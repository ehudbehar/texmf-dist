% \iffalse meta-comment
%
% File: pagesel.dtx
% Version: 2020-08-03 v1.10
% Info: Select pages of a document for output
%
% Copyright (C)
%    1999, 2003, 2006-2008 Heiko Oberdiek
%    2016-2020 Oberdiek Package Support Group
%    https://github.com/ho-tex/pagesel/issues
%
% This work may be distributed and/or modified under the
% conditions of the LaTeX Project Public License, either
% version 1.3c of this license or (at your option) any later
% version. This version of this license is in
%    https://www.latex-project.org/lppl/lppl-1-3c.txt
% and the latest version of this license is in
%    https://www.latex-project.org/lppl.txt
% and version 1.3 or later is part of all distributions of
% LaTeX version 2005/12/01 or later.
%
% This work has the LPPL maintenance status "maintained".
%
% The Current Maintainers of this work are
% Heiko Oberdiek and the Oberdiek Package Support Group
% https://github.com/ho-tex/pagesel/issues
%
% This work consists of the main source file pagesel.dtx
% and the derived files
%    pagesel.sty, pagesel-2016-05-16.sty,
%    pagesel.pdf, pagesel.ins, pagesel.drv.
%
% Distribution:
%    CTAN:macros/latex/contrib/pagesel/pagesel.dtx
%    CTAN:macros/latex/contrib/pagesel/pagesel.pdf
%
% Unpacking:
%    (a) If pagesel.ins is present:
%           tex pagesel.ins
%    (b) Without pagesel.ins:
%           tex pagesel.dtx
%    (c) If you insist on using LaTeX
%           latex \let\install=y% \iffalse meta-comment
%
% File: pagesel.dtx
% Version: 2020-08-03 v1.10
% Info: Select pages of a document for output
%
% Copyright (C)
%    1999, 2003, 2006-2008 Heiko Oberdiek
%    2016-2020 Oberdiek Package Support Group
%    https://github.com/ho-tex/pagesel/issues
%
% This work may be distributed and/or modified under the
% conditions of the LaTeX Project Public License, either
% version 1.3c of this license or (at your option) any later
% version. This version of this license is in
%    https://www.latex-project.org/lppl/lppl-1-3c.txt
% and the latest version of this license is in
%    https://www.latex-project.org/lppl.txt
% and version 1.3 or later is part of all distributions of
% LaTeX version 2005/12/01 or later.
%
% This work has the LPPL maintenance status "maintained".
%
% The Current Maintainers of this work are
% Heiko Oberdiek and the Oberdiek Package Support Group
% https://github.com/ho-tex/pagesel/issues
%
% This work consists of the main source file pagesel.dtx
% and the derived files
%    pagesel.sty, pagesel-2016-05-16.sty,
%    pagesel.pdf, pagesel.ins, pagesel.drv.
%
% Distribution:
%    CTAN:macros/latex/contrib/pagesel/pagesel.dtx
%    CTAN:macros/latex/contrib/pagesel/pagesel.pdf
%
% Unpacking:
%    (a) If pagesel.ins is present:
%           tex pagesel.ins
%    (b) Without pagesel.ins:
%           tex pagesel.dtx
%    (c) If you insist on using LaTeX
%           latex \let\install=y% \iffalse meta-comment
%
% File: pagesel.dtx
% Version: 2020-08-03 v1.10
% Info: Select pages of a document for output
%
% Copyright (C)
%    1999, 2003, 2006-2008 Heiko Oberdiek
%    2016-2020 Oberdiek Package Support Group
%    https://github.com/ho-tex/pagesel/issues
%
% This work may be distributed and/or modified under the
% conditions of the LaTeX Project Public License, either
% version 1.3c of this license or (at your option) any later
% version. This version of this license is in
%    https://www.latex-project.org/lppl/lppl-1-3c.txt
% and the latest version of this license is in
%    https://www.latex-project.org/lppl.txt
% and version 1.3 or later is part of all distributions of
% LaTeX version 2005/12/01 or later.
%
% This work has the LPPL maintenance status "maintained".
%
% The Current Maintainers of this work are
% Heiko Oberdiek and the Oberdiek Package Support Group
% https://github.com/ho-tex/pagesel/issues
%
% This work consists of the main source file pagesel.dtx
% and the derived files
%    pagesel.sty, pagesel-2016-05-16.sty,
%    pagesel.pdf, pagesel.ins, pagesel.drv.
%
% Distribution:
%    CTAN:macros/latex/contrib/pagesel/pagesel.dtx
%    CTAN:macros/latex/contrib/pagesel/pagesel.pdf
%
% Unpacking:
%    (a) If pagesel.ins is present:
%           tex pagesel.ins
%    (b) Without pagesel.ins:
%           tex pagesel.dtx
%    (c) If you insist on using LaTeX
%           latex \let\install=y% \iffalse meta-comment
%
% File: pagesel.dtx
% Version: 2020-08-03 v1.10
% Info: Select pages of a document for output
%
% Copyright (C)
%    1999, 2003, 2006-2008 Heiko Oberdiek
%    2016-2020 Oberdiek Package Support Group
%    https://github.com/ho-tex/pagesel/issues
%
% This work may be distributed and/or modified under the
% conditions of the LaTeX Project Public License, either
% version 1.3c of this license or (at your option) any later
% version. This version of this license is in
%    https://www.latex-project.org/lppl/lppl-1-3c.txt
% and the latest version of this license is in
%    https://www.latex-project.org/lppl.txt
% and version 1.3 or later is part of all distributions of
% LaTeX version 2005/12/01 or later.
%
% This work has the LPPL maintenance status "maintained".
%
% The Current Maintainers of this work are
% Heiko Oberdiek and the Oberdiek Package Support Group
% https://github.com/ho-tex/pagesel/issues
%
% This work consists of the main source file pagesel.dtx
% and the derived files
%    pagesel.sty, pagesel-2016-05-16.sty,
%    pagesel.pdf, pagesel.ins, pagesel.drv.
%
% Distribution:
%    CTAN:macros/latex/contrib/pagesel/pagesel.dtx
%    CTAN:macros/latex/contrib/pagesel/pagesel.pdf
%
% Unpacking:
%    (a) If pagesel.ins is present:
%           tex pagesel.ins
%    (b) Without pagesel.ins:
%           tex pagesel.dtx
%    (c) If you insist on using LaTeX
%           latex \let\install=y\input{pagesel.dtx}
%        (quote the arguments according to the demands of your shell)
%
% Documentation:
%    (a) If pagesel.drv is present:
%           latex pagesel.drv
%    (b) Without pagesel.drv:
%           latex pagesel.dtx; ...
%    The class ltxdoc loads the configuration file ltxdoc.cfg
%    if available. Here you can specify further options, e.g.
%    use A4 as paper format:
%       \PassOptionsToClass{a4paper}{article}
%
%    Programm calls to get the documentation (example):
%       pdflatex pagesel.dtx
%       makeindex -s gind.ist pagesel.idx
%       pdflatex pagesel.dtx
%       makeindex -s gind.ist pagesel.idx
%       pdflatex pagesel.dtx
%
% Installation:
%    TDS:tex/latex/pagesel/pagesel.sty
%    TDS:doc/latex/pagesel/pagesel.pdf
%    TDS:source/latex/pagesel/pagesel.dtx
%
%<*ignore>
\begingroup
  \catcode123=1 %
  \catcode125=2 %
  \def\x{LaTeX2e}%
\expandafter\endgroup
\ifcase 0\ifx\install y1\fi\expandafter
         \ifx\csname processbatchFile\endcsname\relax\else1\fi
         \ifx\fmtname\x\else 1\fi\relax
\else\csname fi\endcsname
%</ignore>
%<*install>
\input docstrip.tex
\Msg{************************************************************************}
\Msg{* Installation}
\Msg{* Package: pagesel 2020-08-03 v1.10 Select pages of a document for output (HO)}
\Msg{************************************************************************}

\keepsilent
\askforoverwritefalse

\let\MetaPrefix\relax
\preamble

This is a generated file.

Project: pagesel
Version: 2020-08-03 v1.10

Copyright (C)
   1999, 2003, 2006-2008 Heiko Oberdiek
   2016-2020 Oberdiek Package Support Group

This work may be distributed and/or modified under the
conditions of the LaTeX Project Public License, either
version 1.3c of this license or (at your option) any later
version. This version of this license is in
   https://www.latex-project.org/lppl/lppl-1-3c.txt
and the latest version of this license is in
   https://www.latex-project.org/lppl.txt
and version 1.3 or later is part of all distributions of
LaTeX version 2005/12/01 or later.

This work has the LPPL maintenance status "maintained".

The Current Maintainers of this work are
Heiko Oberdiek and the Oberdiek Package Support Group
https://github.com/ho-tex/pagesel/issues


This work consists of the main source file pagesel.dtx
and the derived files
   pagesel.sty, pagesel-2016-05-16.sty, pagesel.pdf,
   pagesel.ins, pagesel.drv.

\endpreamble
\let\MetaPrefix\DoubleperCent

\generate{%
  \file{pagesel.ins}{\from{pagesel.dtx}{install}}%
  \file{pagesel.drv}{\from{pagesel.dtx}{driver}}%
  \usedir{tex/latex/pagesel}%
  \file{pagesel.sty}{\from{pagesel.dtx}{package}}%
  \file{pagesel-2016-05-16.sty}{\from{pagesel.dtx}{packagefrozen}}
}

\catcode32=13\relax% active space
\let =\space%
\Msg{************************************************************************}
\Msg{*}
\Msg{* To finish the installation you have to move the following}
\Msg{* file into a directory searched by TeX:}
\Msg{*}
\Msg{*     pagesel.sty}
\Msg{*}
\Msg{* To produce the documentation run the file `pagesel.drv'}
\Msg{* through LaTeX.}
\Msg{*}
\Msg{* Happy TeXing!}
\Msg{*}
\Msg{************************************************************************}

\endbatchfile
%</install>
%<*ignore>
\fi
%</ignore>
%<*driver>
\NeedsTeXFormat{LaTeX2e}
\ProvidesFile{pagesel.drv}%
  [2020-08-03 v1.10 Select pages of a document for output (HO)]%
\documentclass{ltxdoc}
\usepackage{holtxdoc}[2011/11/22]
\begin{document}
  \DocInput{pagesel.dtx}%
\end{document}
%</driver>
% \fi
%
%
%
% \GetFileInfo{pagesel.drv}
%
% \title{The \xpackage{pagesel} package}
% \date{2020-08-03 v1.10}
% \author{Heiko Oberdiek\thanks
% {Please report any issues at \url{https://github.com/ho-tex/pagesel/issues}}}
%
% \maketitle
%
% \begin{abstract}
% Single pages or page areas can be selected for output.
% \end{abstract}
%
% \tableofcontents
%
% \newenvironment{param}{^^A
%   \newcommand{\entry}[1]{\meta{\###1}:&}^^A
%   \begin{tabular}[t]{@{}l@{ }l@{}}^^A
% }{^^A
%   \end{tabular}^^A
% }
%
% \newcommand*{\Option}[1]{\textsf{#1}}
%
% \section{Usage}
%    The package \Package{pagesel} is a \LaTeXe\ package:
%    \begin{quote}
%      |\usepackage|\oarg{options}|{pagesel}|
%    \end{quote}
%    (For plain\TeX\ and \LaTeX\,2.09 the similar package
%    \URL{\Package{selectp}}^^A
%    {https://ctan.org/pkg/selectp}
%    from \NameEmail{Donald Arsenau}{asnd@triumf.ca} can be used.)
%
%    Depending on the options the package works in two modes:
%    \begin{enumerate}
%    \item If no page selecting option is present, so the package
%          ignores the other options and finishes itself. So no
%          page will be suppressed by the package and auxiliary files
%          will be written.
%    \item With at least one page selecting option the specified
%          pages are selected and the other are suppressed.
%          The default for this mode is that auxiliary will not be
%          overwritten. (This can be changed by an option.)
%    \end{enumerate}
%
% \subsection{Page selecting}
%    The package \Package{pagesel} sets up a new counter that is
%    incremented by each \cmd{\shipout}.
%    In this way the package counts the output pages regardless the value
%    of the page counter. So each page can individually by addressed,
%    even if there are several pages with the same page number.
%
% \subsubsection{Options\texorpdfstring{ for selecting pages}{}}
%    \begin{description}
%    \item[\Option{odd}:] The output pages must have an odd number.
%         All even output pages are suppressed. If there are no
%         page areas specified so all odd pages are print. With
%         page areas only the odd pages in this areas are selected.
%    \item[\Option{even}:] The opposite of option \Option{odd}.
%    \item[Page area:] A page area consists of three elements:
%         the starting output page number, an ``area'' hyphen, and
%         the output page number of the last page in this area.
%         Each component is optional, so there are four kinds
%         to spezify a page area:
%         \begin{description}
%         \item[\meta{m}\Option{-}\meta{n}:] All pages between
%              \meta{m} and \meta{n} inclusive.
%         \item[\Option{-}\meta{n}:] All pages until \meta{n} inclusive.
%         \item[\meta{m}\Option{-}:] The page area starts with \meta{m}
%              and all pages to the end of document are selected.
%         \item[\Option{-}:] All pages (not very useful).
%         \item[\meta{s}:] The single page \meta{s}.
%         \end{description}
%    \end{description}
%
% \subsubsection{Examples}
%    \newcommand*{\exam}[1]{\texttt{\strut[#1]}}^^A hash-ok
%    \begin{tabular}{ll}
%      Options & Output pages\\
%      \hline
%      \exam{1, 4, 9}&  1, 4, and 9\\
%      \exam{7-10, 3}&  3, 7, 8, 9, and 10\\
%      \exam{odd, 3-6}& 3, and 5\\
%      \exam{-4, 3, even, 7-8}& 2, 4, and 8\\
%    \end{tabular}
%
% \subsection{Auxiliary files}
%    If a page is suppressed, the \cmd{\write} commands are not
%    performed. Labels, index entries, or entries for the
%    table of contents aren't written. So it is likely that
%    the table of contents, registers, and lists are incomplete.
% \subsubsection{Options\texorpdfstring{ for handling auxiliary files}{}}
%    \begin{description}
%    \item[\Option{nofiles}:] This is the default. Auxiliary files are
%         read but not written or changed. Also the job is aborted
%         after the last selected page for saving time.
%    \item[\Option{nonofiles}/\Option{files}:] Auxiliary files are
%         written.
%    \end{description}
% \subsubsection{\texorpdfstring{Package }{}\Package{hyperref}}
%    In old versions of \Package{hyperref} [1999/04/12 v6.55] (and below)
%    there is a bug with \cmd{\nofiles}:
%    \begin{itemize}
%    \item Some ``garbage'' appears on terminal and in the log file.
%          This is harmless and can be ignored.
%    \item The outline auxiliary file \cmd{\jobname.out}, however,
%          is opened and truncated to zero bytes.
%          Version 1.0 of this package had
%          loaded a patch file \File{hypnofil.tex}, if it detects
%          \Package{hyperref} to get \cmd{\nofiles} work.
%
%          With the new version of \Package{hyperref} [1999/04/13 v6.56]
%          \cmd{\nofiles} works now. Therefore the workaround code
%          is no longer needed and removed.
%    \end{itemize}
%
% \StopEventually{
% }
%
% \section{Implementation}
%    \begin{macrocode}
%<*package>
%    \end{macrocode}
% \subsection{New implementation using the LaTeX kernel hooks}
%    \begin{macrocode}
\NeedsTeXFormat{LaTeX2e}
\ProvidesPackage{pagesel}
  [2020-08-03 v1.10 Select pages of a document for output (HO)]%
%    \end{macrocode}
%    \begin{macrocode}
\providecommand\IfFormatAtLeastTF{\@ifl@t@r\fmtversion}
\IfFormatAtLeastTF{2020/10/01}{}{\input{pagesel-2016-05-16.sty}}
\IfFormatAtLeastTF{2020/10/01}{}{\endinput}

%    \end{macrocode}
%    If the package is loaded twice, the package code does not
%    work. So stop loading the package, if it is already loaded.
%    \begin{macrocode}
\@ifundefined{ps@oddpages}{}{%
  \PackageWarningNoLine{pagesel}{Package already loaded.}%
  \endinput
}
%    \end{macrocode}
%    \begin{macrocode}
%</package>
%    \end{macrocode}
% \subsection{Package}
%    \begin{macrocode}
%<*packagefrozen>
\NeedsTeXFormat{LaTeX2e}
\ProvidesPackage{pagesel}
  [2020-08-03 v1.10 Select pages of a document for output (legacy code) (HO)]%
%    \end{macrocode}
%
%    If the package is loaded twice, the package code does not
%    work. So stop loading the package, if it is already loaded.
%    \begin{macrocode}
\@ifundefined{ps@makevoid}{}{%
  \PackageWarningNoLine{pagesel}{Package already loaded.}%
  \endinput
}
%    \end{macrocode}
%
%    \begin{macro}{\ps@makevoid}
%    Macro \cmd{\ps@makevoid} clears the output box. Because
%    nothing is shipped out and this is intended, we reduce
%    the counter \cmd{\deadcycles} in order to avoid problems, if
%    more than \cmd{\maxdeadcycles} pages are omitted.
%    \begin{macrocode}
\newcommand*{\ps@makevoid}{%
  \global\setbox\@cclv\copy\voidb@x
  \begingroup
    \count@=\deadcycles
    \advance\count@ by -1\relax
    \deadcycles=\count@
  \endgroup
}
%</packagefrozen>
%    \end{macrocode}
%    \end{macro}
%
%    \begin{macro}{\ps@oddpages}
%    \begin{macrocode}
%<*package|packagefrozen>
\newcommand*\ps@oddpages{0}
\DeclareOption{odd}{\renewcommand*\ps@oddpages{1}}
\DeclareOption{even}{\renewcommand*\ps@oddpages{2}}
%    \end{macrocode}
%    \end{macro}
%
%    \begin{macrocode}
\DeclareOption{nofiles}{\let\ps@nofiles\nofiles}
\DeclareOption{nonofiles}{\let\ps@nofiles\@empty}
\DeclareOption{files}{\let\ps@nofiles\@empty}
\ExecuteOptions{nofiles}
%    \end{macrocode}
%
%    \begin{macrocode}
\DeclareOption*{%
  \begingroup
    \expandafter\ps@checkoption\CurrentOption-\END
    \edef\x{\endgroup\noexpand\ps@store{\ps@first}{\ps@last}}%
  \x
}
%    \end{macrocode}
%
%    \begin{macro}{\ps@checkoption}
%    \begin{macrocode}
\newcommand\ps@checkoption{}
\def\ps@checkoption#1-#2\END{%
  \ifx\\#2\\%
    \ifx\\#1\\%
      % empty option
      \def\ps@first{\maxdimen}%
      \def\ps@last{\maxdimen}%
    \else
      \edef\ps@first{#1}%
      \edef\ps@last{#1}%
    \fi
  \else
    \ifx\\#1\\%
      \def\ps@first{-\maxdimen}%
    \else
      \edef\ps@first{#1}%
    \fi
    \ps@checklast#2%
  \fi
}
%    \end{macrocode}
%    \end{macro}
%
%    \begin{macro}{\ps@checklast}
%    \begin{macrocode}
\newcommand\ps@checklast{}
\def\ps@checklast#1-{%
  \ifx\\#1\\%
    \def\ps@last{\maxdimen}%
  \else
    \edef\ps@last{#1}%
  \fi
}
%    \end{macrocode}
%    \end{macro}
%
%    \begin{macro}{\ps@store}
%    \begin{macrocode}
\newcommand*{\ps@store}[2]{%
  \expandafter\def\expandafter\ps@testlist\expandafter{%
    \ps@testlist\ps@pagetest{#1}{#2}%
  }%
}
%    \end{macrocode}
%    \end{macro}
%
%    \begin{macro}{\ps@testlist}
%    \begin{macrocode}
\newcommand*\ps@testlist{}
%    \end{macrocode}
%    \end{macro}
%
%    \begin{macrocode}
\ProcessOptions
%    \end{macrocode}
%
%    \begin{macrocode}
\begingroup
  \edef\x{%
    \ifnum\ps@oddpages>0 \relax\fi
    \ifx\ps@testlist\@empty\else\relax\fi
  }%
  \ifx\x\@empty
    \endgroup
    \PackageInfo{pagesel}{Nothing to do}%
    \expandafter\endinput
  \fi
\endgroup
%    \end{macrocode}
%
%    \begin{macrocode}
%</package|packagefrozen>
%<*packagefrozen>
\RequirePackage{everyshi}
%</packagefrozen>
%    \end{macrocode}
%
%    \begin{macrocode}
%<*package|packagefrozen>
\ps@nofiles
%    \end{macrocode}
%
%    \begin{macro}{\c@ps@count}
%    \begin{macrocode}
\newcounter{ps@count}
\setcounter{ps@count}{0}
%    \end{macrocode}
%    \end{macro}
%
%    \begin{macro}{\ps@ReturnAfterElseFi}
%    \begin{macro}{\ps@ReturnAfterFi}
%    \begin{macrocode}
\long\def\ps@ReturnAfterElseFi#1\else#2\fi{\fi#1}
\long\def\ps@ReturnAfterFi#1\fi{\fi#1}
%    \end{macrocode}
%    \end{macro}
%    \end{macro}
%
%    \begin{macrocode}
\newcommand{\ps@lastpage}{\maxdimen}
\ifx\ps@nofiles\nofiles
  \ifx\ps@testlist\@empty
  \else
    \def\ps@lastpage{0}%
    \newcommand*{\ps@pagetest}[2]{%
      \ifnum#2>\ps@lastpage\relax
        \def\ps@lastpage{#2}%
      \fi
    }%
    \ps@testlist
    \let\ps@pagetest\relax
  \fi
\fi
%    \end{macrocode}
%
%    \begin{macro}{\ps@ifinset}
%    \begin{macrocode}
\newcommand*{\ps@ifinset}[4]{%
  \ifnum#1>\value{ps@count}%
    \ps@ReturnAfterElseFi{#4}%
  \else
    \ps@ReturnAfterFi{%
      \ifnum#2<\value{ps@count}%
        \ps@ReturnAfterElseFi{#4}%
      \else
        \ps@ReturnAfterFi{#3}%
      \fi
    }%
  \fi
}
%    \end{macrocode}
%    \end{macro}
%
%    \begin{macro}{\ps@pagetest}
%    \begin{macrocode}
\newcommand*{\ps@pagetest}[2]{%
  \ps@ifinset{#1}{#2}{\let\ps@next\@empty}{}%
}
%    \end{macrocode}
%    \end{macro}
%
%    \begin{macrocode}
%</package|packagefrozen>
%<packagefrozen>\EveryShipout{%
%<package>\AddToHook{shipout/before}{%
%<*package|packagefrozen>
  \stepcounter{ps@count}%
  \ifnum\value{ps@count}>\ps@lastpage\relax
    \global\output{%
      \ps@cleanup@if
      \ps@group@message
      \typeout{%
        Package pagesel Notice: Aborting LaTeX job %
        after last selected page (\ps@lastpage).%
      }%
      \ps@message@ignore
      \global\setbox\@cclv\box\voidb@x
      \deadcycles0\relax
%    \end{macrocode}
%    First leave the output group before ending the job.
%    \begin{macrocode}
      \aftergroup\@@end
    }%
  \fi
  \let\ps@next\@empty
  \ifx\ps@testlist\@empty
  \else
%<packagefrozen>    \let\ps@next\ps@makevoid
%<package>    \let\ps@next\DiscardShipoutBox
    \ps@testlist
  \fi
  \ifnum\ps@oddpages=1 %
    \ifodd\value{ps@count}%
    \else
%<packagefrozen>    \let\ps@next\ps@makevoid
%<package>    \let\ps@next\DiscardShipoutBox
    \fi
  \fi
  \ifnum\ps@oddpages=2 %
    \ifodd\value{ps@count}%
%<packagefrozen>    \let\ps@next\ps@makevoid
%<package>    \let\ps@next\DiscardShipoutBox
    \else
    \fi
  \fi
%<packagefrozen>  \ps@begindvi
  \ps@next
}
%</package|packagefrozen>
%    \end{macrocode}
%
%    \begin{macrocode}
%<*package|packagefrozen>
%<packagefrozen>\begingroup\expandafter\expandafter\expandafter\endgroup
%<packagefrozen>\expandafter\ifx\csname currentiflevel\endcsname\relax
%<packagefrozen>  \let\ps@cleanup@if\@empty
%<packagefrozen>\else
  \def\ps@cleanup@if{%
    \ifnum\currentiflevel>\@ne
      \csname fi\endcsname
      \expandafter\ps@cleanup@if
    \fi
  }%
%<packagefrozen>\fi
%    \end{macrocode}
%    Because of \cs{aftergroup} it is too dangerous to perform
%    a similar cleanup for groups.
%    \begin{macrocode}
%<packagefrozen> \begingroup\expandafter\expandafter\expandafter\endgroup
%<packagefrozen> \expandafter\ifx\csname currentgrouplevel\endcsname\relax
%<packagefrozen>  \let\ps@group@message\@empty
%<packagefrozen>  \def\ps@message@ignore{%
%<packagefrozen>    \typeout{%
%<packagefrozen>      (pagesel) \space\space\@spaces\@spaces\@spaces
%<packagefrozen>      Messages (\string\end\space occurred ...) can be ignored.%
%<packagefrozen>    }%
%<packagefrozen>  }%
%<packagefrozen>\else
  \def\ps@group@message{%
    \ifnum\currentgrouplevel>\@ne
      \def\ps@message@ignore{%
        \typeout{%
          (pagesel) \space\space\@spaces\@spaces\@spaces
          Message (\string\end\space occurred ...) %
          can be ignored.%
        }%
      }%
    \else
      \let\ps@message@ignore\@empty
    \fi
  }%
%<packagefrozen>\fi
%</package|packagefrozen>
%    \end{macrocode}
%
% \subsection{AtBeginDvi hook support}
%
%    The material of box \cs{@begindvibox} is recorded in parallel
%    in box \cs{ps@begindvibox}.
%    \begin{macrocode}
%<*packagefrozen>
\newbox\ps@begindvibox
\ifvoid\@begindvibox
\else
  \global\setbox\ps@begindvibox\vbox{%
    \unvbox\@begindvibox
  }%
\fi
\let\ps@org@AtBeginDvi\AtBeginDvi
\def\AtBeginDvi#1{%
  \global\setbox\ps@begindvibox\vbox{%
    \unvbox\ps@begindvibox
    #1%
  }%
  \ps@org@AtBeginDvi{#1}%
}
%    \end{macrocode}
%
%    \begin{macro}{\ps@begindvi}
%    Macro \cs{ps@begindvi} is called the similar way as \cs{@begindvi}.
%    If the first page is printed, then \cs{AtBeginDvi} should work
%    as usual. Otherwise the contents of box \cs{ps@begindvibox} is
%    set on the first selected page.
%    \begin{macrocode}
\def\ps@begindvi{%
  \ifx\ps@next\@empty
    \global\let\ps@begindvi\@empty
  \else
    \global\let\ps@begindvi\ps@begindvi@do
  \fi
}
\def\ps@begindvi@do{%
  \ifx\ps@next\@empty
    \setbox\@cclv\vbox{%
      \unvbox\ps@begindvibox
      \box\@cclv
    }%
    \global\let\ps@begindvi\@empty
  \fi
}
%    \end{macrocode}
%    \end{macro}
%
%    \begin{macrocode}
%</packagefrozen>
%    \end{macrocode}
%
% \section{Installation}
%
% \subsection{Download}
%
% \paragraph{Package.} This package is available on
% CTAN\footnote{\CTANpkg{pagesel}}:
% \begin{description}
% \item[\CTAN{macros/latex/contrib/pagesel/pagesel.dtx}] The source file.
% \item[\CTAN{macros/latex/contrib/pagesel/pagesel.pdf}] Documentation.
% \end{description}
%
%
%
% \subsection{Package installation}
%
% \paragraph{Unpacking.} The \xfile{.dtx} file is a self-extracting
% \docstrip\ archive. The files are extracted by running the
% \xfile{.dtx} through \plainTeX:
% \begin{quote}
%   \verb|tex pagesel.dtx|
% \end{quote}
%
% \paragraph{TDS.} Now the different files must be moved into
% the different directories in your installation TDS tree
% (also known as \xfile{texmf} tree):
% \begin{quote}
% \def\t{^^A
% \begin{tabular}{@{}>{\ttfamily}l@{ $\rightarrow$ }>{\ttfamily}l@{}}
%   pagesel.sty & tex/latex/pagesel/pagesel.sty\\
%   pagesel.pdf & doc/latex/pagesel/pagesel.pdf\\
%   pagesel.dtx & source/latex/pagesel/pagesel.dtx\\
% \end{tabular}^^A
% }^^A
% \sbox0{\t}^^A
% \ifdim\wd0>\linewidth
%   \begingroup
%     \advance\linewidth by\leftmargin
%     \advance\linewidth by\rightmargin
%   \edef\x{\endgroup
%     \def\noexpand\lw{\the\linewidth}^^A
%   }\x
%   \def\lwbox{^^A
%     \leavevmode
%     \hbox to \linewidth{^^A
%       \kern-\leftmargin\relax
%       \hss
%       \usebox0
%       \hss
%       \kern-\rightmargin\relax
%     }^^A
%   }^^A
%   \ifdim\wd0>\lw
%     \sbox0{\small\t}^^A
%     \ifdim\wd0>\linewidth
%       \ifdim\wd0>\lw
%         \sbox0{\footnotesize\t}^^A
%         \ifdim\wd0>\linewidth
%           \ifdim\wd0>\lw
%             \sbox0{\scriptsize\t}^^A
%             \ifdim\wd0>\linewidth
%               \ifdim\wd0>\lw
%                 \sbox0{\tiny\t}^^A
%                 \ifdim\wd0>\linewidth
%                   \lwbox
%                 \else
%                   \usebox0
%                 \fi
%               \else
%                 \lwbox
%               \fi
%             \else
%               \usebox0
%             \fi
%           \else
%             \lwbox
%           \fi
%         \else
%           \usebox0
%         \fi
%       \else
%         \lwbox
%       \fi
%     \else
%       \usebox0
%     \fi
%   \else
%     \lwbox
%   \fi
% \else
%   \usebox0
% \fi
% \end{quote}
% If you have a \xfile{docstrip.cfg} that configures and enables \docstrip's
% TDS installing feature, then some files can already be in the right
% place, see the documentation of \docstrip.
%
% \subsection{Refresh file name databases}
%
% If your \TeX~distribution
% (\TeX\,Live, \mikTeX, \dots) relies on file name databases, you must refresh
% these. For example, \TeX\,Live\ users run \verb|texhash| or
% \verb|mktexlsr|.
%
% \subsection{Some details for the interested}
%
% \paragraph{Unpacking with \LaTeX.}
% The \xfile{.dtx} chooses its action depending on the format:
% \begin{description}
% \item[\plainTeX:] Run \docstrip\ and extract the files.
% \item[\LaTeX:] Generate the documentation.
% \end{description}
% If you insist on using \LaTeX\ for \docstrip\ (really,
% \docstrip\ does not need \LaTeX), then inform the autodetect routine
% about your intention:
% \begin{quote}
%   \verb|latex \let\install=y\input{pagesel.dtx}|
% \end{quote}
% Do not forget to quote the argument according to the demands
% of your shell.
%
% \paragraph{Generating the documentation.}
% You can use both the \xfile{.dtx} or the \xfile{.drv} to generate
% the documentation. The process can be configured by the
% configuration file \xfile{ltxdoc.cfg}. For instance, put this
% line into this file, if you want to have A4 as paper format:
% \begin{quote}
%   \verb|\PassOptionsToClass{a4paper}{article}|
% \end{quote}
% An example follows how to generate the
% documentation with pdf\LaTeX:
% \begin{quote}
%\begin{verbatim}
%pdflatex pagesel.dtx
%makeindex -s gind.ist pagesel.idx
%pdflatex pagesel.dtx
%makeindex -s gind.ist pagesel.idx
%pdflatex pagesel.dtx
%\end{verbatim}
% \end{quote}
%
% \begin{History}
%   \begin{Version}{1999/03/01 v0.9}
%   \item
%     The first version was built as a response to a question
%     of \NameEmail{Dirk Kuypers}{dk@comnets.rwth-aachen.de},
%     published in the newsgroup
%     \href{news:de.comp.text.tex}{de.comp.text.tex}:\\
%     \URL{``\link{Re: pdflatex nur fuer bestimmte Seiten?!?}''}^^A
%     {https://groups.google.com/group/de.comp.text.tex/msg/6b68c7b3439fb658}
%   \end{Version}
%   \begin{Version}{1999/04/05 v1.0}
%   \item
%     Documentation added in dtx format.
%   \item
%     Copyright: LPPL (\CTAN{macros/latex/base/lppl.txt})
%   \item
%     Options |odd|, |even| added.
%   \item
%     \cmd{\nofiles} added, bug fix for \Package{hyperref}.
%   \item
%     Abort loading of package, if nothing to do.
%   \end{Version}
%   \begin{Version}{1999/04/13 v1.1}
%   \item
%     \cs{nofiles} bug fix removed
%     because of \xpackage{hyperref} 6.55.
%   \item
%     First CTAN release.
%   \end{Version}
%   \begin{Version}{2003/06/05 v1.2}
%   \item
%     \cs{deadcyles} is decremented for omitted pages.
%   \item
%     LPPL 1.2.
%   \end{Version}
%   \begin{Version}{2006/02/20 v1.3}
%   \item
%     Code is not changed.
%   \item
%     New DTX framework.
%   \item
%     LPPL 1.3
%   \end{Version}
%   \begin{Version}{2006/03/02 v1.4}
%   \item
%     Support for \cs{AtBeginDvi} added.
%   \end{Version}
%   \begin{Version}{2006/03/07 v1.5}
%   \item
%     Job is aborted after last selected page.
%   \end{Version}
%   \begin{Version}{2007/04/11 v1.6}
%   \item
%     Line ends sanitized.
%   \end{Version}
%   \begin{Version}{2007/04/12 v1.7}
%   \item
%     Hard coded box number 255 replaced by macro \cs{@cclv}.
%   \end{Version}
%   \begin{Version}{2008/08/11 v1.8}
%   \item
%     Code is not changed.
%   \item
%     URL updated from \texttt{www.dejanews.com}
%     to \texttt{groups.google.com}.
%   \end{Version}
%   \begin{Version}{2016/05/16 v1.9}
%   \item
%     Documentation updates.
%   \end{Version}
%   \begin{Version}{2020-08-03 v1.10}
%   \item Updated to follow the changes in the hook management
%   of LaTeX 2020/10/01
%   \end{Version}
% \end{History}
%
% \PrintIndex
%
% \Finale
\endinput

%        (quote the arguments according to the demands of your shell)
%
% Documentation:
%    (a) If pagesel.drv is present:
%           latex pagesel.drv
%    (b) Without pagesel.drv:
%           latex pagesel.dtx; ...
%    The class ltxdoc loads the configuration file ltxdoc.cfg
%    if available. Here you can specify further options, e.g.
%    use A4 as paper format:
%       \PassOptionsToClass{a4paper}{article}
%
%    Programm calls to get the documentation (example):
%       pdflatex pagesel.dtx
%       makeindex -s gind.ist pagesel.idx
%       pdflatex pagesel.dtx
%       makeindex -s gind.ist pagesel.idx
%       pdflatex pagesel.dtx
%
% Installation:
%    TDS:tex/latex/pagesel/pagesel.sty
%    TDS:doc/latex/pagesel/pagesel.pdf
%    TDS:source/latex/pagesel/pagesel.dtx
%
%<*ignore>
\begingroup
  \catcode123=1 %
  \catcode125=2 %
  \def\x{LaTeX2e}%
\expandafter\endgroup
\ifcase 0\ifx\install y1\fi\expandafter
         \ifx\csname processbatchFile\endcsname\relax\else1\fi
         \ifx\fmtname\x\else 1\fi\relax
\else\csname fi\endcsname
%</ignore>
%<*install>
\input docstrip.tex
\Msg{************************************************************************}
\Msg{* Installation}
\Msg{* Package: pagesel 2020-08-03 v1.10 Select pages of a document for output (HO)}
\Msg{************************************************************************}

\keepsilent
\askforoverwritefalse

\let\MetaPrefix\relax
\preamble

This is a generated file.

Project: pagesel
Version: 2020-08-03 v1.10

Copyright (C)
   1999, 2003, 2006-2008 Heiko Oberdiek
   2016-2020 Oberdiek Package Support Group

This work may be distributed and/or modified under the
conditions of the LaTeX Project Public License, either
version 1.3c of this license or (at your option) any later
version. This version of this license is in
   https://www.latex-project.org/lppl/lppl-1-3c.txt
and the latest version of this license is in
   https://www.latex-project.org/lppl.txt
and version 1.3 or later is part of all distributions of
LaTeX version 2005/12/01 or later.

This work has the LPPL maintenance status "maintained".

The Current Maintainers of this work are
Heiko Oberdiek and the Oberdiek Package Support Group
https://github.com/ho-tex/pagesel/issues


This work consists of the main source file pagesel.dtx
and the derived files
   pagesel.sty, pagesel-2016-05-16.sty, pagesel.pdf,
   pagesel.ins, pagesel.drv.

\endpreamble
\let\MetaPrefix\DoubleperCent

\generate{%
  \file{pagesel.ins}{\from{pagesel.dtx}{install}}%
  \file{pagesel.drv}{\from{pagesel.dtx}{driver}}%
  \usedir{tex/latex/pagesel}%
  \file{pagesel.sty}{\from{pagesel.dtx}{package}}%
  \file{pagesel-2016-05-16.sty}{\from{pagesel.dtx}{packagefrozen}}
}

\catcode32=13\relax% active space
\let =\space%
\Msg{************************************************************************}
\Msg{*}
\Msg{* To finish the installation you have to move the following}
\Msg{* file into a directory searched by TeX:}
\Msg{*}
\Msg{*     pagesel.sty}
\Msg{*}
\Msg{* To produce the documentation run the file `pagesel.drv'}
\Msg{* through LaTeX.}
\Msg{*}
\Msg{* Happy TeXing!}
\Msg{*}
\Msg{************************************************************************}

\endbatchfile
%</install>
%<*ignore>
\fi
%</ignore>
%<*driver>
\NeedsTeXFormat{LaTeX2e}
\ProvidesFile{pagesel.drv}%
  [2020-08-03 v1.10 Select pages of a document for output (HO)]%
\documentclass{ltxdoc}
\usepackage{holtxdoc}[2011/11/22]
\begin{document}
  \DocInput{pagesel.dtx}%
\end{document}
%</driver>
% \fi
%
%
%
% \GetFileInfo{pagesel.drv}
%
% \title{The \xpackage{pagesel} package}
% \date{2020-08-03 v1.10}
% \author{Heiko Oberdiek\thanks
% {Please report any issues at \url{https://github.com/ho-tex/pagesel/issues}}}
%
% \maketitle
%
% \begin{abstract}
% Single pages or page areas can be selected for output.
% \end{abstract}
%
% \tableofcontents
%
% \newenvironment{param}{^^A
%   \newcommand{\entry}[1]{\meta{\###1}:&}^^A
%   \begin{tabular}[t]{@{}l@{ }l@{}}^^A
% }{^^A
%   \end{tabular}^^A
% }
%
% \newcommand*{\Option}[1]{\textsf{#1}}
%
% \section{Usage}
%    The package \Package{pagesel} is a \LaTeXe\ package:
%    \begin{quote}
%      |\usepackage|\oarg{options}|{pagesel}|
%    \end{quote}
%    (For plain\TeX\ and \LaTeX\,2.09 the similar package
%    \URL{\Package{selectp}}^^A
%    {https://ctan.org/pkg/selectp}
%    from \NameEmail{Donald Arsenau}{asnd@triumf.ca} can be used.)
%
%    Depending on the options the package works in two modes:
%    \begin{enumerate}
%    \item If no page selecting option is present, so the package
%          ignores the other options and finishes itself. So no
%          page will be suppressed by the package and auxiliary files
%          will be written.
%    \item With at least one page selecting option the specified
%          pages are selected and the other are suppressed.
%          The default for this mode is that auxiliary will not be
%          overwritten. (This can be changed by an option.)
%    \end{enumerate}
%
% \subsection{Page selecting}
%    The package \Package{pagesel} sets up a new counter that is
%    incremented by each \cmd{\shipout}.
%    In this way the package counts the output pages regardless the value
%    of the page counter. So each page can individually by addressed,
%    even if there are several pages with the same page number.
%
% \subsubsection{Options\texorpdfstring{ for selecting pages}{}}
%    \begin{description}
%    \item[\Option{odd}:] The output pages must have an odd number.
%         All even output pages are suppressed. If there are no
%         page areas specified so all odd pages are print. With
%         page areas only the odd pages in this areas are selected.
%    \item[\Option{even}:] The opposite of option \Option{odd}.
%    \item[Page area:] A page area consists of three elements:
%         the starting output page number, an ``area'' hyphen, and
%         the output page number of the last page in this area.
%         Each component is optional, so there are four kinds
%         to spezify a page area:
%         \begin{description}
%         \item[\meta{m}\Option{-}\meta{n}:] All pages between
%              \meta{m} and \meta{n} inclusive.
%         \item[\Option{-}\meta{n}:] All pages until \meta{n} inclusive.
%         \item[\meta{m}\Option{-}:] The page area starts with \meta{m}
%              and all pages to the end of document are selected.
%         \item[\Option{-}:] All pages (not very useful).
%         \item[\meta{s}:] The single page \meta{s}.
%         \end{description}
%    \end{description}
%
% \subsubsection{Examples}
%    \newcommand*{\exam}[1]{\texttt{\strut[#1]}}^^A hash-ok
%    \begin{tabular}{ll}
%      Options & Output pages\\
%      \hline
%      \exam{1, 4, 9}&  1, 4, and 9\\
%      \exam{7-10, 3}&  3, 7, 8, 9, and 10\\
%      \exam{odd, 3-6}& 3, and 5\\
%      \exam{-4, 3, even, 7-8}& 2, 4, and 8\\
%    \end{tabular}
%
% \subsection{Auxiliary files}
%    If a page is suppressed, the \cmd{\write} commands are not
%    performed. Labels, index entries, or entries for the
%    table of contents aren't written. So it is likely that
%    the table of contents, registers, and lists are incomplete.
% \subsubsection{Options\texorpdfstring{ for handling auxiliary files}{}}
%    \begin{description}
%    \item[\Option{nofiles}:] This is the default. Auxiliary files are
%         read but not written or changed. Also the job is aborted
%         after the last selected page for saving time.
%    \item[\Option{nonofiles}/\Option{files}:] Auxiliary files are
%         written.
%    \end{description}
% \subsubsection{\texorpdfstring{Package }{}\Package{hyperref}}
%    In old versions of \Package{hyperref} [1999/04/12 v6.55] (and below)
%    there is a bug with \cmd{\nofiles}:
%    \begin{itemize}
%    \item Some ``garbage'' appears on terminal and in the log file.
%          This is harmless and can be ignored.
%    \item The outline auxiliary file \cmd{\jobname.out}, however,
%          is opened and truncated to zero bytes.
%          Version 1.0 of this package had
%          loaded a patch file \File{hypnofil.tex}, if it detects
%          \Package{hyperref} to get \cmd{\nofiles} work.
%
%          With the new version of \Package{hyperref} [1999/04/13 v6.56]
%          \cmd{\nofiles} works now. Therefore the workaround code
%          is no longer needed and removed.
%    \end{itemize}
%
% \StopEventually{
% }
%
% \section{Implementation}
%    \begin{macrocode}
%<*package>
%    \end{macrocode}
% \subsection{New implementation using the LaTeX kernel hooks}
%    \begin{macrocode}
\NeedsTeXFormat{LaTeX2e}
\ProvidesPackage{pagesel}
  [2020-08-03 v1.10 Select pages of a document for output (HO)]%
%    \end{macrocode}
%    \begin{macrocode}
\providecommand\IfFormatAtLeastTF{\@ifl@t@r\fmtversion}
\IfFormatAtLeastTF{2020/10/01}{}{%%
%% This is file `pagesel-2016-05-16.sty',
%% generated with the docstrip utility.
%%
%% The original source files were:
%%
%% pagesel.dtx  (with options: `packagefrozen')
%% 
%% This is a generated file.
%% 
%% Project: pagesel
%% Version: 2020-08-03 v1.10
%% 
%% Copyright (C)
%%    1999, 2003, 2006-2008 Heiko Oberdiek
%%    2016-2020 Oberdiek Package Support Group
%% 
%% This work may be distributed and/or modified under the
%% conditions of the LaTeX Project Public License, either
%% version 1.3c of this license or (at your option) any later
%% version. This version of this license is in
%%    https://www.latex-project.org/lppl/lppl-1-3c.txt
%% and the latest version of this license is in
%%    https://www.latex-project.org/lppl.txt
%% and version 1.3 or later is part of all distributions of
%% LaTeX version 2005/12/01 or later.
%% 
%% This work has the LPPL maintenance status "maintained".
%% 
%% The Current Maintainers of this work are
%% Heiko Oberdiek and the Oberdiek Package Support Group
%% https://github.com/ho-tex/pagesel/issues
%% 
%% This work consists of the main source file pagesel.dtx
%% and the derived files
%%    pagesel.sty, pagesel-2016-05-16.sty, pagesel.pdf,
%%    pagesel.ins, pagesel.drv.
%% 
\NeedsTeXFormat{LaTeX2e}
\ProvidesPackage{pagesel}
  [2020-08-03 v1.10 Select pages of a document for output (legacy code) (HO)]%
\@ifundefined{ps@makevoid}{}{%
  \PackageWarningNoLine{pagesel}{Package already loaded.}%
  \endinput
}
\newcommand*{\ps@makevoid}{%
  \global\setbox\@cclv\copy\voidb@x
  \begingroup
    \count@=\deadcycles
    \advance\count@ by -1\relax
    \deadcycles=\count@
  \endgroup
}
\newcommand*\ps@oddpages{0}
\DeclareOption{odd}{\renewcommand*\ps@oddpages{1}}
\DeclareOption{even}{\renewcommand*\ps@oddpages{2}}
\DeclareOption{nofiles}{\let\ps@nofiles\nofiles}
\DeclareOption{nonofiles}{\let\ps@nofiles\@empty}
\DeclareOption{files}{\let\ps@nofiles\@empty}
\ExecuteOptions{nofiles}
\DeclareOption*{%
  \begingroup
    \expandafter\ps@checkoption\CurrentOption-\END
    \edef\x{\endgroup\noexpand\ps@store{\ps@first}{\ps@last}}%
  \x
}
\newcommand\ps@checkoption{}
\def\ps@checkoption#1-#2\END{%
  \ifx\\#2\\%
    \ifx\\#1\\%
      % empty option
      \def\ps@first{\maxdimen}%
      \def\ps@last{\maxdimen}%
    \else
      \edef\ps@first{#1}%
      \edef\ps@last{#1}%
    \fi
  \else
    \ifx\\#1\\%
      \def\ps@first{-\maxdimen}%
    \else
      \edef\ps@first{#1}%
    \fi
    \ps@checklast#2%
  \fi
}
\newcommand\ps@checklast{}
\def\ps@checklast#1-{%
  \ifx\\#1\\%
    \def\ps@last{\maxdimen}%
  \else
    \edef\ps@last{#1}%
  \fi
}
\newcommand*{\ps@store}[2]{%
  \expandafter\def\expandafter\ps@testlist\expandafter{%
    \ps@testlist\ps@pagetest{#1}{#2}%
  }%
}
\newcommand*\ps@testlist{}
\ProcessOptions
\begingroup
  \edef\x{%
    \ifnum\ps@oddpages>0 \relax\fi
    \ifx\ps@testlist\@empty\else\relax\fi
  }%
  \ifx\x\@empty
    \endgroup
    \PackageInfo{pagesel}{Nothing to do}%
    \expandafter\endinput
  \fi
\endgroup
\RequirePackage{everyshi}
\ps@nofiles
\newcounter{ps@count}
\setcounter{ps@count}{0}
\long\def\ps@ReturnAfterElseFi#1\else#2\fi{\fi#1}
\long\def\ps@ReturnAfterFi#1\fi{\fi#1}
\newcommand{\ps@lastpage}{\maxdimen}
\ifx\ps@nofiles\nofiles
  \ifx\ps@testlist\@empty
  \else
    \def\ps@lastpage{0}%
    \newcommand*{\ps@pagetest}[2]{%
      \ifnum#2>\ps@lastpage\relax
        \def\ps@lastpage{#2}%
      \fi
    }%
    \ps@testlist
    \let\ps@pagetest\relax
  \fi
\fi
\newcommand*{\ps@ifinset}[4]{%
  \ifnum#1>\value{ps@count}%
    \ps@ReturnAfterElseFi{#4}%
  \else
    \ps@ReturnAfterFi{%
      \ifnum#2<\value{ps@count}%
        \ps@ReturnAfterElseFi{#4}%
      \else
        \ps@ReturnAfterFi{#3}%
      \fi
    }%
  \fi
}
\newcommand*{\ps@pagetest}[2]{%
  \ps@ifinset{#1}{#2}{\let\ps@next\@empty}{}%
}
\EveryShipout{%
  \stepcounter{ps@count}%
  \ifnum\value{ps@count}>\ps@lastpage\relax
    \global\output{%
      \ps@cleanup@if
      \ps@group@message
      \typeout{%
        Package pagesel Notice: Aborting LaTeX job %
        after last selected page (\ps@lastpage).%
      }%
      \ps@message@ignore
      \global\setbox\@cclv\box\voidb@x
      \deadcycles0\relax
      \aftergroup\@@end
    }%
  \fi
  \let\ps@next\@empty
  \ifx\ps@testlist\@empty
  \else
    \let\ps@next\ps@makevoid
    \ps@testlist
  \fi
  \ifnum\ps@oddpages=1 %
    \ifodd\value{ps@count}%
    \else
    \let\ps@next\ps@makevoid
    \fi
  \fi
  \ifnum\ps@oddpages=2 %
    \ifodd\value{ps@count}%
    \let\ps@next\ps@makevoid
    \else
    \fi
  \fi
  \ps@begindvi
  \ps@next
}
\begingroup\expandafter\expandafter\expandafter\endgroup
\expandafter\ifx\csname currentiflevel\endcsname\relax
  \let\ps@cleanup@if\@empty
\else
  \def\ps@cleanup@if{%
    \ifnum\currentiflevel>\@ne
      \csname fi\endcsname
      \expandafter\ps@cleanup@if
    \fi
  }%
\fi
 \begingroup\expandafter\expandafter\expandafter\endgroup
 \expandafter\ifx\csname currentgrouplevel\endcsname\relax
  \let\ps@group@message\@empty
  \def\ps@message@ignore{%
    \typeout{%
      (pagesel) \space\space\@spaces\@spaces\@spaces
      Messages (\string\end\space occurred ...) can be ignored.%
    }%
  }%
\else
  \def\ps@group@message{%
    \ifnum\currentgrouplevel>\@ne
      \def\ps@message@ignore{%
        \typeout{%
          (pagesel) \space\space\@spaces\@spaces\@spaces
          Message (\string\end\space occurred ...) %
          can be ignored.%
        }%
      }%
    \else
      \let\ps@message@ignore\@empty
    \fi
  }%
\fi
\newbox\ps@begindvibox
\ifvoid\@begindvibox
\else
  \global\setbox\ps@begindvibox\vbox{%
    \unvbox\@begindvibox
  }%
\fi
\let\ps@org@AtBeginDvi\AtBeginDvi
\def\AtBeginDvi#1{%
  \global\setbox\ps@begindvibox\vbox{%
    \unvbox\ps@begindvibox
    #1%
  }%
  \ps@org@AtBeginDvi{#1}%
}
\def\ps@begindvi{%
  \ifx\ps@next\@empty
    \global\let\ps@begindvi\@empty
  \else
    \global\let\ps@begindvi\ps@begindvi@do
  \fi
}
\def\ps@begindvi@do{%
  \ifx\ps@next\@empty
    \setbox\@cclv\vbox{%
      \unvbox\ps@begindvibox
      \box\@cclv
    }%
    \global\let\ps@begindvi\@empty
  \fi
}
\endinput
%%
%% End of file `pagesel-2016-05-16.sty'.
}
\IfFormatAtLeastTF{2020/10/01}{}{\endinput}

%    \end{macrocode}
%    If the package is loaded twice, the package code does not
%    work. So stop loading the package, if it is already loaded.
%    \begin{macrocode}
\@ifundefined{ps@oddpages}{}{%
  \PackageWarningNoLine{pagesel}{Package already loaded.}%
  \endinput
}
%    \end{macrocode}
%    \begin{macrocode}
%</package>
%    \end{macrocode}
% \subsection{Package}
%    \begin{macrocode}
%<*packagefrozen>
\NeedsTeXFormat{LaTeX2e}
\ProvidesPackage{pagesel}
  [2020-08-03 v1.10 Select pages of a document for output (legacy code) (HO)]%
%    \end{macrocode}
%
%    If the package is loaded twice, the package code does not
%    work. So stop loading the package, if it is already loaded.
%    \begin{macrocode}
\@ifundefined{ps@makevoid}{}{%
  \PackageWarningNoLine{pagesel}{Package already loaded.}%
  \endinput
}
%    \end{macrocode}
%
%    \begin{macro}{\ps@makevoid}
%    Macro \cmd{\ps@makevoid} clears the output box. Because
%    nothing is shipped out and this is intended, we reduce
%    the counter \cmd{\deadcycles} in order to avoid problems, if
%    more than \cmd{\maxdeadcycles} pages are omitted.
%    \begin{macrocode}
\newcommand*{\ps@makevoid}{%
  \global\setbox\@cclv\copy\voidb@x
  \begingroup
    \count@=\deadcycles
    \advance\count@ by -1\relax
    \deadcycles=\count@
  \endgroup
}
%</packagefrozen>
%    \end{macrocode}
%    \end{macro}
%
%    \begin{macro}{\ps@oddpages}
%    \begin{macrocode}
%<*package|packagefrozen>
\newcommand*\ps@oddpages{0}
\DeclareOption{odd}{\renewcommand*\ps@oddpages{1}}
\DeclareOption{even}{\renewcommand*\ps@oddpages{2}}
%    \end{macrocode}
%    \end{macro}
%
%    \begin{macrocode}
\DeclareOption{nofiles}{\let\ps@nofiles\nofiles}
\DeclareOption{nonofiles}{\let\ps@nofiles\@empty}
\DeclareOption{files}{\let\ps@nofiles\@empty}
\ExecuteOptions{nofiles}
%    \end{macrocode}
%
%    \begin{macrocode}
\DeclareOption*{%
  \begingroup
    \expandafter\ps@checkoption\CurrentOption-\END
    \edef\x{\endgroup\noexpand\ps@store{\ps@first}{\ps@last}}%
  \x
}
%    \end{macrocode}
%
%    \begin{macro}{\ps@checkoption}
%    \begin{macrocode}
\newcommand\ps@checkoption{}
\def\ps@checkoption#1-#2\END{%
  \ifx\\#2\\%
    \ifx\\#1\\%
      % empty option
      \def\ps@first{\maxdimen}%
      \def\ps@last{\maxdimen}%
    \else
      \edef\ps@first{#1}%
      \edef\ps@last{#1}%
    \fi
  \else
    \ifx\\#1\\%
      \def\ps@first{-\maxdimen}%
    \else
      \edef\ps@first{#1}%
    \fi
    \ps@checklast#2%
  \fi
}
%    \end{macrocode}
%    \end{macro}
%
%    \begin{macro}{\ps@checklast}
%    \begin{macrocode}
\newcommand\ps@checklast{}
\def\ps@checklast#1-{%
  \ifx\\#1\\%
    \def\ps@last{\maxdimen}%
  \else
    \edef\ps@last{#1}%
  \fi
}
%    \end{macrocode}
%    \end{macro}
%
%    \begin{macro}{\ps@store}
%    \begin{macrocode}
\newcommand*{\ps@store}[2]{%
  \expandafter\def\expandafter\ps@testlist\expandafter{%
    \ps@testlist\ps@pagetest{#1}{#2}%
  }%
}
%    \end{macrocode}
%    \end{macro}
%
%    \begin{macro}{\ps@testlist}
%    \begin{macrocode}
\newcommand*\ps@testlist{}
%    \end{macrocode}
%    \end{macro}
%
%    \begin{macrocode}
\ProcessOptions
%    \end{macrocode}
%
%    \begin{macrocode}
\begingroup
  \edef\x{%
    \ifnum\ps@oddpages>0 \relax\fi
    \ifx\ps@testlist\@empty\else\relax\fi
  }%
  \ifx\x\@empty
    \endgroup
    \PackageInfo{pagesel}{Nothing to do}%
    \expandafter\endinput
  \fi
\endgroup
%    \end{macrocode}
%
%    \begin{macrocode}
%</package|packagefrozen>
%<*packagefrozen>
\RequirePackage{everyshi}
%</packagefrozen>
%    \end{macrocode}
%
%    \begin{macrocode}
%<*package|packagefrozen>
\ps@nofiles
%    \end{macrocode}
%
%    \begin{macro}{\c@ps@count}
%    \begin{macrocode}
\newcounter{ps@count}
\setcounter{ps@count}{0}
%    \end{macrocode}
%    \end{macro}
%
%    \begin{macro}{\ps@ReturnAfterElseFi}
%    \begin{macro}{\ps@ReturnAfterFi}
%    \begin{macrocode}
\long\def\ps@ReturnAfterElseFi#1\else#2\fi{\fi#1}
\long\def\ps@ReturnAfterFi#1\fi{\fi#1}
%    \end{macrocode}
%    \end{macro}
%    \end{macro}
%
%    \begin{macrocode}
\newcommand{\ps@lastpage}{\maxdimen}
\ifx\ps@nofiles\nofiles
  \ifx\ps@testlist\@empty
  \else
    \def\ps@lastpage{0}%
    \newcommand*{\ps@pagetest}[2]{%
      \ifnum#2>\ps@lastpage\relax
        \def\ps@lastpage{#2}%
      \fi
    }%
    \ps@testlist
    \let\ps@pagetest\relax
  \fi
\fi
%    \end{macrocode}
%
%    \begin{macro}{\ps@ifinset}
%    \begin{macrocode}
\newcommand*{\ps@ifinset}[4]{%
  \ifnum#1>\value{ps@count}%
    \ps@ReturnAfterElseFi{#4}%
  \else
    \ps@ReturnAfterFi{%
      \ifnum#2<\value{ps@count}%
        \ps@ReturnAfterElseFi{#4}%
      \else
        \ps@ReturnAfterFi{#3}%
      \fi
    }%
  \fi
}
%    \end{macrocode}
%    \end{macro}
%
%    \begin{macro}{\ps@pagetest}
%    \begin{macrocode}
\newcommand*{\ps@pagetest}[2]{%
  \ps@ifinset{#1}{#2}{\let\ps@next\@empty}{}%
}
%    \end{macrocode}
%    \end{macro}
%
%    \begin{macrocode}
%</package|packagefrozen>
%<packagefrozen>\EveryShipout{%
%<package>\AddToHook{shipout/before}{%
%<*package|packagefrozen>
  \stepcounter{ps@count}%
  \ifnum\value{ps@count}>\ps@lastpage\relax
    \global\output{%
      \ps@cleanup@if
      \ps@group@message
      \typeout{%
        Package pagesel Notice: Aborting LaTeX job %
        after last selected page (\ps@lastpage).%
      }%
      \ps@message@ignore
      \global\setbox\@cclv\box\voidb@x
      \deadcycles0\relax
%    \end{macrocode}
%    First leave the output group before ending the job.
%    \begin{macrocode}
      \aftergroup\@@end
    }%
  \fi
  \let\ps@next\@empty
  \ifx\ps@testlist\@empty
  \else
%<packagefrozen>    \let\ps@next\ps@makevoid
%<package>    \let\ps@next\DiscardShipoutBox
    \ps@testlist
  \fi
  \ifnum\ps@oddpages=1 %
    \ifodd\value{ps@count}%
    \else
%<packagefrozen>    \let\ps@next\ps@makevoid
%<package>    \let\ps@next\DiscardShipoutBox
    \fi
  \fi
  \ifnum\ps@oddpages=2 %
    \ifodd\value{ps@count}%
%<packagefrozen>    \let\ps@next\ps@makevoid
%<package>    \let\ps@next\DiscardShipoutBox
    \else
    \fi
  \fi
%<packagefrozen>  \ps@begindvi
  \ps@next
}
%</package|packagefrozen>
%    \end{macrocode}
%
%    \begin{macrocode}
%<*package|packagefrozen>
%<packagefrozen>\begingroup\expandafter\expandafter\expandafter\endgroup
%<packagefrozen>\expandafter\ifx\csname currentiflevel\endcsname\relax
%<packagefrozen>  \let\ps@cleanup@if\@empty
%<packagefrozen>\else
  \def\ps@cleanup@if{%
    \ifnum\currentiflevel>\@ne
      \csname fi\endcsname
      \expandafter\ps@cleanup@if
    \fi
  }%
%<packagefrozen>\fi
%    \end{macrocode}
%    Because of \cs{aftergroup} it is too dangerous to perform
%    a similar cleanup for groups.
%    \begin{macrocode}
%<packagefrozen> \begingroup\expandafter\expandafter\expandafter\endgroup
%<packagefrozen> \expandafter\ifx\csname currentgrouplevel\endcsname\relax
%<packagefrozen>  \let\ps@group@message\@empty
%<packagefrozen>  \def\ps@message@ignore{%
%<packagefrozen>    \typeout{%
%<packagefrozen>      (pagesel) \space\space\@spaces\@spaces\@spaces
%<packagefrozen>      Messages (\string\end\space occurred ...) can be ignored.%
%<packagefrozen>    }%
%<packagefrozen>  }%
%<packagefrozen>\else
  \def\ps@group@message{%
    \ifnum\currentgrouplevel>\@ne
      \def\ps@message@ignore{%
        \typeout{%
          (pagesel) \space\space\@spaces\@spaces\@spaces
          Message (\string\end\space occurred ...) %
          can be ignored.%
        }%
      }%
    \else
      \let\ps@message@ignore\@empty
    \fi
  }%
%<packagefrozen>\fi
%</package|packagefrozen>
%    \end{macrocode}
%
% \subsection{AtBeginDvi hook support}
%
%    The material of box \cs{@begindvibox} is recorded in parallel
%    in box \cs{ps@begindvibox}.
%    \begin{macrocode}
%<*packagefrozen>
\newbox\ps@begindvibox
\ifvoid\@begindvibox
\else
  \global\setbox\ps@begindvibox\vbox{%
    \unvbox\@begindvibox
  }%
\fi
\let\ps@org@AtBeginDvi\AtBeginDvi
\def\AtBeginDvi#1{%
  \global\setbox\ps@begindvibox\vbox{%
    \unvbox\ps@begindvibox
    #1%
  }%
  \ps@org@AtBeginDvi{#1}%
}
%    \end{macrocode}
%
%    \begin{macro}{\ps@begindvi}
%    Macro \cs{ps@begindvi} is called the similar way as \cs{@begindvi}.
%    If the first page is printed, then \cs{AtBeginDvi} should work
%    as usual. Otherwise the contents of box \cs{ps@begindvibox} is
%    set on the first selected page.
%    \begin{macrocode}
\def\ps@begindvi{%
  \ifx\ps@next\@empty
    \global\let\ps@begindvi\@empty
  \else
    \global\let\ps@begindvi\ps@begindvi@do
  \fi
}
\def\ps@begindvi@do{%
  \ifx\ps@next\@empty
    \setbox\@cclv\vbox{%
      \unvbox\ps@begindvibox
      \box\@cclv
    }%
    \global\let\ps@begindvi\@empty
  \fi
}
%    \end{macrocode}
%    \end{macro}
%
%    \begin{macrocode}
%</packagefrozen>
%    \end{macrocode}
%
% \section{Installation}
%
% \subsection{Download}
%
% \paragraph{Package.} This package is available on
% CTAN\footnote{\CTANpkg{pagesel}}:
% \begin{description}
% \item[\CTAN{macros/latex/contrib/pagesel/pagesel.dtx}] The source file.
% \item[\CTAN{macros/latex/contrib/pagesel/pagesel.pdf}] Documentation.
% \end{description}
%
%
%
% \subsection{Package installation}
%
% \paragraph{Unpacking.} The \xfile{.dtx} file is a self-extracting
% \docstrip\ archive. The files are extracted by running the
% \xfile{.dtx} through \plainTeX:
% \begin{quote}
%   \verb|tex pagesel.dtx|
% \end{quote}
%
% \paragraph{TDS.} Now the different files must be moved into
% the different directories in your installation TDS tree
% (also known as \xfile{texmf} tree):
% \begin{quote}
% \def\t{^^A
% \begin{tabular}{@{}>{\ttfamily}l@{ $\rightarrow$ }>{\ttfamily}l@{}}
%   pagesel.sty & tex/latex/pagesel/pagesel.sty\\
%   pagesel.pdf & doc/latex/pagesel/pagesel.pdf\\
%   pagesel.dtx & source/latex/pagesel/pagesel.dtx\\
% \end{tabular}^^A
% }^^A
% \sbox0{\t}^^A
% \ifdim\wd0>\linewidth
%   \begingroup
%     \advance\linewidth by\leftmargin
%     \advance\linewidth by\rightmargin
%   \edef\x{\endgroup
%     \def\noexpand\lw{\the\linewidth}^^A
%   }\x
%   \def\lwbox{^^A
%     \leavevmode
%     \hbox to \linewidth{^^A
%       \kern-\leftmargin\relax
%       \hss
%       \usebox0
%       \hss
%       \kern-\rightmargin\relax
%     }^^A
%   }^^A
%   \ifdim\wd0>\lw
%     \sbox0{\small\t}^^A
%     \ifdim\wd0>\linewidth
%       \ifdim\wd0>\lw
%         \sbox0{\footnotesize\t}^^A
%         \ifdim\wd0>\linewidth
%           \ifdim\wd0>\lw
%             \sbox0{\scriptsize\t}^^A
%             \ifdim\wd0>\linewidth
%               \ifdim\wd0>\lw
%                 \sbox0{\tiny\t}^^A
%                 \ifdim\wd0>\linewidth
%                   \lwbox
%                 \else
%                   \usebox0
%                 \fi
%               \else
%                 \lwbox
%               \fi
%             \else
%               \usebox0
%             \fi
%           \else
%             \lwbox
%           \fi
%         \else
%           \usebox0
%         \fi
%       \else
%         \lwbox
%       \fi
%     \else
%       \usebox0
%     \fi
%   \else
%     \lwbox
%   \fi
% \else
%   \usebox0
% \fi
% \end{quote}
% If you have a \xfile{docstrip.cfg} that configures and enables \docstrip's
% TDS installing feature, then some files can already be in the right
% place, see the documentation of \docstrip.
%
% \subsection{Refresh file name databases}
%
% If your \TeX~distribution
% (\TeX\,Live, \mikTeX, \dots) relies on file name databases, you must refresh
% these. For example, \TeX\,Live\ users run \verb|texhash| or
% \verb|mktexlsr|.
%
% \subsection{Some details for the interested}
%
% \paragraph{Unpacking with \LaTeX.}
% The \xfile{.dtx} chooses its action depending on the format:
% \begin{description}
% \item[\plainTeX:] Run \docstrip\ and extract the files.
% \item[\LaTeX:] Generate the documentation.
% \end{description}
% If you insist on using \LaTeX\ for \docstrip\ (really,
% \docstrip\ does not need \LaTeX), then inform the autodetect routine
% about your intention:
% \begin{quote}
%   \verb|latex \let\install=y% \iffalse meta-comment
%
% File: pagesel.dtx
% Version: 2020-08-03 v1.10
% Info: Select pages of a document for output
%
% Copyright (C)
%    1999, 2003, 2006-2008 Heiko Oberdiek
%    2016-2020 Oberdiek Package Support Group
%    https://github.com/ho-tex/pagesel/issues
%
% This work may be distributed and/or modified under the
% conditions of the LaTeX Project Public License, either
% version 1.3c of this license or (at your option) any later
% version. This version of this license is in
%    https://www.latex-project.org/lppl/lppl-1-3c.txt
% and the latest version of this license is in
%    https://www.latex-project.org/lppl.txt
% and version 1.3 or later is part of all distributions of
% LaTeX version 2005/12/01 or later.
%
% This work has the LPPL maintenance status "maintained".
%
% The Current Maintainers of this work are
% Heiko Oberdiek and the Oberdiek Package Support Group
% https://github.com/ho-tex/pagesel/issues
%
% This work consists of the main source file pagesel.dtx
% and the derived files
%    pagesel.sty, pagesel-2016-05-16.sty,
%    pagesel.pdf, pagesel.ins, pagesel.drv.
%
% Distribution:
%    CTAN:macros/latex/contrib/pagesel/pagesel.dtx
%    CTAN:macros/latex/contrib/pagesel/pagesel.pdf
%
% Unpacking:
%    (a) If pagesel.ins is present:
%           tex pagesel.ins
%    (b) Without pagesel.ins:
%           tex pagesel.dtx
%    (c) If you insist on using LaTeX
%           latex \let\install=y\input{pagesel.dtx}
%        (quote the arguments according to the demands of your shell)
%
% Documentation:
%    (a) If pagesel.drv is present:
%           latex pagesel.drv
%    (b) Without pagesel.drv:
%           latex pagesel.dtx; ...
%    The class ltxdoc loads the configuration file ltxdoc.cfg
%    if available. Here you can specify further options, e.g.
%    use A4 as paper format:
%       \PassOptionsToClass{a4paper}{article}
%
%    Programm calls to get the documentation (example):
%       pdflatex pagesel.dtx
%       makeindex -s gind.ist pagesel.idx
%       pdflatex pagesel.dtx
%       makeindex -s gind.ist pagesel.idx
%       pdflatex pagesel.dtx
%
% Installation:
%    TDS:tex/latex/pagesel/pagesel.sty
%    TDS:doc/latex/pagesel/pagesel.pdf
%    TDS:source/latex/pagesel/pagesel.dtx
%
%<*ignore>
\begingroup
  \catcode123=1 %
  \catcode125=2 %
  \def\x{LaTeX2e}%
\expandafter\endgroup
\ifcase 0\ifx\install y1\fi\expandafter
         \ifx\csname processbatchFile\endcsname\relax\else1\fi
         \ifx\fmtname\x\else 1\fi\relax
\else\csname fi\endcsname
%</ignore>
%<*install>
\input docstrip.tex
\Msg{************************************************************************}
\Msg{* Installation}
\Msg{* Package: pagesel 2020-08-03 v1.10 Select pages of a document for output (HO)}
\Msg{************************************************************************}

\keepsilent
\askforoverwritefalse

\let\MetaPrefix\relax
\preamble

This is a generated file.

Project: pagesel
Version: 2020-08-03 v1.10

Copyright (C)
   1999, 2003, 2006-2008 Heiko Oberdiek
   2016-2020 Oberdiek Package Support Group

This work may be distributed and/or modified under the
conditions of the LaTeX Project Public License, either
version 1.3c of this license or (at your option) any later
version. This version of this license is in
   https://www.latex-project.org/lppl/lppl-1-3c.txt
and the latest version of this license is in
   https://www.latex-project.org/lppl.txt
and version 1.3 or later is part of all distributions of
LaTeX version 2005/12/01 or later.

This work has the LPPL maintenance status "maintained".

The Current Maintainers of this work are
Heiko Oberdiek and the Oberdiek Package Support Group
https://github.com/ho-tex/pagesel/issues


This work consists of the main source file pagesel.dtx
and the derived files
   pagesel.sty, pagesel-2016-05-16.sty, pagesel.pdf,
   pagesel.ins, pagesel.drv.

\endpreamble
\let\MetaPrefix\DoubleperCent

\generate{%
  \file{pagesel.ins}{\from{pagesel.dtx}{install}}%
  \file{pagesel.drv}{\from{pagesel.dtx}{driver}}%
  \usedir{tex/latex/pagesel}%
  \file{pagesel.sty}{\from{pagesel.dtx}{package}}%
  \file{pagesel-2016-05-16.sty}{\from{pagesel.dtx}{packagefrozen}}
}

\catcode32=13\relax% active space
\let =\space%
\Msg{************************************************************************}
\Msg{*}
\Msg{* To finish the installation you have to move the following}
\Msg{* file into a directory searched by TeX:}
\Msg{*}
\Msg{*     pagesel.sty}
\Msg{*}
\Msg{* To produce the documentation run the file `pagesel.drv'}
\Msg{* through LaTeX.}
\Msg{*}
\Msg{* Happy TeXing!}
\Msg{*}
\Msg{************************************************************************}

\endbatchfile
%</install>
%<*ignore>
\fi
%</ignore>
%<*driver>
\NeedsTeXFormat{LaTeX2e}
\ProvidesFile{pagesel.drv}%
  [2020-08-03 v1.10 Select pages of a document for output (HO)]%
\documentclass{ltxdoc}
\usepackage{holtxdoc}[2011/11/22]
\begin{document}
  \DocInput{pagesel.dtx}%
\end{document}
%</driver>
% \fi
%
%
%
% \GetFileInfo{pagesel.drv}
%
% \title{The \xpackage{pagesel} package}
% \date{2020-08-03 v1.10}
% \author{Heiko Oberdiek\thanks
% {Please report any issues at \url{https://github.com/ho-tex/pagesel/issues}}}
%
% \maketitle
%
% \begin{abstract}
% Single pages or page areas can be selected for output.
% \end{abstract}
%
% \tableofcontents
%
% \newenvironment{param}{^^A
%   \newcommand{\entry}[1]{\meta{\###1}:&}^^A
%   \begin{tabular}[t]{@{}l@{ }l@{}}^^A
% }{^^A
%   \end{tabular}^^A
% }
%
% \newcommand*{\Option}[1]{\textsf{#1}}
%
% \section{Usage}
%    The package \Package{pagesel} is a \LaTeXe\ package:
%    \begin{quote}
%      |\usepackage|\oarg{options}|{pagesel}|
%    \end{quote}
%    (For plain\TeX\ and \LaTeX\,2.09 the similar package
%    \URL{\Package{selectp}}^^A
%    {https://ctan.org/pkg/selectp}
%    from \NameEmail{Donald Arsenau}{asnd@triumf.ca} can be used.)
%
%    Depending on the options the package works in two modes:
%    \begin{enumerate}
%    \item If no page selecting option is present, so the package
%          ignores the other options and finishes itself. So no
%          page will be suppressed by the package and auxiliary files
%          will be written.
%    \item With at least one page selecting option the specified
%          pages are selected and the other are suppressed.
%          The default for this mode is that auxiliary will not be
%          overwritten. (This can be changed by an option.)
%    \end{enumerate}
%
% \subsection{Page selecting}
%    The package \Package{pagesel} sets up a new counter that is
%    incremented by each \cmd{\shipout}.
%    In this way the package counts the output pages regardless the value
%    of the page counter. So each page can individually by addressed,
%    even if there are several pages with the same page number.
%
% \subsubsection{Options\texorpdfstring{ for selecting pages}{}}
%    \begin{description}
%    \item[\Option{odd}:] The output pages must have an odd number.
%         All even output pages are suppressed. If there are no
%         page areas specified so all odd pages are print. With
%         page areas only the odd pages in this areas are selected.
%    \item[\Option{even}:] The opposite of option \Option{odd}.
%    \item[Page area:] A page area consists of three elements:
%         the starting output page number, an ``area'' hyphen, and
%         the output page number of the last page in this area.
%         Each component is optional, so there are four kinds
%         to spezify a page area:
%         \begin{description}
%         \item[\meta{m}\Option{-}\meta{n}:] All pages between
%              \meta{m} and \meta{n} inclusive.
%         \item[\Option{-}\meta{n}:] All pages until \meta{n} inclusive.
%         \item[\meta{m}\Option{-}:] The page area starts with \meta{m}
%              and all pages to the end of document are selected.
%         \item[\Option{-}:] All pages (not very useful).
%         \item[\meta{s}:] The single page \meta{s}.
%         \end{description}
%    \end{description}
%
% \subsubsection{Examples}
%    \newcommand*{\exam}[1]{\texttt{\strut[#1]}}^^A hash-ok
%    \begin{tabular}{ll}
%      Options & Output pages\\
%      \hline
%      \exam{1, 4, 9}&  1, 4, and 9\\
%      \exam{7-10, 3}&  3, 7, 8, 9, and 10\\
%      \exam{odd, 3-6}& 3, and 5\\
%      \exam{-4, 3, even, 7-8}& 2, 4, and 8\\
%    \end{tabular}
%
% \subsection{Auxiliary files}
%    If a page is suppressed, the \cmd{\write} commands are not
%    performed. Labels, index entries, or entries for the
%    table of contents aren't written. So it is likely that
%    the table of contents, registers, and lists are incomplete.
% \subsubsection{Options\texorpdfstring{ for handling auxiliary files}{}}
%    \begin{description}
%    \item[\Option{nofiles}:] This is the default. Auxiliary files are
%         read but not written or changed. Also the job is aborted
%         after the last selected page for saving time.
%    \item[\Option{nonofiles}/\Option{files}:] Auxiliary files are
%         written.
%    \end{description}
% \subsubsection{\texorpdfstring{Package }{}\Package{hyperref}}
%    In old versions of \Package{hyperref} [1999/04/12 v6.55] (and below)
%    there is a bug with \cmd{\nofiles}:
%    \begin{itemize}
%    \item Some ``garbage'' appears on terminal and in the log file.
%          This is harmless and can be ignored.
%    \item The outline auxiliary file \cmd{\jobname.out}, however,
%          is opened and truncated to zero bytes.
%          Version 1.0 of this package had
%          loaded a patch file \File{hypnofil.tex}, if it detects
%          \Package{hyperref} to get \cmd{\nofiles} work.
%
%          With the new version of \Package{hyperref} [1999/04/13 v6.56]
%          \cmd{\nofiles} works now. Therefore the workaround code
%          is no longer needed and removed.
%    \end{itemize}
%
% \StopEventually{
% }
%
% \section{Implementation}
%    \begin{macrocode}
%<*package>
%    \end{macrocode}
% \subsection{New implementation using the LaTeX kernel hooks}
%    \begin{macrocode}
\NeedsTeXFormat{LaTeX2e}
\ProvidesPackage{pagesel}
  [2020-08-03 v1.10 Select pages of a document for output (HO)]%
%    \end{macrocode}
%    \begin{macrocode}
\providecommand\IfFormatAtLeastTF{\@ifl@t@r\fmtversion}
\IfFormatAtLeastTF{2020/10/01}{}{\input{pagesel-2016-05-16.sty}}
\IfFormatAtLeastTF{2020/10/01}{}{\endinput}

%    \end{macrocode}
%    If the package is loaded twice, the package code does not
%    work. So stop loading the package, if it is already loaded.
%    \begin{macrocode}
\@ifundefined{ps@oddpages}{}{%
  \PackageWarningNoLine{pagesel}{Package already loaded.}%
  \endinput
}
%    \end{macrocode}
%    \begin{macrocode}
%</package>
%    \end{macrocode}
% \subsection{Package}
%    \begin{macrocode}
%<*packagefrozen>
\NeedsTeXFormat{LaTeX2e}
\ProvidesPackage{pagesel}
  [2020-08-03 v1.10 Select pages of a document for output (legacy code) (HO)]%
%    \end{macrocode}
%
%    If the package is loaded twice, the package code does not
%    work. So stop loading the package, if it is already loaded.
%    \begin{macrocode}
\@ifundefined{ps@makevoid}{}{%
  \PackageWarningNoLine{pagesel}{Package already loaded.}%
  \endinput
}
%    \end{macrocode}
%
%    \begin{macro}{\ps@makevoid}
%    Macro \cmd{\ps@makevoid} clears the output box. Because
%    nothing is shipped out and this is intended, we reduce
%    the counter \cmd{\deadcycles} in order to avoid problems, if
%    more than \cmd{\maxdeadcycles} pages are omitted.
%    \begin{macrocode}
\newcommand*{\ps@makevoid}{%
  \global\setbox\@cclv\copy\voidb@x
  \begingroup
    \count@=\deadcycles
    \advance\count@ by -1\relax
    \deadcycles=\count@
  \endgroup
}
%</packagefrozen>
%    \end{macrocode}
%    \end{macro}
%
%    \begin{macro}{\ps@oddpages}
%    \begin{macrocode}
%<*package|packagefrozen>
\newcommand*\ps@oddpages{0}
\DeclareOption{odd}{\renewcommand*\ps@oddpages{1}}
\DeclareOption{even}{\renewcommand*\ps@oddpages{2}}
%    \end{macrocode}
%    \end{macro}
%
%    \begin{macrocode}
\DeclareOption{nofiles}{\let\ps@nofiles\nofiles}
\DeclareOption{nonofiles}{\let\ps@nofiles\@empty}
\DeclareOption{files}{\let\ps@nofiles\@empty}
\ExecuteOptions{nofiles}
%    \end{macrocode}
%
%    \begin{macrocode}
\DeclareOption*{%
  \begingroup
    \expandafter\ps@checkoption\CurrentOption-\END
    \edef\x{\endgroup\noexpand\ps@store{\ps@first}{\ps@last}}%
  \x
}
%    \end{macrocode}
%
%    \begin{macro}{\ps@checkoption}
%    \begin{macrocode}
\newcommand\ps@checkoption{}
\def\ps@checkoption#1-#2\END{%
  \ifx\\#2\\%
    \ifx\\#1\\%
      % empty option
      \def\ps@first{\maxdimen}%
      \def\ps@last{\maxdimen}%
    \else
      \edef\ps@first{#1}%
      \edef\ps@last{#1}%
    \fi
  \else
    \ifx\\#1\\%
      \def\ps@first{-\maxdimen}%
    \else
      \edef\ps@first{#1}%
    \fi
    \ps@checklast#2%
  \fi
}
%    \end{macrocode}
%    \end{macro}
%
%    \begin{macro}{\ps@checklast}
%    \begin{macrocode}
\newcommand\ps@checklast{}
\def\ps@checklast#1-{%
  \ifx\\#1\\%
    \def\ps@last{\maxdimen}%
  \else
    \edef\ps@last{#1}%
  \fi
}
%    \end{macrocode}
%    \end{macro}
%
%    \begin{macro}{\ps@store}
%    \begin{macrocode}
\newcommand*{\ps@store}[2]{%
  \expandafter\def\expandafter\ps@testlist\expandafter{%
    \ps@testlist\ps@pagetest{#1}{#2}%
  }%
}
%    \end{macrocode}
%    \end{macro}
%
%    \begin{macro}{\ps@testlist}
%    \begin{macrocode}
\newcommand*\ps@testlist{}
%    \end{macrocode}
%    \end{macro}
%
%    \begin{macrocode}
\ProcessOptions
%    \end{macrocode}
%
%    \begin{macrocode}
\begingroup
  \edef\x{%
    \ifnum\ps@oddpages>0 \relax\fi
    \ifx\ps@testlist\@empty\else\relax\fi
  }%
  \ifx\x\@empty
    \endgroup
    \PackageInfo{pagesel}{Nothing to do}%
    \expandafter\endinput
  \fi
\endgroup
%    \end{macrocode}
%
%    \begin{macrocode}
%</package|packagefrozen>
%<*packagefrozen>
\RequirePackage{everyshi}
%</packagefrozen>
%    \end{macrocode}
%
%    \begin{macrocode}
%<*package|packagefrozen>
\ps@nofiles
%    \end{macrocode}
%
%    \begin{macro}{\c@ps@count}
%    \begin{macrocode}
\newcounter{ps@count}
\setcounter{ps@count}{0}
%    \end{macrocode}
%    \end{macro}
%
%    \begin{macro}{\ps@ReturnAfterElseFi}
%    \begin{macro}{\ps@ReturnAfterFi}
%    \begin{macrocode}
\long\def\ps@ReturnAfterElseFi#1\else#2\fi{\fi#1}
\long\def\ps@ReturnAfterFi#1\fi{\fi#1}
%    \end{macrocode}
%    \end{macro}
%    \end{macro}
%
%    \begin{macrocode}
\newcommand{\ps@lastpage}{\maxdimen}
\ifx\ps@nofiles\nofiles
  \ifx\ps@testlist\@empty
  \else
    \def\ps@lastpage{0}%
    \newcommand*{\ps@pagetest}[2]{%
      \ifnum#2>\ps@lastpage\relax
        \def\ps@lastpage{#2}%
      \fi
    }%
    \ps@testlist
    \let\ps@pagetest\relax
  \fi
\fi
%    \end{macrocode}
%
%    \begin{macro}{\ps@ifinset}
%    \begin{macrocode}
\newcommand*{\ps@ifinset}[4]{%
  \ifnum#1>\value{ps@count}%
    \ps@ReturnAfterElseFi{#4}%
  \else
    \ps@ReturnAfterFi{%
      \ifnum#2<\value{ps@count}%
        \ps@ReturnAfterElseFi{#4}%
      \else
        \ps@ReturnAfterFi{#3}%
      \fi
    }%
  \fi
}
%    \end{macrocode}
%    \end{macro}
%
%    \begin{macro}{\ps@pagetest}
%    \begin{macrocode}
\newcommand*{\ps@pagetest}[2]{%
  \ps@ifinset{#1}{#2}{\let\ps@next\@empty}{}%
}
%    \end{macrocode}
%    \end{macro}
%
%    \begin{macrocode}
%</package|packagefrozen>
%<packagefrozen>\EveryShipout{%
%<package>\AddToHook{shipout/before}{%
%<*package|packagefrozen>
  \stepcounter{ps@count}%
  \ifnum\value{ps@count}>\ps@lastpage\relax
    \global\output{%
      \ps@cleanup@if
      \ps@group@message
      \typeout{%
        Package pagesel Notice: Aborting LaTeX job %
        after last selected page (\ps@lastpage).%
      }%
      \ps@message@ignore
      \global\setbox\@cclv\box\voidb@x
      \deadcycles0\relax
%    \end{macrocode}
%    First leave the output group before ending the job.
%    \begin{macrocode}
      \aftergroup\@@end
    }%
  \fi
  \let\ps@next\@empty
  \ifx\ps@testlist\@empty
  \else
%<packagefrozen>    \let\ps@next\ps@makevoid
%<package>    \let\ps@next\DiscardShipoutBox
    \ps@testlist
  \fi
  \ifnum\ps@oddpages=1 %
    \ifodd\value{ps@count}%
    \else
%<packagefrozen>    \let\ps@next\ps@makevoid
%<package>    \let\ps@next\DiscardShipoutBox
    \fi
  \fi
  \ifnum\ps@oddpages=2 %
    \ifodd\value{ps@count}%
%<packagefrozen>    \let\ps@next\ps@makevoid
%<package>    \let\ps@next\DiscardShipoutBox
    \else
    \fi
  \fi
%<packagefrozen>  \ps@begindvi
  \ps@next
}
%</package|packagefrozen>
%    \end{macrocode}
%
%    \begin{macrocode}
%<*package|packagefrozen>
%<packagefrozen>\begingroup\expandafter\expandafter\expandafter\endgroup
%<packagefrozen>\expandafter\ifx\csname currentiflevel\endcsname\relax
%<packagefrozen>  \let\ps@cleanup@if\@empty
%<packagefrozen>\else
  \def\ps@cleanup@if{%
    \ifnum\currentiflevel>\@ne
      \csname fi\endcsname
      \expandafter\ps@cleanup@if
    \fi
  }%
%<packagefrozen>\fi
%    \end{macrocode}
%    Because of \cs{aftergroup} it is too dangerous to perform
%    a similar cleanup for groups.
%    \begin{macrocode}
%<packagefrozen> \begingroup\expandafter\expandafter\expandafter\endgroup
%<packagefrozen> \expandafter\ifx\csname currentgrouplevel\endcsname\relax
%<packagefrozen>  \let\ps@group@message\@empty
%<packagefrozen>  \def\ps@message@ignore{%
%<packagefrozen>    \typeout{%
%<packagefrozen>      (pagesel) \space\space\@spaces\@spaces\@spaces
%<packagefrozen>      Messages (\string\end\space occurred ...) can be ignored.%
%<packagefrozen>    }%
%<packagefrozen>  }%
%<packagefrozen>\else
  \def\ps@group@message{%
    \ifnum\currentgrouplevel>\@ne
      \def\ps@message@ignore{%
        \typeout{%
          (pagesel) \space\space\@spaces\@spaces\@spaces
          Message (\string\end\space occurred ...) %
          can be ignored.%
        }%
      }%
    \else
      \let\ps@message@ignore\@empty
    \fi
  }%
%<packagefrozen>\fi
%</package|packagefrozen>
%    \end{macrocode}
%
% \subsection{AtBeginDvi hook support}
%
%    The material of box \cs{@begindvibox} is recorded in parallel
%    in box \cs{ps@begindvibox}.
%    \begin{macrocode}
%<*packagefrozen>
\newbox\ps@begindvibox
\ifvoid\@begindvibox
\else
  \global\setbox\ps@begindvibox\vbox{%
    \unvbox\@begindvibox
  }%
\fi
\let\ps@org@AtBeginDvi\AtBeginDvi
\def\AtBeginDvi#1{%
  \global\setbox\ps@begindvibox\vbox{%
    \unvbox\ps@begindvibox
    #1%
  }%
  \ps@org@AtBeginDvi{#1}%
}
%    \end{macrocode}
%
%    \begin{macro}{\ps@begindvi}
%    Macro \cs{ps@begindvi} is called the similar way as \cs{@begindvi}.
%    If the first page is printed, then \cs{AtBeginDvi} should work
%    as usual. Otherwise the contents of box \cs{ps@begindvibox} is
%    set on the first selected page.
%    \begin{macrocode}
\def\ps@begindvi{%
  \ifx\ps@next\@empty
    \global\let\ps@begindvi\@empty
  \else
    \global\let\ps@begindvi\ps@begindvi@do
  \fi
}
\def\ps@begindvi@do{%
  \ifx\ps@next\@empty
    \setbox\@cclv\vbox{%
      \unvbox\ps@begindvibox
      \box\@cclv
    }%
    \global\let\ps@begindvi\@empty
  \fi
}
%    \end{macrocode}
%    \end{macro}
%
%    \begin{macrocode}
%</packagefrozen>
%    \end{macrocode}
%
% \section{Installation}
%
% \subsection{Download}
%
% \paragraph{Package.} This package is available on
% CTAN\footnote{\CTANpkg{pagesel}}:
% \begin{description}
% \item[\CTAN{macros/latex/contrib/pagesel/pagesel.dtx}] The source file.
% \item[\CTAN{macros/latex/contrib/pagesel/pagesel.pdf}] Documentation.
% \end{description}
%
%
%
% \subsection{Package installation}
%
% \paragraph{Unpacking.} The \xfile{.dtx} file is a self-extracting
% \docstrip\ archive. The files are extracted by running the
% \xfile{.dtx} through \plainTeX:
% \begin{quote}
%   \verb|tex pagesel.dtx|
% \end{quote}
%
% \paragraph{TDS.} Now the different files must be moved into
% the different directories in your installation TDS tree
% (also known as \xfile{texmf} tree):
% \begin{quote}
% \def\t{^^A
% \begin{tabular}{@{}>{\ttfamily}l@{ $\rightarrow$ }>{\ttfamily}l@{}}
%   pagesel.sty & tex/latex/pagesel/pagesel.sty\\
%   pagesel.pdf & doc/latex/pagesel/pagesel.pdf\\
%   pagesel.dtx & source/latex/pagesel/pagesel.dtx\\
% \end{tabular}^^A
% }^^A
% \sbox0{\t}^^A
% \ifdim\wd0>\linewidth
%   \begingroup
%     \advance\linewidth by\leftmargin
%     \advance\linewidth by\rightmargin
%   \edef\x{\endgroup
%     \def\noexpand\lw{\the\linewidth}^^A
%   }\x
%   \def\lwbox{^^A
%     \leavevmode
%     \hbox to \linewidth{^^A
%       \kern-\leftmargin\relax
%       \hss
%       \usebox0
%       \hss
%       \kern-\rightmargin\relax
%     }^^A
%   }^^A
%   \ifdim\wd0>\lw
%     \sbox0{\small\t}^^A
%     \ifdim\wd0>\linewidth
%       \ifdim\wd0>\lw
%         \sbox0{\footnotesize\t}^^A
%         \ifdim\wd0>\linewidth
%           \ifdim\wd0>\lw
%             \sbox0{\scriptsize\t}^^A
%             \ifdim\wd0>\linewidth
%               \ifdim\wd0>\lw
%                 \sbox0{\tiny\t}^^A
%                 \ifdim\wd0>\linewidth
%                   \lwbox
%                 \else
%                   \usebox0
%                 \fi
%               \else
%                 \lwbox
%               \fi
%             \else
%               \usebox0
%             \fi
%           \else
%             \lwbox
%           \fi
%         \else
%           \usebox0
%         \fi
%       \else
%         \lwbox
%       \fi
%     \else
%       \usebox0
%     \fi
%   \else
%     \lwbox
%   \fi
% \else
%   \usebox0
% \fi
% \end{quote}
% If you have a \xfile{docstrip.cfg} that configures and enables \docstrip's
% TDS installing feature, then some files can already be in the right
% place, see the documentation of \docstrip.
%
% \subsection{Refresh file name databases}
%
% If your \TeX~distribution
% (\TeX\,Live, \mikTeX, \dots) relies on file name databases, you must refresh
% these. For example, \TeX\,Live\ users run \verb|texhash| or
% \verb|mktexlsr|.
%
% \subsection{Some details for the interested}
%
% \paragraph{Unpacking with \LaTeX.}
% The \xfile{.dtx} chooses its action depending on the format:
% \begin{description}
% \item[\plainTeX:] Run \docstrip\ and extract the files.
% \item[\LaTeX:] Generate the documentation.
% \end{description}
% If you insist on using \LaTeX\ for \docstrip\ (really,
% \docstrip\ does not need \LaTeX), then inform the autodetect routine
% about your intention:
% \begin{quote}
%   \verb|latex \let\install=y\input{pagesel.dtx}|
% \end{quote}
% Do not forget to quote the argument according to the demands
% of your shell.
%
% \paragraph{Generating the documentation.}
% You can use both the \xfile{.dtx} or the \xfile{.drv} to generate
% the documentation. The process can be configured by the
% configuration file \xfile{ltxdoc.cfg}. For instance, put this
% line into this file, if you want to have A4 as paper format:
% \begin{quote}
%   \verb|\PassOptionsToClass{a4paper}{article}|
% \end{quote}
% An example follows how to generate the
% documentation with pdf\LaTeX:
% \begin{quote}
%\begin{verbatim}
%pdflatex pagesel.dtx
%makeindex -s gind.ist pagesel.idx
%pdflatex pagesel.dtx
%makeindex -s gind.ist pagesel.idx
%pdflatex pagesel.dtx
%\end{verbatim}
% \end{quote}
%
% \begin{History}
%   \begin{Version}{1999/03/01 v0.9}
%   \item
%     The first version was built as a response to a question
%     of \NameEmail{Dirk Kuypers}{dk@comnets.rwth-aachen.de},
%     published in the newsgroup
%     \href{news:de.comp.text.tex}{de.comp.text.tex}:\\
%     \URL{``\link{Re: pdflatex nur fuer bestimmte Seiten?!?}''}^^A
%     {https://groups.google.com/group/de.comp.text.tex/msg/6b68c7b3439fb658}
%   \end{Version}
%   \begin{Version}{1999/04/05 v1.0}
%   \item
%     Documentation added in dtx format.
%   \item
%     Copyright: LPPL (\CTAN{macros/latex/base/lppl.txt})
%   \item
%     Options |odd|, |even| added.
%   \item
%     \cmd{\nofiles} added, bug fix for \Package{hyperref}.
%   \item
%     Abort loading of package, if nothing to do.
%   \end{Version}
%   \begin{Version}{1999/04/13 v1.1}
%   \item
%     \cs{nofiles} bug fix removed
%     because of \xpackage{hyperref} 6.55.
%   \item
%     First CTAN release.
%   \end{Version}
%   \begin{Version}{2003/06/05 v1.2}
%   \item
%     \cs{deadcyles} is decremented for omitted pages.
%   \item
%     LPPL 1.2.
%   \end{Version}
%   \begin{Version}{2006/02/20 v1.3}
%   \item
%     Code is not changed.
%   \item
%     New DTX framework.
%   \item
%     LPPL 1.3
%   \end{Version}
%   \begin{Version}{2006/03/02 v1.4}
%   \item
%     Support for \cs{AtBeginDvi} added.
%   \end{Version}
%   \begin{Version}{2006/03/07 v1.5}
%   \item
%     Job is aborted after last selected page.
%   \end{Version}
%   \begin{Version}{2007/04/11 v1.6}
%   \item
%     Line ends sanitized.
%   \end{Version}
%   \begin{Version}{2007/04/12 v1.7}
%   \item
%     Hard coded box number 255 replaced by macro \cs{@cclv}.
%   \end{Version}
%   \begin{Version}{2008/08/11 v1.8}
%   \item
%     Code is not changed.
%   \item
%     URL updated from \texttt{www.dejanews.com}
%     to \texttt{groups.google.com}.
%   \end{Version}
%   \begin{Version}{2016/05/16 v1.9}
%   \item
%     Documentation updates.
%   \end{Version}
%   \begin{Version}{2020-08-03 v1.10}
%   \item Updated to follow the changes in the hook management
%   of LaTeX 2020/10/01
%   \end{Version}
% \end{History}
%
% \PrintIndex
%
% \Finale
\endinput
|
% \end{quote}
% Do not forget to quote the argument according to the demands
% of your shell.
%
% \paragraph{Generating the documentation.}
% You can use both the \xfile{.dtx} or the \xfile{.drv} to generate
% the documentation. The process can be configured by the
% configuration file \xfile{ltxdoc.cfg}. For instance, put this
% line into this file, if you want to have A4 as paper format:
% \begin{quote}
%   \verb|\PassOptionsToClass{a4paper}{article}|
% \end{quote}
% An example follows how to generate the
% documentation with pdf\LaTeX:
% \begin{quote}
%\begin{verbatim}
%pdflatex pagesel.dtx
%makeindex -s gind.ist pagesel.idx
%pdflatex pagesel.dtx
%makeindex -s gind.ist pagesel.idx
%pdflatex pagesel.dtx
%\end{verbatim}
% \end{quote}
%
% \begin{History}
%   \begin{Version}{1999/03/01 v0.9}
%   \item
%     The first version was built as a response to a question
%     of \NameEmail{Dirk Kuypers}{dk@comnets.rwth-aachen.de},
%     published in the newsgroup
%     \href{news:de.comp.text.tex}{de.comp.text.tex}:\\
%     \URL{``\link{Re: pdflatex nur fuer bestimmte Seiten?!?}''}^^A
%     {https://groups.google.com/group/de.comp.text.tex/msg/6b68c7b3439fb658}
%   \end{Version}
%   \begin{Version}{1999/04/05 v1.0}
%   \item
%     Documentation added in dtx format.
%   \item
%     Copyright: LPPL (\CTAN{macros/latex/base/lppl.txt})
%   \item
%     Options |odd|, |even| added.
%   \item
%     \cmd{\nofiles} added, bug fix for \Package{hyperref}.
%   \item
%     Abort loading of package, if nothing to do.
%   \end{Version}
%   \begin{Version}{1999/04/13 v1.1}
%   \item
%     \cs{nofiles} bug fix removed
%     because of \xpackage{hyperref} 6.55.
%   \item
%     First CTAN release.
%   \end{Version}
%   \begin{Version}{2003/06/05 v1.2}
%   \item
%     \cs{deadcyles} is decremented for omitted pages.
%   \item
%     LPPL 1.2.
%   \end{Version}
%   \begin{Version}{2006/02/20 v1.3}
%   \item
%     Code is not changed.
%   \item
%     New DTX framework.
%   \item
%     LPPL 1.3
%   \end{Version}
%   \begin{Version}{2006/03/02 v1.4}
%   \item
%     Support for \cs{AtBeginDvi} added.
%   \end{Version}
%   \begin{Version}{2006/03/07 v1.5}
%   \item
%     Job is aborted after last selected page.
%   \end{Version}
%   \begin{Version}{2007/04/11 v1.6}
%   \item
%     Line ends sanitized.
%   \end{Version}
%   \begin{Version}{2007/04/12 v1.7}
%   \item
%     Hard coded box number 255 replaced by macro \cs{@cclv}.
%   \end{Version}
%   \begin{Version}{2008/08/11 v1.8}
%   \item
%     Code is not changed.
%   \item
%     URL updated from \texttt{www.dejanews.com}
%     to \texttt{groups.google.com}.
%   \end{Version}
%   \begin{Version}{2016/05/16 v1.9}
%   \item
%     Documentation updates.
%   \end{Version}
%   \begin{Version}{2020-08-03 v1.10}
%   \item Updated to follow the changes in the hook management
%   of LaTeX 2020/10/01
%   \end{Version}
% \end{History}
%
% \PrintIndex
%
% \Finale
\endinput

%        (quote the arguments according to the demands of your shell)
%
% Documentation:
%    (a) If pagesel.drv is present:
%           latex pagesel.drv
%    (b) Without pagesel.drv:
%           latex pagesel.dtx; ...
%    The class ltxdoc loads the configuration file ltxdoc.cfg
%    if available. Here you can specify further options, e.g.
%    use A4 as paper format:
%       \PassOptionsToClass{a4paper}{article}
%
%    Programm calls to get the documentation (example):
%       pdflatex pagesel.dtx
%       makeindex -s gind.ist pagesel.idx
%       pdflatex pagesel.dtx
%       makeindex -s gind.ist pagesel.idx
%       pdflatex pagesel.dtx
%
% Installation:
%    TDS:tex/latex/pagesel/pagesel.sty
%    TDS:doc/latex/pagesel/pagesel.pdf
%    TDS:source/latex/pagesel/pagesel.dtx
%
%<*ignore>
\begingroup
  \catcode123=1 %
  \catcode125=2 %
  \def\x{LaTeX2e}%
\expandafter\endgroup
\ifcase 0\ifx\install y1\fi\expandafter
         \ifx\csname processbatchFile\endcsname\relax\else1\fi
         \ifx\fmtname\x\else 1\fi\relax
\else\csname fi\endcsname
%</ignore>
%<*install>
\input docstrip.tex
\Msg{************************************************************************}
\Msg{* Installation}
\Msg{* Package: pagesel 2020-08-03 v1.10 Select pages of a document for output (HO)}
\Msg{************************************************************************}

\keepsilent
\askforoverwritefalse

\let\MetaPrefix\relax
\preamble

This is a generated file.

Project: pagesel
Version: 2020-08-03 v1.10

Copyright (C)
   1999, 2003, 2006-2008 Heiko Oberdiek
   2016-2020 Oberdiek Package Support Group

This work may be distributed and/or modified under the
conditions of the LaTeX Project Public License, either
version 1.3c of this license or (at your option) any later
version. This version of this license is in
   https://www.latex-project.org/lppl/lppl-1-3c.txt
and the latest version of this license is in
   https://www.latex-project.org/lppl.txt
and version 1.3 or later is part of all distributions of
LaTeX version 2005/12/01 or later.

This work has the LPPL maintenance status "maintained".

The Current Maintainers of this work are
Heiko Oberdiek and the Oberdiek Package Support Group
https://github.com/ho-tex/pagesel/issues


This work consists of the main source file pagesel.dtx
and the derived files
   pagesel.sty, pagesel-2016-05-16.sty, pagesel.pdf,
   pagesel.ins, pagesel.drv.

\endpreamble
\let\MetaPrefix\DoubleperCent

\generate{%
  \file{pagesel.ins}{\from{pagesel.dtx}{install}}%
  \file{pagesel.drv}{\from{pagesel.dtx}{driver}}%
  \usedir{tex/latex/pagesel}%
  \file{pagesel.sty}{\from{pagesel.dtx}{package}}%
  \file{pagesel-2016-05-16.sty}{\from{pagesel.dtx}{packagefrozen}}
}

\catcode32=13\relax% active space
\let =\space%
\Msg{************************************************************************}
\Msg{*}
\Msg{* To finish the installation you have to move the following}
\Msg{* file into a directory searched by TeX:}
\Msg{*}
\Msg{*     pagesel.sty}
\Msg{*}
\Msg{* To produce the documentation run the file `pagesel.drv'}
\Msg{* through LaTeX.}
\Msg{*}
\Msg{* Happy TeXing!}
\Msg{*}
\Msg{************************************************************************}

\endbatchfile
%</install>
%<*ignore>
\fi
%</ignore>
%<*driver>
\NeedsTeXFormat{LaTeX2e}
\ProvidesFile{pagesel.drv}%
  [2020-08-03 v1.10 Select pages of a document for output (HO)]%
\documentclass{ltxdoc}
\usepackage{holtxdoc}[2011/11/22]
\begin{document}
  \DocInput{pagesel.dtx}%
\end{document}
%</driver>
% \fi
%
%
%
% \GetFileInfo{pagesel.drv}
%
% \title{The \xpackage{pagesel} package}
% \date{2020-08-03 v1.10}
% \author{Heiko Oberdiek\thanks
% {Please report any issues at \url{https://github.com/ho-tex/pagesel/issues}}}
%
% \maketitle
%
% \begin{abstract}
% Single pages or page areas can be selected for output.
% \end{abstract}
%
% \tableofcontents
%
% \newenvironment{param}{^^A
%   \newcommand{\entry}[1]{\meta{\###1}:&}^^A
%   \begin{tabular}[t]{@{}l@{ }l@{}}^^A
% }{^^A
%   \end{tabular}^^A
% }
%
% \newcommand*{\Option}[1]{\textsf{#1}}
%
% \section{Usage}
%    The package \Package{pagesel} is a \LaTeXe\ package:
%    \begin{quote}
%      |\usepackage|\oarg{options}|{pagesel}|
%    \end{quote}
%    (For plain\TeX\ and \LaTeX\,2.09 the similar package
%    \URL{\Package{selectp}}^^A
%    {https://ctan.org/pkg/selectp}
%    from \NameEmail{Donald Arsenau}{asnd@triumf.ca} can be used.)
%
%    Depending on the options the package works in two modes:
%    \begin{enumerate}
%    \item If no page selecting option is present, so the package
%          ignores the other options and finishes itself. So no
%          page will be suppressed by the package and auxiliary files
%          will be written.
%    \item With at least one page selecting option the specified
%          pages are selected and the other are suppressed.
%          The default for this mode is that auxiliary will not be
%          overwritten. (This can be changed by an option.)
%    \end{enumerate}
%
% \subsection{Page selecting}
%    The package \Package{pagesel} sets up a new counter that is
%    incremented by each \cmd{\shipout}.
%    In this way the package counts the output pages regardless the value
%    of the page counter. So each page can individually by addressed,
%    even if there are several pages with the same page number.
%
% \subsubsection{Options\texorpdfstring{ for selecting pages}{}}
%    \begin{description}
%    \item[\Option{odd}:] The output pages must have an odd number.
%         All even output pages are suppressed. If there are no
%         page areas specified so all odd pages are print. With
%         page areas only the odd pages in this areas are selected.
%    \item[\Option{even}:] The opposite of option \Option{odd}.
%    \item[Page area:] A page area consists of three elements:
%         the starting output page number, an ``area'' hyphen, and
%         the output page number of the last page in this area.
%         Each component is optional, so there are four kinds
%         to spezify a page area:
%         \begin{description}
%         \item[\meta{m}\Option{-}\meta{n}:] All pages between
%              \meta{m} and \meta{n} inclusive.
%         \item[\Option{-}\meta{n}:] All pages until \meta{n} inclusive.
%         \item[\meta{m}\Option{-}:] The page area starts with \meta{m}
%              and all pages to the end of document are selected.
%         \item[\Option{-}:] All pages (not very useful).
%         \item[\meta{s}:] The single page \meta{s}.
%         \end{description}
%    \end{description}
%
% \subsubsection{Examples}
%    \newcommand*{\exam}[1]{\texttt{\strut[#1]}}^^A hash-ok
%    \begin{tabular}{ll}
%      Options & Output pages\\
%      \hline
%      \exam{1, 4, 9}&  1, 4, and 9\\
%      \exam{7-10, 3}&  3, 7, 8, 9, and 10\\
%      \exam{odd, 3-6}& 3, and 5\\
%      \exam{-4, 3, even, 7-8}& 2, 4, and 8\\
%    \end{tabular}
%
% \subsection{Auxiliary files}
%    If a page is suppressed, the \cmd{\write} commands are not
%    performed. Labels, index entries, or entries for the
%    table of contents aren't written. So it is likely that
%    the table of contents, registers, and lists are incomplete.
% \subsubsection{Options\texorpdfstring{ for handling auxiliary files}{}}
%    \begin{description}
%    \item[\Option{nofiles}:] This is the default. Auxiliary files are
%         read but not written or changed. Also the job is aborted
%         after the last selected page for saving time.
%    \item[\Option{nonofiles}/\Option{files}:] Auxiliary files are
%         written.
%    \end{description}
% \subsubsection{\texorpdfstring{Package }{}\Package{hyperref}}
%    In old versions of \Package{hyperref} [1999/04/12 v6.55] (and below)
%    there is a bug with \cmd{\nofiles}:
%    \begin{itemize}
%    \item Some ``garbage'' appears on terminal and in the log file.
%          This is harmless and can be ignored.
%    \item The outline auxiliary file \cmd{\jobname.out}, however,
%          is opened and truncated to zero bytes.
%          Version 1.0 of this package had
%          loaded a patch file \File{hypnofil.tex}, if it detects
%          \Package{hyperref} to get \cmd{\nofiles} work.
%
%          With the new version of \Package{hyperref} [1999/04/13 v6.56]
%          \cmd{\nofiles} works now. Therefore the workaround code
%          is no longer needed and removed.
%    \end{itemize}
%
% \StopEventually{
% }
%
% \section{Implementation}
%    \begin{macrocode}
%<*package>
%    \end{macrocode}
% \subsection{New implementation using the LaTeX kernel hooks}
%    \begin{macrocode}
\NeedsTeXFormat{LaTeX2e}
\ProvidesPackage{pagesel}
  [2020-08-03 v1.10 Select pages of a document for output (HO)]%
%    \end{macrocode}
%    \begin{macrocode}
\providecommand\IfFormatAtLeastTF{\@ifl@t@r\fmtversion}
\IfFormatAtLeastTF{2020/10/01}{}{%%
%% This is file `pagesel-2016-05-16.sty',
%% generated with the docstrip utility.
%%
%% The original source files were:
%%
%% pagesel.dtx  (with options: `packagefrozen')
%% 
%% This is a generated file.
%% 
%% Project: pagesel
%% Version: 2020-08-03 v1.10
%% 
%% Copyright (C)
%%    1999, 2003, 2006-2008 Heiko Oberdiek
%%    2016-2020 Oberdiek Package Support Group
%% 
%% This work may be distributed and/or modified under the
%% conditions of the LaTeX Project Public License, either
%% version 1.3c of this license or (at your option) any later
%% version. This version of this license is in
%%    https://www.latex-project.org/lppl/lppl-1-3c.txt
%% and the latest version of this license is in
%%    https://www.latex-project.org/lppl.txt
%% and version 1.3 or later is part of all distributions of
%% LaTeX version 2005/12/01 or later.
%% 
%% This work has the LPPL maintenance status "maintained".
%% 
%% The Current Maintainers of this work are
%% Heiko Oberdiek and the Oberdiek Package Support Group
%% https://github.com/ho-tex/pagesel/issues
%% 
%% This work consists of the main source file pagesel.dtx
%% and the derived files
%%    pagesel.sty, pagesel-2016-05-16.sty, pagesel.pdf,
%%    pagesel.ins, pagesel.drv.
%% 
\NeedsTeXFormat{LaTeX2e}
\ProvidesPackage{pagesel}
  [2020-08-03 v1.10 Select pages of a document for output (legacy code) (HO)]%
\@ifundefined{ps@makevoid}{}{%
  \PackageWarningNoLine{pagesel}{Package already loaded.}%
  \endinput
}
\newcommand*{\ps@makevoid}{%
  \global\setbox\@cclv\copy\voidb@x
  \begingroup
    \count@=\deadcycles
    \advance\count@ by -1\relax
    \deadcycles=\count@
  \endgroup
}
\newcommand*\ps@oddpages{0}
\DeclareOption{odd}{\renewcommand*\ps@oddpages{1}}
\DeclareOption{even}{\renewcommand*\ps@oddpages{2}}
\DeclareOption{nofiles}{\let\ps@nofiles\nofiles}
\DeclareOption{nonofiles}{\let\ps@nofiles\@empty}
\DeclareOption{files}{\let\ps@nofiles\@empty}
\ExecuteOptions{nofiles}
\DeclareOption*{%
  \begingroup
    \expandafter\ps@checkoption\CurrentOption-\END
    \edef\x{\endgroup\noexpand\ps@store{\ps@first}{\ps@last}}%
  \x
}
\newcommand\ps@checkoption{}
\def\ps@checkoption#1-#2\END{%
  \ifx\\#2\\%
    \ifx\\#1\\%
      % empty option
      \def\ps@first{\maxdimen}%
      \def\ps@last{\maxdimen}%
    \else
      \edef\ps@first{#1}%
      \edef\ps@last{#1}%
    \fi
  \else
    \ifx\\#1\\%
      \def\ps@first{-\maxdimen}%
    \else
      \edef\ps@first{#1}%
    \fi
    \ps@checklast#2%
  \fi
}
\newcommand\ps@checklast{}
\def\ps@checklast#1-{%
  \ifx\\#1\\%
    \def\ps@last{\maxdimen}%
  \else
    \edef\ps@last{#1}%
  \fi
}
\newcommand*{\ps@store}[2]{%
  \expandafter\def\expandafter\ps@testlist\expandafter{%
    \ps@testlist\ps@pagetest{#1}{#2}%
  }%
}
\newcommand*\ps@testlist{}
\ProcessOptions
\begingroup
  \edef\x{%
    \ifnum\ps@oddpages>0 \relax\fi
    \ifx\ps@testlist\@empty\else\relax\fi
  }%
  \ifx\x\@empty
    \endgroup
    \PackageInfo{pagesel}{Nothing to do}%
    \expandafter\endinput
  \fi
\endgroup
\RequirePackage{everyshi}
\ps@nofiles
\newcounter{ps@count}
\setcounter{ps@count}{0}
\long\def\ps@ReturnAfterElseFi#1\else#2\fi{\fi#1}
\long\def\ps@ReturnAfterFi#1\fi{\fi#1}
\newcommand{\ps@lastpage}{\maxdimen}
\ifx\ps@nofiles\nofiles
  \ifx\ps@testlist\@empty
  \else
    \def\ps@lastpage{0}%
    \newcommand*{\ps@pagetest}[2]{%
      \ifnum#2>\ps@lastpage\relax
        \def\ps@lastpage{#2}%
      \fi
    }%
    \ps@testlist
    \let\ps@pagetest\relax
  \fi
\fi
\newcommand*{\ps@ifinset}[4]{%
  \ifnum#1>\value{ps@count}%
    \ps@ReturnAfterElseFi{#4}%
  \else
    \ps@ReturnAfterFi{%
      \ifnum#2<\value{ps@count}%
        \ps@ReturnAfterElseFi{#4}%
      \else
        \ps@ReturnAfterFi{#3}%
      \fi
    }%
  \fi
}
\newcommand*{\ps@pagetest}[2]{%
  \ps@ifinset{#1}{#2}{\let\ps@next\@empty}{}%
}
\EveryShipout{%
  \stepcounter{ps@count}%
  \ifnum\value{ps@count}>\ps@lastpage\relax
    \global\output{%
      \ps@cleanup@if
      \ps@group@message
      \typeout{%
        Package pagesel Notice: Aborting LaTeX job %
        after last selected page (\ps@lastpage).%
      }%
      \ps@message@ignore
      \global\setbox\@cclv\box\voidb@x
      \deadcycles0\relax
      \aftergroup\@@end
    }%
  \fi
  \let\ps@next\@empty
  \ifx\ps@testlist\@empty
  \else
    \let\ps@next\ps@makevoid
    \ps@testlist
  \fi
  \ifnum\ps@oddpages=1 %
    \ifodd\value{ps@count}%
    \else
    \let\ps@next\ps@makevoid
    \fi
  \fi
  \ifnum\ps@oddpages=2 %
    \ifodd\value{ps@count}%
    \let\ps@next\ps@makevoid
    \else
    \fi
  \fi
  \ps@begindvi
  \ps@next
}
\begingroup\expandafter\expandafter\expandafter\endgroup
\expandafter\ifx\csname currentiflevel\endcsname\relax
  \let\ps@cleanup@if\@empty
\else
  \def\ps@cleanup@if{%
    \ifnum\currentiflevel>\@ne
      \csname fi\endcsname
      \expandafter\ps@cleanup@if
    \fi
  }%
\fi
 \begingroup\expandafter\expandafter\expandafter\endgroup
 \expandafter\ifx\csname currentgrouplevel\endcsname\relax
  \let\ps@group@message\@empty
  \def\ps@message@ignore{%
    \typeout{%
      (pagesel) \space\space\@spaces\@spaces\@spaces
      Messages (\string\end\space occurred ...) can be ignored.%
    }%
  }%
\else
  \def\ps@group@message{%
    \ifnum\currentgrouplevel>\@ne
      \def\ps@message@ignore{%
        \typeout{%
          (pagesel) \space\space\@spaces\@spaces\@spaces
          Message (\string\end\space occurred ...) %
          can be ignored.%
        }%
      }%
    \else
      \let\ps@message@ignore\@empty
    \fi
  }%
\fi
\newbox\ps@begindvibox
\ifvoid\@begindvibox
\else
  \global\setbox\ps@begindvibox\vbox{%
    \unvbox\@begindvibox
  }%
\fi
\let\ps@org@AtBeginDvi\AtBeginDvi
\def\AtBeginDvi#1{%
  \global\setbox\ps@begindvibox\vbox{%
    \unvbox\ps@begindvibox
    #1%
  }%
  \ps@org@AtBeginDvi{#1}%
}
\def\ps@begindvi{%
  \ifx\ps@next\@empty
    \global\let\ps@begindvi\@empty
  \else
    \global\let\ps@begindvi\ps@begindvi@do
  \fi
}
\def\ps@begindvi@do{%
  \ifx\ps@next\@empty
    \setbox\@cclv\vbox{%
      \unvbox\ps@begindvibox
      \box\@cclv
    }%
    \global\let\ps@begindvi\@empty
  \fi
}
\endinput
%%
%% End of file `pagesel-2016-05-16.sty'.
}
\IfFormatAtLeastTF{2020/10/01}{}{\endinput}

%    \end{macrocode}
%    If the package is loaded twice, the package code does not
%    work. So stop loading the package, if it is already loaded.
%    \begin{macrocode}
\@ifundefined{ps@oddpages}{}{%
  \PackageWarningNoLine{pagesel}{Package already loaded.}%
  \endinput
}
%    \end{macrocode}
%    \begin{macrocode}
%</package>
%    \end{macrocode}
% \subsection{Package}
%    \begin{macrocode}
%<*packagefrozen>
\NeedsTeXFormat{LaTeX2e}
\ProvidesPackage{pagesel}
  [2020-08-03 v1.10 Select pages of a document for output (legacy code) (HO)]%
%    \end{macrocode}
%
%    If the package is loaded twice, the package code does not
%    work. So stop loading the package, if it is already loaded.
%    \begin{macrocode}
\@ifundefined{ps@makevoid}{}{%
  \PackageWarningNoLine{pagesel}{Package already loaded.}%
  \endinput
}
%    \end{macrocode}
%
%    \begin{macro}{\ps@makevoid}
%    Macro \cmd{\ps@makevoid} clears the output box. Because
%    nothing is shipped out and this is intended, we reduce
%    the counter \cmd{\deadcycles} in order to avoid problems, if
%    more than \cmd{\maxdeadcycles} pages are omitted.
%    \begin{macrocode}
\newcommand*{\ps@makevoid}{%
  \global\setbox\@cclv\copy\voidb@x
  \begingroup
    \count@=\deadcycles
    \advance\count@ by -1\relax
    \deadcycles=\count@
  \endgroup
}
%</packagefrozen>
%    \end{macrocode}
%    \end{macro}
%
%    \begin{macro}{\ps@oddpages}
%    \begin{macrocode}
%<*package|packagefrozen>
\newcommand*\ps@oddpages{0}
\DeclareOption{odd}{\renewcommand*\ps@oddpages{1}}
\DeclareOption{even}{\renewcommand*\ps@oddpages{2}}
%    \end{macrocode}
%    \end{macro}
%
%    \begin{macrocode}
\DeclareOption{nofiles}{\let\ps@nofiles\nofiles}
\DeclareOption{nonofiles}{\let\ps@nofiles\@empty}
\DeclareOption{files}{\let\ps@nofiles\@empty}
\ExecuteOptions{nofiles}
%    \end{macrocode}
%
%    \begin{macrocode}
\DeclareOption*{%
  \begingroup
    \expandafter\ps@checkoption\CurrentOption-\END
    \edef\x{\endgroup\noexpand\ps@store{\ps@first}{\ps@last}}%
  \x
}
%    \end{macrocode}
%
%    \begin{macro}{\ps@checkoption}
%    \begin{macrocode}
\newcommand\ps@checkoption{}
\def\ps@checkoption#1-#2\END{%
  \ifx\\#2\\%
    \ifx\\#1\\%
      % empty option
      \def\ps@first{\maxdimen}%
      \def\ps@last{\maxdimen}%
    \else
      \edef\ps@first{#1}%
      \edef\ps@last{#1}%
    \fi
  \else
    \ifx\\#1\\%
      \def\ps@first{-\maxdimen}%
    \else
      \edef\ps@first{#1}%
    \fi
    \ps@checklast#2%
  \fi
}
%    \end{macrocode}
%    \end{macro}
%
%    \begin{macro}{\ps@checklast}
%    \begin{macrocode}
\newcommand\ps@checklast{}
\def\ps@checklast#1-{%
  \ifx\\#1\\%
    \def\ps@last{\maxdimen}%
  \else
    \edef\ps@last{#1}%
  \fi
}
%    \end{macrocode}
%    \end{macro}
%
%    \begin{macro}{\ps@store}
%    \begin{macrocode}
\newcommand*{\ps@store}[2]{%
  \expandafter\def\expandafter\ps@testlist\expandafter{%
    \ps@testlist\ps@pagetest{#1}{#2}%
  }%
}
%    \end{macrocode}
%    \end{macro}
%
%    \begin{macro}{\ps@testlist}
%    \begin{macrocode}
\newcommand*\ps@testlist{}
%    \end{macrocode}
%    \end{macro}
%
%    \begin{macrocode}
\ProcessOptions
%    \end{macrocode}
%
%    \begin{macrocode}
\begingroup
  \edef\x{%
    \ifnum\ps@oddpages>0 \relax\fi
    \ifx\ps@testlist\@empty\else\relax\fi
  }%
  \ifx\x\@empty
    \endgroup
    \PackageInfo{pagesel}{Nothing to do}%
    \expandafter\endinput
  \fi
\endgroup
%    \end{macrocode}
%
%    \begin{macrocode}
%</package|packagefrozen>
%<*packagefrozen>
\RequirePackage{everyshi}
%</packagefrozen>
%    \end{macrocode}
%
%    \begin{macrocode}
%<*package|packagefrozen>
\ps@nofiles
%    \end{macrocode}
%
%    \begin{macro}{\c@ps@count}
%    \begin{macrocode}
\newcounter{ps@count}
\setcounter{ps@count}{0}
%    \end{macrocode}
%    \end{macro}
%
%    \begin{macro}{\ps@ReturnAfterElseFi}
%    \begin{macro}{\ps@ReturnAfterFi}
%    \begin{macrocode}
\long\def\ps@ReturnAfterElseFi#1\else#2\fi{\fi#1}
\long\def\ps@ReturnAfterFi#1\fi{\fi#1}
%    \end{macrocode}
%    \end{macro}
%    \end{macro}
%
%    \begin{macrocode}
\newcommand{\ps@lastpage}{\maxdimen}
\ifx\ps@nofiles\nofiles
  \ifx\ps@testlist\@empty
  \else
    \def\ps@lastpage{0}%
    \newcommand*{\ps@pagetest}[2]{%
      \ifnum#2>\ps@lastpage\relax
        \def\ps@lastpage{#2}%
      \fi
    }%
    \ps@testlist
    \let\ps@pagetest\relax
  \fi
\fi
%    \end{macrocode}
%
%    \begin{macro}{\ps@ifinset}
%    \begin{macrocode}
\newcommand*{\ps@ifinset}[4]{%
  \ifnum#1>\value{ps@count}%
    \ps@ReturnAfterElseFi{#4}%
  \else
    \ps@ReturnAfterFi{%
      \ifnum#2<\value{ps@count}%
        \ps@ReturnAfterElseFi{#4}%
      \else
        \ps@ReturnAfterFi{#3}%
      \fi
    }%
  \fi
}
%    \end{macrocode}
%    \end{macro}
%
%    \begin{macro}{\ps@pagetest}
%    \begin{macrocode}
\newcommand*{\ps@pagetest}[2]{%
  \ps@ifinset{#1}{#2}{\let\ps@next\@empty}{}%
}
%    \end{macrocode}
%    \end{macro}
%
%    \begin{macrocode}
%</package|packagefrozen>
%<packagefrozen>\EveryShipout{%
%<package>\AddToHook{shipout/before}{%
%<*package|packagefrozen>
  \stepcounter{ps@count}%
  \ifnum\value{ps@count}>\ps@lastpage\relax
    \global\output{%
      \ps@cleanup@if
      \ps@group@message
      \typeout{%
        Package pagesel Notice: Aborting LaTeX job %
        after last selected page (\ps@lastpage).%
      }%
      \ps@message@ignore
      \global\setbox\@cclv\box\voidb@x
      \deadcycles0\relax
%    \end{macrocode}
%    First leave the output group before ending the job.
%    \begin{macrocode}
      \aftergroup\@@end
    }%
  \fi
  \let\ps@next\@empty
  \ifx\ps@testlist\@empty
  \else
%<packagefrozen>    \let\ps@next\ps@makevoid
%<package>    \let\ps@next\DiscardShipoutBox
    \ps@testlist
  \fi
  \ifnum\ps@oddpages=1 %
    \ifodd\value{ps@count}%
    \else
%<packagefrozen>    \let\ps@next\ps@makevoid
%<package>    \let\ps@next\DiscardShipoutBox
    \fi
  \fi
  \ifnum\ps@oddpages=2 %
    \ifodd\value{ps@count}%
%<packagefrozen>    \let\ps@next\ps@makevoid
%<package>    \let\ps@next\DiscardShipoutBox
    \else
    \fi
  \fi
%<packagefrozen>  \ps@begindvi
  \ps@next
}
%</package|packagefrozen>
%    \end{macrocode}
%
%    \begin{macrocode}
%<*package|packagefrozen>
%<packagefrozen>\begingroup\expandafter\expandafter\expandafter\endgroup
%<packagefrozen>\expandafter\ifx\csname currentiflevel\endcsname\relax
%<packagefrozen>  \let\ps@cleanup@if\@empty
%<packagefrozen>\else
  \def\ps@cleanup@if{%
    \ifnum\currentiflevel>\@ne
      \csname fi\endcsname
      \expandafter\ps@cleanup@if
    \fi
  }%
%<packagefrozen>\fi
%    \end{macrocode}
%    Because of \cs{aftergroup} it is too dangerous to perform
%    a similar cleanup for groups.
%    \begin{macrocode}
%<packagefrozen> \begingroup\expandafter\expandafter\expandafter\endgroup
%<packagefrozen> \expandafter\ifx\csname currentgrouplevel\endcsname\relax
%<packagefrozen>  \let\ps@group@message\@empty
%<packagefrozen>  \def\ps@message@ignore{%
%<packagefrozen>    \typeout{%
%<packagefrozen>      (pagesel) \space\space\@spaces\@spaces\@spaces
%<packagefrozen>      Messages (\string\end\space occurred ...) can be ignored.%
%<packagefrozen>    }%
%<packagefrozen>  }%
%<packagefrozen>\else
  \def\ps@group@message{%
    \ifnum\currentgrouplevel>\@ne
      \def\ps@message@ignore{%
        \typeout{%
          (pagesel) \space\space\@spaces\@spaces\@spaces
          Message (\string\end\space occurred ...) %
          can be ignored.%
        }%
      }%
    \else
      \let\ps@message@ignore\@empty
    \fi
  }%
%<packagefrozen>\fi
%</package|packagefrozen>
%    \end{macrocode}
%
% \subsection{AtBeginDvi hook support}
%
%    The material of box \cs{@begindvibox} is recorded in parallel
%    in box \cs{ps@begindvibox}.
%    \begin{macrocode}
%<*packagefrozen>
\newbox\ps@begindvibox
\ifvoid\@begindvibox
\else
  \global\setbox\ps@begindvibox\vbox{%
    \unvbox\@begindvibox
  }%
\fi
\let\ps@org@AtBeginDvi\AtBeginDvi
\def\AtBeginDvi#1{%
  \global\setbox\ps@begindvibox\vbox{%
    \unvbox\ps@begindvibox
    #1%
  }%
  \ps@org@AtBeginDvi{#1}%
}
%    \end{macrocode}
%
%    \begin{macro}{\ps@begindvi}
%    Macro \cs{ps@begindvi} is called the similar way as \cs{@begindvi}.
%    If the first page is printed, then \cs{AtBeginDvi} should work
%    as usual. Otherwise the contents of box \cs{ps@begindvibox} is
%    set on the first selected page.
%    \begin{macrocode}
\def\ps@begindvi{%
  \ifx\ps@next\@empty
    \global\let\ps@begindvi\@empty
  \else
    \global\let\ps@begindvi\ps@begindvi@do
  \fi
}
\def\ps@begindvi@do{%
  \ifx\ps@next\@empty
    \setbox\@cclv\vbox{%
      \unvbox\ps@begindvibox
      \box\@cclv
    }%
    \global\let\ps@begindvi\@empty
  \fi
}
%    \end{macrocode}
%    \end{macro}
%
%    \begin{macrocode}
%</packagefrozen>
%    \end{macrocode}
%
% \section{Installation}
%
% \subsection{Download}
%
% \paragraph{Package.} This package is available on
% CTAN\footnote{\CTANpkg{pagesel}}:
% \begin{description}
% \item[\CTAN{macros/latex/contrib/pagesel/pagesel.dtx}] The source file.
% \item[\CTAN{macros/latex/contrib/pagesel/pagesel.pdf}] Documentation.
% \end{description}
%
%
%
% \subsection{Package installation}
%
% \paragraph{Unpacking.} The \xfile{.dtx} file is a self-extracting
% \docstrip\ archive. The files are extracted by running the
% \xfile{.dtx} through \plainTeX:
% \begin{quote}
%   \verb|tex pagesel.dtx|
% \end{quote}
%
% \paragraph{TDS.} Now the different files must be moved into
% the different directories in your installation TDS tree
% (also known as \xfile{texmf} tree):
% \begin{quote}
% \def\t{^^A
% \begin{tabular}{@{}>{\ttfamily}l@{ $\rightarrow$ }>{\ttfamily}l@{}}
%   pagesel.sty & tex/latex/pagesel/pagesel.sty\\
%   pagesel.pdf & doc/latex/pagesel/pagesel.pdf\\
%   pagesel.dtx & source/latex/pagesel/pagesel.dtx\\
% \end{tabular}^^A
% }^^A
% \sbox0{\t}^^A
% \ifdim\wd0>\linewidth
%   \begingroup
%     \advance\linewidth by\leftmargin
%     \advance\linewidth by\rightmargin
%   \edef\x{\endgroup
%     \def\noexpand\lw{\the\linewidth}^^A
%   }\x
%   \def\lwbox{^^A
%     \leavevmode
%     \hbox to \linewidth{^^A
%       \kern-\leftmargin\relax
%       \hss
%       \usebox0
%       \hss
%       \kern-\rightmargin\relax
%     }^^A
%   }^^A
%   \ifdim\wd0>\lw
%     \sbox0{\small\t}^^A
%     \ifdim\wd0>\linewidth
%       \ifdim\wd0>\lw
%         \sbox0{\footnotesize\t}^^A
%         \ifdim\wd0>\linewidth
%           \ifdim\wd0>\lw
%             \sbox0{\scriptsize\t}^^A
%             \ifdim\wd0>\linewidth
%               \ifdim\wd0>\lw
%                 \sbox0{\tiny\t}^^A
%                 \ifdim\wd0>\linewidth
%                   \lwbox
%                 \else
%                   \usebox0
%                 \fi
%               \else
%                 \lwbox
%               \fi
%             \else
%               \usebox0
%             \fi
%           \else
%             \lwbox
%           \fi
%         \else
%           \usebox0
%         \fi
%       \else
%         \lwbox
%       \fi
%     \else
%       \usebox0
%     \fi
%   \else
%     \lwbox
%   \fi
% \else
%   \usebox0
% \fi
% \end{quote}
% If you have a \xfile{docstrip.cfg} that configures and enables \docstrip's
% TDS installing feature, then some files can already be in the right
% place, see the documentation of \docstrip.
%
% \subsection{Refresh file name databases}
%
% If your \TeX~distribution
% (\TeX\,Live, \mikTeX, \dots) relies on file name databases, you must refresh
% these. For example, \TeX\,Live\ users run \verb|texhash| or
% \verb|mktexlsr|.
%
% \subsection{Some details for the interested}
%
% \paragraph{Unpacking with \LaTeX.}
% The \xfile{.dtx} chooses its action depending on the format:
% \begin{description}
% \item[\plainTeX:] Run \docstrip\ and extract the files.
% \item[\LaTeX:] Generate the documentation.
% \end{description}
% If you insist on using \LaTeX\ for \docstrip\ (really,
% \docstrip\ does not need \LaTeX), then inform the autodetect routine
% about your intention:
% \begin{quote}
%   \verb|latex \let\install=y% \iffalse meta-comment
%
% File: pagesel.dtx
% Version: 2020-08-03 v1.10
% Info: Select pages of a document for output
%
% Copyright (C)
%    1999, 2003, 2006-2008 Heiko Oberdiek
%    2016-2020 Oberdiek Package Support Group
%    https://github.com/ho-tex/pagesel/issues
%
% This work may be distributed and/or modified under the
% conditions of the LaTeX Project Public License, either
% version 1.3c of this license or (at your option) any later
% version. This version of this license is in
%    https://www.latex-project.org/lppl/lppl-1-3c.txt
% and the latest version of this license is in
%    https://www.latex-project.org/lppl.txt
% and version 1.3 or later is part of all distributions of
% LaTeX version 2005/12/01 or later.
%
% This work has the LPPL maintenance status "maintained".
%
% The Current Maintainers of this work are
% Heiko Oberdiek and the Oberdiek Package Support Group
% https://github.com/ho-tex/pagesel/issues
%
% This work consists of the main source file pagesel.dtx
% and the derived files
%    pagesel.sty, pagesel-2016-05-16.sty,
%    pagesel.pdf, pagesel.ins, pagesel.drv.
%
% Distribution:
%    CTAN:macros/latex/contrib/pagesel/pagesel.dtx
%    CTAN:macros/latex/contrib/pagesel/pagesel.pdf
%
% Unpacking:
%    (a) If pagesel.ins is present:
%           tex pagesel.ins
%    (b) Without pagesel.ins:
%           tex pagesel.dtx
%    (c) If you insist on using LaTeX
%           latex \let\install=y% \iffalse meta-comment
%
% File: pagesel.dtx
% Version: 2020-08-03 v1.10
% Info: Select pages of a document for output
%
% Copyright (C)
%    1999, 2003, 2006-2008 Heiko Oberdiek
%    2016-2020 Oberdiek Package Support Group
%    https://github.com/ho-tex/pagesel/issues
%
% This work may be distributed and/or modified under the
% conditions of the LaTeX Project Public License, either
% version 1.3c of this license or (at your option) any later
% version. This version of this license is in
%    https://www.latex-project.org/lppl/lppl-1-3c.txt
% and the latest version of this license is in
%    https://www.latex-project.org/lppl.txt
% and version 1.3 or later is part of all distributions of
% LaTeX version 2005/12/01 or later.
%
% This work has the LPPL maintenance status "maintained".
%
% The Current Maintainers of this work are
% Heiko Oberdiek and the Oberdiek Package Support Group
% https://github.com/ho-tex/pagesel/issues
%
% This work consists of the main source file pagesel.dtx
% and the derived files
%    pagesel.sty, pagesel-2016-05-16.sty,
%    pagesel.pdf, pagesel.ins, pagesel.drv.
%
% Distribution:
%    CTAN:macros/latex/contrib/pagesel/pagesel.dtx
%    CTAN:macros/latex/contrib/pagesel/pagesel.pdf
%
% Unpacking:
%    (a) If pagesel.ins is present:
%           tex pagesel.ins
%    (b) Without pagesel.ins:
%           tex pagesel.dtx
%    (c) If you insist on using LaTeX
%           latex \let\install=y\input{pagesel.dtx}
%        (quote the arguments according to the demands of your shell)
%
% Documentation:
%    (a) If pagesel.drv is present:
%           latex pagesel.drv
%    (b) Without pagesel.drv:
%           latex pagesel.dtx; ...
%    The class ltxdoc loads the configuration file ltxdoc.cfg
%    if available. Here you can specify further options, e.g.
%    use A4 as paper format:
%       \PassOptionsToClass{a4paper}{article}
%
%    Programm calls to get the documentation (example):
%       pdflatex pagesel.dtx
%       makeindex -s gind.ist pagesel.idx
%       pdflatex pagesel.dtx
%       makeindex -s gind.ist pagesel.idx
%       pdflatex pagesel.dtx
%
% Installation:
%    TDS:tex/latex/pagesel/pagesel.sty
%    TDS:doc/latex/pagesel/pagesel.pdf
%    TDS:source/latex/pagesel/pagesel.dtx
%
%<*ignore>
\begingroup
  \catcode123=1 %
  \catcode125=2 %
  \def\x{LaTeX2e}%
\expandafter\endgroup
\ifcase 0\ifx\install y1\fi\expandafter
         \ifx\csname processbatchFile\endcsname\relax\else1\fi
         \ifx\fmtname\x\else 1\fi\relax
\else\csname fi\endcsname
%</ignore>
%<*install>
\input docstrip.tex
\Msg{************************************************************************}
\Msg{* Installation}
\Msg{* Package: pagesel 2020-08-03 v1.10 Select pages of a document for output (HO)}
\Msg{************************************************************************}

\keepsilent
\askforoverwritefalse

\let\MetaPrefix\relax
\preamble

This is a generated file.

Project: pagesel
Version: 2020-08-03 v1.10

Copyright (C)
   1999, 2003, 2006-2008 Heiko Oberdiek
   2016-2020 Oberdiek Package Support Group

This work may be distributed and/or modified under the
conditions of the LaTeX Project Public License, either
version 1.3c of this license or (at your option) any later
version. This version of this license is in
   https://www.latex-project.org/lppl/lppl-1-3c.txt
and the latest version of this license is in
   https://www.latex-project.org/lppl.txt
and version 1.3 or later is part of all distributions of
LaTeX version 2005/12/01 or later.

This work has the LPPL maintenance status "maintained".

The Current Maintainers of this work are
Heiko Oberdiek and the Oberdiek Package Support Group
https://github.com/ho-tex/pagesel/issues


This work consists of the main source file pagesel.dtx
and the derived files
   pagesel.sty, pagesel-2016-05-16.sty, pagesel.pdf,
   pagesel.ins, pagesel.drv.

\endpreamble
\let\MetaPrefix\DoubleperCent

\generate{%
  \file{pagesel.ins}{\from{pagesel.dtx}{install}}%
  \file{pagesel.drv}{\from{pagesel.dtx}{driver}}%
  \usedir{tex/latex/pagesel}%
  \file{pagesel.sty}{\from{pagesel.dtx}{package}}%
  \file{pagesel-2016-05-16.sty}{\from{pagesel.dtx}{packagefrozen}}
}

\catcode32=13\relax% active space
\let =\space%
\Msg{************************************************************************}
\Msg{*}
\Msg{* To finish the installation you have to move the following}
\Msg{* file into a directory searched by TeX:}
\Msg{*}
\Msg{*     pagesel.sty}
\Msg{*}
\Msg{* To produce the documentation run the file `pagesel.drv'}
\Msg{* through LaTeX.}
\Msg{*}
\Msg{* Happy TeXing!}
\Msg{*}
\Msg{************************************************************************}

\endbatchfile
%</install>
%<*ignore>
\fi
%</ignore>
%<*driver>
\NeedsTeXFormat{LaTeX2e}
\ProvidesFile{pagesel.drv}%
  [2020-08-03 v1.10 Select pages of a document for output (HO)]%
\documentclass{ltxdoc}
\usepackage{holtxdoc}[2011/11/22]
\begin{document}
  \DocInput{pagesel.dtx}%
\end{document}
%</driver>
% \fi
%
%
%
% \GetFileInfo{pagesel.drv}
%
% \title{The \xpackage{pagesel} package}
% \date{2020-08-03 v1.10}
% \author{Heiko Oberdiek\thanks
% {Please report any issues at \url{https://github.com/ho-tex/pagesel/issues}}}
%
% \maketitle
%
% \begin{abstract}
% Single pages or page areas can be selected for output.
% \end{abstract}
%
% \tableofcontents
%
% \newenvironment{param}{^^A
%   \newcommand{\entry}[1]{\meta{\###1}:&}^^A
%   \begin{tabular}[t]{@{}l@{ }l@{}}^^A
% }{^^A
%   \end{tabular}^^A
% }
%
% \newcommand*{\Option}[1]{\textsf{#1}}
%
% \section{Usage}
%    The package \Package{pagesel} is a \LaTeXe\ package:
%    \begin{quote}
%      |\usepackage|\oarg{options}|{pagesel}|
%    \end{quote}
%    (For plain\TeX\ and \LaTeX\,2.09 the similar package
%    \URL{\Package{selectp}}^^A
%    {https://ctan.org/pkg/selectp}
%    from \NameEmail{Donald Arsenau}{asnd@triumf.ca} can be used.)
%
%    Depending on the options the package works in two modes:
%    \begin{enumerate}
%    \item If no page selecting option is present, so the package
%          ignores the other options and finishes itself. So no
%          page will be suppressed by the package and auxiliary files
%          will be written.
%    \item With at least one page selecting option the specified
%          pages are selected and the other are suppressed.
%          The default for this mode is that auxiliary will not be
%          overwritten. (This can be changed by an option.)
%    \end{enumerate}
%
% \subsection{Page selecting}
%    The package \Package{pagesel} sets up a new counter that is
%    incremented by each \cmd{\shipout}.
%    In this way the package counts the output pages regardless the value
%    of the page counter. So each page can individually by addressed,
%    even if there are several pages with the same page number.
%
% \subsubsection{Options\texorpdfstring{ for selecting pages}{}}
%    \begin{description}
%    \item[\Option{odd}:] The output pages must have an odd number.
%         All even output pages are suppressed. If there are no
%         page areas specified so all odd pages are print. With
%         page areas only the odd pages in this areas are selected.
%    \item[\Option{even}:] The opposite of option \Option{odd}.
%    \item[Page area:] A page area consists of three elements:
%         the starting output page number, an ``area'' hyphen, and
%         the output page number of the last page in this area.
%         Each component is optional, so there are four kinds
%         to spezify a page area:
%         \begin{description}
%         \item[\meta{m}\Option{-}\meta{n}:] All pages between
%              \meta{m} and \meta{n} inclusive.
%         \item[\Option{-}\meta{n}:] All pages until \meta{n} inclusive.
%         \item[\meta{m}\Option{-}:] The page area starts with \meta{m}
%              and all pages to the end of document are selected.
%         \item[\Option{-}:] All pages (not very useful).
%         \item[\meta{s}:] The single page \meta{s}.
%         \end{description}
%    \end{description}
%
% \subsubsection{Examples}
%    \newcommand*{\exam}[1]{\texttt{\strut[#1]}}^^A hash-ok
%    \begin{tabular}{ll}
%      Options & Output pages\\
%      \hline
%      \exam{1, 4, 9}&  1, 4, and 9\\
%      \exam{7-10, 3}&  3, 7, 8, 9, and 10\\
%      \exam{odd, 3-6}& 3, and 5\\
%      \exam{-4, 3, even, 7-8}& 2, 4, and 8\\
%    \end{tabular}
%
% \subsection{Auxiliary files}
%    If a page is suppressed, the \cmd{\write} commands are not
%    performed. Labels, index entries, or entries for the
%    table of contents aren't written. So it is likely that
%    the table of contents, registers, and lists are incomplete.
% \subsubsection{Options\texorpdfstring{ for handling auxiliary files}{}}
%    \begin{description}
%    \item[\Option{nofiles}:] This is the default. Auxiliary files are
%         read but not written or changed. Also the job is aborted
%         after the last selected page for saving time.
%    \item[\Option{nonofiles}/\Option{files}:] Auxiliary files are
%         written.
%    \end{description}
% \subsubsection{\texorpdfstring{Package }{}\Package{hyperref}}
%    In old versions of \Package{hyperref} [1999/04/12 v6.55] (and below)
%    there is a bug with \cmd{\nofiles}:
%    \begin{itemize}
%    \item Some ``garbage'' appears on terminal and in the log file.
%          This is harmless and can be ignored.
%    \item The outline auxiliary file \cmd{\jobname.out}, however,
%          is opened and truncated to zero bytes.
%          Version 1.0 of this package had
%          loaded a patch file \File{hypnofil.tex}, if it detects
%          \Package{hyperref} to get \cmd{\nofiles} work.
%
%          With the new version of \Package{hyperref} [1999/04/13 v6.56]
%          \cmd{\nofiles} works now. Therefore the workaround code
%          is no longer needed and removed.
%    \end{itemize}
%
% \StopEventually{
% }
%
% \section{Implementation}
%    \begin{macrocode}
%<*package>
%    \end{macrocode}
% \subsection{New implementation using the LaTeX kernel hooks}
%    \begin{macrocode}
\NeedsTeXFormat{LaTeX2e}
\ProvidesPackage{pagesel}
  [2020-08-03 v1.10 Select pages of a document for output (HO)]%
%    \end{macrocode}
%    \begin{macrocode}
\providecommand\IfFormatAtLeastTF{\@ifl@t@r\fmtversion}
\IfFormatAtLeastTF{2020/10/01}{}{\input{pagesel-2016-05-16.sty}}
\IfFormatAtLeastTF{2020/10/01}{}{\endinput}

%    \end{macrocode}
%    If the package is loaded twice, the package code does not
%    work. So stop loading the package, if it is already loaded.
%    \begin{macrocode}
\@ifundefined{ps@oddpages}{}{%
  \PackageWarningNoLine{pagesel}{Package already loaded.}%
  \endinput
}
%    \end{macrocode}
%    \begin{macrocode}
%</package>
%    \end{macrocode}
% \subsection{Package}
%    \begin{macrocode}
%<*packagefrozen>
\NeedsTeXFormat{LaTeX2e}
\ProvidesPackage{pagesel}
  [2020-08-03 v1.10 Select pages of a document for output (legacy code) (HO)]%
%    \end{macrocode}
%
%    If the package is loaded twice, the package code does not
%    work. So stop loading the package, if it is already loaded.
%    \begin{macrocode}
\@ifundefined{ps@makevoid}{}{%
  \PackageWarningNoLine{pagesel}{Package already loaded.}%
  \endinput
}
%    \end{macrocode}
%
%    \begin{macro}{\ps@makevoid}
%    Macro \cmd{\ps@makevoid} clears the output box. Because
%    nothing is shipped out and this is intended, we reduce
%    the counter \cmd{\deadcycles} in order to avoid problems, if
%    more than \cmd{\maxdeadcycles} pages are omitted.
%    \begin{macrocode}
\newcommand*{\ps@makevoid}{%
  \global\setbox\@cclv\copy\voidb@x
  \begingroup
    \count@=\deadcycles
    \advance\count@ by -1\relax
    \deadcycles=\count@
  \endgroup
}
%</packagefrozen>
%    \end{macrocode}
%    \end{macro}
%
%    \begin{macro}{\ps@oddpages}
%    \begin{macrocode}
%<*package|packagefrozen>
\newcommand*\ps@oddpages{0}
\DeclareOption{odd}{\renewcommand*\ps@oddpages{1}}
\DeclareOption{even}{\renewcommand*\ps@oddpages{2}}
%    \end{macrocode}
%    \end{macro}
%
%    \begin{macrocode}
\DeclareOption{nofiles}{\let\ps@nofiles\nofiles}
\DeclareOption{nonofiles}{\let\ps@nofiles\@empty}
\DeclareOption{files}{\let\ps@nofiles\@empty}
\ExecuteOptions{nofiles}
%    \end{macrocode}
%
%    \begin{macrocode}
\DeclareOption*{%
  \begingroup
    \expandafter\ps@checkoption\CurrentOption-\END
    \edef\x{\endgroup\noexpand\ps@store{\ps@first}{\ps@last}}%
  \x
}
%    \end{macrocode}
%
%    \begin{macro}{\ps@checkoption}
%    \begin{macrocode}
\newcommand\ps@checkoption{}
\def\ps@checkoption#1-#2\END{%
  \ifx\\#2\\%
    \ifx\\#1\\%
      % empty option
      \def\ps@first{\maxdimen}%
      \def\ps@last{\maxdimen}%
    \else
      \edef\ps@first{#1}%
      \edef\ps@last{#1}%
    \fi
  \else
    \ifx\\#1\\%
      \def\ps@first{-\maxdimen}%
    \else
      \edef\ps@first{#1}%
    \fi
    \ps@checklast#2%
  \fi
}
%    \end{macrocode}
%    \end{macro}
%
%    \begin{macro}{\ps@checklast}
%    \begin{macrocode}
\newcommand\ps@checklast{}
\def\ps@checklast#1-{%
  \ifx\\#1\\%
    \def\ps@last{\maxdimen}%
  \else
    \edef\ps@last{#1}%
  \fi
}
%    \end{macrocode}
%    \end{macro}
%
%    \begin{macro}{\ps@store}
%    \begin{macrocode}
\newcommand*{\ps@store}[2]{%
  \expandafter\def\expandafter\ps@testlist\expandafter{%
    \ps@testlist\ps@pagetest{#1}{#2}%
  }%
}
%    \end{macrocode}
%    \end{macro}
%
%    \begin{macro}{\ps@testlist}
%    \begin{macrocode}
\newcommand*\ps@testlist{}
%    \end{macrocode}
%    \end{macro}
%
%    \begin{macrocode}
\ProcessOptions
%    \end{macrocode}
%
%    \begin{macrocode}
\begingroup
  \edef\x{%
    \ifnum\ps@oddpages>0 \relax\fi
    \ifx\ps@testlist\@empty\else\relax\fi
  }%
  \ifx\x\@empty
    \endgroup
    \PackageInfo{pagesel}{Nothing to do}%
    \expandafter\endinput
  \fi
\endgroup
%    \end{macrocode}
%
%    \begin{macrocode}
%</package|packagefrozen>
%<*packagefrozen>
\RequirePackage{everyshi}
%</packagefrozen>
%    \end{macrocode}
%
%    \begin{macrocode}
%<*package|packagefrozen>
\ps@nofiles
%    \end{macrocode}
%
%    \begin{macro}{\c@ps@count}
%    \begin{macrocode}
\newcounter{ps@count}
\setcounter{ps@count}{0}
%    \end{macrocode}
%    \end{macro}
%
%    \begin{macro}{\ps@ReturnAfterElseFi}
%    \begin{macro}{\ps@ReturnAfterFi}
%    \begin{macrocode}
\long\def\ps@ReturnAfterElseFi#1\else#2\fi{\fi#1}
\long\def\ps@ReturnAfterFi#1\fi{\fi#1}
%    \end{macrocode}
%    \end{macro}
%    \end{macro}
%
%    \begin{macrocode}
\newcommand{\ps@lastpage}{\maxdimen}
\ifx\ps@nofiles\nofiles
  \ifx\ps@testlist\@empty
  \else
    \def\ps@lastpage{0}%
    \newcommand*{\ps@pagetest}[2]{%
      \ifnum#2>\ps@lastpage\relax
        \def\ps@lastpage{#2}%
      \fi
    }%
    \ps@testlist
    \let\ps@pagetest\relax
  \fi
\fi
%    \end{macrocode}
%
%    \begin{macro}{\ps@ifinset}
%    \begin{macrocode}
\newcommand*{\ps@ifinset}[4]{%
  \ifnum#1>\value{ps@count}%
    \ps@ReturnAfterElseFi{#4}%
  \else
    \ps@ReturnAfterFi{%
      \ifnum#2<\value{ps@count}%
        \ps@ReturnAfterElseFi{#4}%
      \else
        \ps@ReturnAfterFi{#3}%
      \fi
    }%
  \fi
}
%    \end{macrocode}
%    \end{macro}
%
%    \begin{macro}{\ps@pagetest}
%    \begin{macrocode}
\newcommand*{\ps@pagetest}[2]{%
  \ps@ifinset{#1}{#2}{\let\ps@next\@empty}{}%
}
%    \end{macrocode}
%    \end{macro}
%
%    \begin{macrocode}
%</package|packagefrozen>
%<packagefrozen>\EveryShipout{%
%<package>\AddToHook{shipout/before}{%
%<*package|packagefrozen>
  \stepcounter{ps@count}%
  \ifnum\value{ps@count}>\ps@lastpage\relax
    \global\output{%
      \ps@cleanup@if
      \ps@group@message
      \typeout{%
        Package pagesel Notice: Aborting LaTeX job %
        after last selected page (\ps@lastpage).%
      }%
      \ps@message@ignore
      \global\setbox\@cclv\box\voidb@x
      \deadcycles0\relax
%    \end{macrocode}
%    First leave the output group before ending the job.
%    \begin{macrocode}
      \aftergroup\@@end
    }%
  \fi
  \let\ps@next\@empty
  \ifx\ps@testlist\@empty
  \else
%<packagefrozen>    \let\ps@next\ps@makevoid
%<package>    \let\ps@next\DiscardShipoutBox
    \ps@testlist
  \fi
  \ifnum\ps@oddpages=1 %
    \ifodd\value{ps@count}%
    \else
%<packagefrozen>    \let\ps@next\ps@makevoid
%<package>    \let\ps@next\DiscardShipoutBox
    \fi
  \fi
  \ifnum\ps@oddpages=2 %
    \ifodd\value{ps@count}%
%<packagefrozen>    \let\ps@next\ps@makevoid
%<package>    \let\ps@next\DiscardShipoutBox
    \else
    \fi
  \fi
%<packagefrozen>  \ps@begindvi
  \ps@next
}
%</package|packagefrozen>
%    \end{macrocode}
%
%    \begin{macrocode}
%<*package|packagefrozen>
%<packagefrozen>\begingroup\expandafter\expandafter\expandafter\endgroup
%<packagefrozen>\expandafter\ifx\csname currentiflevel\endcsname\relax
%<packagefrozen>  \let\ps@cleanup@if\@empty
%<packagefrozen>\else
  \def\ps@cleanup@if{%
    \ifnum\currentiflevel>\@ne
      \csname fi\endcsname
      \expandafter\ps@cleanup@if
    \fi
  }%
%<packagefrozen>\fi
%    \end{macrocode}
%    Because of \cs{aftergroup} it is too dangerous to perform
%    a similar cleanup for groups.
%    \begin{macrocode}
%<packagefrozen> \begingroup\expandafter\expandafter\expandafter\endgroup
%<packagefrozen> \expandafter\ifx\csname currentgrouplevel\endcsname\relax
%<packagefrozen>  \let\ps@group@message\@empty
%<packagefrozen>  \def\ps@message@ignore{%
%<packagefrozen>    \typeout{%
%<packagefrozen>      (pagesel) \space\space\@spaces\@spaces\@spaces
%<packagefrozen>      Messages (\string\end\space occurred ...) can be ignored.%
%<packagefrozen>    }%
%<packagefrozen>  }%
%<packagefrozen>\else
  \def\ps@group@message{%
    \ifnum\currentgrouplevel>\@ne
      \def\ps@message@ignore{%
        \typeout{%
          (pagesel) \space\space\@spaces\@spaces\@spaces
          Message (\string\end\space occurred ...) %
          can be ignored.%
        }%
      }%
    \else
      \let\ps@message@ignore\@empty
    \fi
  }%
%<packagefrozen>\fi
%</package|packagefrozen>
%    \end{macrocode}
%
% \subsection{AtBeginDvi hook support}
%
%    The material of box \cs{@begindvibox} is recorded in parallel
%    in box \cs{ps@begindvibox}.
%    \begin{macrocode}
%<*packagefrozen>
\newbox\ps@begindvibox
\ifvoid\@begindvibox
\else
  \global\setbox\ps@begindvibox\vbox{%
    \unvbox\@begindvibox
  }%
\fi
\let\ps@org@AtBeginDvi\AtBeginDvi
\def\AtBeginDvi#1{%
  \global\setbox\ps@begindvibox\vbox{%
    \unvbox\ps@begindvibox
    #1%
  }%
  \ps@org@AtBeginDvi{#1}%
}
%    \end{macrocode}
%
%    \begin{macro}{\ps@begindvi}
%    Macro \cs{ps@begindvi} is called the similar way as \cs{@begindvi}.
%    If the first page is printed, then \cs{AtBeginDvi} should work
%    as usual. Otherwise the contents of box \cs{ps@begindvibox} is
%    set on the first selected page.
%    \begin{macrocode}
\def\ps@begindvi{%
  \ifx\ps@next\@empty
    \global\let\ps@begindvi\@empty
  \else
    \global\let\ps@begindvi\ps@begindvi@do
  \fi
}
\def\ps@begindvi@do{%
  \ifx\ps@next\@empty
    \setbox\@cclv\vbox{%
      \unvbox\ps@begindvibox
      \box\@cclv
    }%
    \global\let\ps@begindvi\@empty
  \fi
}
%    \end{macrocode}
%    \end{macro}
%
%    \begin{macrocode}
%</packagefrozen>
%    \end{macrocode}
%
% \section{Installation}
%
% \subsection{Download}
%
% \paragraph{Package.} This package is available on
% CTAN\footnote{\CTANpkg{pagesel}}:
% \begin{description}
% \item[\CTAN{macros/latex/contrib/pagesel/pagesel.dtx}] The source file.
% \item[\CTAN{macros/latex/contrib/pagesel/pagesel.pdf}] Documentation.
% \end{description}
%
%
%
% \subsection{Package installation}
%
% \paragraph{Unpacking.} The \xfile{.dtx} file is a self-extracting
% \docstrip\ archive. The files are extracted by running the
% \xfile{.dtx} through \plainTeX:
% \begin{quote}
%   \verb|tex pagesel.dtx|
% \end{quote}
%
% \paragraph{TDS.} Now the different files must be moved into
% the different directories in your installation TDS tree
% (also known as \xfile{texmf} tree):
% \begin{quote}
% \def\t{^^A
% \begin{tabular}{@{}>{\ttfamily}l@{ $\rightarrow$ }>{\ttfamily}l@{}}
%   pagesel.sty & tex/latex/pagesel/pagesel.sty\\
%   pagesel.pdf & doc/latex/pagesel/pagesel.pdf\\
%   pagesel.dtx & source/latex/pagesel/pagesel.dtx\\
% \end{tabular}^^A
% }^^A
% \sbox0{\t}^^A
% \ifdim\wd0>\linewidth
%   \begingroup
%     \advance\linewidth by\leftmargin
%     \advance\linewidth by\rightmargin
%   \edef\x{\endgroup
%     \def\noexpand\lw{\the\linewidth}^^A
%   }\x
%   \def\lwbox{^^A
%     \leavevmode
%     \hbox to \linewidth{^^A
%       \kern-\leftmargin\relax
%       \hss
%       \usebox0
%       \hss
%       \kern-\rightmargin\relax
%     }^^A
%   }^^A
%   \ifdim\wd0>\lw
%     \sbox0{\small\t}^^A
%     \ifdim\wd0>\linewidth
%       \ifdim\wd0>\lw
%         \sbox0{\footnotesize\t}^^A
%         \ifdim\wd0>\linewidth
%           \ifdim\wd0>\lw
%             \sbox0{\scriptsize\t}^^A
%             \ifdim\wd0>\linewidth
%               \ifdim\wd0>\lw
%                 \sbox0{\tiny\t}^^A
%                 \ifdim\wd0>\linewidth
%                   \lwbox
%                 \else
%                   \usebox0
%                 \fi
%               \else
%                 \lwbox
%               \fi
%             \else
%               \usebox0
%             \fi
%           \else
%             \lwbox
%           \fi
%         \else
%           \usebox0
%         \fi
%       \else
%         \lwbox
%       \fi
%     \else
%       \usebox0
%     \fi
%   \else
%     \lwbox
%   \fi
% \else
%   \usebox0
% \fi
% \end{quote}
% If you have a \xfile{docstrip.cfg} that configures and enables \docstrip's
% TDS installing feature, then some files can already be in the right
% place, see the documentation of \docstrip.
%
% \subsection{Refresh file name databases}
%
% If your \TeX~distribution
% (\TeX\,Live, \mikTeX, \dots) relies on file name databases, you must refresh
% these. For example, \TeX\,Live\ users run \verb|texhash| or
% \verb|mktexlsr|.
%
% \subsection{Some details for the interested}
%
% \paragraph{Unpacking with \LaTeX.}
% The \xfile{.dtx} chooses its action depending on the format:
% \begin{description}
% \item[\plainTeX:] Run \docstrip\ and extract the files.
% \item[\LaTeX:] Generate the documentation.
% \end{description}
% If you insist on using \LaTeX\ for \docstrip\ (really,
% \docstrip\ does not need \LaTeX), then inform the autodetect routine
% about your intention:
% \begin{quote}
%   \verb|latex \let\install=y\input{pagesel.dtx}|
% \end{quote}
% Do not forget to quote the argument according to the demands
% of your shell.
%
% \paragraph{Generating the documentation.}
% You can use both the \xfile{.dtx} or the \xfile{.drv} to generate
% the documentation. The process can be configured by the
% configuration file \xfile{ltxdoc.cfg}. For instance, put this
% line into this file, if you want to have A4 as paper format:
% \begin{quote}
%   \verb|\PassOptionsToClass{a4paper}{article}|
% \end{quote}
% An example follows how to generate the
% documentation with pdf\LaTeX:
% \begin{quote}
%\begin{verbatim}
%pdflatex pagesel.dtx
%makeindex -s gind.ist pagesel.idx
%pdflatex pagesel.dtx
%makeindex -s gind.ist pagesel.idx
%pdflatex pagesel.dtx
%\end{verbatim}
% \end{quote}
%
% \begin{History}
%   \begin{Version}{1999/03/01 v0.9}
%   \item
%     The first version was built as a response to a question
%     of \NameEmail{Dirk Kuypers}{dk@comnets.rwth-aachen.de},
%     published in the newsgroup
%     \href{news:de.comp.text.tex}{de.comp.text.tex}:\\
%     \URL{``\link{Re: pdflatex nur fuer bestimmte Seiten?!?}''}^^A
%     {https://groups.google.com/group/de.comp.text.tex/msg/6b68c7b3439fb658}
%   \end{Version}
%   \begin{Version}{1999/04/05 v1.0}
%   \item
%     Documentation added in dtx format.
%   \item
%     Copyright: LPPL (\CTAN{macros/latex/base/lppl.txt})
%   \item
%     Options |odd|, |even| added.
%   \item
%     \cmd{\nofiles} added, bug fix for \Package{hyperref}.
%   \item
%     Abort loading of package, if nothing to do.
%   \end{Version}
%   \begin{Version}{1999/04/13 v1.1}
%   \item
%     \cs{nofiles} bug fix removed
%     because of \xpackage{hyperref} 6.55.
%   \item
%     First CTAN release.
%   \end{Version}
%   \begin{Version}{2003/06/05 v1.2}
%   \item
%     \cs{deadcyles} is decremented for omitted pages.
%   \item
%     LPPL 1.2.
%   \end{Version}
%   \begin{Version}{2006/02/20 v1.3}
%   \item
%     Code is not changed.
%   \item
%     New DTX framework.
%   \item
%     LPPL 1.3
%   \end{Version}
%   \begin{Version}{2006/03/02 v1.4}
%   \item
%     Support for \cs{AtBeginDvi} added.
%   \end{Version}
%   \begin{Version}{2006/03/07 v1.5}
%   \item
%     Job is aborted after last selected page.
%   \end{Version}
%   \begin{Version}{2007/04/11 v1.6}
%   \item
%     Line ends sanitized.
%   \end{Version}
%   \begin{Version}{2007/04/12 v1.7}
%   \item
%     Hard coded box number 255 replaced by macro \cs{@cclv}.
%   \end{Version}
%   \begin{Version}{2008/08/11 v1.8}
%   \item
%     Code is not changed.
%   \item
%     URL updated from \texttt{www.dejanews.com}
%     to \texttt{groups.google.com}.
%   \end{Version}
%   \begin{Version}{2016/05/16 v1.9}
%   \item
%     Documentation updates.
%   \end{Version}
%   \begin{Version}{2020-08-03 v1.10}
%   \item Updated to follow the changes in the hook management
%   of LaTeX 2020/10/01
%   \end{Version}
% \end{History}
%
% \PrintIndex
%
% \Finale
\endinput

%        (quote the arguments according to the demands of your shell)
%
% Documentation:
%    (a) If pagesel.drv is present:
%           latex pagesel.drv
%    (b) Without pagesel.drv:
%           latex pagesel.dtx; ...
%    The class ltxdoc loads the configuration file ltxdoc.cfg
%    if available. Here you can specify further options, e.g.
%    use A4 as paper format:
%       \PassOptionsToClass{a4paper}{article}
%
%    Programm calls to get the documentation (example):
%       pdflatex pagesel.dtx
%       makeindex -s gind.ist pagesel.idx
%       pdflatex pagesel.dtx
%       makeindex -s gind.ist pagesel.idx
%       pdflatex pagesel.dtx
%
% Installation:
%    TDS:tex/latex/pagesel/pagesel.sty
%    TDS:doc/latex/pagesel/pagesel.pdf
%    TDS:source/latex/pagesel/pagesel.dtx
%
%<*ignore>
\begingroup
  \catcode123=1 %
  \catcode125=2 %
  \def\x{LaTeX2e}%
\expandafter\endgroup
\ifcase 0\ifx\install y1\fi\expandafter
         \ifx\csname processbatchFile\endcsname\relax\else1\fi
         \ifx\fmtname\x\else 1\fi\relax
\else\csname fi\endcsname
%</ignore>
%<*install>
\input docstrip.tex
\Msg{************************************************************************}
\Msg{* Installation}
\Msg{* Package: pagesel 2020-08-03 v1.10 Select pages of a document for output (HO)}
\Msg{************************************************************************}

\keepsilent
\askforoverwritefalse

\let\MetaPrefix\relax
\preamble

This is a generated file.

Project: pagesel
Version: 2020-08-03 v1.10

Copyright (C)
   1999, 2003, 2006-2008 Heiko Oberdiek
   2016-2020 Oberdiek Package Support Group

This work may be distributed and/or modified under the
conditions of the LaTeX Project Public License, either
version 1.3c of this license or (at your option) any later
version. This version of this license is in
   https://www.latex-project.org/lppl/lppl-1-3c.txt
and the latest version of this license is in
   https://www.latex-project.org/lppl.txt
and version 1.3 or later is part of all distributions of
LaTeX version 2005/12/01 or later.

This work has the LPPL maintenance status "maintained".

The Current Maintainers of this work are
Heiko Oberdiek and the Oberdiek Package Support Group
https://github.com/ho-tex/pagesel/issues


This work consists of the main source file pagesel.dtx
and the derived files
   pagesel.sty, pagesel-2016-05-16.sty, pagesel.pdf,
   pagesel.ins, pagesel.drv.

\endpreamble
\let\MetaPrefix\DoubleperCent

\generate{%
  \file{pagesel.ins}{\from{pagesel.dtx}{install}}%
  \file{pagesel.drv}{\from{pagesel.dtx}{driver}}%
  \usedir{tex/latex/pagesel}%
  \file{pagesel.sty}{\from{pagesel.dtx}{package}}%
  \file{pagesel-2016-05-16.sty}{\from{pagesel.dtx}{packagefrozen}}
}

\catcode32=13\relax% active space
\let =\space%
\Msg{************************************************************************}
\Msg{*}
\Msg{* To finish the installation you have to move the following}
\Msg{* file into a directory searched by TeX:}
\Msg{*}
\Msg{*     pagesel.sty}
\Msg{*}
\Msg{* To produce the documentation run the file `pagesel.drv'}
\Msg{* through LaTeX.}
\Msg{*}
\Msg{* Happy TeXing!}
\Msg{*}
\Msg{************************************************************************}

\endbatchfile
%</install>
%<*ignore>
\fi
%</ignore>
%<*driver>
\NeedsTeXFormat{LaTeX2e}
\ProvidesFile{pagesel.drv}%
  [2020-08-03 v1.10 Select pages of a document for output (HO)]%
\documentclass{ltxdoc}
\usepackage{holtxdoc}[2011/11/22]
\begin{document}
  \DocInput{pagesel.dtx}%
\end{document}
%</driver>
% \fi
%
%
%
% \GetFileInfo{pagesel.drv}
%
% \title{The \xpackage{pagesel} package}
% \date{2020-08-03 v1.10}
% \author{Heiko Oberdiek\thanks
% {Please report any issues at \url{https://github.com/ho-tex/pagesel/issues}}}
%
% \maketitle
%
% \begin{abstract}
% Single pages or page areas can be selected for output.
% \end{abstract}
%
% \tableofcontents
%
% \newenvironment{param}{^^A
%   \newcommand{\entry}[1]{\meta{\###1}:&}^^A
%   \begin{tabular}[t]{@{}l@{ }l@{}}^^A
% }{^^A
%   \end{tabular}^^A
% }
%
% \newcommand*{\Option}[1]{\textsf{#1}}
%
% \section{Usage}
%    The package \Package{pagesel} is a \LaTeXe\ package:
%    \begin{quote}
%      |\usepackage|\oarg{options}|{pagesel}|
%    \end{quote}
%    (For plain\TeX\ and \LaTeX\,2.09 the similar package
%    \URL{\Package{selectp}}^^A
%    {https://ctan.org/pkg/selectp}
%    from \NameEmail{Donald Arsenau}{asnd@triumf.ca} can be used.)
%
%    Depending on the options the package works in two modes:
%    \begin{enumerate}
%    \item If no page selecting option is present, so the package
%          ignores the other options and finishes itself. So no
%          page will be suppressed by the package and auxiliary files
%          will be written.
%    \item With at least one page selecting option the specified
%          pages are selected and the other are suppressed.
%          The default for this mode is that auxiliary will not be
%          overwritten. (This can be changed by an option.)
%    \end{enumerate}
%
% \subsection{Page selecting}
%    The package \Package{pagesel} sets up a new counter that is
%    incremented by each \cmd{\shipout}.
%    In this way the package counts the output pages regardless the value
%    of the page counter. So each page can individually by addressed,
%    even if there are several pages with the same page number.
%
% \subsubsection{Options\texorpdfstring{ for selecting pages}{}}
%    \begin{description}
%    \item[\Option{odd}:] The output pages must have an odd number.
%         All even output pages are suppressed. If there are no
%         page areas specified so all odd pages are print. With
%         page areas only the odd pages in this areas are selected.
%    \item[\Option{even}:] The opposite of option \Option{odd}.
%    \item[Page area:] A page area consists of three elements:
%         the starting output page number, an ``area'' hyphen, and
%         the output page number of the last page in this area.
%         Each component is optional, so there are four kinds
%         to spezify a page area:
%         \begin{description}
%         \item[\meta{m}\Option{-}\meta{n}:] All pages between
%              \meta{m} and \meta{n} inclusive.
%         \item[\Option{-}\meta{n}:] All pages until \meta{n} inclusive.
%         \item[\meta{m}\Option{-}:] The page area starts with \meta{m}
%              and all pages to the end of document are selected.
%         \item[\Option{-}:] All pages (not very useful).
%         \item[\meta{s}:] The single page \meta{s}.
%         \end{description}
%    \end{description}
%
% \subsubsection{Examples}
%    \newcommand*{\exam}[1]{\texttt{\strut[#1]}}^^A hash-ok
%    \begin{tabular}{ll}
%      Options & Output pages\\
%      \hline
%      \exam{1, 4, 9}&  1, 4, and 9\\
%      \exam{7-10, 3}&  3, 7, 8, 9, and 10\\
%      \exam{odd, 3-6}& 3, and 5\\
%      \exam{-4, 3, even, 7-8}& 2, 4, and 8\\
%    \end{tabular}
%
% \subsection{Auxiliary files}
%    If a page is suppressed, the \cmd{\write} commands are not
%    performed. Labels, index entries, or entries for the
%    table of contents aren't written. So it is likely that
%    the table of contents, registers, and lists are incomplete.
% \subsubsection{Options\texorpdfstring{ for handling auxiliary files}{}}
%    \begin{description}
%    \item[\Option{nofiles}:] This is the default. Auxiliary files are
%         read but not written or changed. Also the job is aborted
%         after the last selected page for saving time.
%    \item[\Option{nonofiles}/\Option{files}:] Auxiliary files are
%         written.
%    \end{description}
% \subsubsection{\texorpdfstring{Package }{}\Package{hyperref}}
%    In old versions of \Package{hyperref} [1999/04/12 v6.55] (and below)
%    there is a bug with \cmd{\nofiles}:
%    \begin{itemize}
%    \item Some ``garbage'' appears on terminal and in the log file.
%          This is harmless and can be ignored.
%    \item The outline auxiliary file \cmd{\jobname.out}, however,
%          is opened and truncated to zero bytes.
%          Version 1.0 of this package had
%          loaded a patch file \File{hypnofil.tex}, if it detects
%          \Package{hyperref} to get \cmd{\nofiles} work.
%
%          With the new version of \Package{hyperref} [1999/04/13 v6.56]
%          \cmd{\nofiles} works now. Therefore the workaround code
%          is no longer needed and removed.
%    \end{itemize}
%
% \StopEventually{
% }
%
% \section{Implementation}
%    \begin{macrocode}
%<*package>
%    \end{macrocode}
% \subsection{New implementation using the LaTeX kernel hooks}
%    \begin{macrocode}
\NeedsTeXFormat{LaTeX2e}
\ProvidesPackage{pagesel}
  [2020-08-03 v1.10 Select pages of a document for output (HO)]%
%    \end{macrocode}
%    \begin{macrocode}
\providecommand\IfFormatAtLeastTF{\@ifl@t@r\fmtversion}
\IfFormatAtLeastTF{2020/10/01}{}{%%
%% This is file `pagesel-2016-05-16.sty',
%% generated with the docstrip utility.
%%
%% The original source files were:
%%
%% pagesel.dtx  (with options: `packagefrozen')
%% 
%% This is a generated file.
%% 
%% Project: pagesel
%% Version: 2020-08-03 v1.10
%% 
%% Copyright (C)
%%    1999, 2003, 2006-2008 Heiko Oberdiek
%%    2016-2020 Oberdiek Package Support Group
%% 
%% This work may be distributed and/or modified under the
%% conditions of the LaTeX Project Public License, either
%% version 1.3c of this license or (at your option) any later
%% version. This version of this license is in
%%    https://www.latex-project.org/lppl/lppl-1-3c.txt
%% and the latest version of this license is in
%%    https://www.latex-project.org/lppl.txt
%% and version 1.3 or later is part of all distributions of
%% LaTeX version 2005/12/01 or later.
%% 
%% This work has the LPPL maintenance status "maintained".
%% 
%% The Current Maintainers of this work are
%% Heiko Oberdiek and the Oberdiek Package Support Group
%% https://github.com/ho-tex/pagesel/issues
%% 
%% This work consists of the main source file pagesel.dtx
%% and the derived files
%%    pagesel.sty, pagesel-2016-05-16.sty, pagesel.pdf,
%%    pagesel.ins, pagesel.drv.
%% 
\NeedsTeXFormat{LaTeX2e}
\ProvidesPackage{pagesel}
  [2020-08-03 v1.10 Select pages of a document for output (legacy code) (HO)]%
\@ifundefined{ps@makevoid}{}{%
  \PackageWarningNoLine{pagesel}{Package already loaded.}%
  \endinput
}
\newcommand*{\ps@makevoid}{%
  \global\setbox\@cclv\copy\voidb@x
  \begingroup
    \count@=\deadcycles
    \advance\count@ by -1\relax
    \deadcycles=\count@
  \endgroup
}
\newcommand*\ps@oddpages{0}
\DeclareOption{odd}{\renewcommand*\ps@oddpages{1}}
\DeclareOption{even}{\renewcommand*\ps@oddpages{2}}
\DeclareOption{nofiles}{\let\ps@nofiles\nofiles}
\DeclareOption{nonofiles}{\let\ps@nofiles\@empty}
\DeclareOption{files}{\let\ps@nofiles\@empty}
\ExecuteOptions{nofiles}
\DeclareOption*{%
  \begingroup
    \expandafter\ps@checkoption\CurrentOption-\END
    \edef\x{\endgroup\noexpand\ps@store{\ps@first}{\ps@last}}%
  \x
}
\newcommand\ps@checkoption{}
\def\ps@checkoption#1-#2\END{%
  \ifx\\#2\\%
    \ifx\\#1\\%
      % empty option
      \def\ps@first{\maxdimen}%
      \def\ps@last{\maxdimen}%
    \else
      \edef\ps@first{#1}%
      \edef\ps@last{#1}%
    \fi
  \else
    \ifx\\#1\\%
      \def\ps@first{-\maxdimen}%
    \else
      \edef\ps@first{#1}%
    \fi
    \ps@checklast#2%
  \fi
}
\newcommand\ps@checklast{}
\def\ps@checklast#1-{%
  \ifx\\#1\\%
    \def\ps@last{\maxdimen}%
  \else
    \edef\ps@last{#1}%
  \fi
}
\newcommand*{\ps@store}[2]{%
  \expandafter\def\expandafter\ps@testlist\expandafter{%
    \ps@testlist\ps@pagetest{#1}{#2}%
  }%
}
\newcommand*\ps@testlist{}
\ProcessOptions
\begingroup
  \edef\x{%
    \ifnum\ps@oddpages>0 \relax\fi
    \ifx\ps@testlist\@empty\else\relax\fi
  }%
  \ifx\x\@empty
    \endgroup
    \PackageInfo{pagesel}{Nothing to do}%
    \expandafter\endinput
  \fi
\endgroup
\RequirePackage{everyshi}
\ps@nofiles
\newcounter{ps@count}
\setcounter{ps@count}{0}
\long\def\ps@ReturnAfterElseFi#1\else#2\fi{\fi#1}
\long\def\ps@ReturnAfterFi#1\fi{\fi#1}
\newcommand{\ps@lastpage}{\maxdimen}
\ifx\ps@nofiles\nofiles
  \ifx\ps@testlist\@empty
  \else
    \def\ps@lastpage{0}%
    \newcommand*{\ps@pagetest}[2]{%
      \ifnum#2>\ps@lastpage\relax
        \def\ps@lastpage{#2}%
      \fi
    }%
    \ps@testlist
    \let\ps@pagetest\relax
  \fi
\fi
\newcommand*{\ps@ifinset}[4]{%
  \ifnum#1>\value{ps@count}%
    \ps@ReturnAfterElseFi{#4}%
  \else
    \ps@ReturnAfterFi{%
      \ifnum#2<\value{ps@count}%
        \ps@ReturnAfterElseFi{#4}%
      \else
        \ps@ReturnAfterFi{#3}%
      \fi
    }%
  \fi
}
\newcommand*{\ps@pagetest}[2]{%
  \ps@ifinset{#1}{#2}{\let\ps@next\@empty}{}%
}
\EveryShipout{%
  \stepcounter{ps@count}%
  \ifnum\value{ps@count}>\ps@lastpage\relax
    \global\output{%
      \ps@cleanup@if
      \ps@group@message
      \typeout{%
        Package pagesel Notice: Aborting LaTeX job %
        after last selected page (\ps@lastpage).%
      }%
      \ps@message@ignore
      \global\setbox\@cclv\box\voidb@x
      \deadcycles0\relax
      \aftergroup\@@end
    }%
  \fi
  \let\ps@next\@empty
  \ifx\ps@testlist\@empty
  \else
    \let\ps@next\ps@makevoid
    \ps@testlist
  \fi
  \ifnum\ps@oddpages=1 %
    \ifodd\value{ps@count}%
    \else
    \let\ps@next\ps@makevoid
    \fi
  \fi
  \ifnum\ps@oddpages=2 %
    \ifodd\value{ps@count}%
    \let\ps@next\ps@makevoid
    \else
    \fi
  \fi
  \ps@begindvi
  \ps@next
}
\begingroup\expandafter\expandafter\expandafter\endgroup
\expandafter\ifx\csname currentiflevel\endcsname\relax
  \let\ps@cleanup@if\@empty
\else
  \def\ps@cleanup@if{%
    \ifnum\currentiflevel>\@ne
      \csname fi\endcsname
      \expandafter\ps@cleanup@if
    \fi
  }%
\fi
 \begingroup\expandafter\expandafter\expandafter\endgroup
 \expandafter\ifx\csname currentgrouplevel\endcsname\relax
  \let\ps@group@message\@empty
  \def\ps@message@ignore{%
    \typeout{%
      (pagesel) \space\space\@spaces\@spaces\@spaces
      Messages (\string\end\space occurred ...) can be ignored.%
    }%
  }%
\else
  \def\ps@group@message{%
    \ifnum\currentgrouplevel>\@ne
      \def\ps@message@ignore{%
        \typeout{%
          (pagesel) \space\space\@spaces\@spaces\@spaces
          Message (\string\end\space occurred ...) %
          can be ignored.%
        }%
      }%
    \else
      \let\ps@message@ignore\@empty
    \fi
  }%
\fi
\newbox\ps@begindvibox
\ifvoid\@begindvibox
\else
  \global\setbox\ps@begindvibox\vbox{%
    \unvbox\@begindvibox
  }%
\fi
\let\ps@org@AtBeginDvi\AtBeginDvi
\def\AtBeginDvi#1{%
  \global\setbox\ps@begindvibox\vbox{%
    \unvbox\ps@begindvibox
    #1%
  }%
  \ps@org@AtBeginDvi{#1}%
}
\def\ps@begindvi{%
  \ifx\ps@next\@empty
    \global\let\ps@begindvi\@empty
  \else
    \global\let\ps@begindvi\ps@begindvi@do
  \fi
}
\def\ps@begindvi@do{%
  \ifx\ps@next\@empty
    \setbox\@cclv\vbox{%
      \unvbox\ps@begindvibox
      \box\@cclv
    }%
    \global\let\ps@begindvi\@empty
  \fi
}
\endinput
%%
%% End of file `pagesel-2016-05-16.sty'.
}
\IfFormatAtLeastTF{2020/10/01}{}{\endinput}

%    \end{macrocode}
%    If the package is loaded twice, the package code does not
%    work. So stop loading the package, if it is already loaded.
%    \begin{macrocode}
\@ifundefined{ps@oddpages}{}{%
  \PackageWarningNoLine{pagesel}{Package already loaded.}%
  \endinput
}
%    \end{macrocode}
%    \begin{macrocode}
%</package>
%    \end{macrocode}
% \subsection{Package}
%    \begin{macrocode}
%<*packagefrozen>
\NeedsTeXFormat{LaTeX2e}
\ProvidesPackage{pagesel}
  [2020-08-03 v1.10 Select pages of a document for output (legacy code) (HO)]%
%    \end{macrocode}
%
%    If the package is loaded twice, the package code does not
%    work. So stop loading the package, if it is already loaded.
%    \begin{macrocode}
\@ifundefined{ps@makevoid}{}{%
  \PackageWarningNoLine{pagesel}{Package already loaded.}%
  \endinput
}
%    \end{macrocode}
%
%    \begin{macro}{\ps@makevoid}
%    Macro \cmd{\ps@makevoid} clears the output box. Because
%    nothing is shipped out and this is intended, we reduce
%    the counter \cmd{\deadcycles} in order to avoid problems, if
%    more than \cmd{\maxdeadcycles} pages are omitted.
%    \begin{macrocode}
\newcommand*{\ps@makevoid}{%
  \global\setbox\@cclv\copy\voidb@x
  \begingroup
    \count@=\deadcycles
    \advance\count@ by -1\relax
    \deadcycles=\count@
  \endgroup
}
%</packagefrozen>
%    \end{macrocode}
%    \end{macro}
%
%    \begin{macro}{\ps@oddpages}
%    \begin{macrocode}
%<*package|packagefrozen>
\newcommand*\ps@oddpages{0}
\DeclareOption{odd}{\renewcommand*\ps@oddpages{1}}
\DeclareOption{even}{\renewcommand*\ps@oddpages{2}}
%    \end{macrocode}
%    \end{macro}
%
%    \begin{macrocode}
\DeclareOption{nofiles}{\let\ps@nofiles\nofiles}
\DeclareOption{nonofiles}{\let\ps@nofiles\@empty}
\DeclareOption{files}{\let\ps@nofiles\@empty}
\ExecuteOptions{nofiles}
%    \end{macrocode}
%
%    \begin{macrocode}
\DeclareOption*{%
  \begingroup
    \expandafter\ps@checkoption\CurrentOption-\END
    \edef\x{\endgroup\noexpand\ps@store{\ps@first}{\ps@last}}%
  \x
}
%    \end{macrocode}
%
%    \begin{macro}{\ps@checkoption}
%    \begin{macrocode}
\newcommand\ps@checkoption{}
\def\ps@checkoption#1-#2\END{%
  \ifx\\#2\\%
    \ifx\\#1\\%
      % empty option
      \def\ps@first{\maxdimen}%
      \def\ps@last{\maxdimen}%
    \else
      \edef\ps@first{#1}%
      \edef\ps@last{#1}%
    \fi
  \else
    \ifx\\#1\\%
      \def\ps@first{-\maxdimen}%
    \else
      \edef\ps@first{#1}%
    \fi
    \ps@checklast#2%
  \fi
}
%    \end{macrocode}
%    \end{macro}
%
%    \begin{macro}{\ps@checklast}
%    \begin{macrocode}
\newcommand\ps@checklast{}
\def\ps@checklast#1-{%
  \ifx\\#1\\%
    \def\ps@last{\maxdimen}%
  \else
    \edef\ps@last{#1}%
  \fi
}
%    \end{macrocode}
%    \end{macro}
%
%    \begin{macro}{\ps@store}
%    \begin{macrocode}
\newcommand*{\ps@store}[2]{%
  \expandafter\def\expandafter\ps@testlist\expandafter{%
    \ps@testlist\ps@pagetest{#1}{#2}%
  }%
}
%    \end{macrocode}
%    \end{macro}
%
%    \begin{macro}{\ps@testlist}
%    \begin{macrocode}
\newcommand*\ps@testlist{}
%    \end{macrocode}
%    \end{macro}
%
%    \begin{macrocode}
\ProcessOptions
%    \end{macrocode}
%
%    \begin{macrocode}
\begingroup
  \edef\x{%
    \ifnum\ps@oddpages>0 \relax\fi
    \ifx\ps@testlist\@empty\else\relax\fi
  }%
  \ifx\x\@empty
    \endgroup
    \PackageInfo{pagesel}{Nothing to do}%
    \expandafter\endinput
  \fi
\endgroup
%    \end{macrocode}
%
%    \begin{macrocode}
%</package|packagefrozen>
%<*packagefrozen>
\RequirePackage{everyshi}
%</packagefrozen>
%    \end{macrocode}
%
%    \begin{macrocode}
%<*package|packagefrozen>
\ps@nofiles
%    \end{macrocode}
%
%    \begin{macro}{\c@ps@count}
%    \begin{macrocode}
\newcounter{ps@count}
\setcounter{ps@count}{0}
%    \end{macrocode}
%    \end{macro}
%
%    \begin{macro}{\ps@ReturnAfterElseFi}
%    \begin{macro}{\ps@ReturnAfterFi}
%    \begin{macrocode}
\long\def\ps@ReturnAfterElseFi#1\else#2\fi{\fi#1}
\long\def\ps@ReturnAfterFi#1\fi{\fi#1}
%    \end{macrocode}
%    \end{macro}
%    \end{macro}
%
%    \begin{macrocode}
\newcommand{\ps@lastpage}{\maxdimen}
\ifx\ps@nofiles\nofiles
  \ifx\ps@testlist\@empty
  \else
    \def\ps@lastpage{0}%
    \newcommand*{\ps@pagetest}[2]{%
      \ifnum#2>\ps@lastpage\relax
        \def\ps@lastpage{#2}%
      \fi
    }%
    \ps@testlist
    \let\ps@pagetest\relax
  \fi
\fi
%    \end{macrocode}
%
%    \begin{macro}{\ps@ifinset}
%    \begin{macrocode}
\newcommand*{\ps@ifinset}[4]{%
  \ifnum#1>\value{ps@count}%
    \ps@ReturnAfterElseFi{#4}%
  \else
    \ps@ReturnAfterFi{%
      \ifnum#2<\value{ps@count}%
        \ps@ReturnAfterElseFi{#4}%
      \else
        \ps@ReturnAfterFi{#3}%
      \fi
    }%
  \fi
}
%    \end{macrocode}
%    \end{macro}
%
%    \begin{macro}{\ps@pagetest}
%    \begin{macrocode}
\newcommand*{\ps@pagetest}[2]{%
  \ps@ifinset{#1}{#2}{\let\ps@next\@empty}{}%
}
%    \end{macrocode}
%    \end{macro}
%
%    \begin{macrocode}
%</package|packagefrozen>
%<packagefrozen>\EveryShipout{%
%<package>\AddToHook{shipout/before}{%
%<*package|packagefrozen>
  \stepcounter{ps@count}%
  \ifnum\value{ps@count}>\ps@lastpage\relax
    \global\output{%
      \ps@cleanup@if
      \ps@group@message
      \typeout{%
        Package pagesel Notice: Aborting LaTeX job %
        after last selected page (\ps@lastpage).%
      }%
      \ps@message@ignore
      \global\setbox\@cclv\box\voidb@x
      \deadcycles0\relax
%    \end{macrocode}
%    First leave the output group before ending the job.
%    \begin{macrocode}
      \aftergroup\@@end
    }%
  \fi
  \let\ps@next\@empty
  \ifx\ps@testlist\@empty
  \else
%<packagefrozen>    \let\ps@next\ps@makevoid
%<package>    \let\ps@next\DiscardShipoutBox
    \ps@testlist
  \fi
  \ifnum\ps@oddpages=1 %
    \ifodd\value{ps@count}%
    \else
%<packagefrozen>    \let\ps@next\ps@makevoid
%<package>    \let\ps@next\DiscardShipoutBox
    \fi
  \fi
  \ifnum\ps@oddpages=2 %
    \ifodd\value{ps@count}%
%<packagefrozen>    \let\ps@next\ps@makevoid
%<package>    \let\ps@next\DiscardShipoutBox
    \else
    \fi
  \fi
%<packagefrozen>  \ps@begindvi
  \ps@next
}
%</package|packagefrozen>
%    \end{macrocode}
%
%    \begin{macrocode}
%<*package|packagefrozen>
%<packagefrozen>\begingroup\expandafter\expandafter\expandafter\endgroup
%<packagefrozen>\expandafter\ifx\csname currentiflevel\endcsname\relax
%<packagefrozen>  \let\ps@cleanup@if\@empty
%<packagefrozen>\else
  \def\ps@cleanup@if{%
    \ifnum\currentiflevel>\@ne
      \csname fi\endcsname
      \expandafter\ps@cleanup@if
    \fi
  }%
%<packagefrozen>\fi
%    \end{macrocode}
%    Because of \cs{aftergroup} it is too dangerous to perform
%    a similar cleanup for groups.
%    \begin{macrocode}
%<packagefrozen> \begingroup\expandafter\expandafter\expandafter\endgroup
%<packagefrozen> \expandafter\ifx\csname currentgrouplevel\endcsname\relax
%<packagefrozen>  \let\ps@group@message\@empty
%<packagefrozen>  \def\ps@message@ignore{%
%<packagefrozen>    \typeout{%
%<packagefrozen>      (pagesel) \space\space\@spaces\@spaces\@spaces
%<packagefrozen>      Messages (\string\end\space occurred ...) can be ignored.%
%<packagefrozen>    }%
%<packagefrozen>  }%
%<packagefrozen>\else
  \def\ps@group@message{%
    \ifnum\currentgrouplevel>\@ne
      \def\ps@message@ignore{%
        \typeout{%
          (pagesel) \space\space\@spaces\@spaces\@spaces
          Message (\string\end\space occurred ...) %
          can be ignored.%
        }%
      }%
    \else
      \let\ps@message@ignore\@empty
    \fi
  }%
%<packagefrozen>\fi
%</package|packagefrozen>
%    \end{macrocode}
%
% \subsection{AtBeginDvi hook support}
%
%    The material of box \cs{@begindvibox} is recorded in parallel
%    in box \cs{ps@begindvibox}.
%    \begin{macrocode}
%<*packagefrozen>
\newbox\ps@begindvibox
\ifvoid\@begindvibox
\else
  \global\setbox\ps@begindvibox\vbox{%
    \unvbox\@begindvibox
  }%
\fi
\let\ps@org@AtBeginDvi\AtBeginDvi
\def\AtBeginDvi#1{%
  \global\setbox\ps@begindvibox\vbox{%
    \unvbox\ps@begindvibox
    #1%
  }%
  \ps@org@AtBeginDvi{#1}%
}
%    \end{macrocode}
%
%    \begin{macro}{\ps@begindvi}
%    Macro \cs{ps@begindvi} is called the similar way as \cs{@begindvi}.
%    If the first page is printed, then \cs{AtBeginDvi} should work
%    as usual. Otherwise the contents of box \cs{ps@begindvibox} is
%    set on the first selected page.
%    \begin{macrocode}
\def\ps@begindvi{%
  \ifx\ps@next\@empty
    \global\let\ps@begindvi\@empty
  \else
    \global\let\ps@begindvi\ps@begindvi@do
  \fi
}
\def\ps@begindvi@do{%
  \ifx\ps@next\@empty
    \setbox\@cclv\vbox{%
      \unvbox\ps@begindvibox
      \box\@cclv
    }%
    \global\let\ps@begindvi\@empty
  \fi
}
%    \end{macrocode}
%    \end{macro}
%
%    \begin{macrocode}
%</packagefrozen>
%    \end{macrocode}
%
% \section{Installation}
%
% \subsection{Download}
%
% \paragraph{Package.} This package is available on
% CTAN\footnote{\CTANpkg{pagesel}}:
% \begin{description}
% \item[\CTAN{macros/latex/contrib/pagesel/pagesel.dtx}] The source file.
% \item[\CTAN{macros/latex/contrib/pagesel/pagesel.pdf}] Documentation.
% \end{description}
%
%
%
% \subsection{Package installation}
%
% \paragraph{Unpacking.} The \xfile{.dtx} file is a self-extracting
% \docstrip\ archive. The files are extracted by running the
% \xfile{.dtx} through \plainTeX:
% \begin{quote}
%   \verb|tex pagesel.dtx|
% \end{quote}
%
% \paragraph{TDS.} Now the different files must be moved into
% the different directories in your installation TDS tree
% (also known as \xfile{texmf} tree):
% \begin{quote}
% \def\t{^^A
% \begin{tabular}{@{}>{\ttfamily}l@{ $\rightarrow$ }>{\ttfamily}l@{}}
%   pagesel.sty & tex/latex/pagesel/pagesel.sty\\
%   pagesel.pdf & doc/latex/pagesel/pagesel.pdf\\
%   pagesel.dtx & source/latex/pagesel/pagesel.dtx\\
% \end{tabular}^^A
% }^^A
% \sbox0{\t}^^A
% \ifdim\wd0>\linewidth
%   \begingroup
%     \advance\linewidth by\leftmargin
%     \advance\linewidth by\rightmargin
%   \edef\x{\endgroup
%     \def\noexpand\lw{\the\linewidth}^^A
%   }\x
%   \def\lwbox{^^A
%     \leavevmode
%     \hbox to \linewidth{^^A
%       \kern-\leftmargin\relax
%       \hss
%       \usebox0
%       \hss
%       \kern-\rightmargin\relax
%     }^^A
%   }^^A
%   \ifdim\wd0>\lw
%     \sbox0{\small\t}^^A
%     \ifdim\wd0>\linewidth
%       \ifdim\wd0>\lw
%         \sbox0{\footnotesize\t}^^A
%         \ifdim\wd0>\linewidth
%           \ifdim\wd0>\lw
%             \sbox0{\scriptsize\t}^^A
%             \ifdim\wd0>\linewidth
%               \ifdim\wd0>\lw
%                 \sbox0{\tiny\t}^^A
%                 \ifdim\wd0>\linewidth
%                   \lwbox
%                 \else
%                   \usebox0
%                 \fi
%               \else
%                 \lwbox
%               \fi
%             \else
%               \usebox0
%             \fi
%           \else
%             \lwbox
%           \fi
%         \else
%           \usebox0
%         \fi
%       \else
%         \lwbox
%       \fi
%     \else
%       \usebox0
%     \fi
%   \else
%     \lwbox
%   \fi
% \else
%   \usebox0
% \fi
% \end{quote}
% If you have a \xfile{docstrip.cfg} that configures and enables \docstrip's
% TDS installing feature, then some files can already be in the right
% place, see the documentation of \docstrip.
%
% \subsection{Refresh file name databases}
%
% If your \TeX~distribution
% (\TeX\,Live, \mikTeX, \dots) relies on file name databases, you must refresh
% these. For example, \TeX\,Live\ users run \verb|texhash| or
% \verb|mktexlsr|.
%
% \subsection{Some details for the interested}
%
% \paragraph{Unpacking with \LaTeX.}
% The \xfile{.dtx} chooses its action depending on the format:
% \begin{description}
% \item[\plainTeX:] Run \docstrip\ and extract the files.
% \item[\LaTeX:] Generate the documentation.
% \end{description}
% If you insist on using \LaTeX\ for \docstrip\ (really,
% \docstrip\ does not need \LaTeX), then inform the autodetect routine
% about your intention:
% \begin{quote}
%   \verb|latex \let\install=y% \iffalse meta-comment
%
% File: pagesel.dtx
% Version: 2020-08-03 v1.10
% Info: Select pages of a document for output
%
% Copyright (C)
%    1999, 2003, 2006-2008 Heiko Oberdiek
%    2016-2020 Oberdiek Package Support Group
%    https://github.com/ho-tex/pagesel/issues
%
% This work may be distributed and/or modified under the
% conditions of the LaTeX Project Public License, either
% version 1.3c of this license or (at your option) any later
% version. This version of this license is in
%    https://www.latex-project.org/lppl/lppl-1-3c.txt
% and the latest version of this license is in
%    https://www.latex-project.org/lppl.txt
% and version 1.3 or later is part of all distributions of
% LaTeX version 2005/12/01 or later.
%
% This work has the LPPL maintenance status "maintained".
%
% The Current Maintainers of this work are
% Heiko Oberdiek and the Oberdiek Package Support Group
% https://github.com/ho-tex/pagesel/issues
%
% This work consists of the main source file pagesel.dtx
% and the derived files
%    pagesel.sty, pagesel-2016-05-16.sty,
%    pagesel.pdf, pagesel.ins, pagesel.drv.
%
% Distribution:
%    CTAN:macros/latex/contrib/pagesel/pagesel.dtx
%    CTAN:macros/latex/contrib/pagesel/pagesel.pdf
%
% Unpacking:
%    (a) If pagesel.ins is present:
%           tex pagesel.ins
%    (b) Without pagesel.ins:
%           tex pagesel.dtx
%    (c) If you insist on using LaTeX
%           latex \let\install=y\input{pagesel.dtx}
%        (quote the arguments according to the demands of your shell)
%
% Documentation:
%    (a) If pagesel.drv is present:
%           latex pagesel.drv
%    (b) Without pagesel.drv:
%           latex pagesel.dtx; ...
%    The class ltxdoc loads the configuration file ltxdoc.cfg
%    if available. Here you can specify further options, e.g.
%    use A4 as paper format:
%       \PassOptionsToClass{a4paper}{article}
%
%    Programm calls to get the documentation (example):
%       pdflatex pagesel.dtx
%       makeindex -s gind.ist pagesel.idx
%       pdflatex pagesel.dtx
%       makeindex -s gind.ist pagesel.idx
%       pdflatex pagesel.dtx
%
% Installation:
%    TDS:tex/latex/pagesel/pagesel.sty
%    TDS:doc/latex/pagesel/pagesel.pdf
%    TDS:source/latex/pagesel/pagesel.dtx
%
%<*ignore>
\begingroup
  \catcode123=1 %
  \catcode125=2 %
  \def\x{LaTeX2e}%
\expandafter\endgroup
\ifcase 0\ifx\install y1\fi\expandafter
         \ifx\csname processbatchFile\endcsname\relax\else1\fi
         \ifx\fmtname\x\else 1\fi\relax
\else\csname fi\endcsname
%</ignore>
%<*install>
\input docstrip.tex
\Msg{************************************************************************}
\Msg{* Installation}
\Msg{* Package: pagesel 2020-08-03 v1.10 Select pages of a document for output (HO)}
\Msg{************************************************************************}

\keepsilent
\askforoverwritefalse

\let\MetaPrefix\relax
\preamble

This is a generated file.

Project: pagesel
Version: 2020-08-03 v1.10

Copyright (C)
   1999, 2003, 2006-2008 Heiko Oberdiek
   2016-2020 Oberdiek Package Support Group

This work may be distributed and/or modified under the
conditions of the LaTeX Project Public License, either
version 1.3c of this license or (at your option) any later
version. This version of this license is in
   https://www.latex-project.org/lppl/lppl-1-3c.txt
and the latest version of this license is in
   https://www.latex-project.org/lppl.txt
and version 1.3 or later is part of all distributions of
LaTeX version 2005/12/01 or later.

This work has the LPPL maintenance status "maintained".

The Current Maintainers of this work are
Heiko Oberdiek and the Oberdiek Package Support Group
https://github.com/ho-tex/pagesel/issues


This work consists of the main source file pagesel.dtx
and the derived files
   pagesel.sty, pagesel-2016-05-16.sty, pagesel.pdf,
   pagesel.ins, pagesel.drv.

\endpreamble
\let\MetaPrefix\DoubleperCent

\generate{%
  \file{pagesel.ins}{\from{pagesel.dtx}{install}}%
  \file{pagesel.drv}{\from{pagesel.dtx}{driver}}%
  \usedir{tex/latex/pagesel}%
  \file{pagesel.sty}{\from{pagesel.dtx}{package}}%
  \file{pagesel-2016-05-16.sty}{\from{pagesel.dtx}{packagefrozen}}
}

\catcode32=13\relax% active space
\let =\space%
\Msg{************************************************************************}
\Msg{*}
\Msg{* To finish the installation you have to move the following}
\Msg{* file into a directory searched by TeX:}
\Msg{*}
\Msg{*     pagesel.sty}
\Msg{*}
\Msg{* To produce the documentation run the file `pagesel.drv'}
\Msg{* through LaTeX.}
\Msg{*}
\Msg{* Happy TeXing!}
\Msg{*}
\Msg{************************************************************************}

\endbatchfile
%</install>
%<*ignore>
\fi
%</ignore>
%<*driver>
\NeedsTeXFormat{LaTeX2e}
\ProvidesFile{pagesel.drv}%
  [2020-08-03 v1.10 Select pages of a document for output (HO)]%
\documentclass{ltxdoc}
\usepackage{holtxdoc}[2011/11/22]
\begin{document}
  \DocInput{pagesel.dtx}%
\end{document}
%</driver>
% \fi
%
%
%
% \GetFileInfo{pagesel.drv}
%
% \title{The \xpackage{pagesel} package}
% \date{2020-08-03 v1.10}
% \author{Heiko Oberdiek\thanks
% {Please report any issues at \url{https://github.com/ho-tex/pagesel/issues}}}
%
% \maketitle
%
% \begin{abstract}
% Single pages or page areas can be selected for output.
% \end{abstract}
%
% \tableofcontents
%
% \newenvironment{param}{^^A
%   \newcommand{\entry}[1]{\meta{\###1}:&}^^A
%   \begin{tabular}[t]{@{}l@{ }l@{}}^^A
% }{^^A
%   \end{tabular}^^A
% }
%
% \newcommand*{\Option}[1]{\textsf{#1}}
%
% \section{Usage}
%    The package \Package{pagesel} is a \LaTeXe\ package:
%    \begin{quote}
%      |\usepackage|\oarg{options}|{pagesel}|
%    \end{quote}
%    (For plain\TeX\ and \LaTeX\,2.09 the similar package
%    \URL{\Package{selectp}}^^A
%    {https://ctan.org/pkg/selectp}
%    from \NameEmail{Donald Arsenau}{asnd@triumf.ca} can be used.)
%
%    Depending on the options the package works in two modes:
%    \begin{enumerate}
%    \item If no page selecting option is present, so the package
%          ignores the other options and finishes itself. So no
%          page will be suppressed by the package and auxiliary files
%          will be written.
%    \item With at least one page selecting option the specified
%          pages are selected and the other are suppressed.
%          The default for this mode is that auxiliary will not be
%          overwritten. (This can be changed by an option.)
%    \end{enumerate}
%
% \subsection{Page selecting}
%    The package \Package{pagesel} sets up a new counter that is
%    incremented by each \cmd{\shipout}.
%    In this way the package counts the output pages regardless the value
%    of the page counter. So each page can individually by addressed,
%    even if there are several pages with the same page number.
%
% \subsubsection{Options\texorpdfstring{ for selecting pages}{}}
%    \begin{description}
%    \item[\Option{odd}:] The output pages must have an odd number.
%         All even output pages are suppressed. If there are no
%         page areas specified so all odd pages are print. With
%         page areas only the odd pages in this areas are selected.
%    \item[\Option{even}:] The opposite of option \Option{odd}.
%    \item[Page area:] A page area consists of three elements:
%         the starting output page number, an ``area'' hyphen, and
%         the output page number of the last page in this area.
%         Each component is optional, so there are four kinds
%         to spezify a page area:
%         \begin{description}
%         \item[\meta{m}\Option{-}\meta{n}:] All pages between
%              \meta{m} and \meta{n} inclusive.
%         \item[\Option{-}\meta{n}:] All pages until \meta{n} inclusive.
%         \item[\meta{m}\Option{-}:] The page area starts with \meta{m}
%              and all pages to the end of document are selected.
%         \item[\Option{-}:] All pages (not very useful).
%         \item[\meta{s}:] The single page \meta{s}.
%         \end{description}
%    \end{description}
%
% \subsubsection{Examples}
%    \newcommand*{\exam}[1]{\texttt{\strut[#1]}}^^A hash-ok
%    \begin{tabular}{ll}
%      Options & Output pages\\
%      \hline
%      \exam{1, 4, 9}&  1, 4, and 9\\
%      \exam{7-10, 3}&  3, 7, 8, 9, and 10\\
%      \exam{odd, 3-6}& 3, and 5\\
%      \exam{-4, 3, even, 7-8}& 2, 4, and 8\\
%    \end{tabular}
%
% \subsection{Auxiliary files}
%    If a page is suppressed, the \cmd{\write} commands are not
%    performed. Labels, index entries, or entries for the
%    table of contents aren't written. So it is likely that
%    the table of contents, registers, and lists are incomplete.
% \subsubsection{Options\texorpdfstring{ for handling auxiliary files}{}}
%    \begin{description}
%    \item[\Option{nofiles}:] This is the default. Auxiliary files are
%         read but not written or changed. Also the job is aborted
%         after the last selected page for saving time.
%    \item[\Option{nonofiles}/\Option{files}:] Auxiliary files are
%         written.
%    \end{description}
% \subsubsection{\texorpdfstring{Package }{}\Package{hyperref}}
%    In old versions of \Package{hyperref} [1999/04/12 v6.55] (and below)
%    there is a bug with \cmd{\nofiles}:
%    \begin{itemize}
%    \item Some ``garbage'' appears on terminal and in the log file.
%          This is harmless and can be ignored.
%    \item The outline auxiliary file \cmd{\jobname.out}, however,
%          is opened and truncated to zero bytes.
%          Version 1.0 of this package had
%          loaded a patch file \File{hypnofil.tex}, if it detects
%          \Package{hyperref} to get \cmd{\nofiles} work.
%
%          With the new version of \Package{hyperref} [1999/04/13 v6.56]
%          \cmd{\nofiles} works now. Therefore the workaround code
%          is no longer needed and removed.
%    \end{itemize}
%
% \StopEventually{
% }
%
% \section{Implementation}
%    \begin{macrocode}
%<*package>
%    \end{macrocode}
% \subsection{New implementation using the LaTeX kernel hooks}
%    \begin{macrocode}
\NeedsTeXFormat{LaTeX2e}
\ProvidesPackage{pagesel}
  [2020-08-03 v1.10 Select pages of a document for output (HO)]%
%    \end{macrocode}
%    \begin{macrocode}
\providecommand\IfFormatAtLeastTF{\@ifl@t@r\fmtversion}
\IfFormatAtLeastTF{2020/10/01}{}{\input{pagesel-2016-05-16.sty}}
\IfFormatAtLeastTF{2020/10/01}{}{\endinput}

%    \end{macrocode}
%    If the package is loaded twice, the package code does not
%    work. So stop loading the package, if it is already loaded.
%    \begin{macrocode}
\@ifundefined{ps@oddpages}{}{%
  \PackageWarningNoLine{pagesel}{Package already loaded.}%
  \endinput
}
%    \end{macrocode}
%    \begin{macrocode}
%</package>
%    \end{macrocode}
% \subsection{Package}
%    \begin{macrocode}
%<*packagefrozen>
\NeedsTeXFormat{LaTeX2e}
\ProvidesPackage{pagesel}
  [2020-08-03 v1.10 Select pages of a document for output (legacy code) (HO)]%
%    \end{macrocode}
%
%    If the package is loaded twice, the package code does not
%    work. So stop loading the package, if it is already loaded.
%    \begin{macrocode}
\@ifundefined{ps@makevoid}{}{%
  \PackageWarningNoLine{pagesel}{Package already loaded.}%
  \endinput
}
%    \end{macrocode}
%
%    \begin{macro}{\ps@makevoid}
%    Macro \cmd{\ps@makevoid} clears the output box. Because
%    nothing is shipped out and this is intended, we reduce
%    the counter \cmd{\deadcycles} in order to avoid problems, if
%    more than \cmd{\maxdeadcycles} pages are omitted.
%    \begin{macrocode}
\newcommand*{\ps@makevoid}{%
  \global\setbox\@cclv\copy\voidb@x
  \begingroup
    \count@=\deadcycles
    \advance\count@ by -1\relax
    \deadcycles=\count@
  \endgroup
}
%</packagefrozen>
%    \end{macrocode}
%    \end{macro}
%
%    \begin{macro}{\ps@oddpages}
%    \begin{macrocode}
%<*package|packagefrozen>
\newcommand*\ps@oddpages{0}
\DeclareOption{odd}{\renewcommand*\ps@oddpages{1}}
\DeclareOption{even}{\renewcommand*\ps@oddpages{2}}
%    \end{macrocode}
%    \end{macro}
%
%    \begin{macrocode}
\DeclareOption{nofiles}{\let\ps@nofiles\nofiles}
\DeclareOption{nonofiles}{\let\ps@nofiles\@empty}
\DeclareOption{files}{\let\ps@nofiles\@empty}
\ExecuteOptions{nofiles}
%    \end{macrocode}
%
%    \begin{macrocode}
\DeclareOption*{%
  \begingroup
    \expandafter\ps@checkoption\CurrentOption-\END
    \edef\x{\endgroup\noexpand\ps@store{\ps@first}{\ps@last}}%
  \x
}
%    \end{macrocode}
%
%    \begin{macro}{\ps@checkoption}
%    \begin{macrocode}
\newcommand\ps@checkoption{}
\def\ps@checkoption#1-#2\END{%
  \ifx\\#2\\%
    \ifx\\#1\\%
      % empty option
      \def\ps@first{\maxdimen}%
      \def\ps@last{\maxdimen}%
    \else
      \edef\ps@first{#1}%
      \edef\ps@last{#1}%
    \fi
  \else
    \ifx\\#1\\%
      \def\ps@first{-\maxdimen}%
    \else
      \edef\ps@first{#1}%
    \fi
    \ps@checklast#2%
  \fi
}
%    \end{macrocode}
%    \end{macro}
%
%    \begin{macro}{\ps@checklast}
%    \begin{macrocode}
\newcommand\ps@checklast{}
\def\ps@checklast#1-{%
  \ifx\\#1\\%
    \def\ps@last{\maxdimen}%
  \else
    \edef\ps@last{#1}%
  \fi
}
%    \end{macrocode}
%    \end{macro}
%
%    \begin{macro}{\ps@store}
%    \begin{macrocode}
\newcommand*{\ps@store}[2]{%
  \expandafter\def\expandafter\ps@testlist\expandafter{%
    \ps@testlist\ps@pagetest{#1}{#2}%
  }%
}
%    \end{macrocode}
%    \end{macro}
%
%    \begin{macro}{\ps@testlist}
%    \begin{macrocode}
\newcommand*\ps@testlist{}
%    \end{macrocode}
%    \end{macro}
%
%    \begin{macrocode}
\ProcessOptions
%    \end{macrocode}
%
%    \begin{macrocode}
\begingroup
  \edef\x{%
    \ifnum\ps@oddpages>0 \relax\fi
    \ifx\ps@testlist\@empty\else\relax\fi
  }%
  \ifx\x\@empty
    \endgroup
    \PackageInfo{pagesel}{Nothing to do}%
    \expandafter\endinput
  \fi
\endgroup
%    \end{macrocode}
%
%    \begin{macrocode}
%</package|packagefrozen>
%<*packagefrozen>
\RequirePackage{everyshi}
%</packagefrozen>
%    \end{macrocode}
%
%    \begin{macrocode}
%<*package|packagefrozen>
\ps@nofiles
%    \end{macrocode}
%
%    \begin{macro}{\c@ps@count}
%    \begin{macrocode}
\newcounter{ps@count}
\setcounter{ps@count}{0}
%    \end{macrocode}
%    \end{macro}
%
%    \begin{macro}{\ps@ReturnAfterElseFi}
%    \begin{macro}{\ps@ReturnAfterFi}
%    \begin{macrocode}
\long\def\ps@ReturnAfterElseFi#1\else#2\fi{\fi#1}
\long\def\ps@ReturnAfterFi#1\fi{\fi#1}
%    \end{macrocode}
%    \end{macro}
%    \end{macro}
%
%    \begin{macrocode}
\newcommand{\ps@lastpage}{\maxdimen}
\ifx\ps@nofiles\nofiles
  \ifx\ps@testlist\@empty
  \else
    \def\ps@lastpage{0}%
    \newcommand*{\ps@pagetest}[2]{%
      \ifnum#2>\ps@lastpage\relax
        \def\ps@lastpage{#2}%
      \fi
    }%
    \ps@testlist
    \let\ps@pagetest\relax
  \fi
\fi
%    \end{macrocode}
%
%    \begin{macro}{\ps@ifinset}
%    \begin{macrocode}
\newcommand*{\ps@ifinset}[4]{%
  \ifnum#1>\value{ps@count}%
    \ps@ReturnAfterElseFi{#4}%
  \else
    \ps@ReturnAfterFi{%
      \ifnum#2<\value{ps@count}%
        \ps@ReturnAfterElseFi{#4}%
      \else
        \ps@ReturnAfterFi{#3}%
      \fi
    }%
  \fi
}
%    \end{macrocode}
%    \end{macro}
%
%    \begin{macro}{\ps@pagetest}
%    \begin{macrocode}
\newcommand*{\ps@pagetest}[2]{%
  \ps@ifinset{#1}{#2}{\let\ps@next\@empty}{}%
}
%    \end{macrocode}
%    \end{macro}
%
%    \begin{macrocode}
%</package|packagefrozen>
%<packagefrozen>\EveryShipout{%
%<package>\AddToHook{shipout/before}{%
%<*package|packagefrozen>
  \stepcounter{ps@count}%
  \ifnum\value{ps@count}>\ps@lastpage\relax
    \global\output{%
      \ps@cleanup@if
      \ps@group@message
      \typeout{%
        Package pagesel Notice: Aborting LaTeX job %
        after last selected page (\ps@lastpage).%
      }%
      \ps@message@ignore
      \global\setbox\@cclv\box\voidb@x
      \deadcycles0\relax
%    \end{macrocode}
%    First leave the output group before ending the job.
%    \begin{macrocode}
      \aftergroup\@@end
    }%
  \fi
  \let\ps@next\@empty
  \ifx\ps@testlist\@empty
  \else
%<packagefrozen>    \let\ps@next\ps@makevoid
%<package>    \let\ps@next\DiscardShipoutBox
    \ps@testlist
  \fi
  \ifnum\ps@oddpages=1 %
    \ifodd\value{ps@count}%
    \else
%<packagefrozen>    \let\ps@next\ps@makevoid
%<package>    \let\ps@next\DiscardShipoutBox
    \fi
  \fi
  \ifnum\ps@oddpages=2 %
    \ifodd\value{ps@count}%
%<packagefrozen>    \let\ps@next\ps@makevoid
%<package>    \let\ps@next\DiscardShipoutBox
    \else
    \fi
  \fi
%<packagefrozen>  \ps@begindvi
  \ps@next
}
%</package|packagefrozen>
%    \end{macrocode}
%
%    \begin{macrocode}
%<*package|packagefrozen>
%<packagefrozen>\begingroup\expandafter\expandafter\expandafter\endgroup
%<packagefrozen>\expandafter\ifx\csname currentiflevel\endcsname\relax
%<packagefrozen>  \let\ps@cleanup@if\@empty
%<packagefrozen>\else
  \def\ps@cleanup@if{%
    \ifnum\currentiflevel>\@ne
      \csname fi\endcsname
      \expandafter\ps@cleanup@if
    \fi
  }%
%<packagefrozen>\fi
%    \end{macrocode}
%    Because of \cs{aftergroup} it is too dangerous to perform
%    a similar cleanup for groups.
%    \begin{macrocode}
%<packagefrozen> \begingroup\expandafter\expandafter\expandafter\endgroup
%<packagefrozen> \expandafter\ifx\csname currentgrouplevel\endcsname\relax
%<packagefrozen>  \let\ps@group@message\@empty
%<packagefrozen>  \def\ps@message@ignore{%
%<packagefrozen>    \typeout{%
%<packagefrozen>      (pagesel) \space\space\@spaces\@spaces\@spaces
%<packagefrozen>      Messages (\string\end\space occurred ...) can be ignored.%
%<packagefrozen>    }%
%<packagefrozen>  }%
%<packagefrozen>\else
  \def\ps@group@message{%
    \ifnum\currentgrouplevel>\@ne
      \def\ps@message@ignore{%
        \typeout{%
          (pagesel) \space\space\@spaces\@spaces\@spaces
          Message (\string\end\space occurred ...) %
          can be ignored.%
        }%
      }%
    \else
      \let\ps@message@ignore\@empty
    \fi
  }%
%<packagefrozen>\fi
%</package|packagefrozen>
%    \end{macrocode}
%
% \subsection{AtBeginDvi hook support}
%
%    The material of box \cs{@begindvibox} is recorded in parallel
%    in box \cs{ps@begindvibox}.
%    \begin{macrocode}
%<*packagefrozen>
\newbox\ps@begindvibox
\ifvoid\@begindvibox
\else
  \global\setbox\ps@begindvibox\vbox{%
    \unvbox\@begindvibox
  }%
\fi
\let\ps@org@AtBeginDvi\AtBeginDvi
\def\AtBeginDvi#1{%
  \global\setbox\ps@begindvibox\vbox{%
    \unvbox\ps@begindvibox
    #1%
  }%
  \ps@org@AtBeginDvi{#1}%
}
%    \end{macrocode}
%
%    \begin{macro}{\ps@begindvi}
%    Macro \cs{ps@begindvi} is called the similar way as \cs{@begindvi}.
%    If the first page is printed, then \cs{AtBeginDvi} should work
%    as usual. Otherwise the contents of box \cs{ps@begindvibox} is
%    set on the first selected page.
%    \begin{macrocode}
\def\ps@begindvi{%
  \ifx\ps@next\@empty
    \global\let\ps@begindvi\@empty
  \else
    \global\let\ps@begindvi\ps@begindvi@do
  \fi
}
\def\ps@begindvi@do{%
  \ifx\ps@next\@empty
    \setbox\@cclv\vbox{%
      \unvbox\ps@begindvibox
      \box\@cclv
    }%
    \global\let\ps@begindvi\@empty
  \fi
}
%    \end{macrocode}
%    \end{macro}
%
%    \begin{macrocode}
%</packagefrozen>
%    \end{macrocode}
%
% \section{Installation}
%
% \subsection{Download}
%
% \paragraph{Package.} This package is available on
% CTAN\footnote{\CTANpkg{pagesel}}:
% \begin{description}
% \item[\CTAN{macros/latex/contrib/pagesel/pagesel.dtx}] The source file.
% \item[\CTAN{macros/latex/contrib/pagesel/pagesel.pdf}] Documentation.
% \end{description}
%
%
%
% \subsection{Package installation}
%
% \paragraph{Unpacking.} The \xfile{.dtx} file is a self-extracting
% \docstrip\ archive. The files are extracted by running the
% \xfile{.dtx} through \plainTeX:
% \begin{quote}
%   \verb|tex pagesel.dtx|
% \end{quote}
%
% \paragraph{TDS.} Now the different files must be moved into
% the different directories in your installation TDS tree
% (also known as \xfile{texmf} tree):
% \begin{quote}
% \def\t{^^A
% \begin{tabular}{@{}>{\ttfamily}l@{ $\rightarrow$ }>{\ttfamily}l@{}}
%   pagesel.sty & tex/latex/pagesel/pagesel.sty\\
%   pagesel.pdf & doc/latex/pagesel/pagesel.pdf\\
%   pagesel.dtx & source/latex/pagesel/pagesel.dtx\\
% \end{tabular}^^A
% }^^A
% \sbox0{\t}^^A
% \ifdim\wd0>\linewidth
%   \begingroup
%     \advance\linewidth by\leftmargin
%     \advance\linewidth by\rightmargin
%   \edef\x{\endgroup
%     \def\noexpand\lw{\the\linewidth}^^A
%   }\x
%   \def\lwbox{^^A
%     \leavevmode
%     \hbox to \linewidth{^^A
%       \kern-\leftmargin\relax
%       \hss
%       \usebox0
%       \hss
%       \kern-\rightmargin\relax
%     }^^A
%   }^^A
%   \ifdim\wd0>\lw
%     \sbox0{\small\t}^^A
%     \ifdim\wd0>\linewidth
%       \ifdim\wd0>\lw
%         \sbox0{\footnotesize\t}^^A
%         \ifdim\wd0>\linewidth
%           \ifdim\wd0>\lw
%             \sbox0{\scriptsize\t}^^A
%             \ifdim\wd0>\linewidth
%               \ifdim\wd0>\lw
%                 \sbox0{\tiny\t}^^A
%                 \ifdim\wd0>\linewidth
%                   \lwbox
%                 \else
%                   \usebox0
%                 \fi
%               \else
%                 \lwbox
%               \fi
%             \else
%               \usebox0
%             \fi
%           \else
%             \lwbox
%           \fi
%         \else
%           \usebox0
%         \fi
%       \else
%         \lwbox
%       \fi
%     \else
%       \usebox0
%     \fi
%   \else
%     \lwbox
%   \fi
% \else
%   \usebox0
% \fi
% \end{quote}
% If you have a \xfile{docstrip.cfg} that configures and enables \docstrip's
% TDS installing feature, then some files can already be in the right
% place, see the documentation of \docstrip.
%
% \subsection{Refresh file name databases}
%
% If your \TeX~distribution
% (\TeX\,Live, \mikTeX, \dots) relies on file name databases, you must refresh
% these. For example, \TeX\,Live\ users run \verb|texhash| or
% \verb|mktexlsr|.
%
% \subsection{Some details for the interested}
%
% \paragraph{Unpacking with \LaTeX.}
% The \xfile{.dtx} chooses its action depending on the format:
% \begin{description}
% \item[\plainTeX:] Run \docstrip\ and extract the files.
% \item[\LaTeX:] Generate the documentation.
% \end{description}
% If you insist on using \LaTeX\ for \docstrip\ (really,
% \docstrip\ does not need \LaTeX), then inform the autodetect routine
% about your intention:
% \begin{quote}
%   \verb|latex \let\install=y\input{pagesel.dtx}|
% \end{quote}
% Do not forget to quote the argument according to the demands
% of your shell.
%
% \paragraph{Generating the documentation.}
% You can use both the \xfile{.dtx} or the \xfile{.drv} to generate
% the documentation. The process can be configured by the
% configuration file \xfile{ltxdoc.cfg}. For instance, put this
% line into this file, if you want to have A4 as paper format:
% \begin{quote}
%   \verb|\PassOptionsToClass{a4paper}{article}|
% \end{quote}
% An example follows how to generate the
% documentation with pdf\LaTeX:
% \begin{quote}
%\begin{verbatim}
%pdflatex pagesel.dtx
%makeindex -s gind.ist pagesel.idx
%pdflatex pagesel.dtx
%makeindex -s gind.ist pagesel.idx
%pdflatex pagesel.dtx
%\end{verbatim}
% \end{quote}
%
% \begin{History}
%   \begin{Version}{1999/03/01 v0.9}
%   \item
%     The first version was built as a response to a question
%     of \NameEmail{Dirk Kuypers}{dk@comnets.rwth-aachen.de},
%     published in the newsgroup
%     \href{news:de.comp.text.tex}{de.comp.text.tex}:\\
%     \URL{``\link{Re: pdflatex nur fuer bestimmte Seiten?!?}''}^^A
%     {https://groups.google.com/group/de.comp.text.tex/msg/6b68c7b3439fb658}
%   \end{Version}
%   \begin{Version}{1999/04/05 v1.0}
%   \item
%     Documentation added in dtx format.
%   \item
%     Copyright: LPPL (\CTAN{macros/latex/base/lppl.txt})
%   \item
%     Options |odd|, |even| added.
%   \item
%     \cmd{\nofiles} added, bug fix for \Package{hyperref}.
%   \item
%     Abort loading of package, if nothing to do.
%   \end{Version}
%   \begin{Version}{1999/04/13 v1.1}
%   \item
%     \cs{nofiles} bug fix removed
%     because of \xpackage{hyperref} 6.55.
%   \item
%     First CTAN release.
%   \end{Version}
%   \begin{Version}{2003/06/05 v1.2}
%   \item
%     \cs{deadcyles} is decremented for omitted pages.
%   \item
%     LPPL 1.2.
%   \end{Version}
%   \begin{Version}{2006/02/20 v1.3}
%   \item
%     Code is not changed.
%   \item
%     New DTX framework.
%   \item
%     LPPL 1.3
%   \end{Version}
%   \begin{Version}{2006/03/02 v1.4}
%   \item
%     Support for \cs{AtBeginDvi} added.
%   \end{Version}
%   \begin{Version}{2006/03/07 v1.5}
%   \item
%     Job is aborted after last selected page.
%   \end{Version}
%   \begin{Version}{2007/04/11 v1.6}
%   \item
%     Line ends sanitized.
%   \end{Version}
%   \begin{Version}{2007/04/12 v1.7}
%   \item
%     Hard coded box number 255 replaced by macro \cs{@cclv}.
%   \end{Version}
%   \begin{Version}{2008/08/11 v1.8}
%   \item
%     Code is not changed.
%   \item
%     URL updated from \texttt{www.dejanews.com}
%     to \texttt{groups.google.com}.
%   \end{Version}
%   \begin{Version}{2016/05/16 v1.9}
%   \item
%     Documentation updates.
%   \end{Version}
%   \begin{Version}{2020-08-03 v1.10}
%   \item Updated to follow the changes in the hook management
%   of LaTeX 2020/10/01
%   \end{Version}
% \end{History}
%
% \PrintIndex
%
% \Finale
\endinput
|
% \end{quote}
% Do not forget to quote the argument according to the demands
% of your shell.
%
% \paragraph{Generating the documentation.}
% You can use both the \xfile{.dtx} or the \xfile{.drv} to generate
% the documentation. The process can be configured by the
% configuration file \xfile{ltxdoc.cfg}. For instance, put this
% line into this file, if you want to have A4 as paper format:
% \begin{quote}
%   \verb|\PassOptionsToClass{a4paper}{article}|
% \end{quote}
% An example follows how to generate the
% documentation with pdf\LaTeX:
% \begin{quote}
%\begin{verbatim}
%pdflatex pagesel.dtx
%makeindex -s gind.ist pagesel.idx
%pdflatex pagesel.dtx
%makeindex -s gind.ist pagesel.idx
%pdflatex pagesel.dtx
%\end{verbatim}
% \end{quote}
%
% \begin{History}
%   \begin{Version}{1999/03/01 v0.9}
%   \item
%     The first version was built as a response to a question
%     of \NameEmail{Dirk Kuypers}{dk@comnets.rwth-aachen.de},
%     published in the newsgroup
%     \href{news:de.comp.text.tex}{de.comp.text.tex}:\\
%     \URL{``\link{Re: pdflatex nur fuer bestimmte Seiten?!?}''}^^A
%     {https://groups.google.com/group/de.comp.text.tex/msg/6b68c7b3439fb658}
%   \end{Version}
%   \begin{Version}{1999/04/05 v1.0}
%   \item
%     Documentation added in dtx format.
%   \item
%     Copyright: LPPL (\CTAN{macros/latex/base/lppl.txt})
%   \item
%     Options |odd|, |even| added.
%   \item
%     \cmd{\nofiles} added, bug fix for \Package{hyperref}.
%   \item
%     Abort loading of package, if nothing to do.
%   \end{Version}
%   \begin{Version}{1999/04/13 v1.1}
%   \item
%     \cs{nofiles} bug fix removed
%     because of \xpackage{hyperref} 6.55.
%   \item
%     First CTAN release.
%   \end{Version}
%   \begin{Version}{2003/06/05 v1.2}
%   \item
%     \cs{deadcyles} is decremented for omitted pages.
%   \item
%     LPPL 1.2.
%   \end{Version}
%   \begin{Version}{2006/02/20 v1.3}
%   \item
%     Code is not changed.
%   \item
%     New DTX framework.
%   \item
%     LPPL 1.3
%   \end{Version}
%   \begin{Version}{2006/03/02 v1.4}
%   \item
%     Support for \cs{AtBeginDvi} added.
%   \end{Version}
%   \begin{Version}{2006/03/07 v1.5}
%   \item
%     Job is aborted after last selected page.
%   \end{Version}
%   \begin{Version}{2007/04/11 v1.6}
%   \item
%     Line ends sanitized.
%   \end{Version}
%   \begin{Version}{2007/04/12 v1.7}
%   \item
%     Hard coded box number 255 replaced by macro \cs{@cclv}.
%   \end{Version}
%   \begin{Version}{2008/08/11 v1.8}
%   \item
%     Code is not changed.
%   \item
%     URL updated from \texttt{www.dejanews.com}
%     to \texttt{groups.google.com}.
%   \end{Version}
%   \begin{Version}{2016/05/16 v1.9}
%   \item
%     Documentation updates.
%   \end{Version}
%   \begin{Version}{2020-08-03 v1.10}
%   \item Updated to follow the changes in the hook management
%   of LaTeX 2020/10/01
%   \end{Version}
% \end{History}
%
% \PrintIndex
%
% \Finale
\endinput
|
% \end{quote}
% Do not forget to quote the argument according to the demands
% of your shell.
%
% \paragraph{Generating the documentation.}
% You can use both the \xfile{.dtx} or the \xfile{.drv} to generate
% the documentation. The process can be configured by the
% configuration file \xfile{ltxdoc.cfg}. For instance, put this
% line into this file, if you want to have A4 as paper format:
% \begin{quote}
%   \verb|\PassOptionsToClass{a4paper}{article}|
% \end{quote}
% An example follows how to generate the
% documentation with pdf\LaTeX:
% \begin{quote}
%\begin{verbatim}
%pdflatex pagesel.dtx
%makeindex -s gind.ist pagesel.idx
%pdflatex pagesel.dtx
%makeindex -s gind.ist pagesel.idx
%pdflatex pagesel.dtx
%\end{verbatim}
% \end{quote}
%
% \begin{History}
%   \begin{Version}{1999/03/01 v0.9}
%   \item
%     The first version was built as a response to a question
%     of \NameEmail{Dirk Kuypers}{dk@comnets.rwth-aachen.de},
%     published in the newsgroup
%     \href{news:de.comp.text.tex}{de.comp.text.tex}:\\
%     \URL{``\link{Re: pdflatex nur fuer bestimmte Seiten?!?}''}^^A
%     {https://groups.google.com/group/de.comp.text.tex/msg/6b68c7b3439fb658}
%   \end{Version}
%   \begin{Version}{1999/04/05 v1.0}
%   \item
%     Documentation added in dtx format.
%   \item
%     Copyright: LPPL (\CTAN{macros/latex/base/lppl.txt})
%   \item
%     Options |odd|, |even| added.
%   \item
%     \cmd{\nofiles} added, bug fix for \Package{hyperref}.
%   \item
%     Abort loading of package, if nothing to do.
%   \end{Version}
%   \begin{Version}{1999/04/13 v1.1}
%   \item
%     \cs{nofiles} bug fix removed
%     because of \xpackage{hyperref} 6.55.
%   \item
%     First CTAN release.
%   \end{Version}
%   \begin{Version}{2003/06/05 v1.2}
%   \item
%     \cs{deadcyles} is decremented for omitted pages.
%   \item
%     LPPL 1.2.
%   \end{Version}
%   \begin{Version}{2006/02/20 v1.3}
%   \item
%     Code is not changed.
%   \item
%     New DTX framework.
%   \item
%     LPPL 1.3
%   \end{Version}
%   \begin{Version}{2006/03/02 v1.4}
%   \item
%     Support for \cs{AtBeginDvi} added.
%   \end{Version}
%   \begin{Version}{2006/03/07 v1.5}
%   \item
%     Job is aborted after last selected page.
%   \end{Version}
%   \begin{Version}{2007/04/11 v1.6}
%   \item
%     Line ends sanitized.
%   \end{Version}
%   \begin{Version}{2007/04/12 v1.7}
%   \item
%     Hard coded box number 255 replaced by macro \cs{@cclv}.
%   \end{Version}
%   \begin{Version}{2008/08/11 v1.8}
%   \item
%     Code is not changed.
%   \item
%     URL updated from \texttt{www.dejanews.com}
%     to \texttt{groups.google.com}.
%   \end{Version}
%   \begin{Version}{2016/05/16 v1.9}
%   \item
%     Documentation updates.
%   \end{Version}
%   \begin{Version}{2020-08-03 v1.10}
%   \item Updated to follow the changes in the hook management
%   of LaTeX 2020/10/01
%   \end{Version}
% \end{History}
%
% \PrintIndex
%
% \Finale
\endinput

%        (quote the arguments according to the demands of your shell)
%
% Documentation:
%    (a) If pagesel.drv is present:
%           latex pagesel.drv
%    (b) Without pagesel.drv:
%           latex pagesel.dtx; ...
%    The class ltxdoc loads the configuration file ltxdoc.cfg
%    if available. Here you can specify further options, e.g.
%    use A4 as paper format:
%       \PassOptionsToClass{a4paper}{article}
%
%    Programm calls to get the documentation (example):
%       pdflatex pagesel.dtx
%       makeindex -s gind.ist pagesel.idx
%       pdflatex pagesel.dtx
%       makeindex -s gind.ist pagesel.idx
%       pdflatex pagesel.dtx
%
% Installation:
%    TDS:tex/latex/pagesel/pagesel.sty
%    TDS:doc/latex/pagesel/pagesel.pdf
%    TDS:source/latex/pagesel/pagesel.dtx
%
%<*ignore>
\begingroup
  \catcode123=1 %
  \catcode125=2 %
  \def\x{LaTeX2e}%
\expandafter\endgroup
\ifcase 0\ifx\install y1\fi\expandafter
         \ifx\csname processbatchFile\endcsname\relax\else1\fi
         \ifx\fmtname\x\else 1\fi\relax
\else\csname fi\endcsname
%</ignore>
%<*install>
\input docstrip.tex
\Msg{************************************************************************}
\Msg{* Installation}
\Msg{* Package: pagesel 2020-08-03 v1.10 Select pages of a document for output (HO)}
\Msg{************************************************************************}

\keepsilent
\askforoverwritefalse

\let\MetaPrefix\relax
\preamble

This is a generated file.

Project: pagesel
Version: 2020-08-03 v1.10

Copyright (C)
   1999, 2003, 2006-2008 Heiko Oberdiek
   2016-2020 Oberdiek Package Support Group

This work may be distributed and/or modified under the
conditions of the LaTeX Project Public License, either
version 1.3c of this license or (at your option) any later
version. This version of this license is in
   https://www.latex-project.org/lppl/lppl-1-3c.txt
and the latest version of this license is in
   https://www.latex-project.org/lppl.txt
and version 1.3 or later is part of all distributions of
LaTeX version 2005/12/01 or later.

This work has the LPPL maintenance status "maintained".

The Current Maintainers of this work are
Heiko Oberdiek and the Oberdiek Package Support Group
https://github.com/ho-tex/pagesel/issues


This work consists of the main source file pagesel.dtx
and the derived files
   pagesel.sty, pagesel-2016-05-16.sty, pagesel.pdf,
   pagesel.ins, pagesel.drv.

\endpreamble
\let\MetaPrefix\DoubleperCent

\generate{%
  \file{pagesel.ins}{\from{pagesel.dtx}{install}}%
  \file{pagesel.drv}{\from{pagesel.dtx}{driver}}%
  \usedir{tex/latex/pagesel}%
  \file{pagesel.sty}{\from{pagesel.dtx}{package}}%
  \file{pagesel-2016-05-16.sty}{\from{pagesel.dtx}{packagefrozen}}
}

\catcode32=13\relax% active space
\let =\space%
\Msg{************************************************************************}
\Msg{*}
\Msg{* To finish the installation you have to move the following}
\Msg{* file into a directory searched by TeX:}
\Msg{*}
\Msg{*     pagesel.sty}
\Msg{*}
\Msg{* To produce the documentation run the file `pagesel.drv'}
\Msg{* through LaTeX.}
\Msg{*}
\Msg{* Happy TeXing!}
\Msg{*}
\Msg{************************************************************************}

\endbatchfile
%</install>
%<*ignore>
\fi
%</ignore>
%<*driver>
\NeedsTeXFormat{LaTeX2e}
\ProvidesFile{pagesel.drv}%
  [2020-08-03 v1.10 Select pages of a document for output (HO)]%
\documentclass{ltxdoc}
\usepackage{holtxdoc}[2011/11/22]
\begin{document}
  \DocInput{pagesel.dtx}%
\end{document}
%</driver>
% \fi
%
%
%
% \GetFileInfo{pagesel.drv}
%
% \title{The \xpackage{pagesel} package}
% \date{2020-08-03 v1.10}
% \author{Heiko Oberdiek\thanks
% {Please report any issues at \url{https://github.com/ho-tex/pagesel/issues}}}
%
% \maketitle
%
% \begin{abstract}
% Single pages or page areas can be selected for output.
% \end{abstract}
%
% \tableofcontents
%
% \newenvironment{param}{^^A
%   \newcommand{\entry}[1]{\meta{\###1}:&}^^A
%   \begin{tabular}[t]{@{}l@{ }l@{}}^^A
% }{^^A
%   \end{tabular}^^A
% }
%
% \newcommand*{\Option}[1]{\textsf{#1}}
%
% \section{Usage}
%    The package \Package{pagesel} is a \LaTeXe\ package:
%    \begin{quote}
%      |\usepackage|\oarg{options}|{pagesel}|
%    \end{quote}
%    (For plain\TeX\ and \LaTeX\,2.09 the similar package
%    \URL{\Package{selectp}}^^A
%    {https://ctan.org/pkg/selectp}
%    from \NameEmail{Donald Arsenau}{asnd@triumf.ca} can be used.)
%
%    Depending on the options the package works in two modes:
%    \begin{enumerate}
%    \item If no page selecting option is present, so the package
%          ignores the other options and finishes itself. So no
%          page will be suppressed by the package and auxiliary files
%          will be written.
%    \item With at least one page selecting option the specified
%          pages are selected and the other are suppressed.
%          The default for this mode is that auxiliary will not be
%          overwritten. (This can be changed by an option.)
%    \end{enumerate}
%
% \subsection{Page selecting}
%    The package \Package{pagesel} sets up a new counter that is
%    incremented by each \cmd{\shipout}.
%    In this way the package counts the output pages regardless the value
%    of the page counter. So each page can individually by addressed,
%    even if there are several pages with the same page number.
%
% \subsubsection{Options\texorpdfstring{ for selecting pages}{}}
%    \begin{description}
%    \item[\Option{odd}:] The output pages must have an odd number.
%         All even output pages are suppressed. If there are no
%         page areas specified so all odd pages are print. With
%         page areas only the odd pages in this areas are selected.
%    \item[\Option{even}:] The opposite of option \Option{odd}.
%    \item[Page area:] A page area consists of three elements:
%         the starting output page number, an ``area'' hyphen, and
%         the output page number of the last page in this area.
%         Each component is optional, so there are four kinds
%         to spezify a page area:
%         \begin{description}
%         \item[\meta{m}\Option{-}\meta{n}:] All pages between
%              \meta{m} and \meta{n} inclusive.
%         \item[\Option{-}\meta{n}:] All pages until \meta{n} inclusive.
%         \item[\meta{m}\Option{-}:] The page area starts with \meta{m}
%              and all pages to the end of document are selected.
%         \item[\Option{-}:] All pages (not very useful).
%         \item[\meta{s}:] The single page \meta{s}.
%         \end{description}
%    \end{description}
%
% \subsubsection{Examples}
%    \newcommand*{\exam}[1]{\texttt{\strut[#1]}}^^A hash-ok
%    \begin{tabular}{ll}
%      Options & Output pages\\
%      \hline
%      \exam{1, 4, 9}&  1, 4, and 9\\
%      \exam{7-10, 3}&  3, 7, 8, 9, and 10\\
%      \exam{odd, 3-6}& 3, and 5\\
%      \exam{-4, 3, even, 7-8}& 2, 4, and 8\\
%    \end{tabular}
%
% \subsection{Auxiliary files}
%    If a page is suppressed, the \cmd{\write} commands are not
%    performed. Labels, index entries, or entries for the
%    table of contents aren't written. So it is likely that
%    the table of contents, registers, and lists are incomplete.
% \subsubsection{Options\texorpdfstring{ for handling auxiliary files}{}}
%    \begin{description}
%    \item[\Option{nofiles}:] This is the default. Auxiliary files are
%         read but not written or changed. Also the job is aborted
%         after the last selected page for saving time.
%    \item[\Option{nonofiles}/\Option{files}:] Auxiliary files are
%         written.
%    \end{description}
% \subsubsection{\texorpdfstring{Package }{}\Package{hyperref}}
%    In old versions of \Package{hyperref} [1999/04/12 v6.55] (and below)
%    there is a bug with \cmd{\nofiles}:
%    \begin{itemize}
%    \item Some ``garbage'' appears on terminal and in the log file.
%          This is harmless and can be ignored.
%    \item The outline auxiliary file \cmd{\jobname.out}, however,
%          is opened and truncated to zero bytes.
%          Version 1.0 of this package had
%          loaded a patch file \File{hypnofil.tex}, if it detects
%          \Package{hyperref} to get \cmd{\nofiles} work.
%
%          With the new version of \Package{hyperref} [1999/04/13 v6.56]
%          \cmd{\nofiles} works now. Therefore the workaround code
%          is no longer needed and removed.
%    \end{itemize}
%
% \StopEventually{
% }
%
% \section{Implementation}
%    \begin{macrocode}
%<*package>
%    \end{macrocode}
% \subsection{New implementation using the LaTeX kernel hooks}
%    \begin{macrocode}
\NeedsTeXFormat{LaTeX2e}
\ProvidesPackage{pagesel}
  [2020-08-03 v1.10 Select pages of a document for output (HO)]%
%    \end{macrocode}
%    \begin{macrocode}
\providecommand\IfFormatAtLeastTF{\@ifl@t@r\fmtversion}
\IfFormatAtLeastTF{2020/10/01}{}{%%
%% This is file `pagesel-2016-05-16.sty',
%% generated with the docstrip utility.
%%
%% The original source files were:
%%
%% pagesel.dtx  (with options: `packagefrozen')
%% 
%% This is a generated file.
%% 
%% Project: pagesel
%% Version: 2020-08-03 v1.10
%% 
%% Copyright (C)
%%    1999, 2003, 2006-2008 Heiko Oberdiek
%%    2016-2020 Oberdiek Package Support Group
%% 
%% This work may be distributed and/or modified under the
%% conditions of the LaTeX Project Public License, either
%% version 1.3c of this license or (at your option) any later
%% version. This version of this license is in
%%    https://www.latex-project.org/lppl/lppl-1-3c.txt
%% and the latest version of this license is in
%%    https://www.latex-project.org/lppl.txt
%% and version 1.3 or later is part of all distributions of
%% LaTeX version 2005/12/01 or later.
%% 
%% This work has the LPPL maintenance status "maintained".
%% 
%% The Current Maintainers of this work are
%% Heiko Oberdiek and the Oberdiek Package Support Group
%% https://github.com/ho-tex/pagesel/issues
%% 
%% This work consists of the main source file pagesel.dtx
%% and the derived files
%%    pagesel.sty, pagesel-2016-05-16.sty, pagesel.pdf,
%%    pagesel.ins, pagesel.drv.
%% 
\NeedsTeXFormat{LaTeX2e}
\ProvidesPackage{pagesel}
  [2020-08-03 v1.10 Select pages of a document for output (legacy code) (HO)]%
\@ifundefined{ps@makevoid}{}{%
  \PackageWarningNoLine{pagesel}{Package already loaded.}%
  \endinput
}
\newcommand*{\ps@makevoid}{%
  \global\setbox\@cclv\copy\voidb@x
  \begingroup
    \count@=\deadcycles
    \advance\count@ by -1\relax
    \deadcycles=\count@
  \endgroup
}
\newcommand*\ps@oddpages{0}
\DeclareOption{odd}{\renewcommand*\ps@oddpages{1}}
\DeclareOption{even}{\renewcommand*\ps@oddpages{2}}
\DeclareOption{nofiles}{\let\ps@nofiles\nofiles}
\DeclareOption{nonofiles}{\let\ps@nofiles\@empty}
\DeclareOption{files}{\let\ps@nofiles\@empty}
\ExecuteOptions{nofiles}
\DeclareOption*{%
  \begingroup
    \expandafter\ps@checkoption\CurrentOption-\END
    \edef\x{\endgroup\noexpand\ps@store{\ps@first}{\ps@last}}%
  \x
}
\newcommand\ps@checkoption{}
\def\ps@checkoption#1-#2\END{%
  \ifx\\#2\\%
    \ifx\\#1\\%
      % empty option
      \def\ps@first{\maxdimen}%
      \def\ps@last{\maxdimen}%
    \else
      \edef\ps@first{#1}%
      \edef\ps@last{#1}%
    \fi
  \else
    \ifx\\#1\\%
      \def\ps@first{-\maxdimen}%
    \else
      \edef\ps@first{#1}%
    \fi
    \ps@checklast#2%
  \fi
}
\newcommand\ps@checklast{}
\def\ps@checklast#1-{%
  \ifx\\#1\\%
    \def\ps@last{\maxdimen}%
  \else
    \edef\ps@last{#1}%
  \fi
}
\newcommand*{\ps@store}[2]{%
  \expandafter\def\expandafter\ps@testlist\expandafter{%
    \ps@testlist\ps@pagetest{#1}{#2}%
  }%
}
\newcommand*\ps@testlist{}
\ProcessOptions
\begingroup
  \edef\x{%
    \ifnum\ps@oddpages>0 \relax\fi
    \ifx\ps@testlist\@empty\else\relax\fi
  }%
  \ifx\x\@empty
    \endgroup
    \PackageInfo{pagesel}{Nothing to do}%
    \expandafter\endinput
  \fi
\endgroup
\RequirePackage{everyshi}
\ps@nofiles
\newcounter{ps@count}
\setcounter{ps@count}{0}
\long\def\ps@ReturnAfterElseFi#1\else#2\fi{\fi#1}
\long\def\ps@ReturnAfterFi#1\fi{\fi#1}
\newcommand{\ps@lastpage}{\maxdimen}
\ifx\ps@nofiles\nofiles
  \ifx\ps@testlist\@empty
  \else
    \def\ps@lastpage{0}%
    \newcommand*{\ps@pagetest}[2]{%
      \ifnum#2>\ps@lastpage\relax
        \def\ps@lastpage{#2}%
      \fi
    }%
    \ps@testlist
    \let\ps@pagetest\relax
  \fi
\fi
\newcommand*{\ps@ifinset}[4]{%
  \ifnum#1>\value{ps@count}%
    \ps@ReturnAfterElseFi{#4}%
  \else
    \ps@ReturnAfterFi{%
      \ifnum#2<\value{ps@count}%
        \ps@ReturnAfterElseFi{#4}%
      \else
        \ps@ReturnAfterFi{#3}%
      \fi
    }%
  \fi
}
\newcommand*{\ps@pagetest}[2]{%
  \ps@ifinset{#1}{#2}{\let\ps@next\@empty}{}%
}
\EveryShipout{%
  \stepcounter{ps@count}%
  \ifnum\value{ps@count}>\ps@lastpage\relax
    \global\output{%
      \ps@cleanup@if
      \ps@group@message
      \typeout{%
        Package pagesel Notice: Aborting LaTeX job %
        after last selected page (\ps@lastpage).%
      }%
      \ps@message@ignore
      \global\setbox\@cclv\box\voidb@x
      \deadcycles0\relax
      \aftergroup\@@end
    }%
  \fi
  \let\ps@next\@empty
  \ifx\ps@testlist\@empty
  \else
    \let\ps@next\ps@makevoid
    \ps@testlist
  \fi
  \ifnum\ps@oddpages=1 %
    \ifodd\value{ps@count}%
    \else
    \let\ps@next\ps@makevoid
    \fi
  \fi
  \ifnum\ps@oddpages=2 %
    \ifodd\value{ps@count}%
    \let\ps@next\ps@makevoid
    \else
    \fi
  \fi
  \ps@begindvi
  \ps@next
}
\begingroup\expandafter\expandafter\expandafter\endgroup
\expandafter\ifx\csname currentiflevel\endcsname\relax
  \let\ps@cleanup@if\@empty
\else
  \def\ps@cleanup@if{%
    \ifnum\currentiflevel>\@ne
      \csname fi\endcsname
      \expandafter\ps@cleanup@if
    \fi
  }%
\fi
 \begingroup\expandafter\expandafter\expandafter\endgroup
 \expandafter\ifx\csname currentgrouplevel\endcsname\relax
  \let\ps@group@message\@empty
  \def\ps@message@ignore{%
    \typeout{%
      (pagesel) \space\space\@spaces\@spaces\@spaces
      Messages (\string\end\space occurred ...) can be ignored.%
    }%
  }%
\else
  \def\ps@group@message{%
    \ifnum\currentgrouplevel>\@ne
      \def\ps@message@ignore{%
        \typeout{%
          (pagesel) \space\space\@spaces\@spaces\@spaces
          Message (\string\end\space occurred ...) %
          can be ignored.%
        }%
      }%
    \else
      \let\ps@message@ignore\@empty
    \fi
  }%
\fi
\newbox\ps@begindvibox
\ifvoid\@begindvibox
\else
  \global\setbox\ps@begindvibox\vbox{%
    \unvbox\@begindvibox
  }%
\fi
\let\ps@org@AtBeginDvi\AtBeginDvi
\def\AtBeginDvi#1{%
  \global\setbox\ps@begindvibox\vbox{%
    \unvbox\ps@begindvibox
    #1%
  }%
  \ps@org@AtBeginDvi{#1}%
}
\def\ps@begindvi{%
  \ifx\ps@next\@empty
    \global\let\ps@begindvi\@empty
  \else
    \global\let\ps@begindvi\ps@begindvi@do
  \fi
}
\def\ps@begindvi@do{%
  \ifx\ps@next\@empty
    \setbox\@cclv\vbox{%
      \unvbox\ps@begindvibox
      \box\@cclv
    }%
    \global\let\ps@begindvi\@empty
  \fi
}
\endinput
%%
%% End of file `pagesel-2016-05-16.sty'.
}
\IfFormatAtLeastTF{2020/10/01}{}{\endinput}

%    \end{macrocode}
%    If the package is loaded twice, the package code does not
%    work. So stop loading the package, if it is already loaded.
%    \begin{macrocode}
\@ifundefined{ps@oddpages}{}{%
  \PackageWarningNoLine{pagesel}{Package already loaded.}%
  \endinput
}
%    \end{macrocode}
%    \begin{macrocode}
%</package>
%    \end{macrocode}
% \subsection{Package}
%    \begin{macrocode}
%<*packagefrozen>
\NeedsTeXFormat{LaTeX2e}
\ProvidesPackage{pagesel}
  [2020-08-03 v1.10 Select pages of a document for output (legacy code) (HO)]%
%    \end{macrocode}
%
%    If the package is loaded twice, the package code does not
%    work. So stop loading the package, if it is already loaded.
%    \begin{macrocode}
\@ifundefined{ps@makevoid}{}{%
  \PackageWarningNoLine{pagesel}{Package already loaded.}%
  \endinput
}
%    \end{macrocode}
%
%    \begin{macro}{\ps@makevoid}
%    Macro \cmd{\ps@makevoid} clears the output box. Because
%    nothing is shipped out and this is intended, we reduce
%    the counter \cmd{\deadcycles} in order to avoid problems, if
%    more than \cmd{\maxdeadcycles} pages are omitted.
%    \begin{macrocode}
\newcommand*{\ps@makevoid}{%
  \global\setbox\@cclv\copy\voidb@x
  \begingroup
    \count@=\deadcycles
    \advance\count@ by -1\relax
    \deadcycles=\count@
  \endgroup
}
%</packagefrozen>
%    \end{macrocode}
%    \end{macro}
%
%    \begin{macro}{\ps@oddpages}
%    \begin{macrocode}
%<*package|packagefrozen>
\newcommand*\ps@oddpages{0}
\DeclareOption{odd}{\renewcommand*\ps@oddpages{1}}
\DeclareOption{even}{\renewcommand*\ps@oddpages{2}}
%    \end{macrocode}
%    \end{macro}
%
%    \begin{macrocode}
\DeclareOption{nofiles}{\let\ps@nofiles\nofiles}
\DeclareOption{nonofiles}{\let\ps@nofiles\@empty}
\DeclareOption{files}{\let\ps@nofiles\@empty}
\ExecuteOptions{nofiles}
%    \end{macrocode}
%
%    \begin{macrocode}
\DeclareOption*{%
  \begingroup
    \expandafter\ps@checkoption\CurrentOption-\END
    \edef\x{\endgroup\noexpand\ps@store{\ps@first}{\ps@last}}%
  \x
}
%    \end{macrocode}
%
%    \begin{macro}{\ps@checkoption}
%    \begin{macrocode}
\newcommand\ps@checkoption{}
\def\ps@checkoption#1-#2\END{%
  \ifx\\#2\\%
    \ifx\\#1\\%
      % empty option
      \def\ps@first{\maxdimen}%
      \def\ps@last{\maxdimen}%
    \else
      \edef\ps@first{#1}%
      \edef\ps@last{#1}%
    \fi
  \else
    \ifx\\#1\\%
      \def\ps@first{-\maxdimen}%
    \else
      \edef\ps@first{#1}%
    \fi
    \ps@checklast#2%
  \fi
}
%    \end{macrocode}
%    \end{macro}
%
%    \begin{macro}{\ps@checklast}
%    \begin{macrocode}
\newcommand\ps@checklast{}
\def\ps@checklast#1-{%
  \ifx\\#1\\%
    \def\ps@last{\maxdimen}%
  \else
    \edef\ps@last{#1}%
  \fi
}
%    \end{macrocode}
%    \end{macro}
%
%    \begin{macro}{\ps@store}
%    \begin{macrocode}
\newcommand*{\ps@store}[2]{%
  \expandafter\def\expandafter\ps@testlist\expandafter{%
    \ps@testlist\ps@pagetest{#1}{#2}%
  }%
}
%    \end{macrocode}
%    \end{macro}
%
%    \begin{macro}{\ps@testlist}
%    \begin{macrocode}
\newcommand*\ps@testlist{}
%    \end{macrocode}
%    \end{macro}
%
%    \begin{macrocode}
\ProcessOptions
%    \end{macrocode}
%
%    \begin{macrocode}
\begingroup
  \edef\x{%
    \ifnum\ps@oddpages>0 \relax\fi
    \ifx\ps@testlist\@empty\else\relax\fi
  }%
  \ifx\x\@empty
    \endgroup
    \PackageInfo{pagesel}{Nothing to do}%
    \expandafter\endinput
  \fi
\endgroup
%    \end{macrocode}
%
%    \begin{macrocode}
%</package|packagefrozen>
%<*packagefrozen>
\RequirePackage{everyshi}
%</packagefrozen>
%    \end{macrocode}
%
%    \begin{macrocode}
%<*package|packagefrozen>
\ps@nofiles
%    \end{macrocode}
%
%    \begin{macro}{\c@ps@count}
%    \begin{macrocode}
\newcounter{ps@count}
\setcounter{ps@count}{0}
%    \end{macrocode}
%    \end{macro}
%
%    \begin{macro}{\ps@ReturnAfterElseFi}
%    \begin{macro}{\ps@ReturnAfterFi}
%    \begin{macrocode}
\long\def\ps@ReturnAfterElseFi#1\else#2\fi{\fi#1}
\long\def\ps@ReturnAfterFi#1\fi{\fi#1}
%    \end{macrocode}
%    \end{macro}
%    \end{macro}
%
%    \begin{macrocode}
\newcommand{\ps@lastpage}{\maxdimen}
\ifx\ps@nofiles\nofiles
  \ifx\ps@testlist\@empty
  \else
    \def\ps@lastpage{0}%
    \newcommand*{\ps@pagetest}[2]{%
      \ifnum#2>\ps@lastpage\relax
        \def\ps@lastpage{#2}%
      \fi
    }%
    \ps@testlist
    \let\ps@pagetest\relax
  \fi
\fi
%    \end{macrocode}
%
%    \begin{macro}{\ps@ifinset}
%    \begin{macrocode}
\newcommand*{\ps@ifinset}[4]{%
  \ifnum#1>\value{ps@count}%
    \ps@ReturnAfterElseFi{#4}%
  \else
    \ps@ReturnAfterFi{%
      \ifnum#2<\value{ps@count}%
        \ps@ReturnAfterElseFi{#4}%
      \else
        \ps@ReturnAfterFi{#3}%
      \fi
    }%
  \fi
}
%    \end{macrocode}
%    \end{macro}
%
%    \begin{macro}{\ps@pagetest}
%    \begin{macrocode}
\newcommand*{\ps@pagetest}[2]{%
  \ps@ifinset{#1}{#2}{\let\ps@next\@empty}{}%
}
%    \end{macrocode}
%    \end{macro}
%
%    \begin{macrocode}
%</package|packagefrozen>
%<packagefrozen>\EveryShipout{%
%<package>\AddToHook{shipout/before}{%
%<*package|packagefrozen>
  \stepcounter{ps@count}%
  \ifnum\value{ps@count}>\ps@lastpage\relax
    \global\output{%
      \ps@cleanup@if
      \ps@group@message
      \typeout{%
        Package pagesel Notice: Aborting LaTeX job %
        after last selected page (\ps@lastpage).%
      }%
      \ps@message@ignore
      \global\setbox\@cclv\box\voidb@x
      \deadcycles0\relax
%    \end{macrocode}
%    First leave the output group before ending the job.
%    \begin{macrocode}
      \aftergroup\@@end
    }%
  \fi
  \let\ps@next\@empty
  \ifx\ps@testlist\@empty
  \else
%<packagefrozen>    \let\ps@next\ps@makevoid
%<package>    \let\ps@next\DiscardShipoutBox
    \ps@testlist
  \fi
  \ifnum\ps@oddpages=1 %
    \ifodd\value{ps@count}%
    \else
%<packagefrozen>    \let\ps@next\ps@makevoid
%<package>    \let\ps@next\DiscardShipoutBox
    \fi
  \fi
  \ifnum\ps@oddpages=2 %
    \ifodd\value{ps@count}%
%<packagefrozen>    \let\ps@next\ps@makevoid
%<package>    \let\ps@next\DiscardShipoutBox
    \else
    \fi
  \fi
%<packagefrozen>  \ps@begindvi
  \ps@next
}
%</package|packagefrozen>
%    \end{macrocode}
%
%    \begin{macrocode}
%<*package|packagefrozen>
%<packagefrozen>\begingroup\expandafter\expandafter\expandafter\endgroup
%<packagefrozen>\expandafter\ifx\csname currentiflevel\endcsname\relax
%<packagefrozen>  \let\ps@cleanup@if\@empty
%<packagefrozen>\else
  \def\ps@cleanup@if{%
    \ifnum\currentiflevel>\@ne
      \csname fi\endcsname
      \expandafter\ps@cleanup@if
    \fi
  }%
%<packagefrozen>\fi
%    \end{macrocode}
%    Because of \cs{aftergroup} it is too dangerous to perform
%    a similar cleanup for groups.
%    \begin{macrocode}
%<packagefrozen> \begingroup\expandafter\expandafter\expandafter\endgroup
%<packagefrozen> \expandafter\ifx\csname currentgrouplevel\endcsname\relax
%<packagefrozen>  \let\ps@group@message\@empty
%<packagefrozen>  \def\ps@message@ignore{%
%<packagefrozen>    \typeout{%
%<packagefrozen>      (pagesel) \space\space\@spaces\@spaces\@spaces
%<packagefrozen>      Messages (\string\end\space occurred ...) can be ignored.%
%<packagefrozen>    }%
%<packagefrozen>  }%
%<packagefrozen>\else
  \def\ps@group@message{%
    \ifnum\currentgrouplevel>\@ne
      \def\ps@message@ignore{%
        \typeout{%
          (pagesel) \space\space\@spaces\@spaces\@spaces
          Message (\string\end\space occurred ...) %
          can be ignored.%
        }%
      }%
    \else
      \let\ps@message@ignore\@empty
    \fi
  }%
%<packagefrozen>\fi
%</package|packagefrozen>
%    \end{macrocode}
%
% \subsection{AtBeginDvi hook support}
%
%    The material of box \cs{@begindvibox} is recorded in parallel
%    in box \cs{ps@begindvibox}.
%    \begin{macrocode}
%<*packagefrozen>
\newbox\ps@begindvibox
\ifvoid\@begindvibox
\else
  \global\setbox\ps@begindvibox\vbox{%
    \unvbox\@begindvibox
  }%
\fi
\let\ps@org@AtBeginDvi\AtBeginDvi
\def\AtBeginDvi#1{%
  \global\setbox\ps@begindvibox\vbox{%
    \unvbox\ps@begindvibox
    #1%
  }%
  \ps@org@AtBeginDvi{#1}%
}
%    \end{macrocode}
%
%    \begin{macro}{\ps@begindvi}
%    Macro \cs{ps@begindvi} is called the similar way as \cs{@begindvi}.
%    If the first page is printed, then \cs{AtBeginDvi} should work
%    as usual. Otherwise the contents of box \cs{ps@begindvibox} is
%    set on the first selected page.
%    \begin{macrocode}
\def\ps@begindvi{%
  \ifx\ps@next\@empty
    \global\let\ps@begindvi\@empty
  \else
    \global\let\ps@begindvi\ps@begindvi@do
  \fi
}
\def\ps@begindvi@do{%
  \ifx\ps@next\@empty
    \setbox\@cclv\vbox{%
      \unvbox\ps@begindvibox
      \box\@cclv
    }%
    \global\let\ps@begindvi\@empty
  \fi
}
%    \end{macrocode}
%    \end{macro}
%
%    \begin{macrocode}
%</packagefrozen>
%    \end{macrocode}
%
% \section{Installation}
%
% \subsection{Download}
%
% \paragraph{Package.} This package is available on
% CTAN\footnote{\CTANpkg{pagesel}}:
% \begin{description}
% \item[\CTAN{macros/latex/contrib/pagesel/pagesel.dtx}] The source file.
% \item[\CTAN{macros/latex/contrib/pagesel/pagesel.pdf}] Documentation.
% \end{description}
%
%
%
% \subsection{Package installation}
%
% \paragraph{Unpacking.} The \xfile{.dtx} file is a self-extracting
% \docstrip\ archive. The files are extracted by running the
% \xfile{.dtx} through \plainTeX:
% \begin{quote}
%   \verb|tex pagesel.dtx|
% \end{quote}
%
% \paragraph{TDS.} Now the different files must be moved into
% the different directories in your installation TDS tree
% (also known as \xfile{texmf} tree):
% \begin{quote}
% \def\t{^^A
% \begin{tabular}{@{}>{\ttfamily}l@{ $\rightarrow$ }>{\ttfamily}l@{}}
%   pagesel.sty & tex/latex/pagesel/pagesel.sty\\
%   pagesel.pdf & doc/latex/pagesel/pagesel.pdf\\
%   pagesel.dtx & source/latex/pagesel/pagesel.dtx\\
% \end{tabular}^^A
% }^^A
% \sbox0{\t}^^A
% \ifdim\wd0>\linewidth
%   \begingroup
%     \advance\linewidth by\leftmargin
%     \advance\linewidth by\rightmargin
%   \edef\x{\endgroup
%     \def\noexpand\lw{\the\linewidth}^^A
%   }\x
%   \def\lwbox{^^A
%     \leavevmode
%     \hbox to \linewidth{^^A
%       \kern-\leftmargin\relax
%       \hss
%       \usebox0
%       \hss
%       \kern-\rightmargin\relax
%     }^^A
%   }^^A
%   \ifdim\wd0>\lw
%     \sbox0{\small\t}^^A
%     \ifdim\wd0>\linewidth
%       \ifdim\wd0>\lw
%         \sbox0{\footnotesize\t}^^A
%         \ifdim\wd0>\linewidth
%           \ifdim\wd0>\lw
%             \sbox0{\scriptsize\t}^^A
%             \ifdim\wd0>\linewidth
%               \ifdim\wd0>\lw
%                 \sbox0{\tiny\t}^^A
%                 \ifdim\wd0>\linewidth
%                   \lwbox
%                 \else
%                   \usebox0
%                 \fi
%               \else
%                 \lwbox
%               \fi
%             \else
%               \usebox0
%             \fi
%           \else
%             \lwbox
%           \fi
%         \else
%           \usebox0
%         \fi
%       \else
%         \lwbox
%       \fi
%     \else
%       \usebox0
%     \fi
%   \else
%     \lwbox
%   \fi
% \else
%   \usebox0
% \fi
% \end{quote}
% If you have a \xfile{docstrip.cfg} that configures and enables \docstrip's
% TDS installing feature, then some files can already be in the right
% place, see the documentation of \docstrip.
%
% \subsection{Refresh file name databases}
%
% If your \TeX~distribution
% (\TeX\,Live, \mikTeX, \dots) relies on file name databases, you must refresh
% these. For example, \TeX\,Live\ users run \verb|texhash| or
% \verb|mktexlsr|.
%
% \subsection{Some details for the interested}
%
% \paragraph{Unpacking with \LaTeX.}
% The \xfile{.dtx} chooses its action depending on the format:
% \begin{description}
% \item[\plainTeX:] Run \docstrip\ and extract the files.
% \item[\LaTeX:] Generate the documentation.
% \end{description}
% If you insist on using \LaTeX\ for \docstrip\ (really,
% \docstrip\ does not need \LaTeX), then inform the autodetect routine
% about your intention:
% \begin{quote}
%   \verb|latex \let\install=y% \iffalse meta-comment
%
% File: pagesel.dtx
% Version: 2020-08-03 v1.10
% Info: Select pages of a document for output
%
% Copyright (C)
%    1999, 2003, 2006-2008 Heiko Oberdiek
%    2016-2020 Oberdiek Package Support Group
%    https://github.com/ho-tex/pagesel/issues
%
% This work may be distributed and/or modified under the
% conditions of the LaTeX Project Public License, either
% version 1.3c of this license or (at your option) any later
% version. This version of this license is in
%    https://www.latex-project.org/lppl/lppl-1-3c.txt
% and the latest version of this license is in
%    https://www.latex-project.org/lppl.txt
% and version 1.3 or later is part of all distributions of
% LaTeX version 2005/12/01 or later.
%
% This work has the LPPL maintenance status "maintained".
%
% The Current Maintainers of this work are
% Heiko Oberdiek and the Oberdiek Package Support Group
% https://github.com/ho-tex/pagesel/issues
%
% This work consists of the main source file pagesel.dtx
% and the derived files
%    pagesel.sty, pagesel-2016-05-16.sty,
%    pagesel.pdf, pagesel.ins, pagesel.drv.
%
% Distribution:
%    CTAN:macros/latex/contrib/pagesel/pagesel.dtx
%    CTAN:macros/latex/contrib/pagesel/pagesel.pdf
%
% Unpacking:
%    (a) If pagesel.ins is present:
%           tex pagesel.ins
%    (b) Without pagesel.ins:
%           tex pagesel.dtx
%    (c) If you insist on using LaTeX
%           latex \let\install=y% \iffalse meta-comment
%
% File: pagesel.dtx
% Version: 2020-08-03 v1.10
% Info: Select pages of a document for output
%
% Copyright (C)
%    1999, 2003, 2006-2008 Heiko Oberdiek
%    2016-2020 Oberdiek Package Support Group
%    https://github.com/ho-tex/pagesel/issues
%
% This work may be distributed and/or modified under the
% conditions of the LaTeX Project Public License, either
% version 1.3c of this license or (at your option) any later
% version. This version of this license is in
%    https://www.latex-project.org/lppl/lppl-1-3c.txt
% and the latest version of this license is in
%    https://www.latex-project.org/lppl.txt
% and version 1.3 or later is part of all distributions of
% LaTeX version 2005/12/01 or later.
%
% This work has the LPPL maintenance status "maintained".
%
% The Current Maintainers of this work are
% Heiko Oberdiek and the Oberdiek Package Support Group
% https://github.com/ho-tex/pagesel/issues
%
% This work consists of the main source file pagesel.dtx
% and the derived files
%    pagesel.sty, pagesel-2016-05-16.sty,
%    pagesel.pdf, pagesel.ins, pagesel.drv.
%
% Distribution:
%    CTAN:macros/latex/contrib/pagesel/pagesel.dtx
%    CTAN:macros/latex/contrib/pagesel/pagesel.pdf
%
% Unpacking:
%    (a) If pagesel.ins is present:
%           tex pagesel.ins
%    (b) Without pagesel.ins:
%           tex pagesel.dtx
%    (c) If you insist on using LaTeX
%           latex \let\install=y% \iffalse meta-comment
%
% File: pagesel.dtx
% Version: 2020-08-03 v1.10
% Info: Select pages of a document for output
%
% Copyright (C)
%    1999, 2003, 2006-2008 Heiko Oberdiek
%    2016-2020 Oberdiek Package Support Group
%    https://github.com/ho-tex/pagesel/issues
%
% This work may be distributed and/or modified under the
% conditions of the LaTeX Project Public License, either
% version 1.3c of this license or (at your option) any later
% version. This version of this license is in
%    https://www.latex-project.org/lppl/lppl-1-3c.txt
% and the latest version of this license is in
%    https://www.latex-project.org/lppl.txt
% and version 1.3 or later is part of all distributions of
% LaTeX version 2005/12/01 or later.
%
% This work has the LPPL maintenance status "maintained".
%
% The Current Maintainers of this work are
% Heiko Oberdiek and the Oberdiek Package Support Group
% https://github.com/ho-tex/pagesel/issues
%
% This work consists of the main source file pagesel.dtx
% and the derived files
%    pagesel.sty, pagesel-2016-05-16.sty,
%    pagesel.pdf, pagesel.ins, pagesel.drv.
%
% Distribution:
%    CTAN:macros/latex/contrib/pagesel/pagesel.dtx
%    CTAN:macros/latex/contrib/pagesel/pagesel.pdf
%
% Unpacking:
%    (a) If pagesel.ins is present:
%           tex pagesel.ins
%    (b) Without pagesel.ins:
%           tex pagesel.dtx
%    (c) If you insist on using LaTeX
%           latex \let\install=y\input{pagesel.dtx}
%        (quote the arguments according to the demands of your shell)
%
% Documentation:
%    (a) If pagesel.drv is present:
%           latex pagesel.drv
%    (b) Without pagesel.drv:
%           latex pagesel.dtx; ...
%    The class ltxdoc loads the configuration file ltxdoc.cfg
%    if available. Here you can specify further options, e.g.
%    use A4 as paper format:
%       \PassOptionsToClass{a4paper}{article}
%
%    Programm calls to get the documentation (example):
%       pdflatex pagesel.dtx
%       makeindex -s gind.ist pagesel.idx
%       pdflatex pagesel.dtx
%       makeindex -s gind.ist pagesel.idx
%       pdflatex pagesel.dtx
%
% Installation:
%    TDS:tex/latex/pagesel/pagesel.sty
%    TDS:doc/latex/pagesel/pagesel.pdf
%    TDS:source/latex/pagesel/pagesel.dtx
%
%<*ignore>
\begingroup
  \catcode123=1 %
  \catcode125=2 %
  \def\x{LaTeX2e}%
\expandafter\endgroup
\ifcase 0\ifx\install y1\fi\expandafter
         \ifx\csname processbatchFile\endcsname\relax\else1\fi
         \ifx\fmtname\x\else 1\fi\relax
\else\csname fi\endcsname
%</ignore>
%<*install>
\input docstrip.tex
\Msg{************************************************************************}
\Msg{* Installation}
\Msg{* Package: pagesel 2020-08-03 v1.10 Select pages of a document for output (HO)}
\Msg{************************************************************************}

\keepsilent
\askforoverwritefalse

\let\MetaPrefix\relax
\preamble

This is a generated file.

Project: pagesel
Version: 2020-08-03 v1.10

Copyright (C)
   1999, 2003, 2006-2008 Heiko Oberdiek
   2016-2020 Oberdiek Package Support Group

This work may be distributed and/or modified under the
conditions of the LaTeX Project Public License, either
version 1.3c of this license or (at your option) any later
version. This version of this license is in
   https://www.latex-project.org/lppl/lppl-1-3c.txt
and the latest version of this license is in
   https://www.latex-project.org/lppl.txt
and version 1.3 or later is part of all distributions of
LaTeX version 2005/12/01 or later.

This work has the LPPL maintenance status "maintained".

The Current Maintainers of this work are
Heiko Oberdiek and the Oberdiek Package Support Group
https://github.com/ho-tex/pagesel/issues


This work consists of the main source file pagesel.dtx
and the derived files
   pagesel.sty, pagesel-2016-05-16.sty, pagesel.pdf,
   pagesel.ins, pagesel.drv.

\endpreamble
\let\MetaPrefix\DoubleperCent

\generate{%
  \file{pagesel.ins}{\from{pagesel.dtx}{install}}%
  \file{pagesel.drv}{\from{pagesel.dtx}{driver}}%
  \usedir{tex/latex/pagesel}%
  \file{pagesel.sty}{\from{pagesel.dtx}{package}}%
  \file{pagesel-2016-05-16.sty}{\from{pagesel.dtx}{packagefrozen}}
}

\catcode32=13\relax% active space
\let =\space%
\Msg{************************************************************************}
\Msg{*}
\Msg{* To finish the installation you have to move the following}
\Msg{* file into a directory searched by TeX:}
\Msg{*}
\Msg{*     pagesel.sty}
\Msg{*}
\Msg{* To produce the documentation run the file `pagesel.drv'}
\Msg{* through LaTeX.}
\Msg{*}
\Msg{* Happy TeXing!}
\Msg{*}
\Msg{************************************************************************}

\endbatchfile
%</install>
%<*ignore>
\fi
%</ignore>
%<*driver>
\NeedsTeXFormat{LaTeX2e}
\ProvidesFile{pagesel.drv}%
  [2020-08-03 v1.10 Select pages of a document for output (HO)]%
\documentclass{ltxdoc}
\usepackage{holtxdoc}[2011/11/22]
\begin{document}
  \DocInput{pagesel.dtx}%
\end{document}
%</driver>
% \fi
%
%
%
% \GetFileInfo{pagesel.drv}
%
% \title{The \xpackage{pagesel} package}
% \date{2020-08-03 v1.10}
% \author{Heiko Oberdiek\thanks
% {Please report any issues at \url{https://github.com/ho-tex/pagesel/issues}}}
%
% \maketitle
%
% \begin{abstract}
% Single pages or page areas can be selected for output.
% \end{abstract}
%
% \tableofcontents
%
% \newenvironment{param}{^^A
%   \newcommand{\entry}[1]{\meta{\###1}:&}^^A
%   \begin{tabular}[t]{@{}l@{ }l@{}}^^A
% }{^^A
%   \end{tabular}^^A
% }
%
% \newcommand*{\Option}[1]{\textsf{#1}}
%
% \section{Usage}
%    The package \Package{pagesel} is a \LaTeXe\ package:
%    \begin{quote}
%      |\usepackage|\oarg{options}|{pagesel}|
%    \end{quote}
%    (For plain\TeX\ and \LaTeX\,2.09 the similar package
%    \URL{\Package{selectp}}^^A
%    {https://ctan.org/pkg/selectp}
%    from \NameEmail{Donald Arsenau}{asnd@triumf.ca} can be used.)
%
%    Depending on the options the package works in two modes:
%    \begin{enumerate}
%    \item If no page selecting option is present, so the package
%          ignores the other options and finishes itself. So no
%          page will be suppressed by the package and auxiliary files
%          will be written.
%    \item With at least one page selecting option the specified
%          pages are selected and the other are suppressed.
%          The default for this mode is that auxiliary will not be
%          overwritten. (This can be changed by an option.)
%    \end{enumerate}
%
% \subsection{Page selecting}
%    The package \Package{pagesel} sets up a new counter that is
%    incremented by each \cmd{\shipout}.
%    In this way the package counts the output pages regardless the value
%    of the page counter. So each page can individually by addressed,
%    even if there are several pages with the same page number.
%
% \subsubsection{Options\texorpdfstring{ for selecting pages}{}}
%    \begin{description}
%    \item[\Option{odd}:] The output pages must have an odd number.
%         All even output pages are suppressed. If there are no
%         page areas specified so all odd pages are print. With
%         page areas only the odd pages in this areas are selected.
%    \item[\Option{even}:] The opposite of option \Option{odd}.
%    \item[Page area:] A page area consists of three elements:
%         the starting output page number, an ``area'' hyphen, and
%         the output page number of the last page in this area.
%         Each component is optional, so there are four kinds
%         to spezify a page area:
%         \begin{description}
%         \item[\meta{m}\Option{-}\meta{n}:] All pages between
%              \meta{m} and \meta{n} inclusive.
%         \item[\Option{-}\meta{n}:] All pages until \meta{n} inclusive.
%         \item[\meta{m}\Option{-}:] The page area starts with \meta{m}
%              and all pages to the end of document are selected.
%         \item[\Option{-}:] All pages (not very useful).
%         \item[\meta{s}:] The single page \meta{s}.
%         \end{description}
%    \end{description}
%
% \subsubsection{Examples}
%    \newcommand*{\exam}[1]{\texttt{\strut[#1]}}^^A hash-ok
%    \begin{tabular}{ll}
%      Options & Output pages\\
%      \hline
%      \exam{1, 4, 9}&  1, 4, and 9\\
%      \exam{7-10, 3}&  3, 7, 8, 9, and 10\\
%      \exam{odd, 3-6}& 3, and 5\\
%      \exam{-4, 3, even, 7-8}& 2, 4, and 8\\
%    \end{tabular}
%
% \subsection{Auxiliary files}
%    If a page is suppressed, the \cmd{\write} commands are not
%    performed. Labels, index entries, or entries for the
%    table of contents aren't written. So it is likely that
%    the table of contents, registers, and lists are incomplete.
% \subsubsection{Options\texorpdfstring{ for handling auxiliary files}{}}
%    \begin{description}
%    \item[\Option{nofiles}:] This is the default. Auxiliary files are
%         read but not written or changed. Also the job is aborted
%         after the last selected page for saving time.
%    \item[\Option{nonofiles}/\Option{files}:] Auxiliary files are
%         written.
%    \end{description}
% \subsubsection{\texorpdfstring{Package }{}\Package{hyperref}}
%    In old versions of \Package{hyperref} [1999/04/12 v6.55] (and below)
%    there is a bug with \cmd{\nofiles}:
%    \begin{itemize}
%    \item Some ``garbage'' appears on terminal and in the log file.
%          This is harmless and can be ignored.
%    \item The outline auxiliary file \cmd{\jobname.out}, however,
%          is opened and truncated to zero bytes.
%          Version 1.0 of this package had
%          loaded a patch file \File{hypnofil.tex}, if it detects
%          \Package{hyperref} to get \cmd{\nofiles} work.
%
%          With the new version of \Package{hyperref} [1999/04/13 v6.56]
%          \cmd{\nofiles} works now. Therefore the workaround code
%          is no longer needed and removed.
%    \end{itemize}
%
% \StopEventually{
% }
%
% \section{Implementation}
%    \begin{macrocode}
%<*package>
%    \end{macrocode}
% \subsection{New implementation using the LaTeX kernel hooks}
%    \begin{macrocode}
\NeedsTeXFormat{LaTeX2e}
\ProvidesPackage{pagesel}
  [2020-08-03 v1.10 Select pages of a document for output (HO)]%
%    \end{macrocode}
%    \begin{macrocode}
\providecommand\IfFormatAtLeastTF{\@ifl@t@r\fmtversion}
\IfFormatAtLeastTF{2020/10/01}{}{\input{pagesel-2016-05-16.sty}}
\IfFormatAtLeastTF{2020/10/01}{}{\endinput}

%    \end{macrocode}
%    If the package is loaded twice, the package code does not
%    work. So stop loading the package, if it is already loaded.
%    \begin{macrocode}
\@ifundefined{ps@oddpages}{}{%
  \PackageWarningNoLine{pagesel}{Package already loaded.}%
  \endinput
}
%    \end{macrocode}
%    \begin{macrocode}
%</package>
%    \end{macrocode}
% \subsection{Package}
%    \begin{macrocode}
%<*packagefrozen>
\NeedsTeXFormat{LaTeX2e}
\ProvidesPackage{pagesel}
  [2020-08-03 v1.10 Select pages of a document for output (legacy code) (HO)]%
%    \end{macrocode}
%
%    If the package is loaded twice, the package code does not
%    work. So stop loading the package, if it is already loaded.
%    \begin{macrocode}
\@ifundefined{ps@makevoid}{}{%
  \PackageWarningNoLine{pagesel}{Package already loaded.}%
  \endinput
}
%    \end{macrocode}
%
%    \begin{macro}{\ps@makevoid}
%    Macro \cmd{\ps@makevoid} clears the output box. Because
%    nothing is shipped out and this is intended, we reduce
%    the counter \cmd{\deadcycles} in order to avoid problems, if
%    more than \cmd{\maxdeadcycles} pages are omitted.
%    \begin{macrocode}
\newcommand*{\ps@makevoid}{%
  \global\setbox\@cclv\copy\voidb@x
  \begingroup
    \count@=\deadcycles
    \advance\count@ by -1\relax
    \deadcycles=\count@
  \endgroup
}
%</packagefrozen>
%    \end{macrocode}
%    \end{macro}
%
%    \begin{macro}{\ps@oddpages}
%    \begin{macrocode}
%<*package|packagefrozen>
\newcommand*\ps@oddpages{0}
\DeclareOption{odd}{\renewcommand*\ps@oddpages{1}}
\DeclareOption{even}{\renewcommand*\ps@oddpages{2}}
%    \end{macrocode}
%    \end{macro}
%
%    \begin{macrocode}
\DeclareOption{nofiles}{\let\ps@nofiles\nofiles}
\DeclareOption{nonofiles}{\let\ps@nofiles\@empty}
\DeclareOption{files}{\let\ps@nofiles\@empty}
\ExecuteOptions{nofiles}
%    \end{macrocode}
%
%    \begin{macrocode}
\DeclareOption*{%
  \begingroup
    \expandafter\ps@checkoption\CurrentOption-\END
    \edef\x{\endgroup\noexpand\ps@store{\ps@first}{\ps@last}}%
  \x
}
%    \end{macrocode}
%
%    \begin{macro}{\ps@checkoption}
%    \begin{macrocode}
\newcommand\ps@checkoption{}
\def\ps@checkoption#1-#2\END{%
  \ifx\\#2\\%
    \ifx\\#1\\%
      % empty option
      \def\ps@first{\maxdimen}%
      \def\ps@last{\maxdimen}%
    \else
      \edef\ps@first{#1}%
      \edef\ps@last{#1}%
    \fi
  \else
    \ifx\\#1\\%
      \def\ps@first{-\maxdimen}%
    \else
      \edef\ps@first{#1}%
    \fi
    \ps@checklast#2%
  \fi
}
%    \end{macrocode}
%    \end{macro}
%
%    \begin{macro}{\ps@checklast}
%    \begin{macrocode}
\newcommand\ps@checklast{}
\def\ps@checklast#1-{%
  \ifx\\#1\\%
    \def\ps@last{\maxdimen}%
  \else
    \edef\ps@last{#1}%
  \fi
}
%    \end{macrocode}
%    \end{macro}
%
%    \begin{macro}{\ps@store}
%    \begin{macrocode}
\newcommand*{\ps@store}[2]{%
  \expandafter\def\expandafter\ps@testlist\expandafter{%
    \ps@testlist\ps@pagetest{#1}{#2}%
  }%
}
%    \end{macrocode}
%    \end{macro}
%
%    \begin{macro}{\ps@testlist}
%    \begin{macrocode}
\newcommand*\ps@testlist{}
%    \end{macrocode}
%    \end{macro}
%
%    \begin{macrocode}
\ProcessOptions
%    \end{macrocode}
%
%    \begin{macrocode}
\begingroup
  \edef\x{%
    \ifnum\ps@oddpages>0 \relax\fi
    \ifx\ps@testlist\@empty\else\relax\fi
  }%
  \ifx\x\@empty
    \endgroup
    \PackageInfo{pagesel}{Nothing to do}%
    \expandafter\endinput
  \fi
\endgroup
%    \end{macrocode}
%
%    \begin{macrocode}
%</package|packagefrozen>
%<*packagefrozen>
\RequirePackage{everyshi}
%</packagefrozen>
%    \end{macrocode}
%
%    \begin{macrocode}
%<*package|packagefrozen>
\ps@nofiles
%    \end{macrocode}
%
%    \begin{macro}{\c@ps@count}
%    \begin{macrocode}
\newcounter{ps@count}
\setcounter{ps@count}{0}
%    \end{macrocode}
%    \end{macro}
%
%    \begin{macro}{\ps@ReturnAfterElseFi}
%    \begin{macro}{\ps@ReturnAfterFi}
%    \begin{macrocode}
\long\def\ps@ReturnAfterElseFi#1\else#2\fi{\fi#1}
\long\def\ps@ReturnAfterFi#1\fi{\fi#1}
%    \end{macrocode}
%    \end{macro}
%    \end{macro}
%
%    \begin{macrocode}
\newcommand{\ps@lastpage}{\maxdimen}
\ifx\ps@nofiles\nofiles
  \ifx\ps@testlist\@empty
  \else
    \def\ps@lastpage{0}%
    \newcommand*{\ps@pagetest}[2]{%
      \ifnum#2>\ps@lastpage\relax
        \def\ps@lastpage{#2}%
      \fi
    }%
    \ps@testlist
    \let\ps@pagetest\relax
  \fi
\fi
%    \end{macrocode}
%
%    \begin{macro}{\ps@ifinset}
%    \begin{macrocode}
\newcommand*{\ps@ifinset}[4]{%
  \ifnum#1>\value{ps@count}%
    \ps@ReturnAfterElseFi{#4}%
  \else
    \ps@ReturnAfterFi{%
      \ifnum#2<\value{ps@count}%
        \ps@ReturnAfterElseFi{#4}%
      \else
        \ps@ReturnAfterFi{#3}%
      \fi
    }%
  \fi
}
%    \end{macrocode}
%    \end{macro}
%
%    \begin{macro}{\ps@pagetest}
%    \begin{macrocode}
\newcommand*{\ps@pagetest}[2]{%
  \ps@ifinset{#1}{#2}{\let\ps@next\@empty}{}%
}
%    \end{macrocode}
%    \end{macro}
%
%    \begin{macrocode}
%</package|packagefrozen>
%<packagefrozen>\EveryShipout{%
%<package>\AddToHook{shipout/before}{%
%<*package|packagefrozen>
  \stepcounter{ps@count}%
  \ifnum\value{ps@count}>\ps@lastpage\relax
    \global\output{%
      \ps@cleanup@if
      \ps@group@message
      \typeout{%
        Package pagesel Notice: Aborting LaTeX job %
        after last selected page (\ps@lastpage).%
      }%
      \ps@message@ignore
      \global\setbox\@cclv\box\voidb@x
      \deadcycles0\relax
%    \end{macrocode}
%    First leave the output group before ending the job.
%    \begin{macrocode}
      \aftergroup\@@end
    }%
  \fi
  \let\ps@next\@empty
  \ifx\ps@testlist\@empty
  \else
%<packagefrozen>    \let\ps@next\ps@makevoid
%<package>    \let\ps@next\DiscardShipoutBox
    \ps@testlist
  \fi
  \ifnum\ps@oddpages=1 %
    \ifodd\value{ps@count}%
    \else
%<packagefrozen>    \let\ps@next\ps@makevoid
%<package>    \let\ps@next\DiscardShipoutBox
    \fi
  \fi
  \ifnum\ps@oddpages=2 %
    \ifodd\value{ps@count}%
%<packagefrozen>    \let\ps@next\ps@makevoid
%<package>    \let\ps@next\DiscardShipoutBox
    \else
    \fi
  \fi
%<packagefrozen>  \ps@begindvi
  \ps@next
}
%</package|packagefrozen>
%    \end{macrocode}
%
%    \begin{macrocode}
%<*package|packagefrozen>
%<packagefrozen>\begingroup\expandafter\expandafter\expandafter\endgroup
%<packagefrozen>\expandafter\ifx\csname currentiflevel\endcsname\relax
%<packagefrozen>  \let\ps@cleanup@if\@empty
%<packagefrozen>\else
  \def\ps@cleanup@if{%
    \ifnum\currentiflevel>\@ne
      \csname fi\endcsname
      \expandafter\ps@cleanup@if
    \fi
  }%
%<packagefrozen>\fi
%    \end{macrocode}
%    Because of \cs{aftergroup} it is too dangerous to perform
%    a similar cleanup for groups.
%    \begin{macrocode}
%<packagefrozen> \begingroup\expandafter\expandafter\expandafter\endgroup
%<packagefrozen> \expandafter\ifx\csname currentgrouplevel\endcsname\relax
%<packagefrozen>  \let\ps@group@message\@empty
%<packagefrozen>  \def\ps@message@ignore{%
%<packagefrozen>    \typeout{%
%<packagefrozen>      (pagesel) \space\space\@spaces\@spaces\@spaces
%<packagefrozen>      Messages (\string\end\space occurred ...) can be ignored.%
%<packagefrozen>    }%
%<packagefrozen>  }%
%<packagefrozen>\else
  \def\ps@group@message{%
    \ifnum\currentgrouplevel>\@ne
      \def\ps@message@ignore{%
        \typeout{%
          (pagesel) \space\space\@spaces\@spaces\@spaces
          Message (\string\end\space occurred ...) %
          can be ignored.%
        }%
      }%
    \else
      \let\ps@message@ignore\@empty
    \fi
  }%
%<packagefrozen>\fi
%</package|packagefrozen>
%    \end{macrocode}
%
% \subsection{AtBeginDvi hook support}
%
%    The material of box \cs{@begindvibox} is recorded in parallel
%    in box \cs{ps@begindvibox}.
%    \begin{macrocode}
%<*packagefrozen>
\newbox\ps@begindvibox
\ifvoid\@begindvibox
\else
  \global\setbox\ps@begindvibox\vbox{%
    \unvbox\@begindvibox
  }%
\fi
\let\ps@org@AtBeginDvi\AtBeginDvi
\def\AtBeginDvi#1{%
  \global\setbox\ps@begindvibox\vbox{%
    \unvbox\ps@begindvibox
    #1%
  }%
  \ps@org@AtBeginDvi{#1}%
}
%    \end{macrocode}
%
%    \begin{macro}{\ps@begindvi}
%    Macro \cs{ps@begindvi} is called the similar way as \cs{@begindvi}.
%    If the first page is printed, then \cs{AtBeginDvi} should work
%    as usual. Otherwise the contents of box \cs{ps@begindvibox} is
%    set on the first selected page.
%    \begin{macrocode}
\def\ps@begindvi{%
  \ifx\ps@next\@empty
    \global\let\ps@begindvi\@empty
  \else
    \global\let\ps@begindvi\ps@begindvi@do
  \fi
}
\def\ps@begindvi@do{%
  \ifx\ps@next\@empty
    \setbox\@cclv\vbox{%
      \unvbox\ps@begindvibox
      \box\@cclv
    }%
    \global\let\ps@begindvi\@empty
  \fi
}
%    \end{macrocode}
%    \end{macro}
%
%    \begin{macrocode}
%</packagefrozen>
%    \end{macrocode}
%
% \section{Installation}
%
% \subsection{Download}
%
% \paragraph{Package.} This package is available on
% CTAN\footnote{\CTANpkg{pagesel}}:
% \begin{description}
% \item[\CTAN{macros/latex/contrib/pagesel/pagesel.dtx}] The source file.
% \item[\CTAN{macros/latex/contrib/pagesel/pagesel.pdf}] Documentation.
% \end{description}
%
%
%
% \subsection{Package installation}
%
% \paragraph{Unpacking.} The \xfile{.dtx} file is a self-extracting
% \docstrip\ archive. The files are extracted by running the
% \xfile{.dtx} through \plainTeX:
% \begin{quote}
%   \verb|tex pagesel.dtx|
% \end{quote}
%
% \paragraph{TDS.} Now the different files must be moved into
% the different directories in your installation TDS tree
% (also known as \xfile{texmf} tree):
% \begin{quote}
% \def\t{^^A
% \begin{tabular}{@{}>{\ttfamily}l@{ $\rightarrow$ }>{\ttfamily}l@{}}
%   pagesel.sty & tex/latex/pagesel/pagesel.sty\\
%   pagesel.pdf & doc/latex/pagesel/pagesel.pdf\\
%   pagesel.dtx & source/latex/pagesel/pagesel.dtx\\
% \end{tabular}^^A
% }^^A
% \sbox0{\t}^^A
% \ifdim\wd0>\linewidth
%   \begingroup
%     \advance\linewidth by\leftmargin
%     \advance\linewidth by\rightmargin
%   \edef\x{\endgroup
%     \def\noexpand\lw{\the\linewidth}^^A
%   }\x
%   \def\lwbox{^^A
%     \leavevmode
%     \hbox to \linewidth{^^A
%       \kern-\leftmargin\relax
%       \hss
%       \usebox0
%       \hss
%       \kern-\rightmargin\relax
%     }^^A
%   }^^A
%   \ifdim\wd0>\lw
%     \sbox0{\small\t}^^A
%     \ifdim\wd0>\linewidth
%       \ifdim\wd0>\lw
%         \sbox0{\footnotesize\t}^^A
%         \ifdim\wd0>\linewidth
%           \ifdim\wd0>\lw
%             \sbox0{\scriptsize\t}^^A
%             \ifdim\wd0>\linewidth
%               \ifdim\wd0>\lw
%                 \sbox0{\tiny\t}^^A
%                 \ifdim\wd0>\linewidth
%                   \lwbox
%                 \else
%                   \usebox0
%                 \fi
%               \else
%                 \lwbox
%               \fi
%             \else
%               \usebox0
%             \fi
%           \else
%             \lwbox
%           \fi
%         \else
%           \usebox0
%         \fi
%       \else
%         \lwbox
%       \fi
%     \else
%       \usebox0
%     \fi
%   \else
%     \lwbox
%   \fi
% \else
%   \usebox0
% \fi
% \end{quote}
% If you have a \xfile{docstrip.cfg} that configures and enables \docstrip's
% TDS installing feature, then some files can already be in the right
% place, see the documentation of \docstrip.
%
% \subsection{Refresh file name databases}
%
% If your \TeX~distribution
% (\TeX\,Live, \mikTeX, \dots) relies on file name databases, you must refresh
% these. For example, \TeX\,Live\ users run \verb|texhash| or
% \verb|mktexlsr|.
%
% \subsection{Some details for the interested}
%
% \paragraph{Unpacking with \LaTeX.}
% The \xfile{.dtx} chooses its action depending on the format:
% \begin{description}
% \item[\plainTeX:] Run \docstrip\ and extract the files.
% \item[\LaTeX:] Generate the documentation.
% \end{description}
% If you insist on using \LaTeX\ for \docstrip\ (really,
% \docstrip\ does not need \LaTeX), then inform the autodetect routine
% about your intention:
% \begin{quote}
%   \verb|latex \let\install=y\input{pagesel.dtx}|
% \end{quote}
% Do not forget to quote the argument according to the demands
% of your shell.
%
% \paragraph{Generating the documentation.}
% You can use both the \xfile{.dtx} or the \xfile{.drv} to generate
% the documentation. The process can be configured by the
% configuration file \xfile{ltxdoc.cfg}. For instance, put this
% line into this file, if you want to have A4 as paper format:
% \begin{quote}
%   \verb|\PassOptionsToClass{a4paper}{article}|
% \end{quote}
% An example follows how to generate the
% documentation with pdf\LaTeX:
% \begin{quote}
%\begin{verbatim}
%pdflatex pagesel.dtx
%makeindex -s gind.ist pagesel.idx
%pdflatex pagesel.dtx
%makeindex -s gind.ist pagesel.idx
%pdflatex pagesel.dtx
%\end{verbatim}
% \end{quote}
%
% \begin{History}
%   \begin{Version}{1999/03/01 v0.9}
%   \item
%     The first version was built as a response to a question
%     of \NameEmail{Dirk Kuypers}{dk@comnets.rwth-aachen.de},
%     published in the newsgroup
%     \href{news:de.comp.text.tex}{de.comp.text.tex}:\\
%     \URL{``\link{Re: pdflatex nur fuer bestimmte Seiten?!?}''}^^A
%     {https://groups.google.com/group/de.comp.text.tex/msg/6b68c7b3439fb658}
%   \end{Version}
%   \begin{Version}{1999/04/05 v1.0}
%   \item
%     Documentation added in dtx format.
%   \item
%     Copyright: LPPL (\CTAN{macros/latex/base/lppl.txt})
%   \item
%     Options |odd|, |even| added.
%   \item
%     \cmd{\nofiles} added, bug fix for \Package{hyperref}.
%   \item
%     Abort loading of package, if nothing to do.
%   \end{Version}
%   \begin{Version}{1999/04/13 v1.1}
%   \item
%     \cs{nofiles} bug fix removed
%     because of \xpackage{hyperref} 6.55.
%   \item
%     First CTAN release.
%   \end{Version}
%   \begin{Version}{2003/06/05 v1.2}
%   \item
%     \cs{deadcyles} is decremented for omitted pages.
%   \item
%     LPPL 1.2.
%   \end{Version}
%   \begin{Version}{2006/02/20 v1.3}
%   \item
%     Code is not changed.
%   \item
%     New DTX framework.
%   \item
%     LPPL 1.3
%   \end{Version}
%   \begin{Version}{2006/03/02 v1.4}
%   \item
%     Support for \cs{AtBeginDvi} added.
%   \end{Version}
%   \begin{Version}{2006/03/07 v1.5}
%   \item
%     Job is aborted after last selected page.
%   \end{Version}
%   \begin{Version}{2007/04/11 v1.6}
%   \item
%     Line ends sanitized.
%   \end{Version}
%   \begin{Version}{2007/04/12 v1.7}
%   \item
%     Hard coded box number 255 replaced by macro \cs{@cclv}.
%   \end{Version}
%   \begin{Version}{2008/08/11 v1.8}
%   \item
%     Code is not changed.
%   \item
%     URL updated from \texttt{www.dejanews.com}
%     to \texttt{groups.google.com}.
%   \end{Version}
%   \begin{Version}{2016/05/16 v1.9}
%   \item
%     Documentation updates.
%   \end{Version}
%   \begin{Version}{2020-08-03 v1.10}
%   \item Updated to follow the changes in the hook management
%   of LaTeX 2020/10/01
%   \end{Version}
% \end{History}
%
% \PrintIndex
%
% \Finale
\endinput

%        (quote the arguments according to the demands of your shell)
%
% Documentation:
%    (a) If pagesel.drv is present:
%           latex pagesel.drv
%    (b) Without pagesel.drv:
%           latex pagesel.dtx; ...
%    The class ltxdoc loads the configuration file ltxdoc.cfg
%    if available. Here you can specify further options, e.g.
%    use A4 as paper format:
%       \PassOptionsToClass{a4paper}{article}
%
%    Programm calls to get the documentation (example):
%       pdflatex pagesel.dtx
%       makeindex -s gind.ist pagesel.idx
%       pdflatex pagesel.dtx
%       makeindex -s gind.ist pagesel.idx
%       pdflatex pagesel.dtx
%
% Installation:
%    TDS:tex/latex/pagesel/pagesel.sty
%    TDS:doc/latex/pagesel/pagesel.pdf
%    TDS:source/latex/pagesel/pagesel.dtx
%
%<*ignore>
\begingroup
  \catcode123=1 %
  \catcode125=2 %
  \def\x{LaTeX2e}%
\expandafter\endgroup
\ifcase 0\ifx\install y1\fi\expandafter
         \ifx\csname processbatchFile\endcsname\relax\else1\fi
         \ifx\fmtname\x\else 1\fi\relax
\else\csname fi\endcsname
%</ignore>
%<*install>
\input docstrip.tex
\Msg{************************************************************************}
\Msg{* Installation}
\Msg{* Package: pagesel 2020-08-03 v1.10 Select pages of a document for output (HO)}
\Msg{************************************************************************}

\keepsilent
\askforoverwritefalse

\let\MetaPrefix\relax
\preamble

This is a generated file.

Project: pagesel
Version: 2020-08-03 v1.10

Copyright (C)
   1999, 2003, 2006-2008 Heiko Oberdiek
   2016-2020 Oberdiek Package Support Group

This work may be distributed and/or modified under the
conditions of the LaTeX Project Public License, either
version 1.3c of this license or (at your option) any later
version. This version of this license is in
   https://www.latex-project.org/lppl/lppl-1-3c.txt
and the latest version of this license is in
   https://www.latex-project.org/lppl.txt
and version 1.3 or later is part of all distributions of
LaTeX version 2005/12/01 or later.

This work has the LPPL maintenance status "maintained".

The Current Maintainers of this work are
Heiko Oberdiek and the Oberdiek Package Support Group
https://github.com/ho-tex/pagesel/issues


This work consists of the main source file pagesel.dtx
and the derived files
   pagesel.sty, pagesel-2016-05-16.sty, pagesel.pdf,
   pagesel.ins, pagesel.drv.

\endpreamble
\let\MetaPrefix\DoubleperCent

\generate{%
  \file{pagesel.ins}{\from{pagesel.dtx}{install}}%
  \file{pagesel.drv}{\from{pagesel.dtx}{driver}}%
  \usedir{tex/latex/pagesel}%
  \file{pagesel.sty}{\from{pagesel.dtx}{package}}%
  \file{pagesel-2016-05-16.sty}{\from{pagesel.dtx}{packagefrozen}}
}

\catcode32=13\relax% active space
\let =\space%
\Msg{************************************************************************}
\Msg{*}
\Msg{* To finish the installation you have to move the following}
\Msg{* file into a directory searched by TeX:}
\Msg{*}
\Msg{*     pagesel.sty}
\Msg{*}
\Msg{* To produce the documentation run the file `pagesel.drv'}
\Msg{* through LaTeX.}
\Msg{*}
\Msg{* Happy TeXing!}
\Msg{*}
\Msg{************************************************************************}

\endbatchfile
%</install>
%<*ignore>
\fi
%</ignore>
%<*driver>
\NeedsTeXFormat{LaTeX2e}
\ProvidesFile{pagesel.drv}%
  [2020-08-03 v1.10 Select pages of a document for output (HO)]%
\documentclass{ltxdoc}
\usepackage{holtxdoc}[2011/11/22]
\begin{document}
  \DocInput{pagesel.dtx}%
\end{document}
%</driver>
% \fi
%
%
%
% \GetFileInfo{pagesel.drv}
%
% \title{The \xpackage{pagesel} package}
% \date{2020-08-03 v1.10}
% \author{Heiko Oberdiek\thanks
% {Please report any issues at \url{https://github.com/ho-tex/pagesel/issues}}}
%
% \maketitle
%
% \begin{abstract}
% Single pages or page areas can be selected for output.
% \end{abstract}
%
% \tableofcontents
%
% \newenvironment{param}{^^A
%   \newcommand{\entry}[1]{\meta{\###1}:&}^^A
%   \begin{tabular}[t]{@{}l@{ }l@{}}^^A
% }{^^A
%   \end{tabular}^^A
% }
%
% \newcommand*{\Option}[1]{\textsf{#1}}
%
% \section{Usage}
%    The package \Package{pagesel} is a \LaTeXe\ package:
%    \begin{quote}
%      |\usepackage|\oarg{options}|{pagesel}|
%    \end{quote}
%    (For plain\TeX\ and \LaTeX\,2.09 the similar package
%    \URL{\Package{selectp}}^^A
%    {https://ctan.org/pkg/selectp}
%    from \NameEmail{Donald Arsenau}{asnd@triumf.ca} can be used.)
%
%    Depending on the options the package works in two modes:
%    \begin{enumerate}
%    \item If no page selecting option is present, so the package
%          ignores the other options and finishes itself. So no
%          page will be suppressed by the package and auxiliary files
%          will be written.
%    \item With at least one page selecting option the specified
%          pages are selected and the other are suppressed.
%          The default for this mode is that auxiliary will not be
%          overwritten. (This can be changed by an option.)
%    \end{enumerate}
%
% \subsection{Page selecting}
%    The package \Package{pagesel} sets up a new counter that is
%    incremented by each \cmd{\shipout}.
%    In this way the package counts the output pages regardless the value
%    of the page counter. So each page can individually by addressed,
%    even if there are several pages with the same page number.
%
% \subsubsection{Options\texorpdfstring{ for selecting pages}{}}
%    \begin{description}
%    \item[\Option{odd}:] The output pages must have an odd number.
%         All even output pages are suppressed. If there are no
%         page areas specified so all odd pages are print. With
%         page areas only the odd pages in this areas are selected.
%    \item[\Option{even}:] The opposite of option \Option{odd}.
%    \item[Page area:] A page area consists of three elements:
%         the starting output page number, an ``area'' hyphen, and
%         the output page number of the last page in this area.
%         Each component is optional, so there are four kinds
%         to spezify a page area:
%         \begin{description}
%         \item[\meta{m}\Option{-}\meta{n}:] All pages between
%              \meta{m} and \meta{n} inclusive.
%         \item[\Option{-}\meta{n}:] All pages until \meta{n} inclusive.
%         \item[\meta{m}\Option{-}:] The page area starts with \meta{m}
%              and all pages to the end of document are selected.
%         \item[\Option{-}:] All pages (not very useful).
%         \item[\meta{s}:] The single page \meta{s}.
%         \end{description}
%    \end{description}
%
% \subsubsection{Examples}
%    \newcommand*{\exam}[1]{\texttt{\strut[#1]}}^^A hash-ok
%    \begin{tabular}{ll}
%      Options & Output pages\\
%      \hline
%      \exam{1, 4, 9}&  1, 4, and 9\\
%      \exam{7-10, 3}&  3, 7, 8, 9, and 10\\
%      \exam{odd, 3-6}& 3, and 5\\
%      \exam{-4, 3, even, 7-8}& 2, 4, and 8\\
%    \end{tabular}
%
% \subsection{Auxiliary files}
%    If a page is suppressed, the \cmd{\write} commands are not
%    performed. Labels, index entries, or entries for the
%    table of contents aren't written. So it is likely that
%    the table of contents, registers, and lists are incomplete.
% \subsubsection{Options\texorpdfstring{ for handling auxiliary files}{}}
%    \begin{description}
%    \item[\Option{nofiles}:] This is the default. Auxiliary files are
%         read but not written or changed. Also the job is aborted
%         after the last selected page for saving time.
%    \item[\Option{nonofiles}/\Option{files}:] Auxiliary files are
%         written.
%    \end{description}
% \subsubsection{\texorpdfstring{Package }{}\Package{hyperref}}
%    In old versions of \Package{hyperref} [1999/04/12 v6.55] (and below)
%    there is a bug with \cmd{\nofiles}:
%    \begin{itemize}
%    \item Some ``garbage'' appears on terminal and in the log file.
%          This is harmless and can be ignored.
%    \item The outline auxiliary file \cmd{\jobname.out}, however,
%          is opened and truncated to zero bytes.
%          Version 1.0 of this package had
%          loaded a patch file \File{hypnofil.tex}, if it detects
%          \Package{hyperref} to get \cmd{\nofiles} work.
%
%          With the new version of \Package{hyperref} [1999/04/13 v6.56]
%          \cmd{\nofiles} works now. Therefore the workaround code
%          is no longer needed and removed.
%    \end{itemize}
%
% \StopEventually{
% }
%
% \section{Implementation}
%    \begin{macrocode}
%<*package>
%    \end{macrocode}
% \subsection{New implementation using the LaTeX kernel hooks}
%    \begin{macrocode}
\NeedsTeXFormat{LaTeX2e}
\ProvidesPackage{pagesel}
  [2020-08-03 v1.10 Select pages of a document for output (HO)]%
%    \end{macrocode}
%    \begin{macrocode}
\providecommand\IfFormatAtLeastTF{\@ifl@t@r\fmtversion}
\IfFormatAtLeastTF{2020/10/01}{}{%%
%% This is file `pagesel-2016-05-16.sty',
%% generated with the docstrip utility.
%%
%% The original source files were:
%%
%% pagesel.dtx  (with options: `packagefrozen')
%% 
%% This is a generated file.
%% 
%% Project: pagesel
%% Version: 2020-08-03 v1.10
%% 
%% Copyright (C)
%%    1999, 2003, 2006-2008 Heiko Oberdiek
%%    2016-2020 Oberdiek Package Support Group
%% 
%% This work may be distributed and/or modified under the
%% conditions of the LaTeX Project Public License, either
%% version 1.3c of this license or (at your option) any later
%% version. This version of this license is in
%%    https://www.latex-project.org/lppl/lppl-1-3c.txt
%% and the latest version of this license is in
%%    https://www.latex-project.org/lppl.txt
%% and version 1.3 or later is part of all distributions of
%% LaTeX version 2005/12/01 or later.
%% 
%% This work has the LPPL maintenance status "maintained".
%% 
%% The Current Maintainers of this work are
%% Heiko Oberdiek and the Oberdiek Package Support Group
%% https://github.com/ho-tex/pagesel/issues
%% 
%% This work consists of the main source file pagesel.dtx
%% and the derived files
%%    pagesel.sty, pagesel-2016-05-16.sty, pagesel.pdf,
%%    pagesel.ins, pagesel.drv.
%% 
\NeedsTeXFormat{LaTeX2e}
\ProvidesPackage{pagesel}
  [2020-08-03 v1.10 Select pages of a document for output (legacy code) (HO)]%
\@ifundefined{ps@makevoid}{}{%
  \PackageWarningNoLine{pagesel}{Package already loaded.}%
  \endinput
}
\newcommand*{\ps@makevoid}{%
  \global\setbox\@cclv\copy\voidb@x
  \begingroup
    \count@=\deadcycles
    \advance\count@ by -1\relax
    \deadcycles=\count@
  \endgroup
}
\newcommand*\ps@oddpages{0}
\DeclareOption{odd}{\renewcommand*\ps@oddpages{1}}
\DeclareOption{even}{\renewcommand*\ps@oddpages{2}}
\DeclareOption{nofiles}{\let\ps@nofiles\nofiles}
\DeclareOption{nonofiles}{\let\ps@nofiles\@empty}
\DeclareOption{files}{\let\ps@nofiles\@empty}
\ExecuteOptions{nofiles}
\DeclareOption*{%
  \begingroup
    \expandafter\ps@checkoption\CurrentOption-\END
    \edef\x{\endgroup\noexpand\ps@store{\ps@first}{\ps@last}}%
  \x
}
\newcommand\ps@checkoption{}
\def\ps@checkoption#1-#2\END{%
  \ifx\\#2\\%
    \ifx\\#1\\%
      % empty option
      \def\ps@first{\maxdimen}%
      \def\ps@last{\maxdimen}%
    \else
      \edef\ps@first{#1}%
      \edef\ps@last{#1}%
    \fi
  \else
    \ifx\\#1\\%
      \def\ps@first{-\maxdimen}%
    \else
      \edef\ps@first{#1}%
    \fi
    \ps@checklast#2%
  \fi
}
\newcommand\ps@checklast{}
\def\ps@checklast#1-{%
  \ifx\\#1\\%
    \def\ps@last{\maxdimen}%
  \else
    \edef\ps@last{#1}%
  \fi
}
\newcommand*{\ps@store}[2]{%
  \expandafter\def\expandafter\ps@testlist\expandafter{%
    \ps@testlist\ps@pagetest{#1}{#2}%
  }%
}
\newcommand*\ps@testlist{}
\ProcessOptions
\begingroup
  \edef\x{%
    \ifnum\ps@oddpages>0 \relax\fi
    \ifx\ps@testlist\@empty\else\relax\fi
  }%
  \ifx\x\@empty
    \endgroup
    \PackageInfo{pagesel}{Nothing to do}%
    \expandafter\endinput
  \fi
\endgroup
\RequirePackage{everyshi}
\ps@nofiles
\newcounter{ps@count}
\setcounter{ps@count}{0}
\long\def\ps@ReturnAfterElseFi#1\else#2\fi{\fi#1}
\long\def\ps@ReturnAfterFi#1\fi{\fi#1}
\newcommand{\ps@lastpage}{\maxdimen}
\ifx\ps@nofiles\nofiles
  \ifx\ps@testlist\@empty
  \else
    \def\ps@lastpage{0}%
    \newcommand*{\ps@pagetest}[2]{%
      \ifnum#2>\ps@lastpage\relax
        \def\ps@lastpage{#2}%
      \fi
    }%
    \ps@testlist
    \let\ps@pagetest\relax
  \fi
\fi
\newcommand*{\ps@ifinset}[4]{%
  \ifnum#1>\value{ps@count}%
    \ps@ReturnAfterElseFi{#4}%
  \else
    \ps@ReturnAfterFi{%
      \ifnum#2<\value{ps@count}%
        \ps@ReturnAfterElseFi{#4}%
      \else
        \ps@ReturnAfterFi{#3}%
      \fi
    }%
  \fi
}
\newcommand*{\ps@pagetest}[2]{%
  \ps@ifinset{#1}{#2}{\let\ps@next\@empty}{}%
}
\EveryShipout{%
  \stepcounter{ps@count}%
  \ifnum\value{ps@count}>\ps@lastpage\relax
    \global\output{%
      \ps@cleanup@if
      \ps@group@message
      \typeout{%
        Package pagesel Notice: Aborting LaTeX job %
        after last selected page (\ps@lastpage).%
      }%
      \ps@message@ignore
      \global\setbox\@cclv\box\voidb@x
      \deadcycles0\relax
      \aftergroup\@@end
    }%
  \fi
  \let\ps@next\@empty
  \ifx\ps@testlist\@empty
  \else
    \let\ps@next\ps@makevoid
    \ps@testlist
  \fi
  \ifnum\ps@oddpages=1 %
    \ifodd\value{ps@count}%
    \else
    \let\ps@next\ps@makevoid
    \fi
  \fi
  \ifnum\ps@oddpages=2 %
    \ifodd\value{ps@count}%
    \let\ps@next\ps@makevoid
    \else
    \fi
  \fi
  \ps@begindvi
  \ps@next
}
\begingroup\expandafter\expandafter\expandafter\endgroup
\expandafter\ifx\csname currentiflevel\endcsname\relax
  \let\ps@cleanup@if\@empty
\else
  \def\ps@cleanup@if{%
    \ifnum\currentiflevel>\@ne
      \csname fi\endcsname
      \expandafter\ps@cleanup@if
    \fi
  }%
\fi
 \begingroup\expandafter\expandafter\expandafter\endgroup
 \expandafter\ifx\csname currentgrouplevel\endcsname\relax
  \let\ps@group@message\@empty
  \def\ps@message@ignore{%
    \typeout{%
      (pagesel) \space\space\@spaces\@spaces\@spaces
      Messages (\string\end\space occurred ...) can be ignored.%
    }%
  }%
\else
  \def\ps@group@message{%
    \ifnum\currentgrouplevel>\@ne
      \def\ps@message@ignore{%
        \typeout{%
          (pagesel) \space\space\@spaces\@spaces\@spaces
          Message (\string\end\space occurred ...) %
          can be ignored.%
        }%
      }%
    \else
      \let\ps@message@ignore\@empty
    \fi
  }%
\fi
\newbox\ps@begindvibox
\ifvoid\@begindvibox
\else
  \global\setbox\ps@begindvibox\vbox{%
    \unvbox\@begindvibox
  }%
\fi
\let\ps@org@AtBeginDvi\AtBeginDvi
\def\AtBeginDvi#1{%
  \global\setbox\ps@begindvibox\vbox{%
    \unvbox\ps@begindvibox
    #1%
  }%
  \ps@org@AtBeginDvi{#1}%
}
\def\ps@begindvi{%
  \ifx\ps@next\@empty
    \global\let\ps@begindvi\@empty
  \else
    \global\let\ps@begindvi\ps@begindvi@do
  \fi
}
\def\ps@begindvi@do{%
  \ifx\ps@next\@empty
    \setbox\@cclv\vbox{%
      \unvbox\ps@begindvibox
      \box\@cclv
    }%
    \global\let\ps@begindvi\@empty
  \fi
}
\endinput
%%
%% End of file `pagesel-2016-05-16.sty'.
}
\IfFormatAtLeastTF{2020/10/01}{}{\endinput}

%    \end{macrocode}
%    If the package is loaded twice, the package code does not
%    work. So stop loading the package, if it is already loaded.
%    \begin{macrocode}
\@ifundefined{ps@oddpages}{}{%
  \PackageWarningNoLine{pagesel}{Package already loaded.}%
  \endinput
}
%    \end{macrocode}
%    \begin{macrocode}
%</package>
%    \end{macrocode}
% \subsection{Package}
%    \begin{macrocode}
%<*packagefrozen>
\NeedsTeXFormat{LaTeX2e}
\ProvidesPackage{pagesel}
  [2020-08-03 v1.10 Select pages of a document for output (legacy code) (HO)]%
%    \end{macrocode}
%
%    If the package is loaded twice, the package code does not
%    work. So stop loading the package, if it is already loaded.
%    \begin{macrocode}
\@ifundefined{ps@makevoid}{}{%
  \PackageWarningNoLine{pagesel}{Package already loaded.}%
  \endinput
}
%    \end{macrocode}
%
%    \begin{macro}{\ps@makevoid}
%    Macro \cmd{\ps@makevoid} clears the output box. Because
%    nothing is shipped out and this is intended, we reduce
%    the counter \cmd{\deadcycles} in order to avoid problems, if
%    more than \cmd{\maxdeadcycles} pages are omitted.
%    \begin{macrocode}
\newcommand*{\ps@makevoid}{%
  \global\setbox\@cclv\copy\voidb@x
  \begingroup
    \count@=\deadcycles
    \advance\count@ by -1\relax
    \deadcycles=\count@
  \endgroup
}
%</packagefrozen>
%    \end{macrocode}
%    \end{macro}
%
%    \begin{macro}{\ps@oddpages}
%    \begin{macrocode}
%<*package|packagefrozen>
\newcommand*\ps@oddpages{0}
\DeclareOption{odd}{\renewcommand*\ps@oddpages{1}}
\DeclareOption{even}{\renewcommand*\ps@oddpages{2}}
%    \end{macrocode}
%    \end{macro}
%
%    \begin{macrocode}
\DeclareOption{nofiles}{\let\ps@nofiles\nofiles}
\DeclareOption{nonofiles}{\let\ps@nofiles\@empty}
\DeclareOption{files}{\let\ps@nofiles\@empty}
\ExecuteOptions{nofiles}
%    \end{macrocode}
%
%    \begin{macrocode}
\DeclareOption*{%
  \begingroup
    \expandafter\ps@checkoption\CurrentOption-\END
    \edef\x{\endgroup\noexpand\ps@store{\ps@first}{\ps@last}}%
  \x
}
%    \end{macrocode}
%
%    \begin{macro}{\ps@checkoption}
%    \begin{macrocode}
\newcommand\ps@checkoption{}
\def\ps@checkoption#1-#2\END{%
  \ifx\\#2\\%
    \ifx\\#1\\%
      % empty option
      \def\ps@first{\maxdimen}%
      \def\ps@last{\maxdimen}%
    \else
      \edef\ps@first{#1}%
      \edef\ps@last{#1}%
    \fi
  \else
    \ifx\\#1\\%
      \def\ps@first{-\maxdimen}%
    \else
      \edef\ps@first{#1}%
    \fi
    \ps@checklast#2%
  \fi
}
%    \end{macrocode}
%    \end{macro}
%
%    \begin{macro}{\ps@checklast}
%    \begin{macrocode}
\newcommand\ps@checklast{}
\def\ps@checklast#1-{%
  \ifx\\#1\\%
    \def\ps@last{\maxdimen}%
  \else
    \edef\ps@last{#1}%
  \fi
}
%    \end{macrocode}
%    \end{macro}
%
%    \begin{macro}{\ps@store}
%    \begin{macrocode}
\newcommand*{\ps@store}[2]{%
  \expandafter\def\expandafter\ps@testlist\expandafter{%
    \ps@testlist\ps@pagetest{#1}{#2}%
  }%
}
%    \end{macrocode}
%    \end{macro}
%
%    \begin{macro}{\ps@testlist}
%    \begin{macrocode}
\newcommand*\ps@testlist{}
%    \end{macrocode}
%    \end{macro}
%
%    \begin{macrocode}
\ProcessOptions
%    \end{macrocode}
%
%    \begin{macrocode}
\begingroup
  \edef\x{%
    \ifnum\ps@oddpages>0 \relax\fi
    \ifx\ps@testlist\@empty\else\relax\fi
  }%
  \ifx\x\@empty
    \endgroup
    \PackageInfo{pagesel}{Nothing to do}%
    \expandafter\endinput
  \fi
\endgroup
%    \end{macrocode}
%
%    \begin{macrocode}
%</package|packagefrozen>
%<*packagefrozen>
\RequirePackage{everyshi}
%</packagefrozen>
%    \end{macrocode}
%
%    \begin{macrocode}
%<*package|packagefrozen>
\ps@nofiles
%    \end{macrocode}
%
%    \begin{macro}{\c@ps@count}
%    \begin{macrocode}
\newcounter{ps@count}
\setcounter{ps@count}{0}
%    \end{macrocode}
%    \end{macro}
%
%    \begin{macro}{\ps@ReturnAfterElseFi}
%    \begin{macro}{\ps@ReturnAfterFi}
%    \begin{macrocode}
\long\def\ps@ReturnAfterElseFi#1\else#2\fi{\fi#1}
\long\def\ps@ReturnAfterFi#1\fi{\fi#1}
%    \end{macrocode}
%    \end{macro}
%    \end{macro}
%
%    \begin{macrocode}
\newcommand{\ps@lastpage}{\maxdimen}
\ifx\ps@nofiles\nofiles
  \ifx\ps@testlist\@empty
  \else
    \def\ps@lastpage{0}%
    \newcommand*{\ps@pagetest}[2]{%
      \ifnum#2>\ps@lastpage\relax
        \def\ps@lastpage{#2}%
      \fi
    }%
    \ps@testlist
    \let\ps@pagetest\relax
  \fi
\fi
%    \end{macrocode}
%
%    \begin{macro}{\ps@ifinset}
%    \begin{macrocode}
\newcommand*{\ps@ifinset}[4]{%
  \ifnum#1>\value{ps@count}%
    \ps@ReturnAfterElseFi{#4}%
  \else
    \ps@ReturnAfterFi{%
      \ifnum#2<\value{ps@count}%
        \ps@ReturnAfterElseFi{#4}%
      \else
        \ps@ReturnAfterFi{#3}%
      \fi
    }%
  \fi
}
%    \end{macrocode}
%    \end{macro}
%
%    \begin{macro}{\ps@pagetest}
%    \begin{macrocode}
\newcommand*{\ps@pagetest}[2]{%
  \ps@ifinset{#1}{#2}{\let\ps@next\@empty}{}%
}
%    \end{macrocode}
%    \end{macro}
%
%    \begin{macrocode}
%</package|packagefrozen>
%<packagefrozen>\EveryShipout{%
%<package>\AddToHook{shipout/before}{%
%<*package|packagefrozen>
  \stepcounter{ps@count}%
  \ifnum\value{ps@count}>\ps@lastpage\relax
    \global\output{%
      \ps@cleanup@if
      \ps@group@message
      \typeout{%
        Package pagesel Notice: Aborting LaTeX job %
        after last selected page (\ps@lastpage).%
      }%
      \ps@message@ignore
      \global\setbox\@cclv\box\voidb@x
      \deadcycles0\relax
%    \end{macrocode}
%    First leave the output group before ending the job.
%    \begin{macrocode}
      \aftergroup\@@end
    }%
  \fi
  \let\ps@next\@empty
  \ifx\ps@testlist\@empty
  \else
%<packagefrozen>    \let\ps@next\ps@makevoid
%<package>    \let\ps@next\DiscardShipoutBox
    \ps@testlist
  \fi
  \ifnum\ps@oddpages=1 %
    \ifodd\value{ps@count}%
    \else
%<packagefrozen>    \let\ps@next\ps@makevoid
%<package>    \let\ps@next\DiscardShipoutBox
    \fi
  \fi
  \ifnum\ps@oddpages=2 %
    \ifodd\value{ps@count}%
%<packagefrozen>    \let\ps@next\ps@makevoid
%<package>    \let\ps@next\DiscardShipoutBox
    \else
    \fi
  \fi
%<packagefrozen>  \ps@begindvi
  \ps@next
}
%</package|packagefrozen>
%    \end{macrocode}
%
%    \begin{macrocode}
%<*package|packagefrozen>
%<packagefrozen>\begingroup\expandafter\expandafter\expandafter\endgroup
%<packagefrozen>\expandafter\ifx\csname currentiflevel\endcsname\relax
%<packagefrozen>  \let\ps@cleanup@if\@empty
%<packagefrozen>\else
  \def\ps@cleanup@if{%
    \ifnum\currentiflevel>\@ne
      \csname fi\endcsname
      \expandafter\ps@cleanup@if
    \fi
  }%
%<packagefrozen>\fi
%    \end{macrocode}
%    Because of \cs{aftergroup} it is too dangerous to perform
%    a similar cleanup for groups.
%    \begin{macrocode}
%<packagefrozen> \begingroup\expandafter\expandafter\expandafter\endgroup
%<packagefrozen> \expandafter\ifx\csname currentgrouplevel\endcsname\relax
%<packagefrozen>  \let\ps@group@message\@empty
%<packagefrozen>  \def\ps@message@ignore{%
%<packagefrozen>    \typeout{%
%<packagefrozen>      (pagesel) \space\space\@spaces\@spaces\@spaces
%<packagefrozen>      Messages (\string\end\space occurred ...) can be ignored.%
%<packagefrozen>    }%
%<packagefrozen>  }%
%<packagefrozen>\else
  \def\ps@group@message{%
    \ifnum\currentgrouplevel>\@ne
      \def\ps@message@ignore{%
        \typeout{%
          (pagesel) \space\space\@spaces\@spaces\@spaces
          Message (\string\end\space occurred ...) %
          can be ignored.%
        }%
      }%
    \else
      \let\ps@message@ignore\@empty
    \fi
  }%
%<packagefrozen>\fi
%</package|packagefrozen>
%    \end{macrocode}
%
% \subsection{AtBeginDvi hook support}
%
%    The material of box \cs{@begindvibox} is recorded in parallel
%    in box \cs{ps@begindvibox}.
%    \begin{macrocode}
%<*packagefrozen>
\newbox\ps@begindvibox
\ifvoid\@begindvibox
\else
  \global\setbox\ps@begindvibox\vbox{%
    \unvbox\@begindvibox
  }%
\fi
\let\ps@org@AtBeginDvi\AtBeginDvi
\def\AtBeginDvi#1{%
  \global\setbox\ps@begindvibox\vbox{%
    \unvbox\ps@begindvibox
    #1%
  }%
  \ps@org@AtBeginDvi{#1}%
}
%    \end{macrocode}
%
%    \begin{macro}{\ps@begindvi}
%    Macro \cs{ps@begindvi} is called the similar way as \cs{@begindvi}.
%    If the first page is printed, then \cs{AtBeginDvi} should work
%    as usual. Otherwise the contents of box \cs{ps@begindvibox} is
%    set on the first selected page.
%    \begin{macrocode}
\def\ps@begindvi{%
  \ifx\ps@next\@empty
    \global\let\ps@begindvi\@empty
  \else
    \global\let\ps@begindvi\ps@begindvi@do
  \fi
}
\def\ps@begindvi@do{%
  \ifx\ps@next\@empty
    \setbox\@cclv\vbox{%
      \unvbox\ps@begindvibox
      \box\@cclv
    }%
    \global\let\ps@begindvi\@empty
  \fi
}
%    \end{macrocode}
%    \end{macro}
%
%    \begin{macrocode}
%</packagefrozen>
%    \end{macrocode}
%
% \section{Installation}
%
% \subsection{Download}
%
% \paragraph{Package.} This package is available on
% CTAN\footnote{\CTANpkg{pagesel}}:
% \begin{description}
% \item[\CTAN{macros/latex/contrib/pagesel/pagesel.dtx}] The source file.
% \item[\CTAN{macros/latex/contrib/pagesel/pagesel.pdf}] Documentation.
% \end{description}
%
%
%
% \subsection{Package installation}
%
% \paragraph{Unpacking.} The \xfile{.dtx} file is a self-extracting
% \docstrip\ archive. The files are extracted by running the
% \xfile{.dtx} through \plainTeX:
% \begin{quote}
%   \verb|tex pagesel.dtx|
% \end{quote}
%
% \paragraph{TDS.} Now the different files must be moved into
% the different directories in your installation TDS tree
% (also known as \xfile{texmf} tree):
% \begin{quote}
% \def\t{^^A
% \begin{tabular}{@{}>{\ttfamily}l@{ $\rightarrow$ }>{\ttfamily}l@{}}
%   pagesel.sty & tex/latex/pagesel/pagesel.sty\\
%   pagesel.pdf & doc/latex/pagesel/pagesel.pdf\\
%   pagesel.dtx & source/latex/pagesel/pagesel.dtx\\
% \end{tabular}^^A
% }^^A
% \sbox0{\t}^^A
% \ifdim\wd0>\linewidth
%   \begingroup
%     \advance\linewidth by\leftmargin
%     \advance\linewidth by\rightmargin
%   \edef\x{\endgroup
%     \def\noexpand\lw{\the\linewidth}^^A
%   }\x
%   \def\lwbox{^^A
%     \leavevmode
%     \hbox to \linewidth{^^A
%       \kern-\leftmargin\relax
%       \hss
%       \usebox0
%       \hss
%       \kern-\rightmargin\relax
%     }^^A
%   }^^A
%   \ifdim\wd0>\lw
%     \sbox0{\small\t}^^A
%     \ifdim\wd0>\linewidth
%       \ifdim\wd0>\lw
%         \sbox0{\footnotesize\t}^^A
%         \ifdim\wd0>\linewidth
%           \ifdim\wd0>\lw
%             \sbox0{\scriptsize\t}^^A
%             \ifdim\wd0>\linewidth
%               \ifdim\wd0>\lw
%                 \sbox0{\tiny\t}^^A
%                 \ifdim\wd0>\linewidth
%                   \lwbox
%                 \else
%                   \usebox0
%                 \fi
%               \else
%                 \lwbox
%               \fi
%             \else
%               \usebox0
%             \fi
%           \else
%             \lwbox
%           \fi
%         \else
%           \usebox0
%         \fi
%       \else
%         \lwbox
%       \fi
%     \else
%       \usebox0
%     \fi
%   \else
%     \lwbox
%   \fi
% \else
%   \usebox0
% \fi
% \end{quote}
% If you have a \xfile{docstrip.cfg} that configures and enables \docstrip's
% TDS installing feature, then some files can already be in the right
% place, see the documentation of \docstrip.
%
% \subsection{Refresh file name databases}
%
% If your \TeX~distribution
% (\TeX\,Live, \mikTeX, \dots) relies on file name databases, you must refresh
% these. For example, \TeX\,Live\ users run \verb|texhash| or
% \verb|mktexlsr|.
%
% \subsection{Some details for the interested}
%
% \paragraph{Unpacking with \LaTeX.}
% The \xfile{.dtx} chooses its action depending on the format:
% \begin{description}
% \item[\plainTeX:] Run \docstrip\ and extract the files.
% \item[\LaTeX:] Generate the documentation.
% \end{description}
% If you insist on using \LaTeX\ for \docstrip\ (really,
% \docstrip\ does not need \LaTeX), then inform the autodetect routine
% about your intention:
% \begin{quote}
%   \verb|latex \let\install=y% \iffalse meta-comment
%
% File: pagesel.dtx
% Version: 2020-08-03 v1.10
% Info: Select pages of a document for output
%
% Copyright (C)
%    1999, 2003, 2006-2008 Heiko Oberdiek
%    2016-2020 Oberdiek Package Support Group
%    https://github.com/ho-tex/pagesel/issues
%
% This work may be distributed and/or modified under the
% conditions of the LaTeX Project Public License, either
% version 1.3c of this license or (at your option) any later
% version. This version of this license is in
%    https://www.latex-project.org/lppl/lppl-1-3c.txt
% and the latest version of this license is in
%    https://www.latex-project.org/lppl.txt
% and version 1.3 or later is part of all distributions of
% LaTeX version 2005/12/01 or later.
%
% This work has the LPPL maintenance status "maintained".
%
% The Current Maintainers of this work are
% Heiko Oberdiek and the Oberdiek Package Support Group
% https://github.com/ho-tex/pagesel/issues
%
% This work consists of the main source file pagesel.dtx
% and the derived files
%    pagesel.sty, pagesel-2016-05-16.sty,
%    pagesel.pdf, pagesel.ins, pagesel.drv.
%
% Distribution:
%    CTAN:macros/latex/contrib/pagesel/pagesel.dtx
%    CTAN:macros/latex/contrib/pagesel/pagesel.pdf
%
% Unpacking:
%    (a) If pagesel.ins is present:
%           tex pagesel.ins
%    (b) Without pagesel.ins:
%           tex pagesel.dtx
%    (c) If you insist on using LaTeX
%           latex \let\install=y\input{pagesel.dtx}
%        (quote the arguments according to the demands of your shell)
%
% Documentation:
%    (a) If pagesel.drv is present:
%           latex pagesel.drv
%    (b) Without pagesel.drv:
%           latex pagesel.dtx; ...
%    The class ltxdoc loads the configuration file ltxdoc.cfg
%    if available. Here you can specify further options, e.g.
%    use A4 as paper format:
%       \PassOptionsToClass{a4paper}{article}
%
%    Programm calls to get the documentation (example):
%       pdflatex pagesel.dtx
%       makeindex -s gind.ist pagesel.idx
%       pdflatex pagesel.dtx
%       makeindex -s gind.ist pagesel.idx
%       pdflatex pagesel.dtx
%
% Installation:
%    TDS:tex/latex/pagesel/pagesel.sty
%    TDS:doc/latex/pagesel/pagesel.pdf
%    TDS:source/latex/pagesel/pagesel.dtx
%
%<*ignore>
\begingroup
  \catcode123=1 %
  \catcode125=2 %
  \def\x{LaTeX2e}%
\expandafter\endgroup
\ifcase 0\ifx\install y1\fi\expandafter
         \ifx\csname processbatchFile\endcsname\relax\else1\fi
         \ifx\fmtname\x\else 1\fi\relax
\else\csname fi\endcsname
%</ignore>
%<*install>
\input docstrip.tex
\Msg{************************************************************************}
\Msg{* Installation}
\Msg{* Package: pagesel 2020-08-03 v1.10 Select pages of a document for output (HO)}
\Msg{************************************************************************}

\keepsilent
\askforoverwritefalse

\let\MetaPrefix\relax
\preamble

This is a generated file.

Project: pagesel
Version: 2020-08-03 v1.10

Copyright (C)
   1999, 2003, 2006-2008 Heiko Oberdiek
   2016-2020 Oberdiek Package Support Group

This work may be distributed and/or modified under the
conditions of the LaTeX Project Public License, either
version 1.3c of this license or (at your option) any later
version. This version of this license is in
   https://www.latex-project.org/lppl/lppl-1-3c.txt
and the latest version of this license is in
   https://www.latex-project.org/lppl.txt
and version 1.3 or later is part of all distributions of
LaTeX version 2005/12/01 or later.

This work has the LPPL maintenance status "maintained".

The Current Maintainers of this work are
Heiko Oberdiek and the Oberdiek Package Support Group
https://github.com/ho-tex/pagesel/issues


This work consists of the main source file pagesel.dtx
and the derived files
   pagesel.sty, pagesel-2016-05-16.sty, pagesel.pdf,
   pagesel.ins, pagesel.drv.

\endpreamble
\let\MetaPrefix\DoubleperCent

\generate{%
  \file{pagesel.ins}{\from{pagesel.dtx}{install}}%
  \file{pagesel.drv}{\from{pagesel.dtx}{driver}}%
  \usedir{tex/latex/pagesel}%
  \file{pagesel.sty}{\from{pagesel.dtx}{package}}%
  \file{pagesel-2016-05-16.sty}{\from{pagesel.dtx}{packagefrozen}}
}

\catcode32=13\relax% active space
\let =\space%
\Msg{************************************************************************}
\Msg{*}
\Msg{* To finish the installation you have to move the following}
\Msg{* file into a directory searched by TeX:}
\Msg{*}
\Msg{*     pagesel.sty}
\Msg{*}
\Msg{* To produce the documentation run the file `pagesel.drv'}
\Msg{* through LaTeX.}
\Msg{*}
\Msg{* Happy TeXing!}
\Msg{*}
\Msg{************************************************************************}

\endbatchfile
%</install>
%<*ignore>
\fi
%</ignore>
%<*driver>
\NeedsTeXFormat{LaTeX2e}
\ProvidesFile{pagesel.drv}%
  [2020-08-03 v1.10 Select pages of a document for output (HO)]%
\documentclass{ltxdoc}
\usepackage{holtxdoc}[2011/11/22]
\begin{document}
  \DocInput{pagesel.dtx}%
\end{document}
%</driver>
% \fi
%
%
%
% \GetFileInfo{pagesel.drv}
%
% \title{The \xpackage{pagesel} package}
% \date{2020-08-03 v1.10}
% \author{Heiko Oberdiek\thanks
% {Please report any issues at \url{https://github.com/ho-tex/pagesel/issues}}}
%
% \maketitle
%
% \begin{abstract}
% Single pages or page areas can be selected for output.
% \end{abstract}
%
% \tableofcontents
%
% \newenvironment{param}{^^A
%   \newcommand{\entry}[1]{\meta{\###1}:&}^^A
%   \begin{tabular}[t]{@{}l@{ }l@{}}^^A
% }{^^A
%   \end{tabular}^^A
% }
%
% \newcommand*{\Option}[1]{\textsf{#1}}
%
% \section{Usage}
%    The package \Package{pagesel} is a \LaTeXe\ package:
%    \begin{quote}
%      |\usepackage|\oarg{options}|{pagesel}|
%    \end{quote}
%    (For plain\TeX\ and \LaTeX\,2.09 the similar package
%    \URL{\Package{selectp}}^^A
%    {https://ctan.org/pkg/selectp}
%    from \NameEmail{Donald Arsenau}{asnd@triumf.ca} can be used.)
%
%    Depending on the options the package works in two modes:
%    \begin{enumerate}
%    \item If no page selecting option is present, so the package
%          ignores the other options and finishes itself. So no
%          page will be suppressed by the package and auxiliary files
%          will be written.
%    \item With at least one page selecting option the specified
%          pages are selected and the other are suppressed.
%          The default for this mode is that auxiliary will not be
%          overwritten. (This can be changed by an option.)
%    \end{enumerate}
%
% \subsection{Page selecting}
%    The package \Package{pagesel} sets up a new counter that is
%    incremented by each \cmd{\shipout}.
%    In this way the package counts the output pages regardless the value
%    of the page counter. So each page can individually by addressed,
%    even if there are several pages with the same page number.
%
% \subsubsection{Options\texorpdfstring{ for selecting pages}{}}
%    \begin{description}
%    \item[\Option{odd}:] The output pages must have an odd number.
%         All even output pages are suppressed. If there are no
%         page areas specified so all odd pages are print. With
%         page areas only the odd pages in this areas are selected.
%    \item[\Option{even}:] The opposite of option \Option{odd}.
%    \item[Page area:] A page area consists of three elements:
%         the starting output page number, an ``area'' hyphen, and
%         the output page number of the last page in this area.
%         Each component is optional, so there are four kinds
%         to spezify a page area:
%         \begin{description}
%         \item[\meta{m}\Option{-}\meta{n}:] All pages between
%              \meta{m} and \meta{n} inclusive.
%         \item[\Option{-}\meta{n}:] All pages until \meta{n} inclusive.
%         \item[\meta{m}\Option{-}:] The page area starts with \meta{m}
%              and all pages to the end of document are selected.
%         \item[\Option{-}:] All pages (not very useful).
%         \item[\meta{s}:] The single page \meta{s}.
%         \end{description}
%    \end{description}
%
% \subsubsection{Examples}
%    \newcommand*{\exam}[1]{\texttt{\strut[#1]}}^^A hash-ok
%    \begin{tabular}{ll}
%      Options & Output pages\\
%      \hline
%      \exam{1, 4, 9}&  1, 4, and 9\\
%      \exam{7-10, 3}&  3, 7, 8, 9, and 10\\
%      \exam{odd, 3-6}& 3, and 5\\
%      \exam{-4, 3, even, 7-8}& 2, 4, and 8\\
%    \end{tabular}
%
% \subsection{Auxiliary files}
%    If a page is suppressed, the \cmd{\write} commands are not
%    performed. Labels, index entries, or entries for the
%    table of contents aren't written. So it is likely that
%    the table of contents, registers, and lists are incomplete.
% \subsubsection{Options\texorpdfstring{ for handling auxiliary files}{}}
%    \begin{description}
%    \item[\Option{nofiles}:] This is the default. Auxiliary files are
%         read but not written or changed. Also the job is aborted
%         after the last selected page for saving time.
%    \item[\Option{nonofiles}/\Option{files}:] Auxiliary files are
%         written.
%    \end{description}
% \subsubsection{\texorpdfstring{Package }{}\Package{hyperref}}
%    In old versions of \Package{hyperref} [1999/04/12 v6.55] (and below)
%    there is a bug with \cmd{\nofiles}:
%    \begin{itemize}
%    \item Some ``garbage'' appears on terminal and in the log file.
%          This is harmless and can be ignored.
%    \item The outline auxiliary file \cmd{\jobname.out}, however,
%          is opened and truncated to zero bytes.
%          Version 1.0 of this package had
%          loaded a patch file \File{hypnofil.tex}, if it detects
%          \Package{hyperref} to get \cmd{\nofiles} work.
%
%          With the new version of \Package{hyperref} [1999/04/13 v6.56]
%          \cmd{\nofiles} works now. Therefore the workaround code
%          is no longer needed and removed.
%    \end{itemize}
%
% \StopEventually{
% }
%
% \section{Implementation}
%    \begin{macrocode}
%<*package>
%    \end{macrocode}
% \subsection{New implementation using the LaTeX kernel hooks}
%    \begin{macrocode}
\NeedsTeXFormat{LaTeX2e}
\ProvidesPackage{pagesel}
  [2020-08-03 v1.10 Select pages of a document for output (HO)]%
%    \end{macrocode}
%    \begin{macrocode}
\providecommand\IfFormatAtLeastTF{\@ifl@t@r\fmtversion}
\IfFormatAtLeastTF{2020/10/01}{}{\input{pagesel-2016-05-16.sty}}
\IfFormatAtLeastTF{2020/10/01}{}{\endinput}

%    \end{macrocode}
%    If the package is loaded twice, the package code does not
%    work. So stop loading the package, if it is already loaded.
%    \begin{macrocode}
\@ifundefined{ps@oddpages}{}{%
  \PackageWarningNoLine{pagesel}{Package already loaded.}%
  \endinput
}
%    \end{macrocode}
%    \begin{macrocode}
%</package>
%    \end{macrocode}
% \subsection{Package}
%    \begin{macrocode}
%<*packagefrozen>
\NeedsTeXFormat{LaTeX2e}
\ProvidesPackage{pagesel}
  [2020-08-03 v1.10 Select pages of a document for output (legacy code) (HO)]%
%    \end{macrocode}
%
%    If the package is loaded twice, the package code does not
%    work. So stop loading the package, if it is already loaded.
%    \begin{macrocode}
\@ifundefined{ps@makevoid}{}{%
  \PackageWarningNoLine{pagesel}{Package already loaded.}%
  \endinput
}
%    \end{macrocode}
%
%    \begin{macro}{\ps@makevoid}
%    Macro \cmd{\ps@makevoid} clears the output box. Because
%    nothing is shipped out and this is intended, we reduce
%    the counter \cmd{\deadcycles} in order to avoid problems, if
%    more than \cmd{\maxdeadcycles} pages are omitted.
%    \begin{macrocode}
\newcommand*{\ps@makevoid}{%
  \global\setbox\@cclv\copy\voidb@x
  \begingroup
    \count@=\deadcycles
    \advance\count@ by -1\relax
    \deadcycles=\count@
  \endgroup
}
%</packagefrozen>
%    \end{macrocode}
%    \end{macro}
%
%    \begin{macro}{\ps@oddpages}
%    \begin{macrocode}
%<*package|packagefrozen>
\newcommand*\ps@oddpages{0}
\DeclareOption{odd}{\renewcommand*\ps@oddpages{1}}
\DeclareOption{even}{\renewcommand*\ps@oddpages{2}}
%    \end{macrocode}
%    \end{macro}
%
%    \begin{macrocode}
\DeclareOption{nofiles}{\let\ps@nofiles\nofiles}
\DeclareOption{nonofiles}{\let\ps@nofiles\@empty}
\DeclareOption{files}{\let\ps@nofiles\@empty}
\ExecuteOptions{nofiles}
%    \end{macrocode}
%
%    \begin{macrocode}
\DeclareOption*{%
  \begingroup
    \expandafter\ps@checkoption\CurrentOption-\END
    \edef\x{\endgroup\noexpand\ps@store{\ps@first}{\ps@last}}%
  \x
}
%    \end{macrocode}
%
%    \begin{macro}{\ps@checkoption}
%    \begin{macrocode}
\newcommand\ps@checkoption{}
\def\ps@checkoption#1-#2\END{%
  \ifx\\#2\\%
    \ifx\\#1\\%
      % empty option
      \def\ps@first{\maxdimen}%
      \def\ps@last{\maxdimen}%
    \else
      \edef\ps@first{#1}%
      \edef\ps@last{#1}%
    \fi
  \else
    \ifx\\#1\\%
      \def\ps@first{-\maxdimen}%
    \else
      \edef\ps@first{#1}%
    \fi
    \ps@checklast#2%
  \fi
}
%    \end{macrocode}
%    \end{macro}
%
%    \begin{macro}{\ps@checklast}
%    \begin{macrocode}
\newcommand\ps@checklast{}
\def\ps@checklast#1-{%
  \ifx\\#1\\%
    \def\ps@last{\maxdimen}%
  \else
    \edef\ps@last{#1}%
  \fi
}
%    \end{macrocode}
%    \end{macro}
%
%    \begin{macro}{\ps@store}
%    \begin{macrocode}
\newcommand*{\ps@store}[2]{%
  \expandafter\def\expandafter\ps@testlist\expandafter{%
    \ps@testlist\ps@pagetest{#1}{#2}%
  }%
}
%    \end{macrocode}
%    \end{macro}
%
%    \begin{macro}{\ps@testlist}
%    \begin{macrocode}
\newcommand*\ps@testlist{}
%    \end{macrocode}
%    \end{macro}
%
%    \begin{macrocode}
\ProcessOptions
%    \end{macrocode}
%
%    \begin{macrocode}
\begingroup
  \edef\x{%
    \ifnum\ps@oddpages>0 \relax\fi
    \ifx\ps@testlist\@empty\else\relax\fi
  }%
  \ifx\x\@empty
    \endgroup
    \PackageInfo{pagesel}{Nothing to do}%
    \expandafter\endinput
  \fi
\endgroup
%    \end{macrocode}
%
%    \begin{macrocode}
%</package|packagefrozen>
%<*packagefrozen>
\RequirePackage{everyshi}
%</packagefrozen>
%    \end{macrocode}
%
%    \begin{macrocode}
%<*package|packagefrozen>
\ps@nofiles
%    \end{macrocode}
%
%    \begin{macro}{\c@ps@count}
%    \begin{macrocode}
\newcounter{ps@count}
\setcounter{ps@count}{0}
%    \end{macrocode}
%    \end{macro}
%
%    \begin{macro}{\ps@ReturnAfterElseFi}
%    \begin{macro}{\ps@ReturnAfterFi}
%    \begin{macrocode}
\long\def\ps@ReturnAfterElseFi#1\else#2\fi{\fi#1}
\long\def\ps@ReturnAfterFi#1\fi{\fi#1}
%    \end{macrocode}
%    \end{macro}
%    \end{macro}
%
%    \begin{macrocode}
\newcommand{\ps@lastpage}{\maxdimen}
\ifx\ps@nofiles\nofiles
  \ifx\ps@testlist\@empty
  \else
    \def\ps@lastpage{0}%
    \newcommand*{\ps@pagetest}[2]{%
      \ifnum#2>\ps@lastpage\relax
        \def\ps@lastpage{#2}%
      \fi
    }%
    \ps@testlist
    \let\ps@pagetest\relax
  \fi
\fi
%    \end{macrocode}
%
%    \begin{macro}{\ps@ifinset}
%    \begin{macrocode}
\newcommand*{\ps@ifinset}[4]{%
  \ifnum#1>\value{ps@count}%
    \ps@ReturnAfterElseFi{#4}%
  \else
    \ps@ReturnAfterFi{%
      \ifnum#2<\value{ps@count}%
        \ps@ReturnAfterElseFi{#4}%
      \else
        \ps@ReturnAfterFi{#3}%
      \fi
    }%
  \fi
}
%    \end{macrocode}
%    \end{macro}
%
%    \begin{macro}{\ps@pagetest}
%    \begin{macrocode}
\newcommand*{\ps@pagetest}[2]{%
  \ps@ifinset{#1}{#2}{\let\ps@next\@empty}{}%
}
%    \end{macrocode}
%    \end{macro}
%
%    \begin{macrocode}
%</package|packagefrozen>
%<packagefrozen>\EveryShipout{%
%<package>\AddToHook{shipout/before}{%
%<*package|packagefrozen>
  \stepcounter{ps@count}%
  \ifnum\value{ps@count}>\ps@lastpage\relax
    \global\output{%
      \ps@cleanup@if
      \ps@group@message
      \typeout{%
        Package pagesel Notice: Aborting LaTeX job %
        after last selected page (\ps@lastpage).%
      }%
      \ps@message@ignore
      \global\setbox\@cclv\box\voidb@x
      \deadcycles0\relax
%    \end{macrocode}
%    First leave the output group before ending the job.
%    \begin{macrocode}
      \aftergroup\@@end
    }%
  \fi
  \let\ps@next\@empty
  \ifx\ps@testlist\@empty
  \else
%<packagefrozen>    \let\ps@next\ps@makevoid
%<package>    \let\ps@next\DiscardShipoutBox
    \ps@testlist
  \fi
  \ifnum\ps@oddpages=1 %
    \ifodd\value{ps@count}%
    \else
%<packagefrozen>    \let\ps@next\ps@makevoid
%<package>    \let\ps@next\DiscardShipoutBox
    \fi
  \fi
  \ifnum\ps@oddpages=2 %
    \ifodd\value{ps@count}%
%<packagefrozen>    \let\ps@next\ps@makevoid
%<package>    \let\ps@next\DiscardShipoutBox
    \else
    \fi
  \fi
%<packagefrozen>  \ps@begindvi
  \ps@next
}
%</package|packagefrozen>
%    \end{macrocode}
%
%    \begin{macrocode}
%<*package|packagefrozen>
%<packagefrozen>\begingroup\expandafter\expandafter\expandafter\endgroup
%<packagefrozen>\expandafter\ifx\csname currentiflevel\endcsname\relax
%<packagefrozen>  \let\ps@cleanup@if\@empty
%<packagefrozen>\else
  \def\ps@cleanup@if{%
    \ifnum\currentiflevel>\@ne
      \csname fi\endcsname
      \expandafter\ps@cleanup@if
    \fi
  }%
%<packagefrozen>\fi
%    \end{macrocode}
%    Because of \cs{aftergroup} it is too dangerous to perform
%    a similar cleanup for groups.
%    \begin{macrocode}
%<packagefrozen> \begingroup\expandafter\expandafter\expandafter\endgroup
%<packagefrozen> \expandafter\ifx\csname currentgrouplevel\endcsname\relax
%<packagefrozen>  \let\ps@group@message\@empty
%<packagefrozen>  \def\ps@message@ignore{%
%<packagefrozen>    \typeout{%
%<packagefrozen>      (pagesel) \space\space\@spaces\@spaces\@spaces
%<packagefrozen>      Messages (\string\end\space occurred ...) can be ignored.%
%<packagefrozen>    }%
%<packagefrozen>  }%
%<packagefrozen>\else
  \def\ps@group@message{%
    \ifnum\currentgrouplevel>\@ne
      \def\ps@message@ignore{%
        \typeout{%
          (pagesel) \space\space\@spaces\@spaces\@spaces
          Message (\string\end\space occurred ...) %
          can be ignored.%
        }%
      }%
    \else
      \let\ps@message@ignore\@empty
    \fi
  }%
%<packagefrozen>\fi
%</package|packagefrozen>
%    \end{macrocode}
%
% \subsection{AtBeginDvi hook support}
%
%    The material of box \cs{@begindvibox} is recorded in parallel
%    in box \cs{ps@begindvibox}.
%    \begin{macrocode}
%<*packagefrozen>
\newbox\ps@begindvibox
\ifvoid\@begindvibox
\else
  \global\setbox\ps@begindvibox\vbox{%
    \unvbox\@begindvibox
  }%
\fi
\let\ps@org@AtBeginDvi\AtBeginDvi
\def\AtBeginDvi#1{%
  \global\setbox\ps@begindvibox\vbox{%
    \unvbox\ps@begindvibox
    #1%
  }%
  \ps@org@AtBeginDvi{#1}%
}
%    \end{macrocode}
%
%    \begin{macro}{\ps@begindvi}
%    Macro \cs{ps@begindvi} is called the similar way as \cs{@begindvi}.
%    If the first page is printed, then \cs{AtBeginDvi} should work
%    as usual. Otherwise the contents of box \cs{ps@begindvibox} is
%    set on the first selected page.
%    \begin{macrocode}
\def\ps@begindvi{%
  \ifx\ps@next\@empty
    \global\let\ps@begindvi\@empty
  \else
    \global\let\ps@begindvi\ps@begindvi@do
  \fi
}
\def\ps@begindvi@do{%
  \ifx\ps@next\@empty
    \setbox\@cclv\vbox{%
      \unvbox\ps@begindvibox
      \box\@cclv
    }%
    \global\let\ps@begindvi\@empty
  \fi
}
%    \end{macrocode}
%    \end{macro}
%
%    \begin{macrocode}
%</packagefrozen>
%    \end{macrocode}
%
% \section{Installation}
%
% \subsection{Download}
%
% \paragraph{Package.} This package is available on
% CTAN\footnote{\CTANpkg{pagesel}}:
% \begin{description}
% \item[\CTAN{macros/latex/contrib/pagesel/pagesel.dtx}] The source file.
% \item[\CTAN{macros/latex/contrib/pagesel/pagesel.pdf}] Documentation.
% \end{description}
%
%
%
% \subsection{Package installation}
%
% \paragraph{Unpacking.} The \xfile{.dtx} file is a self-extracting
% \docstrip\ archive. The files are extracted by running the
% \xfile{.dtx} through \plainTeX:
% \begin{quote}
%   \verb|tex pagesel.dtx|
% \end{quote}
%
% \paragraph{TDS.} Now the different files must be moved into
% the different directories in your installation TDS tree
% (also known as \xfile{texmf} tree):
% \begin{quote}
% \def\t{^^A
% \begin{tabular}{@{}>{\ttfamily}l@{ $\rightarrow$ }>{\ttfamily}l@{}}
%   pagesel.sty & tex/latex/pagesel/pagesel.sty\\
%   pagesel.pdf & doc/latex/pagesel/pagesel.pdf\\
%   pagesel.dtx & source/latex/pagesel/pagesel.dtx\\
% \end{tabular}^^A
% }^^A
% \sbox0{\t}^^A
% \ifdim\wd0>\linewidth
%   \begingroup
%     \advance\linewidth by\leftmargin
%     \advance\linewidth by\rightmargin
%   \edef\x{\endgroup
%     \def\noexpand\lw{\the\linewidth}^^A
%   }\x
%   \def\lwbox{^^A
%     \leavevmode
%     \hbox to \linewidth{^^A
%       \kern-\leftmargin\relax
%       \hss
%       \usebox0
%       \hss
%       \kern-\rightmargin\relax
%     }^^A
%   }^^A
%   \ifdim\wd0>\lw
%     \sbox0{\small\t}^^A
%     \ifdim\wd0>\linewidth
%       \ifdim\wd0>\lw
%         \sbox0{\footnotesize\t}^^A
%         \ifdim\wd0>\linewidth
%           \ifdim\wd0>\lw
%             \sbox0{\scriptsize\t}^^A
%             \ifdim\wd0>\linewidth
%               \ifdim\wd0>\lw
%                 \sbox0{\tiny\t}^^A
%                 \ifdim\wd0>\linewidth
%                   \lwbox
%                 \else
%                   \usebox0
%                 \fi
%               \else
%                 \lwbox
%               \fi
%             \else
%               \usebox0
%             \fi
%           \else
%             \lwbox
%           \fi
%         \else
%           \usebox0
%         \fi
%       \else
%         \lwbox
%       \fi
%     \else
%       \usebox0
%     \fi
%   \else
%     \lwbox
%   \fi
% \else
%   \usebox0
% \fi
% \end{quote}
% If you have a \xfile{docstrip.cfg} that configures and enables \docstrip's
% TDS installing feature, then some files can already be in the right
% place, see the documentation of \docstrip.
%
% \subsection{Refresh file name databases}
%
% If your \TeX~distribution
% (\TeX\,Live, \mikTeX, \dots) relies on file name databases, you must refresh
% these. For example, \TeX\,Live\ users run \verb|texhash| or
% \verb|mktexlsr|.
%
% \subsection{Some details for the interested}
%
% \paragraph{Unpacking with \LaTeX.}
% The \xfile{.dtx} chooses its action depending on the format:
% \begin{description}
% \item[\plainTeX:] Run \docstrip\ and extract the files.
% \item[\LaTeX:] Generate the documentation.
% \end{description}
% If you insist on using \LaTeX\ for \docstrip\ (really,
% \docstrip\ does not need \LaTeX), then inform the autodetect routine
% about your intention:
% \begin{quote}
%   \verb|latex \let\install=y\input{pagesel.dtx}|
% \end{quote}
% Do not forget to quote the argument according to the demands
% of your shell.
%
% \paragraph{Generating the documentation.}
% You can use both the \xfile{.dtx} or the \xfile{.drv} to generate
% the documentation. The process can be configured by the
% configuration file \xfile{ltxdoc.cfg}. For instance, put this
% line into this file, if you want to have A4 as paper format:
% \begin{quote}
%   \verb|\PassOptionsToClass{a4paper}{article}|
% \end{quote}
% An example follows how to generate the
% documentation with pdf\LaTeX:
% \begin{quote}
%\begin{verbatim}
%pdflatex pagesel.dtx
%makeindex -s gind.ist pagesel.idx
%pdflatex pagesel.dtx
%makeindex -s gind.ist pagesel.idx
%pdflatex pagesel.dtx
%\end{verbatim}
% \end{quote}
%
% \begin{History}
%   \begin{Version}{1999/03/01 v0.9}
%   \item
%     The first version was built as a response to a question
%     of \NameEmail{Dirk Kuypers}{dk@comnets.rwth-aachen.de},
%     published in the newsgroup
%     \href{news:de.comp.text.tex}{de.comp.text.tex}:\\
%     \URL{``\link{Re: pdflatex nur fuer bestimmte Seiten?!?}''}^^A
%     {https://groups.google.com/group/de.comp.text.tex/msg/6b68c7b3439fb658}
%   \end{Version}
%   \begin{Version}{1999/04/05 v1.0}
%   \item
%     Documentation added in dtx format.
%   \item
%     Copyright: LPPL (\CTAN{macros/latex/base/lppl.txt})
%   \item
%     Options |odd|, |even| added.
%   \item
%     \cmd{\nofiles} added, bug fix for \Package{hyperref}.
%   \item
%     Abort loading of package, if nothing to do.
%   \end{Version}
%   \begin{Version}{1999/04/13 v1.1}
%   \item
%     \cs{nofiles} bug fix removed
%     because of \xpackage{hyperref} 6.55.
%   \item
%     First CTAN release.
%   \end{Version}
%   \begin{Version}{2003/06/05 v1.2}
%   \item
%     \cs{deadcyles} is decremented for omitted pages.
%   \item
%     LPPL 1.2.
%   \end{Version}
%   \begin{Version}{2006/02/20 v1.3}
%   \item
%     Code is not changed.
%   \item
%     New DTX framework.
%   \item
%     LPPL 1.3
%   \end{Version}
%   \begin{Version}{2006/03/02 v1.4}
%   \item
%     Support for \cs{AtBeginDvi} added.
%   \end{Version}
%   \begin{Version}{2006/03/07 v1.5}
%   \item
%     Job is aborted after last selected page.
%   \end{Version}
%   \begin{Version}{2007/04/11 v1.6}
%   \item
%     Line ends sanitized.
%   \end{Version}
%   \begin{Version}{2007/04/12 v1.7}
%   \item
%     Hard coded box number 255 replaced by macro \cs{@cclv}.
%   \end{Version}
%   \begin{Version}{2008/08/11 v1.8}
%   \item
%     Code is not changed.
%   \item
%     URL updated from \texttt{www.dejanews.com}
%     to \texttt{groups.google.com}.
%   \end{Version}
%   \begin{Version}{2016/05/16 v1.9}
%   \item
%     Documentation updates.
%   \end{Version}
%   \begin{Version}{2020-08-03 v1.10}
%   \item Updated to follow the changes in the hook management
%   of LaTeX 2020/10/01
%   \end{Version}
% \end{History}
%
% \PrintIndex
%
% \Finale
\endinput
|
% \end{quote}
% Do not forget to quote the argument according to the demands
% of your shell.
%
% \paragraph{Generating the documentation.}
% You can use both the \xfile{.dtx} or the \xfile{.drv} to generate
% the documentation. The process can be configured by the
% configuration file \xfile{ltxdoc.cfg}. For instance, put this
% line into this file, if you want to have A4 as paper format:
% \begin{quote}
%   \verb|\PassOptionsToClass{a4paper}{article}|
% \end{quote}
% An example follows how to generate the
% documentation with pdf\LaTeX:
% \begin{quote}
%\begin{verbatim}
%pdflatex pagesel.dtx
%makeindex -s gind.ist pagesel.idx
%pdflatex pagesel.dtx
%makeindex -s gind.ist pagesel.idx
%pdflatex pagesel.dtx
%\end{verbatim}
% \end{quote}
%
% \begin{History}
%   \begin{Version}{1999/03/01 v0.9}
%   \item
%     The first version was built as a response to a question
%     of \NameEmail{Dirk Kuypers}{dk@comnets.rwth-aachen.de},
%     published in the newsgroup
%     \href{news:de.comp.text.tex}{de.comp.text.tex}:\\
%     \URL{``\link{Re: pdflatex nur fuer bestimmte Seiten?!?}''}^^A
%     {https://groups.google.com/group/de.comp.text.tex/msg/6b68c7b3439fb658}
%   \end{Version}
%   \begin{Version}{1999/04/05 v1.0}
%   \item
%     Documentation added in dtx format.
%   \item
%     Copyright: LPPL (\CTAN{macros/latex/base/lppl.txt})
%   \item
%     Options |odd|, |even| added.
%   \item
%     \cmd{\nofiles} added, bug fix for \Package{hyperref}.
%   \item
%     Abort loading of package, if nothing to do.
%   \end{Version}
%   \begin{Version}{1999/04/13 v1.1}
%   \item
%     \cs{nofiles} bug fix removed
%     because of \xpackage{hyperref} 6.55.
%   \item
%     First CTAN release.
%   \end{Version}
%   \begin{Version}{2003/06/05 v1.2}
%   \item
%     \cs{deadcyles} is decremented for omitted pages.
%   \item
%     LPPL 1.2.
%   \end{Version}
%   \begin{Version}{2006/02/20 v1.3}
%   \item
%     Code is not changed.
%   \item
%     New DTX framework.
%   \item
%     LPPL 1.3
%   \end{Version}
%   \begin{Version}{2006/03/02 v1.4}
%   \item
%     Support for \cs{AtBeginDvi} added.
%   \end{Version}
%   \begin{Version}{2006/03/07 v1.5}
%   \item
%     Job is aborted after last selected page.
%   \end{Version}
%   \begin{Version}{2007/04/11 v1.6}
%   \item
%     Line ends sanitized.
%   \end{Version}
%   \begin{Version}{2007/04/12 v1.7}
%   \item
%     Hard coded box number 255 replaced by macro \cs{@cclv}.
%   \end{Version}
%   \begin{Version}{2008/08/11 v1.8}
%   \item
%     Code is not changed.
%   \item
%     URL updated from \texttt{www.dejanews.com}
%     to \texttt{groups.google.com}.
%   \end{Version}
%   \begin{Version}{2016/05/16 v1.9}
%   \item
%     Documentation updates.
%   \end{Version}
%   \begin{Version}{2020-08-03 v1.10}
%   \item Updated to follow the changes in the hook management
%   of LaTeX 2020/10/01
%   \end{Version}
% \end{History}
%
% \PrintIndex
%
% \Finale
\endinput

%        (quote the arguments according to the demands of your shell)
%
% Documentation:
%    (a) If pagesel.drv is present:
%           latex pagesel.drv
%    (b) Without pagesel.drv:
%           latex pagesel.dtx; ...
%    The class ltxdoc loads the configuration file ltxdoc.cfg
%    if available. Here you can specify further options, e.g.
%    use A4 as paper format:
%       \PassOptionsToClass{a4paper}{article}
%
%    Programm calls to get the documentation (example):
%       pdflatex pagesel.dtx
%       makeindex -s gind.ist pagesel.idx
%       pdflatex pagesel.dtx
%       makeindex -s gind.ist pagesel.idx
%       pdflatex pagesel.dtx
%
% Installation:
%    TDS:tex/latex/pagesel/pagesel.sty
%    TDS:doc/latex/pagesel/pagesel.pdf
%    TDS:source/latex/pagesel/pagesel.dtx
%
%<*ignore>
\begingroup
  \catcode123=1 %
  \catcode125=2 %
  \def\x{LaTeX2e}%
\expandafter\endgroup
\ifcase 0\ifx\install y1\fi\expandafter
         \ifx\csname processbatchFile\endcsname\relax\else1\fi
         \ifx\fmtname\x\else 1\fi\relax
\else\csname fi\endcsname
%</ignore>
%<*install>
\input docstrip.tex
\Msg{************************************************************************}
\Msg{* Installation}
\Msg{* Package: pagesel 2020-08-03 v1.10 Select pages of a document for output (HO)}
\Msg{************************************************************************}

\keepsilent
\askforoverwritefalse

\let\MetaPrefix\relax
\preamble

This is a generated file.

Project: pagesel
Version: 2020-08-03 v1.10

Copyright (C)
   1999, 2003, 2006-2008 Heiko Oberdiek
   2016-2020 Oberdiek Package Support Group

This work may be distributed and/or modified under the
conditions of the LaTeX Project Public License, either
version 1.3c of this license or (at your option) any later
version. This version of this license is in
   https://www.latex-project.org/lppl/lppl-1-3c.txt
and the latest version of this license is in
   https://www.latex-project.org/lppl.txt
and version 1.3 or later is part of all distributions of
LaTeX version 2005/12/01 or later.

This work has the LPPL maintenance status "maintained".

The Current Maintainers of this work are
Heiko Oberdiek and the Oberdiek Package Support Group
https://github.com/ho-tex/pagesel/issues


This work consists of the main source file pagesel.dtx
and the derived files
   pagesel.sty, pagesel-2016-05-16.sty, pagesel.pdf,
   pagesel.ins, pagesel.drv.

\endpreamble
\let\MetaPrefix\DoubleperCent

\generate{%
  \file{pagesel.ins}{\from{pagesel.dtx}{install}}%
  \file{pagesel.drv}{\from{pagesel.dtx}{driver}}%
  \usedir{tex/latex/pagesel}%
  \file{pagesel.sty}{\from{pagesel.dtx}{package}}%
  \file{pagesel-2016-05-16.sty}{\from{pagesel.dtx}{packagefrozen}}
}

\catcode32=13\relax% active space
\let =\space%
\Msg{************************************************************************}
\Msg{*}
\Msg{* To finish the installation you have to move the following}
\Msg{* file into a directory searched by TeX:}
\Msg{*}
\Msg{*     pagesel.sty}
\Msg{*}
\Msg{* To produce the documentation run the file `pagesel.drv'}
\Msg{* through LaTeX.}
\Msg{*}
\Msg{* Happy TeXing!}
\Msg{*}
\Msg{************************************************************************}

\endbatchfile
%</install>
%<*ignore>
\fi
%</ignore>
%<*driver>
\NeedsTeXFormat{LaTeX2e}
\ProvidesFile{pagesel.drv}%
  [2020-08-03 v1.10 Select pages of a document for output (HO)]%
\documentclass{ltxdoc}
\usepackage{holtxdoc}[2011/11/22]
\begin{document}
  \DocInput{pagesel.dtx}%
\end{document}
%</driver>
% \fi
%
%
%
% \GetFileInfo{pagesel.drv}
%
% \title{The \xpackage{pagesel} package}
% \date{2020-08-03 v1.10}
% \author{Heiko Oberdiek\thanks
% {Please report any issues at \url{https://github.com/ho-tex/pagesel/issues}}}
%
% \maketitle
%
% \begin{abstract}
% Single pages or page areas can be selected for output.
% \end{abstract}
%
% \tableofcontents
%
% \newenvironment{param}{^^A
%   \newcommand{\entry}[1]{\meta{\###1}:&}^^A
%   \begin{tabular}[t]{@{}l@{ }l@{}}^^A
% }{^^A
%   \end{tabular}^^A
% }
%
% \newcommand*{\Option}[1]{\textsf{#1}}
%
% \section{Usage}
%    The package \Package{pagesel} is a \LaTeXe\ package:
%    \begin{quote}
%      |\usepackage|\oarg{options}|{pagesel}|
%    \end{quote}
%    (For plain\TeX\ and \LaTeX\,2.09 the similar package
%    \URL{\Package{selectp}}^^A
%    {https://ctan.org/pkg/selectp}
%    from \NameEmail{Donald Arsenau}{asnd@triumf.ca} can be used.)
%
%    Depending on the options the package works in two modes:
%    \begin{enumerate}
%    \item If no page selecting option is present, so the package
%          ignores the other options and finishes itself. So no
%          page will be suppressed by the package and auxiliary files
%          will be written.
%    \item With at least one page selecting option the specified
%          pages are selected and the other are suppressed.
%          The default for this mode is that auxiliary will not be
%          overwritten. (This can be changed by an option.)
%    \end{enumerate}
%
% \subsection{Page selecting}
%    The package \Package{pagesel} sets up a new counter that is
%    incremented by each \cmd{\shipout}.
%    In this way the package counts the output pages regardless the value
%    of the page counter. So each page can individually by addressed,
%    even if there are several pages with the same page number.
%
% \subsubsection{Options\texorpdfstring{ for selecting pages}{}}
%    \begin{description}
%    \item[\Option{odd}:] The output pages must have an odd number.
%         All even output pages are suppressed. If there are no
%         page areas specified so all odd pages are print. With
%         page areas only the odd pages in this areas are selected.
%    \item[\Option{even}:] The opposite of option \Option{odd}.
%    \item[Page area:] A page area consists of three elements:
%         the starting output page number, an ``area'' hyphen, and
%         the output page number of the last page in this area.
%         Each component is optional, so there are four kinds
%         to spezify a page area:
%         \begin{description}
%         \item[\meta{m}\Option{-}\meta{n}:] All pages between
%              \meta{m} and \meta{n} inclusive.
%         \item[\Option{-}\meta{n}:] All pages until \meta{n} inclusive.
%         \item[\meta{m}\Option{-}:] The page area starts with \meta{m}
%              and all pages to the end of document are selected.
%         \item[\Option{-}:] All pages (not very useful).
%         \item[\meta{s}:] The single page \meta{s}.
%         \end{description}
%    \end{description}
%
% \subsubsection{Examples}
%    \newcommand*{\exam}[1]{\texttt{\strut[#1]}}^^A hash-ok
%    \begin{tabular}{ll}
%      Options & Output pages\\
%      \hline
%      \exam{1, 4, 9}&  1, 4, and 9\\
%      \exam{7-10, 3}&  3, 7, 8, 9, and 10\\
%      \exam{odd, 3-6}& 3, and 5\\
%      \exam{-4, 3, even, 7-8}& 2, 4, and 8\\
%    \end{tabular}
%
% \subsection{Auxiliary files}
%    If a page is suppressed, the \cmd{\write} commands are not
%    performed. Labels, index entries, or entries for the
%    table of contents aren't written. So it is likely that
%    the table of contents, registers, and lists are incomplete.
% \subsubsection{Options\texorpdfstring{ for handling auxiliary files}{}}
%    \begin{description}
%    \item[\Option{nofiles}:] This is the default. Auxiliary files are
%         read but not written or changed. Also the job is aborted
%         after the last selected page for saving time.
%    \item[\Option{nonofiles}/\Option{files}:] Auxiliary files are
%         written.
%    \end{description}
% \subsubsection{\texorpdfstring{Package }{}\Package{hyperref}}
%    In old versions of \Package{hyperref} [1999/04/12 v6.55] (and below)
%    there is a bug with \cmd{\nofiles}:
%    \begin{itemize}
%    \item Some ``garbage'' appears on terminal and in the log file.
%          This is harmless and can be ignored.
%    \item The outline auxiliary file \cmd{\jobname.out}, however,
%          is opened and truncated to zero bytes.
%          Version 1.0 of this package had
%          loaded a patch file \File{hypnofil.tex}, if it detects
%          \Package{hyperref} to get \cmd{\nofiles} work.
%
%          With the new version of \Package{hyperref} [1999/04/13 v6.56]
%          \cmd{\nofiles} works now. Therefore the workaround code
%          is no longer needed and removed.
%    \end{itemize}
%
% \StopEventually{
% }
%
% \section{Implementation}
%    \begin{macrocode}
%<*package>
%    \end{macrocode}
% \subsection{New implementation using the LaTeX kernel hooks}
%    \begin{macrocode}
\NeedsTeXFormat{LaTeX2e}
\ProvidesPackage{pagesel}
  [2020-08-03 v1.10 Select pages of a document for output (HO)]%
%    \end{macrocode}
%    \begin{macrocode}
\providecommand\IfFormatAtLeastTF{\@ifl@t@r\fmtversion}
\IfFormatAtLeastTF{2020/10/01}{}{%%
%% This is file `pagesel-2016-05-16.sty',
%% generated with the docstrip utility.
%%
%% The original source files were:
%%
%% pagesel.dtx  (with options: `packagefrozen')
%% 
%% This is a generated file.
%% 
%% Project: pagesel
%% Version: 2020-08-03 v1.10
%% 
%% Copyright (C)
%%    1999, 2003, 2006-2008 Heiko Oberdiek
%%    2016-2020 Oberdiek Package Support Group
%% 
%% This work may be distributed and/or modified under the
%% conditions of the LaTeX Project Public License, either
%% version 1.3c of this license or (at your option) any later
%% version. This version of this license is in
%%    https://www.latex-project.org/lppl/lppl-1-3c.txt
%% and the latest version of this license is in
%%    https://www.latex-project.org/lppl.txt
%% and version 1.3 or later is part of all distributions of
%% LaTeX version 2005/12/01 or later.
%% 
%% This work has the LPPL maintenance status "maintained".
%% 
%% The Current Maintainers of this work are
%% Heiko Oberdiek and the Oberdiek Package Support Group
%% https://github.com/ho-tex/pagesel/issues
%% 
%% This work consists of the main source file pagesel.dtx
%% and the derived files
%%    pagesel.sty, pagesel-2016-05-16.sty, pagesel.pdf,
%%    pagesel.ins, pagesel.drv.
%% 
\NeedsTeXFormat{LaTeX2e}
\ProvidesPackage{pagesel}
  [2020-08-03 v1.10 Select pages of a document for output (legacy code) (HO)]%
\@ifundefined{ps@makevoid}{}{%
  \PackageWarningNoLine{pagesel}{Package already loaded.}%
  \endinput
}
\newcommand*{\ps@makevoid}{%
  \global\setbox\@cclv\copy\voidb@x
  \begingroup
    \count@=\deadcycles
    \advance\count@ by -1\relax
    \deadcycles=\count@
  \endgroup
}
\newcommand*\ps@oddpages{0}
\DeclareOption{odd}{\renewcommand*\ps@oddpages{1}}
\DeclareOption{even}{\renewcommand*\ps@oddpages{2}}
\DeclareOption{nofiles}{\let\ps@nofiles\nofiles}
\DeclareOption{nonofiles}{\let\ps@nofiles\@empty}
\DeclareOption{files}{\let\ps@nofiles\@empty}
\ExecuteOptions{nofiles}
\DeclareOption*{%
  \begingroup
    \expandafter\ps@checkoption\CurrentOption-\END
    \edef\x{\endgroup\noexpand\ps@store{\ps@first}{\ps@last}}%
  \x
}
\newcommand\ps@checkoption{}
\def\ps@checkoption#1-#2\END{%
  \ifx\\#2\\%
    \ifx\\#1\\%
      % empty option
      \def\ps@first{\maxdimen}%
      \def\ps@last{\maxdimen}%
    \else
      \edef\ps@first{#1}%
      \edef\ps@last{#1}%
    \fi
  \else
    \ifx\\#1\\%
      \def\ps@first{-\maxdimen}%
    \else
      \edef\ps@first{#1}%
    \fi
    \ps@checklast#2%
  \fi
}
\newcommand\ps@checklast{}
\def\ps@checklast#1-{%
  \ifx\\#1\\%
    \def\ps@last{\maxdimen}%
  \else
    \edef\ps@last{#1}%
  \fi
}
\newcommand*{\ps@store}[2]{%
  \expandafter\def\expandafter\ps@testlist\expandafter{%
    \ps@testlist\ps@pagetest{#1}{#2}%
  }%
}
\newcommand*\ps@testlist{}
\ProcessOptions
\begingroup
  \edef\x{%
    \ifnum\ps@oddpages>0 \relax\fi
    \ifx\ps@testlist\@empty\else\relax\fi
  }%
  \ifx\x\@empty
    \endgroup
    \PackageInfo{pagesel}{Nothing to do}%
    \expandafter\endinput
  \fi
\endgroup
\RequirePackage{everyshi}
\ps@nofiles
\newcounter{ps@count}
\setcounter{ps@count}{0}
\long\def\ps@ReturnAfterElseFi#1\else#2\fi{\fi#1}
\long\def\ps@ReturnAfterFi#1\fi{\fi#1}
\newcommand{\ps@lastpage}{\maxdimen}
\ifx\ps@nofiles\nofiles
  \ifx\ps@testlist\@empty
  \else
    \def\ps@lastpage{0}%
    \newcommand*{\ps@pagetest}[2]{%
      \ifnum#2>\ps@lastpage\relax
        \def\ps@lastpage{#2}%
      \fi
    }%
    \ps@testlist
    \let\ps@pagetest\relax
  \fi
\fi
\newcommand*{\ps@ifinset}[4]{%
  \ifnum#1>\value{ps@count}%
    \ps@ReturnAfterElseFi{#4}%
  \else
    \ps@ReturnAfterFi{%
      \ifnum#2<\value{ps@count}%
        \ps@ReturnAfterElseFi{#4}%
      \else
        \ps@ReturnAfterFi{#3}%
      \fi
    }%
  \fi
}
\newcommand*{\ps@pagetest}[2]{%
  \ps@ifinset{#1}{#2}{\let\ps@next\@empty}{}%
}
\EveryShipout{%
  \stepcounter{ps@count}%
  \ifnum\value{ps@count}>\ps@lastpage\relax
    \global\output{%
      \ps@cleanup@if
      \ps@group@message
      \typeout{%
        Package pagesel Notice: Aborting LaTeX job %
        after last selected page (\ps@lastpage).%
      }%
      \ps@message@ignore
      \global\setbox\@cclv\box\voidb@x
      \deadcycles0\relax
      \aftergroup\@@end
    }%
  \fi
  \let\ps@next\@empty
  \ifx\ps@testlist\@empty
  \else
    \let\ps@next\ps@makevoid
    \ps@testlist
  \fi
  \ifnum\ps@oddpages=1 %
    \ifodd\value{ps@count}%
    \else
    \let\ps@next\ps@makevoid
    \fi
  \fi
  \ifnum\ps@oddpages=2 %
    \ifodd\value{ps@count}%
    \let\ps@next\ps@makevoid
    \else
    \fi
  \fi
  \ps@begindvi
  \ps@next
}
\begingroup\expandafter\expandafter\expandafter\endgroup
\expandafter\ifx\csname currentiflevel\endcsname\relax
  \let\ps@cleanup@if\@empty
\else
  \def\ps@cleanup@if{%
    \ifnum\currentiflevel>\@ne
      \csname fi\endcsname
      \expandafter\ps@cleanup@if
    \fi
  }%
\fi
 \begingroup\expandafter\expandafter\expandafter\endgroup
 \expandafter\ifx\csname currentgrouplevel\endcsname\relax
  \let\ps@group@message\@empty
  \def\ps@message@ignore{%
    \typeout{%
      (pagesel) \space\space\@spaces\@spaces\@spaces
      Messages (\string\end\space occurred ...) can be ignored.%
    }%
  }%
\else
  \def\ps@group@message{%
    \ifnum\currentgrouplevel>\@ne
      \def\ps@message@ignore{%
        \typeout{%
          (pagesel) \space\space\@spaces\@spaces\@spaces
          Message (\string\end\space occurred ...) %
          can be ignored.%
        }%
      }%
    \else
      \let\ps@message@ignore\@empty
    \fi
  }%
\fi
\newbox\ps@begindvibox
\ifvoid\@begindvibox
\else
  \global\setbox\ps@begindvibox\vbox{%
    \unvbox\@begindvibox
  }%
\fi
\let\ps@org@AtBeginDvi\AtBeginDvi
\def\AtBeginDvi#1{%
  \global\setbox\ps@begindvibox\vbox{%
    \unvbox\ps@begindvibox
    #1%
  }%
  \ps@org@AtBeginDvi{#1}%
}
\def\ps@begindvi{%
  \ifx\ps@next\@empty
    \global\let\ps@begindvi\@empty
  \else
    \global\let\ps@begindvi\ps@begindvi@do
  \fi
}
\def\ps@begindvi@do{%
  \ifx\ps@next\@empty
    \setbox\@cclv\vbox{%
      \unvbox\ps@begindvibox
      \box\@cclv
    }%
    \global\let\ps@begindvi\@empty
  \fi
}
\endinput
%%
%% End of file `pagesel-2016-05-16.sty'.
}
\IfFormatAtLeastTF{2020/10/01}{}{\endinput}

%    \end{macrocode}
%    If the package is loaded twice, the package code does not
%    work. So stop loading the package, if it is already loaded.
%    \begin{macrocode}
\@ifundefined{ps@oddpages}{}{%
  \PackageWarningNoLine{pagesel}{Package already loaded.}%
  \endinput
}
%    \end{macrocode}
%    \begin{macrocode}
%</package>
%    \end{macrocode}
% \subsection{Package}
%    \begin{macrocode}
%<*packagefrozen>
\NeedsTeXFormat{LaTeX2e}
\ProvidesPackage{pagesel}
  [2020-08-03 v1.10 Select pages of a document for output (legacy code) (HO)]%
%    \end{macrocode}
%
%    If the package is loaded twice, the package code does not
%    work. So stop loading the package, if it is already loaded.
%    \begin{macrocode}
\@ifundefined{ps@makevoid}{}{%
  \PackageWarningNoLine{pagesel}{Package already loaded.}%
  \endinput
}
%    \end{macrocode}
%
%    \begin{macro}{\ps@makevoid}
%    Macro \cmd{\ps@makevoid} clears the output box. Because
%    nothing is shipped out and this is intended, we reduce
%    the counter \cmd{\deadcycles} in order to avoid problems, if
%    more than \cmd{\maxdeadcycles} pages are omitted.
%    \begin{macrocode}
\newcommand*{\ps@makevoid}{%
  \global\setbox\@cclv\copy\voidb@x
  \begingroup
    \count@=\deadcycles
    \advance\count@ by -1\relax
    \deadcycles=\count@
  \endgroup
}
%</packagefrozen>
%    \end{macrocode}
%    \end{macro}
%
%    \begin{macro}{\ps@oddpages}
%    \begin{macrocode}
%<*package|packagefrozen>
\newcommand*\ps@oddpages{0}
\DeclareOption{odd}{\renewcommand*\ps@oddpages{1}}
\DeclareOption{even}{\renewcommand*\ps@oddpages{2}}
%    \end{macrocode}
%    \end{macro}
%
%    \begin{macrocode}
\DeclareOption{nofiles}{\let\ps@nofiles\nofiles}
\DeclareOption{nonofiles}{\let\ps@nofiles\@empty}
\DeclareOption{files}{\let\ps@nofiles\@empty}
\ExecuteOptions{nofiles}
%    \end{macrocode}
%
%    \begin{macrocode}
\DeclareOption*{%
  \begingroup
    \expandafter\ps@checkoption\CurrentOption-\END
    \edef\x{\endgroup\noexpand\ps@store{\ps@first}{\ps@last}}%
  \x
}
%    \end{macrocode}
%
%    \begin{macro}{\ps@checkoption}
%    \begin{macrocode}
\newcommand\ps@checkoption{}
\def\ps@checkoption#1-#2\END{%
  \ifx\\#2\\%
    \ifx\\#1\\%
      % empty option
      \def\ps@first{\maxdimen}%
      \def\ps@last{\maxdimen}%
    \else
      \edef\ps@first{#1}%
      \edef\ps@last{#1}%
    \fi
  \else
    \ifx\\#1\\%
      \def\ps@first{-\maxdimen}%
    \else
      \edef\ps@first{#1}%
    \fi
    \ps@checklast#2%
  \fi
}
%    \end{macrocode}
%    \end{macro}
%
%    \begin{macro}{\ps@checklast}
%    \begin{macrocode}
\newcommand\ps@checklast{}
\def\ps@checklast#1-{%
  \ifx\\#1\\%
    \def\ps@last{\maxdimen}%
  \else
    \edef\ps@last{#1}%
  \fi
}
%    \end{macrocode}
%    \end{macro}
%
%    \begin{macro}{\ps@store}
%    \begin{macrocode}
\newcommand*{\ps@store}[2]{%
  \expandafter\def\expandafter\ps@testlist\expandafter{%
    \ps@testlist\ps@pagetest{#1}{#2}%
  }%
}
%    \end{macrocode}
%    \end{macro}
%
%    \begin{macro}{\ps@testlist}
%    \begin{macrocode}
\newcommand*\ps@testlist{}
%    \end{macrocode}
%    \end{macro}
%
%    \begin{macrocode}
\ProcessOptions
%    \end{macrocode}
%
%    \begin{macrocode}
\begingroup
  \edef\x{%
    \ifnum\ps@oddpages>0 \relax\fi
    \ifx\ps@testlist\@empty\else\relax\fi
  }%
  \ifx\x\@empty
    \endgroup
    \PackageInfo{pagesel}{Nothing to do}%
    \expandafter\endinput
  \fi
\endgroup
%    \end{macrocode}
%
%    \begin{macrocode}
%</package|packagefrozen>
%<*packagefrozen>
\RequirePackage{everyshi}
%</packagefrozen>
%    \end{macrocode}
%
%    \begin{macrocode}
%<*package|packagefrozen>
\ps@nofiles
%    \end{macrocode}
%
%    \begin{macro}{\c@ps@count}
%    \begin{macrocode}
\newcounter{ps@count}
\setcounter{ps@count}{0}
%    \end{macrocode}
%    \end{macro}
%
%    \begin{macro}{\ps@ReturnAfterElseFi}
%    \begin{macro}{\ps@ReturnAfterFi}
%    \begin{macrocode}
\long\def\ps@ReturnAfterElseFi#1\else#2\fi{\fi#1}
\long\def\ps@ReturnAfterFi#1\fi{\fi#1}
%    \end{macrocode}
%    \end{macro}
%    \end{macro}
%
%    \begin{macrocode}
\newcommand{\ps@lastpage}{\maxdimen}
\ifx\ps@nofiles\nofiles
  \ifx\ps@testlist\@empty
  \else
    \def\ps@lastpage{0}%
    \newcommand*{\ps@pagetest}[2]{%
      \ifnum#2>\ps@lastpage\relax
        \def\ps@lastpage{#2}%
      \fi
    }%
    \ps@testlist
    \let\ps@pagetest\relax
  \fi
\fi
%    \end{macrocode}
%
%    \begin{macro}{\ps@ifinset}
%    \begin{macrocode}
\newcommand*{\ps@ifinset}[4]{%
  \ifnum#1>\value{ps@count}%
    \ps@ReturnAfterElseFi{#4}%
  \else
    \ps@ReturnAfterFi{%
      \ifnum#2<\value{ps@count}%
        \ps@ReturnAfterElseFi{#4}%
      \else
        \ps@ReturnAfterFi{#3}%
      \fi
    }%
  \fi
}
%    \end{macrocode}
%    \end{macro}
%
%    \begin{macro}{\ps@pagetest}
%    \begin{macrocode}
\newcommand*{\ps@pagetest}[2]{%
  \ps@ifinset{#1}{#2}{\let\ps@next\@empty}{}%
}
%    \end{macrocode}
%    \end{macro}
%
%    \begin{macrocode}
%</package|packagefrozen>
%<packagefrozen>\EveryShipout{%
%<package>\AddToHook{shipout/before}{%
%<*package|packagefrozen>
  \stepcounter{ps@count}%
  \ifnum\value{ps@count}>\ps@lastpage\relax
    \global\output{%
      \ps@cleanup@if
      \ps@group@message
      \typeout{%
        Package pagesel Notice: Aborting LaTeX job %
        after last selected page (\ps@lastpage).%
      }%
      \ps@message@ignore
      \global\setbox\@cclv\box\voidb@x
      \deadcycles0\relax
%    \end{macrocode}
%    First leave the output group before ending the job.
%    \begin{macrocode}
      \aftergroup\@@end
    }%
  \fi
  \let\ps@next\@empty
  \ifx\ps@testlist\@empty
  \else
%<packagefrozen>    \let\ps@next\ps@makevoid
%<package>    \let\ps@next\DiscardShipoutBox
    \ps@testlist
  \fi
  \ifnum\ps@oddpages=1 %
    \ifodd\value{ps@count}%
    \else
%<packagefrozen>    \let\ps@next\ps@makevoid
%<package>    \let\ps@next\DiscardShipoutBox
    \fi
  \fi
  \ifnum\ps@oddpages=2 %
    \ifodd\value{ps@count}%
%<packagefrozen>    \let\ps@next\ps@makevoid
%<package>    \let\ps@next\DiscardShipoutBox
    \else
    \fi
  \fi
%<packagefrozen>  \ps@begindvi
  \ps@next
}
%</package|packagefrozen>
%    \end{macrocode}
%
%    \begin{macrocode}
%<*package|packagefrozen>
%<packagefrozen>\begingroup\expandafter\expandafter\expandafter\endgroup
%<packagefrozen>\expandafter\ifx\csname currentiflevel\endcsname\relax
%<packagefrozen>  \let\ps@cleanup@if\@empty
%<packagefrozen>\else
  \def\ps@cleanup@if{%
    \ifnum\currentiflevel>\@ne
      \csname fi\endcsname
      \expandafter\ps@cleanup@if
    \fi
  }%
%<packagefrozen>\fi
%    \end{macrocode}
%    Because of \cs{aftergroup} it is too dangerous to perform
%    a similar cleanup for groups.
%    \begin{macrocode}
%<packagefrozen> \begingroup\expandafter\expandafter\expandafter\endgroup
%<packagefrozen> \expandafter\ifx\csname currentgrouplevel\endcsname\relax
%<packagefrozen>  \let\ps@group@message\@empty
%<packagefrozen>  \def\ps@message@ignore{%
%<packagefrozen>    \typeout{%
%<packagefrozen>      (pagesel) \space\space\@spaces\@spaces\@spaces
%<packagefrozen>      Messages (\string\end\space occurred ...) can be ignored.%
%<packagefrozen>    }%
%<packagefrozen>  }%
%<packagefrozen>\else
  \def\ps@group@message{%
    \ifnum\currentgrouplevel>\@ne
      \def\ps@message@ignore{%
        \typeout{%
          (pagesel) \space\space\@spaces\@spaces\@spaces
          Message (\string\end\space occurred ...) %
          can be ignored.%
        }%
      }%
    \else
      \let\ps@message@ignore\@empty
    \fi
  }%
%<packagefrozen>\fi
%</package|packagefrozen>
%    \end{macrocode}
%
% \subsection{AtBeginDvi hook support}
%
%    The material of box \cs{@begindvibox} is recorded in parallel
%    in box \cs{ps@begindvibox}.
%    \begin{macrocode}
%<*packagefrozen>
\newbox\ps@begindvibox
\ifvoid\@begindvibox
\else
  \global\setbox\ps@begindvibox\vbox{%
    \unvbox\@begindvibox
  }%
\fi
\let\ps@org@AtBeginDvi\AtBeginDvi
\def\AtBeginDvi#1{%
  \global\setbox\ps@begindvibox\vbox{%
    \unvbox\ps@begindvibox
    #1%
  }%
  \ps@org@AtBeginDvi{#1}%
}
%    \end{macrocode}
%
%    \begin{macro}{\ps@begindvi}
%    Macro \cs{ps@begindvi} is called the similar way as \cs{@begindvi}.
%    If the first page is printed, then \cs{AtBeginDvi} should work
%    as usual. Otherwise the contents of box \cs{ps@begindvibox} is
%    set on the first selected page.
%    \begin{macrocode}
\def\ps@begindvi{%
  \ifx\ps@next\@empty
    \global\let\ps@begindvi\@empty
  \else
    \global\let\ps@begindvi\ps@begindvi@do
  \fi
}
\def\ps@begindvi@do{%
  \ifx\ps@next\@empty
    \setbox\@cclv\vbox{%
      \unvbox\ps@begindvibox
      \box\@cclv
    }%
    \global\let\ps@begindvi\@empty
  \fi
}
%    \end{macrocode}
%    \end{macro}
%
%    \begin{macrocode}
%</packagefrozen>
%    \end{macrocode}
%
% \section{Installation}
%
% \subsection{Download}
%
% \paragraph{Package.} This package is available on
% CTAN\footnote{\CTANpkg{pagesel}}:
% \begin{description}
% \item[\CTAN{macros/latex/contrib/pagesel/pagesel.dtx}] The source file.
% \item[\CTAN{macros/latex/contrib/pagesel/pagesel.pdf}] Documentation.
% \end{description}
%
%
%
% \subsection{Package installation}
%
% \paragraph{Unpacking.} The \xfile{.dtx} file is a self-extracting
% \docstrip\ archive. The files are extracted by running the
% \xfile{.dtx} through \plainTeX:
% \begin{quote}
%   \verb|tex pagesel.dtx|
% \end{quote}
%
% \paragraph{TDS.} Now the different files must be moved into
% the different directories in your installation TDS tree
% (also known as \xfile{texmf} tree):
% \begin{quote}
% \def\t{^^A
% \begin{tabular}{@{}>{\ttfamily}l@{ $\rightarrow$ }>{\ttfamily}l@{}}
%   pagesel.sty & tex/latex/pagesel/pagesel.sty\\
%   pagesel.pdf & doc/latex/pagesel/pagesel.pdf\\
%   pagesel.dtx & source/latex/pagesel/pagesel.dtx\\
% \end{tabular}^^A
% }^^A
% \sbox0{\t}^^A
% \ifdim\wd0>\linewidth
%   \begingroup
%     \advance\linewidth by\leftmargin
%     \advance\linewidth by\rightmargin
%   \edef\x{\endgroup
%     \def\noexpand\lw{\the\linewidth}^^A
%   }\x
%   \def\lwbox{^^A
%     \leavevmode
%     \hbox to \linewidth{^^A
%       \kern-\leftmargin\relax
%       \hss
%       \usebox0
%       \hss
%       \kern-\rightmargin\relax
%     }^^A
%   }^^A
%   \ifdim\wd0>\lw
%     \sbox0{\small\t}^^A
%     \ifdim\wd0>\linewidth
%       \ifdim\wd0>\lw
%         \sbox0{\footnotesize\t}^^A
%         \ifdim\wd0>\linewidth
%           \ifdim\wd0>\lw
%             \sbox0{\scriptsize\t}^^A
%             \ifdim\wd0>\linewidth
%               \ifdim\wd0>\lw
%                 \sbox0{\tiny\t}^^A
%                 \ifdim\wd0>\linewidth
%                   \lwbox
%                 \else
%                   \usebox0
%                 \fi
%               \else
%                 \lwbox
%               \fi
%             \else
%               \usebox0
%             \fi
%           \else
%             \lwbox
%           \fi
%         \else
%           \usebox0
%         \fi
%       \else
%         \lwbox
%       \fi
%     \else
%       \usebox0
%     \fi
%   \else
%     \lwbox
%   \fi
% \else
%   \usebox0
% \fi
% \end{quote}
% If you have a \xfile{docstrip.cfg} that configures and enables \docstrip's
% TDS installing feature, then some files can already be in the right
% place, see the documentation of \docstrip.
%
% \subsection{Refresh file name databases}
%
% If your \TeX~distribution
% (\TeX\,Live, \mikTeX, \dots) relies on file name databases, you must refresh
% these. For example, \TeX\,Live\ users run \verb|texhash| or
% \verb|mktexlsr|.
%
% \subsection{Some details for the interested}
%
% \paragraph{Unpacking with \LaTeX.}
% The \xfile{.dtx} chooses its action depending on the format:
% \begin{description}
% \item[\plainTeX:] Run \docstrip\ and extract the files.
% \item[\LaTeX:] Generate the documentation.
% \end{description}
% If you insist on using \LaTeX\ for \docstrip\ (really,
% \docstrip\ does not need \LaTeX), then inform the autodetect routine
% about your intention:
% \begin{quote}
%   \verb|latex \let\install=y% \iffalse meta-comment
%
% File: pagesel.dtx
% Version: 2020-08-03 v1.10
% Info: Select pages of a document for output
%
% Copyright (C)
%    1999, 2003, 2006-2008 Heiko Oberdiek
%    2016-2020 Oberdiek Package Support Group
%    https://github.com/ho-tex/pagesel/issues
%
% This work may be distributed and/or modified under the
% conditions of the LaTeX Project Public License, either
% version 1.3c of this license or (at your option) any later
% version. This version of this license is in
%    https://www.latex-project.org/lppl/lppl-1-3c.txt
% and the latest version of this license is in
%    https://www.latex-project.org/lppl.txt
% and version 1.3 or later is part of all distributions of
% LaTeX version 2005/12/01 or later.
%
% This work has the LPPL maintenance status "maintained".
%
% The Current Maintainers of this work are
% Heiko Oberdiek and the Oberdiek Package Support Group
% https://github.com/ho-tex/pagesel/issues
%
% This work consists of the main source file pagesel.dtx
% and the derived files
%    pagesel.sty, pagesel-2016-05-16.sty,
%    pagesel.pdf, pagesel.ins, pagesel.drv.
%
% Distribution:
%    CTAN:macros/latex/contrib/pagesel/pagesel.dtx
%    CTAN:macros/latex/contrib/pagesel/pagesel.pdf
%
% Unpacking:
%    (a) If pagesel.ins is present:
%           tex pagesel.ins
%    (b) Without pagesel.ins:
%           tex pagesel.dtx
%    (c) If you insist on using LaTeX
%           latex \let\install=y% \iffalse meta-comment
%
% File: pagesel.dtx
% Version: 2020-08-03 v1.10
% Info: Select pages of a document for output
%
% Copyright (C)
%    1999, 2003, 2006-2008 Heiko Oberdiek
%    2016-2020 Oberdiek Package Support Group
%    https://github.com/ho-tex/pagesel/issues
%
% This work may be distributed and/or modified under the
% conditions of the LaTeX Project Public License, either
% version 1.3c of this license or (at your option) any later
% version. This version of this license is in
%    https://www.latex-project.org/lppl/lppl-1-3c.txt
% and the latest version of this license is in
%    https://www.latex-project.org/lppl.txt
% and version 1.3 or later is part of all distributions of
% LaTeX version 2005/12/01 or later.
%
% This work has the LPPL maintenance status "maintained".
%
% The Current Maintainers of this work are
% Heiko Oberdiek and the Oberdiek Package Support Group
% https://github.com/ho-tex/pagesel/issues
%
% This work consists of the main source file pagesel.dtx
% and the derived files
%    pagesel.sty, pagesel-2016-05-16.sty,
%    pagesel.pdf, pagesel.ins, pagesel.drv.
%
% Distribution:
%    CTAN:macros/latex/contrib/pagesel/pagesel.dtx
%    CTAN:macros/latex/contrib/pagesel/pagesel.pdf
%
% Unpacking:
%    (a) If pagesel.ins is present:
%           tex pagesel.ins
%    (b) Without pagesel.ins:
%           tex pagesel.dtx
%    (c) If you insist on using LaTeX
%           latex \let\install=y\input{pagesel.dtx}
%        (quote the arguments according to the demands of your shell)
%
% Documentation:
%    (a) If pagesel.drv is present:
%           latex pagesel.drv
%    (b) Without pagesel.drv:
%           latex pagesel.dtx; ...
%    The class ltxdoc loads the configuration file ltxdoc.cfg
%    if available. Here you can specify further options, e.g.
%    use A4 as paper format:
%       \PassOptionsToClass{a4paper}{article}
%
%    Programm calls to get the documentation (example):
%       pdflatex pagesel.dtx
%       makeindex -s gind.ist pagesel.idx
%       pdflatex pagesel.dtx
%       makeindex -s gind.ist pagesel.idx
%       pdflatex pagesel.dtx
%
% Installation:
%    TDS:tex/latex/pagesel/pagesel.sty
%    TDS:doc/latex/pagesel/pagesel.pdf
%    TDS:source/latex/pagesel/pagesel.dtx
%
%<*ignore>
\begingroup
  \catcode123=1 %
  \catcode125=2 %
  \def\x{LaTeX2e}%
\expandafter\endgroup
\ifcase 0\ifx\install y1\fi\expandafter
         \ifx\csname processbatchFile\endcsname\relax\else1\fi
         \ifx\fmtname\x\else 1\fi\relax
\else\csname fi\endcsname
%</ignore>
%<*install>
\input docstrip.tex
\Msg{************************************************************************}
\Msg{* Installation}
\Msg{* Package: pagesel 2020-08-03 v1.10 Select pages of a document for output (HO)}
\Msg{************************************************************************}

\keepsilent
\askforoverwritefalse

\let\MetaPrefix\relax
\preamble

This is a generated file.

Project: pagesel
Version: 2020-08-03 v1.10

Copyright (C)
   1999, 2003, 2006-2008 Heiko Oberdiek
   2016-2020 Oberdiek Package Support Group

This work may be distributed and/or modified under the
conditions of the LaTeX Project Public License, either
version 1.3c of this license or (at your option) any later
version. This version of this license is in
   https://www.latex-project.org/lppl/lppl-1-3c.txt
and the latest version of this license is in
   https://www.latex-project.org/lppl.txt
and version 1.3 or later is part of all distributions of
LaTeX version 2005/12/01 or later.

This work has the LPPL maintenance status "maintained".

The Current Maintainers of this work are
Heiko Oberdiek and the Oberdiek Package Support Group
https://github.com/ho-tex/pagesel/issues


This work consists of the main source file pagesel.dtx
and the derived files
   pagesel.sty, pagesel-2016-05-16.sty, pagesel.pdf,
   pagesel.ins, pagesel.drv.

\endpreamble
\let\MetaPrefix\DoubleperCent

\generate{%
  \file{pagesel.ins}{\from{pagesel.dtx}{install}}%
  \file{pagesel.drv}{\from{pagesel.dtx}{driver}}%
  \usedir{tex/latex/pagesel}%
  \file{pagesel.sty}{\from{pagesel.dtx}{package}}%
  \file{pagesel-2016-05-16.sty}{\from{pagesel.dtx}{packagefrozen}}
}

\catcode32=13\relax% active space
\let =\space%
\Msg{************************************************************************}
\Msg{*}
\Msg{* To finish the installation you have to move the following}
\Msg{* file into a directory searched by TeX:}
\Msg{*}
\Msg{*     pagesel.sty}
\Msg{*}
\Msg{* To produce the documentation run the file `pagesel.drv'}
\Msg{* through LaTeX.}
\Msg{*}
\Msg{* Happy TeXing!}
\Msg{*}
\Msg{************************************************************************}

\endbatchfile
%</install>
%<*ignore>
\fi
%</ignore>
%<*driver>
\NeedsTeXFormat{LaTeX2e}
\ProvidesFile{pagesel.drv}%
  [2020-08-03 v1.10 Select pages of a document for output (HO)]%
\documentclass{ltxdoc}
\usepackage{holtxdoc}[2011/11/22]
\begin{document}
  \DocInput{pagesel.dtx}%
\end{document}
%</driver>
% \fi
%
%
%
% \GetFileInfo{pagesel.drv}
%
% \title{The \xpackage{pagesel} package}
% \date{2020-08-03 v1.10}
% \author{Heiko Oberdiek\thanks
% {Please report any issues at \url{https://github.com/ho-tex/pagesel/issues}}}
%
% \maketitle
%
% \begin{abstract}
% Single pages or page areas can be selected for output.
% \end{abstract}
%
% \tableofcontents
%
% \newenvironment{param}{^^A
%   \newcommand{\entry}[1]{\meta{\###1}:&}^^A
%   \begin{tabular}[t]{@{}l@{ }l@{}}^^A
% }{^^A
%   \end{tabular}^^A
% }
%
% \newcommand*{\Option}[1]{\textsf{#1}}
%
% \section{Usage}
%    The package \Package{pagesel} is a \LaTeXe\ package:
%    \begin{quote}
%      |\usepackage|\oarg{options}|{pagesel}|
%    \end{quote}
%    (For plain\TeX\ and \LaTeX\,2.09 the similar package
%    \URL{\Package{selectp}}^^A
%    {https://ctan.org/pkg/selectp}
%    from \NameEmail{Donald Arsenau}{asnd@triumf.ca} can be used.)
%
%    Depending on the options the package works in two modes:
%    \begin{enumerate}
%    \item If no page selecting option is present, so the package
%          ignores the other options and finishes itself. So no
%          page will be suppressed by the package and auxiliary files
%          will be written.
%    \item With at least one page selecting option the specified
%          pages are selected and the other are suppressed.
%          The default for this mode is that auxiliary will not be
%          overwritten. (This can be changed by an option.)
%    \end{enumerate}
%
% \subsection{Page selecting}
%    The package \Package{pagesel} sets up a new counter that is
%    incremented by each \cmd{\shipout}.
%    In this way the package counts the output pages regardless the value
%    of the page counter. So each page can individually by addressed,
%    even if there are several pages with the same page number.
%
% \subsubsection{Options\texorpdfstring{ for selecting pages}{}}
%    \begin{description}
%    \item[\Option{odd}:] The output pages must have an odd number.
%         All even output pages are suppressed. If there are no
%         page areas specified so all odd pages are print. With
%         page areas only the odd pages in this areas are selected.
%    \item[\Option{even}:] The opposite of option \Option{odd}.
%    \item[Page area:] A page area consists of three elements:
%         the starting output page number, an ``area'' hyphen, and
%         the output page number of the last page in this area.
%         Each component is optional, so there are four kinds
%         to spezify a page area:
%         \begin{description}
%         \item[\meta{m}\Option{-}\meta{n}:] All pages between
%              \meta{m} and \meta{n} inclusive.
%         \item[\Option{-}\meta{n}:] All pages until \meta{n} inclusive.
%         \item[\meta{m}\Option{-}:] The page area starts with \meta{m}
%              and all pages to the end of document are selected.
%         \item[\Option{-}:] All pages (not very useful).
%         \item[\meta{s}:] The single page \meta{s}.
%         \end{description}
%    \end{description}
%
% \subsubsection{Examples}
%    \newcommand*{\exam}[1]{\texttt{\strut[#1]}}^^A hash-ok
%    \begin{tabular}{ll}
%      Options & Output pages\\
%      \hline
%      \exam{1, 4, 9}&  1, 4, and 9\\
%      \exam{7-10, 3}&  3, 7, 8, 9, and 10\\
%      \exam{odd, 3-6}& 3, and 5\\
%      \exam{-4, 3, even, 7-8}& 2, 4, and 8\\
%    \end{tabular}
%
% \subsection{Auxiliary files}
%    If a page is suppressed, the \cmd{\write} commands are not
%    performed. Labels, index entries, or entries for the
%    table of contents aren't written. So it is likely that
%    the table of contents, registers, and lists are incomplete.
% \subsubsection{Options\texorpdfstring{ for handling auxiliary files}{}}
%    \begin{description}
%    \item[\Option{nofiles}:] This is the default. Auxiliary files are
%         read but not written or changed. Also the job is aborted
%         after the last selected page for saving time.
%    \item[\Option{nonofiles}/\Option{files}:] Auxiliary files are
%         written.
%    \end{description}
% \subsubsection{\texorpdfstring{Package }{}\Package{hyperref}}
%    In old versions of \Package{hyperref} [1999/04/12 v6.55] (and below)
%    there is a bug with \cmd{\nofiles}:
%    \begin{itemize}
%    \item Some ``garbage'' appears on terminal and in the log file.
%          This is harmless and can be ignored.
%    \item The outline auxiliary file \cmd{\jobname.out}, however,
%          is opened and truncated to zero bytes.
%          Version 1.0 of this package had
%          loaded a patch file \File{hypnofil.tex}, if it detects
%          \Package{hyperref} to get \cmd{\nofiles} work.
%
%          With the new version of \Package{hyperref} [1999/04/13 v6.56]
%          \cmd{\nofiles} works now. Therefore the workaround code
%          is no longer needed and removed.
%    \end{itemize}
%
% \StopEventually{
% }
%
% \section{Implementation}
%    \begin{macrocode}
%<*package>
%    \end{macrocode}
% \subsection{New implementation using the LaTeX kernel hooks}
%    \begin{macrocode}
\NeedsTeXFormat{LaTeX2e}
\ProvidesPackage{pagesel}
  [2020-08-03 v1.10 Select pages of a document for output (HO)]%
%    \end{macrocode}
%    \begin{macrocode}
\providecommand\IfFormatAtLeastTF{\@ifl@t@r\fmtversion}
\IfFormatAtLeastTF{2020/10/01}{}{\input{pagesel-2016-05-16.sty}}
\IfFormatAtLeastTF{2020/10/01}{}{\endinput}

%    \end{macrocode}
%    If the package is loaded twice, the package code does not
%    work. So stop loading the package, if it is already loaded.
%    \begin{macrocode}
\@ifundefined{ps@oddpages}{}{%
  \PackageWarningNoLine{pagesel}{Package already loaded.}%
  \endinput
}
%    \end{macrocode}
%    \begin{macrocode}
%</package>
%    \end{macrocode}
% \subsection{Package}
%    \begin{macrocode}
%<*packagefrozen>
\NeedsTeXFormat{LaTeX2e}
\ProvidesPackage{pagesel}
  [2020-08-03 v1.10 Select pages of a document for output (legacy code) (HO)]%
%    \end{macrocode}
%
%    If the package is loaded twice, the package code does not
%    work. So stop loading the package, if it is already loaded.
%    \begin{macrocode}
\@ifundefined{ps@makevoid}{}{%
  \PackageWarningNoLine{pagesel}{Package already loaded.}%
  \endinput
}
%    \end{macrocode}
%
%    \begin{macro}{\ps@makevoid}
%    Macro \cmd{\ps@makevoid} clears the output box. Because
%    nothing is shipped out and this is intended, we reduce
%    the counter \cmd{\deadcycles} in order to avoid problems, if
%    more than \cmd{\maxdeadcycles} pages are omitted.
%    \begin{macrocode}
\newcommand*{\ps@makevoid}{%
  \global\setbox\@cclv\copy\voidb@x
  \begingroup
    \count@=\deadcycles
    \advance\count@ by -1\relax
    \deadcycles=\count@
  \endgroup
}
%</packagefrozen>
%    \end{macrocode}
%    \end{macro}
%
%    \begin{macro}{\ps@oddpages}
%    \begin{macrocode}
%<*package|packagefrozen>
\newcommand*\ps@oddpages{0}
\DeclareOption{odd}{\renewcommand*\ps@oddpages{1}}
\DeclareOption{even}{\renewcommand*\ps@oddpages{2}}
%    \end{macrocode}
%    \end{macro}
%
%    \begin{macrocode}
\DeclareOption{nofiles}{\let\ps@nofiles\nofiles}
\DeclareOption{nonofiles}{\let\ps@nofiles\@empty}
\DeclareOption{files}{\let\ps@nofiles\@empty}
\ExecuteOptions{nofiles}
%    \end{macrocode}
%
%    \begin{macrocode}
\DeclareOption*{%
  \begingroup
    \expandafter\ps@checkoption\CurrentOption-\END
    \edef\x{\endgroup\noexpand\ps@store{\ps@first}{\ps@last}}%
  \x
}
%    \end{macrocode}
%
%    \begin{macro}{\ps@checkoption}
%    \begin{macrocode}
\newcommand\ps@checkoption{}
\def\ps@checkoption#1-#2\END{%
  \ifx\\#2\\%
    \ifx\\#1\\%
      % empty option
      \def\ps@first{\maxdimen}%
      \def\ps@last{\maxdimen}%
    \else
      \edef\ps@first{#1}%
      \edef\ps@last{#1}%
    \fi
  \else
    \ifx\\#1\\%
      \def\ps@first{-\maxdimen}%
    \else
      \edef\ps@first{#1}%
    \fi
    \ps@checklast#2%
  \fi
}
%    \end{macrocode}
%    \end{macro}
%
%    \begin{macro}{\ps@checklast}
%    \begin{macrocode}
\newcommand\ps@checklast{}
\def\ps@checklast#1-{%
  \ifx\\#1\\%
    \def\ps@last{\maxdimen}%
  \else
    \edef\ps@last{#1}%
  \fi
}
%    \end{macrocode}
%    \end{macro}
%
%    \begin{macro}{\ps@store}
%    \begin{macrocode}
\newcommand*{\ps@store}[2]{%
  \expandafter\def\expandafter\ps@testlist\expandafter{%
    \ps@testlist\ps@pagetest{#1}{#2}%
  }%
}
%    \end{macrocode}
%    \end{macro}
%
%    \begin{macro}{\ps@testlist}
%    \begin{macrocode}
\newcommand*\ps@testlist{}
%    \end{macrocode}
%    \end{macro}
%
%    \begin{macrocode}
\ProcessOptions
%    \end{macrocode}
%
%    \begin{macrocode}
\begingroup
  \edef\x{%
    \ifnum\ps@oddpages>0 \relax\fi
    \ifx\ps@testlist\@empty\else\relax\fi
  }%
  \ifx\x\@empty
    \endgroup
    \PackageInfo{pagesel}{Nothing to do}%
    \expandafter\endinput
  \fi
\endgroup
%    \end{macrocode}
%
%    \begin{macrocode}
%</package|packagefrozen>
%<*packagefrozen>
\RequirePackage{everyshi}
%</packagefrozen>
%    \end{macrocode}
%
%    \begin{macrocode}
%<*package|packagefrozen>
\ps@nofiles
%    \end{macrocode}
%
%    \begin{macro}{\c@ps@count}
%    \begin{macrocode}
\newcounter{ps@count}
\setcounter{ps@count}{0}
%    \end{macrocode}
%    \end{macro}
%
%    \begin{macro}{\ps@ReturnAfterElseFi}
%    \begin{macro}{\ps@ReturnAfterFi}
%    \begin{macrocode}
\long\def\ps@ReturnAfterElseFi#1\else#2\fi{\fi#1}
\long\def\ps@ReturnAfterFi#1\fi{\fi#1}
%    \end{macrocode}
%    \end{macro}
%    \end{macro}
%
%    \begin{macrocode}
\newcommand{\ps@lastpage}{\maxdimen}
\ifx\ps@nofiles\nofiles
  \ifx\ps@testlist\@empty
  \else
    \def\ps@lastpage{0}%
    \newcommand*{\ps@pagetest}[2]{%
      \ifnum#2>\ps@lastpage\relax
        \def\ps@lastpage{#2}%
      \fi
    }%
    \ps@testlist
    \let\ps@pagetest\relax
  \fi
\fi
%    \end{macrocode}
%
%    \begin{macro}{\ps@ifinset}
%    \begin{macrocode}
\newcommand*{\ps@ifinset}[4]{%
  \ifnum#1>\value{ps@count}%
    \ps@ReturnAfterElseFi{#4}%
  \else
    \ps@ReturnAfterFi{%
      \ifnum#2<\value{ps@count}%
        \ps@ReturnAfterElseFi{#4}%
      \else
        \ps@ReturnAfterFi{#3}%
      \fi
    }%
  \fi
}
%    \end{macrocode}
%    \end{macro}
%
%    \begin{macro}{\ps@pagetest}
%    \begin{macrocode}
\newcommand*{\ps@pagetest}[2]{%
  \ps@ifinset{#1}{#2}{\let\ps@next\@empty}{}%
}
%    \end{macrocode}
%    \end{macro}
%
%    \begin{macrocode}
%</package|packagefrozen>
%<packagefrozen>\EveryShipout{%
%<package>\AddToHook{shipout/before}{%
%<*package|packagefrozen>
  \stepcounter{ps@count}%
  \ifnum\value{ps@count}>\ps@lastpage\relax
    \global\output{%
      \ps@cleanup@if
      \ps@group@message
      \typeout{%
        Package pagesel Notice: Aborting LaTeX job %
        after last selected page (\ps@lastpage).%
      }%
      \ps@message@ignore
      \global\setbox\@cclv\box\voidb@x
      \deadcycles0\relax
%    \end{macrocode}
%    First leave the output group before ending the job.
%    \begin{macrocode}
      \aftergroup\@@end
    }%
  \fi
  \let\ps@next\@empty
  \ifx\ps@testlist\@empty
  \else
%<packagefrozen>    \let\ps@next\ps@makevoid
%<package>    \let\ps@next\DiscardShipoutBox
    \ps@testlist
  \fi
  \ifnum\ps@oddpages=1 %
    \ifodd\value{ps@count}%
    \else
%<packagefrozen>    \let\ps@next\ps@makevoid
%<package>    \let\ps@next\DiscardShipoutBox
    \fi
  \fi
  \ifnum\ps@oddpages=2 %
    \ifodd\value{ps@count}%
%<packagefrozen>    \let\ps@next\ps@makevoid
%<package>    \let\ps@next\DiscardShipoutBox
    \else
    \fi
  \fi
%<packagefrozen>  \ps@begindvi
  \ps@next
}
%</package|packagefrozen>
%    \end{macrocode}
%
%    \begin{macrocode}
%<*package|packagefrozen>
%<packagefrozen>\begingroup\expandafter\expandafter\expandafter\endgroup
%<packagefrozen>\expandafter\ifx\csname currentiflevel\endcsname\relax
%<packagefrozen>  \let\ps@cleanup@if\@empty
%<packagefrozen>\else
  \def\ps@cleanup@if{%
    \ifnum\currentiflevel>\@ne
      \csname fi\endcsname
      \expandafter\ps@cleanup@if
    \fi
  }%
%<packagefrozen>\fi
%    \end{macrocode}
%    Because of \cs{aftergroup} it is too dangerous to perform
%    a similar cleanup for groups.
%    \begin{macrocode}
%<packagefrozen> \begingroup\expandafter\expandafter\expandafter\endgroup
%<packagefrozen> \expandafter\ifx\csname currentgrouplevel\endcsname\relax
%<packagefrozen>  \let\ps@group@message\@empty
%<packagefrozen>  \def\ps@message@ignore{%
%<packagefrozen>    \typeout{%
%<packagefrozen>      (pagesel) \space\space\@spaces\@spaces\@spaces
%<packagefrozen>      Messages (\string\end\space occurred ...) can be ignored.%
%<packagefrozen>    }%
%<packagefrozen>  }%
%<packagefrozen>\else
  \def\ps@group@message{%
    \ifnum\currentgrouplevel>\@ne
      \def\ps@message@ignore{%
        \typeout{%
          (pagesel) \space\space\@spaces\@spaces\@spaces
          Message (\string\end\space occurred ...) %
          can be ignored.%
        }%
      }%
    \else
      \let\ps@message@ignore\@empty
    \fi
  }%
%<packagefrozen>\fi
%</package|packagefrozen>
%    \end{macrocode}
%
% \subsection{AtBeginDvi hook support}
%
%    The material of box \cs{@begindvibox} is recorded in parallel
%    in box \cs{ps@begindvibox}.
%    \begin{macrocode}
%<*packagefrozen>
\newbox\ps@begindvibox
\ifvoid\@begindvibox
\else
  \global\setbox\ps@begindvibox\vbox{%
    \unvbox\@begindvibox
  }%
\fi
\let\ps@org@AtBeginDvi\AtBeginDvi
\def\AtBeginDvi#1{%
  \global\setbox\ps@begindvibox\vbox{%
    \unvbox\ps@begindvibox
    #1%
  }%
  \ps@org@AtBeginDvi{#1}%
}
%    \end{macrocode}
%
%    \begin{macro}{\ps@begindvi}
%    Macro \cs{ps@begindvi} is called the similar way as \cs{@begindvi}.
%    If the first page is printed, then \cs{AtBeginDvi} should work
%    as usual. Otherwise the contents of box \cs{ps@begindvibox} is
%    set on the first selected page.
%    \begin{macrocode}
\def\ps@begindvi{%
  \ifx\ps@next\@empty
    \global\let\ps@begindvi\@empty
  \else
    \global\let\ps@begindvi\ps@begindvi@do
  \fi
}
\def\ps@begindvi@do{%
  \ifx\ps@next\@empty
    \setbox\@cclv\vbox{%
      \unvbox\ps@begindvibox
      \box\@cclv
    }%
    \global\let\ps@begindvi\@empty
  \fi
}
%    \end{macrocode}
%    \end{macro}
%
%    \begin{macrocode}
%</packagefrozen>
%    \end{macrocode}
%
% \section{Installation}
%
% \subsection{Download}
%
% \paragraph{Package.} This package is available on
% CTAN\footnote{\CTANpkg{pagesel}}:
% \begin{description}
% \item[\CTAN{macros/latex/contrib/pagesel/pagesel.dtx}] The source file.
% \item[\CTAN{macros/latex/contrib/pagesel/pagesel.pdf}] Documentation.
% \end{description}
%
%
%
% \subsection{Package installation}
%
% \paragraph{Unpacking.} The \xfile{.dtx} file is a self-extracting
% \docstrip\ archive. The files are extracted by running the
% \xfile{.dtx} through \plainTeX:
% \begin{quote}
%   \verb|tex pagesel.dtx|
% \end{quote}
%
% \paragraph{TDS.} Now the different files must be moved into
% the different directories in your installation TDS tree
% (also known as \xfile{texmf} tree):
% \begin{quote}
% \def\t{^^A
% \begin{tabular}{@{}>{\ttfamily}l@{ $\rightarrow$ }>{\ttfamily}l@{}}
%   pagesel.sty & tex/latex/pagesel/pagesel.sty\\
%   pagesel.pdf & doc/latex/pagesel/pagesel.pdf\\
%   pagesel.dtx & source/latex/pagesel/pagesel.dtx\\
% \end{tabular}^^A
% }^^A
% \sbox0{\t}^^A
% \ifdim\wd0>\linewidth
%   \begingroup
%     \advance\linewidth by\leftmargin
%     \advance\linewidth by\rightmargin
%   \edef\x{\endgroup
%     \def\noexpand\lw{\the\linewidth}^^A
%   }\x
%   \def\lwbox{^^A
%     \leavevmode
%     \hbox to \linewidth{^^A
%       \kern-\leftmargin\relax
%       \hss
%       \usebox0
%       \hss
%       \kern-\rightmargin\relax
%     }^^A
%   }^^A
%   \ifdim\wd0>\lw
%     \sbox0{\small\t}^^A
%     \ifdim\wd0>\linewidth
%       \ifdim\wd0>\lw
%         \sbox0{\footnotesize\t}^^A
%         \ifdim\wd0>\linewidth
%           \ifdim\wd0>\lw
%             \sbox0{\scriptsize\t}^^A
%             \ifdim\wd0>\linewidth
%               \ifdim\wd0>\lw
%                 \sbox0{\tiny\t}^^A
%                 \ifdim\wd0>\linewidth
%                   \lwbox
%                 \else
%                   \usebox0
%                 \fi
%               \else
%                 \lwbox
%               \fi
%             \else
%               \usebox0
%             \fi
%           \else
%             \lwbox
%           \fi
%         \else
%           \usebox0
%         \fi
%       \else
%         \lwbox
%       \fi
%     \else
%       \usebox0
%     \fi
%   \else
%     \lwbox
%   \fi
% \else
%   \usebox0
% \fi
% \end{quote}
% If you have a \xfile{docstrip.cfg} that configures and enables \docstrip's
% TDS installing feature, then some files can already be in the right
% place, see the documentation of \docstrip.
%
% \subsection{Refresh file name databases}
%
% If your \TeX~distribution
% (\TeX\,Live, \mikTeX, \dots) relies on file name databases, you must refresh
% these. For example, \TeX\,Live\ users run \verb|texhash| or
% \verb|mktexlsr|.
%
% \subsection{Some details for the interested}
%
% \paragraph{Unpacking with \LaTeX.}
% The \xfile{.dtx} chooses its action depending on the format:
% \begin{description}
% \item[\plainTeX:] Run \docstrip\ and extract the files.
% \item[\LaTeX:] Generate the documentation.
% \end{description}
% If you insist on using \LaTeX\ for \docstrip\ (really,
% \docstrip\ does not need \LaTeX), then inform the autodetect routine
% about your intention:
% \begin{quote}
%   \verb|latex \let\install=y\input{pagesel.dtx}|
% \end{quote}
% Do not forget to quote the argument according to the demands
% of your shell.
%
% \paragraph{Generating the documentation.}
% You can use both the \xfile{.dtx} or the \xfile{.drv} to generate
% the documentation. The process can be configured by the
% configuration file \xfile{ltxdoc.cfg}. For instance, put this
% line into this file, if you want to have A4 as paper format:
% \begin{quote}
%   \verb|\PassOptionsToClass{a4paper}{article}|
% \end{quote}
% An example follows how to generate the
% documentation with pdf\LaTeX:
% \begin{quote}
%\begin{verbatim}
%pdflatex pagesel.dtx
%makeindex -s gind.ist pagesel.idx
%pdflatex pagesel.dtx
%makeindex -s gind.ist pagesel.idx
%pdflatex pagesel.dtx
%\end{verbatim}
% \end{quote}
%
% \begin{History}
%   \begin{Version}{1999/03/01 v0.9}
%   \item
%     The first version was built as a response to a question
%     of \NameEmail{Dirk Kuypers}{dk@comnets.rwth-aachen.de},
%     published in the newsgroup
%     \href{news:de.comp.text.tex}{de.comp.text.tex}:\\
%     \URL{``\link{Re: pdflatex nur fuer bestimmte Seiten?!?}''}^^A
%     {https://groups.google.com/group/de.comp.text.tex/msg/6b68c7b3439fb658}
%   \end{Version}
%   \begin{Version}{1999/04/05 v1.0}
%   \item
%     Documentation added in dtx format.
%   \item
%     Copyright: LPPL (\CTAN{macros/latex/base/lppl.txt})
%   \item
%     Options |odd|, |even| added.
%   \item
%     \cmd{\nofiles} added, bug fix for \Package{hyperref}.
%   \item
%     Abort loading of package, if nothing to do.
%   \end{Version}
%   \begin{Version}{1999/04/13 v1.1}
%   \item
%     \cs{nofiles} bug fix removed
%     because of \xpackage{hyperref} 6.55.
%   \item
%     First CTAN release.
%   \end{Version}
%   \begin{Version}{2003/06/05 v1.2}
%   \item
%     \cs{deadcyles} is decremented for omitted pages.
%   \item
%     LPPL 1.2.
%   \end{Version}
%   \begin{Version}{2006/02/20 v1.3}
%   \item
%     Code is not changed.
%   \item
%     New DTX framework.
%   \item
%     LPPL 1.3
%   \end{Version}
%   \begin{Version}{2006/03/02 v1.4}
%   \item
%     Support for \cs{AtBeginDvi} added.
%   \end{Version}
%   \begin{Version}{2006/03/07 v1.5}
%   \item
%     Job is aborted after last selected page.
%   \end{Version}
%   \begin{Version}{2007/04/11 v1.6}
%   \item
%     Line ends sanitized.
%   \end{Version}
%   \begin{Version}{2007/04/12 v1.7}
%   \item
%     Hard coded box number 255 replaced by macro \cs{@cclv}.
%   \end{Version}
%   \begin{Version}{2008/08/11 v1.8}
%   \item
%     Code is not changed.
%   \item
%     URL updated from \texttt{www.dejanews.com}
%     to \texttt{groups.google.com}.
%   \end{Version}
%   \begin{Version}{2016/05/16 v1.9}
%   \item
%     Documentation updates.
%   \end{Version}
%   \begin{Version}{2020-08-03 v1.10}
%   \item Updated to follow the changes in the hook management
%   of LaTeX 2020/10/01
%   \end{Version}
% \end{History}
%
% \PrintIndex
%
% \Finale
\endinput

%        (quote the arguments according to the demands of your shell)
%
% Documentation:
%    (a) If pagesel.drv is present:
%           latex pagesel.drv
%    (b) Without pagesel.drv:
%           latex pagesel.dtx; ...
%    The class ltxdoc loads the configuration file ltxdoc.cfg
%    if available. Here you can specify further options, e.g.
%    use A4 as paper format:
%       \PassOptionsToClass{a4paper}{article}
%
%    Programm calls to get the documentation (example):
%       pdflatex pagesel.dtx
%       makeindex -s gind.ist pagesel.idx
%       pdflatex pagesel.dtx
%       makeindex -s gind.ist pagesel.idx
%       pdflatex pagesel.dtx
%
% Installation:
%    TDS:tex/latex/pagesel/pagesel.sty
%    TDS:doc/latex/pagesel/pagesel.pdf
%    TDS:source/latex/pagesel/pagesel.dtx
%
%<*ignore>
\begingroup
  \catcode123=1 %
  \catcode125=2 %
  \def\x{LaTeX2e}%
\expandafter\endgroup
\ifcase 0\ifx\install y1\fi\expandafter
         \ifx\csname processbatchFile\endcsname\relax\else1\fi
         \ifx\fmtname\x\else 1\fi\relax
\else\csname fi\endcsname
%</ignore>
%<*install>
\input docstrip.tex
\Msg{************************************************************************}
\Msg{* Installation}
\Msg{* Package: pagesel 2020-08-03 v1.10 Select pages of a document for output (HO)}
\Msg{************************************************************************}

\keepsilent
\askforoverwritefalse

\let\MetaPrefix\relax
\preamble

This is a generated file.

Project: pagesel
Version: 2020-08-03 v1.10

Copyright (C)
   1999, 2003, 2006-2008 Heiko Oberdiek
   2016-2020 Oberdiek Package Support Group

This work may be distributed and/or modified under the
conditions of the LaTeX Project Public License, either
version 1.3c of this license or (at your option) any later
version. This version of this license is in
   https://www.latex-project.org/lppl/lppl-1-3c.txt
and the latest version of this license is in
   https://www.latex-project.org/lppl.txt
and version 1.3 or later is part of all distributions of
LaTeX version 2005/12/01 or later.

This work has the LPPL maintenance status "maintained".

The Current Maintainers of this work are
Heiko Oberdiek and the Oberdiek Package Support Group
https://github.com/ho-tex/pagesel/issues


This work consists of the main source file pagesel.dtx
and the derived files
   pagesel.sty, pagesel-2016-05-16.sty, pagesel.pdf,
   pagesel.ins, pagesel.drv.

\endpreamble
\let\MetaPrefix\DoubleperCent

\generate{%
  \file{pagesel.ins}{\from{pagesel.dtx}{install}}%
  \file{pagesel.drv}{\from{pagesel.dtx}{driver}}%
  \usedir{tex/latex/pagesel}%
  \file{pagesel.sty}{\from{pagesel.dtx}{package}}%
  \file{pagesel-2016-05-16.sty}{\from{pagesel.dtx}{packagefrozen}}
}

\catcode32=13\relax% active space
\let =\space%
\Msg{************************************************************************}
\Msg{*}
\Msg{* To finish the installation you have to move the following}
\Msg{* file into a directory searched by TeX:}
\Msg{*}
\Msg{*     pagesel.sty}
\Msg{*}
\Msg{* To produce the documentation run the file `pagesel.drv'}
\Msg{* through LaTeX.}
\Msg{*}
\Msg{* Happy TeXing!}
\Msg{*}
\Msg{************************************************************************}

\endbatchfile
%</install>
%<*ignore>
\fi
%</ignore>
%<*driver>
\NeedsTeXFormat{LaTeX2e}
\ProvidesFile{pagesel.drv}%
  [2020-08-03 v1.10 Select pages of a document for output (HO)]%
\documentclass{ltxdoc}
\usepackage{holtxdoc}[2011/11/22]
\begin{document}
  \DocInput{pagesel.dtx}%
\end{document}
%</driver>
% \fi
%
%
%
% \GetFileInfo{pagesel.drv}
%
% \title{The \xpackage{pagesel} package}
% \date{2020-08-03 v1.10}
% \author{Heiko Oberdiek\thanks
% {Please report any issues at \url{https://github.com/ho-tex/pagesel/issues}}}
%
% \maketitle
%
% \begin{abstract}
% Single pages or page areas can be selected for output.
% \end{abstract}
%
% \tableofcontents
%
% \newenvironment{param}{^^A
%   \newcommand{\entry}[1]{\meta{\###1}:&}^^A
%   \begin{tabular}[t]{@{}l@{ }l@{}}^^A
% }{^^A
%   \end{tabular}^^A
% }
%
% \newcommand*{\Option}[1]{\textsf{#1}}
%
% \section{Usage}
%    The package \Package{pagesel} is a \LaTeXe\ package:
%    \begin{quote}
%      |\usepackage|\oarg{options}|{pagesel}|
%    \end{quote}
%    (For plain\TeX\ and \LaTeX\,2.09 the similar package
%    \URL{\Package{selectp}}^^A
%    {https://ctan.org/pkg/selectp}
%    from \NameEmail{Donald Arsenau}{asnd@triumf.ca} can be used.)
%
%    Depending on the options the package works in two modes:
%    \begin{enumerate}
%    \item If no page selecting option is present, so the package
%          ignores the other options and finishes itself. So no
%          page will be suppressed by the package and auxiliary files
%          will be written.
%    \item With at least one page selecting option the specified
%          pages are selected and the other are suppressed.
%          The default for this mode is that auxiliary will not be
%          overwritten. (This can be changed by an option.)
%    \end{enumerate}
%
% \subsection{Page selecting}
%    The package \Package{pagesel} sets up a new counter that is
%    incremented by each \cmd{\shipout}.
%    In this way the package counts the output pages regardless the value
%    of the page counter. So each page can individually by addressed,
%    even if there are several pages with the same page number.
%
% \subsubsection{Options\texorpdfstring{ for selecting pages}{}}
%    \begin{description}
%    \item[\Option{odd}:] The output pages must have an odd number.
%         All even output pages are suppressed. If there are no
%         page areas specified so all odd pages are print. With
%         page areas only the odd pages in this areas are selected.
%    \item[\Option{even}:] The opposite of option \Option{odd}.
%    \item[Page area:] A page area consists of three elements:
%         the starting output page number, an ``area'' hyphen, and
%         the output page number of the last page in this area.
%         Each component is optional, so there are four kinds
%         to spezify a page area:
%         \begin{description}
%         \item[\meta{m}\Option{-}\meta{n}:] All pages between
%              \meta{m} and \meta{n} inclusive.
%         \item[\Option{-}\meta{n}:] All pages until \meta{n} inclusive.
%         \item[\meta{m}\Option{-}:] The page area starts with \meta{m}
%              and all pages to the end of document are selected.
%         \item[\Option{-}:] All pages (not very useful).
%         \item[\meta{s}:] The single page \meta{s}.
%         \end{description}
%    \end{description}
%
% \subsubsection{Examples}
%    \newcommand*{\exam}[1]{\texttt{\strut[#1]}}^^A hash-ok
%    \begin{tabular}{ll}
%      Options & Output pages\\
%      \hline
%      \exam{1, 4, 9}&  1, 4, and 9\\
%      \exam{7-10, 3}&  3, 7, 8, 9, and 10\\
%      \exam{odd, 3-6}& 3, and 5\\
%      \exam{-4, 3, even, 7-8}& 2, 4, and 8\\
%    \end{tabular}
%
% \subsection{Auxiliary files}
%    If a page is suppressed, the \cmd{\write} commands are not
%    performed. Labels, index entries, or entries for the
%    table of contents aren't written. So it is likely that
%    the table of contents, registers, and lists are incomplete.
% \subsubsection{Options\texorpdfstring{ for handling auxiliary files}{}}
%    \begin{description}
%    \item[\Option{nofiles}:] This is the default. Auxiliary files are
%         read but not written or changed. Also the job is aborted
%         after the last selected page for saving time.
%    \item[\Option{nonofiles}/\Option{files}:] Auxiliary files are
%         written.
%    \end{description}
% \subsubsection{\texorpdfstring{Package }{}\Package{hyperref}}
%    In old versions of \Package{hyperref} [1999/04/12 v6.55] (and below)
%    there is a bug with \cmd{\nofiles}:
%    \begin{itemize}
%    \item Some ``garbage'' appears on terminal and in the log file.
%          This is harmless and can be ignored.
%    \item The outline auxiliary file \cmd{\jobname.out}, however,
%          is opened and truncated to zero bytes.
%          Version 1.0 of this package had
%          loaded a patch file \File{hypnofil.tex}, if it detects
%          \Package{hyperref} to get \cmd{\nofiles} work.
%
%          With the new version of \Package{hyperref} [1999/04/13 v6.56]
%          \cmd{\nofiles} works now. Therefore the workaround code
%          is no longer needed and removed.
%    \end{itemize}
%
% \StopEventually{
% }
%
% \section{Implementation}
%    \begin{macrocode}
%<*package>
%    \end{macrocode}
% \subsection{New implementation using the LaTeX kernel hooks}
%    \begin{macrocode}
\NeedsTeXFormat{LaTeX2e}
\ProvidesPackage{pagesel}
  [2020-08-03 v1.10 Select pages of a document for output (HO)]%
%    \end{macrocode}
%    \begin{macrocode}
\providecommand\IfFormatAtLeastTF{\@ifl@t@r\fmtversion}
\IfFormatAtLeastTF{2020/10/01}{}{%%
%% This is file `pagesel-2016-05-16.sty',
%% generated with the docstrip utility.
%%
%% The original source files were:
%%
%% pagesel.dtx  (with options: `packagefrozen')
%% 
%% This is a generated file.
%% 
%% Project: pagesel
%% Version: 2020-08-03 v1.10
%% 
%% Copyright (C)
%%    1999, 2003, 2006-2008 Heiko Oberdiek
%%    2016-2020 Oberdiek Package Support Group
%% 
%% This work may be distributed and/or modified under the
%% conditions of the LaTeX Project Public License, either
%% version 1.3c of this license or (at your option) any later
%% version. This version of this license is in
%%    https://www.latex-project.org/lppl/lppl-1-3c.txt
%% and the latest version of this license is in
%%    https://www.latex-project.org/lppl.txt
%% and version 1.3 or later is part of all distributions of
%% LaTeX version 2005/12/01 or later.
%% 
%% This work has the LPPL maintenance status "maintained".
%% 
%% The Current Maintainers of this work are
%% Heiko Oberdiek and the Oberdiek Package Support Group
%% https://github.com/ho-tex/pagesel/issues
%% 
%% This work consists of the main source file pagesel.dtx
%% and the derived files
%%    pagesel.sty, pagesel-2016-05-16.sty, pagesel.pdf,
%%    pagesel.ins, pagesel.drv.
%% 
\NeedsTeXFormat{LaTeX2e}
\ProvidesPackage{pagesel}
  [2020-08-03 v1.10 Select pages of a document for output (legacy code) (HO)]%
\@ifundefined{ps@makevoid}{}{%
  \PackageWarningNoLine{pagesel}{Package already loaded.}%
  \endinput
}
\newcommand*{\ps@makevoid}{%
  \global\setbox\@cclv\copy\voidb@x
  \begingroup
    \count@=\deadcycles
    \advance\count@ by -1\relax
    \deadcycles=\count@
  \endgroup
}
\newcommand*\ps@oddpages{0}
\DeclareOption{odd}{\renewcommand*\ps@oddpages{1}}
\DeclareOption{even}{\renewcommand*\ps@oddpages{2}}
\DeclareOption{nofiles}{\let\ps@nofiles\nofiles}
\DeclareOption{nonofiles}{\let\ps@nofiles\@empty}
\DeclareOption{files}{\let\ps@nofiles\@empty}
\ExecuteOptions{nofiles}
\DeclareOption*{%
  \begingroup
    \expandafter\ps@checkoption\CurrentOption-\END
    \edef\x{\endgroup\noexpand\ps@store{\ps@first}{\ps@last}}%
  \x
}
\newcommand\ps@checkoption{}
\def\ps@checkoption#1-#2\END{%
  \ifx\\#2\\%
    \ifx\\#1\\%
      % empty option
      \def\ps@first{\maxdimen}%
      \def\ps@last{\maxdimen}%
    \else
      \edef\ps@first{#1}%
      \edef\ps@last{#1}%
    \fi
  \else
    \ifx\\#1\\%
      \def\ps@first{-\maxdimen}%
    \else
      \edef\ps@first{#1}%
    \fi
    \ps@checklast#2%
  \fi
}
\newcommand\ps@checklast{}
\def\ps@checklast#1-{%
  \ifx\\#1\\%
    \def\ps@last{\maxdimen}%
  \else
    \edef\ps@last{#1}%
  \fi
}
\newcommand*{\ps@store}[2]{%
  \expandafter\def\expandafter\ps@testlist\expandafter{%
    \ps@testlist\ps@pagetest{#1}{#2}%
  }%
}
\newcommand*\ps@testlist{}
\ProcessOptions
\begingroup
  \edef\x{%
    \ifnum\ps@oddpages>0 \relax\fi
    \ifx\ps@testlist\@empty\else\relax\fi
  }%
  \ifx\x\@empty
    \endgroup
    \PackageInfo{pagesel}{Nothing to do}%
    \expandafter\endinput
  \fi
\endgroup
\RequirePackage{everyshi}
\ps@nofiles
\newcounter{ps@count}
\setcounter{ps@count}{0}
\long\def\ps@ReturnAfterElseFi#1\else#2\fi{\fi#1}
\long\def\ps@ReturnAfterFi#1\fi{\fi#1}
\newcommand{\ps@lastpage}{\maxdimen}
\ifx\ps@nofiles\nofiles
  \ifx\ps@testlist\@empty
  \else
    \def\ps@lastpage{0}%
    \newcommand*{\ps@pagetest}[2]{%
      \ifnum#2>\ps@lastpage\relax
        \def\ps@lastpage{#2}%
      \fi
    }%
    \ps@testlist
    \let\ps@pagetest\relax
  \fi
\fi
\newcommand*{\ps@ifinset}[4]{%
  \ifnum#1>\value{ps@count}%
    \ps@ReturnAfterElseFi{#4}%
  \else
    \ps@ReturnAfterFi{%
      \ifnum#2<\value{ps@count}%
        \ps@ReturnAfterElseFi{#4}%
      \else
        \ps@ReturnAfterFi{#3}%
      \fi
    }%
  \fi
}
\newcommand*{\ps@pagetest}[2]{%
  \ps@ifinset{#1}{#2}{\let\ps@next\@empty}{}%
}
\EveryShipout{%
  \stepcounter{ps@count}%
  \ifnum\value{ps@count}>\ps@lastpage\relax
    \global\output{%
      \ps@cleanup@if
      \ps@group@message
      \typeout{%
        Package pagesel Notice: Aborting LaTeX job %
        after last selected page (\ps@lastpage).%
      }%
      \ps@message@ignore
      \global\setbox\@cclv\box\voidb@x
      \deadcycles0\relax
      \aftergroup\@@end
    }%
  \fi
  \let\ps@next\@empty
  \ifx\ps@testlist\@empty
  \else
    \let\ps@next\ps@makevoid
    \ps@testlist
  \fi
  \ifnum\ps@oddpages=1 %
    \ifodd\value{ps@count}%
    \else
    \let\ps@next\ps@makevoid
    \fi
  \fi
  \ifnum\ps@oddpages=2 %
    \ifodd\value{ps@count}%
    \let\ps@next\ps@makevoid
    \else
    \fi
  \fi
  \ps@begindvi
  \ps@next
}
\begingroup\expandafter\expandafter\expandafter\endgroup
\expandafter\ifx\csname currentiflevel\endcsname\relax
  \let\ps@cleanup@if\@empty
\else
  \def\ps@cleanup@if{%
    \ifnum\currentiflevel>\@ne
      \csname fi\endcsname
      \expandafter\ps@cleanup@if
    \fi
  }%
\fi
 \begingroup\expandafter\expandafter\expandafter\endgroup
 \expandafter\ifx\csname currentgrouplevel\endcsname\relax
  \let\ps@group@message\@empty
  \def\ps@message@ignore{%
    \typeout{%
      (pagesel) \space\space\@spaces\@spaces\@spaces
      Messages (\string\end\space occurred ...) can be ignored.%
    }%
  }%
\else
  \def\ps@group@message{%
    \ifnum\currentgrouplevel>\@ne
      \def\ps@message@ignore{%
        \typeout{%
          (pagesel) \space\space\@spaces\@spaces\@spaces
          Message (\string\end\space occurred ...) %
          can be ignored.%
        }%
      }%
    \else
      \let\ps@message@ignore\@empty
    \fi
  }%
\fi
\newbox\ps@begindvibox
\ifvoid\@begindvibox
\else
  \global\setbox\ps@begindvibox\vbox{%
    \unvbox\@begindvibox
  }%
\fi
\let\ps@org@AtBeginDvi\AtBeginDvi
\def\AtBeginDvi#1{%
  \global\setbox\ps@begindvibox\vbox{%
    \unvbox\ps@begindvibox
    #1%
  }%
  \ps@org@AtBeginDvi{#1}%
}
\def\ps@begindvi{%
  \ifx\ps@next\@empty
    \global\let\ps@begindvi\@empty
  \else
    \global\let\ps@begindvi\ps@begindvi@do
  \fi
}
\def\ps@begindvi@do{%
  \ifx\ps@next\@empty
    \setbox\@cclv\vbox{%
      \unvbox\ps@begindvibox
      \box\@cclv
    }%
    \global\let\ps@begindvi\@empty
  \fi
}
\endinput
%%
%% End of file `pagesel-2016-05-16.sty'.
}
\IfFormatAtLeastTF{2020/10/01}{}{\endinput}

%    \end{macrocode}
%    If the package is loaded twice, the package code does not
%    work. So stop loading the package, if it is already loaded.
%    \begin{macrocode}
\@ifundefined{ps@oddpages}{}{%
  \PackageWarningNoLine{pagesel}{Package already loaded.}%
  \endinput
}
%    \end{macrocode}
%    \begin{macrocode}
%</package>
%    \end{macrocode}
% \subsection{Package}
%    \begin{macrocode}
%<*packagefrozen>
\NeedsTeXFormat{LaTeX2e}
\ProvidesPackage{pagesel}
  [2020-08-03 v1.10 Select pages of a document for output (legacy code) (HO)]%
%    \end{macrocode}
%
%    If the package is loaded twice, the package code does not
%    work. So stop loading the package, if it is already loaded.
%    \begin{macrocode}
\@ifundefined{ps@makevoid}{}{%
  \PackageWarningNoLine{pagesel}{Package already loaded.}%
  \endinput
}
%    \end{macrocode}
%
%    \begin{macro}{\ps@makevoid}
%    Macro \cmd{\ps@makevoid} clears the output box. Because
%    nothing is shipped out and this is intended, we reduce
%    the counter \cmd{\deadcycles} in order to avoid problems, if
%    more than \cmd{\maxdeadcycles} pages are omitted.
%    \begin{macrocode}
\newcommand*{\ps@makevoid}{%
  \global\setbox\@cclv\copy\voidb@x
  \begingroup
    \count@=\deadcycles
    \advance\count@ by -1\relax
    \deadcycles=\count@
  \endgroup
}
%</packagefrozen>
%    \end{macrocode}
%    \end{macro}
%
%    \begin{macro}{\ps@oddpages}
%    \begin{macrocode}
%<*package|packagefrozen>
\newcommand*\ps@oddpages{0}
\DeclareOption{odd}{\renewcommand*\ps@oddpages{1}}
\DeclareOption{even}{\renewcommand*\ps@oddpages{2}}
%    \end{macrocode}
%    \end{macro}
%
%    \begin{macrocode}
\DeclareOption{nofiles}{\let\ps@nofiles\nofiles}
\DeclareOption{nonofiles}{\let\ps@nofiles\@empty}
\DeclareOption{files}{\let\ps@nofiles\@empty}
\ExecuteOptions{nofiles}
%    \end{macrocode}
%
%    \begin{macrocode}
\DeclareOption*{%
  \begingroup
    \expandafter\ps@checkoption\CurrentOption-\END
    \edef\x{\endgroup\noexpand\ps@store{\ps@first}{\ps@last}}%
  \x
}
%    \end{macrocode}
%
%    \begin{macro}{\ps@checkoption}
%    \begin{macrocode}
\newcommand\ps@checkoption{}
\def\ps@checkoption#1-#2\END{%
  \ifx\\#2\\%
    \ifx\\#1\\%
      % empty option
      \def\ps@first{\maxdimen}%
      \def\ps@last{\maxdimen}%
    \else
      \edef\ps@first{#1}%
      \edef\ps@last{#1}%
    \fi
  \else
    \ifx\\#1\\%
      \def\ps@first{-\maxdimen}%
    \else
      \edef\ps@first{#1}%
    \fi
    \ps@checklast#2%
  \fi
}
%    \end{macrocode}
%    \end{macro}
%
%    \begin{macro}{\ps@checklast}
%    \begin{macrocode}
\newcommand\ps@checklast{}
\def\ps@checklast#1-{%
  \ifx\\#1\\%
    \def\ps@last{\maxdimen}%
  \else
    \edef\ps@last{#1}%
  \fi
}
%    \end{macrocode}
%    \end{macro}
%
%    \begin{macro}{\ps@store}
%    \begin{macrocode}
\newcommand*{\ps@store}[2]{%
  \expandafter\def\expandafter\ps@testlist\expandafter{%
    \ps@testlist\ps@pagetest{#1}{#2}%
  }%
}
%    \end{macrocode}
%    \end{macro}
%
%    \begin{macro}{\ps@testlist}
%    \begin{macrocode}
\newcommand*\ps@testlist{}
%    \end{macrocode}
%    \end{macro}
%
%    \begin{macrocode}
\ProcessOptions
%    \end{macrocode}
%
%    \begin{macrocode}
\begingroup
  \edef\x{%
    \ifnum\ps@oddpages>0 \relax\fi
    \ifx\ps@testlist\@empty\else\relax\fi
  }%
  \ifx\x\@empty
    \endgroup
    \PackageInfo{pagesel}{Nothing to do}%
    \expandafter\endinput
  \fi
\endgroup
%    \end{macrocode}
%
%    \begin{macrocode}
%</package|packagefrozen>
%<*packagefrozen>
\RequirePackage{everyshi}
%</packagefrozen>
%    \end{macrocode}
%
%    \begin{macrocode}
%<*package|packagefrozen>
\ps@nofiles
%    \end{macrocode}
%
%    \begin{macro}{\c@ps@count}
%    \begin{macrocode}
\newcounter{ps@count}
\setcounter{ps@count}{0}
%    \end{macrocode}
%    \end{macro}
%
%    \begin{macro}{\ps@ReturnAfterElseFi}
%    \begin{macro}{\ps@ReturnAfterFi}
%    \begin{macrocode}
\long\def\ps@ReturnAfterElseFi#1\else#2\fi{\fi#1}
\long\def\ps@ReturnAfterFi#1\fi{\fi#1}
%    \end{macrocode}
%    \end{macro}
%    \end{macro}
%
%    \begin{macrocode}
\newcommand{\ps@lastpage}{\maxdimen}
\ifx\ps@nofiles\nofiles
  \ifx\ps@testlist\@empty
  \else
    \def\ps@lastpage{0}%
    \newcommand*{\ps@pagetest}[2]{%
      \ifnum#2>\ps@lastpage\relax
        \def\ps@lastpage{#2}%
      \fi
    }%
    \ps@testlist
    \let\ps@pagetest\relax
  \fi
\fi
%    \end{macrocode}
%
%    \begin{macro}{\ps@ifinset}
%    \begin{macrocode}
\newcommand*{\ps@ifinset}[4]{%
  \ifnum#1>\value{ps@count}%
    \ps@ReturnAfterElseFi{#4}%
  \else
    \ps@ReturnAfterFi{%
      \ifnum#2<\value{ps@count}%
        \ps@ReturnAfterElseFi{#4}%
      \else
        \ps@ReturnAfterFi{#3}%
      \fi
    }%
  \fi
}
%    \end{macrocode}
%    \end{macro}
%
%    \begin{macro}{\ps@pagetest}
%    \begin{macrocode}
\newcommand*{\ps@pagetest}[2]{%
  \ps@ifinset{#1}{#2}{\let\ps@next\@empty}{}%
}
%    \end{macrocode}
%    \end{macro}
%
%    \begin{macrocode}
%</package|packagefrozen>
%<packagefrozen>\EveryShipout{%
%<package>\AddToHook{shipout/before}{%
%<*package|packagefrozen>
  \stepcounter{ps@count}%
  \ifnum\value{ps@count}>\ps@lastpage\relax
    \global\output{%
      \ps@cleanup@if
      \ps@group@message
      \typeout{%
        Package pagesel Notice: Aborting LaTeX job %
        after last selected page (\ps@lastpage).%
      }%
      \ps@message@ignore
      \global\setbox\@cclv\box\voidb@x
      \deadcycles0\relax
%    \end{macrocode}
%    First leave the output group before ending the job.
%    \begin{macrocode}
      \aftergroup\@@end
    }%
  \fi
  \let\ps@next\@empty
  \ifx\ps@testlist\@empty
  \else
%<packagefrozen>    \let\ps@next\ps@makevoid
%<package>    \let\ps@next\DiscardShipoutBox
    \ps@testlist
  \fi
  \ifnum\ps@oddpages=1 %
    \ifodd\value{ps@count}%
    \else
%<packagefrozen>    \let\ps@next\ps@makevoid
%<package>    \let\ps@next\DiscardShipoutBox
    \fi
  \fi
  \ifnum\ps@oddpages=2 %
    \ifodd\value{ps@count}%
%<packagefrozen>    \let\ps@next\ps@makevoid
%<package>    \let\ps@next\DiscardShipoutBox
    \else
    \fi
  \fi
%<packagefrozen>  \ps@begindvi
  \ps@next
}
%</package|packagefrozen>
%    \end{macrocode}
%
%    \begin{macrocode}
%<*package|packagefrozen>
%<packagefrozen>\begingroup\expandafter\expandafter\expandafter\endgroup
%<packagefrozen>\expandafter\ifx\csname currentiflevel\endcsname\relax
%<packagefrozen>  \let\ps@cleanup@if\@empty
%<packagefrozen>\else
  \def\ps@cleanup@if{%
    \ifnum\currentiflevel>\@ne
      \csname fi\endcsname
      \expandafter\ps@cleanup@if
    \fi
  }%
%<packagefrozen>\fi
%    \end{macrocode}
%    Because of \cs{aftergroup} it is too dangerous to perform
%    a similar cleanup for groups.
%    \begin{macrocode}
%<packagefrozen> \begingroup\expandafter\expandafter\expandafter\endgroup
%<packagefrozen> \expandafter\ifx\csname currentgrouplevel\endcsname\relax
%<packagefrozen>  \let\ps@group@message\@empty
%<packagefrozen>  \def\ps@message@ignore{%
%<packagefrozen>    \typeout{%
%<packagefrozen>      (pagesel) \space\space\@spaces\@spaces\@spaces
%<packagefrozen>      Messages (\string\end\space occurred ...) can be ignored.%
%<packagefrozen>    }%
%<packagefrozen>  }%
%<packagefrozen>\else
  \def\ps@group@message{%
    \ifnum\currentgrouplevel>\@ne
      \def\ps@message@ignore{%
        \typeout{%
          (pagesel) \space\space\@spaces\@spaces\@spaces
          Message (\string\end\space occurred ...) %
          can be ignored.%
        }%
      }%
    \else
      \let\ps@message@ignore\@empty
    \fi
  }%
%<packagefrozen>\fi
%</package|packagefrozen>
%    \end{macrocode}
%
% \subsection{AtBeginDvi hook support}
%
%    The material of box \cs{@begindvibox} is recorded in parallel
%    in box \cs{ps@begindvibox}.
%    \begin{macrocode}
%<*packagefrozen>
\newbox\ps@begindvibox
\ifvoid\@begindvibox
\else
  \global\setbox\ps@begindvibox\vbox{%
    \unvbox\@begindvibox
  }%
\fi
\let\ps@org@AtBeginDvi\AtBeginDvi
\def\AtBeginDvi#1{%
  \global\setbox\ps@begindvibox\vbox{%
    \unvbox\ps@begindvibox
    #1%
  }%
  \ps@org@AtBeginDvi{#1}%
}
%    \end{macrocode}
%
%    \begin{macro}{\ps@begindvi}
%    Macro \cs{ps@begindvi} is called the similar way as \cs{@begindvi}.
%    If the first page is printed, then \cs{AtBeginDvi} should work
%    as usual. Otherwise the contents of box \cs{ps@begindvibox} is
%    set on the first selected page.
%    \begin{macrocode}
\def\ps@begindvi{%
  \ifx\ps@next\@empty
    \global\let\ps@begindvi\@empty
  \else
    \global\let\ps@begindvi\ps@begindvi@do
  \fi
}
\def\ps@begindvi@do{%
  \ifx\ps@next\@empty
    \setbox\@cclv\vbox{%
      \unvbox\ps@begindvibox
      \box\@cclv
    }%
    \global\let\ps@begindvi\@empty
  \fi
}
%    \end{macrocode}
%    \end{macro}
%
%    \begin{macrocode}
%</packagefrozen>
%    \end{macrocode}
%
% \section{Installation}
%
% \subsection{Download}
%
% \paragraph{Package.} This package is available on
% CTAN\footnote{\CTANpkg{pagesel}}:
% \begin{description}
% \item[\CTAN{macros/latex/contrib/pagesel/pagesel.dtx}] The source file.
% \item[\CTAN{macros/latex/contrib/pagesel/pagesel.pdf}] Documentation.
% \end{description}
%
%
%
% \subsection{Package installation}
%
% \paragraph{Unpacking.} The \xfile{.dtx} file is a self-extracting
% \docstrip\ archive. The files are extracted by running the
% \xfile{.dtx} through \plainTeX:
% \begin{quote}
%   \verb|tex pagesel.dtx|
% \end{quote}
%
% \paragraph{TDS.} Now the different files must be moved into
% the different directories in your installation TDS tree
% (also known as \xfile{texmf} tree):
% \begin{quote}
% \def\t{^^A
% \begin{tabular}{@{}>{\ttfamily}l@{ $\rightarrow$ }>{\ttfamily}l@{}}
%   pagesel.sty & tex/latex/pagesel/pagesel.sty\\
%   pagesel.pdf & doc/latex/pagesel/pagesel.pdf\\
%   pagesel.dtx & source/latex/pagesel/pagesel.dtx\\
% \end{tabular}^^A
% }^^A
% \sbox0{\t}^^A
% \ifdim\wd0>\linewidth
%   \begingroup
%     \advance\linewidth by\leftmargin
%     \advance\linewidth by\rightmargin
%   \edef\x{\endgroup
%     \def\noexpand\lw{\the\linewidth}^^A
%   }\x
%   \def\lwbox{^^A
%     \leavevmode
%     \hbox to \linewidth{^^A
%       \kern-\leftmargin\relax
%       \hss
%       \usebox0
%       \hss
%       \kern-\rightmargin\relax
%     }^^A
%   }^^A
%   \ifdim\wd0>\lw
%     \sbox0{\small\t}^^A
%     \ifdim\wd0>\linewidth
%       \ifdim\wd0>\lw
%         \sbox0{\footnotesize\t}^^A
%         \ifdim\wd0>\linewidth
%           \ifdim\wd0>\lw
%             \sbox0{\scriptsize\t}^^A
%             \ifdim\wd0>\linewidth
%               \ifdim\wd0>\lw
%                 \sbox0{\tiny\t}^^A
%                 \ifdim\wd0>\linewidth
%                   \lwbox
%                 \else
%                   \usebox0
%                 \fi
%               \else
%                 \lwbox
%               \fi
%             \else
%               \usebox0
%             \fi
%           \else
%             \lwbox
%           \fi
%         \else
%           \usebox0
%         \fi
%       \else
%         \lwbox
%       \fi
%     \else
%       \usebox0
%     \fi
%   \else
%     \lwbox
%   \fi
% \else
%   \usebox0
% \fi
% \end{quote}
% If you have a \xfile{docstrip.cfg} that configures and enables \docstrip's
% TDS installing feature, then some files can already be in the right
% place, see the documentation of \docstrip.
%
% \subsection{Refresh file name databases}
%
% If your \TeX~distribution
% (\TeX\,Live, \mikTeX, \dots) relies on file name databases, you must refresh
% these. For example, \TeX\,Live\ users run \verb|texhash| or
% \verb|mktexlsr|.
%
% \subsection{Some details for the interested}
%
% \paragraph{Unpacking with \LaTeX.}
% The \xfile{.dtx} chooses its action depending on the format:
% \begin{description}
% \item[\plainTeX:] Run \docstrip\ and extract the files.
% \item[\LaTeX:] Generate the documentation.
% \end{description}
% If you insist on using \LaTeX\ for \docstrip\ (really,
% \docstrip\ does not need \LaTeX), then inform the autodetect routine
% about your intention:
% \begin{quote}
%   \verb|latex \let\install=y% \iffalse meta-comment
%
% File: pagesel.dtx
% Version: 2020-08-03 v1.10
% Info: Select pages of a document for output
%
% Copyright (C)
%    1999, 2003, 2006-2008 Heiko Oberdiek
%    2016-2020 Oberdiek Package Support Group
%    https://github.com/ho-tex/pagesel/issues
%
% This work may be distributed and/or modified under the
% conditions of the LaTeX Project Public License, either
% version 1.3c of this license or (at your option) any later
% version. This version of this license is in
%    https://www.latex-project.org/lppl/lppl-1-3c.txt
% and the latest version of this license is in
%    https://www.latex-project.org/lppl.txt
% and version 1.3 or later is part of all distributions of
% LaTeX version 2005/12/01 or later.
%
% This work has the LPPL maintenance status "maintained".
%
% The Current Maintainers of this work are
% Heiko Oberdiek and the Oberdiek Package Support Group
% https://github.com/ho-tex/pagesel/issues
%
% This work consists of the main source file pagesel.dtx
% and the derived files
%    pagesel.sty, pagesel-2016-05-16.sty,
%    pagesel.pdf, pagesel.ins, pagesel.drv.
%
% Distribution:
%    CTAN:macros/latex/contrib/pagesel/pagesel.dtx
%    CTAN:macros/latex/contrib/pagesel/pagesel.pdf
%
% Unpacking:
%    (a) If pagesel.ins is present:
%           tex pagesel.ins
%    (b) Without pagesel.ins:
%           tex pagesel.dtx
%    (c) If you insist on using LaTeX
%           latex \let\install=y\input{pagesel.dtx}
%        (quote the arguments according to the demands of your shell)
%
% Documentation:
%    (a) If pagesel.drv is present:
%           latex pagesel.drv
%    (b) Without pagesel.drv:
%           latex pagesel.dtx; ...
%    The class ltxdoc loads the configuration file ltxdoc.cfg
%    if available. Here you can specify further options, e.g.
%    use A4 as paper format:
%       \PassOptionsToClass{a4paper}{article}
%
%    Programm calls to get the documentation (example):
%       pdflatex pagesel.dtx
%       makeindex -s gind.ist pagesel.idx
%       pdflatex pagesel.dtx
%       makeindex -s gind.ist pagesel.idx
%       pdflatex pagesel.dtx
%
% Installation:
%    TDS:tex/latex/pagesel/pagesel.sty
%    TDS:doc/latex/pagesel/pagesel.pdf
%    TDS:source/latex/pagesel/pagesel.dtx
%
%<*ignore>
\begingroup
  \catcode123=1 %
  \catcode125=2 %
  \def\x{LaTeX2e}%
\expandafter\endgroup
\ifcase 0\ifx\install y1\fi\expandafter
         \ifx\csname processbatchFile\endcsname\relax\else1\fi
         \ifx\fmtname\x\else 1\fi\relax
\else\csname fi\endcsname
%</ignore>
%<*install>
\input docstrip.tex
\Msg{************************************************************************}
\Msg{* Installation}
\Msg{* Package: pagesel 2020-08-03 v1.10 Select pages of a document for output (HO)}
\Msg{************************************************************************}

\keepsilent
\askforoverwritefalse

\let\MetaPrefix\relax
\preamble

This is a generated file.

Project: pagesel
Version: 2020-08-03 v1.10

Copyright (C)
   1999, 2003, 2006-2008 Heiko Oberdiek
   2016-2020 Oberdiek Package Support Group

This work may be distributed and/or modified under the
conditions of the LaTeX Project Public License, either
version 1.3c of this license or (at your option) any later
version. This version of this license is in
   https://www.latex-project.org/lppl/lppl-1-3c.txt
and the latest version of this license is in
   https://www.latex-project.org/lppl.txt
and version 1.3 or later is part of all distributions of
LaTeX version 2005/12/01 or later.

This work has the LPPL maintenance status "maintained".

The Current Maintainers of this work are
Heiko Oberdiek and the Oberdiek Package Support Group
https://github.com/ho-tex/pagesel/issues


This work consists of the main source file pagesel.dtx
and the derived files
   pagesel.sty, pagesel-2016-05-16.sty, pagesel.pdf,
   pagesel.ins, pagesel.drv.

\endpreamble
\let\MetaPrefix\DoubleperCent

\generate{%
  \file{pagesel.ins}{\from{pagesel.dtx}{install}}%
  \file{pagesel.drv}{\from{pagesel.dtx}{driver}}%
  \usedir{tex/latex/pagesel}%
  \file{pagesel.sty}{\from{pagesel.dtx}{package}}%
  \file{pagesel-2016-05-16.sty}{\from{pagesel.dtx}{packagefrozen}}
}

\catcode32=13\relax% active space
\let =\space%
\Msg{************************************************************************}
\Msg{*}
\Msg{* To finish the installation you have to move the following}
\Msg{* file into a directory searched by TeX:}
\Msg{*}
\Msg{*     pagesel.sty}
\Msg{*}
\Msg{* To produce the documentation run the file `pagesel.drv'}
\Msg{* through LaTeX.}
\Msg{*}
\Msg{* Happy TeXing!}
\Msg{*}
\Msg{************************************************************************}

\endbatchfile
%</install>
%<*ignore>
\fi
%</ignore>
%<*driver>
\NeedsTeXFormat{LaTeX2e}
\ProvidesFile{pagesel.drv}%
  [2020-08-03 v1.10 Select pages of a document for output (HO)]%
\documentclass{ltxdoc}
\usepackage{holtxdoc}[2011/11/22]
\begin{document}
  \DocInput{pagesel.dtx}%
\end{document}
%</driver>
% \fi
%
%
%
% \GetFileInfo{pagesel.drv}
%
% \title{The \xpackage{pagesel} package}
% \date{2020-08-03 v1.10}
% \author{Heiko Oberdiek\thanks
% {Please report any issues at \url{https://github.com/ho-tex/pagesel/issues}}}
%
% \maketitle
%
% \begin{abstract}
% Single pages or page areas can be selected for output.
% \end{abstract}
%
% \tableofcontents
%
% \newenvironment{param}{^^A
%   \newcommand{\entry}[1]{\meta{\###1}:&}^^A
%   \begin{tabular}[t]{@{}l@{ }l@{}}^^A
% }{^^A
%   \end{tabular}^^A
% }
%
% \newcommand*{\Option}[1]{\textsf{#1}}
%
% \section{Usage}
%    The package \Package{pagesel} is a \LaTeXe\ package:
%    \begin{quote}
%      |\usepackage|\oarg{options}|{pagesel}|
%    \end{quote}
%    (For plain\TeX\ and \LaTeX\,2.09 the similar package
%    \URL{\Package{selectp}}^^A
%    {https://ctan.org/pkg/selectp}
%    from \NameEmail{Donald Arsenau}{asnd@triumf.ca} can be used.)
%
%    Depending on the options the package works in two modes:
%    \begin{enumerate}
%    \item If no page selecting option is present, so the package
%          ignores the other options and finishes itself. So no
%          page will be suppressed by the package and auxiliary files
%          will be written.
%    \item With at least one page selecting option the specified
%          pages are selected and the other are suppressed.
%          The default for this mode is that auxiliary will not be
%          overwritten. (This can be changed by an option.)
%    \end{enumerate}
%
% \subsection{Page selecting}
%    The package \Package{pagesel} sets up a new counter that is
%    incremented by each \cmd{\shipout}.
%    In this way the package counts the output pages regardless the value
%    of the page counter. So each page can individually by addressed,
%    even if there are several pages with the same page number.
%
% \subsubsection{Options\texorpdfstring{ for selecting pages}{}}
%    \begin{description}
%    \item[\Option{odd}:] The output pages must have an odd number.
%         All even output pages are suppressed. If there are no
%         page areas specified so all odd pages are print. With
%         page areas only the odd pages in this areas are selected.
%    \item[\Option{even}:] The opposite of option \Option{odd}.
%    \item[Page area:] A page area consists of three elements:
%         the starting output page number, an ``area'' hyphen, and
%         the output page number of the last page in this area.
%         Each component is optional, so there are four kinds
%         to spezify a page area:
%         \begin{description}
%         \item[\meta{m}\Option{-}\meta{n}:] All pages between
%              \meta{m} and \meta{n} inclusive.
%         \item[\Option{-}\meta{n}:] All pages until \meta{n} inclusive.
%         \item[\meta{m}\Option{-}:] The page area starts with \meta{m}
%              and all pages to the end of document are selected.
%         \item[\Option{-}:] All pages (not very useful).
%         \item[\meta{s}:] The single page \meta{s}.
%         \end{description}
%    \end{description}
%
% \subsubsection{Examples}
%    \newcommand*{\exam}[1]{\texttt{\strut[#1]}}^^A hash-ok
%    \begin{tabular}{ll}
%      Options & Output pages\\
%      \hline
%      \exam{1, 4, 9}&  1, 4, and 9\\
%      \exam{7-10, 3}&  3, 7, 8, 9, and 10\\
%      \exam{odd, 3-6}& 3, and 5\\
%      \exam{-4, 3, even, 7-8}& 2, 4, and 8\\
%    \end{tabular}
%
% \subsection{Auxiliary files}
%    If a page is suppressed, the \cmd{\write} commands are not
%    performed. Labels, index entries, or entries for the
%    table of contents aren't written. So it is likely that
%    the table of contents, registers, and lists are incomplete.
% \subsubsection{Options\texorpdfstring{ for handling auxiliary files}{}}
%    \begin{description}
%    \item[\Option{nofiles}:] This is the default. Auxiliary files are
%         read but not written or changed. Also the job is aborted
%         after the last selected page for saving time.
%    \item[\Option{nonofiles}/\Option{files}:] Auxiliary files are
%         written.
%    \end{description}
% \subsubsection{\texorpdfstring{Package }{}\Package{hyperref}}
%    In old versions of \Package{hyperref} [1999/04/12 v6.55] (and below)
%    there is a bug with \cmd{\nofiles}:
%    \begin{itemize}
%    \item Some ``garbage'' appears on terminal and in the log file.
%          This is harmless and can be ignored.
%    \item The outline auxiliary file \cmd{\jobname.out}, however,
%          is opened and truncated to zero bytes.
%          Version 1.0 of this package had
%          loaded a patch file \File{hypnofil.tex}, if it detects
%          \Package{hyperref} to get \cmd{\nofiles} work.
%
%          With the new version of \Package{hyperref} [1999/04/13 v6.56]
%          \cmd{\nofiles} works now. Therefore the workaround code
%          is no longer needed and removed.
%    \end{itemize}
%
% \StopEventually{
% }
%
% \section{Implementation}
%    \begin{macrocode}
%<*package>
%    \end{macrocode}
% \subsection{New implementation using the LaTeX kernel hooks}
%    \begin{macrocode}
\NeedsTeXFormat{LaTeX2e}
\ProvidesPackage{pagesel}
  [2020-08-03 v1.10 Select pages of a document for output (HO)]%
%    \end{macrocode}
%    \begin{macrocode}
\providecommand\IfFormatAtLeastTF{\@ifl@t@r\fmtversion}
\IfFormatAtLeastTF{2020/10/01}{}{\input{pagesel-2016-05-16.sty}}
\IfFormatAtLeastTF{2020/10/01}{}{\endinput}

%    \end{macrocode}
%    If the package is loaded twice, the package code does not
%    work. So stop loading the package, if it is already loaded.
%    \begin{macrocode}
\@ifundefined{ps@oddpages}{}{%
  \PackageWarningNoLine{pagesel}{Package already loaded.}%
  \endinput
}
%    \end{macrocode}
%    \begin{macrocode}
%</package>
%    \end{macrocode}
% \subsection{Package}
%    \begin{macrocode}
%<*packagefrozen>
\NeedsTeXFormat{LaTeX2e}
\ProvidesPackage{pagesel}
  [2020-08-03 v1.10 Select pages of a document for output (legacy code) (HO)]%
%    \end{macrocode}
%
%    If the package is loaded twice, the package code does not
%    work. So stop loading the package, if it is already loaded.
%    \begin{macrocode}
\@ifundefined{ps@makevoid}{}{%
  \PackageWarningNoLine{pagesel}{Package already loaded.}%
  \endinput
}
%    \end{macrocode}
%
%    \begin{macro}{\ps@makevoid}
%    Macro \cmd{\ps@makevoid} clears the output box. Because
%    nothing is shipped out and this is intended, we reduce
%    the counter \cmd{\deadcycles} in order to avoid problems, if
%    more than \cmd{\maxdeadcycles} pages are omitted.
%    \begin{macrocode}
\newcommand*{\ps@makevoid}{%
  \global\setbox\@cclv\copy\voidb@x
  \begingroup
    \count@=\deadcycles
    \advance\count@ by -1\relax
    \deadcycles=\count@
  \endgroup
}
%</packagefrozen>
%    \end{macrocode}
%    \end{macro}
%
%    \begin{macro}{\ps@oddpages}
%    \begin{macrocode}
%<*package|packagefrozen>
\newcommand*\ps@oddpages{0}
\DeclareOption{odd}{\renewcommand*\ps@oddpages{1}}
\DeclareOption{even}{\renewcommand*\ps@oddpages{2}}
%    \end{macrocode}
%    \end{macro}
%
%    \begin{macrocode}
\DeclareOption{nofiles}{\let\ps@nofiles\nofiles}
\DeclareOption{nonofiles}{\let\ps@nofiles\@empty}
\DeclareOption{files}{\let\ps@nofiles\@empty}
\ExecuteOptions{nofiles}
%    \end{macrocode}
%
%    \begin{macrocode}
\DeclareOption*{%
  \begingroup
    \expandafter\ps@checkoption\CurrentOption-\END
    \edef\x{\endgroup\noexpand\ps@store{\ps@first}{\ps@last}}%
  \x
}
%    \end{macrocode}
%
%    \begin{macro}{\ps@checkoption}
%    \begin{macrocode}
\newcommand\ps@checkoption{}
\def\ps@checkoption#1-#2\END{%
  \ifx\\#2\\%
    \ifx\\#1\\%
      % empty option
      \def\ps@first{\maxdimen}%
      \def\ps@last{\maxdimen}%
    \else
      \edef\ps@first{#1}%
      \edef\ps@last{#1}%
    \fi
  \else
    \ifx\\#1\\%
      \def\ps@first{-\maxdimen}%
    \else
      \edef\ps@first{#1}%
    \fi
    \ps@checklast#2%
  \fi
}
%    \end{macrocode}
%    \end{macro}
%
%    \begin{macro}{\ps@checklast}
%    \begin{macrocode}
\newcommand\ps@checklast{}
\def\ps@checklast#1-{%
  \ifx\\#1\\%
    \def\ps@last{\maxdimen}%
  \else
    \edef\ps@last{#1}%
  \fi
}
%    \end{macrocode}
%    \end{macro}
%
%    \begin{macro}{\ps@store}
%    \begin{macrocode}
\newcommand*{\ps@store}[2]{%
  \expandafter\def\expandafter\ps@testlist\expandafter{%
    \ps@testlist\ps@pagetest{#1}{#2}%
  }%
}
%    \end{macrocode}
%    \end{macro}
%
%    \begin{macro}{\ps@testlist}
%    \begin{macrocode}
\newcommand*\ps@testlist{}
%    \end{macrocode}
%    \end{macro}
%
%    \begin{macrocode}
\ProcessOptions
%    \end{macrocode}
%
%    \begin{macrocode}
\begingroup
  \edef\x{%
    \ifnum\ps@oddpages>0 \relax\fi
    \ifx\ps@testlist\@empty\else\relax\fi
  }%
  \ifx\x\@empty
    \endgroup
    \PackageInfo{pagesel}{Nothing to do}%
    \expandafter\endinput
  \fi
\endgroup
%    \end{macrocode}
%
%    \begin{macrocode}
%</package|packagefrozen>
%<*packagefrozen>
\RequirePackage{everyshi}
%</packagefrozen>
%    \end{macrocode}
%
%    \begin{macrocode}
%<*package|packagefrozen>
\ps@nofiles
%    \end{macrocode}
%
%    \begin{macro}{\c@ps@count}
%    \begin{macrocode}
\newcounter{ps@count}
\setcounter{ps@count}{0}
%    \end{macrocode}
%    \end{macro}
%
%    \begin{macro}{\ps@ReturnAfterElseFi}
%    \begin{macro}{\ps@ReturnAfterFi}
%    \begin{macrocode}
\long\def\ps@ReturnAfterElseFi#1\else#2\fi{\fi#1}
\long\def\ps@ReturnAfterFi#1\fi{\fi#1}
%    \end{macrocode}
%    \end{macro}
%    \end{macro}
%
%    \begin{macrocode}
\newcommand{\ps@lastpage}{\maxdimen}
\ifx\ps@nofiles\nofiles
  \ifx\ps@testlist\@empty
  \else
    \def\ps@lastpage{0}%
    \newcommand*{\ps@pagetest}[2]{%
      \ifnum#2>\ps@lastpage\relax
        \def\ps@lastpage{#2}%
      \fi
    }%
    \ps@testlist
    \let\ps@pagetest\relax
  \fi
\fi
%    \end{macrocode}
%
%    \begin{macro}{\ps@ifinset}
%    \begin{macrocode}
\newcommand*{\ps@ifinset}[4]{%
  \ifnum#1>\value{ps@count}%
    \ps@ReturnAfterElseFi{#4}%
  \else
    \ps@ReturnAfterFi{%
      \ifnum#2<\value{ps@count}%
        \ps@ReturnAfterElseFi{#4}%
      \else
        \ps@ReturnAfterFi{#3}%
      \fi
    }%
  \fi
}
%    \end{macrocode}
%    \end{macro}
%
%    \begin{macro}{\ps@pagetest}
%    \begin{macrocode}
\newcommand*{\ps@pagetest}[2]{%
  \ps@ifinset{#1}{#2}{\let\ps@next\@empty}{}%
}
%    \end{macrocode}
%    \end{macro}
%
%    \begin{macrocode}
%</package|packagefrozen>
%<packagefrozen>\EveryShipout{%
%<package>\AddToHook{shipout/before}{%
%<*package|packagefrozen>
  \stepcounter{ps@count}%
  \ifnum\value{ps@count}>\ps@lastpage\relax
    \global\output{%
      \ps@cleanup@if
      \ps@group@message
      \typeout{%
        Package pagesel Notice: Aborting LaTeX job %
        after last selected page (\ps@lastpage).%
      }%
      \ps@message@ignore
      \global\setbox\@cclv\box\voidb@x
      \deadcycles0\relax
%    \end{macrocode}
%    First leave the output group before ending the job.
%    \begin{macrocode}
      \aftergroup\@@end
    }%
  \fi
  \let\ps@next\@empty
  \ifx\ps@testlist\@empty
  \else
%<packagefrozen>    \let\ps@next\ps@makevoid
%<package>    \let\ps@next\DiscardShipoutBox
    \ps@testlist
  \fi
  \ifnum\ps@oddpages=1 %
    \ifodd\value{ps@count}%
    \else
%<packagefrozen>    \let\ps@next\ps@makevoid
%<package>    \let\ps@next\DiscardShipoutBox
    \fi
  \fi
  \ifnum\ps@oddpages=2 %
    \ifodd\value{ps@count}%
%<packagefrozen>    \let\ps@next\ps@makevoid
%<package>    \let\ps@next\DiscardShipoutBox
    \else
    \fi
  \fi
%<packagefrozen>  \ps@begindvi
  \ps@next
}
%</package|packagefrozen>
%    \end{macrocode}
%
%    \begin{macrocode}
%<*package|packagefrozen>
%<packagefrozen>\begingroup\expandafter\expandafter\expandafter\endgroup
%<packagefrozen>\expandafter\ifx\csname currentiflevel\endcsname\relax
%<packagefrozen>  \let\ps@cleanup@if\@empty
%<packagefrozen>\else
  \def\ps@cleanup@if{%
    \ifnum\currentiflevel>\@ne
      \csname fi\endcsname
      \expandafter\ps@cleanup@if
    \fi
  }%
%<packagefrozen>\fi
%    \end{macrocode}
%    Because of \cs{aftergroup} it is too dangerous to perform
%    a similar cleanup for groups.
%    \begin{macrocode}
%<packagefrozen> \begingroup\expandafter\expandafter\expandafter\endgroup
%<packagefrozen> \expandafter\ifx\csname currentgrouplevel\endcsname\relax
%<packagefrozen>  \let\ps@group@message\@empty
%<packagefrozen>  \def\ps@message@ignore{%
%<packagefrozen>    \typeout{%
%<packagefrozen>      (pagesel) \space\space\@spaces\@spaces\@spaces
%<packagefrozen>      Messages (\string\end\space occurred ...) can be ignored.%
%<packagefrozen>    }%
%<packagefrozen>  }%
%<packagefrozen>\else
  \def\ps@group@message{%
    \ifnum\currentgrouplevel>\@ne
      \def\ps@message@ignore{%
        \typeout{%
          (pagesel) \space\space\@spaces\@spaces\@spaces
          Message (\string\end\space occurred ...) %
          can be ignored.%
        }%
      }%
    \else
      \let\ps@message@ignore\@empty
    \fi
  }%
%<packagefrozen>\fi
%</package|packagefrozen>
%    \end{macrocode}
%
% \subsection{AtBeginDvi hook support}
%
%    The material of box \cs{@begindvibox} is recorded in parallel
%    in box \cs{ps@begindvibox}.
%    \begin{macrocode}
%<*packagefrozen>
\newbox\ps@begindvibox
\ifvoid\@begindvibox
\else
  \global\setbox\ps@begindvibox\vbox{%
    \unvbox\@begindvibox
  }%
\fi
\let\ps@org@AtBeginDvi\AtBeginDvi
\def\AtBeginDvi#1{%
  \global\setbox\ps@begindvibox\vbox{%
    \unvbox\ps@begindvibox
    #1%
  }%
  \ps@org@AtBeginDvi{#1}%
}
%    \end{macrocode}
%
%    \begin{macro}{\ps@begindvi}
%    Macro \cs{ps@begindvi} is called the similar way as \cs{@begindvi}.
%    If the first page is printed, then \cs{AtBeginDvi} should work
%    as usual. Otherwise the contents of box \cs{ps@begindvibox} is
%    set on the first selected page.
%    \begin{macrocode}
\def\ps@begindvi{%
  \ifx\ps@next\@empty
    \global\let\ps@begindvi\@empty
  \else
    \global\let\ps@begindvi\ps@begindvi@do
  \fi
}
\def\ps@begindvi@do{%
  \ifx\ps@next\@empty
    \setbox\@cclv\vbox{%
      \unvbox\ps@begindvibox
      \box\@cclv
    }%
    \global\let\ps@begindvi\@empty
  \fi
}
%    \end{macrocode}
%    \end{macro}
%
%    \begin{macrocode}
%</packagefrozen>
%    \end{macrocode}
%
% \section{Installation}
%
% \subsection{Download}
%
% \paragraph{Package.} This package is available on
% CTAN\footnote{\CTANpkg{pagesel}}:
% \begin{description}
% \item[\CTAN{macros/latex/contrib/pagesel/pagesel.dtx}] The source file.
% \item[\CTAN{macros/latex/contrib/pagesel/pagesel.pdf}] Documentation.
% \end{description}
%
%
%
% \subsection{Package installation}
%
% \paragraph{Unpacking.} The \xfile{.dtx} file is a self-extracting
% \docstrip\ archive. The files are extracted by running the
% \xfile{.dtx} through \plainTeX:
% \begin{quote}
%   \verb|tex pagesel.dtx|
% \end{quote}
%
% \paragraph{TDS.} Now the different files must be moved into
% the different directories in your installation TDS tree
% (also known as \xfile{texmf} tree):
% \begin{quote}
% \def\t{^^A
% \begin{tabular}{@{}>{\ttfamily}l@{ $\rightarrow$ }>{\ttfamily}l@{}}
%   pagesel.sty & tex/latex/pagesel/pagesel.sty\\
%   pagesel.pdf & doc/latex/pagesel/pagesel.pdf\\
%   pagesel.dtx & source/latex/pagesel/pagesel.dtx\\
% \end{tabular}^^A
% }^^A
% \sbox0{\t}^^A
% \ifdim\wd0>\linewidth
%   \begingroup
%     \advance\linewidth by\leftmargin
%     \advance\linewidth by\rightmargin
%   \edef\x{\endgroup
%     \def\noexpand\lw{\the\linewidth}^^A
%   }\x
%   \def\lwbox{^^A
%     \leavevmode
%     \hbox to \linewidth{^^A
%       \kern-\leftmargin\relax
%       \hss
%       \usebox0
%       \hss
%       \kern-\rightmargin\relax
%     }^^A
%   }^^A
%   \ifdim\wd0>\lw
%     \sbox0{\small\t}^^A
%     \ifdim\wd0>\linewidth
%       \ifdim\wd0>\lw
%         \sbox0{\footnotesize\t}^^A
%         \ifdim\wd0>\linewidth
%           \ifdim\wd0>\lw
%             \sbox0{\scriptsize\t}^^A
%             \ifdim\wd0>\linewidth
%               \ifdim\wd0>\lw
%                 \sbox0{\tiny\t}^^A
%                 \ifdim\wd0>\linewidth
%                   \lwbox
%                 \else
%                   \usebox0
%                 \fi
%               \else
%                 \lwbox
%               \fi
%             \else
%               \usebox0
%             \fi
%           \else
%             \lwbox
%           \fi
%         \else
%           \usebox0
%         \fi
%       \else
%         \lwbox
%       \fi
%     \else
%       \usebox0
%     \fi
%   \else
%     \lwbox
%   \fi
% \else
%   \usebox0
% \fi
% \end{quote}
% If you have a \xfile{docstrip.cfg} that configures and enables \docstrip's
% TDS installing feature, then some files can already be in the right
% place, see the documentation of \docstrip.
%
% \subsection{Refresh file name databases}
%
% If your \TeX~distribution
% (\TeX\,Live, \mikTeX, \dots) relies on file name databases, you must refresh
% these. For example, \TeX\,Live\ users run \verb|texhash| or
% \verb|mktexlsr|.
%
% \subsection{Some details for the interested}
%
% \paragraph{Unpacking with \LaTeX.}
% The \xfile{.dtx} chooses its action depending on the format:
% \begin{description}
% \item[\plainTeX:] Run \docstrip\ and extract the files.
% \item[\LaTeX:] Generate the documentation.
% \end{description}
% If you insist on using \LaTeX\ for \docstrip\ (really,
% \docstrip\ does not need \LaTeX), then inform the autodetect routine
% about your intention:
% \begin{quote}
%   \verb|latex \let\install=y\input{pagesel.dtx}|
% \end{quote}
% Do not forget to quote the argument according to the demands
% of your shell.
%
% \paragraph{Generating the documentation.}
% You can use both the \xfile{.dtx} or the \xfile{.drv} to generate
% the documentation. The process can be configured by the
% configuration file \xfile{ltxdoc.cfg}. For instance, put this
% line into this file, if you want to have A4 as paper format:
% \begin{quote}
%   \verb|\PassOptionsToClass{a4paper}{article}|
% \end{quote}
% An example follows how to generate the
% documentation with pdf\LaTeX:
% \begin{quote}
%\begin{verbatim}
%pdflatex pagesel.dtx
%makeindex -s gind.ist pagesel.idx
%pdflatex pagesel.dtx
%makeindex -s gind.ist pagesel.idx
%pdflatex pagesel.dtx
%\end{verbatim}
% \end{quote}
%
% \begin{History}
%   \begin{Version}{1999/03/01 v0.9}
%   \item
%     The first version was built as a response to a question
%     of \NameEmail{Dirk Kuypers}{dk@comnets.rwth-aachen.de},
%     published in the newsgroup
%     \href{news:de.comp.text.tex}{de.comp.text.tex}:\\
%     \URL{``\link{Re: pdflatex nur fuer bestimmte Seiten?!?}''}^^A
%     {https://groups.google.com/group/de.comp.text.tex/msg/6b68c7b3439fb658}
%   \end{Version}
%   \begin{Version}{1999/04/05 v1.0}
%   \item
%     Documentation added in dtx format.
%   \item
%     Copyright: LPPL (\CTAN{macros/latex/base/lppl.txt})
%   \item
%     Options |odd|, |even| added.
%   \item
%     \cmd{\nofiles} added, bug fix for \Package{hyperref}.
%   \item
%     Abort loading of package, if nothing to do.
%   \end{Version}
%   \begin{Version}{1999/04/13 v1.1}
%   \item
%     \cs{nofiles} bug fix removed
%     because of \xpackage{hyperref} 6.55.
%   \item
%     First CTAN release.
%   \end{Version}
%   \begin{Version}{2003/06/05 v1.2}
%   \item
%     \cs{deadcyles} is decremented for omitted pages.
%   \item
%     LPPL 1.2.
%   \end{Version}
%   \begin{Version}{2006/02/20 v1.3}
%   \item
%     Code is not changed.
%   \item
%     New DTX framework.
%   \item
%     LPPL 1.3
%   \end{Version}
%   \begin{Version}{2006/03/02 v1.4}
%   \item
%     Support for \cs{AtBeginDvi} added.
%   \end{Version}
%   \begin{Version}{2006/03/07 v1.5}
%   \item
%     Job is aborted after last selected page.
%   \end{Version}
%   \begin{Version}{2007/04/11 v1.6}
%   \item
%     Line ends sanitized.
%   \end{Version}
%   \begin{Version}{2007/04/12 v1.7}
%   \item
%     Hard coded box number 255 replaced by macro \cs{@cclv}.
%   \end{Version}
%   \begin{Version}{2008/08/11 v1.8}
%   \item
%     Code is not changed.
%   \item
%     URL updated from \texttt{www.dejanews.com}
%     to \texttt{groups.google.com}.
%   \end{Version}
%   \begin{Version}{2016/05/16 v1.9}
%   \item
%     Documentation updates.
%   \end{Version}
%   \begin{Version}{2020-08-03 v1.10}
%   \item Updated to follow the changes in the hook management
%   of LaTeX 2020/10/01
%   \end{Version}
% \end{History}
%
% \PrintIndex
%
% \Finale
\endinput
|
% \end{quote}
% Do not forget to quote the argument according to the demands
% of your shell.
%
% \paragraph{Generating the documentation.}
% You can use both the \xfile{.dtx} or the \xfile{.drv} to generate
% the documentation. The process can be configured by the
% configuration file \xfile{ltxdoc.cfg}. For instance, put this
% line into this file, if you want to have A4 as paper format:
% \begin{quote}
%   \verb|\PassOptionsToClass{a4paper}{article}|
% \end{quote}
% An example follows how to generate the
% documentation with pdf\LaTeX:
% \begin{quote}
%\begin{verbatim}
%pdflatex pagesel.dtx
%makeindex -s gind.ist pagesel.idx
%pdflatex pagesel.dtx
%makeindex -s gind.ist pagesel.idx
%pdflatex pagesel.dtx
%\end{verbatim}
% \end{quote}
%
% \begin{History}
%   \begin{Version}{1999/03/01 v0.9}
%   \item
%     The first version was built as a response to a question
%     of \NameEmail{Dirk Kuypers}{dk@comnets.rwth-aachen.de},
%     published in the newsgroup
%     \href{news:de.comp.text.tex}{de.comp.text.tex}:\\
%     \URL{``\link{Re: pdflatex nur fuer bestimmte Seiten?!?}''}^^A
%     {https://groups.google.com/group/de.comp.text.tex/msg/6b68c7b3439fb658}
%   \end{Version}
%   \begin{Version}{1999/04/05 v1.0}
%   \item
%     Documentation added in dtx format.
%   \item
%     Copyright: LPPL (\CTAN{macros/latex/base/lppl.txt})
%   \item
%     Options |odd|, |even| added.
%   \item
%     \cmd{\nofiles} added, bug fix for \Package{hyperref}.
%   \item
%     Abort loading of package, if nothing to do.
%   \end{Version}
%   \begin{Version}{1999/04/13 v1.1}
%   \item
%     \cs{nofiles} bug fix removed
%     because of \xpackage{hyperref} 6.55.
%   \item
%     First CTAN release.
%   \end{Version}
%   \begin{Version}{2003/06/05 v1.2}
%   \item
%     \cs{deadcyles} is decremented for omitted pages.
%   \item
%     LPPL 1.2.
%   \end{Version}
%   \begin{Version}{2006/02/20 v1.3}
%   \item
%     Code is not changed.
%   \item
%     New DTX framework.
%   \item
%     LPPL 1.3
%   \end{Version}
%   \begin{Version}{2006/03/02 v1.4}
%   \item
%     Support for \cs{AtBeginDvi} added.
%   \end{Version}
%   \begin{Version}{2006/03/07 v1.5}
%   \item
%     Job is aborted after last selected page.
%   \end{Version}
%   \begin{Version}{2007/04/11 v1.6}
%   \item
%     Line ends sanitized.
%   \end{Version}
%   \begin{Version}{2007/04/12 v1.7}
%   \item
%     Hard coded box number 255 replaced by macro \cs{@cclv}.
%   \end{Version}
%   \begin{Version}{2008/08/11 v1.8}
%   \item
%     Code is not changed.
%   \item
%     URL updated from \texttt{www.dejanews.com}
%     to \texttt{groups.google.com}.
%   \end{Version}
%   \begin{Version}{2016/05/16 v1.9}
%   \item
%     Documentation updates.
%   \end{Version}
%   \begin{Version}{2020-08-03 v1.10}
%   \item Updated to follow the changes in the hook management
%   of LaTeX 2020/10/01
%   \end{Version}
% \end{History}
%
% \PrintIndex
%
% \Finale
\endinput
|
% \end{quote}
% Do not forget to quote the argument according to the demands
% of your shell.
%
% \paragraph{Generating the documentation.}
% You can use both the \xfile{.dtx} or the \xfile{.drv} to generate
% the documentation. The process can be configured by the
% configuration file \xfile{ltxdoc.cfg}. For instance, put this
% line into this file, if you want to have A4 as paper format:
% \begin{quote}
%   \verb|\PassOptionsToClass{a4paper}{article}|
% \end{quote}
% An example follows how to generate the
% documentation with pdf\LaTeX:
% \begin{quote}
%\begin{verbatim}
%pdflatex pagesel.dtx
%makeindex -s gind.ist pagesel.idx
%pdflatex pagesel.dtx
%makeindex -s gind.ist pagesel.idx
%pdflatex pagesel.dtx
%\end{verbatim}
% \end{quote}
%
% \begin{History}
%   \begin{Version}{1999/03/01 v0.9}
%   \item
%     The first version was built as a response to a question
%     of \NameEmail{Dirk Kuypers}{dk@comnets.rwth-aachen.de},
%     published in the newsgroup
%     \href{news:de.comp.text.tex}{de.comp.text.tex}:\\
%     \URL{``\link{Re: pdflatex nur fuer bestimmte Seiten?!?}''}^^A
%     {https://groups.google.com/group/de.comp.text.tex/msg/6b68c7b3439fb658}
%   \end{Version}
%   \begin{Version}{1999/04/05 v1.0}
%   \item
%     Documentation added in dtx format.
%   \item
%     Copyright: LPPL (\CTAN{macros/latex/base/lppl.txt})
%   \item
%     Options |odd|, |even| added.
%   \item
%     \cmd{\nofiles} added, bug fix for \Package{hyperref}.
%   \item
%     Abort loading of package, if nothing to do.
%   \end{Version}
%   \begin{Version}{1999/04/13 v1.1}
%   \item
%     \cs{nofiles} bug fix removed
%     because of \xpackage{hyperref} 6.55.
%   \item
%     First CTAN release.
%   \end{Version}
%   \begin{Version}{2003/06/05 v1.2}
%   \item
%     \cs{deadcyles} is decremented for omitted pages.
%   \item
%     LPPL 1.2.
%   \end{Version}
%   \begin{Version}{2006/02/20 v1.3}
%   \item
%     Code is not changed.
%   \item
%     New DTX framework.
%   \item
%     LPPL 1.3
%   \end{Version}
%   \begin{Version}{2006/03/02 v1.4}
%   \item
%     Support for \cs{AtBeginDvi} added.
%   \end{Version}
%   \begin{Version}{2006/03/07 v1.5}
%   \item
%     Job is aborted after last selected page.
%   \end{Version}
%   \begin{Version}{2007/04/11 v1.6}
%   \item
%     Line ends sanitized.
%   \end{Version}
%   \begin{Version}{2007/04/12 v1.7}
%   \item
%     Hard coded box number 255 replaced by macro \cs{@cclv}.
%   \end{Version}
%   \begin{Version}{2008/08/11 v1.8}
%   \item
%     Code is not changed.
%   \item
%     URL updated from \texttt{www.dejanews.com}
%     to \texttt{groups.google.com}.
%   \end{Version}
%   \begin{Version}{2016/05/16 v1.9}
%   \item
%     Documentation updates.
%   \end{Version}
%   \begin{Version}{2020-08-03 v1.10}
%   \item Updated to follow the changes in the hook management
%   of LaTeX 2020/10/01
%   \end{Version}
% \end{History}
%
% \PrintIndex
%
% \Finale
\endinput
|
% \end{quote}
% Do not forget to quote the argument according to the demands
% of your shell.
%
% \paragraph{Generating the documentation.}
% You can use both the \xfile{.dtx} or the \xfile{.drv} to generate
% the documentation. The process can be configured by the
% configuration file \xfile{ltxdoc.cfg}. For instance, put this
% line into this file, if you want to have A4 as paper format:
% \begin{quote}
%   \verb|\PassOptionsToClass{a4paper}{article}|
% \end{quote}
% An example follows how to generate the
% documentation with pdf\LaTeX:
% \begin{quote}
%\begin{verbatim}
%pdflatex pagesel.dtx
%makeindex -s gind.ist pagesel.idx
%pdflatex pagesel.dtx
%makeindex -s gind.ist pagesel.idx
%pdflatex pagesel.dtx
%\end{verbatim}
% \end{quote}
%
% \begin{History}
%   \begin{Version}{1999/03/01 v0.9}
%   \item
%     The first version was built as a response to a question
%     of \NameEmail{Dirk Kuypers}{dk@comnets.rwth-aachen.de},
%     published in the newsgroup
%     \href{news:de.comp.text.tex}{de.comp.text.tex}:\\
%     \URL{``\link{Re: pdflatex nur fuer bestimmte Seiten?!?}''}^^A
%     {https://groups.google.com/group/de.comp.text.tex/msg/6b68c7b3439fb658}
%   \end{Version}
%   \begin{Version}{1999/04/05 v1.0}
%   \item
%     Documentation added in dtx format.
%   \item
%     Copyright: LPPL (\CTAN{macros/latex/base/lppl.txt})
%   \item
%     Options |odd|, |even| added.
%   \item
%     \cmd{\nofiles} added, bug fix for \Package{hyperref}.
%   \item
%     Abort loading of package, if nothing to do.
%   \end{Version}
%   \begin{Version}{1999/04/13 v1.1}
%   \item
%     \cs{nofiles} bug fix removed
%     because of \xpackage{hyperref} 6.55.
%   \item
%     First CTAN release.
%   \end{Version}
%   \begin{Version}{2003/06/05 v1.2}
%   \item
%     \cs{deadcyles} is decremented for omitted pages.
%   \item
%     LPPL 1.2.
%   \end{Version}
%   \begin{Version}{2006/02/20 v1.3}
%   \item
%     Code is not changed.
%   \item
%     New DTX framework.
%   \item
%     LPPL 1.3
%   \end{Version}
%   \begin{Version}{2006/03/02 v1.4}
%   \item
%     Support for \cs{AtBeginDvi} added.
%   \end{Version}
%   \begin{Version}{2006/03/07 v1.5}
%   \item
%     Job is aborted after last selected page.
%   \end{Version}
%   \begin{Version}{2007/04/11 v1.6}
%   \item
%     Line ends sanitized.
%   \end{Version}
%   \begin{Version}{2007/04/12 v1.7}
%   \item
%     Hard coded box number 255 replaced by macro \cs{@cclv}.
%   \end{Version}
%   \begin{Version}{2008/08/11 v1.8}
%   \item
%     Code is not changed.
%   \item
%     URL updated from \texttt{www.dejanews.com}
%     to \texttt{groups.google.com}.
%   \end{Version}
%   \begin{Version}{2016/05/16 v1.9}
%   \item
%     Documentation updates.
%   \end{Version}
%   \begin{Version}{2020-08-03 v1.10}
%   \item Updated to follow the changes in the hook management
%   of LaTeX 2020/10/01
%   \end{Version}
% \end{History}
%
% \PrintIndex
%
% \Finale
\endinput
