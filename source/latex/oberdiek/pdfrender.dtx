% \iffalse meta-comment
%
% File: pdfrender.dtx
% Version: 2019/12/29 v1.6
% Info: Access to some PDF graphics parameters
%
% Copyright (C)
%    2010 Heiko Oberdiek
%    2016-2019 Oberdiek Package Support Group
%    https://github.com/ho-tex/oberdiek/issues
%
% This work may be distributed and/or modified under the
% conditions of the LaTeX Project Public License, either
% version 1.3c of this license or (at your option) any later
% version. This version of this license is in
%    https://www.latex-project.org/lppl/lppl-1-3c.txt
% and the latest version of this license is in
%    https://www.latex-project.org/lppl.txt
% and version 1.3 or later is part of all distributions of
% LaTeX version 2005/12/01 or later.
%
% This work has the LPPL maintenance status "maintained".
%
% The Current Maintainers of this work are
% Heiko Oberdiek and the Oberdiek Package Support Group
% https://github.com/ho-tex/oberdiek/issues
%
% The Base Interpreter refers to any `TeX-Format',
% because some files are installed in TDS:tex/generic//.
%
% This work consists of the main source file pdfrender.dtx
% and the derived files
%    pdfrender.sty, pdfrender.pdf, pdfrender.ins, pdfrender.drv,
%    pdfrender-test1.tex, pdfrender-test2.tex,
%    pdfrender-test3.tex, pdfrender-test4.tex,
%    pdfrender-test5.tex.
%
% Distribution:
%    CTAN:macros/latex/contrib/oberdiek/pdfrender.dtx
%    CTAN:macros/latex/contrib/oberdiek/pdfrender.pdf
%
% Unpacking:
%    (a) If pdfrender.ins is present:
%           tex pdfrender.ins
%    (b) Without pdfrender.ins:
%           tex pdfrender.dtx
%    (c) If you insist on using LaTeX
%           latex \let\install=y% \iffalse meta-comment
%
% File: pdfrender.dtx
% Version: 2019/12/29 v1.6
% Info: Access to some PDF graphics parameters
%
% Copyright (C)
%    2010 Heiko Oberdiek
%    2016-2019 Oberdiek Package Support Group
%    https://github.com/ho-tex/oberdiek/issues
%
% This work may be distributed and/or modified under the
% conditions of the LaTeX Project Public License, either
% version 1.3c of this license or (at your option) any later
% version. This version of this license is in
%    https://www.latex-project.org/lppl/lppl-1-3c.txt
% and the latest version of this license is in
%    https://www.latex-project.org/lppl.txt
% and version 1.3 or later is part of all distributions of
% LaTeX version 2005/12/01 or later.
%
% This work has the LPPL maintenance status "maintained".
%
% The Current Maintainers of this work are
% Heiko Oberdiek and the Oberdiek Package Support Group
% https://github.com/ho-tex/oberdiek/issues
%
% The Base Interpreter refers to any `TeX-Format',
% because some files are installed in TDS:tex/generic//.
%
% This work consists of the main source file pdfrender.dtx
% and the derived files
%    pdfrender.sty, pdfrender.pdf, pdfrender.ins, pdfrender.drv,
%    pdfrender-test1.tex, pdfrender-test2.tex,
%    pdfrender-test3.tex, pdfrender-test4.tex,
%    pdfrender-test5.tex.
%
% Distribution:
%    CTAN:macros/latex/contrib/oberdiek/pdfrender.dtx
%    CTAN:macros/latex/contrib/oberdiek/pdfrender.pdf
%
% Unpacking:
%    (a) If pdfrender.ins is present:
%           tex pdfrender.ins
%    (b) Without pdfrender.ins:
%           tex pdfrender.dtx
%    (c) If you insist on using LaTeX
%           latex \let\install=y% \iffalse meta-comment
%
% File: pdfrender.dtx
% Version: 2019/12/29 v1.6
% Info: Access to some PDF graphics parameters
%
% Copyright (C)
%    2010 Heiko Oberdiek
%    2016-2019 Oberdiek Package Support Group
%    https://github.com/ho-tex/oberdiek/issues
%
% This work may be distributed and/or modified under the
% conditions of the LaTeX Project Public License, either
% version 1.3c of this license or (at your option) any later
% version. This version of this license is in
%    https://www.latex-project.org/lppl/lppl-1-3c.txt
% and the latest version of this license is in
%    https://www.latex-project.org/lppl.txt
% and version 1.3 or later is part of all distributions of
% LaTeX version 2005/12/01 or later.
%
% This work has the LPPL maintenance status "maintained".
%
% The Current Maintainers of this work are
% Heiko Oberdiek and the Oberdiek Package Support Group
% https://github.com/ho-tex/oberdiek/issues
%
% The Base Interpreter refers to any `TeX-Format',
% because some files are installed in TDS:tex/generic//.
%
% This work consists of the main source file pdfrender.dtx
% and the derived files
%    pdfrender.sty, pdfrender.pdf, pdfrender.ins, pdfrender.drv,
%    pdfrender-test1.tex, pdfrender-test2.tex,
%    pdfrender-test3.tex, pdfrender-test4.tex,
%    pdfrender-test5.tex.
%
% Distribution:
%    CTAN:macros/latex/contrib/oberdiek/pdfrender.dtx
%    CTAN:macros/latex/contrib/oberdiek/pdfrender.pdf
%
% Unpacking:
%    (a) If pdfrender.ins is present:
%           tex pdfrender.ins
%    (b) Without pdfrender.ins:
%           tex pdfrender.dtx
%    (c) If you insist on using LaTeX
%           latex \let\install=y% \iffalse meta-comment
%
% File: pdfrender.dtx
% Version: 2019/12/29 v1.6
% Info: Access to some PDF graphics parameters
%
% Copyright (C)
%    2010 Heiko Oberdiek
%    2016-2019 Oberdiek Package Support Group
%    https://github.com/ho-tex/oberdiek/issues
%
% This work may be distributed and/or modified under the
% conditions of the LaTeX Project Public License, either
% version 1.3c of this license or (at your option) any later
% version. This version of this license is in
%    https://www.latex-project.org/lppl/lppl-1-3c.txt
% and the latest version of this license is in
%    https://www.latex-project.org/lppl.txt
% and version 1.3 or later is part of all distributions of
% LaTeX version 2005/12/01 or later.
%
% This work has the LPPL maintenance status "maintained".
%
% The Current Maintainers of this work are
% Heiko Oberdiek and the Oberdiek Package Support Group
% https://github.com/ho-tex/oberdiek/issues
%
% The Base Interpreter refers to any `TeX-Format',
% because some files are installed in TDS:tex/generic//.
%
% This work consists of the main source file pdfrender.dtx
% and the derived files
%    pdfrender.sty, pdfrender.pdf, pdfrender.ins, pdfrender.drv,
%    pdfrender-test1.tex, pdfrender-test2.tex,
%    pdfrender-test3.tex, pdfrender-test4.tex,
%    pdfrender-test5.tex.
%
% Distribution:
%    CTAN:macros/latex/contrib/oberdiek/pdfrender.dtx
%    CTAN:macros/latex/contrib/oberdiek/pdfrender.pdf
%
% Unpacking:
%    (a) If pdfrender.ins is present:
%           tex pdfrender.ins
%    (b) Without pdfrender.ins:
%           tex pdfrender.dtx
%    (c) If you insist on using LaTeX
%           latex \let\install=y\input{pdfrender.dtx}
%        (quote the arguments according to the demands of your shell)
%
% Documentation:
%    (a) If pdfrender.drv is present:
%           latex pdfrender.drv
%    (b) Without pdfrender.drv:
%           latex pdfrender.dtx; ...
%    The class ltxdoc loads the configuration file ltxdoc.cfg
%    if available. Here you can specify further options, e.g.
%    use A4 as paper format:
%       \PassOptionsToClass{a4paper}{article}
%
%    Programm calls to get the documentation (example):
%       pdflatex pdfrender.dtx
%       makeindex -s gind.ist pdfrender.idx
%       pdflatex pdfrender.dtx
%       makeindex -s gind.ist pdfrender.idx
%       pdflatex pdfrender.dtx
%
% Installation:
%    TDS:tex/generic/oberdiek/pdfrender.sty
%    TDS:doc/latex/oberdiek/pdfrender.pdf
%    TDS:source/latex/oberdiek/pdfrender.dtx
%
%<*ignore>
\begingroup
  \catcode123=1 %
  \catcode125=2 %
  \def\x{LaTeX2e}%
\expandafter\endgroup
\ifcase 0\ifx\install y1\fi\expandafter
         \ifx\csname processbatchFile\endcsname\relax\else1\fi
         \ifx\fmtname\x\else 1\fi\relax
\else\csname fi\endcsname
%</ignore>
%<*install>
\input docstrip.tex
\Msg{************************************************************************}
\Msg{* Installation}
\Msg{* Package: pdfrender 2019/12/29 v1.6 Access to some PDF graphics parameters (HO)}
\Msg{************************************************************************}

\keepsilent
\askforoverwritefalse

\let\MetaPrefix\relax
\preamble

This is a generated file.

Project: pdfrender
Version: 2019/12/29 v1.6

Copyright (C)
   2010 Heiko Oberdiek
   2016-2019 Oberdiek Package Support Group

This work may be distributed and/or modified under the
conditions of the LaTeX Project Public License, either
version 1.3c of this license or (at your option) any later
version. This version of this license is in
   https://www.latex-project.org/lppl/lppl-1-3c.txt
and the latest version of this license is in
   https://www.latex-project.org/lppl.txt
and version 1.3 or later is part of all distributions of
LaTeX version 2005/12/01 or later.

This work has the LPPL maintenance status "maintained".

The Current Maintainers of this work are
Heiko Oberdiek and the Oberdiek Package Support Group
https://github.com/ho-tex/oberdiek/issues


The Base Interpreter refers to any `TeX-Format',
because some files are installed in TDS:tex/generic//.

This work consists of the main source file pdfrender.dtx
and the derived files
   pdfrender.sty, pdfrender.pdf, pdfrender.ins, pdfrender.drv,
   pdfrender-test1.tex, pdfrender-test2.tex,
   pdfrender-test3.tex, pdfrender-test4.tex,
   pdfrender-test5.tex.

\endpreamble
\let\MetaPrefix\DoubleperCent

\generate{%
  \file{pdfrender.ins}{\from{pdfrender.dtx}{install}}%
  \file{pdfrender.drv}{\from{pdfrender.dtx}{driver}}%
  \usedir{tex/generic/oberdiek}%
  \file{pdfrender.sty}{\from{pdfrender.dtx}{package}}%
%  \usedir{doc/latex/oberdiek/test}%
%  \file{pdfrender-test1.tex}{\from{pdfrender.dtx}{test1}}%
%  \file{pdfrender-test2.tex}{\from{pdfrender.dtx}{test2}}%
%  \file{pdfrender-test3.tex}{\from{pdfrender.dtx}{test3}}%
%  \file{pdfrender-test4.tex}{\from{pdfrender.dtx}{test4}}%
%  \file{pdfrender-test5.tex}{\from{pdfrender.dtx}{test5}}%
}

\catcode32=13\relax% active space
\let =\space%
\Msg{************************************************************************}
\Msg{*}
\Msg{* To finish the installation you have to move the following}
\Msg{* file into a directory searched by TeX:}
\Msg{*}
\Msg{*     pdfrender.sty}
\Msg{*}
\Msg{* To produce the documentation run the file `pdfrender.drv'}
\Msg{* through LaTeX.}
\Msg{*}
\Msg{* Happy TeXing!}
\Msg{*}
\Msg{************************************************************************}

\endbatchfile
%</install>
%<*ignore>
\fi
%</ignore>
%<*driver>
\NeedsTeXFormat{LaTeX2e}
\ProvidesFile{pdfrender.drv}%
  [2019/12/29 v1.6 Access to some PDF graphics parameters (HO)]%
\documentclass{ltxdoc}
\usepackage{holtxdoc}[2011/11/22]
\usepackage{longtable}
\renewcommand*{\arraystretch}{1.2}
\begin{document}
  \DocInput{pdfrender.dtx}%
\end{document}
%</driver>
% \fi
%
%
%
% \GetFileInfo{pdfrender.drv}
%
% \title{The \xpackage{pdfrender} package}
% \date{2019/12/29 v1.6}
% \author{Heiko Oberdiek\thanks
% {Please report any issues at \url{https://github.com/ho-tex/oberdiek/issues}}}
%
% \maketitle
%
% \begin{abstract}
% The PDF format has some graphics parameter like
% line width or text rendering mode. This package
% provides an interface for setting these parameters.
% \end{abstract}
%
% \tableofcontents
%
% \section{Documentation}
%
% This package \xpackage{pdfrender} defines an interface for
% PDF specific parameters that affects the rendering of
% graphics or text. The interface and its implementation
% uses the same technique as package \xpackage{color}
% for color settings. Therefore this package is loaded to
% enable \LaTeX's color interface.
%
% At different places \LaTeX\ uses \cs{normalcolor} to avoid that
% header, footer or floats are print in the current color of the
% main text. \cs{setgroup@color} is used to start a save box
% with the color that is set at box saving time.
% Package \xpackage{pdfrender} extends these macros to
% add its own hooks of its parameters. Therefore \LaTeX3 should
% generalize \LaTeXe's color interface.
%
% \subsection{Usage}
%
% In \LaTeX\ the package is loaded as normal package. Options
% are not defined for this package.
% \begin{quote}
% |\usepackage{pdfrender}|
% \end{quote}
%
% This package can also be used in \plainTeX\ and even ini\TeX:
% \begin{quote}
% |input pdfrender.sty|
% \end{quote}
%
% \subsection{Macros}
%
% \begin{declcs}{pdfrender} \M{key value list}
% \end{declcs}
% The first parameter \meta{key value list} contains a
% list of parameter settings. The key entry is the parameter name.
% The macro works like
% \cs{color} (without optional argument) for color setting.
%
% \begin{declcs}{textpdfrender} \M{key value list} \M{text}
% \end{declcs}
% In the same way as \cs{pdfrender} the first argument
% specifies the parameters that should be set. This parameter
% setting affects \meta{text} only. Basically it works
% the same way as \cs{textcolor} (without optional argument).
%
% \subsection{Parameters}
%
% The following table shows an overview for the supported
% parameters and values:
%
% \begin{longtable}{@{}l|l|l@{}}
% \bfseries Parameter & \bfseries Value & \bfseries Alias\\*
% \hline
% \endhead
% \xoption{TextRenderingMode}
%   & \texttt{0} & \texttt{Fill}\\*
%   & \texttt{1} & \texttt{Stroke}\\*
%   & \texttt{2} & \texttt{FillStroke}\\*
%   & \texttt{3} & \texttt{Invisible}\\*
%   & \texttt{4} & \texttt{FillClip}\\*
%   & \texttt{5} & \texttt{StrokeClip}\\*
%   & \texttt{6} & \texttt{FillStrokeClip}\\*
%   & \texttt{7} & \texttt{Clip}\\
% \hline
% \xoption{LineWidth}
%   & \itshape positive number, & \itshape \TeX\ dimen\\*[-.2ex]
%   & \qquad \itshape unit is bp & \\
% \hline
% \xoption{LineCapStyle}
%   & \texttt{0} & \texttt{Butt}\\*
%   & \texttt{1} & \texttt{Round}\\*
%   & \texttt{2} & \texttt{ProjectingSquare}\\
% \hline
% \xoption{LineJoinStyle}
%   & \texttt{0} & \texttt{Miter}\\*
%   & \texttt{1} & \texttt{Round}\\*
%   & \texttt{2} & \texttt{Bevel}\\
% \hline
% \xoption{MiterLimit}
%   & \multicolumn{2}{l@{}}{\itshape positive number} \\
% \hline
% \xoption{Flatness}
%   & \multicolumn{2}{l@{}}{\itshape number between 0 and 100} \\
% \hline
% \xoption{LineDashPattern}
%   & \multicolumn{2}{l@{}}{^^A
%       \itshape numbers in square brackets,^^A
%     } \\*[-.2ex]
%   & \multicolumn{2}{l@{}}{^^A
%       \qquad \itshape followed by number,^^A
%     } \\*
%   & \multicolumn{2}{l@{}}{^^A
%       \qquad \itshape units are bp^^A
%     } \\
% \hline
% \xoption{RenderingIntent}
%   & \multicolumn{2}{l@{}}{\texttt{AbsoluteColorimetric}} \\*
%   & \multicolumn{2}{l@{}}{\texttt{RelativeColorimetric}} \\*
%   & \multicolumn{2}{l@{}}{\texttt{Saturation}} \\*
%   & \multicolumn{2}{l@{}}{\texttt{Perceptual}} \\
% \hline
% \xoption{FillColor}
%   & & \itshape color specification \\
% \hline
% \xoption{StrokeColor}
%   & & \itshape color specification \\
% \end{longtable}
%
% \subsubsection{Details}
%
% The description and specification of these parameters
% are available in the PDF specification \cite{pdf-spec}.
% Therefore they are not repeated here.
%
% \begin{description}
% \item[Value:]
%   The values in the second column lists or describe
%   the values that are specified by the PDF specification.
% \item[Alias:] Instead of magic numbers the package also
%   defines some aliases that can be given as value.
%   Example: |LineCapStyle=Round| has the same effect as
%   |LineCapStyle=1|.
% \item[Number:] The term \emph{number} means an integer
%   or real number. The real number is given as plain
%   decimal number without exponent. The decimal separator is a period.
%   At least one digit must be present.
% \item[\xoption{LineWidth}:] As alias a \TeX\ dimen specification
%   can be given. This includes explicit specifications with number
%   and unit, e.g. |LineWidth=0.5pt|. Also \LaTeX\ length registers
%   may be used. If \eTeX's \cs{dimexpr} is available, then it is
%   automatically added. However package \xpackage{calc} is not
%   supported.
% \item[\xoption{FillColor}, \xoption{StrokeColor}:]
%   Package \xpackage{color} or \xpackage{xcolor} must be loaded
%   before these options can be used (since version 1.2).
%   \LaTeX's color support sets both colors at the same time
%   to the same value. However parameter \xoption{TextRenderingMode}
%   offers the value \texttt{FillStroke} that makes only sense, if
%   the two color types can be set separately.
%   If one of the options \xoption{FillColor} or \xoption{StrokeColor}
%   is specified, then also the color is set. For compatibility
%   with the \LaTeX\ color packages (\xpackage{color} or \xpackage{xcolor}),
%   always both colors must be set. Thus if one of them is not specified,
%   it is taken from the current color.
%
%   Both options \xoption{FillColor} and \xoption{StrokeColor}
%   expect a \LaTeX\ color specification as value.
%   Also the optional color model argument is supported.
%   Example:
%   \begin{quote}
%     |FillColor=yellow|,\\
%     |StrokeColor=[cmyk]{1,.5,0,0}|\\
%   \end{quote}
% \end{description}
%
% \subsection{Color stack}
%
% If the \pdfTeX\ version provides color stacks, then each
% parameter is assigned a page based color stack. The assignment
% of a stack takes place, when its parameter is set the first time.
% This avoids the use of color stacks that are not needed.
%
% \StopEventually{
% }
%
% \section{Implementation}
%    \begin{macrocode}
%<*package>
%    \end{macrocode}
%    Reload check, especially if the package is not used with \LaTeX.
%    \begin{macrocode}
\begingroup\catcode61\catcode48\catcode32=10\relax%
  \catcode13=5 % ^^M
  \endlinechar=13 %
  \catcode35=6 % #
  \catcode39=12 % '
  \catcode44=12 % ,
  \catcode45=12 % -
  \catcode46=12 % .
  \catcode58=12 % :
  \catcode64=11 % @
  \catcode123=1 % {
  \catcode125=2 % }
  \expandafter\let\expandafter\x\csname ver@pdfrender.sty\endcsname
  \ifx\x\relax % plain-TeX, first loading
  \else
    \def\empty{}%
    \ifx\x\empty % LaTeX, first loading,
      % variable is initialized, but \ProvidesPackage not yet seen
    \else
      \expandafter\ifx\csname PackageInfo\endcsname\relax
        \def\x#1#2{%
          \immediate\write-1{Package #1 Info: #2.}%
        }%
      \else
        \def\x#1#2{\PackageInfo{#1}{#2, stopped}}%
      \fi
      \x{pdfrender}{The package is already loaded}%
      \aftergroup\endinput
    \fi
  \fi
\endgroup%
%    \end{macrocode}
%    Package identification:
%    \begin{macrocode}
\begingroup\catcode61\catcode48\catcode32=10\relax%
  \catcode13=5 % ^^M
  \endlinechar=13 %
  \catcode35=6 % #
  \catcode39=12 % '
  \catcode40=12 % (
  \catcode41=12 % )
  \catcode44=12 % ,
  \catcode45=12 % -
  \catcode46=12 % .
  \catcode47=12 % /
  \catcode58=12 % :
  \catcode64=11 % @
  \catcode91=12 % [
  \catcode93=12 % ]
  \catcode123=1 % {
  \catcode125=2 % }
  \expandafter\ifx\csname ProvidesPackage\endcsname\relax
    \def\x#1#2#3[#4]{\endgroup
      \immediate\write-1{Package: #3 #4}%
      \xdef#1{#4}%
    }%
  \else
    \def\x#1#2[#3]{\endgroup
      #2[{#3}]%
      \ifx#1\@undefined
        \xdef#1{#3}%
      \fi
      \ifx#1\relax
        \xdef#1{#3}%
      \fi
    }%
  \fi
\expandafter\x\csname ver@pdfrender.sty\endcsname
\ProvidesPackage{pdfrender}%
  [2019/12/29 v1.6 Access to some PDF graphics parameters (HO)]%
%    \end{macrocode}
%
%    \begin{macrocode}
\begingroup\catcode61\catcode48\catcode32=10\relax%
  \catcode13=5 % ^^M
  \endlinechar=13 %
  \catcode123=1 % {
  \catcode125=2 % }
  \catcode64=11 % @
  \def\x{\endgroup
    \expandafter\edef\csname PdfRender@AtEnd\endcsname{%
      \endlinechar=\the\endlinechar\relax
      \catcode13=\the\catcode13\relax
      \catcode32=\the\catcode32\relax
      \catcode35=\the\catcode35\relax
      \catcode61=\the\catcode61\relax
      \catcode64=\the\catcode64\relax
      \catcode123=\the\catcode123\relax
      \catcode125=\the\catcode125\relax
    }%
  }%
\x\catcode61\catcode48\catcode32=10\relax%
\catcode13=5 % ^^M
\endlinechar=13 %
\catcode35=6 % #
\catcode64=11 % @
\catcode123=1 % {
\catcode125=2 % }
\def\TMP@EnsureCode#1#2{%
  \edef\PdfRender@AtEnd{%
    \PdfRender@AtEnd
    \catcode#1=\the\catcode#1\relax
  }%
  \catcode#1=#2\relax
}
\TMP@EnsureCode{10}{12}% ^^J
\TMP@EnsureCode{36}{3}% $
\TMP@EnsureCode{39}{12}% '
\TMP@EnsureCode{40}{12}% (
\TMP@EnsureCode{41}{12}% )
\TMP@EnsureCode{42}{12}% *
\TMP@EnsureCode{43}{12}% +
\TMP@EnsureCode{44}{12}% ,
\TMP@EnsureCode{45}{12}% -
\TMP@EnsureCode{46}{12}% .
\TMP@EnsureCode{47}{12}% /
\TMP@EnsureCode{58}{12}% :
\TMP@EnsureCode{59}{12}% ;
\TMP@EnsureCode{60}{12}% <
\TMP@EnsureCode{62}{12}% >
\TMP@EnsureCode{63}{12}% ?
\TMP@EnsureCode{91}{12}% [
\TMP@EnsureCode{93}{12}% ]
\TMP@EnsureCode{94}{7}% ^ (superscript)
\TMP@EnsureCode{96}{12}% `
\TMP@EnsureCode{124}{12}% |
%    \end{macrocode}
%
%    \begin{macrocode}
\def\PdfRender@AtEndHook{}
\expandafter\def\expandafter\PdfRender@AtEnd\expandafter{%
  \expandafter\PdfRender@AtEndHook
  \PdfRender@AtEnd
  \endinput
}
%    \end{macrocode}
%
% \subsection{Look for \pdfTeX, its mode and features}
%
%    \begin{macro}{\PdfRender@newif}
%    \begin{macrocode}
\def\PdfRender@newif#1{%
  \expandafter\edef\csname PdfRender@#1true\endcsname{%
    \let
    \expandafter\noexpand\csname ifPdfRender@#1\endcsname
    \noexpand\iftrue
  }%
  \expandafter\edef\csname PdfRender@#1false\endcsname{%
    \let
    \expandafter\noexpand\csname ifPdfRender@#1\endcsname
    \noexpand\iffalse
  }%
  \csname PdfRender@#1false\endcsname
}
%    \end{macrocode}
%    \end{macro}
%
%    \begin{macro}{\ifPdfRender@Stack}
%    \begin{macrocode}
\PdfRender@newif{Stack}
%    \end{macrocode}
%    \end{macro}
%    \begin{macro}{\ifPdfRender@Match}
%    \begin{macrocode}
\PdfRender@newif{Match}
%    \end{macrocode}
%    \end{macro}
%
%    \begin{macro}{\PdfRender@RequirePackage}
%    \begin{macrocode}
\begingroup\expandafter\expandafter\expandafter\endgroup
\expandafter\ifx\csname RequirePackage\endcsname\relax
  \def\PdfRender@RequirePackage#1[#2]{%
    \expandafter\def\expandafter\PdfRender@AtEndHook\expandafter{%
      \PdfRender@AtEndHook
      \ltx@ifpackagelater{#1}{#2}{}{%
        \@PackageWarningNoLine{pdfrender}{%
          You have requested version\MessageBreak
          `#2' of package `#1',\MessageBreak
          but only version\MessageBreak
          `\csname ver@#1.\ltx@pkgextension\endcsname'\MessageBreak
          is available%
        }%
      }%
    }%
    \input #1.sty\relax
  }%
\else
  \let\PdfRender@RequirePackage\RequirePackage
\fi
%    \end{macrocode}
%    \end{macro}
% Luatex compatibility
%    \begin{macrocode}
\ifx\pdfextension\@undefined\else
  \def\pdfcolorstackinit {\pdffeedback colorstackinit}
  \protected\def\pdfcolorstack     {\pdfextension colorstack}
  \protected\def\pdfliteral        {\pdfextension literal}
\fi
%    \end{macrocode}
%
%    \begin{macrocode}
\PdfRender@RequirePackage{iftex}[2019/11/07]
\PdfRender@RequirePackage{infwarerr}[2007/09/09]
\PdfRender@RequirePackage{ltxcmds}[2010/01/28]
%    \end{macrocode}
%
%    \begin{macrocode}
\ifpdf
  \ltx@IfUndefined{pdfcolorstackinit}{%
    \@PackageWarning{pdfrender}{%
      Missing \string\pdfcolorstackinit
    }%
  }{%
    \PdfRender@Stacktrue
  }%
  \ltx@IfUndefined{pdfmatch}{%
    \@PackageInfoNoLine{pdfrender}{%
      \string\pdfmatch\ltx@space not found. %
      Therefore the values\MessageBreak
      of some parameters are not validated%
    }%
  }{%
    \PdfRender@Matchtrue
  }%
\else
  \@PackageWarning{pdfrender}{%
    Missing pdfTeX in PDF mode%
  }%
%    \end{macrocode}
%    \begin{macrocode}
  \ltx@IfUndefined{newcommand}{%
%    \end{macrocode}
%    \begin{macro}{\pdfrender}
%    \begin{macrocode}
    \def\pdfrender#1{}%
%    \end{macrocode}
%    \end{macro}
%    \begin{macro}{\textpdfrender}
%    \begin{macrocode}
    \long\def\textpdfrender#1#2{#2}%
%    \end{macrocode}
%    \end{macro}
%    \begin{macrocode}
  }{%
%    \end{macrocode}
%    \begin{macro}{\pdfrender}
%    \begin{macrocode}
    \newcommand*{\pdfrender}[1]{}%
%    \end{macrocode}
%    \end{macro}
%    \begin{macro}{\textpdfrender}
%    \begin{macrocode}
    \newcommand{\textpdfrender}[2]{#2}%
%    \end{macrocode}
%    \end{macro}
%    \begin{macrocode}
  }%
  \expandafter\PdfRender@AtEnd
\fi%
%    \end{macrocode}
%
% \subsection{Enable color support of \LaTeX}
%
%    \begin{macrocode}
\ltx@ifpackageloaded{color}{}{%
  \def\color@setgroup{\begingroup\set@color}%
  \let\color@begingroup\begingroup
  \def\color@endgroup{\endgraf\endgroup}%
  \def\color@hbox{\hbox\bgroup\color@begingroup}%
  \def\color@vbox{\vbox\bgroup\color@begingroup}%
  \def\color@endbox{\color@endgroup\egroup}%
  \ltx@ifundefined{bgroup}{%
    \let\bgroup={\let\egroup=}%
  }{}%
  \ltx@ifundefined{endgraf}{%
    \let\endgraf=\par
  }{}%
}
%    \end{macrocode}
%
% \subsection{Hook into \cs{normalcolor}}
%
%    The problem is that packages \xpackage{color} and \xpackage{xcolor}
%    each overwrite \cs{normalcolor}. For example, after the package
%    loading order \xpackage{color}, \xpackage{pdfrender} and \xpackage{xcolor}
%    the patched version of \cs{normalcolor} is overwritten by
%    package \xpackage{xcolor}. Also using \cs{AtBeginDocument} for
%    patching is not enough. If package \xpackage{hyperref} is loaded
%    later, it might load package \xpackage{color} using \cs{AtBeginDocument}.
%    \begin{macro}{\PdfRender@NormalColorHook}
%    \begin{macrocode}
\def\PdfRender@NormalColorHook{}
%    \end{macrocode}
%    \end{macro}
%    \begin{macro}{\PdfRender@ColorSetGroupHook}
%    \begin{macrocode}
\def\PdfRender@ColorSetGroupHook{}
%    \end{macrocode}
%    \end{macro}
%    \begin{macro}{\PdfRender@TestBox}
%    \begin{macrocode}
\def\PdfRender@TestBox#1{%
  \setbox0=\color@hbox#1\color@endbox
}
%    \end{macrocode}
%    \end{macro}
%    \begin{macro}{\PdfRender@PatchNormalColor}
%    \begin{macrocode}
\def\PdfRender@PatchNormalColor{%
  \ltx@ifundefined{normalcolor}{%
    \gdef\normalcolor{\PdfRender@NormalColorHook}%
  }{%
    \begingroup
      \def\PdfRender@NormalColorHook{\let\PdfRender@temp=Y}%
      \PdfRender@TestBox{%
        \let\set@color\relax
        \normalcolor
        \ifx\PdfRender@temp Y%
        \else
          \ltx@GlobalAppendToMacro\normalcolor{%
            \PdfRender@NormalColorHook
          }%
        \fi
      }%
    \endgroup
  }%
  \ifx\@nodocument\relax
    \global\let\PdfRender@PatchNormalColor\relax
  \fi
}%
%    \end{macrocode}
%    \end{macro}
%    \begin{macro}{\PdfRender@PatchColorSetGroup}
%    \begin{macrocode}
\def\PdfRender@PatchColorSetGroup{%
  \begingroup
    \def\PdfRender@ColorSetGroupHook{\let\PdfRender@temp=Y}%
    \PdfRender@TestBox{%
      \let\set@color\relax
      \color@setgroup\color@endgroup
      \ifx\PdfRender@temp Y%
      \else
        \ltx@GlobalAppendToMacro\color@setgroup{%
          \PdfRender@ColorSetGroupHook
        }%
      \fi
    }%
  \endgroup
  \ifx\@nodocument\relax
    \global\let\PdfRender@PatchColorSetGroup\relax
  \fi
}%
%    \end{macrocode}
%    \end{macro}
%    \begin{macro}{\PdfRender@PatchColor}
%    \begin{macrocode}
\def\PdfRender@PatchColor{%
  \PdfRender@PatchNormalColor
  \PdfRender@PatchColorSetGroup
}
%    \end{macrocode}
%    \end{macro}
%    \begin{macrocode}
\PdfRender@PatchColor
\ltx@IfUndefined{AtBeginDocument}{}{%
  \AtBeginDocument{\PdfRender@PatchColor}%
}
%    \end{macrocode}
%    \cs{AfterPackage} is provided by package \xpackage{scrlfile}.
%    \begin{macrocode}
\ltx@IfUndefined{AfterPackage}{%
}{%
  \AfterPackage{color}{\PdfRender@PatchColor}%
  \AfterPackage{xcolor}{\PdfRender@PatchColor}%
  \AfterPackage{etoolbox}{%
    \AfterEndPreamble{\PdfRender@PatchColor}%
  }%
}%
%    \end{macrocode}
%    \cs{AfterEndPreamble} is provided by package \xpackage{etoolbox}.
%    \begin{macrocode}
\ltx@IfUndefined{AfterEndPreamble}{%
}{%
  \AfterEndPreamble{\PdfRender@PatchColor}%
}%
%    \end{macrocode}
%
%    \begin{macrocode}
\PdfRender@RequirePackage{kvsetkeys}[2010/01/28]
%    \end{macrocode}
%    \begin{macro}{\PdfRender@texorpdfstring}
%    \begin{macrocode}
\def\PdfRender@texorpdfstring{%
  \ltx@IfUndefined{texorpdfstring}\ltx@firstoftwo\texorpdfstring
}
%    \end{macrocode}
%    \end{macro}
%    \begin{macro}{\pdfrender}
%    \begin{macrocode}
\ltx@IfUndefined{DeclareRobustCommand}%
\ltx@firstoftwo\ltx@secondoftwo
{%
  \def\pdfrender#1%
}{%
  \newcommand{\pdfrender}{}%
  \DeclareRobustCommand*{\pdfrender}[1]%
}%
{%
  \PdfRender@texorpdfstring{%
    \PdfRender@PatchNormalColor
    \global\let\PdfRender@FillColor\ltx@empty
    \global\let\PdfRender@StrokeColor\ltx@empty
    \kvsetkeys{PDFRENDER}{#1}%
    \PdfRender@SetColor
  }{}%
}
%    \end{macrocode}
%    \end{macro}
%    \begin{macro}{\textpdfrender}
%    \begin{macrocode}
\ltx@IfUndefined{DeclareRobustCommand}%
\ltx@firstoftwo\ltx@secondoftwo
{%
  \long\def\textpdfrender#1#2%
}{%
  \newcommand{\textpdfrender}{}%
  \DeclareRobustCommand{\textpdfrender}[2]%
}%
{%
  \PdfRender@texorpdfstring{%
    \begingroup
      \pdfrender{#1}%
      #2%
    \endgroup
  }{#2}%
}
%    \end{macrocode}
%    \end{macro}
%
%    \begin{macro}{\ifPdfRender@Values}
%    \begin{macrocode}
\PdfRender@newif{Values}
%    \end{macrocode}
%    \end{macro}
%    \begin{macro}{\PdfRender@NewClassValues}
%    \begin{macrocode}
\def\PdfRender@NewClassValues#1#2#3#4{%
  \PdfRender@Valuestrue
  \PdfRender@NewClass{#1}{#2}{#3}{#4}{}%
}
%    \end{macrocode}
%    \end{macro}
%    \begin{macro}{\PdfRender@NewClass}
%    \begin{macrocode}
\def\PdfRender@NewClass#1#2#3#4#5{%
  \PdfRender@newif{Active#1}%
  \expandafter\def\csname PdfRender@Default#1\endcsname{#2}%
  \expandafter\let\csname PdfRender@Current#1\expandafter\endcsname
      \csname PdfRender@Default#1\endcsname
  \ifPdfRender@Stack
    \expandafter\edef\csname PdfRender@Init#1\endcsname{%
      \global\chardef
      \expandafter\noexpand\csname PdfRender@Stack#1\endcsname=%
          \noexpand\pdfcolorstackinit page direct{%
            \noexpand#3%
            \expandafter\noexpand\csname PdfRender@Default#1\endcsname
          }\relax
      \noexpand\@PackageInfo{pdfrender}{%
        New color stack `#1' = \noexpand\number
        \expandafter\noexpand\csname PdfRender@Stack#1\endcsname
      }%
      \gdef\expandafter\noexpand\csname PdfRender@Init#1\endcsname{}%
    }%
    \expandafter\edef\csname PdfRender@Set#1\endcsname{%
      \expandafter\noexpand\csname PdfRender@Init#1\endcsname
      \noexpand\pdfcolorstack
      \expandafter\noexpand\csname PdfRender@Stack#1\endcsname
      push{%
        #3{\expandafter\noexpand\csname PdfRender@Current#1\endcsname}%
      }%
      \noexpand\aftergroup
      \expandafter\noexpand\csname PdfRender@Reset#1\endcsname
    }%
    \expandafter\edef\csname PdfRender@Reset#1\endcsname{%
      \expandafter\noexpand\csname PdfRender@Init#1\endcsname
      \noexpand\pdfcolorstack
      \expandafter\noexpand\csname PdfRender@Stack#1\endcsname
      pop\relax
    }%
  \else
    \expandafter\edef\csname PdfRender@Set#1\endcsname{%
      \noexpand\pdfliteral direct{%
        #3{\expandafter\noexpand\csname PdfRender@Current#1\endcsname}%
      }%
      \noexpand\aftergroup
      \expandafter\noexpand\csname PdfRender@Reset#1\endcsname
    }%
    \expandafter\edef\csname PdfRender@Reset#1\endcsname{%
      \noexpand\pdfliteral direct{%
        #3{\expandafter\noexpand\csname PdfRender@Current#1\endcsname}%
      }%
    }%
  \fi
  \expandafter\edef\csname PdfRender@Normal#1\endcsname{%
    \let
    \expandafter\noexpand\csname PdfRender@Current#1\endcsname
    \expandafter\noexpand\csname PdfRender@Default#1\endcsname
    \noexpand\PdfRender@Set{#1}%
  }%
  \expandafter\ltx@GlobalAppendToMacro\expandafter\PdfRender@NormalColorHook
  \expandafter{%
    \csname PdfRender@Normal#1\endcsname
  }%
  \ltx@GlobalAppendToMacro\PdfRender@ColorSetGroupHook{%
    \PdfRender@Set{#1}%
  }%
  \ifPdfRender@Values
    \kv@parse@normalized{#4}{%
      \expandafter\let\csname PdfRender@#1@\kv@key\endcsname\kv@key
      \ifx\kv@value\relax
      \else
        \expandafter\let\csname PdfRender@#1@\kv@value\endcsname\kv@key
      \fi
      \ltx@gobbletwo
    }%
    \PdfRender@define@key{PDFRENDER}{#1}{%
      \global\csname PdfRender@Active#1true\endcsname
      \def\PdfRender@Current{##1}%
      \PdfRender@SetValidateValues{#1}%
    }%
    \PdfRender@Valuesfalse
  \else
    \PdfRender@define@key{PDFRENDER}{#1}{%
      \global\csname PdfRender@Active#1true\endcsname
      \expandafter\def\csname PdfRender@Current#1\endcsname{##1}%
      \ltx@IfUndefined{PdfRender@PostProcess#1}{%
      }{%
        \csname PdfRender@PostProcess#1\endcsname
      }%
      \PdfRender@SetValidate{#1}{#4}{#5}%
    }%
  \fi
}%
%    \end{macrocode}
%    \end{macro}
%    \begin{macro}{\PdfRender@define@key}
%    \begin{macrocode}
\ltx@IfUndefined{define@key}{%
  \def\PdfRender@define@key#1#2{%
    \expandafter\def\csname KV@#1@#2\endcsname##1%
  }%
}{%
  \let\PdfRender@define@key\define@key
}
%    \end{macrocode}
%    \end{macro}
%    \begin{macro}{\PdfRender@Set}
%    \begin{macrocode}
\def\PdfRender@Set#1{%
  \csname ifPdfRender@Active#1\endcsname
    \csname PdfRender@Set#1\expandafter\endcsname
  \fi
}
%    \end{macrocode}
%    \end{macro}
%    \begin{macro}{\PdfRender@Reset}
%    \begin{macrocode}
\def\PdfRender@Reset#1{%
  \csname ifPdfRender@Active#1\endcsname
    \csname PdfRender@Reset#1\expandafter\endcsname
  \fi
}
%    \end{macrocode}
%    \end{macro}
%    \begin{macro}{\PdfRender@ErrorInvalidValue}
%    \begin{macrocode}
\def\PdfRender@ErrorInvalidValue#1{%
  \PackageError{pdfrender}{%
    Ignoring parameter setting for `#1'\MessageBreak
    because of invalid value %
    `\csname PdfRender@Current#1\endcsname'%
  }\@ehc
  \expandafter\let\csname PdfRender@Current#1\endcsname\ltx@empty
}%
%    \end{macrocode}
%    \end{macro}
%    \begin{macro}{\PdfRender@SetValidate}
%    \begin{macrocode}
\ifPdfRender@Match
  \def\PdfRender@SetValidate#1#2#3{%
    \ifnum\pdfmatch{^(#2)$}{\csname PdfRender@Current#1\endcsname}=1 %
      \csname PdfRender@Set#1\expandafter\endcsname
    \else
      \PdfRender@ErrorInvalidValue{#1}%
    \fi
  }%
\else
  \def\PdfRender@SetValidate#1#2#3{%
    \expandafter\let\expandafter\PdfRender@Current
    \csname PdfRender@Current#1\endcsname
    #3%
    \ifx\PdfRender@Current\@empty
      \PdfRender@ErrorInvalidValue{#1}%
    \else
      \csname PdfRender@Set#1\expandafter\endcsname
    \fi
  }%
\fi
%    \end{macrocode}
%    \end{macro}
%    \begin{macro}{\PdfRender@SetValidateValues}
%    \begin{macrocode}
\def\PdfRender@SetValidateValues#1{%
  \ltx@IfUndefined{PdfRender@#1@\PdfRender@Current}{%
    \expandafter\let\csname PdfRender@Current#1\endcsname
                    \PdfRender@Current
    \PdfRender@ErrorInvalidValue{#1}%
  }{%
    \expandafter\edef\csname PdfRender@Current#1\endcsname{%
      \csname PdfRender@#1@\PdfRender@Current\endcsname
    }%
    \csname PdfRender@Set#1\endcsname
  }%
}
%    \end{macrocode}
%    \end{macro}
%    \begin{macro}{\PdfRender@OpValue}
%    \begin{macrocode}
\def\PdfRender@OpValue#1#2{#2\ltx@space#1}%
%    \end{macrocode}
%    \end{macro}
%    \begin{macro}{\PdfRender@OpName}
%    \begin{macrocode}
\def\PdfRender@OpName#1#2{/#2\ltx@space#1}%
%    \end{macrocode}
%    \end{macro}
%
% \subsection{Declare and setup parameters}
%
%    \begin{macrocode}
\PdfRender@NewClassValues{TextRenderingMode}%
                      {0}%
                      {\PdfRender@OpValue{Tr}}{%
  0=Fill,%
  1=Stroke,%
  2=FillStroke,%
  3=Invisible,%
  4=FillClip,%
  5=StrokeClip,%
  6=FillStrokeClip,%
  7=Clip,%
}%
\PdfRender@NewClass{LineWidth}{1}{\PdfRender@OpValue{w}}{%
  [0-9]+\string\.?[0-9]*|\string\.[0-9]+%
}{}%
\ltx@IfUndefined{dimexpr}{%
  \def\PdfRender@dimexpr{}%
}{%
  \let\PdfRender@dimexpr\dimexpr
}
\def\PdfRender@PostProcessLineWidth{%
  \begingroup
  \afterassignment\PdfRender@@PostProcessLineWidth
  \dimen0=\PdfRender@dimexpr\PdfRender@CurrentLineWidth bp %
  \PdfRender@let\PdfRender@relax\PdfRender@relax
}
\let\PdfRender@let\let
\let\PdfRender@relax\relax
\def\PdfRender@@PostProcessLineWidth#1\PdfRender@let{%
  \ifx\\#1\\%
    \endgroup
  \else
    \dimen0=.996264\dimen0 % 72/72.27
    \edef\x{\endgroup
      \def\noexpand\PdfRender@CurrentLineWidth{%
        \strip@pt\dimen0%
      }%
    }%
    \expandafter\x
  \fi
}
\PdfRender@NewClassValues{LineCapStyle}{0}{\PdfRender@OpValue{J}}{%
  0=Butt,%
  1=Round,%
  2=ProjectingSquare,%
}%
\PdfRender@NewClassValues{LineJoinStyle}{0}{\PdfRender@OpValue{j}}{%
  0=Miter,%
  1=Round,%
  2=Bevel,%
}%
\PdfRender@NewClass{MiterLimit}{10}{\PdfRender@OpValue{M}}{%
  [0-9]*[1-9][0-9]*\string\.?[0-9]*|%
  [0-9]*\string\.?[0-9]*[1-9][0-9]*%
}{}%
\PdfRender@NewClass{Flatness}{0}{\PdfRender@OpValue{i}}{%
  100(\string\.0*)?|[0-9][0-9](\string\.[0-9]*)?|\string\.[0-9]+%
}{}%
\PdfRender@NewClass{LineDashPattern}{[]0}{\PdfRender@OpValue{d}}{%
  \string\[%
  ( ?([0-9]+\string\.?[0-9]*|\string\.[0-9]+) ?)*%
  \string\] ?%
  ([0-9]+\string\.?[0-9]*|\string\.[0-9]+)%
}{}%
\PdfRender@NewClassValues{RenderingIntent}%
                      {RelativeColorimetric}%
                      {\PdfRender@OpName{ri}}{%
  AbsoluteColorimetric,%
  RelativeColorimetric,%
  Saturation,%
  Perceptual,%
}%
%    \end{macrocode}
%
% \subsection{Fill and stroke color support}
%
%    \begin{macrocode}
\PdfRender@define@key{PDFRENDER}{FillColor}{%
  \begingroup
    \def\PdfRender@Color{#1}%
    \ifx\PdfRender@Color\ltx@empty
      \global\let\PdfRender@FillColor\ltx@empty
    \else
      \PdfRender@ColorAvailable{%
        \PdfRender@TestBox{%
          \expandafter\PdfRender@TryColor\PdfRender@Color\ltx@empty
          \PdfRender@GetFillColor
          \ifx\PdfRender@FillColor\ltx@empty
            \@PackageWarning{pdfrender}{%
              Cannot extract fill color\MessageBreak
              from value `#1'%
            }%
          \fi
        }%
      }%
    \fi
  \endgroup
}
\PdfRender@define@key{PDFRENDER}{StrokeColor}{%
  \begingroup
    \def\PdfRender@Color{#1}%
    \ifx\PdfRender@Color\ltx@empty
      \global\let\PdfRender@StrokeColor\ltx@empty
    \else
      \PdfRender@ColorAvailable{%
        \PdfRender@TestBox{%
          \expandafter\PdfRender@TryColor\PdfRender@Color\ltx@empty
          \PdfRender@GetStrokeColor
          \ifx\PdfRender@StrokeColor\ltx@empty
            \@PackageWarning{pdfrender}{%
              Cannot extract stroke color\MessageBreak
              from value `#1'%
            }%
          \fi
        }%
      }%
    \fi
  \endgroup
}
%    \end{macrocode}
%    \begin{macro}{\PdfRender@ColorAvailable}
%    \begin{macrocode}
\def\PdfRender@ColorAvailable{%
  \@ifundefined{set@color}{%
    \@PackageError{pdfrender}{%
      Ignoring color options, because neither\MessageBreak
      package `color' nor package `xcolor' is loaded%
    }\@ehc
    \global\let\PdfRender@ColorAvailable\ltx@gobble
  }{%
    \global\let\PdfRender@ColorAvailable\ltx@firstofone
  }%
  \PdfRender@ColorAvailable
}
%    \end{macrocode}
%    \end{macro}
%    \begin{macro}{\PdfRender@TryColor}
%    \begin{macrocode}
\def\PdfRender@TryColor{%
  \@ifnextchar[\color\PdfRender@@TryColor
}
%    \end{macrocode}
%    \end{macro}
%    \begin{macro}{\PdfRender@@TryColor}
%    \begin{macrocode}
\def\PdfRender@@TryColor#1\ltx@empty{%
  \expandafter\color\expandafter{\PdfRender@Color}%
}
%    \end{macrocode}
%    \end{macro}
%    \begin{macro}{\PdfRender@SetColor}
%    \begin{macrocode}
\def\PdfRender@SetColor{%
  \chardef\PdfRender@NeedsCurrentColor=0 %
  \ifx\PdfRender@FillColor\ltx@empty
    \ifx\PdfRender@StrokeColor\ltx@empty
    \else
      \edef\PdfRender@CurrentColor{%
        \noexpand\PdfRender@FillColor\ltx@space\PdfRender@StrokeColor
      }%
      \chardef\PdfRender@NeedsCurrentColor=1 %
    \fi
  \else
    \ifx\PdfRender@StrokeColor\ltx@empty
      \edef\PdfRender@CurrentColor{%
        \PdfRender@FillColor\ltx@space\noexpand\PdfRender@StrokeColor
      }%
      \chardef\PdfRender@NeedsCurrentColor=2 %
    \else
      \edef\current@color{%
        \PdfRender@FillColor\ltx@space\PdfRender@StrokeColor
      }%
      \set@color
    \fi
  \fi
  \ifnum\PdfRender@NeedsCurrentColor=1 %
    \PdfRender@GetFillColor
    \ifx\PdfRender@FillColor\ltx@empty
      \@PackageWarning{pdfrender}{%
        Cannot extract current fill color%
      }%
    \else
      \edef\current@color{\PdfRender@CurrentColor}%
      \set@color
    \fi
  \else
    \ifnum\PdfRender@NeedsCurrentColor=2 %
      \PdfRender@GetStrokeColor
      \ifx\PdfRender@StrokeColor\ltx@empty
        \@PackageWarning{pdfrender}{%
          Cannot extract current stroke color%
        }%
      \else
        \edef\current@color{\PdfRender@CurrentColor}%
        \set@color
      \fi
    \fi
  \fi
}
%    \end{macrocode}
%    \end{macro}
%    \begin{macro}{\PdfRender@PatternFillColor}
%    \begin{macrocode}
\edef\PdfRender@PatternFillColor{ % space
  (%
    [0-9\string\.]+ g|%
    [0-9\string\.]+ [0-9\string\.]+ [0-9\string\.]+ rg|%
    [0-9\string\.]+ [0-9\string\.]+ %
    [0-9\string\.]+ [0-9\string\.]+ k%
  ) % space
  (.*)$%
}
%    \end{macrocode}
%    \end{macro}
%    \begin{macro}{\PdfRender@PatternStrokeColor}
%    \begin{macrocode}
\edef\PdfRender@PatternStrokeColor{ % space
  (%
    [0-9\string\.]+ G|%
    [0-9\string\.]+ [0-9\string\.]+ [0-9\string\.]+ RG|%
    [0-9\string\.]+ [0-9\string\.]+ %
    [0-9\string\.]+ [0-9\string\.]+ K%
  ) % space
  (.*)$%
}
%    \end{macrocode}
%    \end{macro}
%    \begin{macro}{\PdfRender@MatchPattern}
%    \begin{macrocode}
\def\PdfRender@MatchPattern#1{%
  \ifnum\pdfmatch{\PdfRender@Pattern}{\PdfRender@String}=1 %
    \xdef#1{%
      \expandafter\strip@prefix\pdflastmatch 1%
    }%
    \edef\PdfRender@String{%
      \expandafter\strip@prefix\pdflastmatch 2%
    }%
    \ifx\PdfRender@String\ltx@empty
    \else
      \expandafter\expandafter\expandafter\PdfRender@MatchPattern
      \expandafter\expandafter\expandafter#1%
    \fi
  \fi
}
%    \end{macrocode}
%    \end{macro}
%    \begin{macro}{\PdfRender@GetFillColor}
%    \begin{macrocode}
\def\PdfRender@GetFillColor{%
  \global\let\PdfRender@FillColor\ltx@empty
  \begingroup
    \ifPdfRender@Match
      \let\PdfRender@Pattern\PdfRender@PatternFillColor
      \edef\PdfRender@String{\ltx@space\current@color\ltx@space}%
      \PdfRender@MatchPattern\PdfRender@FillColor
    \else
      \edef\current@color{\current@color\ltx@space}%
      \let\PdfRender@OP\relax
      \PdfRender@FindOp{g}0%
      \PdfRender@FindOp{G}1%
      \PdfRender@FindOp{rg}0%
      \PdfRender@FindOp{RG}1%
      \PdfRender@FindOp{k}0%
      \PdfRender@FindOp{K}1%
      \PdfRender@FilterOp 0\PdfRender@FillColor
    \fi
  \endgroup
}
%    \end{macrocode}
%    \end{macro}
%    \begin{macro}{\PdfRender@GetStrokeColor}
%    \begin{macrocode}
\def\PdfRender@GetStrokeColor{%
  \global\let\PdfRender@StrokeColor\ltx@empty
  \begingroup
    \ifPdfRender@Match
      \let\PdfRender@Pattern\PdfRender@PatternStrokeColor
      \edef\PdfRender@String{\ltx@space\current@color\ltx@space}%
      \PdfRender@MatchPattern\PdfRender@StrokeColor
    \else
      \edef\current@color{\current@color\ltx@space}%
      \let\PdfRender@OP\relax
      \PdfRender@FindOp{g}0%
      \PdfRender@FindOp{G}1%
      \PdfRender@FindOp{rg}0%
      \PdfRender@FindOp{RG}1%
      \PdfRender@FindOp{k}0%
      \PdfRender@FindOp{K}1%
      \PdfRender@FilterOp 1\PdfRender@StrokeColor
    \fi
  \endgroup
}
%    \end{macrocode}
%    \end{macro}
%    \begin{macrocode}
\ifPdfRender@Match
  \expandafter\PdfRender@AtEnd
\fi%
%    \end{macrocode}
%    \begin{macro}{\PdfRender@FindOp}
%    \begin{macrocode}
\def\PdfRender@FindOp#1#2{%
  \def\PdfRender@temp##1 #1 ##2\@nil{%
    ##1%
    \ifx\\##2\\%
      \expandafter\@gobble
    \else
      \PdfRender@OP{#1}#2%
      \expandafter\@firstofone
    \fi
    {%
      \PdfRender@temp##2\@nil
    }%
  }%
  \edef\current@color{%
    \@firstofone{\expandafter\PdfRender@temp\current@color} #1 \@nil
  }%
}
%    \end{macrocode}
%    \end{macro}
%    \begin{macro}{\PdfRender@FilterOp}
%    \begin{macrocode}
\def\PdfRender@FilterOp#1#2{%
  \expandafter\PdfRender@@FilterOp\expandafter#1\expandafter#2%
      \current@color\PdfRender@OP{}{}%
}
%    \end{macrocode}
%    \end{macro}
%    \begin{macro}{\PdfRender@@FilterOp}
%    \begin{macrocode}
\def\PdfRender@@FilterOp#1#2#3\PdfRender@OP#4#5{%
  \ifx\\#4#5\\%
  \else
    \ifnum#1=#5 %
      \xdef#2{#3 #4}%
    \fi
    \expandafter\PdfRender@@FilterOp\expandafter#1\expandafter#2%
  \fi
}
%    \end{macrocode}
%    \end{macro}
%
%    \begin{macrocode}
\PdfRender@AtEnd%
%</package>
%    \end{macrocode}
%% \section{Installation}
%
% \subsection{Download}
%
% \paragraph{Package.} This package is available on
% CTAN\footnote{\CTANpkg{pdfrender}}:
% \begin{description}
% \item[\CTAN{macros/latex/contrib/oberdiek/pdfrender.dtx}] The source file.
% \item[\CTAN{macros/latex/contrib/oberdiek/pdfrender.pdf}] Documentation.
% \end{description}
%
%
% \paragraph{Bundle.} All the packages of the bundle `oberdiek'
% are also available in a TDS compliant ZIP archive. There
% the packages are already unpacked and the documentation files
% are generated. The files and directories obey the TDS standard.
% \begin{description}
% \item[\CTANinstall{install/macros/latex/contrib/oberdiek.tds.zip}]
% \end{description}
% \emph{TDS} refers to the standard ``A Directory Structure
% for \TeX\ Files'' (\CTANpkg{tds}). Directories
% with \xfile{texmf} in their name are usually organized this way.
%
% \subsection{Bundle installation}
%
% \paragraph{Unpacking.} Unpack the \xfile{oberdiek.tds.zip} in the
% TDS tree (also known as \xfile{texmf} tree) of your choice.
% Example (linux):
% \begin{quote}
%   |unzip oberdiek.tds.zip -d ~/texmf|
% \end{quote}
%
% \subsection{Package installation}
%
% \paragraph{Unpacking.} The \xfile{.dtx} file is a self-extracting
% \docstrip\ archive. The files are extracted by running the
% \xfile{.dtx} through \plainTeX:
% \begin{quote}
%   \verb|tex pdfrender.dtx|
% \end{quote}
%
% \paragraph{TDS.} Now the different files must be moved into
% the different directories in your installation TDS tree
% (also known as \xfile{texmf} tree):
% \begin{quote}
% \def\t{^^A
% \begin{tabular}{@{}>{\ttfamily}l@{ $\rightarrow$ }>{\ttfamily}l@{}}
%   pdfrender.sty & tex/generic/oberdiek/pdfrender.sty\\
%   pdfrender.pdf & doc/latex/oberdiek/pdfrender.pdf\\
%   pdfrender.dtx & source/latex/oberdiek/pdfrender.dtx\\
% \end{tabular}^^A
% }^^A
% \sbox0{\t}^^A
% \ifdim\wd0>\linewidth
%   \begingroup
%     \advance\linewidth by\leftmargin
%     \advance\linewidth by\rightmargin
%   \edef\x{\endgroup
%     \def\noexpand\lw{\the\linewidth}^^A
%   }\x
%   \def\lwbox{^^A
%     \leavevmode
%     \hbox to \linewidth{^^A
%       \kern-\leftmargin\relax
%       \hss
%       \usebox0
%       \hss
%       \kern-\rightmargin\relax
%     }^^A
%   }^^A
%   \ifdim\wd0>\lw
%     \sbox0{\small\t}^^A
%     \ifdim\wd0>\linewidth
%       \ifdim\wd0>\lw
%         \sbox0{\footnotesize\t}^^A
%         \ifdim\wd0>\linewidth
%           \ifdim\wd0>\lw
%             \sbox0{\scriptsize\t}^^A
%             \ifdim\wd0>\linewidth
%               \ifdim\wd0>\lw
%                 \sbox0{\tiny\t}^^A
%                 \ifdim\wd0>\linewidth
%                   \lwbox
%                 \else
%                   \usebox0
%                 \fi
%               \else
%                 \lwbox
%               \fi
%             \else
%               \usebox0
%             \fi
%           \else
%             \lwbox
%           \fi
%         \else
%           \usebox0
%         \fi
%       \else
%         \lwbox
%       \fi
%     \else
%       \usebox0
%     \fi
%   \else
%     \lwbox
%   \fi
% \else
%   \usebox0
% \fi
% \end{quote}
% If you have a \xfile{docstrip.cfg} that configures and enables \docstrip's
% TDS installing feature, then some files can already be in the right
% place, see the documentation of \docstrip.
%
% \subsection{Refresh file name databases}
%
% If your \TeX~distribution
% (\TeX\,Live, \mikTeX, \dots) relies on file name databases, you must refresh
% these. For example, \TeX\,Live\ users run \verb|texhash| or
% \verb|mktexlsr|.
%
% \subsection{Some details for the interested}
%
% \paragraph{Unpacking with \LaTeX.}
% The \xfile{.dtx} chooses its action depending on the format:
% \begin{description}
% \item[\plainTeX:] Run \docstrip\ and extract the files.
% \item[\LaTeX:] Generate the documentation.
% \end{description}
% If you insist on using \LaTeX\ for \docstrip\ (really,
% \docstrip\ does not need \LaTeX), then inform the autodetect routine
% about your intention:
% \begin{quote}
%   \verb|latex \let\install=y\input{pdfrender.dtx}|
% \end{quote}
% Do not forget to quote the argument according to the demands
% of your shell.
%
% \paragraph{Generating the documentation.}
% You can use both the \xfile{.dtx} or the \xfile{.drv} to generate
% the documentation. The process can be configured by the
% configuration file \xfile{ltxdoc.cfg}. For instance, put this
% line into this file, if you want to have A4 as paper format:
% \begin{quote}
%   \verb|\PassOptionsToClass{a4paper}{article}|
% \end{quote}
% An example follows how to generate the
% documentation with pdf\LaTeX:
% \begin{quote}
%\begin{verbatim}
%pdflatex pdfrender.dtx
%makeindex -s gind.ist pdfrender.idx
%pdflatex pdfrender.dtx
%makeindex -s gind.ist pdfrender.idx
%pdflatex pdfrender.dtx
%\end{verbatim}
% \end{quote}
%
% \section{Acknowledgement}
%
% \begin{description}
% \item[Friedrich Vosberg] asked in the
%   newsgroup \textsf{de.comp.text.tex} for the font
%   outline feature \cite{news-Vosberg}.
% \item[Gaius Pupus] proposed the basic method
%   using \cs{pdfliteral} in this thread \cite{news-Pupus}.
% \item[Rolf Niepraschk] added color support \cite{news-Niepraschk}.
% \end{description}
%
% \begin{thebibliography}{9}
% \raggedright
%
% \bibitem{pdf-spec}
%   Adobe Systems Incorporated. \textit{PDF Reference --
%   Adobe Portable Document format -- Version 1.7}.
%   6th ed. 2006.
%   \textsc{URL:}\,\url{http://www.adobe.com/devnet/acrobat/pdfs/pdf_reference_1-7.pdf}.
%
% \bibitem{news-Vosberg}
%   Friedrich Vosberg, \textit{Text in Buchstabenumrissen},
%   de.comp.text.tex, 2010-01-22.
%   \textsc{URL:}\,\url{https://groups.google.com/group/de.comp.text.tex/msg/f442310ac8b2d506}.
%
% \bibitem{news-Pupus}
%   Gaius Pupus, \textit{Re: Text in Buchstabenumrissen},
%   de.comp.text.tex, 2010-01-23.
%   \textsc{URL:}\,\url{https://groups.google.com/group/de.comp.text.tex/msg/95d890d77ac47eb1}.
%
% \bibitem{news-Niepraschk}
%   Rolf Niepraschk, \textit{Re: Text in Buchstabenumrissen},
%   de.comp.text.tex, 2010-01-24.
%   \textsc{URL:}\,\url{https://groups.google.com/group/de.comp.text.tex/msg/4eb61a5879db54db}.
%
% \end{thebibliography}
%
% \begin{History}
%   \begin{Version}{2010/01/26 v1.0}
%   \item
%     The first version.
%   \end{Version}
%   \begin{Version}{2010/01/27 v1.1}
%   \item
%     Macros \cs{pdfrender} and \cs{textpdfrender} are made robust.
%   \item
%     Color extraction rewritten for the case that \cs{pdfmatch}
%     is not available. This fixes wrong color assigments in
%     case of nesting.
%   \item
%     Color extraction of case \cs{pdfmatch} is fixed
%     for the case that the color string contains several
%     fill or several stroke operations.
%   \end{Version}
%   \begin{Version}{2010/01/28 v1.2}
%   \item
%     Dependency from package \xpackage{color} is removed.
%   \item
%     Compatibility for \plainTeX\ and even ini\TeX\ added.
%   \end{Version}
%   \begin{Version}{2016/05/14 v1.3}
%   \item
%     Use package \xpackage{luatex85} for compatibility with new LuaTeX.
%   \end{Version}
%   \begin{Version}{2016/05/17 v1.4}
%   \item
%     Documentation updates.
%   \item adjust luatex85 reference so that it works in plain TeX.
%   \end{Version}
%   \begin{Version}{2018/11/01 v1.5}
%   \item
%     Remove luatex85 dependency
%   \end{Version}
%   \begin{Version}{2019/12/29 v1.6}
%   \item
%     \xpackage{iftex} package.
%   \end{Version}
% \end{History}
%
% \PrintIndex
%
% \Finale
\endinput

%        (quote the arguments according to the demands of your shell)
%
% Documentation:
%    (a) If pdfrender.drv is present:
%           latex pdfrender.drv
%    (b) Without pdfrender.drv:
%           latex pdfrender.dtx; ...
%    The class ltxdoc loads the configuration file ltxdoc.cfg
%    if available. Here you can specify further options, e.g.
%    use A4 as paper format:
%       \PassOptionsToClass{a4paper}{article}
%
%    Programm calls to get the documentation (example):
%       pdflatex pdfrender.dtx
%       makeindex -s gind.ist pdfrender.idx
%       pdflatex pdfrender.dtx
%       makeindex -s gind.ist pdfrender.idx
%       pdflatex pdfrender.dtx
%
% Installation:
%    TDS:tex/generic/oberdiek/pdfrender.sty
%    TDS:doc/latex/oberdiek/pdfrender.pdf
%    TDS:source/latex/oberdiek/pdfrender.dtx
%
%<*ignore>
\begingroup
  \catcode123=1 %
  \catcode125=2 %
  \def\x{LaTeX2e}%
\expandafter\endgroup
\ifcase 0\ifx\install y1\fi\expandafter
         \ifx\csname processbatchFile\endcsname\relax\else1\fi
         \ifx\fmtname\x\else 1\fi\relax
\else\csname fi\endcsname
%</ignore>
%<*install>
\input docstrip.tex
\Msg{************************************************************************}
\Msg{* Installation}
\Msg{* Package: pdfrender 2019/12/29 v1.6 Access to some PDF graphics parameters (HO)}
\Msg{************************************************************************}

\keepsilent
\askforoverwritefalse

\let\MetaPrefix\relax
\preamble

This is a generated file.

Project: pdfrender
Version: 2019/12/29 v1.6

Copyright (C)
   2010 Heiko Oberdiek
   2016-2019 Oberdiek Package Support Group

This work may be distributed and/or modified under the
conditions of the LaTeX Project Public License, either
version 1.3c of this license or (at your option) any later
version. This version of this license is in
   https://www.latex-project.org/lppl/lppl-1-3c.txt
and the latest version of this license is in
   https://www.latex-project.org/lppl.txt
and version 1.3 or later is part of all distributions of
LaTeX version 2005/12/01 or later.

This work has the LPPL maintenance status "maintained".

The Current Maintainers of this work are
Heiko Oberdiek and the Oberdiek Package Support Group
https://github.com/ho-tex/oberdiek/issues


The Base Interpreter refers to any `TeX-Format',
because some files are installed in TDS:tex/generic//.

This work consists of the main source file pdfrender.dtx
and the derived files
   pdfrender.sty, pdfrender.pdf, pdfrender.ins, pdfrender.drv,
   pdfrender-test1.tex, pdfrender-test2.tex,
   pdfrender-test3.tex, pdfrender-test4.tex,
   pdfrender-test5.tex.

\endpreamble
\let\MetaPrefix\DoubleperCent

\generate{%
  \file{pdfrender.ins}{\from{pdfrender.dtx}{install}}%
  \file{pdfrender.drv}{\from{pdfrender.dtx}{driver}}%
  \usedir{tex/generic/oberdiek}%
  \file{pdfrender.sty}{\from{pdfrender.dtx}{package}}%
%  \usedir{doc/latex/oberdiek/test}%
%  \file{pdfrender-test1.tex}{\from{pdfrender.dtx}{test1}}%
%  \file{pdfrender-test2.tex}{\from{pdfrender.dtx}{test2}}%
%  \file{pdfrender-test3.tex}{\from{pdfrender.dtx}{test3}}%
%  \file{pdfrender-test4.tex}{\from{pdfrender.dtx}{test4}}%
%  \file{pdfrender-test5.tex}{\from{pdfrender.dtx}{test5}}%
}

\catcode32=13\relax% active space
\let =\space%
\Msg{************************************************************************}
\Msg{*}
\Msg{* To finish the installation you have to move the following}
\Msg{* file into a directory searched by TeX:}
\Msg{*}
\Msg{*     pdfrender.sty}
\Msg{*}
\Msg{* To produce the documentation run the file `pdfrender.drv'}
\Msg{* through LaTeX.}
\Msg{*}
\Msg{* Happy TeXing!}
\Msg{*}
\Msg{************************************************************************}

\endbatchfile
%</install>
%<*ignore>
\fi
%</ignore>
%<*driver>
\NeedsTeXFormat{LaTeX2e}
\ProvidesFile{pdfrender.drv}%
  [2019/12/29 v1.6 Access to some PDF graphics parameters (HO)]%
\documentclass{ltxdoc}
\usepackage{holtxdoc}[2011/11/22]
\usepackage{longtable}
\renewcommand*{\arraystretch}{1.2}
\begin{document}
  \DocInput{pdfrender.dtx}%
\end{document}
%</driver>
% \fi
%
%
%
% \GetFileInfo{pdfrender.drv}
%
% \title{The \xpackage{pdfrender} package}
% \date{2019/12/29 v1.6}
% \author{Heiko Oberdiek\thanks
% {Please report any issues at \url{https://github.com/ho-tex/oberdiek/issues}}}
%
% \maketitle
%
% \begin{abstract}
% The PDF format has some graphics parameter like
% line width or text rendering mode. This package
% provides an interface for setting these parameters.
% \end{abstract}
%
% \tableofcontents
%
% \section{Documentation}
%
% This package \xpackage{pdfrender} defines an interface for
% PDF specific parameters that affects the rendering of
% graphics or text. The interface and its implementation
% uses the same technique as package \xpackage{color}
% for color settings. Therefore this package is loaded to
% enable \LaTeX's color interface.
%
% At different places \LaTeX\ uses \cs{normalcolor} to avoid that
% header, footer or floats are print in the current color of the
% main text. \cs{setgroup@color} is used to start a save box
% with the color that is set at box saving time.
% Package \xpackage{pdfrender} extends these macros to
% add its own hooks of its parameters. Therefore \LaTeX3 should
% generalize \LaTeXe's color interface.
%
% \subsection{Usage}
%
% In \LaTeX\ the package is loaded as normal package. Options
% are not defined for this package.
% \begin{quote}
% |\usepackage{pdfrender}|
% \end{quote}
%
% This package can also be used in \plainTeX\ and even ini\TeX:
% \begin{quote}
% |input pdfrender.sty|
% \end{quote}
%
% \subsection{Macros}
%
% \begin{declcs}{pdfrender} \M{key value list}
% \end{declcs}
% The first parameter \meta{key value list} contains a
% list of parameter settings. The key entry is the parameter name.
% The macro works like
% \cs{color} (without optional argument) for color setting.
%
% \begin{declcs}{textpdfrender} \M{key value list} \M{text}
% \end{declcs}
% In the same way as \cs{pdfrender} the first argument
% specifies the parameters that should be set. This parameter
% setting affects \meta{text} only. Basically it works
% the same way as \cs{textcolor} (without optional argument).
%
% \subsection{Parameters}
%
% The following table shows an overview for the supported
% parameters and values:
%
% \begin{longtable}{@{}l|l|l@{}}
% \bfseries Parameter & \bfseries Value & \bfseries Alias\\*
% \hline
% \endhead
% \xoption{TextRenderingMode}
%   & \texttt{0} & \texttt{Fill}\\*
%   & \texttt{1} & \texttt{Stroke}\\*
%   & \texttt{2} & \texttt{FillStroke}\\*
%   & \texttt{3} & \texttt{Invisible}\\*
%   & \texttt{4} & \texttt{FillClip}\\*
%   & \texttt{5} & \texttt{StrokeClip}\\*
%   & \texttt{6} & \texttt{FillStrokeClip}\\*
%   & \texttt{7} & \texttt{Clip}\\
% \hline
% \xoption{LineWidth}
%   & \itshape positive number, & \itshape \TeX\ dimen\\*[-.2ex]
%   & \qquad \itshape unit is bp & \\
% \hline
% \xoption{LineCapStyle}
%   & \texttt{0} & \texttt{Butt}\\*
%   & \texttt{1} & \texttt{Round}\\*
%   & \texttt{2} & \texttt{ProjectingSquare}\\
% \hline
% \xoption{LineJoinStyle}
%   & \texttt{0} & \texttt{Miter}\\*
%   & \texttt{1} & \texttt{Round}\\*
%   & \texttt{2} & \texttt{Bevel}\\
% \hline
% \xoption{MiterLimit}
%   & \multicolumn{2}{l@{}}{\itshape positive number} \\
% \hline
% \xoption{Flatness}
%   & \multicolumn{2}{l@{}}{\itshape number between 0 and 100} \\
% \hline
% \xoption{LineDashPattern}
%   & \multicolumn{2}{l@{}}{^^A
%       \itshape numbers in square brackets,^^A
%     } \\*[-.2ex]
%   & \multicolumn{2}{l@{}}{^^A
%       \qquad \itshape followed by number,^^A
%     } \\*
%   & \multicolumn{2}{l@{}}{^^A
%       \qquad \itshape units are bp^^A
%     } \\
% \hline
% \xoption{RenderingIntent}
%   & \multicolumn{2}{l@{}}{\texttt{AbsoluteColorimetric}} \\*
%   & \multicolumn{2}{l@{}}{\texttt{RelativeColorimetric}} \\*
%   & \multicolumn{2}{l@{}}{\texttt{Saturation}} \\*
%   & \multicolumn{2}{l@{}}{\texttt{Perceptual}} \\
% \hline
% \xoption{FillColor}
%   & & \itshape color specification \\
% \hline
% \xoption{StrokeColor}
%   & & \itshape color specification \\
% \end{longtable}
%
% \subsubsection{Details}
%
% The description and specification of these parameters
% are available in the PDF specification \cite{pdf-spec}.
% Therefore they are not repeated here.
%
% \begin{description}
% \item[Value:]
%   The values in the second column lists or describe
%   the values that are specified by the PDF specification.
% \item[Alias:] Instead of magic numbers the package also
%   defines some aliases that can be given as value.
%   Example: |LineCapStyle=Round| has the same effect as
%   |LineCapStyle=1|.
% \item[Number:] The term \emph{number} means an integer
%   or real number. The real number is given as plain
%   decimal number without exponent. The decimal separator is a period.
%   At least one digit must be present.
% \item[\xoption{LineWidth}:] As alias a \TeX\ dimen specification
%   can be given. This includes explicit specifications with number
%   and unit, e.g. |LineWidth=0.5pt|. Also \LaTeX\ length registers
%   may be used. If \eTeX's \cs{dimexpr} is available, then it is
%   automatically added. However package \xpackage{calc} is not
%   supported.
% \item[\xoption{FillColor}, \xoption{StrokeColor}:]
%   Package \xpackage{color} or \xpackage{xcolor} must be loaded
%   before these options can be used (since version 1.2).
%   \LaTeX's color support sets both colors at the same time
%   to the same value. However parameter \xoption{TextRenderingMode}
%   offers the value \texttt{FillStroke} that makes only sense, if
%   the two color types can be set separately.
%   If one of the options \xoption{FillColor} or \xoption{StrokeColor}
%   is specified, then also the color is set. For compatibility
%   with the \LaTeX\ color packages (\xpackage{color} or \xpackage{xcolor}),
%   always both colors must be set. Thus if one of them is not specified,
%   it is taken from the current color.
%
%   Both options \xoption{FillColor} and \xoption{StrokeColor}
%   expect a \LaTeX\ color specification as value.
%   Also the optional color model argument is supported.
%   Example:
%   \begin{quote}
%     |FillColor=yellow|,\\
%     |StrokeColor=[cmyk]{1,.5,0,0}|\\
%   \end{quote}
% \end{description}
%
% \subsection{Color stack}
%
% If the \pdfTeX\ version provides color stacks, then each
% parameter is assigned a page based color stack. The assignment
% of a stack takes place, when its parameter is set the first time.
% This avoids the use of color stacks that are not needed.
%
% \StopEventually{
% }
%
% \section{Implementation}
%    \begin{macrocode}
%<*package>
%    \end{macrocode}
%    Reload check, especially if the package is not used with \LaTeX.
%    \begin{macrocode}
\begingroup\catcode61\catcode48\catcode32=10\relax%
  \catcode13=5 % ^^M
  \endlinechar=13 %
  \catcode35=6 % #
  \catcode39=12 % '
  \catcode44=12 % ,
  \catcode45=12 % -
  \catcode46=12 % .
  \catcode58=12 % :
  \catcode64=11 % @
  \catcode123=1 % {
  \catcode125=2 % }
  \expandafter\let\expandafter\x\csname ver@pdfrender.sty\endcsname
  \ifx\x\relax % plain-TeX, first loading
  \else
    \def\empty{}%
    \ifx\x\empty % LaTeX, first loading,
      % variable is initialized, but \ProvidesPackage not yet seen
    \else
      \expandafter\ifx\csname PackageInfo\endcsname\relax
        \def\x#1#2{%
          \immediate\write-1{Package #1 Info: #2.}%
        }%
      \else
        \def\x#1#2{\PackageInfo{#1}{#2, stopped}}%
      \fi
      \x{pdfrender}{The package is already loaded}%
      \aftergroup\endinput
    \fi
  \fi
\endgroup%
%    \end{macrocode}
%    Package identification:
%    \begin{macrocode}
\begingroup\catcode61\catcode48\catcode32=10\relax%
  \catcode13=5 % ^^M
  \endlinechar=13 %
  \catcode35=6 % #
  \catcode39=12 % '
  \catcode40=12 % (
  \catcode41=12 % )
  \catcode44=12 % ,
  \catcode45=12 % -
  \catcode46=12 % .
  \catcode47=12 % /
  \catcode58=12 % :
  \catcode64=11 % @
  \catcode91=12 % [
  \catcode93=12 % ]
  \catcode123=1 % {
  \catcode125=2 % }
  \expandafter\ifx\csname ProvidesPackage\endcsname\relax
    \def\x#1#2#3[#4]{\endgroup
      \immediate\write-1{Package: #3 #4}%
      \xdef#1{#4}%
    }%
  \else
    \def\x#1#2[#3]{\endgroup
      #2[{#3}]%
      \ifx#1\@undefined
        \xdef#1{#3}%
      \fi
      \ifx#1\relax
        \xdef#1{#3}%
      \fi
    }%
  \fi
\expandafter\x\csname ver@pdfrender.sty\endcsname
\ProvidesPackage{pdfrender}%
  [2019/12/29 v1.6 Access to some PDF graphics parameters (HO)]%
%    \end{macrocode}
%
%    \begin{macrocode}
\begingroup\catcode61\catcode48\catcode32=10\relax%
  \catcode13=5 % ^^M
  \endlinechar=13 %
  \catcode123=1 % {
  \catcode125=2 % }
  \catcode64=11 % @
  \def\x{\endgroup
    \expandafter\edef\csname PdfRender@AtEnd\endcsname{%
      \endlinechar=\the\endlinechar\relax
      \catcode13=\the\catcode13\relax
      \catcode32=\the\catcode32\relax
      \catcode35=\the\catcode35\relax
      \catcode61=\the\catcode61\relax
      \catcode64=\the\catcode64\relax
      \catcode123=\the\catcode123\relax
      \catcode125=\the\catcode125\relax
    }%
  }%
\x\catcode61\catcode48\catcode32=10\relax%
\catcode13=5 % ^^M
\endlinechar=13 %
\catcode35=6 % #
\catcode64=11 % @
\catcode123=1 % {
\catcode125=2 % }
\def\TMP@EnsureCode#1#2{%
  \edef\PdfRender@AtEnd{%
    \PdfRender@AtEnd
    \catcode#1=\the\catcode#1\relax
  }%
  \catcode#1=#2\relax
}
\TMP@EnsureCode{10}{12}% ^^J
\TMP@EnsureCode{36}{3}% $
\TMP@EnsureCode{39}{12}% '
\TMP@EnsureCode{40}{12}% (
\TMP@EnsureCode{41}{12}% )
\TMP@EnsureCode{42}{12}% *
\TMP@EnsureCode{43}{12}% +
\TMP@EnsureCode{44}{12}% ,
\TMP@EnsureCode{45}{12}% -
\TMP@EnsureCode{46}{12}% .
\TMP@EnsureCode{47}{12}% /
\TMP@EnsureCode{58}{12}% :
\TMP@EnsureCode{59}{12}% ;
\TMP@EnsureCode{60}{12}% <
\TMP@EnsureCode{62}{12}% >
\TMP@EnsureCode{63}{12}% ?
\TMP@EnsureCode{91}{12}% [
\TMP@EnsureCode{93}{12}% ]
\TMP@EnsureCode{94}{7}% ^ (superscript)
\TMP@EnsureCode{96}{12}% `
\TMP@EnsureCode{124}{12}% |
%    \end{macrocode}
%
%    \begin{macrocode}
\def\PdfRender@AtEndHook{}
\expandafter\def\expandafter\PdfRender@AtEnd\expandafter{%
  \expandafter\PdfRender@AtEndHook
  \PdfRender@AtEnd
  \endinput
}
%    \end{macrocode}
%
% \subsection{Look for \pdfTeX, its mode and features}
%
%    \begin{macro}{\PdfRender@newif}
%    \begin{macrocode}
\def\PdfRender@newif#1{%
  \expandafter\edef\csname PdfRender@#1true\endcsname{%
    \let
    \expandafter\noexpand\csname ifPdfRender@#1\endcsname
    \noexpand\iftrue
  }%
  \expandafter\edef\csname PdfRender@#1false\endcsname{%
    \let
    \expandafter\noexpand\csname ifPdfRender@#1\endcsname
    \noexpand\iffalse
  }%
  \csname PdfRender@#1false\endcsname
}
%    \end{macrocode}
%    \end{macro}
%
%    \begin{macro}{\ifPdfRender@Stack}
%    \begin{macrocode}
\PdfRender@newif{Stack}
%    \end{macrocode}
%    \end{macro}
%    \begin{macro}{\ifPdfRender@Match}
%    \begin{macrocode}
\PdfRender@newif{Match}
%    \end{macrocode}
%    \end{macro}
%
%    \begin{macro}{\PdfRender@RequirePackage}
%    \begin{macrocode}
\begingroup\expandafter\expandafter\expandafter\endgroup
\expandafter\ifx\csname RequirePackage\endcsname\relax
  \def\PdfRender@RequirePackage#1[#2]{%
    \expandafter\def\expandafter\PdfRender@AtEndHook\expandafter{%
      \PdfRender@AtEndHook
      \ltx@ifpackagelater{#1}{#2}{}{%
        \@PackageWarningNoLine{pdfrender}{%
          You have requested version\MessageBreak
          `#2' of package `#1',\MessageBreak
          but only version\MessageBreak
          `\csname ver@#1.\ltx@pkgextension\endcsname'\MessageBreak
          is available%
        }%
      }%
    }%
    \input #1.sty\relax
  }%
\else
  \let\PdfRender@RequirePackage\RequirePackage
\fi
%    \end{macrocode}
%    \end{macro}
% Luatex compatibility
%    \begin{macrocode}
\ifx\pdfextension\@undefined\else
  \def\pdfcolorstackinit {\pdffeedback colorstackinit}
  \protected\def\pdfcolorstack     {\pdfextension colorstack}
  \protected\def\pdfliteral        {\pdfextension literal}
\fi
%    \end{macrocode}
%
%    \begin{macrocode}
\PdfRender@RequirePackage{iftex}[2019/11/07]
\PdfRender@RequirePackage{infwarerr}[2007/09/09]
\PdfRender@RequirePackage{ltxcmds}[2010/01/28]
%    \end{macrocode}
%
%    \begin{macrocode}
\ifpdf
  \ltx@IfUndefined{pdfcolorstackinit}{%
    \@PackageWarning{pdfrender}{%
      Missing \string\pdfcolorstackinit
    }%
  }{%
    \PdfRender@Stacktrue
  }%
  \ltx@IfUndefined{pdfmatch}{%
    \@PackageInfoNoLine{pdfrender}{%
      \string\pdfmatch\ltx@space not found. %
      Therefore the values\MessageBreak
      of some parameters are not validated%
    }%
  }{%
    \PdfRender@Matchtrue
  }%
\else
  \@PackageWarning{pdfrender}{%
    Missing pdfTeX in PDF mode%
  }%
%    \end{macrocode}
%    \begin{macrocode}
  \ltx@IfUndefined{newcommand}{%
%    \end{macrocode}
%    \begin{macro}{\pdfrender}
%    \begin{macrocode}
    \def\pdfrender#1{}%
%    \end{macrocode}
%    \end{macro}
%    \begin{macro}{\textpdfrender}
%    \begin{macrocode}
    \long\def\textpdfrender#1#2{#2}%
%    \end{macrocode}
%    \end{macro}
%    \begin{macrocode}
  }{%
%    \end{macrocode}
%    \begin{macro}{\pdfrender}
%    \begin{macrocode}
    \newcommand*{\pdfrender}[1]{}%
%    \end{macrocode}
%    \end{macro}
%    \begin{macro}{\textpdfrender}
%    \begin{macrocode}
    \newcommand{\textpdfrender}[2]{#2}%
%    \end{macrocode}
%    \end{macro}
%    \begin{macrocode}
  }%
  \expandafter\PdfRender@AtEnd
\fi%
%    \end{macrocode}
%
% \subsection{Enable color support of \LaTeX}
%
%    \begin{macrocode}
\ltx@ifpackageloaded{color}{}{%
  \def\color@setgroup{\begingroup\set@color}%
  \let\color@begingroup\begingroup
  \def\color@endgroup{\endgraf\endgroup}%
  \def\color@hbox{\hbox\bgroup\color@begingroup}%
  \def\color@vbox{\vbox\bgroup\color@begingroup}%
  \def\color@endbox{\color@endgroup\egroup}%
  \ltx@ifundefined{bgroup}{%
    \let\bgroup={\let\egroup=}%
  }{}%
  \ltx@ifundefined{endgraf}{%
    \let\endgraf=\par
  }{}%
}
%    \end{macrocode}
%
% \subsection{Hook into \cs{normalcolor}}
%
%    The problem is that packages \xpackage{color} and \xpackage{xcolor}
%    each overwrite \cs{normalcolor}. For example, after the package
%    loading order \xpackage{color}, \xpackage{pdfrender} and \xpackage{xcolor}
%    the patched version of \cs{normalcolor} is overwritten by
%    package \xpackage{xcolor}. Also using \cs{AtBeginDocument} for
%    patching is not enough. If package \xpackage{hyperref} is loaded
%    later, it might load package \xpackage{color} using \cs{AtBeginDocument}.
%    \begin{macro}{\PdfRender@NormalColorHook}
%    \begin{macrocode}
\def\PdfRender@NormalColorHook{}
%    \end{macrocode}
%    \end{macro}
%    \begin{macro}{\PdfRender@ColorSetGroupHook}
%    \begin{macrocode}
\def\PdfRender@ColorSetGroupHook{}
%    \end{macrocode}
%    \end{macro}
%    \begin{macro}{\PdfRender@TestBox}
%    \begin{macrocode}
\def\PdfRender@TestBox#1{%
  \setbox0=\color@hbox#1\color@endbox
}
%    \end{macrocode}
%    \end{macro}
%    \begin{macro}{\PdfRender@PatchNormalColor}
%    \begin{macrocode}
\def\PdfRender@PatchNormalColor{%
  \ltx@ifundefined{normalcolor}{%
    \gdef\normalcolor{\PdfRender@NormalColorHook}%
  }{%
    \begingroup
      \def\PdfRender@NormalColorHook{\let\PdfRender@temp=Y}%
      \PdfRender@TestBox{%
        \let\set@color\relax
        \normalcolor
        \ifx\PdfRender@temp Y%
        \else
          \ltx@GlobalAppendToMacro\normalcolor{%
            \PdfRender@NormalColorHook
          }%
        \fi
      }%
    \endgroup
  }%
  \ifx\@nodocument\relax
    \global\let\PdfRender@PatchNormalColor\relax
  \fi
}%
%    \end{macrocode}
%    \end{macro}
%    \begin{macro}{\PdfRender@PatchColorSetGroup}
%    \begin{macrocode}
\def\PdfRender@PatchColorSetGroup{%
  \begingroup
    \def\PdfRender@ColorSetGroupHook{\let\PdfRender@temp=Y}%
    \PdfRender@TestBox{%
      \let\set@color\relax
      \color@setgroup\color@endgroup
      \ifx\PdfRender@temp Y%
      \else
        \ltx@GlobalAppendToMacro\color@setgroup{%
          \PdfRender@ColorSetGroupHook
        }%
      \fi
    }%
  \endgroup
  \ifx\@nodocument\relax
    \global\let\PdfRender@PatchColorSetGroup\relax
  \fi
}%
%    \end{macrocode}
%    \end{macro}
%    \begin{macro}{\PdfRender@PatchColor}
%    \begin{macrocode}
\def\PdfRender@PatchColor{%
  \PdfRender@PatchNormalColor
  \PdfRender@PatchColorSetGroup
}
%    \end{macrocode}
%    \end{macro}
%    \begin{macrocode}
\PdfRender@PatchColor
\ltx@IfUndefined{AtBeginDocument}{}{%
  \AtBeginDocument{\PdfRender@PatchColor}%
}
%    \end{macrocode}
%    \cs{AfterPackage} is provided by package \xpackage{scrlfile}.
%    \begin{macrocode}
\ltx@IfUndefined{AfterPackage}{%
}{%
  \AfterPackage{color}{\PdfRender@PatchColor}%
  \AfterPackage{xcolor}{\PdfRender@PatchColor}%
  \AfterPackage{etoolbox}{%
    \AfterEndPreamble{\PdfRender@PatchColor}%
  }%
}%
%    \end{macrocode}
%    \cs{AfterEndPreamble} is provided by package \xpackage{etoolbox}.
%    \begin{macrocode}
\ltx@IfUndefined{AfterEndPreamble}{%
}{%
  \AfterEndPreamble{\PdfRender@PatchColor}%
}%
%    \end{macrocode}
%
%    \begin{macrocode}
\PdfRender@RequirePackage{kvsetkeys}[2010/01/28]
%    \end{macrocode}
%    \begin{macro}{\PdfRender@texorpdfstring}
%    \begin{macrocode}
\def\PdfRender@texorpdfstring{%
  \ltx@IfUndefined{texorpdfstring}\ltx@firstoftwo\texorpdfstring
}
%    \end{macrocode}
%    \end{macro}
%    \begin{macro}{\pdfrender}
%    \begin{macrocode}
\ltx@IfUndefined{DeclareRobustCommand}%
\ltx@firstoftwo\ltx@secondoftwo
{%
  \def\pdfrender#1%
}{%
  \newcommand{\pdfrender}{}%
  \DeclareRobustCommand*{\pdfrender}[1]%
}%
{%
  \PdfRender@texorpdfstring{%
    \PdfRender@PatchNormalColor
    \global\let\PdfRender@FillColor\ltx@empty
    \global\let\PdfRender@StrokeColor\ltx@empty
    \kvsetkeys{PDFRENDER}{#1}%
    \PdfRender@SetColor
  }{}%
}
%    \end{macrocode}
%    \end{macro}
%    \begin{macro}{\textpdfrender}
%    \begin{macrocode}
\ltx@IfUndefined{DeclareRobustCommand}%
\ltx@firstoftwo\ltx@secondoftwo
{%
  \long\def\textpdfrender#1#2%
}{%
  \newcommand{\textpdfrender}{}%
  \DeclareRobustCommand{\textpdfrender}[2]%
}%
{%
  \PdfRender@texorpdfstring{%
    \begingroup
      \pdfrender{#1}%
      #2%
    \endgroup
  }{#2}%
}
%    \end{macrocode}
%    \end{macro}
%
%    \begin{macro}{\ifPdfRender@Values}
%    \begin{macrocode}
\PdfRender@newif{Values}
%    \end{macrocode}
%    \end{macro}
%    \begin{macro}{\PdfRender@NewClassValues}
%    \begin{macrocode}
\def\PdfRender@NewClassValues#1#2#3#4{%
  \PdfRender@Valuestrue
  \PdfRender@NewClass{#1}{#2}{#3}{#4}{}%
}
%    \end{macrocode}
%    \end{macro}
%    \begin{macro}{\PdfRender@NewClass}
%    \begin{macrocode}
\def\PdfRender@NewClass#1#2#3#4#5{%
  \PdfRender@newif{Active#1}%
  \expandafter\def\csname PdfRender@Default#1\endcsname{#2}%
  \expandafter\let\csname PdfRender@Current#1\expandafter\endcsname
      \csname PdfRender@Default#1\endcsname
  \ifPdfRender@Stack
    \expandafter\edef\csname PdfRender@Init#1\endcsname{%
      \global\chardef
      \expandafter\noexpand\csname PdfRender@Stack#1\endcsname=%
          \noexpand\pdfcolorstackinit page direct{%
            \noexpand#3%
            \expandafter\noexpand\csname PdfRender@Default#1\endcsname
          }\relax
      \noexpand\@PackageInfo{pdfrender}{%
        New color stack `#1' = \noexpand\number
        \expandafter\noexpand\csname PdfRender@Stack#1\endcsname
      }%
      \gdef\expandafter\noexpand\csname PdfRender@Init#1\endcsname{}%
    }%
    \expandafter\edef\csname PdfRender@Set#1\endcsname{%
      \expandafter\noexpand\csname PdfRender@Init#1\endcsname
      \noexpand\pdfcolorstack
      \expandafter\noexpand\csname PdfRender@Stack#1\endcsname
      push{%
        #3{\expandafter\noexpand\csname PdfRender@Current#1\endcsname}%
      }%
      \noexpand\aftergroup
      \expandafter\noexpand\csname PdfRender@Reset#1\endcsname
    }%
    \expandafter\edef\csname PdfRender@Reset#1\endcsname{%
      \expandafter\noexpand\csname PdfRender@Init#1\endcsname
      \noexpand\pdfcolorstack
      \expandafter\noexpand\csname PdfRender@Stack#1\endcsname
      pop\relax
    }%
  \else
    \expandafter\edef\csname PdfRender@Set#1\endcsname{%
      \noexpand\pdfliteral direct{%
        #3{\expandafter\noexpand\csname PdfRender@Current#1\endcsname}%
      }%
      \noexpand\aftergroup
      \expandafter\noexpand\csname PdfRender@Reset#1\endcsname
    }%
    \expandafter\edef\csname PdfRender@Reset#1\endcsname{%
      \noexpand\pdfliteral direct{%
        #3{\expandafter\noexpand\csname PdfRender@Current#1\endcsname}%
      }%
    }%
  \fi
  \expandafter\edef\csname PdfRender@Normal#1\endcsname{%
    \let
    \expandafter\noexpand\csname PdfRender@Current#1\endcsname
    \expandafter\noexpand\csname PdfRender@Default#1\endcsname
    \noexpand\PdfRender@Set{#1}%
  }%
  \expandafter\ltx@GlobalAppendToMacro\expandafter\PdfRender@NormalColorHook
  \expandafter{%
    \csname PdfRender@Normal#1\endcsname
  }%
  \ltx@GlobalAppendToMacro\PdfRender@ColorSetGroupHook{%
    \PdfRender@Set{#1}%
  }%
  \ifPdfRender@Values
    \kv@parse@normalized{#4}{%
      \expandafter\let\csname PdfRender@#1@\kv@key\endcsname\kv@key
      \ifx\kv@value\relax
      \else
        \expandafter\let\csname PdfRender@#1@\kv@value\endcsname\kv@key
      \fi
      \ltx@gobbletwo
    }%
    \PdfRender@define@key{PDFRENDER}{#1}{%
      \global\csname PdfRender@Active#1true\endcsname
      \def\PdfRender@Current{##1}%
      \PdfRender@SetValidateValues{#1}%
    }%
    \PdfRender@Valuesfalse
  \else
    \PdfRender@define@key{PDFRENDER}{#1}{%
      \global\csname PdfRender@Active#1true\endcsname
      \expandafter\def\csname PdfRender@Current#1\endcsname{##1}%
      \ltx@IfUndefined{PdfRender@PostProcess#1}{%
      }{%
        \csname PdfRender@PostProcess#1\endcsname
      }%
      \PdfRender@SetValidate{#1}{#4}{#5}%
    }%
  \fi
}%
%    \end{macrocode}
%    \end{macro}
%    \begin{macro}{\PdfRender@define@key}
%    \begin{macrocode}
\ltx@IfUndefined{define@key}{%
  \def\PdfRender@define@key#1#2{%
    \expandafter\def\csname KV@#1@#2\endcsname##1%
  }%
}{%
  \let\PdfRender@define@key\define@key
}
%    \end{macrocode}
%    \end{macro}
%    \begin{macro}{\PdfRender@Set}
%    \begin{macrocode}
\def\PdfRender@Set#1{%
  \csname ifPdfRender@Active#1\endcsname
    \csname PdfRender@Set#1\expandafter\endcsname
  \fi
}
%    \end{macrocode}
%    \end{macro}
%    \begin{macro}{\PdfRender@Reset}
%    \begin{macrocode}
\def\PdfRender@Reset#1{%
  \csname ifPdfRender@Active#1\endcsname
    \csname PdfRender@Reset#1\expandafter\endcsname
  \fi
}
%    \end{macrocode}
%    \end{macro}
%    \begin{macro}{\PdfRender@ErrorInvalidValue}
%    \begin{macrocode}
\def\PdfRender@ErrorInvalidValue#1{%
  \PackageError{pdfrender}{%
    Ignoring parameter setting for `#1'\MessageBreak
    because of invalid value %
    `\csname PdfRender@Current#1\endcsname'%
  }\@ehc
  \expandafter\let\csname PdfRender@Current#1\endcsname\ltx@empty
}%
%    \end{macrocode}
%    \end{macro}
%    \begin{macro}{\PdfRender@SetValidate}
%    \begin{macrocode}
\ifPdfRender@Match
  \def\PdfRender@SetValidate#1#2#3{%
    \ifnum\pdfmatch{^(#2)$}{\csname PdfRender@Current#1\endcsname}=1 %
      \csname PdfRender@Set#1\expandafter\endcsname
    \else
      \PdfRender@ErrorInvalidValue{#1}%
    \fi
  }%
\else
  \def\PdfRender@SetValidate#1#2#3{%
    \expandafter\let\expandafter\PdfRender@Current
    \csname PdfRender@Current#1\endcsname
    #3%
    \ifx\PdfRender@Current\@empty
      \PdfRender@ErrorInvalidValue{#1}%
    \else
      \csname PdfRender@Set#1\expandafter\endcsname
    \fi
  }%
\fi
%    \end{macrocode}
%    \end{macro}
%    \begin{macro}{\PdfRender@SetValidateValues}
%    \begin{macrocode}
\def\PdfRender@SetValidateValues#1{%
  \ltx@IfUndefined{PdfRender@#1@\PdfRender@Current}{%
    \expandafter\let\csname PdfRender@Current#1\endcsname
                    \PdfRender@Current
    \PdfRender@ErrorInvalidValue{#1}%
  }{%
    \expandafter\edef\csname PdfRender@Current#1\endcsname{%
      \csname PdfRender@#1@\PdfRender@Current\endcsname
    }%
    \csname PdfRender@Set#1\endcsname
  }%
}
%    \end{macrocode}
%    \end{macro}
%    \begin{macro}{\PdfRender@OpValue}
%    \begin{macrocode}
\def\PdfRender@OpValue#1#2{#2\ltx@space#1}%
%    \end{macrocode}
%    \end{macro}
%    \begin{macro}{\PdfRender@OpName}
%    \begin{macrocode}
\def\PdfRender@OpName#1#2{/#2\ltx@space#1}%
%    \end{macrocode}
%    \end{macro}
%
% \subsection{Declare and setup parameters}
%
%    \begin{macrocode}
\PdfRender@NewClassValues{TextRenderingMode}%
                      {0}%
                      {\PdfRender@OpValue{Tr}}{%
  0=Fill,%
  1=Stroke,%
  2=FillStroke,%
  3=Invisible,%
  4=FillClip,%
  5=StrokeClip,%
  6=FillStrokeClip,%
  7=Clip,%
}%
\PdfRender@NewClass{LineWidth}{1}{\PdfRender@OpValue{w}}{%
  [0-9]+\string\.?[0-9]*|\string\.[0-9]+%
}{}%
\ltx@IfUndefined{dimexpr}{%
  \def\PdfRender@dimexpr{}%
}{%
  \let\PdfRender@dimexpr\dimexpr
}
\def\PdfRender@PostProcessLineWidth{%
  \begingroup
  \afterassignment\PdfRender@@PostProcessLineWidth
  \dimen0=\PdfRender@dimexpr\PdfRender@CurrentLineWidth bp %
  \PdfRender@let\PdfRender@relax\PdfRender@relax
}
\let\PdfRender@let\let
\let\PdfRender@relax\relax
\def\PdfRender@@PostProcessLineWidth#1\PdfRender@let{%
  \ifx\\#1\\%
    \endgroup
  \else
    \dimen0=.996264\dimen0 % 72/72.27
    \edef\x{\endgroup
      \def\noexpand\PdfRender@CurrentLineWidth{%
        \strip@pt\dimen0%
      }%
    }%
    \expandafter\x
  \fi
}
\PdfRender@NewClassValues{LineCapStyle}{0}{\PdfRender@OpValue{J}}{%
  0=Butt,%
  1=Round,%
  2=ProjectingSquare,%
}%
\PdfRender@NewClassValues{LineJoinStyle}{0}{\PdfRender@OpValue{j}}{%
  0=Miter,%
  1=Round,%
  2=Bevel,%
}%
\PdfRender@NewClass{MiterLimit}{10}{\PdfRender@OpValue{M}}{%
  [0-9]*[1-9][0-9]*\string\.?[0-9]*|%
  [0-9]*\string\.?[0-9]*[1-9][0-9]*%
}{}%
\PdfRender@NewClass{Flatness}{0}{\PdfRender@OpValue{i}}{%
  100(\string\.0*)?|[0-9][0-9](\string\.[0-9]*)?|\string\.[0-9]+%
}{}%
\PdfRender@NewClass{LineDashPattern}{[]0}{\PdfRender@OpValue{d}}{%
  \string\[%
  ( ?([0-9]+\string\.?[0-9]*|\string\.[0-9]+) ?)*%
  \string\] ?%
  ([0-9]+\string\.?[0-9]*|\string\.[0-9]+)%
}{}%
\PdfRender@NewClassValues{RenderingIntent}%
                      {RelativeColorimetric}%
                      {\PdfRender@OpName{ri}}{%
  AbsoluteColorimetric,%
  RelativeColorimetric,%
  Saturation,%
  Perceptual,%
}%
%    \end{macrocode}
%
% \subsection{Fill and stroke color support}
%
%    \begin{macrocode}
\PdfRender@define@key{PDFRENDER}{FillColor}{%
  \begingroup
    \def\PdfRender@Color{#1}%
    \ifx\PdfRender@Color\ltx@empty
      \global\let\PdfRender@FillColor\ltx@empty
    \else
      \PdfRender@ColorAvailable{%
        \PdfRender@TestBox{%
          \expandafter\PdfRender@TryColor\PdfRender@Color\ltx@empty
          \PdfRender@GetFillColor
          \ifx\PdfRender@FillColor\ltx@empty
            \@PackageWarning{pdfrender}{%
              Cannot extract fill color\MessageBreak
              from value `#1'%
            }%
          \fi
        }%
      }%
    \fi
  \endgroup
}
\PdfRender@define@key{PDFRENDER}{StrokeColor}{%
  \begingroup
    \def\PdfRender@Color{#1}%
    \ifx\PdfRender@Color\ltx@empty
      \global\let\PdfRender@StrokeColor\ltx@empty
    \else
      \PdfRender@ColorAvailable{%
        \PdfRender@TestBox{%
          \expandafter\PdfRender@TryColor\PdfRender@Color\ltx@empty
          \PdfRender@GetStrokeColor
          \ifx\PdfRender@StrokeColor\ltx@empty
            \@PackageWarning{pdfrender}{%
              Cannot extract stroke color\MessageBreak
              from value `#1'%
            }%
          \fi
        }%
      }%
    \fi
  \endgroup
}
%    \end{macrocode}
%    \begin{macro}{\PdfRender@ColorAvailable}
%    \begin{macrocode}
\def\PdfRender@ColorAvailable{%
  \@ifundefined{set@color}{%
    \@PackageError{pdfrender}{%
      Ignoring color options, because neither\MessageBreak
      package `color' nor package `xcolor' is loaded%
    }\@ehc
    \global\let\PdfRender@ColorAvailable\ltx@gobble
  }{%
    \global\let\PdfRender@ColorAvailable\ltx@firstofone
  }%
  \PdfRender@ColorAvailable
}
%    \end{macrocode}
%    \end{macro}
%    \begin{macro}{\PdfRender@TryColor}
%    \begin{macrocode}
\def\PdfRender@TryColor{%
  \@ifnextchar[\color\PdfRender@@TryColor
}
%    \end{macrocode}
%    \end{macro}
%    \begin{macro}{\PdfRender@@TryColor}
%    \begin{macrocode}
\def\PdfRender@@TryColor#1\ltx@empty{%
  \expandafter\color\expandafter{\PdfRender@Color}%
}
%    \end{macrocode}
%    \end{macro}
%    \begin{macro}{\PdfRender@SetColor}
%    \begin{macrocode}
\def\PdfRender@SetColor{%
  \chardef\PdfRender@NeedsCurrentColor=0 %
  \ifx\PdfRender@FillColor\ltx@empty
    \ifx\PdfRender@StrokeColor\ltx@empty
    \else
      \edef\PdfRender@CurrentColor{%
        \noexpand\PdfRender@FillColor\ltx@space\PdfRender@StrokeColor
      }%
      \chardef\PdfRender@NeedsCurrentColor=1 %
    \fi
  \else
    \ifx\PdfRender@StrokeColor\ltx@empty
      \edef\PdfRender@CurrentColor{%
        \PdfRender@FillColor\ltx@space\noexpand\PdfRender@StrokeColor
      }%
      \chardef\PdfRender@NeedsCurrentColor=2 %
    \else
      \edef\current@color{%
        \PdfRender@FillColor\ltx@space\PdfRender@StrokeColor
      }%
      \set@color
    \fi
  \fi
  \ifnum\PdfRender@NeedsCurrentColor=1 %
    \PdfRender@GetFillColor
    \ifx\PdfRender@FillColor\ltx@empty
      \@PackageWarning{pdfrender}{%
        Cannot extract current fill color%
      }%
    \else
      \edef\current@color{\PdfRender@CurrentColor}%
      \set@color
    \fi
  \else
    \ifnum\PdfRender@NeedsCurrentColor=2 %
      \PdfRender@GetStrokeColor
      \ifx\PdfRender@StrokeColor\ltx@empty
        \@PackageWarning{pdfrender}{%
          Cannot extract current stroke color%
        }%
      \else
        \edef\current@color{\PdfRender@CurrentColor}%
        \set@color
      \fi
    \fi
  \fi
}
%    \end{macrocode}
%    \end{macro}
%    \begin{macro}{\PdfRender@PatternFillColor}
%    \begin{macrocode}
\edef\PdfRender@PatternFillColor{ % space
  (%
    [0-9\string\.]+ g|%
    [0-9\string\.]+ [0-9\string\.]+ [0-9\string\.]+ rg|%
    [0-9\string\.]+ [0-9\string\.]+ %
    [0-9\string\.]+ [0-9\string\.]+ k%
  ) % space
  (.*)$%
}
%    \end{macrocode}
%    \end{macro}
%    \begin{macro}{\PdfRender@PatternStrokeColor}
%    \begin{macrocode}
\edef\PdfRender@PatternStrokeColor{ % space
  (%
    [0-9\string\.]+ G|%
    [0-9\string\.]+ [0-9\string\.]+ [0-9\string\.]+ RG|%
    [0-9\string\.]+ [0-9\string\.]+ %
    [0-9\string\.]+ [0-9\string\.]+ K%
  ) % space
  (.*)$%
}
%    \end{macrocode}
%    \end{macro}
%    \begin{macro}{\PdfRender@MatchPattern}
%    \begin{macrocode}
\def\PdfRender@MatchPattern#1{%
  \ifnum\pdfmatch{\PdfRender@Pattern}{\PdfRender@String}=1 %
    \xdef#1{%
      \expandafter\strip@prefix\pdflastmatch 1%
    }%
    \edef\PdfRender@String{%
      \expandafter\strip@prefix\pdflastmatch 2%
    }%
    \ifx\PdfRender@String\ltx@empty
    \else
      \expandafter\expandafter\expandafter\PdfRender@MatchPattern
      \expandafter\expandafter\expandafter#1%
    \fi
  \fi
}
%    \end{macrocode}
%    \end{macro}
%    \begin{macro}{\PdfRender@GetFillColor}
%    \begin{macrocode}
\def\PdfRender@GetFillColor{%
  \global\let\PdfRender@FillColor\ltx@empty
  \begingroup
    \ifPdfRender@Match
      \let\PdfRender@Pattern\PdfRender@PatternFillColor
      \edef\PdfRender@String{\ltx@space\current@color\ltx@space}%
      \PdfRender@MatchPattern\PdfRender@FillColor
    \else
      \edef\current@color{\current@color\ltx@space}%
      \let\PdfRender@OP\relax
      \PdfRender@FindOp{g}0%
      \PdfRender@FindOp{G}1%
      \PdfRender@FindOp{rg}0%
      \PdfRender@FindOp{RG}1%
      \PdfRender@FindOp{k}0%
      \PdfRender@FindOp{K}1%
      \PdfRender@FilterOp 0\PdfRender@FillColor
    \fi
  \endgroup
}
%    \end{macrocode}
%    \end{macro}
%    \begin{macro}{\PdfRender@GetStrokeColor}
%    \begin{macrocode}
\def\PdfRender@GetStrokeColor{%
  \global\let\PdfRender@StrokeColor\ltx@empty
  \begingroup
    \ifPdfRender@Match
      \let\PdfRender@Pattern\PdfRender@PatternStrokeColor
      \edef\PdfRender@String{\ltx@space\current@color\ltx@space}%
      \PdfRender@MatchPattern\PdfRender@StrokeColor
    \else
      \edef\current@color{\current@color\ltx@space}%
      \let\PdfRender@OP\relax
      \PdfRender@FindOp{g}0%
      \PdfRender@FindOp{G}1%
      \PdfRender@FindOp{rg}0%
      \PdfRender@FindOp{RG}1%
      \PdfRender@FindOp{k}0%
      \PdfRender@FindOp{K}1%
      \PdfRender@FilterOp 1\PdfRender@StrokeColor
    \fi
  \endgroup
}
%    \end{macrocode}
%    \end{macro}
%    \begin{macrocode}
\ifPdfRender@Match
  \expandafter\PdfRender@AtEnd
\fi%
%    \end{macrocode}
%    \begin{macro}{\PdfRender@FindOp}
%    \begin{macrocode}
\def\PdfRender@FindOp#1#2{%
  \def\PdfRender@temp##1 #1 ##2\@nil{%
    ##1%
    \ifx\\##2\\%
      \expandafter\@gobble
    \else
      \PdfRender@OP{#1}#2%
      \expandafter\@firstofone
    \fi
    {%
      \PdfRender@temp##2\@nil
    }%
  }%
  \edef\current@color{%
    \@firstofone{\expandafter\PdfRender@temp\current@color} #1 \@nil
  }%
}
%    \end{macrocode}
%    \end{macro}
%    \begin{macro}{\PdfRender@FilterOp}
%    \begin{macrocode}
\def\PdfRender@FilterOp#1#2{%
  \expandafter\PdfRender@@FilterOp\expandafter#1\expandafter#2%
      \current@color\PdfRender@OP{}{}%
}
%    \end{macrocode}
%    \end{macro}
%    \begin{macro}{\PdfRender@@FilterOp}
%    \begin{macrocode}
\def\PdfRender@@FilterOp#1#2#3\PdfRender@OP#4#5{%
  \ifx\\#4#5\\%
  \else
    \ifnum#1=#5 %
      \xdef#2{#3 #4}%
    \fi
    \expandafter\PdfRender@@FilterOp\expandafter#1\expandafter#2%
  \fi
}
%    \end{macrocode}
%    \end{macro}
%
%    \begin{macrocode}
\PdfRender@AtEnd%
%</package>
%    \end{macrocode}
%% \section{Installation}
%
% \subsection{Download}
%
% \paragraph{Package.} This package is available on
% CTAN\footnote{\CTANpkg{pdfrender}}:
% \begin{description}
% \item[\CTAN{macros/latex/contrib/oberdiek/pdfrender.dtx}] The source file.
% \item[\CTAN{macros/latex/contrib/oberdiek/pdfrender.pdf}] Documentation.
% \end{description}
%
%
% \paragraph{Bundle.} All the packages of the bundle `oberdiek'
% are also available in a TDS compliant ZIP archive. There
% the packages are already unpacked and the documentation files
% are generated. The files and directories obey the TDS standard.
% \begin{description}
% \item[\CTANinstall{install/macros/latex/contrib/oberdiek.tds.zip}]
% \end{description}
% \emph{TDS} refers to the standard ``A Directory Structure
% for \TeX\ Files'' (\CTANpkg{tds}). Directories
% with \xfile{texmf} in their name are usually organized this way.
%
% \subsection{Bundle installation}
%
% \paragraph{Unpacking.} Unpack the \xfile{oberdiek.tds.zip} in the
% TDS tree (also known as \xfile{texmf} tree) of your choice.
% Example (linux):
% \begin{quote}
%   |unzip oberdiek.tds.zip -d ~/texmf|
% \end{quote}
%
% \subsection{Package installation}
%
% \paragraph{Unpacking.} The \xfile{.dtx} file is a self-extracting
% \docstrip\ archive. The files are extracted by running the
% \xfile{.dtx} through \plainTeX:
% \begin{quote}
%   \verb|tex pdfrender.dtx|
% \end{quote}
%
% \paragraph{TDS.} Now the different files must be moved into
% the different directories in your installation TDS tree
% (also known as \xfile{texmf} tree):
% \begin{quote}
% \def\t{^^A
% \begin{tabular}{@{}>{\ttfamily}l@{ $\rightarrow$ }>{\ttfamily}l@{}}
%   pdfrender.sty & tex/generic/oberdiek/pdfrender.sty\\
%   pdfrender.pdf & doc/latex/oberdiek/pdfrender.pdf\\
%   pdfrender.dtx & source/latex/oberdiek/pdfrender.dtx\\
% \end{tabular}^^A
% }^^A
% \sbox0{\t}^^A
% \ifdim\wd0>\linewidth
%   \begingroup
%     \advance\linewidth by\leftmargin
%     \advance\linewidth by\rightmargin
%   \edef\x{\endgroup
%     \def\noexpand\lw{\the\linewidth}^^A
%   }\x
%   \def\lwbox{^^A
%     \leavevmode
%     \hbox to \linewidth{^^A
%       \kern-\leftmargin\relax
%       \hss
%       \usebox0
%       \hss
%       \kern-\rightmargin\relax
%     }^^A
%   }^^A
%   \ifdim\wd0>\lw
%     \sbox0{\small\t}^^A
%     \ifdim\wd0>\linewidth
%       \ifdim\wd0>\lw
%         \sbox0{\footnotesize\t}^^A
%         \ifdim\wd0>\linewidth
%           \ifdim\wd0>\lw
%             \sbox0{\scriptsize\t}^^A
%             \ifdim\wd0>\linewidth
%               \ifdim\wd0>\lw
%                 \sbox0{\tiny\t}^^A
%                 \ifdim\wd0>\linewidth
%                   \lwbox
%                 \else
%                   \usebox0
%                 \fi
%               \else
%                 \lwbox
%               \fi
%             \else
%               \usebox0
%             \fi
%           \else
%             \lwbox
%           \fi
%         \else
%           \usebox0
%         \fi
%       \else
%         \lwbox
%       \fi
%     \else
%       \usebox0
%     \fi
%   \else
%     \lwbox
%   \fi
% \else
%   \usebox0
% \fi
% \end{quote}
% If you have a \xfile{docstrip.cfg} that configures and enables \docstrip's
% TDS installing feature, then some files can already be in the right
% place, see the documentation of \docstrip.
%
% \subsection{Refresh file name databases}
%
% If your \TeX~distribution
% (\TeX\,Live, \mikTeX, \dots) relies on file name databases, you must refresh
% these. For example, \TeX\,Live\ users run \verb|texhash| or
% \verb|mktexlsr|.
%
% \subsection{Some details for the interested}
%
% \paragraph{Unpacking with \LaTeX.}
% The \xfile{.dtx} chooses its action depending on the format:
% \begin{description}
% \item[\plainTeX:] Run \docstrip\ and extract the files.
% \item[\LaTeX:] Generate the documentation.
% \end{description}
% If you insist on using \LaTeX\ for \docstrip\ (really,
% \docstrip\ does not need \LaTeX), then inform the autodetect routine
% about your intention:
% \begin{quote}
%   \verb|latex \let\install=y% \iffalse meta-comment
%
% File: pdfrender.dtx
% Version: 2019/12/29 v1.6
% Info: Access to some PDF graphics parameters
%
% Copyright (C)
%    2010 Heiko Oberdiek
%    2016-2019 Oberdiek Package Support Group
%    https://github.com/ho-tex/oberdiek/issues
%
% This work may be distributed and/or modified under the
% conditions of the LaTeX Project Public License, either
% version 1.3c of this license or (at your option) any later
% version. This version of this license is in
%    https://www.latex-project.org/lppl/lppl-1-3c.txt
% and the latest version of this license is in
%    https://www.latex-project.org/lppl.txt
% and version 1.3 or later is part of all distributions of
% LaTeX version 2005/12/01 or later.
%
% This work has the LPPL maintenance status "maintained".
%
% The Current Maintainers of this work are
% Heiko Oberdiek and the Oberdiek Package Support Group
% https://github.com/ho-tex/oberdiek/issues
%
% The Base Interpreter refers to any `TeX-Format',
% because some files are installed in TDS:tex/generic//.
%
% This work consists of the main source file pdfrender.dtx
% and the derived files
%    pdfrender.sty, pdfrender.pdf, pdfrender.ins, pdfrender.drv,
%    pdfrender-test1.tex, pdfrender-test2.tex,
%    pdfrender-test3.tex, pdfrender-test4.tex,
%    pdfrender-test5.tex.
%
% Distribution:
%    CTAN:macros/latex/contrib/oberdiek/pdfrender.dtx
%    CTAN:macros/latex/contrib/oberdiek/pdfrender.pdf
%
% Unpacking:
%    (a) If pdfrender.ins is present:
%           tex pdfrender.ins
%    (b) Without pdfrender.ins:
%           tex pdfrender.dtx
%    (c) If you insist on using LaTeX
%           latex \let\install=y\input{pdfrender.dtx}
%        (quote the arguments according to the demands of your shell)
%
% Documentation:
%    (a) If pdfrender.drv is present:
%           latex pdfrender.drv
%    (b) Without pdfrender.drv:
%           latex pdfrender.dtx; ...
%    The class ltxdoc loads the configuration file ltxdoc.cfg
%    if available. Here you can specify further options, e.g.
%    use A4 as paper format:
%       \PassOptionsToClass{a4paper}{article}
%
%    Programm calls to get the documentation (example):
%       pdflatex pdfrender.dtx
%       makeindex -s gind.ist pdfrender.idx
%       pdflatex pdfrender.dtx
%       makeindex -s gind.ist pdfrender.idx
%       pdflatex pdfrender.dtx
%
% Installation:
%    TDS:tex/generic/oberdiek/pdfrender.sty
%    TDS:doc/latex/oberdiek/pdfrender.pdf
%    TDS:source/latex/oberdiek/pdfrender.dtx
%
%<*ignore>
\begingroup
  \catcode123=1 %
  \catcode125=2 %
  \def\x{LaTeX2e}%
\expandafter\endgroup
\ifcase 0\ifx\install y1\fi\expandafter
         \ifx\csname processbatchFile\endcsname\relax\else1\fi
         \ifx\fmtname\x\else 1\fi\relax
\else\csname fi\endcsname
%</ignore>
%<*install>
\input docstrip.tex
\Msg{************************************************************************}
\Msg{* Installation}
\Msg{* Package: pdfrender 2019/12/29 v1.6 Access to some PDF graphics parameters (HO)}
\Msg{************************************************************************}

\keepsilent
\askforoverwritefalse

\let\MetaPrefix\relax
\preamble

This is a generated file.

Project: pdfrender
Version: 2019/12/29 v1.6

Copyright (C)
   2010 Heiko Oberdiek
   2016-2019 Oberdiek Package Support Group

This work may be distributed and/or modified under the
conditions of the LaTeX Project Public License, either
version 1.3c of this license or (at your option) any later
version. This version of this license is in
   https://www.latex-project.org/lppl/lppl-1-3c.txt
and the latest version of this license is in
   https://www.latex-project.org/lppl.txt
and version 1.3 or later is part of all distributions of
LaTeX version 2005/12/01 or later.

This work has the LPPL maintenance status "maintained".

The Current Maintainers of this work are
Heiko Oberdiek and the Oberdiek Package Support Group
https://github.com/ho-tex/oberdiek/issues


The Base Interpreter refers to any `TeX-Format',
because some files are installed in TDS:tex/generic//.

This work consists of the main source file pdfrender.dtx
and the derived files
   pdfrender.sty, pdfrender.pdf, pdfrender.ins, pdfrender.drv,
   pdfrender-test1.tex, pdfrender-test2.tex,
   pdfrender-test3.tex, pdfrender-test4.tex,
   pdfrender-test5.tex.

\endpreamble
\let\MetaPrefix\DoubleperCent

\generate{%
  \file{pdfrender.ins}{\from{pdfrender.dtx}{install}}%
  \file{pdfrender.drv}{\from{pdfrender.dtx}{driver}}%
  \usedir{tex/generic/oberdiek}%
  \file{pdfrender.sty}{\from{pdfrender.dtx}{package}}%
%  \usedir{doc/latex/oberdiek/test}%
%  \file{pdfrender-test1.tex}{\from{pdfrender.dtx}{test1}}%
%  \file{pdfrender-test2.tex}{\from{pdfrender.dtx}{test2}}%
%  \file{pdfrender-test3.tex}{\from{pdfrender.dtx}{test3}}%
%  \file{pdfrender-test4.tex}{\from{pdfrender.dtx}{test4}}%
%  \file{pdfrender-test5.tex}{\from{pdfrender.dtx}{test5}}%
}

\catcode32=13\relax% active space
\let =\space%
\Msg{************************************************************************}
\Msg{*}
\Msg{* To finish the installation you have to move the following}
\Msg{* file into a directory searched by TeX:}
\Msg{*}
\Msg{*     pdfrender.sty}
\Msg{*}
\Msg{* To produce the documentation run the file `pdfrender.drv'}
\Msg{* through LaTeX.}
\Msg{*}
\Msg{* Happy TeXing!}
\Msg{*}
\Msg{************************************************************************}

\endbatchfile
%</install>
%<*ignore>
\fi
%</ignore>
%<*driver>
\NeedsTeXFormat{LaTeX2e}
\ProvidesFile{pdfrender.drv}%
  [2019/12/29 v1.6 Access to some PDF graphics parameters (HO)]%
\documentclass{ltxdoc}
\usepackage{holtxdoc}[2011/11/22]
\usepackage{longtable}
\renewcommand*{\arraystretch}{1.2}
\begin{document}
  \DocInput{pdfrender.dtx}%
\end{document}
%</driver>
% \fi
%
%
%
% \GetFileInfo{pdfrender.drv}
%
% \title{The \xpackage{pdfrender} package}
% \date{2019/12/29 v1.6}
% \author{Heiko Oberdiek\thanks
% {Please report any issues at \url{https://github.com/ho-tex/oberdiek/issues}}}
%
% \maketitle
%
% \begin{abstract}
% The PDF format has some graphics parameter like
% line width or text rendering mode. This package
% provides an interface for setting these parameters.
% \end{abstract}
%
% \tableofcontents
%
% \section{Documentation}
%
% This package \xpackage{pdfrender} defines an interface for
% PDF specific parameters that affects the rendering of
% graphics or text. The interface and its implementation
% uses the same technique as package \xpackage{color}
% for color settings. Therefore this package is loaded to
% enable \LaTeX's color interface.
%
% At different places \LaTeX\ uses \cs{normalcolor} to avoid that
% header, footer or floats are print in the current color of the
% main text. \cs{setgroup@color} is used to start a save box
% with the color that is set at box saving time.
% Package \xpackage{pdfrender} extends these macros to
% add its own hooks of its parameters. Therefore \LaTeX3 should
% generalize \LaTeXe's color interface.
%
% \subsection{Usage}
%
% In \LaTeX\ the package is loaded as normal package. Options
% are not defined for this package.
% \begin{quote}
% |\usepackage{pdfrender}|
% \end{quote}
%
% This package can also be used in \plainTeX\ and even ini\TeX:
% \begin{quote}
% |input pdfrender.sty|
% \end{quote}
%
% \subsection{Macros}
%
% \begin{declcs}{pdfrender} \M{key value list}
% \end{declcs}
% The first parameter \meta{key value list} contains a
% list of parameter settings. The key entry is the parameter name.
% The macro works like
% \cs{color} (without optional argument) for color setting.
%
% \begin{declcs}{textpdfrender} \M{key value list} \M{text}
% \end{declcs}
% In the same way as \cs{pdfrender} the first argument
% specifies the parameters that should be set. This parameter
% setting affects \meta{text} only. Basically it works
% the same way as \cs{textcolor} (without optional argument).
%
% \subsection{Parameters}
%
% The following table shows an overview for the supported
% parameters and values:
%
% \begin{longtable}{@{}l|l|l@{}}
% \bfseries Parameter & \bfseries Value & \bfseries Alias\\*
% \hline
% \endhead
% \xoption{TextRenderingMode}
%   & \texttt{0} & \texttt{Fill}\\*
%   & \texttt{1} & \texttt{Stroke}\\*
%   & \texttt{2} & \texttt{FillStroke}\\*
%   & \texttt{3} & \texttt{Invisible}\\*
%   & \texttt{4} & \texttt{FillClip}\\*
%   & \texttt{5} & \texttt{StrokeClip}\\*
%   & \texttt{6} & \texttt{FillStrokeClip}\\*
%   & \texttt{7} & \texttt{Clip}\\
% \hline
% \xoption{LineWidth}
%   & \itshape positive number, & \itshape \TeX\ dimen\\*[-.2ex]
%   & \qquad \itshape unit is bp & \\
% \hline
% \xoption{LineCapStyle}
%   & \texttt{0} & \texttt{Butt}\\*
%   & \texttt{1} & \texttt{Round}\\*
%   & \texttt{2} & \texttt{ProjectingSquare}\\
% \hline
% \xoption{LineJoinStyle}
%   & \texttt{0} & \texttt{Miter}\\*
%   & \texttt{1} & \texttt{Round}\\*
%   & \texttt{2} & \texttt{Bevel}\\
% \hline
% \xoption{MiterLimit}
%   & \multicolumn{2}{l@{}}{\itshape positive number} \\
% \hline
% \xoption{Flatness}
%   & \multicolumn{2}{l@{}}{\itshape number between 0 and 100} \\
% \hline
% \xoption{LineDashPattern}
%   & \multicolumn{2}{l@{}}{^^A
%       \itshape numbers in square brackets,^^A
%     } \\*[-.2ex]
%   & \multicolumn{2}{l@{}}{^^A
%       \qquad \itshape followed by number,^^A
%     } \\*
%   & \multicolumn{2}{l@{}}{^^A
%       \qquad \itshape units are bp^^A
%     } \\
% \hline
% \xoption{RenderingIntent}
%   & \multicolumn{2}{l@{}}{\texttt{AbsoluteColorimetric}} \\*
%   & \multicolumn{2}{l@{}}{\texttt{RelativeColorimetric}} \\*
%   & \multicolumn{2}{l@{}}{\texttt{Saturation}} \\*
%   & \multicolumn{2}{l@{}}{\texttt{Perceptual}} \\
% \hline
% \xoption{FillColor}
%   & & \itshape color specification \\
% \hline
% \xoption{StrokeColor}
%   & & \itshape color specification \\
% \end{longtable}
%
% \subsubsection{Details}
%
% The description and specification of these parameters
% are available in the PDF specification \cite{pdf-spec}.
% Therefore they are not repeated here.
%
% \begin{description}
% \item[Value:]
%   The values in the second column lists or describe
%   the values that are specified by the PDF specification.
% \item[Alias:] Instead of magic numbers the package also
%   defines some aliases that can be given as value.
%   Example: |LineCapStyle=Round| has the same effect as
%   |LineCapStyle=1|.
% \item[Number:] The term \emph{number} means an integer
%   or real number. The real number is given as plain
%   decimal number without exponent. The decimal separator is a period.
%   At least one digit must be present.
% \item[\xoption{LineWidth}:] As alias a \TeX\ dimen specification
%   can be given. This includes explicit specifications with number
%   and unit, e.g. |LineWidth=0.5pt|. Also \LaTeX\ length registers
%   may be used. If \eTeX's \cs{dimexpr} is available, then it is
%   automatically added. However package \xpackage{calc} is not
%   supported.
% \item[\xoption{FillColor}, \xoption{StrokeColor}:]
%   Package \xpackage{color} or \xpackage{xcolor} must be loaded
%   before these options can be used (since version 1.2).
%   \LaTeX's color support sets both colors at the same time
%   to the same value. However parameter \xoption{TextRenderingMode}
%   offers the value \texttt{FillStroke} that makes only sense, if
%   the two color types can be set separately.
%   If one of the options \xoption{FillColor} or \xoption{StrokeColor}
%   is specified, then also the color is set. For compatibility
%   with the \LaTeX\ color packages (\xpackage{color} or \xpackage{xcolor}),
%   always both colors must be set. Thus if one of them is not specified,
%   it is taken from the current color.
%
%   Both options \xoption{FillColor} and \xoption{StrokeColor}
%   expect a \LaTeX\ color specification as value.
%   Also the optional color model argument is supported.
%   Example:
%   \begin{quote}
%     |FillColor=yellow|,\\
%     |StrokeColor=[cmyk]{1,.5,0,0}|\\
%   \end{quote}
% \end{description}
%
% \subsection{Color stack}
%
% If the \pdfTeX\ version provides color stacks, then each
% parameter is assigned a page based color stack. The assignment
% of a stack takes place, when its parameter is set the first time.
% This avoids the use of color stacks that are not needed.
%
% \StopEventually{
% }
%
% \section{Implementation}
%    \begin{macrocode}
%<*package>
%    \end{macrocode}
%    Reload check, especially if the package is not used with \LaTeX.
%    \begin{macrocode}
\begingroup\catcode61\catcode48\catcode32=10\relax%
  \catcode13=5 % ^^M
  \endlinechar=13 %
  \catcode35=6 % #
  \catcode39=12 % '
  \catcode44=12 % ,
  \catcode45=12 % -
  \catcode46=12 % .
  \catcode58=12 % :
  \catcode64=11 % @
  \catcode123=1 % {
  \catcode125=2 % }
  \expandafter\let\expandafter\x\csname ver@pdfrender.sty\endcsname
  \ifx\x\relax % plain-TeX, first loading
  \else
    \def\empty{}%
    \ifx\x\empty % LaTeX, first loading,
      % variable is initialized, but \ProvidesPackage not yet seen
    \else
      \expandafter\ifx\csname PackageInfo\endcsname\relax
        \def\x#1#2{%
          \immediate\write-1{Package #1 Info: #2.}%
        }%
      \else
        \def\x#1#2{\PackageInfo{#1}{#2, stopped}}%
      \fi
      \x{pdfrender}{The package is already loaded}%
      \aftergroup\endinput
    \fi
  \fi
\endgroup%
%    \end{macrocode}
%    Package identification:
%    \begin{macrocode}
\begingroup\catcode61\catcode48\catcode32=10\relax%
  \catcode13=5 % ^^M
  \endlinechar=13 %
  \catcode35=6 % #
  \catcode39=12 % '
  \catcode40=12 % (
  \catcode41=12 % )
  \catcode44=12 % ,
  \catcode45=12 % -
  \catcode46=12 % .
  \catcode47=12 % /
  \catcode58=12 % :
  \catcode64=11 % @
  \catcode91=12 % [
  \catcode93=12 % ]
  \catcode123=1 % {
  \catcode125=2 % }
  \expandafter\ifx\csname ProvidesPackage\endcsname\relax
    \def\x#1#2#3[#4]{\endgroup
      \immediate\write-1{Package: #3 #4}%
      \xdef#1{#4}%
    }%
  \else
    \def\x#1#2[#3]{\endgroup
      #2[{#3}]%
      \ifx#1\@undefined
        \xdef#1{#3}%
      \fi
      \ifx#1\relax
        \xdef#1{#3}%
      \fi
    }%
  \fi
\expandafter\x\csname ver@pdfrender.sty\endcsname
\ProvidesPackage{pdfrender}%
  [2019/12/29 v1.6 Access to some PDF graphics parameters (HO)]%
%    \end{macrocode}
%
%    \begin{macrocode}
\begingroup\catcode61\catcode48\catcode32=10\relax%
  \catcode13=5 % ^^M
  \endlinechar=13 %
  \catcode123=1 % {
  \catcode125=2 % }
  \catcode64=11 % @
  \def\x{\endgroup
    \expandafter\edef\csname PdfRender@AtEnd\endcsname{%
      \endlinechar=\the\endlinechar\relax
      \catcode13=\the\catcode13\relax
      \catcode32=\the\catcode32\relax
      \catcode35=\the\catcode35\relax
      \catcode61=\the\catcode61\relax
      \catcode64=\the\catcode64\relax
      \catcode123=\the\catcode123\relax
      \catcode125=\the\catcode125\relax
    }%
  }%
\x\catcode61\catcode48\catcode32=10\relax%
\catcode13=5 % ^^M
\endlinechar=13 %
\catcode35=6 % #
\catcode64=11 % @
\catcode123=1 % {
\catcode125=2 % }
\def\TMP@EnsureCode#1#2{%
  \edef\PdfRender@AtEnd{%
    \PdfRender@AtEnd
    \catcode#1=\the\catcode#1\relax
  }%
  \catcode#1=#2\relax
}
\TMP@EnsureCode{10}{12}% ^^J
\TMP@EnsureCode{36}{3}% $
\TMP@EnsureCode{39}{12}% '
\TMP@EnsureCode{40}{12}% (
\TMP@EnsureCode{41}{12}% )
\TMP@EnsureCode{42}{12}% *
\TMP@EnsureCode{43}{12}% +
\TMP@EnsureCode{44}{12}% ,
\TMP@EnsureCode{45}{12}% -
\TMP@EnsureCode{46}{12}% .
\TMP@EnsureCode{47}{12}% /
\TMP@EnsureCode{58}{12}% :
\TMP@EnsureCode{59}{12}% ;
\TMP@EnsureCode{60}{12}% <
\TMP@EnsureCode{62}{12}% >
\TMP@EnsureCode{63}{12}% ?
\TMP@EnsureCode{91}{12}% [
\TMP@EnsureCode{93}{12}% ]
\TMP@EnsureCode{94}{7}% ^ (superscript)
\TMP@EnsureCode{96}{12}% `
\TMP@EnsureCode{124}{12}% |
%    \end{macrocode}
%
%    \begin{macrocode}
\def\PdfRender@AtEndHook{}
\expandafter\def\expandafter\PdfRender@AtEnd\expandafter{%
  \expandafter\PdfRender@AtEndHook
  \PdfRender@AtEnd
  \endinput
}
%    \end{macrocode}
%
% \subsection{Look for \pdfTeX, its mode and features}
%
%    \begin{macro}{\PdfRender@newif}
%    \begin{macrocode}
\def\PdfRender@newif#1{%
  \expandafter\edef\csname PdfRender@#1true\endcsname{%
    \let
    \expandafter\noexpand\csname ifPdfRender@#1\endcsname
    \noexpand\iftrue
  }%
  \expandafter\edef\csname PdfRender@#1false\endcsname{%
    \let
    \expandafter\noexpand\csname ifPdfRender@#1\endcsname
    \noexpand\iffalse
  }%
  \csname PdfRender@#1false\endcsname
}
%    \end{macrocode}
%    \end{macro}
%
%    \begin{macro}{\ifPdfRender@Stack}
%    \begin{macrocode}
\PdfRender@newif{Stack}
%    \end{macrocode}
%    \end{macro}
%    \begin{macro}{\ifPdfRender@Match}
%    \begin{macrocode}
\PdfRender@newif{Match}
%    \end{macrocode}
%    \end{macro}
%
%    \begin{macro}{\PdfRender@RequirePackage}
%    \begin{macrocode}
\begingroup\expandafter\expandafter\expandafter\endgroup
\expandafter\ifx\csname RequirePackage\endcsname\relax
  \def\PdfRender@RequirePackage#1[#2]{%
    \expandafter\def\expandafter\PdfRender@AtEndHook\expandafter{%
      \PdfRender@AtEndHook
      \ltx@ifpackagelater{#1}{#2}{}{%
        \@PackageWarningNoLine{pdfrender}{%
          You have requested version\MessageBreak
          `#2' of package `#1',\MessageBreak
          but only version\MessageBreak
          `\csname ver@#1.\ltx@pkgextension\endcsname'\MessageBreak
          is available%
        }%
      }%
    }%
    \input #1.sty\relax
  }%
\else
  \let\PdfRender@RequirePackage\RequirePackage
\fi
%    \end{macrocode}
%    \end{macro}
% Luatex compatibility
%    \begin{macrocode}
\ifx\pdfextension\@undefined\else
  \def\pdfcolorstackinit {\pdffeedback colorstackinit}
  \protected\def\pdfcolorstack     {\pdfextension colorstack}
  \protected\def\pdfliteral        {\pdfextension literal}
\fi
%    \end{macrocode}
%
%    \begin{macrocode}
\PdfRender@RequirePackage{iftex}[2019/11/07]
\PdfRender@RequirePackage{infwarerr}[2007/09/09]
\PdfRender@RequirePackage{ltxcmds}[2010/01/28]
%    \end{macrocode}
%
%    \begin{macrocode}
\ifpdf
  \ltx@IfUndefined{pdfcolorstackinit}{%
    \@PackageWarning{pdfrender}{%
      Missing \string\pdfcolorstackinit
    }%
  }{%
    \PdfRender@Stacktrue
  }%
  \ltx@IfUndefined{pdfmatch}{%
    \@PackageInfoNoLine{pdfrender}{%
      \string\pdfmatch\ltx@space not found. %
      Therefore the values\MessageBreak
      of some parameters are not validated%
    }%
  }{%
    \PdfRender@Matchtrue
  }%
\else
  \@PackageWarning{pdfrender}{%
    Missing pdfTeX in PDF mode%
  }%
%    \end{macrocode}
%    \begin{macrocode}
  \ltx@IfUndefined{newcommand}{%
%    \end{macrocode}
%    \begin{macro}{\pdfrender}
%    \begin{macrocode}
    \def\pdfrender#1{}%
%    \end{macrocode}
%    \end{macro}
%    \begin{macro}{\textpdfrender}
%    \begin{macrocode}
    \long\def\textpdfrender#1#2{#2}%
%    \end{macrocode}
%    \end{macro}
%    \begin{macrocode}
  }{%
%    \end{macrocode}
%    \begin{macro}{\pdfrender}
%    \begin{macrocode}
    \newcommand*{\pdfrender}[1]{}%
%    \end{macrocode}
%    \end{macro}
%    \begin{macro}{\textpdfrender}
%    \begin{macrocode}
    \newcommand{\textpdfrender}[2]{#2}%
%    \end{macrocode}
%    \end{macro}
%    \begin{macrocode}
  }%
  \expandafter\PdfRender@AtEnd
\fi%
%    \end{macrocode}
%
% \subsection{Enable color support of \LaTeX}
%
%    \begin{macrocode}
\ltx@ifpackageloaded{color}{}{%
  \def\color@setgroup{\begingroup\set@color}%
  \let\color@begingroup\begingroup
  \def\color@endgroup{\endgraf\endgroup}%
  \def\color@hbox{\hbox\bgroup\color@begingroup}%
  \def\color@vbox{\vbox\bgroup\color@begingroup}%
  \def\color@endbox{\color@endgroup\egroup}%
  \ltx@ifundefined{bgroup}{%
    \let\bgroup={\let\egroup=}%
  }{}%
  \ltx@ifundefined{endgraf}{%
    \let\endgraf=\par
  }{}%
}
%    \end{macrocode}
%
% \subsection{Hook into \cs{normalcolor}}
%
%    The problem is that packages \xpackage{color} and \xpackage{xcolor}
%    each overwrite \cs{normalcolor}. For example, after the package
%    loading order \xpackage{color}, \xpackage{pdfrender} and \xpackage{xcolor}
%    the patched version of \cs{normalcolor} is overwritten by
%    package \xpackage{xcolor}. Also using \cs{AtBeginDocument} for
%    patching is not enough. If package \xpackage{hyperref} is loaded
%    later, it might load package \xpackage{color} using \cs{AtBeginDocument}.
%    \begin{macro}{\PdfRender@NormalColorHook}
%    \begin{macrocode}
\def\PdfRender@NormalColorHook{}
%    \end{macrocode}
%    \end{macro}
%    \begin{macro}{\PdfRender@ColorSetGroupHook}
%    \begin{macrocode}
\def\PdfRender@ColorSetGroupHook{}
%    \end{macrocode}
%    \end{macro}
%    \begin{macro}{\PdfRender@TestBox}
%    \begin{macrocode}
\def\PdfRender@TestBox#1{%
  \setbox0=\color@hbox#1\color@endbox
}
%    \end{macrocode}
%    \end{macro}
%    \begin{macro}{\PdfRender@PatchNormalColor}
%    \begin{macrocode}
\def\PdfRender@PatchNormalColor{%
  \ltx@ifundefined{normalcolor}{%
    \gdef\normalcolor{\PdfRender@NormalColorHook}%
  }{%
    \begingroup
      \def\PdfRender@NormalColorHook{\let\PdfRender@temp=Y}%
      \PdfRender@TestBox{%
        \let\set@color\relax
        \normalcolor
        \ifx\PdfRender@temp Y%
        \else
          \ltx@GlobalAppendToMacro\normalcolor{%
            \PdfRender@NormalColorHook
          }%
        \fi
      }%
    \endgroup
  }%
  \ifx\@nodocument\relax
    \global\let\PdfRender@PatchNormalColor\relax
  \fi
}%
%    \end{macrocode}
%    \end{macro}
%    \begin{macro}{\PdfRender@PatchColorSetGroup}
%    \begin{macrocode}
\def\PdfRender@PatchColorSetGroup{%
  \begingroup
    \def\PdfRender@ColorSetGroupHook{\let\PdfRender@temp=Y}%
    \PdfRender@TestBox{%
      \let\set@color\relax
      \color@setgroup\color@endgroup
      \ifx\PdfRender@temp Y%
      \else
        \ltx@GlobalAppendToMacro\color@setgroup{%
          \PdfRender@ColorSetGroupHook
        }%
      \fi
    }%
  \endgroup
  \ifx\@nodocument\relax
    \global\let\PdfRender@PatchColorSetGroup\relax
  \fi
}%
%    \end{macrocode}
%    \end{macro}
%    \begin{macro}{\PdfRender@PatchColor}
%    \begin{macrocode}
\def\PdfRender@PatchColor{%
  \PdfRender@PatchNormalColor
  \PdfRender@PatchColorSetGroup
}
%    \end{macrocode}
%    \end{macro}
%    \begin{macrocode}
\PdfRender@PatchColor
\ltx@IfUndefined{AtBeginDocument}{}{%
  \AtBeginDocument{\PdfRender@PatchColor}%
}
%    \end{macrocode}
%    \cs{AfterPackage} is provided by package \xpackage{scrlfile}.
%    \begin{macrocode}
\ltx@IfUndefined{AfterPackage}{%
}{%
  \AfterPackage{color}{\PdfRender@PatchColor}%
  \AfterPackage{xcolor}{\PdfRender@PatchColor}%
  \AfterPackage{etoolbox}{%
    \AfterEndPreamble{\PdfRender@PatchColor}%
  }%
}%
%    \end{macrocode}
%    \cs{AfterEndPreamble} is provided by package \xpackage{etoolbox}.
%    \begin{macrocode}
\ltx@IfUndefined{AfterEndPreamble}{%
}{%
  \AfterEndPreamble{\PdfRender@PatchColor}%
}%
%    \end{macrocode}
%
%    \begin{macrocode}
\PdfRender@RequirePackage{kvsetkeys}[2010/01/28]
%    \end{macrocode}
%    \begin{macro}{\PdfRender@texorpdfstring}
%    \begin{macrocode}
\def\PdfRender@texorpdfstring{%
  \ltx@IfUndefined{texorpdfstring}\ltx@firstoftwo\texorpdfstring
}
%    \end{macrocode}
%    \end{macro}
%    \begin{macro}{\pdfrender}
%    \begin{macrocode}
\ltx@IfUndefined{DeclareRobustCommand}%
\ltx@firstoftwo\ltx@secondoftwo
{%
  \def\pdfrender#1%
}{%
  \newcommand{\pdfrender}{}%
  \DeclareRobustCommand*{\pdfrender}[1]%
}%
{%
  \PdfRender@texorpdfstring{%
    \PdfRender@PatchNormalColor
    \global\let\PdfRender@FillColor\ltx@empty
    \global\let\PdfRender@StrokeColor\ltx@empty
    \kvsetkeys{PDFRENDER}{#1}%
    \PdfRender@SetColor
  }{}%
}
%    \end{macrocode}
%    \end{macro}
%    \begin{macro}{\textpdfrender}
%    \begin{macrocode}
\ltx@IfUndefined{DeclareRobustCommand}%
\ltx@firstoftwo\ltx@secondoftwo
{%
  \long\def\textpdfrender#1#2%
}{%
  \newcommand{\textpdfrender}{}%
  \DeclareRobustCommand{\textpdfrender}[2]%
}%
{%
  \PdfRender@texorpdfstring{%
    \begingroup
      \pdfrender{#1}%
      #2%
    \endgroup
  }{#2}%
}
%    \end{macrocode}
%    \end{macro}
%
%    \begin{macro}{\ifPdfRender@Values}
%    \begin{macrocode}
\PdfRender@newif{Values}
%    \end{macrocode}
%    \end{macro}
%    \begin{macro}{\PdfRender@NewClassValues}
%    \begin{macrocode}
\def\PdfRender@NewClassValues#1#2#3#4{%
  \PdfRender@Valuestrue
  \PdfRender@NewClass{#1}{#2}{#3}{#4}{}%
}
%    \end{macrocode}
%    \end{macro}
%    \begin{macro}{\PdfRender@NewClass}
%    \begin{macrocode}
\def\PdfRender@NewClass#1#2#3#4#5{%
  \PdfRender@newif{Active#1}%
  \expandafter\def\csname PdfRender@Default#1\endcsname{#2}%
  \expandafter\let\csname PdfRender@Current#1\expandafter\endcsname
      \csname PdfRender@Default#1\endcsname
  \ifPdfRender@Stack
    \expandafter\edef\csname PdfRender@Init#1\endcsname{%
      \global\chardef
      \expandafter\noexpand\csname PdfRender@Stack#1\endcsname=%
          \noexpand\pdfcolorstackinit page direct{%
            \noexpand#3%
            \expandafter\noexpand\csname PdfRender@Default#1\endcsname
          }\relax
      \noexpand\@PackageInfo{pdfrender}{%
        New color stack `#1' = \noexpand\number
        \expandafter\noexpand\csname PdfRender@Stack#1\endcsname
      }%
      \gdef\expandafter\noexpand\csname PdfRender@Init#1\endcsname{}%
    }%
    \expandafter\edef\csname PdfRender@Set#1\endcsname{%
      \expandafter\noexpand\csname PdfRender@Init#1\endcsname
      \noexpand\pdfcolorstack
      \expandafter\noexpand\csname PdfRender@Stack#1\endcsname
      push{%
        #3{\expandafter\noexpand\csname PdfRender@Current#1\endcsname}%
      }%
      \noexpand\aftergroup
      \expandafter\noexpand\csname PdfRender@Reset#1\endcsname
    }%
    \expandafter\edef\csname PdfRender@Reset#1\endcsname{%
      \expandafter\noexpand\csname PdfRender@Init#1\endcsname
      \noexpand\pdfcolorstack
      \expandafter\noexpand\csname PdfRender@Stack#1\endcsname
      pop\relax
    }%
  \else
    \expandafter\edef\csname PdfRender@Set#1\endcsname{%
      \noexpand\pdfliteral direct{%
        #3{\expandafter\noexpand\csname PdfRender@Current#1\endcsname}%
      }%
      \noexpand\aftergroup
      \expandafter\noexpand\csname PdfRender@Reset#1\endcsname
    }%
    \expandafter\edef\csname PdfRender@Reset#1\endcsname{%
      \noexpand\pdfliteral direct{%
        #3{\expandafter\noexpand\csname PdfRender@Current#1\endcsname}%
      }%
    }%
  \fi
  \expandafter\edef\csname PdfRender@Normal#1\endcsname{%
    \let
    \expandafter\noexpand\csname PdfRender@Current#1\endcsname
    \expandafter\noexpand\csname PdfRender@Default#1\endcsname
    \noexpand\PdfRender@Set{#1}%
  }%
  \expandafter\ltx@GlobalAppendToMacro\expandafter\PdfRender@NormalColorHook
  \expandafter{%
    \csname PdfRender@Normal#1\endcsname
  }%
  \ltx@GlobalAppendToMacro\PdfRender@ColorSetGroupHook{%
    \PdfRender@Set{#1}%
  }%
  \ifPdfRender@Values
    \kv@parse@normalized{#4}{%
      \expandafter\let\csname PdfRender@#1@\kv@key\endcsname\kv@key
      \ifx\kv@value\relax
      \else
        \expandafter\let\csname PdfRender@#1@\kv@value\endcsname\kv@key
      \fi
      \ltx@gobbletwo
    }%
    \PdfRender@define@key{PDFRENDER}{#1}{%
      \global\csname PdfRender@Active#1true\endcsname
      \def\PdfRender@Current{##1}%
      \PdfRender@SetValidateValues{#1}%
    }%
    \PdfRender@Valuesfalse
  \else
    \PdfRender@define@key{PDFRENDER}{#1}{%
      \global\csname PdfRender@Active#1true\endcsname
      \expandafter\def\csname PdfRender@Current#1\endcsname{##1}%
      \ltx@IfUndefined{PdfRender@PostProcess#1}{%
      }{%
        \csname PdfRender@PostProcess#1\endcsname
      }%
      \PdfRender@SetValidate{#1}{#4}{#5}%
    }%
  \fi
}%
%    \end{macrocode}
%    \end{macro}
%    \begin{macro}{\PdfRender@define@key}
%    \begin{macrocode}
\ltx@IfUndefined{define@key}{%
  \def\PdfRender@define@key#1#2{%
    \expandafter\def\csname KV@#1@#2\endcsname##1%
  }%
}{%
  \let\PdfRender@define@key\define@key
}
%    \end{macrocode}
%    \end{macro}
%    \begin{macro}{\PdfRender@Set}
%    \begin{macrocode}
\def\PdfRender@Set#1{%
  \csname ifPdfRender@Active#1\endcsname
    \csname PdfRender@Set#1\expandafter\endcsname
  \fi
}
%    \end{macrocode}
%    \end{macro}
%    \begin{macro}{\PdfRender@Reset}
%    \begin{macrocode}
\def\PdfRender@Reset#1{%
  \csname ifPdfRender@Active#1\endcsname
    \csname PdfRender@Reset#1\expandafter\endcsname
  \fi
}
%    \end{macrocode}
%    \end{macro}
%    \begin{macro}{\PdfRender@ErrorInvalidValue}
%    \begin{macrocode}
\def\PdfRender@ErrorInvalidValue#1{%
  \PackageError{pdfrender}{%
    Ignoring parameter setting for `#1'\MessageBreak
    because of invalid value %
    `\csname PdfRender@Current#1\endcsname'%
  }\@ehc
  \expandafter\let\csname PdfRender@Current#1\endcsname\ltx@empty
}%
%    \end{macrocode}
%    \end{macro}
%    \begin{macro}{\PdfRender@SetValidate}
%    \begin{macrocode}
\ifPdfRender@Match
  \def\PdfRender@SetValidate#1#2#3{%
    \ifnum\pdfmatch{^(#2)$}{\csname PdfRender@Current#1\endcsname}=1 %
      \csname PdfRender@Set#1\expandafter\endcsname
    \else
      \PdfRender@ErrorInvalidValue{#1}%
    \fi
  }%
\else
  \def\PdfRender@SetValidate#1#2#3{%
    \expandafter\let\expandafter\PdfRender@Current
    \csname PdfRender@Current#1\endcsname
    #3%
    \ifx\PdfRender@Current\@empty
      \PdfRender@ErrorInvalidValue{#1}%
    \else
      \csname PdfRender@Set#1\expandafter\endcsname
    \fi
  }%
\fi
%    \end{macrocode}
%    \end{macro}
%    \begin{macro}{\PdfRender@SetValidateValues}
%    \begin{macrocode}
\def\PdfRender@SetValidateValues#1{%
  \ltx@IfUndefined{PdfRender@#1@\PdfRender@Current}{%
    \expandafter\let\csname PdfRender@Current#1\endcsname
                    \PdfRender@Current
    \PdfRender@ErrorInvalidValue{#1}%
  }{%
    \expandafter\edef\csname PdfRender@Current#1\endcsname{%
      \csname PdfRender@#1@\PdfRender@Current\endcsname
    }%
    \csname PdfRender@Set#1\endcsname
  }%
}
%    \end{macrocode}
%    \end{macro}
%    \begin{macro}{\PdfRender@OpValue}
%    \begin{macrocode}
\def\PdfRender@OpValue#1#2{#2\ltx@space#1}%
%    \end{macrocode}
%    \end{macro}
%    \begin{macro}{\PdfRender@OpName}
%    \begin{macrocode}
\def\PdfRender@OpName#1#2{/#2\ltx@space#1}%
%    \end{macrocode}
%    \end{macro}
%
% \subsection{Declare and setup parameters}
%
%    \begin{macrocode}
\PdfRender@NewClassValues{TextRenderingMode}%
                      {0}%
                      {\PdfRender@OpValue{Tr}}{%
  0=Fill,%
  1=Stroke,%
  2=FillStroke,%
  3=Invisible,%
  4=FillClip,%
  5=StrokeClip,%
  6=FillStrokeClip,%
  7=Clip,%
}%
\PdfRender@NewClass{LineWidth}{1}{\PdfRender@OpValue{w}}{%
  [0-9]+\string\.?[0-9]*|\string\.[0-9]+%
}{}%
\ltx@IfUndefined{dimexpr}{%
  \def\PdfRender@dimexpr{}%
}{%
  \let\PdfRender@dimexpr\dimexpr
}
\def\PdfRender@PostProcessLineWidth{%
  \begingroup
  \afterassignment\PdfRender@@PostProcessLineWidth
  \dimen0=\PdfRender@dimexpr\PdfRender@CurrentLineWidth bp %
  \PdfRender@let\PdfRender@relax\PdfRender@relax
}
\let\PdfRender@let\let
\let\PdfRender@relax\relax
\def\PdfRender@@PostProcessLineWidth#1\PdfRender@let{%
  \ifx\\#1\\%
    \endgroup
  \else
    \dimen0=.996264\dimen0 % 72/72.27
    \edef\x{\endgroup
      \def\noexpand\PdfRender@CurrentLineWidth{%
        \strip@pt\dimen0%
      }%
    }%
    \expandafter\x
  \fi
}
\PdfRender@NewClassValues{LineCapStyle}{0}{\PdfRender@OpValue{J}}{%
  0=Butt,%
  1=Round,%
  2=ProjectingSquare,%
}%
\PdfRender@NewClassValues{LineJoinStyle}{0}{\PdfRender@OpValue{j}}{%
  0=Miter,%
  1=Round,%
  2=Bevel,%
}%
\PdfRender@NewClass{MiterLimit}{10}{\PdfRender@OpValue{M}}{%
  [0-9]*[1-9][0-9]*\string\.?[0-9]*|%
  [0-9]*\string\.?[0-9]*[1-9][0-9]*%
}{}%
\PdfRender@NewClass{Flatness}{0}{\PdfRender@OpValue{i}}{%
  100(\string\.0*)?|[0-9][0-9](\string\.[0-9]*)?|\string\.[0-9]+%
}{}%
\PdfRender@NewClass{LineDashPattern}{[]0}{\PdfRender@OpValue{d}}{%
  \string\[%
  ( ?([0-9]+\string\.?[0-9]*|\string\.[0-9]+) ?)*%
  \string\] ?%
  ([0-9]+\string\.?[0-9]*|\string\.[0-9]+)%
}{}%
\PdfRender@NewClassValues{RenderingIntent}%
                      {RelativeColorimetric}%
                      {\PdfRender@OpName{ri}}{%
  AbsoluteColorimetric,%
  RelativeColorimetric,%
  Saturation,%
  Perceptual,%
}%
%    \end{macrocode}
%
% \subsection{Fill and stroke color support}
%
%    \begin{macrocode}
\PdfRender@define@key{PDFRENDER}{FillColor}{%
  \begingroup
    \def\PdfRender@Color{#1}%
    \ifx\PdfRender@Color\ltx@empty
      \global\let\PdfRender@FillColor\ltx@empty
    \else
      \PdfRender@ColorAvailable{%
        \PdfRender@TestBox{%
          \expandafter\PdfRender@TryColor\PdfRender@Color\ltx@empty
          \PdfRender@GetFillColor
          \ifx\PdfRender@FillColor\ltx@empty
            \@PackageWarning{pdfrender}{%
              Cannot extract fill color\MessageBreak
              from value `#1'%
            }%
          \fi
        }%
      }%
    \fi
  \endgroup
}
\PdfRender@define@key{PDFRENDER}{StrokeColor}{%
  \begingroup
    \def\PdfRender@Color{#1}%
    \ifx\PdfRender@Color\ltx@empty
      \global\let\PdfRender@StrokeColor\ltx@empty
    \else
      \PdfRender@ColorAvailable{%
        \PdfRender@TestBox{%
          \expandafter\PdfRender@TryColor\PdfRender@Color\ltx@empty
          \PdfRender@GetStrokeColor
          \ifx\PdfRender@StrokeColor\ltx@empty
            \@PackageWarning{pdfrender}{%
              Cannot extract stroke color\MessageBreak
              from value `#1'%
            }%
          \fi
        }%
      }%
    \fi
  \endgroup
}
%    \end{macrocode}
%    \begin{macro}{\PdfRender@ColorAvailable}
%    \begin{macrocode}
\def\PdfRender@ColorAvailable{%
  \@ifundefined{set@color}{%
    \@PackageError{pdfrender}{%
      Ignoring color options, because neither\MessageBreak
      package `color' nor package `xcolor' is loaded%
    }\@ehc
    \global\let\PdfRender@ColorAvailable\ltx@gobble
  }{%
    \global\let\PdfRender@ColorAvailable\ltx@firstofone
  }%
  \PdfRender@ColorAvailable
}
%    \end{macrocode}
%    \end{macro}
%    \begin{macro}{\PdfRender@TryColor}
%    \begin{macrocode}
\def\PdfRender@TryColor{%
  \@ifnextchar[\color\PdfRender@@TryColor
}
%    \end{macrocode}
%    \end{macro}
%    \begin{macro}{\PdfRender@@TryColor}
%    \begin{macrocode}
\def\PdfRender@@TryColor#1\ltx@empty{%
  \expandafter\color\expandafter{\PdfRender@Color}%
}
%    \end{macrocode}
%    \end{macro}
%    \begin{macro}{\PdfRender@SetColor}
%    \begin{macrocode}
\def\PdfRender@SetColor{%
  \chardef\PdfRender@NeedsCurrentColor=0 %
  \ifx\PdfRender@FillColor\ltx@empty
    \ifx\PdfRender@StrokeColor\ltx@empty
    \else
      \edef\PdfRender@CurrentColor{%
        \noexpand\PdfRender@FillColor\ltx@space\PdfRender@StrokeColor
      }%
      \chardef\PdfRender@NeedsCurrentColor=1 %
    \fi
  \else
    \ifx\PdfRender@StrokeColor\ltx@empty
      \edef\PdfRender@CurrentColor{%
        \PdfRender@FillColor\ltx@space\noexpand\PdfRender@StrokeColor
      }%
      \chardef\PdfRender@NeedsCurrentColor=2 %
    \else
      \edef\current@color{%
        \PdfRender@FillColor\ltx@space\PdfRender@StrokeColor
      }%
      \set@color
    \fi
  \fi
  \ifnum\PdfRender@NeedsCurrentColor=1 %
    \PdfRender@GetFillColor
    \ifx\PdfRender@FillColor\ltx@empty
      \@PackageWarning{pdfrender}{%
        Cannot extract current fill color%
      }%
    \else
      \edef\current@color{\PdfRender@CurrentColor}%
      \set@color
    \fi
  \else
    \ifnum\PdfRender@NeedsCurrentColor=2 %
      \PdfRender@GetStrokeColor
      \ifx\PdfRender@StrokeColor\ltx@empty
        \@PackageWarning{pdfrender}{%
          Cannot extract current stroke color%
        }%
      \else
        \edef\current@color{\PdfRender@CurrentColor}%
        \set@color
      \fi
    \fi
  \fi
}
%    \end{macrocode}
%    \end{macro}
%    \begin{macro}{\PdfRender@PatternFillColor}
%    \begin{macrocode}
\edef\PdfRender@PatternFillColor{ % space
  (%
    [0-9\string\.]+ g|%
    [0-9\string\.]+ [0-9\string\.]+ [0-9\string\.]+ rg|%
    [0-9\string\.]+ [0-9\string\.]+ %
    [0-9\string\.]+ [0-9\string\.]+ k%
  ) % space
  (.*)$%
}
%    \end{macrocode}
%    \end{macro}
%    \begin{macro}{\PdfRender@PatternStrokeColor}
%    \begin{macrocode}
\edef\PdfRender@PatternStrokeColor{ % space
  (%
    [0-9\string\.]+ G|%
    [0-9\string\.]+ [0-9\string\.]+ [0-9\string\.]+ RG|%
    [0-9\string\.]+ [0-9\string\.]+ %
    [0-9\string\.]+ [0-9\string\.]+ K%
  ) % space
  (.*)$%
}
%    \end{macrocode}
%    \end{macro}
%    \begin{macro}{\PdfRender@MatchPattern}
%    \begin{macrocode}
\def\PdfRender@MatchPattern#1{%
  \ifnum\pdfmatch{\PdfRender@Pattern}{\PdfRender@String}=1 %
    \xdef#1{%
      \expandafter\strip@prefix\pdflastmatch 1%
    }%
    \edef\PdfRender@String{%
      \expandafter\strip@prefix\pdflastmatch 2%
    }%
    \ifx\PdfRender@String\ltx@empty
    \else
      \expandafter\expandafter\expandafter\PdfRender@MatchPattern
      \expandafter\expandafter\expandafter#1%
    \fi
  \fi
}
%    \end{macrocode}
%    \end{macro}
%    \begin{macro}{\PdfRender@GetFillColor}
%    \begin{macrocode}
\def\PdfRender@GetFillColor{%
  \global\let\PdfRender@FillColor\ltx@empty
  \begingroup
    \ifPdfRender@Match
      \let\PdfRender@Pattern\PdfRender@PatternFillColor
      \edef\PdfRender@String{\ltx@space\current@color\ltx@space}%
      \PdfRender@MatchPattern\PdfRender@FillColor
    \else
      \edef\current@color{\current@color\ltx@space}%
      \let\PdfRender@OP\relax
      \PdfRender@FindOp{g}0%
      \PdfRender@FindOp{G}1%
      \PdfRender@FindOp{rg}0%
      \PdfRender@FindOp{RG}1%
      \PdfRender@FindOp{k}0%
      \PdfRender@FindOp{K}1%
      \PdfRender@FilterOp 0\PdfRender@FillColor
    \fi
  \endgroup
}
%    \end{macrocode}
%    \end{macro}
%    \begin{macro}{\PdfRender@GetStrokeColor}
%    \begin{macrocode}
\def\PdfRender@GetStrokeColor{%
  \global\let\PdfRender@StrokeColor\ltx@empty
  \begingroup
    \ifPdfRender@Match
      \let\PdfRender@Pattern\PdfRender@PatternStrokeColor
      \edef\PdfRender@String{\ltx@space\current@color\ltx@space}%
      \PdfRender@MatchPattern\PdfRender@StrokeColor
    \else
      \edef\current@color{\current@color\ltx@space}%
      \let\PdfRender@OP\relax
      \PdfRender@FindOp{g}0%
      \PdfRender@FindOp{G}1%
      \PdfRender@FindOp{rg}0%
      \PdfRender@FindOp{RG}1%
      \PdfRender@FindOp{k}0%
      \PdfRender@FindOp{K}1%
      \PdfRender@FilterOp 1\PdfRender@StrokeColor
    \fi
  \endgroup
}
%    \end{macrocode}
%    \end{macro}
%    \begin{macrocode}
\ifPdfRender@Match
  \expandafter\PdfRender@AtEnd
\fi%
%    \end{macrocode}
%    \begin{macro}{\PdfRender@FindOp}
%    \begin{macrocode}
\def\PdfRender@FindOp#1#2{%
  \def\PdfRender@temp##1 #1 ##2\@nil{%
    ##1%
    \ifx\\##2\\%
      \expandafter\@gobble
    \else
      \PdfRender@OP{#1}#2%
      \expandafter\@firstofone
    \fi
    {%
      \PdfRender@temp##2\@nil
    }%
  }%
  \edef\current@color{%
    \@firstofone{\expandafter\PdfRender@temp\current@color} #1 \@nil
  }%
}
%    \end{macrocode}
%    \end{macro}
%    \begin{macro}{\PdfRender@FilterOp}
%    \begin{macrocode}
\def\PdfRender@FilterOp#1#2{%
  \expandafter\PdfRender@@FilterOp\expandafter#1\expandafter#2%
      \current@color\PdfRender@OP{}{}%
}
%    \end{macrocode}
%    \end{macro}
%    \begin{macro}{\PdfRender@@FilterOp}
%    \begin{macrocode}
\def\PdfRender@@FilterOp#1#2#3\PdfRender@OP#4#5{%
  \ifx\\#4#5\\%
  \else
    \ifnum#1=#5 %
      \xdef#2{#3 #4}%
    \fi
    \expandafter\PdfRender@@FilterOp\expandafter#1\expandafter#2%
  \fi
}
%    \end{macrocode}
%    \end{macro}
%
%    \begin{macrocode}
\PdfRender@AtEnd%
%</package>
%    \end{macrocode}
%% \section{Installation}
%
% \subsection{Download}
%
% \paragraph{Package.} This package is available on
% CTAN\footnote{\CTANpkg{pdfrender}}:
% \begin{description}
% \item[\CTAN{macros/latex/contrib/oberdiek/pdfrender.dtx}] The source file.
% \item[\CTAN{macros/latex/contrib/oberdiek/pdfrender.pdf}] Documentation.
% \end{description}
%
%
% \paragraph{Bundle.} All the packages of the bundle `oberdiek'
% are also available in a TDS compliant ZIP archive. There
% the packages are already unpacked and the documentation files
% are generated. The files and directories obey the TDS standard.
% \begin{description}
% \item[\CTANinstall{install/macros/latex/contrib/oberdiek.tds.zip}]
% \end{description}
% \emph{TDS} refers to the standard ``A Directory Structure
% for \TeX\ Files'' (\CTANpkg{tds}). Directories
% with \xfile{texmf} in their name are usually organized this way.
%
% \subsection{Bundle installation}
%
% \paragraph{Unpacking.} Unpack the \xfile{oberdiek.tds.zip} in the
% TDS tree (also known as \xfile{texmf} tree) of your choice.
% Example (linux):
% \begin{quote}
%   |unzip oberdiek.tds.zip -d ~/texmf|
% \end{quote}
%
% \subsection{Package installation}
%
% \paragraph{Unpacking.} The \xfile{.dtx} file is a self-extracting
% \docstrip\ archive. The files are extracted by running the
% \xfile{.dtx} through \plainTeX:
% \begin{quote}
%   \verb|tex pdfrender.dtx|
% \end{quote}
%
% \paragraph{TDS.} Now the different files must be moved into
% the different directories in your installation TDS tree
% (also known as \xfile{texmf} tree):
% \begin{quote}
% \def\t{^^A
% \begin{tabular}{@{}>{\ttfamily}l@{ $\rightarrow$ }>{\ttfamily}l@{}}
%   pdfrender.sty & tex/generic/oberdiek/pdfrender.sty\\
%   pdfrender.pdf & doc/latex/oberdiek/pdfrender.pdf\\
%   pdfrender.dtx & source/latex/oberdiek/pdfrender.dtx\\
% \end{tabular}^^A
% }^^A
% \sbox0{\t}^^A
% \ifdim\wd0>\linewidth
%   \begingroup
%     \advance\linewidth by\leftmargin
%     \advance\linewidth by\rightmargin
%   \edef\x{\endgroup
%     \def\noexpand\lw{\the\linewidth}^^A
%   }\x
%   \def\lwbox{^^A
%     \leavevmode
%     \hbox to \linewidth{^^A
%       \kern-\leftmargin\relax
%       \hss
%       \usebox0
%       \hss
%       \kern-\rightmargin\relax
%     }^^A
%   }^^A
%   \ifdim\wd0>\lw
%     \sbox0{\small\t}^^A
%     \ifdim\wd0>\linewidth
%       \ifdim\wd0>\lw
%         \sbox0{\footnotesize\t}^^A
%         \ifdim\wd0>\linewidth
%           \ifdim\wd0>\lw
%             \sbox0{\scriptsize\t}^^A
%             \ifdim\wd0>\linewidth
%               \ifdim\wd0>\lw
%                 \sbox0{\tiny\t}^^A
%                 \ifdim\wd0>\linewidth
%                   \lwbox
%                 \else
%                   \usebox0
%                 \fi
%               \else
%                 \lwbox
%               \fi
%             \else
%               \usebox0
%             \fi
%           \else
%             \lwbox
%           \fi
%         \else
%           \usebox0
%         \fi
%       \else
%         \lwbox
%       \fi
%     \else
%       \usebox0
%     \fi
%   \else
%     \lwbox
%   \fi
% \else
%   \usebox0
% \fi
% \end{quote}
% If you have a \xfile{docstrip.cfg} that configures and enables \docstrip's
% TDS installing feature, then some files can already be in the right
% place, see the documentation of \docstrip.
%
% \subsection{Refresh file name databases}
%
% If your \TeX~distribution
% (\TeX\,Live, \mikTeX, \dots) relies on file name databases, you must refresh
% these. For example, \TeX\,Live\ users run \verb|texhash| or
% \verb|mktexlsr|.
%
% \subsection{Some details for the interested}
%
% \paragraph{Unpacking with \LaTeX.}
% The \xfile{.dtx} chooses its action depending on the format:
% \begin{description}
% \item[\plainTeX:] Run \docstrip\ and extract the files.
% \item[\LaTeX:] Generate the documentation.
% \end{description}
% If you insist on using \LaTeX\ for \docstrip\ (really,
% \docstrip\ does not need \LaTeX), then inform the autodetect routine
% about your intention:
% \begin{quote}
%   \verb|latex \let\install=y\input{pdfrender.dtx}|
% \end{quote}
% Do not forget to quote the argument according to the demands
% of your shell.
%
% \paragraph{Generating the documentation.}
% You can use both the \xfile{.dtx} or the \xfile{.drv} to generate
% the documentation. The process can be configured by the
% configuration file \xfile{ltxdoc.cfg}. For instance, put this
% line into this file, if you want to have A4 as paper format:
% \begin{quote}
%   \verb|\PassOptionsToClass{a4paper}{article}|
% \end{quote}
% An example follows how to generate the
% documentation with pdf\LaTeX:
% \begin{quote}
%\begin{verbatim}
%pdflatex pdfrender.dtx
%makeindex -s gind.ist pdfrender.idx
%pdflatex pdfrender.dtx
%makeindex -s gind.ist pdfrender.idx
%pdflatex pdfrender.dtx
%\end{verbatim}
% \end{quote}
%
% \section{Acknowledgement}
%
% \begin{description}
% \item[Friedrich Vosberg] asked in the
%   newsgroup \textsf{de.comp.text.tex} for the font
%   outline feature \cite{news-Vosberg}.
% \item[Gaius Pupus] proposed the basic method
%   using \cs{pdfliteral} in this thread \cite{news-Pupus}.
% \item[Rolf Niepraschk] added color support \cite{news-Niepraschk}.
% \end{description}
%
% \begin{thebibliography}{9}
% \raggedright
%
% \bibitem{pdf-spec}
%   Adobe Systems Incorporated. \textit{PDF Reference --
%   Adobe Portable Document format -- Version 1.7}.
%   6th ed. 2006.
%   \textsc{URL:}\,\url{http://www.adobe.com/devnet/acrobat/pdfs/pdf_reference_1-7.pdf}.
%
% \bibitem{news-Vosberg}
%   Friedrich Vosberg, \textit{Text in Buchstabenumrissen},
%   de.comp.text.tex, 2010-01-22.
%   \textsc{URL:}\,\url{https://groups.google.com/group/de.comp.text.tex/msg/f442310ac8b2d506}.
%
% \bibitem{news-Pupus}
%   Gaius Pupus, \textit{Re: Text in Buchstabenumrissen},
%   de.comp.text.tex, 2010-01-23.
%   \textsc{URL:}\,\url{https://groups.google.com/group/de.comp.text.tex/msg/95d890d77ac47eb1}.
%
% \bibitem{news-Niepraschk}
%   Rolf Niepraschk, \textit{Re: Text in Buchstabenumrissen},
%   de.comp.text.tex, 2010-01-24.
%   \textsc{URL:}\,\url{https://groups.google.com/group/de.comp.text.tex/msg/4eb61a5879db54db}.
%
% \end{thebibliography}
%
% \begin{History}
%   \begin{Version}{2010/01/26 v1.0}
%   \item
%     The first version.
%   \end{Version}
%   \begin{Version}{2010/01/27 v1.1}
%   \item
%     Macros \cs{pdfrender} and \cs{textpdfrender} are made robust.
%   \item
%     Color extraction rewritten for the case that \cs{pdfmatch}
%     is not available. This fixes wrong color assigments in
%     case of nesting.
%   \item
%     Color extraction of case \cs{pdfmatch} is fixed
%     for the case that the color string contains several
%     fill or several stroke operations.
%   \end{Version}
%   \begin{Version}{2010/01/28 v1.2}
%   \item
%     Dependency from package \xpackage{color} is removed.
%   \item
%     Compatibility for \plainTeX\ and even ini\TeX\ added.
%   \end{Version}
%   \begin{Version}{2016/05/14 v1.3}
%   \item
%     Use package \xpackage{luatex85} for compatibility with new LuaTeX.
%   \end{Version}
%   \begin{Version}{2016/05/17 v1.4}
%   \item
%     Documentation updates.
%   \item adjust luatex85 reference so that it works in plain TeX.
%   \end{Version}
%   \begin{Version}{2018/11/01 v1.5}
%   \item
%     Remove luatex85 dependency
%   \end{Version}
%   \begin{Version}{2019/12/29 v1.6}
%   \item
%     \xpackage{iftex} package.
%   \end{Version}
% \end{History}
%
% \PrintIndex
%
% \Finale
\endinput
|
% \end{quote}
% Do not forget to quote the argument according to the demands
% of your shell.
%
% \paragraph{Generating the documentation.}
% You can use both the \xfile{.dtx} or the \xfile{.drv} to generate
% the documentation. The process can be configured by the
% configuration file \xfile{ltxdoc.cfg}. For instance, put this
% line into this file, if you want to have A4 as paper format:
% \begin{quote}
%   \verb|\PassOptionsToClass{a4paper}{article}|
% \end{quote}
% An example follows how to generate the
% documentation with pdf\LaTeX:
% \begin{quote}
%\begin{verbatim}
%pdflatex pdfrender.dtx
%makeindex -s gind.ist pdfrender.idx
%pdflatex pdfrender.dtx
%makeindex -s gind.ist pdfrender.idx
%pdflatex pdfrender.dtx
%\end{verbatim}
% \end{quote}
%
% \section{Acknowledgement}
%
% \begin{description}
% \item[Friedrich Vosberg] asked in the
%   newsgroup \textsf{de.comp.text.tex} for the font
%   outline feature \cite{news-Vosberg}.
% \item[Gaius Pupus] proposed the basic method
%   using \cs{pdfliteral} in this thread \cite{news-Pupus}.
% \item[Rolf Niepraschk] added color support \cite{news-Niepraschk}.
% \end{description}
%
% \begin{thebibliography}{9}
% \raggedright
%
% \bibitem{pdf-spec}
%   Adobe Systems Incorporated. \textit{PDF Reference --
%   Adobe Portable Document format -- Version 1.7}.
%   6th ed. 2006.
%   \textsc{URL:}\,\url{http://www.adobe.com/devnet/acrobat/pdfs/pdf_reference_1-7.pdf}.
%
% \bibitem{news-Vosberg}
%   Friedrich Vosberg, \textit{Text in Buchstabenumrissen},
%   de.comp.text.tex, 2010-01-22.
%   \textsc{URL:}\,\url{https://groups.google.com/group/de.comp.text.tex/msg/f442310ac8b2d506}.
%
% \bibitem{news-Pupus}
%   Gaius Pupus, \textit{Re: Text in Buchstabenumrissen},
%   de.comp.text.tex, 2010-01-23.
%   \textsc{URL:}\,\url{https://groups.google.com/group/de.comp.text.tex/msg/95d890d77ac47eb1}.
%
% \bibitem{news-Niepraschk}
%   Rolf Niepraschk, \textit{Re: Text in Buchstabenumrissen},
%   de.comp.text.tex, 2010-01-24.
%   \textsc{URL:}\,\url{https://groups.google.com/group/de.comp.text.tex/msg/4eb61a5879db54db}.
%
% \end{thebibliography}
%
% \begin{History}
%   \begin{Version}{2010/01/26 v1.0}
%   \item
%     The first version.
%   \end{Version}
%   \begin{Version}{2010/01/27 v1.1}
%   \item
%     Macros \cs{pdfrender} and \cs{textpdfrender} are made robust.
%   \item
%     Color extraction rewritten for the case that \cs{pdfmatch}
%     is not available. This fixes wrong color assigments in
%     case of nesting.
%   \item
%     Color extraction of case \cs{pdfmatch} is fixed
%     for the case that the color string contains several
%     fill or several stroke operations.
%   \end{Version}
%   \begin{Version}{2010/01/28 v1.2}
%   \item
%     Dependency from package \xpackage{color} is removed.
%   \item
%     Compatibility for \plainTeX\ and even ini\TeX\ added.
%   \end{Version}
%   \begin{Version}{2016/05/14 v1.3}
%   \item
%     Use package \xpackage{luatex85} for compatibility with new LuaTeX.
%   \end{Version}
%   \begin{Version}{2016/05/17 v1.4}
%   \item
%     Documentation updates.
%   \item adjust luatex85 reference so that it works in plain TeX.
%   \end{Version}
%   \begin{Version}{2018/11/01 v1.5}
%   \item
%     Remove luatex85 dependency
%   \end{Version}
%   \begin{Version}{2019/12/29 v1.6}
%   \item
%     \xpackage{iftex} package.
%   \end{Version}
% \end{History}
%
% \PrintIndex
%
% \Finale
\endinput

%        (quote the arguments according to the demands of your shell)
%
% Documentation:
%    (a) If pdfrender.drv is present:
%           latex pdfrender.drv
%    (b) Without pdfrender.drv:
%           latex pdfrender.dtx; ...
%    The class ltxdoc loads the configuration file ltxdoc.cfg
%    if available. Here you can specify further options, e.g.
%    use A4 as paper format:
%       \PassOptionsToClass{a4paper}{article}
%
%    Programm calls to get the documentation (example):
%       pdflatex pdfrender.dtx
%       makeindex -s gind.ist pdfrender.idx
%       pdflatex pdfrender.dtx
%       makeindex -s gind.ist pdfrender.idx
%       pdflatex pdfrender.dtx
%
% Installation:
%    TDS:tex/generic/oberdiek/pdfrender.sty
%    TDS:doc/latex/oberdiek/pdfrender.pdf
%    TDS:source/latex/oberdiek/pdfrender.dtx
%
%<*ignore>
\begingroup
  \catcode123=1 %
  \catcode125=2 %
  \def\x{LaTeX2e}%
\expandafter\endgroup
\ifcase 0\ifx\install y1\fi\expandafter
         \ifx\csname processbatchFile\endcsname\relax\else1\fi
         \ifx\fmtname\x\else 1\fi\relax
\else\csname fi\endcsname
%</ignore>
%<*install>
\input docstrip.tex
\Msg{************************************************************************}
\Msg{* Installation}
\Msg{* Package: pdfrender 2019/12/29 v1.6 Access to some PDF graphics parameters (HO)}
\Msg{************************************************************************}

\keepsilent
\askforoverwritefalse

\let\MetaPrefix\relax
\preamble

This is a generated file.

Project: pdfrender
Version: 2019/12/29 v1.6

Copyright (C)
   2010 Heiko Oberdiek
   2016-2019 Oberdiek Package Support Group

This work may be distributed and/or modified under the
conditions of the LaTeX Project Public License, either
version 1.3c of this license or (at your option) any later
version. This version of this license is in
   https://www.latex-project.org/lppl/lppl-1-3c.txt
and the latest version of this license is in
   https://www.latex-project.org/lppl.txt
and version 1.3 or later is part of all distributions of
LaTeX version 2005/12/01 or later.

This work has the LPPL maintenance status "maintained".

The Current Maintainers of this work are
Heiko Oberdiek and the Oberdiek Package Support Group
https://github.com/ho-tex/oberdiek/issues


The Base Interpreter refers to any `TeX-Format',
because some files are installed in TDS:tex/generic//.

This work consists of the main source file pdfrender.dtx
and the derived files
   pdfrender.sty, pdfrender.pdf, pdfrender.ins, pdfrender.drv,
   pdfrender-test1.tex, pdfrender-test2.tex,
   pdfrender-test3.tex, pdfrender-test4.tex,
   pdfrender-test5.tex.

\endpreamble
\let\MetaPrefix\DoubleperCent

\generate{%
  \file{pdfrender.ins}{\from{pdfrender.dtx}{install}}%
  \file{pdfrender.drv}{\from{pdfrender.dtx}{driver}}%
  \usedir{tex/generic/oberdiek}%
  \file{pdfrender.sty}{\from{pdfrender.dtx}{package}}%
%  \usedir{doc/latex/oberdiek/test}%
%  \file{pdfrender-test1.tex}{\from{pdfrender.dtx}{test1}}%
%  \file{pdfrender-test2.tex}{\from{pdfrender.dtx}{test2}}%
%  \file{pdfrender-test3.tex}{\from{pdfrender.dtx}{test3}}%
%  \file{pdfrender-test4.tex}{\from{pdfrender.dtx}{test4}}%
%  \file{pdfrender-test5.tex}{\from{pdfrender.dtx}{test5}}%
}

\catcode32=13\relax% active space
\let =\space%
\Msg{************************************************************************}
\Msg{*}
\Msg{* To finish the installation you have to move the following}
\Msg{* file into a directory searched by TeX:}
\Msg{*}
\Msg{*     pdfrender.sty}
\Msg{*}
\Msg{* To produce the documentation run the file `pdfrender.drv'}
\Msg{* through LaTeX.}
\Msg{*}
\Msg{* Happy TeXing!}
\Msg{*}
\Msg{************************************************************************}

\endbatchfile
%</install>
%<*ignore>
\fi
%</ignore>
%<*driver>
\NeedsTeXFormat{LaTeX2e}
\ProvidesFile{pdfrender.drv}%
  [2019/12/29 v1.6 Access to some PDF graphics parameters (HO)]%
\documentclass{ltxdoc}
\usepackage{holtxdoc}[2011/11/22]
\usepackage{longtable}
\renewcommand*{\arraystretch}{1.2}
\begin{document}
  \DocInput{pdfrender.dtx}%
\end{document}
%</driver>
% \fi
%
%
%
% \GetFileInfo{pdfrender.drv}
%
% \title{The \xpackage{pdfrender} package}
% \date{2019/12/29 v1.6}
% \author{Heiko Oberdiek\thanks
% {Please report any issues at \url{https://github.com/ho-tex/oberdiek/issues}}}
%
% \maketitle
%
% \begin{abstract}
% The PDF format has some graphics parameter like
% line width or text rendering mode. This package
% provides an interface for setting these parameters.
% \end{abstract}
%
% \tableofcontents
%
% \section{Documentation}
%
% This package \xpackage{pdfrender} defines an interface for
% PDF specific parameters that affects the rendering of
% graphics or text. The interface and its implementation
% uses the same technique as package \xpackage{color}
% for color settings. Therefore this package is loaded to
% enable \LaTeX's color interface.
%
% At different places \LaTeX\ uses \cs{normalcolor} to avoid that
% header, footer or floats are print in the current color of the
% main text. \cs{setgroup@color} is used to start a save box
% with the color that is set at box saving time.
% Package \xpackage{pdfrender} extends these macros to
% add its own hooks of its parameters. Therefore \LaTeX3 should
% generalize \LaTeXe's color interface.
%
% \subsection{Usage}
%
% In \LaTeX\ the package is loaded as normal package. Options
% are not defined for this package.
% \begin{quote}
% |\usepackage{pdfrender}|
% \end{quote}
%
% This package can also be used in \plainTeX\ and even ini\TeX:
% \begin{quote}
% |input pdfrender.sty|
% \end{quote}
%
% \subsection{Macros}
%
% \begin{declcs}{pdfrender} \M{key value list}
% \end{declcs}
% The first parameter \meta{key value list} contains a
% list of parameter settings. The key entry is the parameter name.
% The macro works like
% \cs{color} (without optional argument) for color setting.
%
% \begin{declcs}{textpdfrender} \M{key value list} \M{text}
% \end{declcs}
% In the same way as \cs{pdfrender} the first argument
% specifies the parameters that should be set. This parameter
% setting affects \meta{text} only. Basically it works
% the same way as \cs{textcolor} (without optional argument).
%
% \subsection{Parameters}
%
% The following table shows an overview for the supported
% parameters and values:
%
% \begin{longtable}{@{}l|l|l@{}}
% \bfseries Parameter & \bfseries Value & \bfseries Alias\\*
% \hline
% \endhead
% \xoption{TextRenderingMode}
%   & \texttt{0} & \texttt{Fill}\\*
%   & \texttt{1} & \texttt{Stroke}\\*
%   & \texttt{2} & \texttt{FillStroke}\\*
%   & \texttt{3} & \texttt{Invisible}\\*
%   & \texttt{4} & \texttt{FillClip}\\*
%   & \texttt{5} & \texttt{StrokeClip}\\*
%   & \texttt{6} & \texttt{FillStrokeClip}\\*
%   & \texttt{7} & \texttt{Clip}\\
% \hline
% \xoption{LineWidth}
%   & \itshape positive number, & \itshape \TeX\ dimen\\*[-.2ex]
%   & \qquad \itshape unit is bp & \\
% \hline
% \xoption{LineCapStyle}
%   & \texttt{0} & \texttt{Butt}\\*
%   & \texttt{1} & \texttt{Round}\\*
%   & \texttt{2} & \texttt{ProjectingSquare}\\
% \hline
% \xoption{LineJoinStyle}
%   & \texttt{0} & \texttt{Miter}\\*
%   & \texttt{1} & \texttt{Round}\\*
%   & \texttt{2} & \texttt{Bevel}\\
% \hline
% \xoption{MiterLimit}
%   & \multicolumn{2}{l@{}}{\itshape positive number} \\
% \hline
% \xoption{Flatness}
%   & \multicolumn{2}{l@{}}{\itshape number between 0 and 100} \\
% \hline
% \xoption{LineDashPattern}
%   & \multicolumn{2}{l@{}}{^^A
%       \itshape numbers in square brackets,^^A
%     } \\*[-.2ex]
%   & \multicolumn{2}{l@{}}{^^A
%       \qquad \itshape followed by number,^^A
%     } \\*
%   & \multicolumn{2}{l@{}}{^^A
%       \qquad \itshape units are bp^^A
%     } \\
% \hline
% \xoption{RenderingIntent}
%   & \multicolumn{2}{l@{}}{\texttt{AbsoluteColorimetric}} \\*
%   & \multicolumn{2}{l@{}}{\texttt{RelativeColorimetric}} \\*
%   & \multicolumn{2}{l@{}}{\texttt{Saturation}} \\*
%   & \multicolumn{2}{l@{}}{\texttt{Perceptual}} \\
% \hline
% \xoption{FillColor}
%   & & \itshape color specification \\
% \hline
% \xoption{StrokeColor}
%   & & \itshape color specification \\
% \end{longtable}
%
% \subsubsection{Details}
%
% The description and specification of these parameters
% are available in the PDF specification \cite{pdf-spec}.
% Therefore they are not repeated here.
%
% \begin{description}
% \item[Value:]
%   The values in the second column lists or describe
%   the values that are specified by the PDF specification.
% \item[Alias:] Instead of magic numbers the package also
%   defines some aliases that can be given as value.
%   Example: |LineCapStyle=Round| has the same effect as
%   |LineCapStyle=1|.
% \item[Number:] The term \emph{number} means an integer
%   or real number. The real number is given as plain
%   decimal number without exponent. The decimal separator is a period.
%   At least one digit must be present.
% \item[\xoption{LineWidth}:] As alias a \TeX\ dimen specification
%   can be given. This includes explicit specifications with number
%   and unit, e.g. |LineWidth=0.5pt|. Also \LaTeX\ length registers
%   may be used. If \eTeX's \cs{dimexpr} is available, then it is
%   automatically added. However package \xpackage{calc} is not
%   supported.
% \item[\xoption{FillColor}, \xoption{StrokeColor}:]
%   Package \xpackage{color} or \xpackage{xcolor} must be loaded
%   before these options can be used (since version 1.2).
%   \LaTeX's color support sets both colors at the same time
%   to the same value. However parameter \xoption{TextRenderingMode}
%   offers the value \texttt{FillStroke} that makes only sense, if
%   the two color types can be set separately.
%   If one of the options \xoption{FillColor} or \xoption{StrokeColor}
%   is specified, then also the color is set. For compatibility
%   with the \LaTeX\ color packages (\xpackage{color} or \xpackage{xcolor}),
%   always both colors must be set. Thus if one of them is not specified,
%   it is taken from the current color.
%
%   Both options \xoption{FillColor} and \xoption{StrokeColor}
%   expect a \LaTeX\ color specification as value.
%   Also the optional color model argument is supported.
%   Example:
%   \begin{quote}
%     |FillColor=yellow|,\\
%     |StrokeColor=[cmyk]{1,.5,0,0}|\\
%   \end{quote}
% \end{description}
%
% \subsection{Color stack}
%
% If the \pdfTeX\ version provides color stacks, then each
% parameter is assigned a page based color stack. The assignment
% of a stack takes place, when its parameter is set the first time.
% This avoids the use of color stacks that are not needed.
%
% \StopEventually{
% }
%
% \section{Implementation}
%    \begin{macrocode}
%<*package>
%    \end{macrocode}
%    Reload check, especially if the package is not used with \LaTeX.
%    \begin{macrocode}
\begingroup\catcode61\catcode48\catcode32=10\relax%
  \catcode13=5 % ^^M
  \endlinechar=13 %
  \catcode35=6 % #
  \catcode39=12 % '
  \catcode44=12 % ,
  \catcode45=12 % -
  \catcode46=12 % .
  \catcode58=12 % :
  \catcode64=11 % @
  \catcode123=1 % {
  \catcode125=2 % }
  \expandafter\let\expandafter\x\csname ver@pdfrender.sty\endcsname
  \ifx\x\relax % plain-TeX, first loading
  \else
    \def\empty{}%
    \ifx\x\empty % LaTeX, first loading,
      % variable is initialized, but \ProvidesPackage not yet seen
    \else
      \expandafter\ifx\csname PackageInfo\endcsname\relax
        \def\x#1#2{%
          \immediate\write-1{Package #1 Info: #2.}%
        }%
      \else
        \def\x#1#2{\PackageInfo{#1}{#2, stopped}}%
      \fi
      \x{pdfrender}{The package is already loaded}%
      \aftergroup\endinput
    \fi
  \fi
\endgroup%
%    \end{macrocode}
%    Package identification:
%    \begin{macrocode}
\begingroup\catcode61\catcode48\catcode32=10\relax%
  \catcode13=5 % ^^M
  \endlinechar=13 %
  \catcode35=6 % #
  \catcode39=12 % '
  \catcode40=12 % (
  \catcode41=12 % )
  \catcode44=12 % ,
  \catcode45=12 % -
  \catcode46=12 % .
  \catcode47=12 % /
  \catcode58=12 % :
  \catcode64=11 % @
  \catcode91=12 % [
  \catcode93=12 % ]
  \catcode123=1 % {
  \catcode125=2 % }
  \expandafter\ifx\csname ProvidesPackage\endcsname\relax
    \def\x#1#2#3[#4]{\endgroup
      \immediate\write-1{Package: #3 #4}%
      \xdef#1{#4}%
    }%
  \else
    \def\x#1#2[#3]{\endgroup
      #2[{#3}]%
      \ifx#1\@undefined
        \xdef#1{#3}%
      \fi
      \ifx#1\relax
        \xdef#1{#3}%
      \fi
    }%
  \fi
\expandafter\x\csname ver@pdfrender.sty\endcsname
\ProvidesPackage{pdfrender}%
  [2019/12/29 v1.6 Access to some PDF graphics parameters (HO)]%
%    \end{macrocode}
%
%    \begin{macrocode}
\begingroup\catcode61\catcode48\catcode32=10\relax%
  \catcode13=5 % ^^M
  \endlinechar=13 %
  \catcode123=1 % {
  \catcode125=2 % }
  \catcode64=11 % @
  \def\x{\endgroup
    \expandafter\edef\csname PdfRender@AtEnd\endcsname{%
      \endlinechar=\the\endlinechar\relax
      \catcode13=\the\catcode13\relax
      \catcode32=\the\catcode32\relax
      \catcode35=\the\catcode35\relax
      \catcode61=\the\catcode61\relax
      \catcode64=\the\catcode64\relax
      \catcode123=\the\catcode123\relax
      \catcode125=\the\catcode125\relax
    }%
  }%
\x\catcode61\catcode48\catcode32=10\relax%
\catcode13=5 % ^^M
\endlinechar=13 %
\catcode35=6 % #
\catcode64=11 % @
\catcode123=1 % {
\catcode125=2 % }
\def\TMP@EnsureCode#1#2{%
  \edef\PdfRender@AtEnd{%
    \PdfRender@AtEnd
    \catcode#1=\the\catcode#1\relax
  }%
  \catcode#1=#2\relax
}
\TMP@EnsureCode{10}{12}% ^^J
\TMP@EnsureCode{36}{3}% $
\TMP@EnsureCode{39}{12}% '
\TMP@EnsureCode{40}{12}% (
\TMP@EnsureCode{41}{12}% )
\TMP@EnsureCode{42}{12}% *
\TMP@EnsureCode{43}{12}% +
\TMP@EnsureCode{44}{12}% ,
\TMP@EnsureCode{45}{12}% -
\TMP@EnsureCode{46}{12}% .
\TMP@EnsureCode{47}{12}% /
\TMP@EnsureCode{58}{12}% :
\TMP@EnsureCode{59}{12}% ;
\TMP@EnsureCode{60}{12}% <
\TMP@EnsureCode{62}{12}% >
\TMP@EnsureCode{63}{12}% ?
\TMP@EnsureCode{91}{12}% [
\TMP@EnsureCode{93}{12}% ]
\TMP@EnsureCode{94}{7}% ^ (superscript)
\TMP@EnsureCode{96}{12}% `
\TMP@EnsureCode{124}{12}% |
%    \end{macrocode}
%
%    \begin{macrocode}
\def\PdfRender@AtEndHook{}
\expandafter\def\expandafter\PdfRender@AtEnd\expandafter{%
  \expandafter\PdfRender@AtEndHook
  \PdfRender@AtEnd
  \endinput
}
%    \end{macrocode}
%
% \subsection{Look for \pdfTeX, its mode and features}
%
%    \begin{macro}{\PdfRender@newif}
%    \begin{macrocode}
\def\PdfRender@newif#1{%
  \expandafter\edef\csname PdfRender@#1true\endcsname{%
    \let
    \expandafter\noexpand\csname ifPdfRender@#1\endcsname
    \noexpand\iftrue
  }%
  \expandafter\edef\csname PdfRender@#1false\endcsname{%
    \let
    \expandafter\noexpand\csname ifPdfRender@#1\endcsname
    \noexpand\iffalse
  }%
  \csname PdfRender@#1false\endcsname
}
%    \end{macrocode}
%    \end{macro}
%
%    \begin{macro}{\ifPdfRender@Stack}
%    \begin{macrocode}
\PdfRender@newif{Stack}
%    \end{macrocode}
%    \end{macro}
%    \begin{macro}{\ifPdfRender@Match}
%    \begin{macrocode}
\PdfRender@newif{Match}
%    \end{macrocode}
%    \end{macro}
%
%    \begin{macro}{\PdfRender@RequirePackage}
%    \begin{macrocode}
\begingroup\expandafter\expandafter\expandafter\endgroup
\expandafter\ifx\csname RequirePackage\endcsname\relax
  \def\PdfRender@RequirePackage#1[#2]{%
    \expandafter\def\expandafter\PdfRender@AtEndHook\expandafter{%
      \PdfRender@AtEndHook
      \ltx@ifpackagelater{#1}{#2}{}{%
        \@PackageWarningNoLine{pdfrender}{%
          You have requested version\MessageBreak
          `#2' of package `#1',\MessageBreak
          but only version\MessageBreak
          `\csname ver@#1.\ltx@pkgextension\endcsname'\MessageBreak
          is available%
        }%
      }%
    }%
    \input #1.sty\relax
  }%
\else
  \let\PdfRender@RequirePackage\RequirePackage
\fi
%    \end{macrocode}
%    \end{macro}
% Luatex compatibility
%    \begin{macrocode}
\ifx\pdfextension\@undefined\else
  \def\pdfcolorstackinit {\pdffeedback colorstackinit}
  \protected\def\pdfcolorstack     {\pdfextension colorstack}
  \protected\def\pdfliteral        {\pdfextension literal}
\fi
%    \end{macrocode}
%
%    \begin{macrocode}
\PdfRender@RequirePackage{iftex}[2019/11/07]
\PdfRender@RequirePackage{infwarerr}[2007/09/09]
\PdfRender@RequirePackage{ltxcmds}[2010/01/28]
%    \end{macrocode}
%
%    \begin{macrocode}
\ifpdf
  \ltx@IfUndefined{pdfcolorstackinit}{%
    \@PackageWarning{pdfrender}{%
      Missing \string\pdfcolorstackinit
    }%
  }{%
    \PdfRender@Stacktrue
  }%
  \ltx@IfUndefined{pdfmatch}{%
    \@PackageInfoNoLine{pdfrender}{%
      \string\pdfmatch\ltx@space not found. %
      Therefore the values\MessageBreak
      of some parameters are not validated%
    }%
  }{%
    \PdfRender@Matchtrue
  }%
\else
  \@PackageWarning{pdfrender}{%
    Missing pdfTeX in PDF mode%
  }%
%    \end{macrocode}
%    \begin{macrocode}
  \ltx@IfUndefined{newcommand}{%
%    \end{macrocode}
%    \begin{macro}{\pdfrender}
%    \begin{macrocode}
    \def\pdfrender#1{}%
%    \end{macrocode}
%    \end{macro}
%    \begin{macro}{\textpdfrender}
%    \begin{macrocode}
    \long\def\textpdfrender#1#2{#2}%
%    \end{macrocode}
%    \end{macro}
%    \begin{macrocode}
  }{%
%    \end{macrocode}
%    \begin{macro}{\pdfrender}
%    \begin{macrocode}
    \newcommand*{\pdfrender}[1]{}%
%    \end{macrocode}
%    \end{macro}
%    \begin{macro}{\textpdfrender}
%    \begin{macrocode}
    \newcommand{\textpdfrender}[2]{#2}%
%    \end{macrocode}
%    \end{macro}
%    \begin{macrocode}
  }%
  \expandafter\PdfRender@AtEnd
\fi%
%    \end{macrocode}
%
% \subsection{Enable color support of \LaTeX}
%
%    \begin{macrocode}
\ltx@ifpackageloaded{color}{}{%
  \def\color@setgroup{\begingroup\set@color}%
  \let\color@begingroup\begingroup
  \def\color@endgroup{\endgraf\endgroup}%
  \def\color@hbox{\hbox\bgroup\color@begingroup}%
  \def\color@vbox{\vbox\bgroup\color@begingroup}%
  \def\color@endbox{\color@endgroup\egroup}%
  \ltx@ifundefined{bgroup}{%
    \let\bgroup={\let\egroup=}%
  }{}%
  \ltx@ifundefined{endgraf}{%
    \let\endgraf=\par
  }{}%
}
%    \end{macrocode}
%
% \subsection{Hook into \cs{normalcolor}}
%
%    The problem is that packages \xpackage{color} and \xpackage{xcolor}
%    each overwrite \cs{normalcolor}. For example, after the package
%    loading order \xpackage{color}, \xpackage{pdfrender} and \xpackage{xcolor}
%    the patched version of \cs{normalcolor} is overwritten by
%    package \xpackage{xcolor}. Also using \cs{AtBeginDocument} for
%    patching is not enough. If package \xpackage{hyperref} is loaded
%    later, it might load package \xpackage{color} using \cs{AtBeginDocument}.
%    \begin{macro}{\PdfRender@NormalColorHook}
%    \begin{macrocode}
\def\PdfRender@NormalColorHook{}
%    \end{macrocode}
%    \end{macro}
%    \begin{macro}{\PdfRender@ColorSetGroupHook}
%    \begin{macrocode}
\def\PdfRender@ColorSetGroupHook{}
%    \end{macrocode}
%    \end{macro}
%    \begin{macro}{\PdfRender@TestBox}
%    \begin{macrocode}
\def\PdfRender@TestBox#1{%
  \setbox0=\color@hbox#1\color@endbox
}
%    \end{macrocode}
%    \end{macro}
%    \begin{macro}{\PdfRender@PatchNormalColor}
%    \begin{macrocode}
\def\PdfRender@PatchNormalColor{%
  \ltx@ifundefined{normalcolor}{%
    \gdef\normalcolor{\PdfRender@NormalColorHook}%
  }{%
    \begingroup
      \def\PdfRender@NormalColorHook{\let\PdfRender@temp=Y}%
      \PdfRender@TestBox{%
        \let\set@color\relax
        \normalcolor
        \ifx\PdfRender@temp Y%
        \else
          \ltx@GlobalAppendToMacro\normalcolor{%
            \PdfRender@NormalColorHook
          }%
        \fi
      }%
    \endgroup
  }%
  \ifx\@nodocument\relax
    \global\let\PdfRender@PatchNormalColor\relax
  \fi
}%
%    \end{macrocode}
%    \end{macro}
%    \begin{macro}{\PdfRender@PatchColorSetGroup}
%    \begin{macrocode}
\def\PdfRender@PatchColorSetGroup{%
  \begingroup
    \def\PdfRender@ColorSetGroupHook{\let\PdfRender@temp=Y}%
    \PdfRender@TestBox{%
      \let\set@color\relax
      \color@setgroup\color@endgroup
      \ifx\PdfRender@temp Y%
      \else
        \ltx@GlobalAppendToMacro\color@setgroup{%
          \PdfRender@ColorSetGroupHook
        }%
      \fi
    }%
  \endgroup
  \ifx\@nodocument\relax
    \global\let\PdfRender@PatchColorSetGroup\relax
  \fi
}%
%    \end{macrocode}
%    \end{macro}
%    \begin{macro}{\PdfRender@PatchColor}
%    \begin{macrocode}
\def\PdfRender@PatchColor{%
  \PdfRender@PatchNormalColor
  \PdfRender@PatchColorSetGroup
}
%    \end{macrocode}
%    \end{macro}
%    \begin{macrocode}
\PdfRender@PatchColor
\ltx@IfUndefined{AtBeginDocument}{}{%
  \AtBeginDocument{\PdfRender@PatchColor}%
}
%    \end{macrocode}
%    \cs{AfterPackage} is provided by package \xpackage{scrlfile}.
%    \begin{macrocode}
\ltx@IfUndefined{AfterPackage}{%
}{%
  \AfterPackage{color}{\PdfRender@PatchColor}%
  \AfterPackage{xcolor}{\PdfRender@PatchColor}%
  \AfterPackage{etoolbox}{%
    \AfterEndPreamble{\PdfRender@PatchColor}%
  }%
}%
%    \end{macrocode}
%    \cs{AfterEndPreamble} is provided by package \xpackage{etoolbox}.
%    \begin{macrocode}
\ltx@IfUndefined{AfterEndPreamble}{%
}{%
  \AfterEndPreamble{\PdfRender@PatchColor}%
}%
%    \end{macrocode}
%
%    \begin{macrocode}
\PdfRender@RequirePackage{kvsetkeys}[2010/01/28]
%    \end{macrocode}
%    \begin{macro}{\PdfRender@texorpdfstring}
%    \begin{macrocode}
\def\PdfRender@texorpdfstring{%
  \ltx@IfUndefined{texorpdfstring}\ltx@firstoftwo\texorpdfstring
}
%    \end{macrocode}
%    \end{macro}
%    \begin{macro}{\pdfrender}
%    \begin{macrocode}
\ltx@IfUndefined{DeclareRobustCommand}%
\ltx@firstoftwo\ltx@secondoftwo
{%
  \def\pdfrender#1%
}{%
  \newcommand{\pdfrender}{}%
  \DeclareRobustCommand*{\pdfrender}[1]%
}%
{%
  \PdfRender@texorpdfstring{%
    \PdfRender@PatchNormalColor
    \global\let\PdfRender@FillColor\ltx@empty
    \global\let\PdfRender@StrokeColor\ltx@empty
    \kvsetkeys{PDFRENDER}{#1}%
    \PdfRender@SetColor
  }{}%
}
%    \end{macrocode}
%    \end{macro}
%    \begin{macro}{\textpdfrender}
%    \begin{macrocode}
\ltx@IfUndefined{DeclareRobustCommand}%
\ltx@firstoftwo\ltx@secondoftwo
{%
  \long\def\textpdfrender#1#2%
}{%
  \newcommand{\textpdfrender}{}%
  \DeclareRobustCommand{\textpdfrender}[2]%
}%
{%
  \PdfRender@texorpdfstring{%
    \begingroup
      \pdfrender{#1}%
      #2%
    \endgroup
  }{#2}%
}
%    \end{macrocode}
%    \end{macro}
%
%    \begin{macro}{\ifPdfRender@Values}
%    \begin{macrocode}
\PdfRender@newif{Values}
%    \end{macrocode}
%    \end{macro}
%    \begin{macro}{\PdfRender@NewClassValues}
%    \begin{macrocode}
\def\PdfRender@NewClassValues#1#2#3#4{%
  \PdfRender@Valuestrue
  \PdfRender@NewClass{#1}{#2}{#3}{#4}{}%
}
%    \end{macrocode}
%    \end{macro}
%    \begin{macro}{\PdfRender@NewClass}
%    \begin{macrocode}
\def\PdfRender@NewClass#1#2#3#4#5{%
  \PdfRender@newif{Active#1}%
  \expandafter\def\csname PdfRender@Default#1\endcsname{#2}%
  \expandafter\let\csname PdfRender@Current#1\expandafter\endcsname
      \csname PdfRender@Default#1\endcsname
  \ifPdfRender@Stack
    \expandafter\edef\csname PdfRender@Init#1\endcsname{%
      \global\chardef
      \expandafter\noexpand\csname PdfRender@Stack#1\endcsname=%
          \noexpand\pdfcolorstackinit page direct{%
            \noexpand#3%
            \expandafter\noexpand\csname PdfRender@Default#1\endcsname
          }\relax
      \noexpand\@PackageInfo{pdfrender}{%
        New color stack `#1' = \noexpand\number
        \expandafter\noexpand\csname PdfRender@Stack#1\endcsname
      }%
      \gdef\expandafter\noexpand\csname PdfRender@Init#1\endcsname{}%
    }%
    \expandafter\edef\csname PdfRender@Set#1\endcsname{%
      \expandafter\noexpand\csname PdfRender@Init#1\endcsname
      \noexpand\pdfcolorstack
      \expandafter\noexpand\csname PdfRender@Stack#1\endcsname
      push{%
        #3{\expandafter\noexpand\csname PdfRender@Current#1\endcsname}%
      }%
      \noexpand\aftergroup
      \expandafter\noexpand\csname PdfRender@Reset#1\endcsname
    }%
    \expandafter\edef\csname PdfRender@Reset#1\endcsname{%
      \expandafter\noexpand\csname PdfRender@Init#1\endcsname
      \noexpand\pdfcolorstack
      \expandafter\noexpand\csname PdfRender@Stack#1\endcsname
      pop\relax
    }%
  \else
    \expandafter\edef\csname PdfRender@Set#1\endcsname{%
      \noexpand\pdfliteral direct{%
        #3{\expandafter\noexpand\csname PdfRender@Current#1\endcsname}%
      }%
      \noexpand\aftergroup
      \expandafter\noexpand\csname PdfRender@Reset#1\endcsname
    }%
    \expandafter\edef\csname PdfRender@Reset#1\endcsname{%
      \noexpand\pdfliteral direct{%
        #3{\expandafter\noexpand\csname PdfRender@Current#1\endcsname}%
      }%
    }%
  \fi
  \expandafter\edef\csname PdfRender@Normal#1\endcsname{%
    \let
    \expandafter\noexpand\csname PdfRender@Current#1\endcsname
    \expandafter\noexpand\csname PdfRender@Default#1\endcsname
    \noexpand\PdfRender@Set{#1}%
  }%
  \expandafter\ltx@GlobalAppendToMacro\expandafter\PdfRender@NormalColorHook
  \expandafter{%
    \csname PdfRender@Normal#1\endcsname
  }%
  \ltx@GlobalAppendToMacro\PdfRender@ColorSetGroupHook{%
    \PdfRender@Set{#1}%
  }%
  \ifPdfRender@Values
    \kv@parse@normalized{#4}{%
      \expandafter\let\csname PdfRender@#1@\kv@key\endcsname\kv@key
      \ifx\kv@value\relax
      \else
        \expandafter\let\csname PdfRender@#1@\kv@value\endcsname\kv@key
      \fi
      \ltx@gobbletwo
    }%
    \PdfRender@define@key{PDFRENDER}{#1}{%
      \global\csname PdfRender@Active#1true\endcsname
      \def\PdfRender@Current{##1}%
      \PdfRender@SetValidateValues{#1}%
    }%
    \PdfRender@Valuesfalse
  \else
    \PdfRender@define@key{PDFRENDER}{#1}{%
      \global\csname PdfRender@Active#1true\endcsname
      \expandafter\def\csname PdfRender@Current#1\endcsname{##1}%
      \ltx@IfUndefined{PdfRender@PostProcess#1}{%
      }{%
        \csname PdfRender@PostProcess#1\endcsname
      }%
      \PdfRender@SetValidate{#1}{#4}{#5}%
    }%
  \fi
}%
%    \end{macrocode}
%    \end{macro}
%    \begin{macro}{\PdfRender@define@key}
%    \begin{macrocode}
\ltx@IfUndefined{define@key}{%
  \def\PdfRender@define@key#1#2{%
    \expandafter\def\csname KV@#1@#2\endcsname##1%
  }%
}{%
  \let\PdfRender@define@key\define@key
}
%    \end{macrocode}
%    \end{macro}
%    \begin{macro}{\PdfRender@Set}
%    \begin{macrocode}
\def\PdfRender@Set#1{%
  \csname ifPdfRender@Active#1\endcsname
    \csname PdfRender@Set#1\expandafter\endcsname
  \fi
}
%    \end{macrocode}
%    \end{macro}
%    \begin{macro}{\PdfRender@Reset}
%    \begin{macrocode}
\def\PdfRender@Reset#1{%
  \csname ifPdfRender@Active#1\endcsname
    \csname PdfRender@Reset#1\expandafter\endcsname
  \fi
}
%    \end{macrocode}
%    \end{macro}
%    \begin{macro}{\PdfRender@ErrorInvalidValue}
%    \begin{macrocode}
\def\PdfRender@ErrorInvalidValue#1{%
  \PackageError{pdfrender}{%
    Ignoring parameter setting for `#1'\MessageBreak
    because of invalid value %
    `\csname PdfRender@Current#1\endcsname'%
  }\@ehc
  \expandafter\let\csname PdfRender@Current#1\endcsname\ltx@empty
}%
%    \end{macrocode}
%    \end{macro}
%    \begin{macro}{\PdfRender@SetValidate}
%    \begin{macrocode}
\ifPdfRender@Match
  \def\PdfRender@SetValidate#1#2#3{%
    \ifnum\pdfmatch{^(#2)$}{\csname PdfRender@Current#1\endcsname}=1 %
      \csname PdfRender@Set#1\expandafter\endcsname
    \else
      \PdfRender@ErrorInvalidValue{#1}%
    \fi
  }%
\else
  \def\PdfRender@SetValidate#1#2#3{%
    \expandafter\let\expandafter\PdfRender@Current
    \csname PdfRender@Current#1\endcsname
    #3%
    \ifx\PdfRender@Current\@empty
      \PdfRender@ErrorInvalidValue{#1}%
    \else
      \csname PdfRender@Set#1\expandafter\endcsname
    \fi
  }%
\fi
%    \end{macrocode}
%    \end{macro}
%    \begin{macro}{\PdfRender@SetValidateValues}
%    \begin{macrocode}
\def\PdfRender@SetValidateValues#1{%
  \ltx@IfUndefined{PdfRender@#1@\PdfRender@Current}{%
    \expandafter\let\csname PdfRender@Current#1\endcsname
                    \PdfRender@Current
    \PdfRender@ErrorInvalidValue{#1}%
  }{%
    \expandafter\edef\csname PdfRender@Current#1\endcsname{%
      \csname PdfRender@#1@\PdfRender@Current\endcsname
    }%
    \csname PdfRender@Set#1\endcsname
  }%
}
%    \end{macrocode}
%    \end{macro}
%    \begin{macro}{\PdfRender@OpValue}
%    \begin{macrocode}
\def\PdfRender@OpValue#1#2{#2\ltx@space#1}%
%    \end{macrocode}
%    \end{macro}
%    \begin{macro}{\PdfRender@OpName}
%    \begin{macrocode}
\def\PdfRender@OpName#1#2{/#2\ltx@space#1}%
%    \end{macrocode}
%    \end{macro}
%
% \subsection{Declare and setup parameters}
%
%    \begin{macrocode}
\PdfRender@NewClassValues{TextRenderingMode}%
                      {0}%
                      {\PdfRender@OpValue{Tr}}{%
  0=Fill,%
  1=Stroke,%
  2=FillStroke,%
  3=Invisible,%
  4=FillClip,%
  5=StrokeClip,%
  6=FillStrokeClip,%
  7=Clip,%
}%
\PdfRender@NewClass{LineWidth}{1}{\PdfRender@OpValue{w}}{%
  [0-9]+\string\.?[0-9]*|\string\.[0-9]+%
}{}%
\ltx@IfUndefined{dimexpr}{%
  \def\PdfRender@dimexpr{}%
}{%
  \let\PdfRender@dimexpr\dimexpr
}
\def\PdfRender@PostProcessLineWidth{%
  \begingroup
  \afterassignment\PdfRender@@PostProcessLineWidth
  \dimen0=\PdfRender@dimexpr\PdfRender@CurrentLineWidth bp %
  \PdfRender@let\PdfRender@relax\PdfRender@relax
}
\let\PdfRender@let\let
\let\PdfRender@relax\relax
\def\PdfRender@@PostProcessLineWidth#1\PdfRender@let{%
  \ifx\\#1\\%
    \endgroup
  \else
    \dimen0=.996264\dimen0 % 72/72.27
    \edef\x{\endgroup
      \def\noexpand\PdfRender@CurrentLineWidth{%
        \strip@pt\dimen0%
      }%
    }%
    \expandafter\x
  \fi
}
\PdfRender@NewClassValues{LineCapStyle}{0}{\PdfRender@OpValue{J}}{%
  0=Butt,%
  1=Round,%
  2=ProjectingSquare,%
}%
\PdfRender@NewClassValues{LineJoinStyle}{0}{\PdfRender@OpValue{j}}{%
  0=Miter,%
  1=Round,%
  2=Bevel,%
}%
\PdfRender@NewClass{MiterLimit}{10}{\PdfRender@OpValue{M}}{%
  [0-9]*[1-9][0-9]*\string\.?[0-9]*|%
  [0-9]*\string\.?[0-9]*[1-9][0-9]*%
}{}%
\PdfRender@NewClass{Flatness}{0}{\PdfRender@OpValue{i}}{%
  100(\string\.0*)?|[0-9][0-9](\string\.[0-9]*)?|\string\.[0-9]+%
}{}%
\PdfRender@NewClass{LineDashPattern}{[]0}{\PdfRender@OpValue{d}}{%
  \string\[%
  ( ?([0-9]+\string\.?[0-9]*|\string\.[0-9]+) ?)*%
  \string\] ?%
  ([0-9]+\string\.?[0-9]*|\string\.[0-9]+)%
}{}%
\PdfRender@NewClassValues{RenderingIntent}%
                      {RelativeColorimetric}%
                      {\PdfRender@OpName{ri}}{%
  AbsoluteColorimetric,%
  RelativeColorimetric,%
  Saturation,%
  Perceptual,%
}%
%    \end{macrocode}
%
% \subsection{Fill and stroke color support}
%
%    \begin{macrocode}
\PdfRender@define@key{PDFRENDER}{FillColor}{%
  \begingroup
    \def\PdfRender@Color{#1}%
    \ifx\PdfRender@Color\ltx@empty
      \global\let\PdfRender@FillColor\ltx@empty
    \else
      \PdfRender@ColorAvailable{%
        \PdfRender@TestBox{%
          \expandafter\PdfRender@TryColor\PdfRender@Color\ltx@empty
          \PdfRender@GetFillColor
          \ifx\PdfRender@FillColor\ltx@empty
            \@PackageWarning{pdfrender}{%
              Cannot extract fill color\MessageBreak
              from value `#1'%
            }%
          \fi
        }%
      }%
    \fi
  \endgroup
}
\PdfRender@define@key{PDFRENDER}{StrokeColor}{%
  \begingroup
    \def\PdfRender@Color{#1}%
    \ifx\PdfRender@Color\ltx@empty
      \global\let\PdfRender@StrokeColor\ltx@empty
    \else
      \PdfRender@ColorAvailable{%
        \PdfRender@TestBox{%
          \expandafter\PdfRender@TryColor\PdfRender@Color\ltx@empty
          \PdfRender@GetStrokeColor
          \ifx\PdfRender@StrokeColor\ltx@empty
            \@PackageWarning{pdfrender}{%
              Cannot extract stroke color\MessageBreak
              from value `#1'%
            }%
          \fi
        }%
      }%
    \fi
  \endgroup
}
%    \end{macrocode}
%    \begin{macro}{\PdfRender@ColorAvailable}
%    \begin{macrocode}
\def\PdfRender@ColorAvailable{%
  \@ifundefined{set@color}{%
    \@PackageError{pdfrender}{%
      Ignoring color options, because neither\MessageBreak
      package `color' nor package `xcolor' is loaded%
    }\@ehc
    \global\let\PdfRender@ColorAvailable\ltx@gobble
  }{%
    \global\let\PdfRender@ColorAvailable\ltx@firstofone
  }%
  \PdfRender@ColorAvailable
}
%    \end{macrocode}
%    \end{macro}
%    \begin{macro}{\PdfRender@TryColor}
%    \begin{macrocode}
\def\PdfRender@TryColor{%
  \@ifnextchar[\color\PdfRender@@TryColor
}
%    \end{macrocode}
%    \end{macro}
%    \begin{macro}{\PdfRender@@TryColor}
%    \begin{macrocode}
\def\PdfRender@@TryColor#1\ltx@empty{%
  \expandafter\color\expandafter{\PdfRender@Color}%
}
%    \end{macrocode}
%    \end{macro}
%    \begin{macro}{\PdfRender@SetColor}
%    \begin{macrocode}
\def\PdfRender@SetColor{%
  \chardef\PdfRender@NeedsCurrentColor=0 %
  \ifx\PdfRender@FillColor\ltx@empty
    \ifx\PdfRender@StrokeColor\ltx@empty
    \else
      \edef\PdfRender@CurrentColor{%
        \noexpand\PdfRender@FillColor\ltx@space\PdfRender@StrokeColor
      }%
      \chardef\PdfRender@NeedsCurrentColor=1 %
    \fi
  \else
    \ifx\PdfRender@StrokeColor\ltx@empty
      \edef\PdfRender@CurrentColor{%
        \PdfRender@FillColor\ltx@space\noexpand\PdfRender@StrokeColor
      }%
      \chardef\PdfRender@NeedsCurrentColor=2 %
    \else
      \edef\current@color{%
        \PdfRender@FillColor\ltx@space\PdfRender@StrokeColor
      }%
      \set@color
    \fi
  \fi
  \ifnum\PdfRender@NeedsCurrentColor=1 %
    \PdfRender@GetFillColor
    \ifx\PdfRender@FillColor\ltx@empty
      \@PackageWarning{pdfrender}{%
        Cannot extract current fill color%
      }%
    \else
      \edef\current@color{\PdfRender@CurrentColor}%
      \set@color
    \fi
  \else
    \ifnum\PdfRender@NeedsCurrentColor=2 %
      \PdfRender@GetStrokeColor
      \ifx\PdfRender@StrokeColor\ltx@empty
        \@PackageWarning{pdfrender}{%
          Cannot extract current stroke color%
        }%
      \else
        \edef\current@color{\PdfRender@CurrentColor}%
        \set@color
      \fi
    \fi
  \fi
}
%    \end{macrocode}
%    \end{macro}
%    \begin{macro}{\PdfRender@PatternFillColor}
%    \begin{macrocode}
\edef\PdfRender@PatternFillColor{ % space
  (%
    [0-9\string\.]+ g|%
    [0-9\string\.]+ [0-9\string\.]+ [0-9\string\.]+ rg|%
    [0-9\string\.]+ [0-9\string\.]+ %
    [0-9\string\.]+ [0-9\string\.]+ k%
  ) % space
  (.*)$%
}
%    \end{macrocode}
%    \end{macro}
%    \begin{macro}{\PdfRender@PatternStrokeColor}
%    \begin{macrocode}
\edef\PdfRender@PatternStrokeColor{ % space
  (%
    [0-9\string\.]+ G|%
    [0-9\string\.]+ [0-9\string\.]+ [0-9\string\.]+ RG|%
    [0-9\string\.]+ [0-9\string\.]+ %
    [0-9\string\.]+ [0-9\string\.]+ K%
  ) % space
  (.*)$%
}
%    \end{macrocode}
%    \end{macro}
%    \begin{macro}{\PdfRender@MatchPattern}
%    \begin{macrocode}
\def\PdfRender@MatchPattern#1{%
  \ifnum\pdfmatch{\PdfRender@Pattern}{\PdfRender@String}=1 %
    \xdef#1{%
      \expandafter\strip@prefix\pdflastmatch 1%
    }%
    \edef\PdfRender@String{%
      \expandafter\strip@prefix\pdflastmatch 2%
    }%
    \ifx\PdfRender@String\ltx@empty
    \else
      \expandafter\expandafter\expandafter\PdfRender@MatchPattern
      \expandafter\expandafter\expandafter#1%
    \fi
  \fi
}
%    \end{macrocode}
%    \end{macro}
%    \begin{macro}{\PdfRender@GetFillColor}
%    \begin{macrocode}
\def\PdfRender@GetFillColor{%
  \global\let\PdfRender@FillColor\ltx@empty
  \begingroup
    \ifPdfRender@Match
      \let\PdfRender@Pattern\PdfRender@PatternFillColor
      \edef\PdfRender@String{\ltx@space\current@color\ltx@space}%
      \PdfRender@MatchPattern\PdfRender@FillColor
    \else
      \edef\current@color{\current@color\ltx@space}%
      \let\PdfRender@OP\relax
      \PdfRender@FindOp{g}0%
      \PdfRender@FindOp{G}1%
      \PdfRender@FindOp{rg}0%
      \PdfRender@FindOp{RG}1%
      \PdfRender@FindOp{k}0%
      \PdfRender@FindOp{K}1%
      \PdfRender@FilterOp 0\PdfRender@FillColor
    \fi
  \endgroup
}
%    \end{macrocode}
%    \end{macro}
%    \begin{macro}{\PdfRender@GetStrokeColor}
%    \begin{macrocode}
\def\PdfRender@GetStrokeColor{%
  \global\let\PdfRender@StrokeColor\ltx@empty
  \begingroup
    \ifPdfRender@Match
      \let\PdfRender@Pattern\PdfRender@PatternStrokeColor
      \edef\PdfRender@String{\ltx@space\current@color\ltx@space}%
      \PdfRender@MatchPattern\PdfRender@StrokeColor
    \else
      \edef\current@color{\current@color\ltx@space}%
      \let\PdfRender@OP\relax
      \PdfRender@FindOp{g}0%
      \PdfRender@FindOp{G}1%
      \PdfRender@FindOp{rg}0%
      \PdfRender@FindOp{RG}1%
      \PdfRender@FindOp{k}0%
      \PdfRender@FindOp{K}1%
      \PdfRender@FilterOp 1\PdfRender@StrokeColor
    \fi
  \endgroup
}
%    \end{macrocode}
%    \end{macro}
%    \begin{macrocode}
\ifPdfRender@Match
  \expandafter\PdfRender@AtEnd
\fi%
%    \end{macrocode}
%    \begin{macro}{\PdfRender@FindOp}
%    \begin{macrocode}
\def\PdfRender@FindOp#1#2{%
  \def\PdfRender@temp##1 #1 ##2\@nil{%
    ##1%
    \ifx\\##2\\%
      \expandafter\@gobble
    \else
      \PdfRender@OP{#1}#2%
      \expandafter\@firstofone
    \fi
    {%
      \PdfRender@temp##2\@nil
    }%
  }%
  \edef\current@color{%
    \@firstofone{\expandafter\PdfRender@temp\current@color} #1 \@nil
  }%
}
%    \end{macrocode}
%    \end{macro}
%    \begin{macro}{\PdfRender@FilterOp}
%    \begin{macrocode}
\def\PdfRender@FilterOp#1#2{%
  \expandafter\PdfRender@@FilterOp\expandafter#1\expandafter#2%
      \current@color\PdfRender@OP{}{}%
}
%    \end{macrocode}
%    \end{macro}
%    \begin{macro}{\PdfRender@@FilterOp}
%    \begin{macrocode}
\def\PdfRender@@FilterOp#1#2#3\PdfRender@OP#4#5{%
  \ifx\\#4#5\\%
  \else
    \ifnum#1=#5 %
      \xdef#2{#3 #4}%
    \fi
    \expandafter\PdfRender@@FilterOp\expandafter#1\expandafter#2%
  \fi
}
%    \end{macrocode}
%    \end{macro}
%
%    \begin{macrocode}
\PdfRender@AtEnd%
%</package>
%    \end{macrocode}
%% \section{Installation}
%
% \subsection{Download}
%
% \paragraph{Package.} This package is available on
% CTAN\footnote{\CTANpkg{pdfrender}}:
% \begin{description}
% \item[\CTAN{macros/latex/contrib/oberdiek/pdfrender.dtx}] The source file.
% \item[\CTAN{macros/latex/contrib/oberdiek/pdfrender.pdf}] Documentation.
% \end{description}
%
%
% \paragraph{Bundle.} All the packages of the bundle `oberdiek'
% are also available in a TDS compliant ZIP archive. There
% the packages are already unpacked and the documentation files
% are generated. The files and directories obey the TDS standard.
% \begin{description}
% \item[\CTANinstall{install/macros/latex/contrib/oberdiek.tds.zip}]
% \end{description}
% \emph{TDS} refers to the standard ``A Directory Structure
% for \TeX\ Files'' (\CTANpkg{tds}). Directories
% with \xfile{texmf} in their name are usually organized this way.
%
% \subsection{Bundle installation}
%
% \paragraph{Unpacking.} Unpack the \xfile{oberdiek.tds.zip} in the
% TDS tree (also known as \xfile{texmf} tree) of your choice.
% Example (linux):
% \begin{quote}
%   |unzip oberdiek.tds.zip -d ~/texmf|
% \end{quote}
%
% \subsection{Package installation}
%
% \paragraph{Unpacking.} The \xfile{.dtx} file is a self-extracting
% \docstrip\ archive. The files are extracted by running the
% \xfile{.dtx} through \plainTeX:
% \begin{quote}
%   \verb|tex pdfrender.dtx|
% \end{quote}
%
% \paragraph{TDS.} Now the different files must be moved into
% the different directories in your installation TDS tree
% (also known as \xfile{texmf} tree):
% \begin{quote}
% \def\t{^^A
% \begin{tabular}{@{}>{\ttfamily}l@{ $\rightarrow$ }>{\ttfamily}l@{}}
%   pdfrender.sty & tex/generic/oberdiek/pdfrender.sty\\
%   pdfrender.pdf & doc/latex/oberdiek/pdfrender.pdf\\
%   pdfrender.dtx & source/latex/oberdiek/pdfrender.dtx\\
% \end{tabular}^^A
% }^^A
% \sbox0{\t}^^A
% \ifdim\wd0>\linewidth
%   \begingroup
%     \advance\linewidth by\leftmargin
%     \advance\linewidth by\rightmargin
%   \edef\x{\endgroup
%     \def\noexpand\lw{\the\linewidth}^^A
%   }\x
%   \def\lwbox{^^A
%     \leavevmode
%     \hbox to \linewidth{^^A
%       \kern-\leftmargin\relax
%       \hss
%       \usebox0
%       \hss
%       \kern-\rightmargin\relax
%     }^^A
%   }^^A
%   \ifdim\wd0>\lw
%     \sbox0{\small\t}^^A
%     \ifdim\wd0>\linewidth
%       \ifdim\wd0>\lw
%         \sbox0{\footnotesize\t}^^A
%         \ifdim\wd0>\linewidth
%           \ifdim\wd0>\lw
%             \sbox0{\scriptsize\t}^^A
%             \ifdim\wd0>\linewidth
%               \ifdim\wd0>\lw
%                 \sbox0{\tiny\t}^^A
%                 \ifdim\wd0>\linewidth
%                   \lwbox
%                 \else
%                   \usebox0
%                 \fi
%               \else
%                 \lwbox
%               \fi
%             \else
%               \usebox0
%             \fi
%           \else
%             \lwbox
%           \fi
%         \else
%           \usebox0
%         \fi
%       \else
%         \lwbox
%       \fi
%     \else
%       \usebox0
%     \fi
%   \else
%     \lwbox
%   \fi
% \else
%   \usebox0
% \fi
% \end{quote}
% If you have a \xfile{docstrip.cfg} that configures and enables \docstrip's
% TDS installing feature, then some files can already be in the right
% place, see the documentation of \docstrip.
%
% \subsection{Refresh file name databases}
%
% If your \TeX~distribution
% (\TeX\,Live, \mikTeX, \dots) relies on file name databases, you must refresh
% these. For example, \TeX\,Live\ users run \verb|texhash| or
% \verb|mktexlsr|.
%
% \subsection{Some details for the interested}
%
% \paragraph{Unpacking with \LaTeX.}
% The \xfile{.dtx} chooses its action depending on the format:
% \begin{description}
% \item[\plainTeX:] Run \docstrip\ and extract the files.
% \item[\LaTeX:] Generate the documentation.
% \end{description}
% If you insist on using \LaTeX\ for \docstrip\ (really,
% \docstrip\ does not need \LaTeX), then inform the autodetect routine
% about your intention:
% \begin{quote}
%   \verb|latex \let\install=y% \iffalse meta-comment
%
% File: pdfrender.dtx
% Version: 2019/12/29 v1.6
% Info: Access to some PDF graphics parameters
%
% Copyright (C)
%    2010 Heiko Oberdiek
%    2016-2019 Oberdiek Package Support Group
%    https://github.com/ho-tex/oberdiek/issues
%
% This work may be distributed and/or modified under the
% conditions of the LaTeX Project Public License, either
% version 1.3c of this license or (at your option) any later
% version. This version of this license is in
%    https://www.latex-project.org/lppl/lppl-1-3c.txt
% and the latest version of this license is in
%    https://www.latex-project.org/lppl.txt
% and version 1.3 or later is part of all distributions of
% LaTeX version 2005/12/01 or later.
%
% This work has the LPPL maintenance status "maintained".
%
% The Current Maintainers of this work are
% Heiko Oberdiek and the Oberdiek Package Support Group
% https://github.com/ho-tex/oberdiek/issues
%
% The Base Interpreter refers to any `TeX-Format',
% because some files are installed in TDS:tex/generic//.
%
% This work consists of the main source file pdfrender.dtx
% and the derived files
%    pdfrender.sty, pdfrender.pdf, pdfrender.ins, pdfrender.drv,
%    pdfrender-test1.tex, pdfrender-test2.tex,
%    pdfrender-test3.tex, pdfrender-test4.tex,
%    pdfrender-test5.tex.
%
% Distribution:
%    CTAN:macros/latex/contrib/oberdiek/pdfrender.dtx
%    CTAN:macros/latex/contrib/oberdiek/pdfrender.pdf
%
% Unpacking:
%    (a) If pdfrender.ins is present:
%           tex pdfrender.ins
%    (b) Without pdfrender.ins:
%           tex pdfrender.dtx
%    (c) If you insist on using LaTeX
%           latex \let\install=y% \iffalse meta-comment
%
% File: pdfrender.dtx
% Version: 2019/12/29 v1.6
% Info: Access to some PDF graphics parameters
%
% Copyright (C)
%    2010 Heiko Oberdiek
%    2016-2019 Oberdiek Package Support Group
%    https://github.com/ho-tex/oberdiek/issues
%
% This work may be distributed and/or modified under the
% conditions of the LaTeX Project Public License, either
% version 1.3c of this license or (at your option) any later
% version. This version of this license is in
%    https://www.latex-project.org/lppl/lppl-1-3c.txt
% and the latest version of this license is in
%    https://www.latex-project.org/lppl.txt
% and version 1.3 or later is part of all distributions of
% LaTeX version 2005/12/01 or later.
%
% This work has the LPPL maintenance status "maintained".
%
% The Current Maintainers of this work are
% Heiko Oberdiek and the Oberdiek Package Support Group
% https://github.com/ho-tex/oberdiek/issues
%
% The Base Interpreter refers to any `TeX-Format',
% because some files are installed in TDS:tex/generic//.
%
% This work consists of the main source file pdfrender.dtx
% and the derived files
%    pdfrender.sty, pdfrender.pdf, pdfrender.ins, pdfrender.drv,
%    pdfrender-test1.tex, pdfrender-test2.tex,
%    pdfrender-test3.tex, pdfrender-test4.tex,
%    pdfrender-test5.tex.
%
% Distribution:
%    CTAN:macros/latex/contrib/oberdiek/pdfrender.dtx
%    CTAN:macros/latex/contrib/oberdiek/pdfrender.pdf
%
% Unpacking:
%    (a) If pdfrender.ins is present:
%           tex pdfrender.ins
%    (b) Without pdfrender.ins:
%           tex pdfrender.dtx
%    (c) If you insist on using LaTeX
%           latex \let\install=y\input{pdfrender.dtx}
%        (quote the arguments according to the demands of your shell)
%
% Documentation:
%    (a) If pdfrender.drv is present:
%           latex pdfrender.drv
%    (b) Without pdfrender.drv:
%           latex pdfrender.dtx; ...
%    The class ltxdoc loads the configuration file ltxdoc.cfg
%    if available. Here you can specify further options, e.g.
%    use A4 as paper format:
%       \PassOptionsToClass{a4paper}{article}
%
%    Programm calls to get the documentation (example):
%       pdflatex pdfrender.dtx
%       makeindex -s gind.ist pdfrender.idx
%       pdflatex pdfrender.dtx
%       makeindex -s gind.ist pdfrender.idx
%       pdflatex pdfrender.dtx
%
% Installation:
%    TDS:tex/generic/oberdiek/pdfrender.sty
%    TDS:doc/latex/oberdiek/pdfrender.pdf
%    TDS:source/latex/oberdiek/pdfrender.dtx
%
%<*ignore>
\begingroup
  \catcode123=1 %
  \catcode125=2 %
  \def\x{LaTeX2e}%
\expandafter\endgroup
\ifcase 0\ifx\install y1\fi\expandafter
         \ifx\csname processbatchFile\endcsname\relax\else1\fi
         \ifx\fmtname\x\else 1\fi\relax
\else\csname fi\endcsname
%</ignore>
%<*install>
\input docstrip.tex
\Msg{************************************************************************}
\Msg{* Installation}
\Msg{* Package: pdfrender 2019/12/29 v1.6 Access to some PDF graphics parameters (HO)}
\Msg{************************************************************************}

\keepsilent
\askforoverwritefalse

\let\MetaPrefix\relax
\preamble

This is a generated file.

Project: pdfrender
Version: 2019/12/29 v1.6

Copyright (C)
   2010 Heiko Oberdiek
   2016-2019 Oberdiek Package Support Group

This work may be distributed and/or modified under the
conditions of the LaTeX Project Public License, either
version 1.3c of this license or (at your option) any later
version. This version of this license is in
   https://www.latex-project.org/lppl/lppl-1-3c.txt
and the latest version of this license is in
   https://www.latex-project.org/lppl.txt
and version 1.3 or later is part of all distributions of
LaTeX version 2005/12/01 or later.

This work has the LPPL maintenance status "maintained".

The Current Maintainers of this work are
Heiko Oberdiek and the Oberdiek Package Support Group
https://github.com/ho-tex/oberdiek/issues


The Base Interpreter refers to any `TeX-Format',
because some files are installed in TDS:tex/generic//.

This work consists of the main source file pdfrender.dtx
and the derived files
   pdfrender.sty, pdfrender.pdf, pdfrender.ins, pdfrender.drv,
   pdfrender-test1.tex, pdfrender-test2.tex,
   pdfrender-test3.tex, pdfrender-test4.tex,
   pdfrender-test5.tex.

\endpreamble
\let\MetaPrefix\DoubleperCent

\generate{%
  \file{pdfrender.ins}{\from{pdfrender.dtx}{install}}%
  \file{pdfrender.drv}{\from{pdfrender.dtx}{driver}}%
  \usedir{tex/generic/oberdiek}%
  \file{pdfrender.sty}{\from{pdfrender.dtx}{package}}%
%  \usedir{doc/latex/oberdiek/test}%
%  \file{pdfrender-test1.tex}{\from{pdfrender.dtx}{test1}}%
%  \file{pdfrender-test2.tex}{\from{pdfrender.dtx}{test2}}%
%  \file{pdfrender-test3.tex}{\from{pdfrender.dtx}{test3}}%
%  \file{pdfrender-test4.tex}{\from{pdfrender.dtx}{test4}}%
%  \file{pdfrender-test5.tex}{\from{pdfrender.dtx}{test5}}%
}

\catcode32=13\relax% active space
\let =\space%
\Msg{************************************************************************}
\Msg{*}
\Msg{* To finish the installation you have to move the following}
\Msg{* file into a directory searched by TeX:}
\Msg{*}
\Msg{*     pdfrender.sty}
\Msg{*}
\Msg{* To produce the documentation run the file `pdfrender.drv'}
\Msg{* through LaTeX.}
\Msg{*}
\Msg{* Happy TeXing!}
\Msg{*}
\Msg{************************************************************************}

\endbatchfile
%</install>
%<*ignore>
\fi
%</ignore>
%<*driver>
\NeedsTeXFormat{LaTeX2e}
\ProvidesFile{pdfrender.drv}%
  [2019/12/29 v1.6 Access to some PDF graphics parameters (HO)]%
\documentclass{ltxdoc}
\usepackage{holtxdoc}[2011/11/22]
\usepackage{longtable}
\renewcommand*{\arraystretch}{1.2}
\begin{document}
  \DocInput{pdfrender.dtx}%
\end{document}
%</driver>
% \fi
%
%
%
% \GetFileInfo{pdfrender.drv}
%
% \title{The \xpackage{pdfrender} package}
% \date{2019/12/29 v1.6}
% \author{Heiko Oberdiek\thanks
% {Please report any issues at \url{https://github.com/ho-tex/oberdiek/issues}}}
%
% \maketitle
%
% \begin{abstract}
% The PDF format has some graphics parameter like
% line width or text rendering mode. This package
% provides an interface for setting these parameters.
% \end{abstract}
%
% \tableofcontents
%
% \section{Documentation}
%
% This package \xpackage{pdfrender} defines an interface for
% PDF specific parameters that affects the rendering of
% graphics or text. The interface and its implementation
% uses the same technique as package \xpackage{color}
% for color settings. Therefore this package is loaded to
% enable \LaTeX's color interface.
%
% At different places \LaTeX\ uses \cs{normalcolor} to avoid that
% header, footer or floats are print in the current color of the
% main text. \cs{setgroup@color} is used to start a save box
% with the color that is set at box saving time.
% Package \xpackage{pdfrender} extends these macros to
% add its own hooks of its parameters. Therefore \LaTeX3 should
% generalize \LaTeXe's color interface.
%
% \subsection{Usage}
%
% In \LaTeX\ the package is loaded as normal package. Options
% are not defined for this package.
% \begin{quote}
% |\usepackage{pdfrender}|
% \end{quote}
%
% This package can also be used in \plainTeX\ and even ini\TeX:
% \begin{quote}
% |input pdfrender.sty|
% \end{quote}
%
% \subsection{Macros}
%
% \begin{declcs}{pdfrender} \M{key value list}
% \end{declcs}
% The first parameter \meta{key value list} contains a
% list of parameter settings. The key entry is the parameter name.
% The macro works like
% \cs{color} (without optional argument) for color setting.
%
% \begin{declcs}{textpdfrender} \M{key value list} \M{text}
% \end{declcs}
% In the same way as \cs{pdfrender} the first argument
% specifies the parameters that should be set. This parameter
% setting affects \meta{text} only. Basically it works
% the same way as \cs{textcolor} (without optional argument).
%
% \subsection{Parameters}
%
% The following table shows an overview for the supported
% parameters and values:
%
% \begin{longtable}{@{}l|l|l@{}}
% \bfseries Parameter & \bfseries Value & \bfseries Alias\\*
% \hline
% \endhead
% \xoption{TextRenderingMode}
%   & \texttt{0} & \texttt{Fill}\\*
%   & \texttt{1} & \texttt{Stroke}\\*
%   & \texttt{2} & \texttt{FillStroke}\\*
%   & \texttt{3} & \texttt{Invisible}\\*
%   & \texttt{4} & \texttt{FillClip}\\*
%   & \texttt{5} & \texttt{StrokeClip}\\*
%   & \texttt{6} & \texttt{FillStrokeClip}\\*
%   & \texttt{7} & \texttt{Clip}\\
% \hline
% \xoption{LineWidth}
%   & \itshape positive number, & \itshape \TeX\ dimen\\*[-.2ex]
%   & \qquad \itshape unit is bp & \\
% \hline
% \xoption{LineCapStyle}
%   & \texttt{0} & \texttt{Butt}\\*
%   & \texttt{1} & \texttt{Round}\\*
%   & \texttt{2} & \texttt{ProjectingSquare}\\
% \hline
% \xoption{LineJoinStyle}
%   & \texttt{0} & \texttt{Miter}\\*
%   & \texttt{1} & \texttt{Round}\\*
%   & \texttt{2} & \texttt{Bevel}\\
% \hline
% \xoption{MiterLimit}
%   & \multicolumn{2}{l@{}}{\itshape positive number} \\
% \hline
% \xoption{Flatness}
%   & \multicolumn{2}{l@{}}{\itshape number between 0 and 100} \\
% \hline
% \xoption{LineDashPattern}
%   & \multicolumn{2}{l@{}}{^^A
%       \itshape numbers in square brackets,^^A
%     } \\*[-.2ex]
%   & \multicolumn{2}{l@{}}{^^A
%       \qquad \itshape followed by number,^^A
%     } \\*
%   & \multicolumn{2}{l@{}}{^^A
%       \qquad \itshape units are bp^^A
%     } \\
% \hline
% \xoption{RenderingIntent}
%   & \multicolumn{2}{l@{}}{\texttt{AbsoluteColorimetric}} \\*
%   & \multicolumn{2}{l@{}}{\texttt{RelativeColorimetric}} \\*
%   & \multicolumn{2}{l@{}}{\texttt{Saturation}} \\*
%   & \multicolumn{2}{l@{}}{\texttt{Perceptual}} \\
% \hline
% \xoption{FillColor}
%   & & \itshape color specification \\
% \hline
% \xoption{StrokeColor}
%   & & \itshape color specification \\
% \end{longtable}
%
% \subsubsection{Details}
%
% The description and specification of these parameters
% are available in the PDF specification \cite{pdf-spec}.
% Therefore they are not repeated here.
%
% \begin{description}
% \item[Value:]
%   The values in the second column lists or describe
%   the values that are specified by the PDF specification.
% \item[Alias:] Instead of magic numbers the package also
%   defines some aliases that can be given as value.
%   Example: |LineCapStyle=Round| has the same effect as
%   |LineCapStyle=1|.
% \item[Number:] The term \emph{number} means an integer
%   or real number. The real number is given as plain
%   decimal number without exponent. The decimal separator is a period.
%   At least one digit must be present.
% \item[\xoption{LineWidth}:] As alias a \TeX\ dimen specification
%   can be given. This includes explicit specifications with number
%   and unit, e.g. |LineWidth=0.5pt|. Also \LaTeX\ length registers
%   may be used. If \eTeX's \cs{dimexpr} is available, then it is
%   automatically added. However package \xpackage{calc} is not
%   supported.
% \item[\xoption{FillColor}, \xoption{StrokeColor}:]
%   Package \xpackage{color} or \xpackage{xcolor} must be loaded
%   before these options can be used (since version 1.2).
%   \LaTeX's color support sets both colors at the same time
%   to the same value. However parameter \xoption{TextRenderingMode}
%   offers the value \texttt{FillStroke} that makes only sense, if
%   the two color types can be set separately.
%   If one of the options \xoption{FillColor} or \xoption{StrokeColor}
%   is specified, then also the color is set. For compatibility
%   with the \LaTeX\ color packages (\xpackage{color} or \xpackage{xcolor}),
%   always both colors must be set. Thus if one of them is not specified,
%   it is taken from the current color.
%
%   Both options \xoption{FillColor} and \xoption{StrokeColor}
%   expect a \LaTeX\ color specification as value.
%   Also the optional color model argument is supported.
%   Example:
%   \begin{quote}
%     |FillColor=yellow|,\\
%     |StrokeColor=[cmyk]{1,.5,0,0}|\\
%   \end{quote}
% \end{description}
%
% \subsection{Color stack}
%
% If the \pdfTeX\ version provides color stacks, then each
% parameter is assigned a page based color stack. The assignment
% of a stack takes place, when its parameter is set the first time.
% This avoids the use of color stacks that are not needed.
%
% \StopEventually{
% }
%
% \section{Implementation}
%    \begin{macrocode}
%<*package>
%    \end{macrocode}
%    Reload check, especially if the package is not used with \LaTeX.
%    \begin{macrocode}
\begingroup\catcode61\catcode48\catcode32=10\relax%
  \catcode13=5 % ^^M
  \endlinechar=13 %
  \catcode35=6 % #
  \catcode39=12 % '
  \catcode44=12 % ,
  \catcode45=12 % -
  \catcode46=12 % .
  \catcode58=12 % :
  \catcode64=11 % @
  \catcode123=1 % {
  \catcode125=2 % }
  \expandafter\let\expandafter\x\csname ver@pdfrender.sty\endcsname
  \ifx\x\relax % plain-TeX, first loading
  \else
    \def\empty{}%
    \ifx\x\empty % LaTeX, first loading,
      % variable is initialized, but \ProvidesPackage not yet seen
    \else
      \expandafter\ifx\csname PackageInfo\endcsname\relax
        \def\x#1#2{%
          \immediate\write-1{Package #1 Info: #2.}%
        }%
      \else
        \def\x#1#2{\PackageInfo{#1}{#2, stopped}}%
      \fi
      \x{pdfrender}{The package is already loaded}%
      \aftergroup\endinput
    \fi
  \fi
\endgroup%
%    \end{macrocode}
%    Package identification:
%    \begin{macrocode}
\begingroup\catcode61\catcode48\catcode32=10\relax%
  \catcode13=5 % ^^M
  \endlinechar=13 %
  \catcode35=6 % #
  \catcode39=12 % '
  \catcode40=12 % (
  \catcode41=12 % )
  \catcode44=12 % ,
  \catcode45=12 % -
  \catcode46=12 % .
  \catcode47=12 % /
  \catcode58=12 % :
  \catcode64=11 % @
  \catcode91=12 % [
  \catcode93=12 % ]
  \catcode123=1 % {
  \catcode125=2 % }
  \expandafter\ifx\csname ProvidesPackage\endcsname\relax
    \def\x#1#2#3[#4]{\endgroup
      \immediate\write-1{Package: #3 #4}%
      \xdef#1{#4}%
    }%
  \else
    \def\x#1#2[#3]{\endgroup
      #2[{#3}]%
      \ifx#1\@undefined
        \xdef#1{#3}%
      \fi
      \ifx#1\relax
        \xdef#1{#3}%
      \fi
    }%
  \fi
\expandafter\x\csname ver@pdfrender.sty\endcsname
\ProvidesPackage{pdfrender}%
  [2019/12/29 v1.6 Access to some PDF graphics parameters (HO)]%
%    \end{macrocode}
%
%    \begin{macrocode}
\begingroup\catcode61\catcode48\catcode32=10\relax%
  \catcode13=5 % ^^M
  \endlinechar=13 %
  \catcode123=1 % {
  \catcode125=2 % }
  \catcode64=11 % @
  \def\x{\endgroup
    \expandafter\edef\csname PdfRender@AtEnd\endcsname{%
      \endlinechar=\the\endlinechar\relax
      \catcode13=\the\catcode13\relax
      \catcode32=\the\catcode32\relax
      \catcode35=\the\catcode35\relax
      \catcode61=\the\catcode61\relax
      \catcode64=\the\catcode64\relax
      \catcode123=\the\catcode123\relax
      \catcode125=\the\catcode125\relax
    }%
  }%
\x\catcode61\catcode48\catcode32=10\relax%
\catcode13=5 % ^^M
\endlinechar=13 %
\catcode35=6 % #
\catcode64=11 % @
\catcode123=1 % {
\catcode125=2 % }
\def\TMP@EnsureCode#1#2{%
  \edef\PdfRender@AtEnd{%
    \PdfRender@AtEnd
    \catcode#1=\the\catcode#1\relax
  }%
  \catcode#1=#2\relax
}
\TMP@EnsureCode{10}{12}% ^^J
\TMP@EnsureCode{36}{3}% $
\TMP@EnsureCode{39}{12}% '
\TMP@EnsureCode{40}{12}% (
\TMP@EnsureCode{41}{12}% )
\TMP@EnsureCode{42}{12}% *
\TMP@EnsureCode{43}{12}% +
\TMP@EnsureCode{44}{12}% ,
\TMP@EnsureCode{45}{12}% -
\TMP@EnsureCode{46}{12}% .
\TMP@EnsureCode{47}{12}% /
\TMP@EnsureCode{58}{12}% :
\TMP@EnsureCode{59}{12}% ;
\TMP@EnsureCode{60}{12}% <
\TMP@EnsureCode{62}{12}% >
\TMP@EnsureCode{63}{12}% ?
\TMP@EnsureCode{91}{12}% [
\TMP@EnsureCode{93}{12}% ]
\TMP@EnsureCode{94}{7}% ^ (superscript)
\TMP@EnsureCode{96}{12}% `
\TMP@EnsureCode{124}{12}% |
%    \end{macrocode}
%
%    \begin{macrocode}
\def\PdfRender@AtEndHook{}
\expandafter\def\expandafter\PdfRender@AtEnd\expandafter{%
  \expandafter\PdfRender@AtEndHook
  \PdfRender@AtEnd
  \endinput
}
%    \end{macrocode}
%
% \subsection{Look for \pdfTeX, its mode and features}
%
%    \begin{macro}{\PdfRender@newif}
%    \begin{macrocode}
\def\PdfRender@newif#1{%
  \expandafter\edef\csname PdfRender@#1true\endcsname{%
    \let
    \expandafter\noexpand\csname ifPdfRender@#1\endcsname
    \noexpand\iftrue
  }%
  \expandafter\edef\csname PdfRender@#1false\endcsname{%
    \let
    \expandafter\noexpand\csname ifPdfRender@#1\endcsname
    \noexpand\iffalse
  }%
  \csname PdfRender@#1false\endcsname
}
%    \end{macrocode}
%    \end{macro}
%
%    \begin{macro}{\ifPdfRender@Stack}
%    \begin{macrocode}
\PdfRender@newif{Stack}
%    \end{macrocode}
%    \end{macro}
%    \begin{macro}{\ifPdfRender@Match}
%    \begin{macrocode}
\PdfRender@newif{Match}
%    \end{macrocode}
%    \end{macro}
%
%    \begin{macro}{\PdfRender@RequirePackage}
%    \begin{macrocode}
\begingroup\expandafter\expandafter\expandafter\endgroup
\expandafter\ifx\csname RequirePackage\endcsname\relax
  \def\PdfRender@RequirePackage#1[#2]{%
    \expandafter\def\expandafter\PdfRender@AtEndHook\expandafter{%
      \PdfRender@AtEndHook
      \ltx@ifpackagelater{#1}{#2}{}{%
        \@PackageWarningNoLine{pdfrender}{%
          You have requested version\MessageBreak
          `#2' of package `#1',\MessageBreak
          but only version\MessageBreak
          `\csname ver@#1.\ltx@pkgextension\endcsname'\MessageBreak
          is available%
        }%
      }%
    }%
    \input #1.sty\relax
  }%
\else
  \let\PdfRender@RequirePackage\RequirePackage
\fi
%    \end{macrocode}
%    \end{macro}
% Luatex compatibility
%    \begin{macrocode}
\ifx\pdfextension\@undefined\else
  \def\pdfcolorstackinit {\pdffeedback colorstackinit}
  \protected\def\pdfcolorstack     {\pdfextension colorstack}
  \protected\def\pdfliteral        {\pdfextension literal}
\fi
%    \end{macrocode}
%
%    \begin{macrocode}
\PdfRender@RequirePackage{iftex}[2019/11/07]
\PdfRender@RequirePackage{infwarerr}[2007/09/09]
\PdfRender@RequirePackage{ltxcmds}[2010/01/28]
%    \end{macrocode}
%
%    \begin{macrocode}
\ifpdf
  \ltx@IfUndefined{pdfcolorstackinit}{%
    \@PackageWarning{pdfrender}{%
      Missing \string\pdfcolorstackinit
    }%
  }{%
    \PdfRender@Stacktrue
  }%
  \ltx@IfUndefined{pdfmatch}{%
    \@PackageInfoNoLine{pdfrender}{%
      \string\pdfmatch\ltx@space not found. %
      Therefore the values\MessageBreak
      of some parameters are not validated%
    }%
  }{%
    \PdfRender@Matchtrue
  }%
\else
  \@PackageWarning{pdfrender}{%
    Missing pdfTeX in PDF mode%
  }%
%    \end{macrocode}
%    \begin{macrocode}
  \ltx@IfUndefined{newcommand}{%
%    \end{macrocode}
%    \begin{macro}{\pdfrender}
%    \begin{macrocode}
    \def\pdfrender#1{}%
%    \end{macrocode}
%    \end{macro}
%    \begin{macro}{\textpdfrender}
%    \begin{macrocode}
    \long\def\textpdfrender#1#2{#2}%
%    \end{macrocode}
%    \end{macro}
%    \begin{macrocode}
  }{%
%    \end{macrocode}
%    \begin{macro}{\pdfrender}
%    \begin{macrocode}
    \newcommand*{\pdfrender}[1]{}%
%    \end{macrocode}
%    \end{macro}
%    \begin{macro}{\textpdfrender}
%    \begin{macrocode}
    \newcommand{\textpdfrender}[2]{#2}%
%    \end{macrocode}
%    \end{macro}
%    \begin{macrocode}
  }%
  \expandafter\PdfRender@AtEnd
\fi%
%    \end{macrocode}
%
% \subsection{Enable color support of \LaTeX}
%
%    \begin{macrocode}
\ltx@ifpackageloaded{color}{}{%
  \def\color@setgroup{\begingroup\set@color}%
  \let\color@begingroup\begingroup
  \def\color@endgroup{\endgraf\endgroup}%
  \def\color@hbox{\hbox\bgroup\color@begingroup}%
  \def\color@vbox{\vbox\bgroup\color@begingroup}%
  \def\color@endbox{\color@endgroup\egroup}%
  \ltx@ifundefined{bgroup}{%
    \let\bgroup={\let\egroup=}%
  }{}%
  \ltx@ifundefined{endgraf}{%
    \let\endgraf=\par
  }{}%
}
%    \end{macrocode}
%
% \subsection{Hook into \cs{normalcolor}}
%
%    The problem is that packages \xpackage{color} and \xpackage{xcolor}
%    each overwrite \cs{normalcolor}. For example, after the package
%    loading order \xpackage{color}, \xpackage{pdfrender} and \xpackage{xcolor}
%    the patched version of \cs{normalcolor} is overwritten by
%    package \xpackage{xcolor}. Also using \cs{AtBeginDocument} for
%    patching is not enough. If package \xpackage{hyperref} is loaded
%    later, it might load package \xpackage{color} using \cs{AtBeginDocument}.
%    \begin{macro}{\PdfRender@NormalColorHook}
%    \begin{macrocode}
\def\PdfRender@NormalColorHook{}
%    \end{macrocode}
%    \end{macro}
%    \begin{macro}{\PdfRender@ColorSetGroupHook}
%    \begin{macrocode}
\def\PdfRender@ColorSetGroupHook{}
%    \end{macrocode}
%    \end{macro}
%    \begin{macro}{\PdfRender@TestBox}
%    \begin{macrocode}
\def\PdfRender@TestBox#1{%
  \setbox0=\color@hbox#1\color@endbox
}
%    \end{macrocode}
%    \end{macro}
%    \begin{macro}{\PdfRender@PatchNormalColor}
%    \begin{macrocode}
\def\PdfRender@PatchNormalColor{%
  \ltx@ifundefined{normalcolor}{%
    \gdef\normalcolor{\PdfRender@NormalColorHook}%
  }{%
    \begingroup
      \def\PdfRender@NormalColorHook{\let\PdfRender@temp=Y}%
      \PdfRender@TestBox{%
        \let\set@color\relax
        \normalcolor
        \ifx\PdfRender@temp Y%
        \else
          \ltx@GlobalAppendToMacro\normalcolor{%
            \PdfRender@NormalColorHook
          }%
        \fi
      }%
    \endgroup
  }%
  \ifx\@nodocument\relax
    \global\let\PdfRender@PatchNormalColor\relax
  \fi
}%
%    \end{macrocode}
%    \end{macro}
%    \begin{macro}{\PdfRender@PatchColorSetGroup}
%    \begin{macrocode}
\def\PdfRender@PatchColorSetGroup{%
  \begingroup
    \def\PdfRender@ColorSetGroupHook{\let\PdfRender@temp=Y}%
    \PdfRender@TestBox{%
      \let\set@color\relax
      \color@setgroup\color@endgroup
      \ifx\PdfRender@temp Y%
      \else
        \ltx@GlobalAppendToMacro\color@setgroup{%
          \PdfRender@ColorSetGroupHook
        }%
      \fi
    }%
  \endgroup
  \ifx\@nodocument\relax
    \global\let\PdfRender@PatchColorSetGroup\relax
  \fi
}%
%    \end{macrocode}
%    \end{macro}
%    \begin{macro}{\PdfRender@PatchColor}
%    \begin{macrocode}
\def\PdfRender@PatchColor{%
  \PdfRender@PatchNormalColor
  \PdfRender@PatchColorSetGroup
}
%    \end{macrocode}
%    \end{macro}
%    \begin{macrocode}
\PdfRender@PatchColor
\ltx@IfUndefined{AtBeginDocument}{}{%
  \AtBeginDocument{\PdfRender@PatchColor}%
}
%    \end{macrocode}
%    \cs{AfterPackage} is provided by package \xpackage{scrlfile}.
%    \begin{macrocode}
\ltx@IfUndefined{AfterPackage}{%
}{%
  \AfterPackage{color}{\PdfRender@PatchColor}%
  \AfterPackage{xcolor}{\PdfRender@PatchColor}%
  \AfterPackage{etoolbox}{%
    \AfterEndPreamble{\PdfRender@PatchColor}%
  }%
}%
%    \end{macrocode}
%    \cs{AfterEndPreamble} is provided by package \xpackage{etoolbox}.
%    \begin{macrocode}
\ltx@IfUndefined{AfterEndPreamble}{%
}{%
  \AfterEndPreamble{\PdfRender@PatchColor}%
}%
%    \end{macrocode}
%
%    \begin{macrocode}
\PdfRender@RequirePackage{kvsetkeys}[2010/01/28]
%    \end{macrocode}
%    \begin{macro}{\PdfRender@texorpdfstring}
%    \begin{macrocode}
\def\PdfRender@texorpdfstring{%
  \ltx@IfUndefined{texorpdfstring}\ltx@firstoftwo\texorpdfstring
}
%    \end{macrocode}
%    \end{macro}
%    \begin{macro}{\pdfrender}
%    \begin{macrocode}
\ltx@IfUndefined{DeclareRobustCommand}%
\ltx@firstoftwo\ltx@secondoftwo
{%
  \def\pdfrender#1%
}{%
  \newcommand{\pdfrender}{}%
  \DeclareRobustCommand*{\pdfrender}[1]%
}%
{%
  \PdfRender@texorpdfstring{%
    \PdfRender@PatchNormalColor
    \global\let\PdfRender@FillColor\ltx@empty
    \global\let\PdfRender@StrokeColor\ltx@empty
    \kvsetkeys{PDFRENDER}{#1}%
    \PdfRender@SetColor
  }{}%
}
%    \end{macrocode}
%    \end{macro}
%    \begin{macro}{\textpdfrender}
%    \begin{macrocode}
\ltx@IfUndefined{DeclareRobustCommand}%
\ltx@firstoftwo\ltx@secondoftwo
{%
  \long\def\textpdfrender#1#2%
}{%
  \newcommand{\textpdfrender}{}%
  \DeclareRobustCommand{\textpdfrender}[2]%
}%
{%
  \PdfRender@texorpdfstring{%
    \begingroup
      \pdfrender{#1}%
      #2%
    \endgroup
  }{#2}%
}
%    \end{macrocode}
%    \end{macro}
%
%    \begin{macro}{\ifPdfRender@Values}
%    \begin{macrocode}
\PdfRender@newif{Values}
%    \end{macrocode}
%    \end{macro}
%    \begin{macro}{\PdfRender@NewClassValues}
%    \begin{macrocode}
\def\PdfRender@NewClassValues#1#2#3#4{%
  \PdfRender@Valuestrue
  \PdfRender@NewClass{#1}{#2}{#3}{#4}{}%
}
%    \end{macrocode}
%    \end{macro}
%    \begin{macro}{\PdfRender@NewClass}
%    \begin{macrocode}
\def\PdfRender@NewClass#1#2#3#4#5{%
  \PdfRender@newif{Active#1}%
  \expandafter\def\csname PdfRender@Default#1\endcsname{#2}%
  \expandafter\let\csname PdfRender@Current#1\expandafter\endcsname
      \csname PdfRender@Default#1\endcsname
  \ifPdfRender@Stack
    \expandafter\edef\csname PdfRender@Init#1\endcsname{%
      \global\chardef
      \expandafter\noexpand\csname PdfRender@Stack#1\endcsname=%
          \noexpand\pdfcolorstackinit page direct{%
            \noexpand#3%
            \expandafter\noexpand\csname PdfRender@Default#1\endcsname
          }\relax
      \noexpand\@PackageInfo{pdfrender}{%
        New color stack `#1' = \noexpand\number
        \expandafter\noexpand\csname PdfRender@Stack#1\endcsname
      }%
      \gdef\expandafter\noexpand\csname PdfRender@Init#1\endcsname{}%
    }%
    \expandafter\edef\csname PdfRender@Set#1\endcsname{%
      \expandafter\noexpand\csname PdfRender@Init#1\endcsname
      \noexpand\pdfcolorstack
      \expandafter\noexpand\csname PdfRender@Stack#1\endcsname
      push{%
        #3{\expandafter\noexpand\csname PdfRender@Current#1\endcsname}%
      }%
      \noexpand\aftergroup
      \expandafter\noexpand\csname PdfRender@Reset#1\endcsname
    }%
    \expandafter\edef\csname PdfRender@Reset#1\endcsname{%
      \expandafter\noexpand\csname PdfRender@Init#1\endcsname
      \noexpand\pdfcolorstack
      \expandafter\noexpand\csname PdfRender@Stack#1\endcsname
      pop\relax
    }%
  \else
    \expandafter\edef\csname PdfRender@Set#1\endcsname{%
      \noexpand\pdfliteral direct{%
        #3{\expandafter\noexpand\csname PdfRender@Current#1\endcsname}%
      }%
      \noexpand\aftergroup
      \expandafter\noexpand\csname PdfRender@Reset#1\endcsname
    }%
    \expandafter\edef\csname PdfRender@Reset#1\endcsname{%
      \noexpand\pdfliteral direct{%
        #3{\expandafter\noexpand\csname PdfRender@Current#1\endcsname}%
      }%
    }%
  \fi
  \expandafter\edef\csname PdfRender@Normal#1\endcsname{%
    \let
    \expandafter\noexpand\csname PdfRender@Current#1\endcsname
    \expandafter\noexpand\csname PdfRender@Default#1\endcsname
    \noexpand\PdfRender@Set{#1}%
  }%
  \expandafter\ltx@GlobalAppendToMacro\expandafter\PdfRender@NormalColorHook
  \expandafter{%
    \csname PdfRender@Normal#1\endcsname
  }%
  \ltx@GlobalAppendToMacro\PdfRender@ColorSetGroupHook{%
    \PdfRender@Set{#1}%
  }%
  \ifPdfRender@Values
    \kv@parse@normalized{#4}{%
      \expandafter\let\csname PdfRender@#1@\kv@key\endcsname\kv@key
      \ifx\kv@value\relax
      \else
        \expandafter\let\csname PdfRender@#1@\kv@value\endcsname\kv@key
      \fi
      \ltx@gobbletwo
    }%
    \PdfRender@define@key{PDFRENDER}{#1}{%
      \global\csname PdfRender@Active#1true\endcsname
      \def\PdfRender@Current{##1}%
      \PdfRender@SetValidateValues{#1}%
    }%
    \PdfRender@Valuesfalse
  \else
    \PdfRender@define@key{PDFRENDER}{#1}{%
      \global\csname PdfRender@Active#1true\endcsname
      \expandafter\def\csname PdfRender@Current#1\endcsname{##1}%
      \ltx@IfUndefined{PdfRender@PostProcess#1}{%
      }{%
        \csname PdfRender@PostProcess#1\endcsname
      }%
      \PdfRender@SetValidate{#1}{#4}{#5}%
    }%
  \fi
}%
%    \end{macrocode}
%    \end{macro}
%    \begin{macro}{\PdfRender@define@key}
%    \begin{macrocode}
\ltx@IfUndefined{define@key}{%
  \def\PdfRender@define@key#1#2{%
    \expandafter\def\csname KV@#1@#2\endcsname##1%
  }%
}{%
  \let\PdfRender@define@key\define@key
}
%    \end{macrocode}
%    \end{macro}
%    \begin{macro}{\PdfRender@Set}
%    \begin{macrocode}
\def\PdfRender@Set#1{%
  \csname ifPdfRender@Active#1\endcsname
    \csname PdfRender@Set#1\expandafter\endcsname
  \fi
}
%    \end{macrocode}
%    \end{macro}
%    \begin{macro}{\PdfRender@Reset}
%    \begin{macrocode}
\def\PdfRender@Reset#1{%
  \csname ifPdfRender@Active#1\endcsname
    \csname PdfRender@Reset#1\expandafter\endcsname
  \fi
}
%    \end{macrocode}
%    \end{macro}
%    \begin{macro}{\PdfRender@ErrorInvalidValue}
%    \begin{macrocode}
\def\PdfRender@ErrorInvalidValue#1{%
  \PackageError{pdfrender}{%
    Ignoring parameter setting for `#1'\MessageBreak
    because of invalid value %
    `\csname PdfRender@Current#1\endcsname'%
  }\@ehc
  \expandafter\let\csname PdfRender@Current#1\endcsname\ltx@empty
}%
%    \end{macrocode}
%    \end{macro}
%    \begin{macro}{\PdfRender@SetValidate}
%    \begin{macrocode}
\ifPdfRender@Match
  \def\PdfRender@SetValidate#1#2#3{%
    \ifnum\pdfmatch{^(#2)$}{\csname PdfRender@Current#1\endcsname}=1 %
      \csname PdfRender@Set#1\expandafter\endcsname
    \else
      \PdfRender@ErrorInvalidValue{#1}%
    \fi
  }%
\else
  \def\PdfRender@SetValidate#1#2#3{%
    \expandafter\let\expandafter\PdfRender@Current
    \csname PdfRender@Current#1\endcsname
    #3%
    \ifx\PdfRender@Current\@empty
      \PdfRender@ErrorInvalidValue{#1}%
    \else
      \csname PdfRender@Set#1\expandafter\endcsname
    \fi
  }%
\fi
%    \end{macrocode}
%    \end{macro}
%    \begin{macro}{\PdfRender@SetValidateValues}
%    \begin{macrocode}
\def\PdfRender@SetValidateValues#1{%
  \ltx@IfUndefined{PdfRender@#1@\PdfRender@Current}{%
    \expandafter\let\csname PdfRender@Current#1\endcsname
                    \PdfRender@Current
    \PdfRender@ErrorInvalidValue{#1}%
  }{%
    \expandafter\edef\csname PdfRender@Current#1\endcsname{%
      \csname PdfRender@#1@\PdfRender@Current\endcsname
    }%
    \csname PdfRender@Set#1\endcsname
  }%
}
%    \end{macrocode}
%    \end{macro}
%    \begin{macro}{\PdfRender@OpValue}
%    \begin{macrocode}
\def\PdfRender@OpValue#1#2{#2\ltx@space#1}%
%    \end{macrocode}
%    \end{macro}
%    \begin{macro}{\PdfRender@OpName}
%    \begin{macrocode}
\def\PdfRender@OpName#1#2{/#2\ltx@space#1}%
%    \end{macrocode}
%    \end{macro}
%
% \subsection{Declare and setup parameters}
%
%    \begin{macrocode}
\PdfRender@NewClassValues{TextRenderingMode}%
                      {0}%
                      {\PdfRender@OpValue{Tr}}{%
  0=Fill,%
  1=Stroke,%
  2=FillStroke,%
  3=Invisible,%
  4=FillClip,%
  5=StrokeClip,%
  6=FillStrokeClip,%
  7=Clip,%
}%
\PdfRender@NewClass{LineWidth}{1}{\PdfRender@OpValue{w}}{%
  [0-9]+\string\.?[0-9]*|\string\.[0-9]+%
}{}%
\ltx@IfUndefined{dimexpr}{%
  \def\PdfRender@dimexpr{}%
}{%
  \let\PdfRender@dimexpr\dimexpr
}
\def\PdfRender@PostProcessLineWidth{%
  \begingroup
  \afterassignment\PdfRender@@PostProcessLineWidth
  \dimen0=\PdfRender@dimexpr\PdfRender@CurrentLineWidth bp %
  \PdfRender@let\PdfRender@relax\PdfRender@relax
}
\let\PdfRender@let\let
\let\PdfRender@relax\relax
\def\PdfRender@@PostProcessLineWidth#1\PdfRender@let{%
  \ifx\\#1\\%
    \endgroup
  \else
    \dimen0=.996264\dimen0 % 72/72.27
    \edef\x{\endgroup
      \def\noexpand\PdfRender@CurrentLineWidth{%
        \strip@pt\dimen0%
      }%
    }%
    \expandafter\x
  \fi
}
\PdfRender@NewClassValues{LineCapStyle}{0}{\PdfRender@OpValue{J}}{%
  0=Butt,%
  1=Round,%
  2=ProjectingSquare,%
}%
\PdfRender@NewClassValues{LineJoinStyle}{0}{\PdfRender@OpValue{j}}{%
  0=Miter,%
  1=Round,%
  2=Bevel,%
}%
\PdfRender@NewClass{MiterLimit}{10}{\PdfRender@OpValue{M}}{%
  [0-9]*[1-9][0-9]*\string\.?[0-9]*|%
  [0-9]*\string\.?[0-9]*[1-9][0-9]*%
}{}%
\PdfRender@NewClass{Flatness}{0}{\PdfRender@OpValue{i}}{%
  100(\string\.0*)?|[0-9][0-9](\string\.[0-9]*)?|\string\.[0-9]+%
}{}%
\PdfRender@NewClass{LineDashPattern}{[]0}{\PdfRender@OpValue{d}}{%
  \string\[%
  ( ?([0-9]+\string\.?[0-9]*|\string\.[0-9]+) ?)*%
  \string\] ?%
  ([0-9]+\string\.?[0-9]*|\string\.[0-9]+)%
}{}%
\PdfRender@NewClassValues{RenderingIntent}%
                      {RelativeColorimetric}%
                      {\PdfRender@OpName{ri}}{%
  AbsoluteColorimetric,%
  RelativeColorimetric,%
  Saturation,%
  Perceptual,%
}%
%    \end{macrocode}
%
% \subsection{Fill and stroke color support}
%
%    \begin{macrocode}
\PdfRender@define@key{PDFRENDER}{FillColor}{%
  \begingroup
    \def\PdfRender@Color{#1}%
    \ifx\PdfRender@Color\ltx@empty
      \global\let\PdfRender@FillColor\ltx@empty
    \else
      \PdfRender@ColorAvailable{%
        \PdfRender@TestBox{%
          \expandafter\PdfRender@TryColor\PdfRender@Color\ltx@empty
          \PdfRender@GetFillColor
          \ifx\PdfRender@FillColor\ltx@empty
            \@PackageWarning{pdfrender}{%
              Cannot extract fill color\MessageBreak
              from value `#1'%
            }%
          \fi
        }%
      }%
    \fi
  \endgroup
}
\PdfRender@define@key{PDFRENDER}{StrokeColor}{%
  \begingroup
    \def\PdfRender@Color{#1}%
    \ifx\PdfRender@Color\ltx@empty
      \global\let\PdfRender@StrokeColor\ltx@empty
    \else
      \PdfRender@ColorAvailable{%
        \PdfRender@TestBox{%
          \expandafter\PdfRender@TryColor\PdfRender@Color\ltx@empty
          \PdfRender@GetStrokeColor
          \ifx\PdfRender@StrokeColor\ltx@empty
            \@PackageWarning{pdfrender}{%
              Cannot extract stroke color\MessageBreak
              from value `#1'%
            }%
          \fi
        }%
      }%
    \fi
  \endgroup
}
%    \end{macrocode}
%    \begin{macro}{\PdfRender@ColorAvailable}
%    \begin{macrocode}
\def\PdfRender@ColorAvailable{%
  \@ifundefined{set@color}{%
    \@PackageError{pdfrender}{%
      Ignoring color options, because neither\MessageBreak
      package `color' nor package `xcolor' is loaded%
    }\@ehc
    \global\let\PdfRender@ColorAvailable\ltx@gobble
  }{%
    \global\let\PdfRender@ColorAvailable\ltx@firstofone
  }%
  \PdfRender@ColorAvailable
}
%    \end{macrocode}
%    \end{macro}
%    \begin{macro}{\PdfRender@TryColor}
%    \begin{macrocode}
\def\PdfRender@TryColor{%
  \@ifnextchar[\color\PdfRender@@TryColor
}
%    \end{macrocode}
%    \end{macro}
%    \begin{macro}{\PdfRender@@TryColor}
%    \begin{macrocode}
\def\PdfRender@@TryColor#1\ltx@empty{%
  \expandafter\color\expandafter{\PdfRender@Color}%
}
%    \end{macrocode}
%    \end{macro}
%    \begin{macro}{\PdfRender@SetColor}
%    \begin{macrocode}
\def\PdfRender@SetColor{%
  \chardef\PdfRender@NeedsCurrentColor=0 %
  \ifx\PdfRender@FillColor\ltx@empty
    \ifx\PdfRender@StrokeColor\ltx@empty
    \else
      \edef\PdfRender@CurrentColor{%
        \noexpand\PdfRender@FillColor\ltx@space\PdfRender@StrokeColor
      }%
      \chardef\PdfRender@NeedsCurrentColor=1 %
    \fi
  \else
    \ifx\PdfRender@StrokeColor\ltx@empty
      \edef\PdfRender@CurrentColor{%
        \PdfRender@FillColor\ltx@space\noexpand\PdfRender@StrokeColor
      }%
      \chardef\PdfRender@NeedsCurrentColor=2 %
    \else
      \edef\current@color{%
        \PdfRender@FillColor\ltx@space\PdfRender@StrokeColor
      }%
      \set@color
    \fi
  \fi
  \ifnum\PdfRender@NeedsCurrentColor=1 %
    \PdfRender@GetFillColor
    \ifx\PdfRender@FillColor\ltx@empty
      \@PackageWarning{pdfrender}{%
        Cannot extract current fill color%
      }%
    \else
      \edef\current@color{\PdfRender@CurrentColor}%
      \set@color
    \fi
  \else
    \ifnum\PdfRender@NeedsCurrentColor=2 %
      \PdfRender@GetStrokeColor
      \ifx\PdfRender@StrokeColor\ltx@empty
        \@PackageWarning{pdfrender}{%
          Cannot extract current stroke color%
        }%
      \else
        \edef\current@color{\PdfRender@CurrentColor}%
        \set@color
      \fi
    \fi
  \fi
}
%    \end{macrocode}
%    \end{macro}
%    \begin{macro}{\PdfRender@PatternFillColor}
%    \begin{macrocode}
\edef\PdfRender@PatternFillColor{ % space
  (%
    [0-9\string\.]+ g|%
    [0-9\string\.]+ [0-9\string\.]+ [0-9\string\.]+ rg|%
    [0-9\string\.]+ [0-9\string\.]+ %
    [0-9\string\.]+ [0-9\string\.]+ k%
  ) % space
  (.*)$%
}
%    \end{macrocode}
%    \end{macro}
%    \begin{macro}{\PdfRender@PatternStrokeColor}
%    \begin{macrocode}
\edef\PdfRender@PatternStrokeColor{ % space
  (%
    [0-9\string\.]+ G|%
    [0-9\string\.]+ [0-9\string\.]+ [0-9\string\.]+ RG|%
    [0-9\string\.]+ [0-9\string\.]+ %
    [0-9\string\.]+ [0-9\string\.]+ K%
  ) % space
  (.*)$%
}
%    \end{macrocode}
%    \end{macro}
%    \begin{macro}{\PdfRender@MatchPattern}
%    \begin{macrocode}
\def\PdfRender@MatchPattern#1{%
  \ifnum\pdfmatch{\PdfRender@Pattern}{\PdfRender@String}=1 %
    \xdef#1{%
      \expandafter\strip@prefix\pdflastmatch 1%
    }%
    \edef\PdfRender@String{%
      \expandafter\strip@prefix\pdflastmatch 2%
    }%
    \ifx\PdfRender@String\ltx@empty
    \else
      \expandafter\expandafter\expandafter\PdfRender@MatchPattern
      \expandafter\expandafter\expandafter#1%
    \fi
  \fi
}
%    \end{macrocode}
%    \end{macro}
%    \begin{macro}{\PdfRender@GetFillColor}
%    \begin{macrocode}
\def\PdfRender@GetFillColor{%
  \global\let\PdfRender@FillColor\ltx@empty
  \begingroup
    \ifPdfRender@Match
      \let\PdfRender@Pattern\PdfRender@PatternFillColor
      \edef\PdfRender@String{\ltx@space\current@color\ltx@space}%
      \PdfRender@MatchPattern\PdfRender@FillColor
    \else
      \edef\current@color{\current@color\ltx@space}%
      \let\PdfRender@OP\relax
      \PdfRender@FindOp{g}0%
      \PdfRender@FindOp{G}1%
      \PdfRender@FindOp{rg}0%
      \PdfRender@FindOp{RG}1%
      \PdfRender@FindOp{k}0%
      \PdfRender@FindOp{K}1%
      \PdfRender@FilterOp 0\PdfRender@FillColor
    \fi
  \endgroup
}
%    \end{macrocode}
%    \end{macro}
%    \begin{macro}{\PdfRender@GetStrokeColor}
%    \begin{macrocode}
\def\PdfRender@GetStrokeColor{%
  \global\let\PdfRender@StrokeColor\ltx@empty
  \begingroup
    \ifPdfRender@Match
      \let\PdfRender@Pattern\PdfRender@PatternStrokeColor
      \edef\PdfRender@String{\ltx@space\current@color\ltx@space}%
      \PdfRender@MatchPattern\PdfRender@StrokeColor
    \else
      \edef\current@color{\current@color\ltx@space}%
      \let\PdfRender@OP\relax
      \PdfRender@FindOp{g}0%
      \PdfRender@FindOp{G}1%
      \PdfRender@FindOp{rg}0%
      \PdfRender@FindOp{RG}1%
      \PdfRender@FindOp{k}0%
      \PdfRender@FindOp{K}1%
      \PdfRender@FilterOp 1\PdfRender@StrokeColor
    \fi
  \endgroup
}
%    \end{macrocode}
%    \end{macro}
%    \begin{macrocode}
\ifPdfRender@Match
  \expandafter\PdfRender@AtEnd
\fi%
%    \end{macrocode}
%    \begin{macro}{\PdfRender@FindOp}
%    \begin{macrocode}
\def\PdfRender@FindOp#1#2{%
  \def\PdfRender@temp##1 #1 ##2\@nil{%
    ##1%
    \ifx\\##2\\%
      \expandafter\@gobble
    \else
      \PdfRender@OP{#1}#2%
      \expandafter\@firstofone
    \fi
    {%
      \PdfRender@temp##2\@nil
    }%
  }%
  \edef\current@color{%
    \@firstofone{\expandafter\PdfRender@temp\current@color} #1 \@nil
  }%
}
%    \end{macrocode}
%    \end{macro}
%    \begin{macro}{\PdfRender@FilterOp}
%    \begin{macrocode}
\def\PdfRender@FilterOp#1#2{%
  \expandafter\PdfRender@@FilterOp\expandafter#1\expandafter#2%
      \current@color\PdfRender@OP{}{}%
}
%    \end{macrocode}
%    \end{macro}
%    \begin{macro}{\PdfRender@@FilterOp}
%    \begin{macrocode}
\def\PdfRender@@FilterOp#1#2#3\PdfRender@OP#4#5{%
  \ifx\\#4#5\\%
  \else
    \ifnum#1=#5 %
      \xdef#2{#3 #4}%
    \fi
    \expandafter\PdfRender@@FilterOp\expandafter#1\expandafter#2%
  \fi
}
%    \end{macrocode}
%    \end{macro}
%
%    \begin{macrocode}
\PdfRender@AtEnd%
%</package>
%    \end{macrocode}
%% \section{Installation}
%
% \subsection{Download}
%
% \paragraph{Package.} This package is available on
% CTAN\footnote{\CTANpkg{pdfrender}}:
% \begin{description}
% \item[\CTAN{macros/latex/contrib/oberdiek/pdfrender.dtx}] The source file.
% \item[\CTAN{macros/latex/contrib/oberdiek/pdfrender.pdf}] Documentation.
% \end{description}
%
%
% \paragraph{Bundle.} All the packages of the bundle `oberdiek'
% are also available in a TDS compliant ZIP archive. There
% the packages are already unpacked and the documentation files
% are generated. The files and directories obey the TDS standard.
% \begin{description}
% \item[\CTANinstall{install/macros/latex/contrib/oberdiek.tds.zip}]
% \end{description}
% \emph{TDS} refers to the standard ``A Directory Structure
% for \TeX\ Files'' (\CTANpkg{tds}). Directories
% with \xfile{texmf} in their name are usually organized this way.
%
% \subsection{Bundle installation}
%
% \paragraph{Unpacking.} Unpack the \xfile{oberdiek.tds.zip} in the
% TDS tree (also known as \xfile{texmf} tree) of your choice.
% Example (linux):
% \begin{quote}
%   |unzip oberdiek.tds.zip -d ~/texmf|
% \end{quote}
%
% \subsection{Package installation}
%
% \paragraph{Unpacking.} The \xfile{.dtx} file is a self-extracting
% \docstrip\ archive. The files are extracted by running the
% \xfile{.dtx} through \plainTeX:
% \begin{quote}
%   \verb|tex pdfrender.dtx|
% \end{quote}
%
% \paragraph{TDS.} Now the different files must be moved into
% the different directories in your installation TDS tree
% (also known as \xfile{texmf} tree):
% \begin{quote}
% \def\t{^^A
% \begin{tabular}{@{}>{\ttfamily}l@{ $\rightarrow$ }>{\ttfamily}l@{}}
%   pdfrender.sty & tex/generic/oberdiek/pdfrender.sty\\
%   pdfrender.pdf & doc/latex/oberdiek/pdfrender.pdf\\
%   pdfrender.dtx & source/latex/oberdiek/pdfrender.dtx\\
% \end{tabular}^^A
% }^^A
% \sbox0{\t}^^A
% \ifdim\wd0>\linewidth
%   \begingroup
%     \advance\linewidth by\leftmargin
%     \advance\linewidth by\rightmargin
%   \edef\x{\endgroup
%     \def\noexpand\lw{\the\linewidth}^^A
%   }\x
%   \def\lwbox{^^A
%     \leavevmode
%     \hbox to \linewidth{^^A
%       \kern-\leftmargin\relax
%       \hss
%       \usebox0
%       \hss
%       \kern-\rightmargin\relax
%     }^^A
%   }^^A
%   \ifdim\wd0>\lw
%     \sbox0{\small\t}^^A
%     \ifdim\wd0>\linewidth
%       \ifdim\wd0>\lw
%         \sbox0{\footnotesize\t}^^A
%         \ifdim\wd0>\linewidth
%           \ifdim\wd0>\lw
%             \sbox0{\scriptsize\t}^^A
%             \ifdim\wd0>\linewidth
%               \ifdim\wd0>\lw
%                 \sbox0{\tiny\t}^^A
%                 \ifdim\wd0>\linewidth
%                   \lwbox
%                 \else
%                   \usebox0
%                 \fi
%               \else
%                 \lwbox
%               \fi
%             \else
%               \usebox0
%             \fi
%           \else
%             \lwbox
%           \fi
%         \else
%           \usebox0
%         \fi
%       \else
%         \lwbox
%       \fi
%     \else
%       \usebox0
%     \fi
%   \else
%     \lwbox
%   \fi
% \else
%   \usebox0
% \fi
% \end{quote}
% If you have a \xfile{docstrip.cfg} that configures and enables \docstrip's
% TDS installing feature, then some files can already be in the right
% place, see the documentation of \docstrip.
%
% \subsection{Refresh file name databases}
%
% If your \TeX~distribution
% (\TeX\,Live, \mikTeX, \dots) relies on file name databases, you must refresh
% these. For example, \TeX\,Live\ users run \verb|texhash| or
% \verb|mktexlsr|.
%
% \subsection{Some details for the interested}
%
% \paragraph{Unpacking with \LaTeX.}
% The \xfile{.dtx} chooses its action depending on the format:
% \begin{description}
% \item[\plainTeX:] Run \docstrip\ and extract the files.
% \item[\LaTeX:] Generate the documentation.
% \end{description}
% If you insist on using \LaTeX\ for \docstrip\ (really,
% \docstrip\ does not need \LaTeX), then inform the autodetect routine
% about your intention:
% \begin{quote}
%   \verb|latex \let\install=y\input{pdfrender.dtx}|
% \end{quote}
% Do not forget to quote the argument according to the demands
% of your shell.
%
% \paragraph{Generating the documentation.}
% You can use both the \xfile{.dtx} or the \xfile{.drv} to generate
% the documentation. The process can be configured by the
% configuration file \xfile{ltxdoc.cfg}. For instance, put this
% line into this file, if you want to have A4 as paper format:
% \begin{quote}
%   \verb|\PassOptionsToClass{a4paper}{article}|
% \end{quote}
% An example follows how to generate the
% documentation with pdf\LaTeX:
% \begin{quote}
%\begin{verbatim}
%pdflatex pdfrender.dtx
%makeindex -s gind.ist pdfrender.idx
%pdflatex pdfrender.dtx
%makeindex -s gind.ist pdfrender.idx
%pdflatex pdfrender.dtx
%\end{verbatim}
% \end{quote}
%
% \section{Acknowledgement}
%
% \begin{description}
% \item[Friedrich Vosberg] asked in the
%   newsgroup \textsf{de.comp.text.tex} for the font
%   outline feature \cite{news-Vosberg}.
% \item[Gaius Pupus] proposed the basic method
%   using \cs{pdfliteral} in this thread \cite{news-Pupus}.
% \item[Rolf Niepraschk] added color support \cite{news-Niepraschk}.
% \end{description}
%
% \begin{thebibliography}{9}
% \raggedright
%
% \bibitem{pdf-spec}
%   Adobe Systems Incorporated. \textit{PDF Reference --
%   Adobe Portable Document format -- Version 1.7}.
%   6th ed. 2006.
%   \textsc{URL:}\,\url{http://www.adobe.com/devnet/acrobat/pdfs/pdf_reference_1-7.pdf}.
%
% \bibitem{news-Vosberg}
%   Friedrich Vosberg, \textit{Text in Buchstabenumrissen},
%   de.comp.text.tex, 2010-01-22.
%   \textsc{URL:}\,\url{https://groups.google.com/group/de.comp.text.tex/msg/f442310ac8b2d506}.
%
% \bibitem{news-Pupus}
%   Gaius Pupus, \textit{Re: Text in Buchstabenumrissen},
%   de.comp.text.tex, 2010-01-23.
%   \textsc{URL:}\,\url{https://groups.google.com/group/de.comp.text.tex/msg/95d890d77ac47eb1}.
%
% \bibitem{news-Niepraschk}
%   Rolf Niepraschk, \textit{Re: Text in Buchstabenumrissen},
%   de.comp.text.tex, 2010-01-24.
%   \textsc{URL:}\,\url{https://groups.google.com/group/de.comp.text.tex/msg/4eb61a5879db54db}.
%
% \end{thebibliography}
%
% \begin{History}
%   \begin{Version}{2010/01/26 v1.0}
%   \item
%     The first version.
%   \end{Version}
%   \begin{Version}{2010/01/27 v1.1}
%   \item
%     Macros \cs{pdfrender} and \cs{textpdfrender} are made robust.
%   \item
%     Color extraction rewritten for the case that \cs{pdfmatch}
%     is not available. This fixes wrong color assigments in
%     case of nesting.
%   \item
%     Color extraction of case \cs{pdfmatch} is fixed
%     for the case that the color string contains several
%     fill or several stroke operations.
%   \end{Version}
%   \begin{Version}{2010/01/28 v1.2}
%   \item
%     Dependency from package \xpackage{color} is removed.
%   \item
%     Compatibility for \plainTeX\ and even ini\TeX\ added.
%   \end{Version}
%   \begin{Version}{2016/05/14 v1.3}
%   \item
%     Use package \xpackage{luatex85} for compatibility with new LuaTeX.
%   \end{Version}
%   \begin{Version}{2016/05/17 v1.4}
%   \item
%     Documentation updates.
%   \item adjust luatex85 reference so that it works in plain TeX.
%   \end{Version}
%   \begin{Version}{2018/11/01 v1.5}
%   \item
%     Remove luatex85 dependency
%   \end{Version}
%   \begin{Version}{2019/12/29 v1.6}
%   \item
%     \xpackage{iftex} package.
%   \end{Version}
% \end{History}
%
% \PrintIndex
%
% \Finale
\endinput

%        (quote the arguments according to the demands of your shell)
%
% Documentation:
%    (a) If pdfrender.drv is present:
%           latex pdfrender.drv
%    (b) Without pdfrender.drv:
%           latex pdfrender.dtx; ...
%    The class ltxdoc loads the configuration file ltxdoc.cfg
%    if available. Here you can specify further options, e.g.
%    use A4 as paper format:
%       \PassOptionsToClass{a4paper}{article}
%
%    Programm calls to get the documentation (example):
%       pdflatex pdfrender.dtx
%       makeindex -s gind.ist pdfrender.idx
%       pdflatex pdfrender.dtx
%       makeindex -s gind.ist pdfrender.idx
%       pdflatex pdfrender.dtx
%
% Installation:
%    TDS:tex/generic/oberdiek/pdfrender.sty
%    TDS:doc/latex/oberdiek/pdfrender.pdf
%    TDS:source/latex/oberdiek/pdfrender.dtx
%
%<*ignore>
\begingroup
  \catcode123=1 %
  \catcode125=2 %
  \def\x{LaTeX2e}%
\expandafter\endgroup
\ifcase 0\ifx\install y1\fi\expandafter
         \ifx\csname processbatchFile\endcsname\relax\else1\fi
         \ifx\fmtname\x\else 1\fi\relax
\else\csname fi\endcsname
%</ignore>
%<*install>
\input docstrip.tex
\Msg{************************************************************************}
\Msg{* Installation}
\Msg{* Package: pdfrender 2019/12/29 v1.6 Access to some PDF graphics parameters (HO)}
\Msg{************************************************************************}

\keepsilent
\askforoverwritefalse

\let\MetaPrefix\relax
\preamble

This is a generated file.

Project: pdfrender
Version: 2019/12/29 v1.6

Copyright (C)
   2010 Heiko Oberdiek
   2016-2019 Oberdiek Package Support Group

This work may be distributed and/or modified under the
conditions of the LaTeX Project Public License, either
version 1.3c of this license or (at your option) any later
version. This version of this license is in
   https://www.latex-project.org/lppl/lppl-1-3c.txt
and the latest version of this license is in
   https://www.latex-project.org/lppl.txt
and version 1.3 or later is part of all distributions of
LaTeX version 2005/12/01 or later.

This work has the LPPL maintenance status "maintained".

The Current Maintainers of this work are
Heiko Oberdiek and the Oberdiek Package Support Group
https://github.com/ho-tex/oberdiek/issues


The Base Interpreter refers to any `TeX-Format',
because some files are installed in TDS:tex/generic//.

This work consists of the main source file pdfrender.dtx
and the derived files
   pdfrender.sty, pdfrender.pdf, pdfrender.ins, pdfrender.drv,
   pdfrender-test1.tex, pdfrender-test2.tex,
   pdfrender-test3.tex, pdfrender-test4.tex,
   pdfrender-test5.tex.

\endpreamble
\let\MetaPrefix\DoubleperCent

\generate{%
  \file{pdfrender.ins}{\from{pdfrender.dtx}{install}}%
  \file{pdfrender.drv}{\from{pdfrender.dtx}{driver}}%
  \usedir{tex/generic/oberdiek}%
  \file{pdfrender.sty}{\from{pdfrender.dtx}{package}}%
%  \usedir{doc/latex/oberdiek/test}%
%  \file{pdfrender-test1.tex}{\from{pdfrender.dtx}{test1}}%
%  \file{pdfrender-test2.tex}{\from{pdfrender.dtx}{test2}}%
%  \file{pdfrender-test3.tex}{\from{pdfrender.dtx}{test3}}%
%  \file{pdfrender-test4.tex}{\from{pdfrender.dtx}{test4}}%
%  \file{pdfrender-test5.tex}{\from{pdfrender.dtx}{test5}}%
}

\catcode32=13\relax% active space
\let =\space%
\Msg{************************************************************************}
\Msg{*}
\Msg{* To finish the installation you have to move the following}
\Msg{* file into a directory searched by TeX:}
\Msg{*}
\Msg{*     pdfrender.sty}
\Msg{*}
\Msg{* To produce the documentation run the file `pdfrender.drv'}
\Msg{* through LaTeX.}
\Msg{*}
\Msg{* Happy TeXing!}
\Msg{*}
\Msg{************************************************************************}

\endbatchfile
%</install>
%<*ignore>
\fi
%</ignore>
%<*driver>
\NeedsTeXFormat{LaTeX2e}
\ProvidesFile{pdfrender.drv}%
  [2019/12/29 v1.6 Access to some PDF graphics parameters (HO)]%
\documentclass{ltxdoc}
\usepackage{holtxdoc}[2011/11/22]
\usepackage{longtable}
\renewcommand*{\arraystretch}{1.2}
\begin{document}
  \DocInput{pdfrender.dtx}%
\end{document}
%</driver>
% \fi
%
%
%
% \GetFileInfo{pdfrender.drv}
%
% \title{The \xpackage{pdfrender} package}
% \date{2019/12/29 v1.6}
% \author{Heiko Oberdiek\thanks
% {Please report any issues at \url{https://github.com/ho-tex/oberdiek/issues}}}
%
% \maketitle
%
% \begin{abstract}
% The PDF format has some graphics parameter like
% line width or text rendering mode. This package
% provides an interface for setting these parameters.
% \end{abstract}
%
% \tableofcontents
%
% \section{Documentation}
%
% This package \xpackage{pdfrender} defines an interface for
% PDF specific parameters that affects the rendering of
% graphics or text. The interface and its implementation
% uses the same technique as package \xpackage{color}
% for color settings. Therefore this package is loaded to
% enable \LaTeX's color interface.
%
% At different places \LaTeX\ uses \cs{normalcolor} to avoid that
% header, footer or floats are print in the current color of the
% main text. \cs{setgroup@color} is used to start a save box
% with the color that is set at box saving time.
% Package \xpackage{pdfrender} extends these macros to
% add its own hooks of its parameters. Therefore \LaTeX3 should
% generalize \LaTeXe's color interface.
%
% \subsection{Usage}
%
% In \LaTeX\ the package is loaded as normal package. Options
% are not defined for this package.
% \begin{quote}
% |\usepackage{pdfrender}|
% \end{quote}
%
% This package can also be used in \plainTeX\ and even ini\TeX:
% \begin{quote}
% |input pdfrender.sty|
% \end{quote}
%
% \subsection{Macros}
%
% \begin{declcs}{pdfrender} \M{key value list}
% \end{declcs}
% The first parameter \meta{key value list} contains a
% list of parameter settings. The key entry is the parameter name.
% The macro works like
% \cs{color} (without optional argument) for color setting.
%
% \begin{declcs}{textpdfrender} \M{key value list} \M{text}
% \end{declcs}
% In the same way as \cs{pdfrender} the first argument
% specifies the parameters that should be set. This parameter
% setting affects \meta{text} only. Basically it works
% the same way as \cs{textcolor} (without optional argument).
%
% \subsection{Parameters}
%
% The following table shows an overview for the supported
% parameters and values:
%
% \begin{longtable}{@{}l|l|l@{}}
% \bfseries Parameter & \bfseries Value & \bfseries Alias\\*
% \hline
% \endhead
% \xoption{TextRenderingMode}
%   & \texttt{0} & \texttt{Fill}\\*
%   & \texttt{1} & \texttt{Stroke}\\*
%   & \texttt{2} & \texttt{FillStroke}\\*
%   & \texttt{3} & \texttt{Invisible}\\*
%   & \texttt{4} & \texttt{FillClip}\\*
%   & \texttt{5} & \texttt{StrokeClip}\\*
%   & \texttt{6} & \texttt{FillStrokeClip}\\*
%   & \texttt{7} & \texttt{Clip}\\
% \hline
% \xoption{LineWidth}
%   & \itshape positive number, & \itshape \TeX\ dimen\\*[-.2ex]
%   & \qquad \itshape unit is bp & \\
% \hline
% \xoption{LineCapStyle}
%   & \texttt{0} & \texttt{Butt}\\*
%   & \texttt{1} & \texttt{Round}\\*
%   & \texttt{2} & \texttt{ProjectingSquare}\\
% \hline
% \xoption{LineJoinStyle}
%   & \texttt{0} & \texttt{Miter}\\*
%   & \texttt{1} & \texttt{Round}\\*
%   & \texttt{2} & \texttt{Bevel}\\
% \hline
% \xoption{MiterLimit}
%   & \multicolumn{2}{l@{}}{\itshape positive number} \\
% \hline
% \xoption{Flatness}
%   & \multicolumn{2}{l@{}}{\itshape number between 0 and 100} \\
% \hline
% \xoption{LineDashPattern}
%   & \multicolumn{2}{l@{}}{^^A
%       \itshape numbers in square brackets,^^A
%     } \\*[-.2ex]
%   & \multicolumn{2}{l@{}}{^^A
%       \qquad \itshape followed by number,^^A
%     } \\*
%   & \multicolumn{2}{l@{}}{^^A
%       \qquad \itshape units are bp^^A
%     } \\
% \hline
% \xoption{RenderingIntent}
%   & \multicolumn{2}{l@{}}{\texttt{AbsoluteColorimetric}} \\*
%   & \multicolumn{2}{l@{}}{\texttt{RelativeColorimetric}} \\*
%   & \multicolumn{2}{l@{}}{\texttt{Saturation}} \\*
%   & \multicolumn{2}{l@{}}{\texttt{Perceptual}} \\
% \hline
% \xoption{FillColor}
%   & & \itshape color specification \\
% \hline
% \xoption{StrokeColor}
%   & & \itshape color specification \\
% \end{longtable}
%
% \subsubsection{Details}
%
% The description and specification of these parameters
% are available in the PDF specification \cite{pdf-spec}.
% Therefore they are not repeated here.
%
% \begin{description}
% \item[Value:]
%   The values in the second column lists or describe
%   the values that are specified by the PDF specification.
% \item[Alias:] Instead of magic numbers the package also
%   defines some aliases that can be given as value.
%   Example: |LineCapStyle=Round| has the same effect as
%   |LineCapStyle=1|.
% \item[Number:] The term \emph{number} means an integer
%   or real number. The real number is given as plain
%   decimal number without exponent. The decimal separator is a period.
%   At least one digit must be present.
% \item[\xoption{LineWidth}:] As alias a \TeX\ dimen specification
%   can be given. This includes explicit specifications with number
%   and unit, e.g. |LineWidth=0.5pt|. Also \LaTeX\ length registers
%   may be used. If \eTeX's \cs{dimexpr} is available, then it is
%   automatically added. However package \xpackage{calc} is not
%   supported.
% \item[\xoption{FillColor}, \xoption{StrokeColor}:]
%   Package \xpackage{color} or \xpackage{xcolor} must be loaded
%   before these options can be used (since version 1.2).
%   \LaTeX's color support sets both colors at the same time
%   to the same value. However parameter \xoption{TextRenderingMode}
%   offers the value \texttt{FillStroke} that makes only sense, if
%   the two color types can be set separately.
%   If one of the options \xoption{FillColor} or \xoption{StrokeColor}
%   is specified, then also the color is set. For compatibility
%   with the \LaTeX\ color packages (\xpackage{color} or \xpackage{xcolor}),
%   always both colors must be set. Thus if one of them is not specified,
%   it is taken from the current color.
%
%   Both options \xoption{FillColor} and \xoption{StrokeColor}
%   expect a \LaTeX\ color specification as value.
%   Also the optional color model argument is supported.
%   Example:
%   \begin{quote}
%     |FillColor=yellow|,\\
%     |StrokeColor=[cmyk]{1,.5,0,0}|\\
%   \end{quote}
% \end{description}
%
% \subsection{Color stack}
%
% If the \pdfTeX\ version provides color stacks, then each
% parameter is assigned a page based color stack. The assignment
% of a stack takes place, when its parameter is set the first time.
% This avoids the use of color stacks that are not needed.
%
% \StopEventually{
% }
%
% \section{Implementation}
%    \begin{macrocode}
%<*package>
%    \end{macrocode}
%    Reload check, especially if the package is not used with \LaTeX.
%    \begin{macrocode}
\begingroup\catcode61\catcode48\catcode32=10\relax%
  \catcode13=5 % ^^M
  \endlinechar=13 %
  \catcode35=6 % #
  \catcode39=12 % '
  \catcode44=12 % ,
  \catcode45=12 % -
  \catcode46=12 % .
  \catcode58=12 % :
  \catcode64=11 % @
  \catcode123=1 % {
  \catcode125=2 % }
  \expandafter\let\expandafter\x\csname ver@pdfrender.sty\endcsname
  \ifx\x\relax % plain-TeX, first loading
  \else
    \def\empty{}%
    \ifx\x\empty % LaTeX, first loading,
      % variable is initialized, but \ProvidesPackage not yet seen
    \else
      \expandafter\ifx\csname PackageInfo\endcsname\relax
        \def\x#1#2{%
          \immediate\write-1{Package #1 Info: #2.}%
        }%
      \else
        \def\x#1#2{\PackageInfo{#1}{#2, stopped}}%
      \fi
      \x{pdfrender}{The package is already loaded}%
      \aftergroup\endinput
    \fi
  \fi
\endgroup%
%    \end{macrocode}
%    Package identification:
%    \begin{macrocode}
\begingroup\catcode61\catcode48\catcode32=10\relax%
  \catcode13=5 % ^^M
  \endlinechar=13 %
  \catcode35=6 % #
  \catcode39=12 % '
  \catcode40=12 % (
  \catcode41=12 % )
  \catcode44=12 % ,
  \catcode45=12 % -
  \catcode46=12 % .
  \catcode47=12 % /
  \catcode58=12 % :
  \catcode64=11 % @
  \catcode91=12 % [
  \catcode93=12 % ]
  \catcode123=1 % {
  \catcode125=2 % }
  \expandafter\ifx\csname ProvidesPackage\endcsname\relax
    \def\x#1#2#3[#4]{\endgroup
      \immediate\write-1{Package: #3 #4}%
      \xdef#1{#4}%
    }%
  \else
    \def\x#1#2[#3]{\endgroup
      #2[{#3}]%
      \ifx#1\@undefined
        \xdef#1{#3}%
      \fi
      \ifx#1\relax
        \xdef#1{#3}%
      \fi
    }%
  \fi
\expandafter\x\csname ver@pdfrender.sty\endcsname
\ProvidesPackage{pdfrender}%
  [2019/12/29 v1.6 Access to some PDF graphics parameters (HO)]%
%    \end{macrocode}
%
%    \begin{macrocode}
\begingroup\catcode61\catcode48\catcode32=10\relax%
  \catcode13=5 % ^^M
  \endlinechar=13 %
  \catcode123=1 % {
  \catcode125=2 % }
  \catcode64=11 % @
  \def\x{\endgroup
    \expandafter\edef\csname PdfRender@AtEnd\endcsname{%
      \endlinechar=\the\endlinechar\relax
      \catcode13=\the\catcode13\relax
      \catcode32=\the\catcode32\relax
      \catcode35=\the\catcode35\relax
      \catcode61=\the\catcode61\relax
      \catcode64=\the\catcode64\relax
      \catcode123=\the\catcode123\relax
      \catcode125=\the\catcode125\relax
    }%
  }%
\x\catcode61\catcode48\catcode32=10\relax%
\catcode13=5 % ^^M
\endlinechar=13 %
\catcode35=6 % #
\catcode64=11 % @
\catcode123=1 % {
\catcode125=2 % }
\def\TMP@EnsureCode#1#2{%
  \edef\PdfRender@AtEnd{%
    \PdfRender@AtEnd
    \catcode#1=\the\catcode#1\relax
  }%
  \catcode#1=#2\relax
}
\TMP@EnsureCode{10}{12}% ^^J
\TMP@EnsureCode{36}{3}% $
\TMP@EnsureCode{39}{12}% '
\TMP@EnsureCode{40}{12}% (
\TMP@EnsureCode{41}{12}% )
\TMP@EnsureCode{42}{12}% *
\TMP@EnsureCode{43}{12}% +
\TMP@EnsureCode{44}{12}% ,
\TMP@EnsureCode{45}{12}% -
\TMP@EnsureCode{46}{12}% .
\TMP@EnsureCode{47}{12}% /
\TMP@EnsureCode{58}{12}% :
\TMP@EnsureCode{59}{12}% ;
\TMP@EnsureCode{60}{12}% <
\TMP@EnsureCode{62}{12}% >
\TMP@EnsureCode{63}{12}% ?
\TMP@EnsureCode{91}{12}% [
\TMP@EnsureCode{93}{12}% ]
\TMP@EnsureCode{94}{7}% ^ (superscript)
\TMP@EnsureCode{96}{12}% `
\TMP@EnsureCode{124}{12}% |
%    \end{macrocode}
%
%    \begin{macrocode}
\def\PdfRender@AtEndHook{}
\expandafter\def\expandafter\PdfRender@AtEnd\expandafter{%
  \expandafter\PdfRender@AtEndHook
  \PdfRender@AtEnd
  \endinput
}
%    \end{macrocode}
%
% \subsection{Look for \pdfTeX, its mode and features}
%
%    \begin{macro}{\PdfRender@newif}
%    \begin{macrocode}
\def\PdfRender@newif#1{%
  \expandafter\edef\csname PdfRender@#1true\endcsname{%
    \let
    \expandafter\noexpand\csname ifPdfRender@#1\endcsname
    \noexpand\iftrue
  }%
  \expandafter\edef\csname PdfRender@#1false\endcsname{%
    \let
    \expandafter\noexpand\csname ifPdfRender@#1\endcsname
    \noexpand\iffalse
  }%
  \csname PdfRender@#1false\endcsname
}
%    \end{macrocode}
%    \end{macro}
%
%    \begin{macro}{\ifPdfRender@Stack}
%    \begin{macrocode}
\PdfRender@newif{Stack}
%    \end{macrocode}
%    \end{macro}
%    \begin{macro}{\ifPdfRender@Match}
%    \begin{macrocode}
\PdfRender@newif{Match}
%    \end{macrocode}
%    \end{macro}
%
%    \begin{macro}{\PdfRender@RequirePackage}
%    \begin{macrocode}
\begingroup\expandafter\expandafter\expandafter\endgroup
\expandafter\ifx\csname RequirePackage\endcsname\relax
  \def\PdfRender@RequirePackage#1[#2]{%
    \expandafter\def\expandafter\PdfRender@AtEndHook\expandafter{%
      \PdfRender@AtEndHook
      \ltx@ifpackagelater{#1}{#2}{}{%
        \@PackageWarningNoLine{pdfrender}{%
          You have requested version\MessageBreak
          `#2' of package `#1',\MessageBreak
          but only version\MessageBreak
          `\csname ver@#1.\ltx@pkgextension\endcsname'\MessageBreak
          is available%
        }%
      }%
    }%
    \input #1.sty\relax
  }%
\else
  \let\PdfRender@RequirePackage\RequirePackage
\fi
%    \end{macrocode}
%    \end{macro}
% Luatex compatibility
%    \begin{macrocode}
\ifx\pdfextension\@undefined\else
  \def\pdfcolorstackinit {\pdffeedback colorstackinit}
  \protected\def\pdfcolorstack     {\pdfextension colorstack}
  \protected\def\pdfliteral        {\pdfextension literal}
\fi
%    \end{macrocode}
%
%    \begin{macrocode}
\PdfRender@RequirePackage{iftex}[2019/11/07]
\PdfRender@RequirePackage{infwarerr}[2007/09/09]
\PdfRender@RequirePackage{ltxcmds}[2010/01/28]
%    \end{macrocode}
%
%    \begin{macrocode}
\ifpdf
  \ltx@IfUndefined{pdfcolorstackinit}{%
    \@PackageWarning{pdfrender}{%
      Missing \string\pdfcolorstackinit
    }%
  }{%
    \PdfRender@Stacktrue
  }%
  \ltx@IfUndefined{pdfmatch}{%
    \@PackageInfoNoLine{pdfrender}{%
      \string\pdfmatch\ltx@space not found. %
      Therefore the values\MessageBreak
      of some parameters are not validated%
    }%
  }{%
    \PdfRender@Matchtrue
  }%
\else
  \@PackageWarning{pdfrender}{%
    Missing pdfTeX in PDF mode%
  }%
%    \end{macrocode}
%    \begin{macrocode}
  \ltx@IfUndefined{newcommand}{%
%    \end{macrocode}
%    \begin{macro}{\pdfrender}
%    \begin{macrocode}
    \def\pdfrender#1{}%
%    \end{macrocode}
%    \end{macro}
%    \begin{macro}{\textpdfrender}
%    \begin{macrocode}
    \long\def\textpdfrender#1#2{#2}%
%    \end{macrocode}
%    \end{macro}
%    \begin{macrocode}
  }{%
%    \end{macrocode}
%    \begin{macro}{\pdfrender}
%    \begin{macrocode}
    \newcommand*{\pdfrender}[1]{}%
%    \end{macrocode}
%    \end{macro}
%    \begin{macro}{\textpdfrender}
%    \begin{macrocode}
    \newcommand{\textpdfrender}[2]{#2}%
%    \end{macrocode}
%    \end{macro}
%    \begin{macrocode}
  }%
  \expandafter\PdfRender@AtEnd
\fi%
%    \end{macrocode}
%
% \subsection{Enable color support of \LaTeX}
%
%    \begin{macrocode}
\ltx@ifpackageloaded{color}{}{%
  \def\color@setgroup{\begingroup\set@color}%
  \let\color@begingroup\begingroup
  \def\color@endgroup{\endgraf\endgroup}%
  \def\color@hbox{\hbox\bgroup\color@begingroup}%
  \def\color@vbox{\vbox\bgroup\color@begingroup}%
  \def\color@endbox{\color@endgroup\egroup}%
  \ltx@ifundefined{bgroup}{%
    \let\bgroup={\let\egroup=}%
  }{}%
  \ltx@ifundefined{endgraf}{%
    \let\endgraf=\par
  }{}%
}
%    \end{macrocode}
%
% \subsection{Hook into \cs{normalcolor}}
%
%    The problem is that packages \xpackage{color} and \xpackage{xcolor}
%    each overwrite \cs{normalcolor}. For example, after the package
%    loading order \xpackage{color}, \xpackage{pdfrender} and \xpackage{xcolor}
%    the patched version of \cs{normalcolor} is overwritten by
%    package \xpackage{xcolor}. Also using \cs{AtBeginDocument} for
%    patching is not enough. If package \xpackage{hyperref} is loaded
%    later, it might load package \xpackage{color} using \cs{AtBeginDocument}.
%    \begin{macro}{\PdfRender@NormalColorHook}
%    \begin{macrocode}
\def\PdfRender@NormalColorHook{}
%    \end{macrocode}
%    \end{macro}
%    \begin{macro}{\PdfRender@ColorSetGroupHook}
%    \begin{macrocode}
\def\PdfRender@ColorSetGroupHook{}
%    \end{macrocode}
%    \end{macro}
%    \begin{macro}{\PdfRender@TestBox}
%    \begin{macrocode}
\def\PdfRender@TestBox#1{%
  \setbox0=\color@hbox#1\color@endbox
}
%    \end{macrocode}
%    \end{macro}
%    \begin{macro}{\PdfRender@PatchNormalColor}
%    \begin{macrocode}
\def\PdfRender@PatchNormalColor{%
  \ltx@ifundefined{normalcolor}{%
    \gdef\normalcolor{\PdfRender@NormalColorHook}%
  }{%
    \begingroup
      \def\PdfRender@NormalColorHook{\let\PdfRender@temp=Y}%
      \PdfRender@TestBox{%
        \let\set@color\relax
        \normalcolor
        \ifx\PdfRender@temp Y%
        \else
          \ltx@GlobalAppendToMacro\normalcolor{%
            \PdfRender@NormalColorHook
          }%
        \fi
      }%
    \endgroup
  }%
  \ifx\@nodocument\relax
    \global\let\PdfRender@PatchNormalColor\relax
  \fi
}%
%    \end{macrocode}
%    \end{macro}
%    \begin{macro}{\PdfRender@PatchColorSetGroup}
%    \begin{macrocode}
\def\PdfRender@PatchColorSetGroup{%
  \begingroup
    \def\PdfRender@ColorSetGroupHook{\let\PdfRender@temp=Y}%
    \PdfRender@TestBox{%
      \let\set@color\relax
      \color@setgroup\color@endgroup
      \ifx\PdfRender@temp Y%
      \else
        \ltx@GlobalAppendToMacro\color@setgroup{%
          \PdfRender@ColorSetGroupHook
        }%
      \fi
    }%
  \endgroup
  \ifx\@nodocument\relax
    \global\let\PdfRender@PatchColorSetGroup\relax
  \fi
}%
%    \end{macrocode}
%    \end{macro}
%    \begin{macro}{\PdfRender@PatchColor}
%    \begin{macrocode}
\def\PdfRender@PatchColor{%
  \PdfRender@PatchNormalColor
  \PdfRender@PatchColorSetGroup
}
%    \end{macrocode}
%    \end{macro}
%    \begin{macrocode}
\PdfRender@PatchColor
\ltx@IfUndefined{AtBeginDocument}{}{%
  \AtBeginDocument{\PdfRender@PatchColor}%
}
%    \end{macrocode}
%    \cs{AfterPackage} is provided by package \xpackage{scrlfile}.
%    \begin{macrocode}
\ltx@IfUndefined{AfterPackage}{%
}{%
  \AfterPackage{color}{\PdfRender@PatchColor}%
  \AfterPackage{xcolor}{\PdfRender@PatchColor}%
  \AfterPackage{etoolbox}{%
    \AfterEndPreamble{\PdfRender@PatchColor}%
  }%
}%
%    \end{macrocode}
%    \cs{AfterEndPreamble} is provided by package \xpackage{etoolbox}.
%    \begin{macrocode}
\ltx@IfUndefined{AfterEndPreamble}{%
}{%
  \AfterEndPreamble{\PdfRender@PatchColor}%
}%
%    \end{macrocode}
%
%    \begin{macrocode}
\PdfRender@RequirePackage{kvsetkeys}[2010/01/28]
%    \end{macrocode}
%    \begin{macro}{\PdfRender@texorpdfstring}
%    \begin{macrocode}
\def\PdfRender@texorpdfstring{%
  \ltx@IfUndefined{texorpdfstring}\ltx@firstoftwo\texorpdfstring
}
%    \end{macrocode}
%    \end{macro}
%    \begin{macro}{\pdfrender}
%    \begin{macrocode}
\ltx@IfUndefined{DeclareRobustCommand}%
\ltx@firstoftwo\ltx@secondoftwo
{%
  \def\pdfrender#1%
}{%
  \newcommand{\pdfrender}{}%
  \DeclareRobustCommand*{\pdfrender}[1]%
}%
{%
  \PdfRender@texorpdfstring{%
    \PdfRender@PatchNormalColor
    \global\let\PdfRender@FillColor\ltx@empty
    \global\let\PdfRender@StrokeColor\ltx@empty
    \kvsetkeys{PDFRENDER}{#1}%
    \PdfRender@SetColor
  }{}%
}
%    \end{macrocode}
%    \end{macro}
%    \begin{macro}{\textpdfrender}
%    \begin{macrocode}
\ltx@IfUndefined{DeclareRobustCommand}%
\ltx@firstoftwo\ltx@secondoftwo
{%
  \long\def\textpdfrender#1#2%
}{%
  \newcommand{\textpdfrender}{}%
  \DeclareRobustCommand{\textpdfrender}[2]%
}%
{%
  \PdfRender@texorpdfstring{%
    \begingroup
      \pdfrender{#1}%
      #2%
    \endgroup
  }{#2}%
}
%    \end{macrocode}
%    \end{macro}
%
%    \begin{macro}{\ifPdfRender@Values}
%    \begin{macrocode}
\PdfRender@newif{Values}
%    \end{macrocode}
%    \end{macro}
%    \begin{macro}{\PdfRender@NewClassValues}
%    \begin{macrocode}
\def\PdfRender@NewClassValues#1#2#3#4{%
  \PdfRender@Valuestrue
  \PdfRender@NewClass{#1}{#2}{#3}{#4}{}%
}
%    \end{macrocode}
%    \end{macro}
%    \begin{macro}{\PdfRender@NewClass}
%    \begin{macrocode}
\def\PdfRender@NewClass#1#2#3#4#5{%
  \PdfRender@newif{Active#1}%
  \expandafter\def\csname PdfRender@Default#1\endcsname{#2}%
  \expandafter\let\csname PdfRender@Current#1\expandafter\endcsname
      \csname PdfRender@Default#1\endcsname
  \ifPdfRender@Stack
    \expandafter\edef\csname PdfRender@Init#1\endcsname{%
      \global\chardef
      \expandafter\noexpand\csname PdfRender@Stack#1\endcsname=%
          \noexpand\pdfcolorstackinit page direct{%
            \noexpand#3%
            \expandafter\noexpand\csname PdfRender@Default#1\endcsname
          }\relax
      \noexpand\@PackageInfo{pdfrender}{%
        New color stack `#1' = \noexpand\number
        \expandafter\noexpand\csname PdfRender@Stack#1\endcsname
      }%
      \gdef\expandafter\noexpand\csname PdfRender@Init#1\endcsname{}%
    }%
    \expandafter\edef\csname PdfRender@Set#1\endcsname{%
      \expandafter\noexpand\csname PdfRender@Init#1\endcsname
      \noexpand\pdfcolorstack
      \expandafter\noexpand\csname PdfRender@Stack#1\endcsname
      push{%
        #3{\expandafter\noexpand\csname PdfRender@Current#1\endcsname}%
      }%
      \noexpand\aftergroup
      \expandafter\noexpand\csname PdfRender@Reset#1\endcsname
    }%
    \expandafter\edef\csname PdfRender@Reset#1\endcsname{%
      \expandafter\noexpand\csname PdfRender@Init#1\endcsname
      \noexpand\pdfcolorstack
      \expandafter\noexpand\csname PdfRender@Stack#1\endcsname
      pop\relax
    }%
  \else
    \expandafter\edef\csname PdfRender@Set#1\endcsname{%
      \noexpand\pdfliteral direct{%
        #3{\expandafter\noexpand\csname PdfRender@Current#1\endcsname}%
      }%
      \noexpand\aftergroup
      \expandafter\noexpand\csname PdfRender@Reset#1\endcsname
    }%
    \expandafter\edef\csname PdfRender@Reset#1\endcsname{%
      \noexpand\pdfliteral direct{%
        #3{\expandafter\noexpand\csname PdfRender@Current#1\endcsname}%
      }%
    }%
  \fi
  \expandafter\edef\csname PdfRender@Normal#1\endcsname{%
    \let
    \expandafter\noexpand\csname PdfRender@Current#1\endcsname
    \expandafter\noexpand\csname PdfRender@Default#1\endcsname
    \noexpand\PdfRender@Set{#1}%
  }%
  \expandafter\ltx@GlobalAppendToMacro\expandafter\PdfRender@NormalColorHook
  \expandafter{%
    \csname PdfRender@Normal#1\endcsname
  }%
  \ltx@GlobalAppendToMacro\PdfRender@ColorSetGroupHook{%
    \PdfRender@Set{#1}%
  }%
  \ifPdfRender@Values
    \kv@parse@normalized{#4}{%
      \expandafter\let\csname PdfRender@#1@\kv@key\endcsname\kv@key
      \ifx\kv@value\relax
      \else
        \expandafter\let\csname PdfRender@#1@\kv@value\endcsname\kv@key
      \fi
      \ltx@gobbletwo
    }%
    \PdfRender@define@key{PDFRENDER}{#1}{%
      \global\csname PdfRender@Active#1true\endcsname
      \def\PdfRender@Current{##1}%
      \PdfRender@SetValidateValues{#1}%
    }%
    \PdfRender@Valuesfalse
  \else
    \PdfRender@define@key{PDFRENDER}{#1}{%
      \global\csname PdfRender@Active#1true\endcsname
      \expandafter\def\csname PdfRender@Current#1\endcsname{##1}%
      \ltx@IfUndefined{PdfRender@PostProcess#1}{%
      }{%
        \csname PdfRender@PostProcess#1\endcsname
      }%
      \PdfRender@SetValidate{#1}{#4}{#5}%
    }%
  \fi
}%
%    \end{macrocode}
%    \end{macro}
%    \begin{macro}{\PdfRender@define@key}
%    \begin{macrocode}
\ltx@IfUndefined{define@key}{%
  \def\PdfRender@define@key#1#2{%
    \expandafter\def\csname KV@#1@#2\endcsname##1%
  }%
}{%
  \let\PdfRender@define@key\define@key
}
%    \end{macrocode}
%    \end{macro}
%    \begin{macro}{\PdfRender@Set}
%    \begin{macrocode}
\def\PdfRender@Set#1{%
  \csname ifPdfRender@Active#1\endcsname
    \csname PdfRender@Set#1\expandafter\endcsname
  \fi
}
%    \end{macrocode}
%    \end{macro}
%    \begin{macro}{\PdfRender@Reset}
%    \begin{macrocode}
\def\PdfRender@Reset#1{%
  \csname ifPdfRender@Active#1\endcsname
    \csname PdfRender@Reset#1\expandafter\endcsname
  \fi
}
%    \end{macrocode}
%    \end{macro}
%    \begin{macro}{\PdfRender@ErrorInvalidValue}
%    \begin{macrocode}
\def\PdfRender@ErrorInvalidValue#1{%
  \PackageError{pdfrender}{%
    Ignoring parameter setting for `#1'\MessageBreak
    because of invalid value %
    `\csname PdfRender@Current#1\endcsname'%
  }\@ehc
  \expandafter\let\csname PdfRender@Current#1\endcsname\ltx@empty
}%
%    \end{macrocode}
%    \end{macro}
%    \begin{macro}{\PdfRender@SetValidate}
%    \begin{macrocode}
\ifPdfRender@Match
  \def\PdfRender@SetValidate#1#2#3{%
    \ifnum\pdfmatch{^(#2)$}{\csname PdfRender@Current#1\endcsname}=1 %
      \csname PdfRender@Set#1\expandafter\endcsname
    \else
      \PdfRender@ErrorInvalidValue{#1}%
    \fi
  }%
\else
  \def\PdfRender@SetValidate#1#2#3{%
    \expandafter\let\expandafter\PdfRender@Current
    \csname PdfRender@Current#1\endcsname
    #3%
    \ifx\PdfRender@Current\@empty
      \PdfRender@ErrorInvalidValue{#1}%
    \else
      \csname PdfRender@Set#1\expandafter\endcsname
    \fi
  }%
\fi
%    \end{macrocode}
%    \end{macro}
%    \begin{macro}{\PdfRender@SetValidateValues}
%    \begin{macrocode}
\def\PdfRender@SetValidateValues#1{%
  \ltx@IfUndefined{PdfRender@#1@\PdfRender@Current}{%
    \expandafter\let\csname PdfRender@Current#1\endcsname
                    \PdfRender@Current
    \PdfRender@ErrorInvalidValue{#1}%
  }{%
    \expandafter\edef\csname PdfRender@Current#1\endcsname{%
      \csname PdfRender@#1@\PdfRender@Current\endcsname
    }%
    \csname PdfRender@Set#1\endcsname
  }%
}
%    \end{macrocode}
%    \end{macro}
%    \begin{macro}{\PdfRender@OpValue}
%    \begin{macrocode}
\def\PdfRender@OpValue#1#2{#2\ltx@space#1}%
%    \end{macrocode}
%    \end{macro}
%    \begin{macro}{\PdfRender@OpName}
%    \begin{macrocode}
\def\PdfRender@OpName#1#2{/#2\ltx@space#1}%
%    \end{macrocode}
%    \end{macro}
%
% \subsection{Declare and setup parameters}
%
%    \begin{macrocode}
\PdfRender@NewClassValues{TextRenderingMode}%
                      {0}%
                      {\PdfRender@OpValue{Tr}}{%
  0=Fill,%
  1=Stroke,%
  2=FillStroke,%
  3=Invisible,%
  4=FillClip,%
  5=StrokeClip,%
  6=FillStrokeClip,%
  7=Clip,%
}%
\PdfRender@NewClass{LineWidth}{1}{\PdfRender@OpValue{w}}{%
  [0-9]+\string\.?[0-9]*|\string\.[0-9]+%
}{}%
\ltx@IfUndefined{dimexpr}{%
  \def\PdfRender@dimexpr{}%
}{%
  \let\PdfRender@dimexpr\dimexpr
}
\def\PdfRender@PostProcessLineWidth{%
  \begingroup
  \afterassignment\PdfRender@@PostProcessLineWidth
  \dimen0=\PdfRender@dimexpr\PdfRender@CurrentLineWidth bp %
  \PdfRender@let\PdfRender@relax\PdfRender@relax
}
\let\PdfRender@let\let
\let\PdfRender@relax\relax
\def\PdfRender@@PostProcessLineWidth#1\PdfRender@let{%
  \ifx\\#1\\%
    \endgroup
  \else
    \dimen0=.996264\dimen0 % 72/72.27
    \edef\x{\endgroup
      \def\noexpand\PdfRender@CurrentLineWidth{%
        \strip@pt\dimen0%
      }%
    }%
    \expandafter\x
  \fi
}
\PdfRender@NewClassValues{LineCapStyle}{0}{\PdfRender@OpValue{J}}{%
  0=Butt,%
  1=Round,%
  2=ProjectingSquare,%
}%
\PdfRender@NewClassValues{LineJoinStyle}{0}{\PdfRender@OpValue{j}}{%
  0=Miter,%
  1=Round,%
  2=Bevel,%
}%
\PdfRender@NewClass{MiterLimit}{10}{\PdfRender@OpValue{M}}{%
  [0-9]*[1-9][0-9]*\string\.?[0-9]*|%
  [0-9]*\string\.?[0-9]*[1-9][0-9]*%
}{}%
\PdfRender@NewClass{Flatness}{0}{\PdfRender@OpValue{i}}{%
  100(\string\.0*)?|[0-9][0-9](\string\.[0-9]*)?|\string\.[0-9]+%
}{}%
\PdfRender@NewClass{LineDashPattern}{[]0}{\PdfRender@OpValue{d}}{%
  \string\[%
  ( ?([0-9]+\string\.?[0-9]*|\string\.[0-9]+) ?)*%
  \string\] ?%
  ([0-9]+\string\.?[0-9]*|\string\.[0-9]+)%
}{}%
\PdfRender@NewClassValues{RenderingIntent}%
                      {RelativeColorimetric}%
                      {\PdfRender@OpName{ri}}{%
  AbsoluteColorimetric,%
  RelativeColorimetric,%
  Saturation,%
  Perceptual,%
}%
%    \end{macrocode}
%
% \subsection{Fill and stroke color support}
%
%    \begin{macrocode}
\PdfRender@define@key{PDFRENDER}{FillColor}{%
  \begingroup
    \def\PdfRender@Color{#1}%
    \ifx\PdfRender@Color\ltx@empty
      \global\let\PdfRender@FillColor\ltx@empty
    \else
      \PdfRender@ColorAvailable{%
        \PdfRender@TestBox{%
          \expandafter\PdfRender@TryColor\PdfRender@Color\ltx@empty
          \PdfRender@GetFillColor
          \ifx\PdfRender@FillColor\ltx@empty
            \@PackageWarning{pdfrender}{%
              Cannot extract fill color\MessageBreak
              from value `#1'%
            }%
          \fi
        }%
      }%
    \fi
  \endgroup
}
\PdfRender@define@key{PDFRENDER}{StrokeColor}{%
  \begingroup
    \def\PdfRender@Color{#1}%
    \ifx\PdfRender@Color\ltx@empty
      \global\let\PdfRender@StrokeColor\ltx@empty
    \else
      \PdfRender@ColorAvailable{%
        \PdfRender@TestBox{%
          \expandafter\PdfRender@TryColor\PdfRender@Color\ltx@empty
          \PdfRender@GetStrokeColor
          \ifx\PdfRender@StrokeColor\ltx@empty
            \@PackageWarning{pdfrender}{%
              Cannot extract stroke color\MessageBreak
              from value `#1'%
            }%
          \fi
        }%
      }%
    \fi
  \endgroup
}
%    \end{macrocode}
%    \begin{macro}{\PdfRender@ColorAvailable}
%    \begin{macrocode}
\def\PdfRender@ColorAvailable{%
  \@ifundefined{set@color}{%
    \@PackageError{pdfrender}{%
      Ignoring color options, because neither\MessageBreak
      package `color' nor package `xcolor' is loaded%
    }\@ehc
    \global\let\PdfRender@ColorAvailable\ltx@gobble
  }{%
    \global\let\PdfRender@ColorAvailable\ltx@firstofone
  }%
  \PdfRender@ColorAvailable
}
%    \end{macrocode}
%    \end{macro}
%    \begin{macro}{\PdfRender@TryColor}
%    \begin{macrocode}
\def\PdfRender@TryColor{%
  \@ifnextchar[\color\PdfRender@@TryColor
}
%    \end{macrocode}
%    \end{macro}
%    \begin{macro}{\PdfRender@@TryColor}
%    \begin{macrocode}
\def\PdfRender@@TryColor#1\ltx@empty{%
  \expandafter\color\expandafter{\PdfRender@Color}%
}
%    \end{macrocode}
%    \end{macro}
%    \begin{macro}{\PdfRender@SetColor}
%    \begin{macrocode}
\def\PdfRender@SetColor{%
  \chardef\PdfRender@NeedsCurrentColor=0 %
  \ifx\PdfRender@FillColor\ltx@empty
    \ifx\PdfRender@StrokeColor\ltx@empty
    \else
      \edef\PdfRender@CurrentColor{%
        \noexpand\PdfRender@FillColor\ltx@space\PdfRender@StrokeColor
      }%
      \chardef\PdfRender@NeedsCurrentColor=1 %
    \fi
  \else
    \ifx\PdfRender@StrokeColor\ltx@empty
      \edef\PdfRender@CurrentColor{%
        \PdfRender@FillColor\ltx@space\noexpand\PdfRender@StrokeColor
      }%
      \chardef\PdfRender@NeedsCurrentColor=2 %
    \else
      \edef\current@color{%
        \PdfRender@FillColor\ltx@space\PdfRender@StrokeColor
      }%
      \set@color
    \fi
  \fi
  \ifnum\PdfRender@NeedsCurrentColor=1 %
    \PdfRender@GetFillColor
    \ifx\PdfRender@FillColor\ltx@empty
      \@PackageWarning{pdfrender}{%
        Cannot extract current fill color%
      }%
    \else
      \edef\current@color{\PdfRender@CurrentColor}%
      \set@color
    \fi
  \else
    \ifnum\PdfRender@NeedsCurrentColor=2 %
      \PdfRender@GetStrokeColor
      \ifx\PdfRender@StrokeColor\ltx@empty
        \@PackageWarning{pdfrender}{%
          Cannot extract current stroke color%
        }%
      \else
        \edef\current@color{\PdfRender@CurrentColor}%
        \set@color
      \fi
    \fi
  \fi
}
%    \end{macrocode}
%    \end{macro}
%    \begin{macro}{\PdfRender@PatternFillColor}
%    \begin{macrocode}
\edef\PdfRender@PatternFillColor{ % space
  (%
    [0-9\string\.]+ g|%
    [0-9\string\.]+ [0-9\string\.]+ [0-9\string\.]+ rg|%
    [0-9\string\.]+ [0-9\string\.]+ %
    [0-9\string\.]+ [0-9\string\.]+ k%
  ) % space
  (.*)$%
}
%    \end{macrocode}
%    \end{macro}
%    \begin{macro}{\PdfRender@PatternStrokeColor}
%    \begin{macrocode}
\edef\PdfRender@PatternStrokeColor{ % space
  (%
    [0-9\string\.]+ G|%
    [0-9\string\.]+ [0-9\string\.]+ [0-9\string\.]+ RG|%
    [0-9\string\.]+ [0-9\string\.]+ %
    [0-9\string\.]+ [0-9\string\.]+ K%
  ) % space
  (.*)$%
}
%    \end{macrocode}
%    \end{macro}
%    \begin{macro}{\PdfRender@MatchPattern}
%    \begin{macrocode}
\def\PdfRender@MatchPattern#1{%
  \ifnum\pdfmatch{\PdfRender@Pattern}{\PdfRender@String}=1 %
    \xdef#1{%
      \expandafter\strip@prefix\pdflastmatch 1%
    }%
    \edef\PdfRender@String{%
      \expandafter\strip@prefix\pdflastmatch 2%
    }%
    \ifx\PdfRender@String\ltx@empty
    \else
      \expandafter\expandafter\expandafter\PdfRender@MatchPattern
      \expandafter\expandafter\expandafter#1%
    \fi
  \fi
}
%    \end{macrocode}
%    \end{macro}
%    \begin{macro}{\PdfRender@GetFillColor}
%    \begin{macrocode}
\def\PdfRender@GetFillColor{%
  \global\let\PdfRender@FillColor\ltx@empty
  \begingroup
    \ifPdfRender@Match
      \let\PdfRender@Pattern\PdfRender@PatternFillColor
      \edef\PdfRender@String{\ltx@space\current@color\ltx@space}%
      \PdfRender@MatchPattern\PdfRender@FillColor
    \else
      \edef\current@color{\current@color\ltx@space}%
      \let\PdfRender@OP\relax
      \PdfRender@FindOp{g}0%
      \PdfRender@FindOp{G}1%
      \PdfRender@FindOp{rg}0%
      \PdfRender@FindOp{RG}1%
      \PdfRender@FindOp{k}0%
      \PdfRender@FindOp{K}1%
      \PdfRender@FilterOp 0\PdfRender@FillColor
    \fi
  \endgroup
}
%    \end{macrocode}
%    \end{macro}
%    \begin{macro}{\PdfRender@GetStrokeColor}
%    \begin{macrocode}
\def\PdfRender@GetStrokeColor{%
  \global\let\PdfRender@StrokeColor\ltx@empty
  \begingroup
    \ifPdfRender@Match
      \let\PdfRender@Pattern\PdfRender@PatternStrokeColor
      \edef\PdfRender@String{\ltx@space\current@color\ltx@space}%
      \PdfRender@MatchPattern\PdfRender@StrokeColor
    \else
      \edef\current@color{\current@color\ltx@space}%
      \let\PdfRender@OP\relax
      \PdfRender@FindOp{g}0%
      \PdfRender@FindOp{G}1%
      \PdfRender@FindOp{rg}0%
      \PdfRender@FindOp{RG}1%
      \PdfRender@FindOp{k}0%
      \PdfRender@FindOp{K}1%
      \PdfRender@FilterOp 1\PdfRender@StrokeColor
    \fi
  \endgroup
}
%    \end{macrocode}
%    \end{macro}
%    \begin{macrocode}
\ifPdfRender@Match
  \expandafter\PdfRender@AtEnd
\fi%
%    \end{macrocode}
%    \begin{macro}{\PdfRender@FindOp}
%    \begin{macrocode}
\def\PdfRender@FindOp#1#2{%
  \def\PdfRender@temp##1 #1 ##2\@nil{%
    ##1%
    \ifx\\##2\\%
      \expandafter\@gobble
    \else
      \PdfRender@OP{#1}#2%
      \expandafter\@firstofone
    \fi
    {%
      \PdfRender@temp##2\@nil
    }%
  }%
  \edef\current@color{%
    \@firstofone{\expandafter\PdfRender@temp\current@color} #1 \@nil
  }%
}
%    \end{macrocode}
%    \end{macro}
%    \begin{macro}{\PdfRender@FilterOp}
%    \begin{macrocode}
\def\PdfRender@FilterOp#1#2{%
  \expandafter\PdfRender@@FilterOp\expandafter#1\expandafter#2%
      \current@color\PdfRender@OP{}{}%
}
%    \end{macrocode}
%    \end{macro}
%    \begin{macro}{\PdfRender@@FilterOp}
%    \begin{macrocode}
\def\PdfRender@@FilterOp#1#2#3\PdfRender@OP#4#5{%
  \ifx\\#4#5\\%
  \else
    \ifnum#1=#5 %
      \xdef#2{#3 #4}%
    \fi
    \expandafter\PdfRender@@FilterOp\expandafter#1\expandafter#2%
  \fi
}
%    \end{macrocode}
%    \end{macro}
%
%    \begin{macrocode}
\PdfRender@AtEnd%
%</package>
%    \end{macrocode}
%% \section{Installation}
%
% \subsection{Download}
%
% \paragraph{Package.} This package is available on
% CTAN\footnote{\CTANpkg{pdfrender}}:
% \begin{description}
% \item[\CTAN{macros/latex/contrib/oberdiek/pdfrender.dtx}] The source file.
% \item[\CTAN{macros/latex/contrib/oberdiek/pdfrender.pdf}] Documentation.
% \end{description}
%
%
% \paragraph{Bundle.} All the packages of the bundle `oberdiek'
% are also available in a TDS compliant ZIP archive. There
% the packages are already unpacked and the documentation files
% are generated. The files and directories obey the TDS standard.
% \begin{description}
% \item[\CTANinstall{install/macros/latex/contrib/oberdiek.tds.zip}]
% \end{description}
% \emph{TDS} refers to the standard ``A Directory Structure
% for \TeX\ Files'' (\CTANpkg{tds}). Directories
% with \xfile{texmf} in their name are usually organized this way.
%
% \subsection{Bundle installation}
%
% \paragraph{Unpacking.} Unpack the \xfile{oberdiek.tds.zip} in the
% TDS tree (also known as \xfile{texmf} tree) of your choice.
% Example (linux):
% \begin{quote}
%   |unzip oberdiek.tds.zip -d ~/texmf|
% \end{quote}
%
% \subsection{Package installation}
%
% \paragraph{Unpacking.} The \xfile{.dtx} file is a self-extracting
% \docstrip\ archive. The files are extracted by running the
% \xfile{.dtx} through \plainTeX:
% \begin{quote}
%   \verb|tex pdfrender.dtx|
% \end{quote}
%
% \paragraph{TDS.} Now the different files must be moved into
% the different directories in your installation TDS tree
% (also known as \xfile{texmf} tree):
% \begin{quote}
% \def\t{^^A
% \begin{tabular}{@{}>{\ttfamily}l@{ $\rightarrow$ }>{\ttfamily}l@{}}
%   pdfrender.sty & tex/generic/oberdiek/pdfrender.sty\\
%   pdfrender.pdf & doc/latex/oberdiek/pdfrender.pdf\\
%   pdfrender.dtx & source/latex/oberdiek/pdfrender.dtx\\
% \end{tabular}^^A
% }^^A
% \sbox0{\t}^^A
% \ifdim\wd0>\linewidth
%   \begingroup
%     \advance\linewidth by\leftmargin
%     \advance\linewidth by\rightmargin
%   \edef\x{\endgroup
%     \def\noexpand\lw{\the\linewidth}^^A
%   }\x
%   \def\lwbox{^^A
%     \leavevmode
%     \hbox to \linewidth{^^A
%       \kern-\leftmargin\relax
%       \hss
%       \usebox0
%       \hss
%       \kern-\rightmargin\relax
%     }^^A
%   }^^A
%   \ifdim\wd0>\lw
%     \sbox0{\small\t}^^A
%     \ifdim\wd0>\linewidth
%       \ifdim\wd0>\lw
%         \sbox0{\footnotesize\t}^^A
%         \ifdim\wd0>\linewidth
%           \ifdim\wd0>\lw
%             \sbox0{\scriptsize\t}^^A
%             \ifdim\wd0>\linewidth
%               \ifdim\wd0>\lw
%                 \sbox0{\tiny\t}^^A
%                 \ifdim\wd0>\linewidth
%                   \lwbox
%                 \else
%                   \usebox0
%                 \fi
%               \else
%                 \lwbox
%               \fi
%             \else
%               \usebox0
%             \fi
%           \else
%             \lwbox
%           \fi
%         \else
%           \usebox0
%         \fi
%       \else
%         \lwbox
%       \fi
%     \else
%       \usebox0
%     \fi
%   \else
%     \lwbox
%   \fi
% \else
%   \usebox0
% \fi
% \end{quote}
% If you have a \xfile{docstrip.cfg} that configures and enables \docstrip's
% TDS installing feature, then some files can already be in the right
% place, see the documentation of \docstrip.
%
% \subsection{Refresh file name databases}
%
% If your \TeX~distribution
% (\TeX\,Live, \mikTeX, \dots) relies on file name databases, you must refresh
% these. For example, \TeX\,Live\ users run \verb|texhash| or
% \verb|mktexlsr|.
%
% \subsection{Some details for the interested}
%
% \paragraph{Unpacking with \LaTeX.}
% The \xfile{.dtx} chooses its action depending on the format:
% \begin{description}
% \item[\plainTeX:] Run \docstrip\ and extract the files.
% \item[\LaTeX:] Generate the documentation.
% \end{description}
% If you insist on using \LaTeX\ for \docstrip\ (really,
% \docstrip\ does not need \LaTeX), then inform the autodetect routine
% about your intention:
% \begin{quote}
%   \verb|latex \let\install=y% \iffalse meta-comment
%
% File: pdfrender.dtx
% Version: 2019/12/29 v1.6
% Info: Access to some PDF graphics parameters
%
% Copyright (C)
%    2010 Heiko Oberdiek
%    2016-2019 Oberdiek Package Support Group
%    https://github.com/ho-tex/oberdiek/issues
%
% This work may be distributed and/or modified under the
% conditions of the LaTeX Project Public License, either
% version 1.3c of this license or (at your option) any later
% version. This version of this license is in
%    https://www.latex-project.org/lppl/lppl-1-3c.txt
% and the latest version of this license is in
%    https://www.latex-project.org/lppl.txt
% and version 1.3 or later is part of all distributions of
% LaTeX version 2005/12/01 or later.
%
% This work has the LPPL maintenance status "maintained".
%
% The Current Maintainers of this work are
% Heiko Oberdiek and the Oberdiek Package Support Group
% https://github.com/ho-tex/oberdiek/issues
%
% The Base Interpreter refers to any `TeX-Format',
% because some files are installed in TDS:tex/generic//.
%
% This work consists of the main source file pdfrender.dtx
% and the derived files
%    pdfrender.sty, pdfrender.pdf, pdfrender.ins, pdfrender.drv,
%    pdfrender-test1.tex, pdfrender-test2.tex,
%    pdfrender-test3.tex, pdfrender-test4.tex,
%    pdfrender-test5.tex.
%
% Distribution:
%    CTAN:macros/latex/contrib/oberdiek/pdfrender.dtx
%    CTAN:macros/latex/contrib/oberdiek/pdfrender.pdf
%
% Unpacking:
%    (a) If pdfrender.ins is present:
%           tex pdfrender.ins
%    (b) Without pdfrender.ins:
%           tex pdfrender.dtx
%    (c) If you insist on using LaTeX
%           latex \let\install=y\input{pdfrender.dtx}
%        (quote the arguments according to the demands of your shell)
%
% Documentation:
%    (a) If pdfrender.drv is present:
%           latex pdfrender.drv
%    (b) Without pdfrender.drv:
%           latex pdfrender.dtx; ...
%    The class ltxdoc loads the configuration file ltxdoc.cfg
%    if available. Here you can specify further options, e.g.
%    use A4 as paper format:
%       \PassOptionsToClass{a4paper}{article}
%
%    Programm calls to get the documentation (example):
%       pdflatex pdfrender.dtx
%       makeindex -s gind.ist pdfrender.idx
%       pdflatex pdfrender.dtx
%       makeindex -s gind.ist pdfrender.idx
%       pdflatex pdfrender.dtx
%
% Installation:
%    TDS:tex/generic/oberdiek/pdfrender.sty
%    TDS:doc/latex/oberdiek/pdfrender.pdf
%    TDS:source/latex/oberdiek/pdfrender.dtx
%
%<*ignore>
\begingroup
  \catcode123=1 %
  \catcode125=2 %
  \def\x{LaTeX2e}%
\expandafter\endgroup
\ifcase 0\ifx\install y1\fi\expandafter
         \ifx\csname processbatchFile\endcsname\relax\else1\fi
         \ifx\fmtname\x\else 1\fi\relax
\else\csname fi\endcsname
%</ignore>
%<*install>
\input docstrip.tex
\Msg{************************************************************************}
\Msg{* Installation}
\Msg{* Package: pdfrender 2019/12/29 v1.6 Access to some PDF graphics parameters (HO)}
\Msg{************************************************************************}

\keepsilent
\askforoverwritefalse

\let\MetaPrefix\relax
\preamble

This is a generated file.

Project: pdfrender
Version: 2019/12/29 v1.6

Copyright (C)
   2010 Heiko Oberdiek
   2016-2019 Oberdiek Package Support Group

This work may be distributed and/or modified under the
conditions of the LaTeX Project Public License, either
version 1.3c of this license or (at your option) any later
version. This version of this license is in
   https://www.latex-project.org/lppl/lppl-1-3c.txt
and the latest version of this license is in
   https://www.latex-project.org/lppl.txt
and version 1.3 or later is part of all distributions of
LaTeX version 2005/12/01 or later.

This work has the LPPL maintenance status "maintained".

The Current Maintainers of this work are
Heiko Oberdiek and the Oberdiek Package Support Group
https://github.com/ho-tex/oberdiek/issues


The Base Interpreter refers to any `TeX-Format',
because some files are installed in TDS:tex/generic//.

This work consists of the main source file pdfrender.dtx
and the derived files
   pdfrender.sty, pdfrender.pdf, pdfrender.ins, pdfrender.drv,
   pdfrender-test1.tex, pdfrender-test2.tex,
   pdfrender-test3.tex, pdfrender-test4.tex,
   pdfrender-test5.tex.

\endpreamble
\let\MetaPrefix\DoubleperCent

\generate{%
  \file{pdfrender.ins}{\from{pdfrender.dtx}{install}}%
  \file{pdfrender.drv}{\from{pdfrender.dtx}{driver}}%
  \usedir{tex/generic/oberdiek}%
  \file{pdfrender.sty}{\from{pdfrender.dtx}{package}}%
%  \usedir{doc/latex/oberdiek/test}%
%  \file{pdfrender-test1.tex}{\from{pdfrender.dtx}{test1}}%
%  \file{pdfrender-test2.tex}{\from{pdfrender.dtx}{test2}}%
%  \file{pdfrender-test3.tex}{\from{pdfrender.dtx}{test3}}%
%  \file{pdfrender-test4.tex}{\from{pdfrender.dtx}{test4}}%
%  \file{pdfrender-test5.tex}{\from{pdfrender.dtx}{test5}}%
}

\catcode32=13\relax% active space
\let =\space%
\Msg{************************************************************************}
\Msg{*}
\Msg{* To finish the installation you have to move the following}
\Msg{* file into a directory searched by TeX:}
\Msg{*}
\Msg{*     pdfrender.sty}
\Msg{*}
\Msg{* To produce the documentation run the file `pdfrender.drv'}
\Msg{* through LaTeX.}
\Msg{*}
\Msg{* Happy TeXing!}
\Msg{*}
\Msg{************************************************************************}

\endbatchfile
%</install>
%<*ignore>
\fi
%</ignore>
%<*driver>
\NeedsTeXFormat{LaTeX2e}
\ProvidesFile{pdfrender.drv}%
  [2019/12/29 v1.6 Access to some PDF graphics parameters (HO)]%
\documentclass{ltxdoc}
\usepackage{holtxdoc}[2011/11/22]
\usepackage{longtable}
\renewcommand*{\arraystretch}{1.2}
\begin{document}
  \DocInput{pdfrender.dtx}%
\end{document}
%</driver>
% \fi
%
%
%
% \GetFileInfo{pdfrender.drv}
%
% \title{The \xpackage{pdfrender} package}
% \date{2019/12/29 v1.6}
% \author{Heiko Oberdiek\thanks
% {Please report any issues at \url{https://github.com/ho-tex/oberdiek/issues}}}
%
% \maketitle
%
% \begin{abstract}
% The PDF format has some graphics parameter like
% line width or text rendering mode. This package
% provides an interface for setting these parameters.
% \end{abstract}
%
% \tableofcontents
%
% \section{Documentation}
%
% This package \xpackage{pdfrender} defines an interface for
% PDF specific parameters that affects the rendering of
% graphics or text. The interface and its implementation
% uses the same technique as package \xpackage{color}
% for color settings. Therefore this package is loaded to
% enable \LaTeX's color interface.
%
% At different places \LaTeX\ uses \cs{normalcolor} to avoid that
% header, footer or floats are print in the current color of the
% main text. \cs{setgroup@color} is used to start a save box
% with the color that is set at box saving time.
% Package \xpackage{pdfrender} extends these macros to
% add its own hooks of its parameters. Therefore \LaTeX3 should
% generalize \LaTeXe's color interface.
%
% \subsection{Usage}
%
% In \LaTeX\ the package is loaded as normal package. Options
% are not defined for this package.
% \begin{quote}
% |\usepackage{pdfrender}|
% \end{quote}
%
% This package can also be used in \plainTeX\ and even ini\TeX:
% \begin{quote}
% |input pdfrender.sty|
% \end{quote}
%
% \subsection{Macros}
%
% \begin{declcs}{pdfrender} \M{key value list}
% \end{declcs}
% The first parameter \meta{key value list} contains a
% list of parameter settings. The key entry is the parameter name.
% The macro works like
% \cs{color} (without optional argument) for color setting.
%
% \begin{declcs}{textpdfrender} \M{key value list} \M{text}
% \end{declcs}
% In the same way as \cs{pdfrender} the first argument
% specifies the parameters that should be set. This parameter
% setting affects \meta{text} only. Basically it works
% the same way as \cs{textcolor} (without optional argument).
%
% \subsection{Parameters}
%
% The following table shows an overview for the supported
% parameters and values:
%
% \begin{longtable}{@{}l|l|l@{}}
% \bfseries Parameter & \bfseries Value & \bfseries Alias\\*
% \hline
% \endhead
% \xoption{TextRenderingMode}
%   & \texttt{0} & \texttt{Fill}\\*
%   & \texttt{1} & \texttt{Stroke}\\*
%   & \texttt{2} & \texttt{FillStroke}\\*
%   & \texttt{3} & \texttt{Invisible}\\*
%   & \texttt{4} & \texttt{FillClip}\\*
%   & \texttt{5} & \texttt{StrokeClip}\\*
%   & \texttt{6} & \texttt{FillStrokeClip}\\*
%   & \texttt{7} & \texttt{Clip}\\
% \hline
% \xoption{LineWidth}
%   & \itshape positive number, & \itshape \TeX\ dimen\\*[-.2ex]
%   & \qquad \itshape unit is bp & \\
% \hline
% \xoption{LineCapStyle}
%   & \texttt{0} & \texttt{Butt}\\*
%   & \texttt{1} & \texttt{Round}\\*
%   & \texttt{2} & \texttt{ProjectingSquare}\\
% \hline
% \xoption{LineJoinStyle}
%   & \texttt{0} & \texttt{Miter}\\*
%   & \texttt{1} & \texttt{Round}\\*
%   & \texttt{2} & \texttt{Bevel}\\
% \hline
% \xoption{MiterLimit}
%   & \multicolumn{2}{l@{}}{\itshape positive number} \\
% \hline
% \xoption{Flatness}
%   & \multicolumn{2}{l@{}}{\itshape number between 0 and 100} \\
% \hline
% \xoption{LineDashPattern}
%   & \multicolumn{2}{l@{}}{^^A
%       \itshape numbers in square brackets,^^A
%     } \\*[-.2ex]
%   & \multicolumn{2}{l@{}}{^^A
%       \qquad \itshape followed by number,^^A
%     } \\*
%   & \multicolumn{2}{l@{}}{^^A
%       \qquad \itshape units are bp^^A
%     } \\
% \hline
% \xoption{RenderingIntent}
%   & \multicolumn{2}{l@{}}{\texttt{AbsoluteColorimetric}} \\*
%   & \multicolumn{2}{l@{}}{\texttt{RelativeColorimetric}} \\*
%   & \multicolumn{2}{l@{}}{\texttt{Saturation}} \\*
%   & \multicolumn{2}{l@{}}{\texttt{Perceptual}} \\
% \hline
% \xoption{FillColor}
%   & & \itshape color specification \\
% \hline
% \xoption{StrokeColor}
%   & & \itshape color specification \\
% \end{longtable}
%
% \subsubsection{Details}
%
% The description and specification of these parameters
% are available in the PDF specification \cite{pdf-spec}.
% Therefore they are not repeated here.
%
% \begin{description}
% \item[Value:]
%   The values in the second column lists or describe
%   the values that are specified by the PDF specification.
% \item[Alias:] Instead of magic numbers the package also
%   defines some aliases that can be given as value.
%   Example: |LineCapStyle=Round| has the same effect as
%   |LineCapStyle=1|.
% \item[Number:] The term \emph{number} means an integer
%   or real number. The real number is given as plain
%   decimal number without exponent. The decimal separator is a period.
%   At least one digit must be present.
% \item[\xoption{LineWidth}:] As alias a \TeX\ dimen specification
%   can be given. This includes explicit specifications with number
%   and unit, e.g. |LineWidth=0.5pt|. Also \LaTeX\ length registers
%   may be used. If \eTeX's \cs{dimexpr} is available, then it is
%   automatically added. However package \xpackage{calc} is not
%   supported.
% \item[\xoption{FillColor}, \xoption{StrokeColor}:]
%   Package \xpackage{color} or \xpackage{xcolor} must be loaded
%   before these options can be used (since version 1.2).
%   \LaTeX's color support sets both colors at the same time
%   to the same value. However parameter \xoption{TextRenderingMode}
%   offers the value \texttt{FillStroke} that makes only sense, if
%   the two color types can be set separately.
%   If one of the options \xoption{FillColor} or \xoption{StrokeColor}
%   is specified, then also the color is set. For compatibility
%   with the \LaTeX\ color packages (\xpackage{color} or \xpackage{xcolor}),
%   always both colors must be set. Thus if one of them is not specified,
%   it is taken from the current color.
%
%   Both options \xoption{FillColor} and \xoption{StrokeColor}
%   expect a \LaTeX\ color specification as value.
%   Also the optional color model argument is supported.
%   Example:
%   \begin{quote}
%     |FillColor=yellow|,\\
%     |StrokeColor=[cmyk]{1,.5,0,0}|\\
%   \end{quote}
% \end{description}
%
% \subsection{Color stack}
%
% If the \pdfTeX\ version provides color stacks, then each
% parameter is assigned a page based color stack. The assignment
% of a stack takes place, when its parameter is set the first time.
% This avoids the use of color stacks that are not needed.
%
% \StopEventually{
% }
%
% \section{Implementation}
%    \begin{macrocode}
%<*package>
%    \end{macrocode}
%    Reload check, especially if the package is not used with \LaTeX.
%    \begin{macrocode}
\begingroup\catcode61\catcode48\catcode32=10\relax%
  \catcode13=5 % ^^M
  \endlinechar=13 %
  \catcode35=6 % #
  \catcode39=12 % '
  \catcode44=12 % ,
  \catcode45=12 % -
  \catcode46=12 % .
  \catcode58=12 % :
  \catcode64=11 % @
  \catcode123=1 % {
  \catcode125=2 % }
  \expandafter\let\expandafter\x\csname ver@pdfrender.sty\endcsname
  \ifx\x\relax % plain-TeX, first loading
  \else
    \def\empty{}%
    \ifx\x\empty % LaTeX, first loading,
      % variable is initialized, but \ProvidesPackage not yet seen
    \else
      \expandafter\ifx\csname PackageInfo\endcsname\relax
        \def\x#1#2{%
          \immediate\write-1{Package #1 Info: #2.}%
        }%
      \else
        \def\x#1#2{\PackageInfo{#1}{#2, stopped}}%
      \fi
      \x{pdfrender}{The package is already loaded}%
      \aftergroup\endinput
    \fi
  \fi
\endgroup%
%    \end{macrocode}
%    Package identification:
%    \begin{macrocode}
\begingroup\catcode61\catcode48\catcode32=10\relax%
  \catcode13=5 % ^^M
  \endlinechar=13 %
  \catcode35=6 % #
  \catcode39=12 % '
  \catcode40=12 % (
  \catcode41=12 % )
  \catcode44=12 % ,
  \catcode45=12 % -
  \catcode46=12 % .
  \catcode47=12 % /
  \catcode58=12 % :
  \catcode64=11 % @
  \catcode91=12 % [
  \catcode93=12 % ]
  \catcode123=1 % {
  \catcode125=2 % }
  \expandafter\ifx\csname ProvidesPackage\endcsname\relax
    \def\x#1#2#3[#4]{\endgroup
      \immediate\write-1{Package: #3 #4}%
      \xdef#1{#4}%
    }%
  \else
    \def\x#1#2[#3]{\endgroup
      #2[{#3}]%
      \ifx#1\@undefined
        \xdef#1{#3}%
      \fi
      \ifx#1\relax
        \xdef#1{#3}%
      \fi
    }%
  \fi
\expandafter\x\csname ver@pdfrender.sty\endcsname
\ProvidesPackage{pdfrender}%
  [2019/12/29 v1.6 Access to some PDF graphics parameters (HO)]%
%    \end{macrocode}
%
%    \begin{macrocode}
\begingroup\catcode61\catcode48\catcode32=10\relax%
  \catcode13=5 % ^^M
  \endlinechar=13 %
  \catcode123=1 % {
  \catcode125=2 % }
  \catcode64=11 % @
  \def\x{\endgroup
    \expandafter\edef\csname PdfRender@AtEnd\endcsname{%
      \endlinechar=\the\endlinechar\relax
      \catcode13=\the\catcode13\relax
      \catcode32=\the\catcode32\relax
      \catcode35=\the\catcode35\relax
      \catcode61=\the\catcode61\relax
      \catcode64=\the\catcode64\relax
      \catcode123=\the\catcode123\relax
      \catcode125=\the\catcode125\relax
    }%
  }%
\x\catcode61\catcode48\catcode32=10\relax%
\catcode13=5 % ^^M
\endlinechar=13 %
\catcode35=6 % #
\catcode64=11 % @
\catcode123=1 % {
\catcode125=2 % }
\def\TMP@EnsureCode#1#2{%
  \edef\PdfRender@AtEnd{%
    \PdfRender@AtEnd
    \catcode#1=\the\catcode#1\relax
  }%
  \catcode#1=#2\relax
}
\TMP@EnsureCode{10}{12}% ^^J
\TMP@EnsureCode{36}{3}% $
\TMP@EnsureCode{39}{12}% '
\TMP@EnsureCode{40}{12}% (
\TMP@EnsureCode{41}{12}% )
\TMP@EnsureCode{42}{12}% *
\TMP@EnsureCode{43}{12}% +
\TMP@EnsureCode{44}{12}% ,
\TMP@EnsureCode{45}{12}% -
\TMP@EnsureCode{46}{12}% .
\TMP@EnsureCode{47}{12}% /
\TMP@EnsureCode{58}{12}% :
\TMP@EnsureCode{59}{12}% ;
\TMP@EnsureCode{60}{12}% <
\TMP@EnsureCode{62}{12}% >
\TMP@EnsureCode{63}{12}% ?
\TMP@EnsureCode{91}{12}% [
\TMP@EnsureCode{93}{12}% ]
\TMP@EnsureCode{94}{7}% ^ (superscript)
\TMP@EnsureCode{96}{12}% `
\TMP@EnsureCode{124}{12}% |
%    \end{macrocode}
%
%    \begin{macrocode}
\def\PdfRender@AtEndHook{}
\expandafter\def\expandafter\PdfRender@AtEnd\expandafter{%
  \expandafter\PdfRender@AtEndHook
  \PdfRender@AtEnd
  \endinput
}
%    \end{macrocode}
%
% \subsection{Look for \pdfTeX, its mode and features}
%
%    \begin{macro}{\PdfRender@newif}
%    \begin{macrocode}
\def\PdfRender@newif#1{%
  \expandafter\edef\csname PdfRender@#1true\endcsname{%
    \let
    \expandafter\noexpand\csname ifPdfRender@#1\endcsname
    \noexpand\iftrue
  }%
  \expandafter\edef\csname PdfRender@#1false\endcsname{%
    \let
    \expandafter\noexpand\csname ifPdfRender@#1\endcsname
    \noexpand\iffalse
  }%
  \csname PdfRender@#1false\endcsname
}
%    \end{macrocode}
%    \end{macro}
%
%    \begin{macro}{\ifPdfRender@Stack}
%    \begin{macrocode}
\PdfRender@newif{Stack}
%    \end{macrocode}
%    \end{macro}
%    \begin{macro}{\ifPdfRender@Match}
%    \begin{macrocode}
\PdfRender@newif{Match}
%    \end{macrocode}
%    \end{macro}
%
%    \begin{macro}{\PdfRender@RequirePackage}
%    \begin{macrocode}
\begingroup\expandafter\expandafter\expandafter\endgroup
\expandafter\ifx\csname RequirePackage\endcsname\relax
  \def\PdfRender@RequirePackage#1[#2]{%
    \expandafter\def\expandafter\PdfRender@AtEndHook\expandafter{%
      \PdfRender@AtEndHook
      \ltx@ifpackagelater{#1}{#2}{}{%
        \@PackageWarningNoLine{pdfrender}{%
          You have requested version\MessageBreak
          `#2' of package `#1',\MessageBreak
          but only version\MessageBreak
          `\csname ver@#1.\ltx@pkgextension\endcsname'\MessageBreak
          is available%
        }%
      }%
    }%
    \input #1.sty\relax
  }%
\else
  \let\PdfRender@RequirePackage\RequirePackage
\fi
%    \end{macrocode}
%    \end{macro}
% Luatex compatibility
%    \begin{macrocode}
\ifx\pdfextension\@undefined\else
  \def\pdfcolorstackinit {\pdffeedback colorstackinit}
  \protected\def\pdfcolorstack     {\pdfextension colorstack}
  \protected\def\pdfliteral        {\pdfextension literal}
\fi
%    \end{macrocode}
%
%    \begin{macrocode}
\PdfRender@RequirePackage{iftex}[2019/11/07]
\PdfRender@RequirePackage{infwarerr}[2007/09/09]
\PdfRender@RequirePackage{ltxcmds}[2010/01/28]
%    \end{macrocode}
%
%    \begin{macrocode}
\ifpdf
  \ltx@IfUndefined{pdfcolorstackinit}{%
    \@PackageWarning{pdfrender}{%
      Missing \string\pdfcolorstackinit
    }%
  }{%
    \PdfRender@Stacktrue
  }%
  \ltx@IfUndefined{pdfmatch}{%
    \@PackageInfoNoLine{pdfrender}{%
      \string\pdfmatch\ltx@space not found. %
      Therefore the values\MessageBreak
      of some parameters are not validated%
    }%
  }{%
    \PdfRender@Matchtrue
  }%
\else
  \@PackageWarning{pdfrender}{%
    Missing pdfTeX in PDF mode%
  }%
%    \end{macrocode}
%    \begin{macrocode}
  \ltx@IfUndefined{newcommand}{%
%    \end{macrocode}
%    \begin{macro}{\pdfrender}
%    \begin{macrocode}
    \def\pdfrender#1{}%
%    \end{macrocode}
%    \end{macro}
%    \begin{macro}{\textpdfrender}
%    \begin{macrocode}
    \long\def\textpdfrender#1#2{#2}%
%    \end{macrocode}
%    \end{macro}
%    \begin{macrocode}
  }{%
%    \end{macrocode}
%    \begin{macro}{\pdfrender}
%    \begin{macrocode}
    \newcommand*{\pdfrender}[1]{}%
%    \end{macrocode}
%    \end{macro}
%    \begin{macro}{\textpdfrender}
%    \begin{macrocode}
    \newcommand{\textpdfrender}[2]{#2}%
%    \end{macrocode}
%    \end{macro}
%    \begin{macrocode}
  }%
  \expandafter\PdfRender@AtEnd
\fi%
%    \end{macrocode}
%
% \subsection{Enable color support of \LaTeX}
%
%    \begin{macrocode}
\ltx@ifpackageloaded{color}{}{%
  \def\color@setgroup{\begingroup\set@color}%
  \let\color@begingroup\begingroup
  \def\color@endgroup{\endgraf\endgroup}%
  \def\color@hbox{\hbox\bgroup\color@begingroup}%
  \def\color@vbox{\vbox\bgroup\color@begingroup}%
  \def\color@endbox{\color@endgroup\egroup}%
  \ltx@ifundefined{bgroup}{%
    \let\bgroup={\let\egroup=}%
  }{}%
  \ltx@ifundefined{endgraf}{%
    \let\endgraf=\par
  }{}%
}
%    \end{macrocode}
%
% \subsection{Hook into \cs{normalcolor}}
%
%    The problem is that packages \xpackage{color} and \xpackage{xcolor}
%    each overwrite \cs{normalcolor}. For example, after the package
%    loading order \xpackage{color}, \xpackage{pdfrender} and \xpackage{xcolor}
%    the patched version of \cs{normalcolor} is overwritten by
%    package \xpackage{xcolor}. Also using \cs{AtBeginDocument} for
%    patching is not enough. If package \xpackage{hyperref} is loaded
%    later, it might load package \xpackage{color} using \cs{AtBeginDocument}.
%    \begin{macro}{\PdfRender@NormalColorHook}
%    \begin{macrocode}
\def\PdfRender@NormalColorHook{}
%    \end{macrocode}
%    \end{macro}
%    \begin{macro}{\PdfRender@ColorSetGroupHook}
%    \begin{macrocode}
\def\PdfRender@ColorSetGroupHook{}
%    \end{macrocode}
%    \end{macro}
%    \begin{macro}{\PdfRender@TestBox}
%    \begin{macrocode}
\def\PdfRender@TestBox#1{%
  \setbox0=\color@hbox#1\color@endbox
}
%    \end{macrocode}
%    \end{macro}
%    \begin{macro}{\PdfRender@PatchNormalColor}
%    \begin{macrocode}
\def\PdfRender@PatchNormalColor{%
  \ltx@ifundefined{normalcolor}{%
    \gdef\normalcolor{\PdfRender@NormalColorHook}%
  }{%
    \begingroup
      \def\PdfRender@NormalColorHook{\let\PdfRender@temp=Y}%
      \PdfRender@TestBox{%
        \let\set@color\relax
        \normalcolor
        \ifx\PdfRender@temp Y%
        \else
          \ltx@GlobalAppendToMacro\normalcolor{%
            \PdfRender@NormalColorHook
          }%
        \fi
      }%
    \endgroup
  }%
  \ifx\@nodocument\relax
    \global\let\PdfRender@PatchNormalColor\relax
  \fi
}%
%    \end{macrocode}
%    \end{macro}
%    \begin{macro}{\PdfRender@PatchColorSetGroup}
%    \begin{macrocode}
\def\PdfRender@PatchColorSetGroup{%
  \begingroup
    \def\PdfRender@ColorSetGroupHook{\let\PdfRender@temp=Y}%
    \PdfRender@TestBox{%
      \let\set@color\relax
      \color@setgroup\color@endgroup
      \ifx\PdfRender@temp Y%
      \else
        \ltx@GlobalAppendToMacro\color@setgroup{%
          \PdfRender@ColorSetGroupHook
        }%
      \fi
    }%
  \endgroup
  \ifx\@nodocument\relax
    \global\let\PdfRender@PatchColorSetGroup\relax
  \fi
}%
%    \end{macrocode}
%    \end{macro}
%    \begin{macro}{\PdfRender@PatchColor}
%    \begin{macrocode}
\def\PdfRender@PatchColor{%
  \PdfRender@PatchNormalColor
  \PdfRender@PatchColorSetGroup
}
%    \end{macrocode}
%    \end{macro}
%    \begin{macrocode}
\PdfRender@PatchColor
\ltx@IfUndefined{AtBeginDocument}{}{%
  \AtBeginDocument{\PdfRender@PatchColor}%
}
%    \end{macrocode}
%    \cs{AfterPackage} is provided by package \xpackage{scrlfile}.
%    \begin{macrocode}
\ltx@IfUndefined{AfterPackage}{%
}{%
  \AfterPackage{color}{\PdfRender@PatchColor}%
  \AfterPackage{xcolor}{\PdfRender@PatchColor}%
  \AfterPackage{etoolbox}{%
    \AfterEndPreamble{\PdfRender@PatchColor}%
  }%
}%
%    \end{macrocode}
%    \cs{AfterEndPreamble} is provided by package \xpackage{etoolbox}.
%    \begin{macrocode}
\ltx@IfUndefined{AfterEndPreamble}{%
}{%
  \AfterEndPreamble{\PdfRender@PatchColor}%
}%
%    \end{macrocode}
%
%    \begin{macrocode}
\PdfRender@RequirePackage{kvsetkeys}[2010/01/28]
%    \end{macrocode}
%    \begin{macro}{\PdfRender@texorpdfstring}
%    \begin{macrocode}
\def\PdfRender@texorpdfstring{%
  \ltx@IfUndefined{texorpdfstring}\ltx@firstoftwo\texorpdfstring
}
%    \end{macrocode}
%    \end{macro}
%    \begin{macro}{\pdfrender}
%    \begin{macrocode}
\ltx@IfUndefined{DeclareRobustCommand}%
\ltx@firstoftwo\ltx@secondoftwo
{%
  \def\pdfrender#1%
}{%
  \newcommand{\pdfrender}{}%
  \DeclareRobustCommand*{\pdfrender}[1]%
}%
{%
  \PdfRender@texorpdfstring{%
    \PdfRender@PatchNormalColor
    \global\let\PdfRender@FillColor\ltx@empty
    \global\let\PdfRender@StrokeColor\ltx@empty
    \kvsetkeys{PDFRENDER}{#1}%
    \PdfRender@SetColor
  }{}%
}
%    \end{macrocode}
%    \end{macro}
%    \begin{macro}{\textpdfrender}
%    \begin{macrocode}
\ltx@IfUndefined{DeclareRobustCommand}%
\ltx@firstoftwo\ltx@secondoftwo
{%
  \long\def\textpdfrender#1#2%
}{%
  \newcommand{\textpdfrender}{}%
  \DeclareRobustCommand{\textpdfrender}[2]%
}%
{%
  \PdfRender@texorpdfstring{%
    \begingroup
      \pdfrender{#1}%
      #2%
    \endgroup
  }{#2}%
}
%    \end{macrocode}
%    \end{macro}
%
%    \begin{macro}{\ifPdfRender@Values}
%    \begin{macrocode}
\PdfRender@newif{Values}
%    \end{macrocode}
%    \end{macro}
%    \begin{macro}{\PdfRender@NewClassValues}
%    \begin{macrocode}
\def\PdfRender@NewClassValues#1#2#3#4{%
  \PdfRender@Valuestrue
  \PdfRender@NewClass{#1}{#2}{#3}{#4}{}%
}
%    \end{macrocode}
%    \end{macro}
%    \begin{macro}{\PdfRender@NewClass}
%    \begin{macrocode}
\def\PdfRender@NewClass#1#2#3#4#5{%
  \PdfRender@newif{Active#1}%
  \expandafter\def\csname PdfRender@Default#1\endcsname{#2}%
  \expandafter\let\csname PdfRender@Current#1\expandafter\endcsname
      \csname PdfRender@Default#1\endcsname
  \ifPdfRender@Stack
    \expandafter\edef\csname PdfRender@Init#1\endcsname{%
      \global\chardef
      \expandafter\noexpand\csname PdfRender@Stack#1\endcsname=%
          \noexpand\pdfcolorstackinit page direct{%
            \noexpand#3%
            \expandafter\noexpand\csname PdfRender@Default#1\endcsname
          }\relax
      \noexpand\@PackageInfo{pdfrender}{%
        New color stack `#1' = \noexpand\number
        \expandafter\noexpand\csname PdfRender@Stack#1\endcsname
      }%
      \gdef\expandafter\noexpand\csname PdfRender@Init#1\endcsname{}%
    }%
    \expandafter\edef\csname PdfRender@Set#1\endcsname{%
      \expandafter\noexpand\csname PdfRender@Init#1\endcsname
      \noexpand\pdfcolorstack
      \expandafter\noexpand\csname PdfRender@Stack#1\endcsname
      push{%
        #3{\expandafter\noexpand\csname PdfRender@Current#1\endcsname}%
      }%
      \noexpand\aftergroup
      \expandafter\noexpand\csname PdfRender@Reset#1\endcsname
    }%
    \expandafter\edef\csname PdfRender@Reset#1\endcsname{%
      \expandafter\noexpand\csname PdfRender@Init#1\endcsname
      \noexpand\pdfcolorstack
      \expandafter\noexpand\csname PdfRender@Stack#1\endcsname
      pop\relax
    }%
  \else
    \expandafter\edef\csname PdfRender@Set#1\endcsname{%
      \noexpand\pdfliteral direct{%
        #3{\expandafter\noexpand\csname PdfRender@Current#1\endcsname}%
      }%
      \noexpand\aftergroup
      \expandafter\noexpand\csname PdfRender@Reset#1\endcsname
    }%
    \expandafter\edef\csname PdfRender@Reset#1\endcsname{%
      \noexpand\pdfliteral direct{%
        #3{\expandafter\noexpand\csname PdfRender@Current#1\endcsname}%
      }%
    }%
  \fi
  \expandafter\edef\csname PdfRender@Normal#1\endcsname{%
    \let
    \expandafter\noexpand\csname PdfRender@Current#1\endcsname
    \expandafter\noexpand\csname PdfRender@Default#1\endcsname
    \noexpand\PdfRender@Set{#1}%
  }%
  \expandafter\ltx@GlobalAppendToMacro\expandafter\PdfRender@NormalColorHook
  \expandafter{%
    \csname PdfRender@Normal#1\endcsname
  }%
  \ltx@GlobalAppendToMacro\PdfRender@ColorSetGroupHook{%
    \PdfRender@Set{#1}%
  }%
  \ifPdfRender@Values
    \kv@parse@normalized{#4}{%
      \expandafter\let\csname PdfRender@#1@\kv@key\endcsname\kv@key
      \ifx\kv@value\relax
      \else
        \expandafter\let\csname PdfRender@#1@\kv@value\endcsname\kv@key
      \fi
      \ltx@gobbletwo
    }%
    \PdfRender@define@key{PDFRENDER}{#1}{%
      \global\csname PdfRender@Active#1true\endcsname
      \def\PdfRender@Current{##1}%
      \PdfRender@SetValidateValues{#1}%
    }%
    \PdfRender@Valuesfalse
  \else
    \PdfRender@define@key{PDFRENDER}{#1}{%
      \global\csname PdfRender@Active#1true\endcsname
      \expandafter\def\csname PdfRender@Current#1\endcsname{##1}%
      \ltx@IfUndefined{PdfRender@PostProcess#1}{%
      }{%
        \csname PdfRender@PostProcess#1\endcsname
      }%
      \PdfRender@SetValidate{#1}{#4}{#5}%
    }%
  \fi
}%
%    \end{macrocode}
%    \end{macro}
%    \begin{macro}{\PdfRender@define@key}
%    \begin{macrocode}
\ltx@IfUndefined{define@key}{%
  \def\PdfRender@define@key#1#2{%
    \expandafter\def\csname KV@#1@#2\endcsname##1%
  }%
}{%
  \let\PdfRender@define@key\define@key
}
%    \end{macrocode}
%    \end{macro}
%    \begin{macro}{\PdfRender@Set}
%    \begin{macrocode}
\def\PdfRender@Set#1{%
  \csname ifPdfRender@Active#1\endcsname
    \csname PdfRender@Set#1\expandafter\endcsname
  \fi
}
%    \end{macrocode}
%    \end{macro}
%    \begin{macro}{\PdfRender@Reset}
%    \begin{macrocode}
\def\PdfRender@Reset#1{%
  \csname ifPdfRender@Active#1\endcsname
    \csname PdfRender@Reset#1\expandafter\endcsname
  \fi
}
%    \end{macrocode}
%    \end{macro}
%    \begin{macro}{\PdfRender@ErrorInvalidValue}
%    \begin{macrocode}
\def\PdfRender@ErrorInvalidValue#1{%
  \PackageError{pdfrender}{%
    Ignoring parameter setting for `#1'\MessageBreak
    because of invalid value %
    `\csname PdfRender@Current#1\endcsname'%
  }\@ehc
  \expandafter\let\csname PdfRender@Current#1\endcsname\ltx@empty
}%
%    \end{macrocode}
%    \end{macro}
%    \begin{macro}{\PdfRender@SetValidate}
%    \begin{macrocode}
\ifPdfRender@Match
  \def\PdfRender@SetValidate#1#2#3{%
    \ifnum\pdfmatch{^(#2)$}{\csname PdfRender@Current#1\endcsname}=1 %
      \csname PdfRender@Set#1\expandafter\endcsname
    \else
      \PdfRender@ErrorInvalidValue{#1}%
    \fi
  }%
\else
  \def\PdfRender@SetValidate#1#2#3{%
    \expandafter\let\expandafter\PdfRender@Current
    \csname PdfRender@Current#1\endcsname
    #3%
    \ifx\PdfRender@Current\@empty
      \PdfRender@ErrorInvalidValue{#1}%
    \else
      \csname PdfRender@Set#1\expandafter\endcsname
    \fi
  }%
\fi
%    \end{macrocode}
%    \end{macro}
%    \begin{macro}{\PdfRender@SetValidateValues}
%    \begin{macrocode}
\def\PdfRender@SetValidateValues#1{%
  \ltx@IfUndefined{PdfRender@#1@\PdfRender@Current}{%
    \expandafter\let\csname PdfRender@Current#1\endcsname
                    \PdfRender@Current
    \PdfRender@ErrorInvalidValue{#1}%
  }{%
    \expandafter\edef\csname PdfRender@Current#1\endcsname{%
      \csname PdfRender@#1@\PdfRender@Current\endcsname
    }%
    \csname PdfRender@Set#1\endcsname
  }%
}
%    \end{macrocode}
%    \end{macro}
%    \begin{macro}{\PdfRender@OpValue}
%    \begin{macrocode}
\def\PdfRender@OpValue#1#2{#2\ltx@space#1}%
%    \end{macrocode}
%    \end{macro}
%    \begin{macro}{\PdfRender@OpName}
%    \begin{macrocode}
\def\PdfRender@OpName#1#2{/#2\ltx@space#1}%
%    \end{macrocode}
%    \end{macro}
%
% \subsection{Declare and setup parameters}
%
%    \begin{macrocode}
\PdfRender@NewClassValues{TextRenderingMode}%
                      {0}%
                      {\PdfRender@OpValue{Tr}}{%
  0=Fill,%
  1=Stroke,%
  2=FillStroke,%
  3=Invisible,%
  4=FillClip,%
  5=StrokeClip,%
  6=FillStrokeClip,%
  7=Clip,%
}%
\PdfRender@NewClass{LineWidth}{1}{\PdfRender@OpValue{w}}{%
  [0-9]+\string\.?[0-9]*|\string\.[0-9]+%
}{}%
\ltx@IfUndefined{dimexpr}{%
  \def\PdfRender@dimexpr{}%
}{%
  \let\PdfRender@dimexpr\dimexpr
}
\def\PdfRender@PostProcessLineWidth{%
  \begingroup
  \afterassignment\PdfRender@@PostProcessLineWidth
  \dimen0=\PdfRender@dimexpr\PdfRender@CurrentLineWidth bp %
  \PdfRender@let\PdfRender@relax\PdfRender@relax
}
\let\PdfRender@let\let
\let\PdfRender@relax\relax
\def\PdfRender@@PostProcessLineWidth#1\PdfRender@let{%
  \ifx\\#1\\%
    \endgroup
  \else
    \dimen0=.996264\dimen0 % 72/72.27
    \edef\x{\endgroup
      \def\noexpand\PdfRender@CurrentLineWidth{%
        \strip@pt\dimen0%
      }%
    }%
    \expandafter\x
  \fi
}
\PdfRender@NewClassValues{LineCapStyle}{0}{\PdfRender@OpValue{J}}{%
  0=Butt,%
  1=Round,%
  2=ProjectingSquare,%
}%
\PdfRender@NewClassValues{LineJoinStyle}{0}{\PdfRender@OpValue{j}}{%
  0=Miter,%
  1=Round,%
  2=Bevel,%
}%
\PdfRender@NewClass{MiterLimit}{10}{\PdfRender@OpValue{M}}{%
  [0-9]*[1-9][0-9]*\string\.?[0-9]*|%
  [0-9]*\string\.?[0-9]*[1-9][0-9]*%
}{}%
\PdfRender@NewClass{Flatness}{0}{\PdfRender@OpValue{i}}{%
  100(\string\.0*)?|[0-9][0-9](\string\.[0-9]*)?|\string\.[0-9]+%
}{}%
\PdfRender@NewClass{LineDashPattern}{[]0}{\PdfRender@OpValue{d}}{%
  \string\[%
  ( ?([0-9]+\string\.?[0-9]*|\string\.[0-9]+) ?)*%
  \string\] ?%
  ([0-9]+\string\.?[0-9]*|\string\.[0-9]+)%
}{}%
\PdfRender@NewClassValues{RenderingIntent}%
                      {RelativeColorimetric}%
                      {\PdfRender@OpName{ri}}{%
  AbsoluteColorimetric,%
  RelativeColorimetric,%
  Saturation,%
  Perceptual,%
}%
%    \end{macrocode}
%
% \subsection{Fill and stroke color support}
%
%    \begin{macrocode}
\PdfRender@define@key{PDFRENDER}{FillColor}{%
  \begingroup
    \def\PdfRender@Color{#1}%
    \ifx\PdfRender@Color\ltx@empty
      \global\let\PdfRender@FillColor\ltx@empty
    \else
      \PdfRender@ColorAvailable{%
        \PdfRender@TestBox{%
          \expandafter\PdfRender@TryColor\PdfRender@Color\ltx@empty
          \PdfRender@GetFillColor
          \ifx\PdfRender@FillColor\ltx@empty
            \@PackageWarning{pdfrender}{%
              Cannot extract fill color\MessageBreak
              from value `#1'%
            }%
          \fi
        }%
      }%
    \fi
  \endgroup
}
\PdfRender@define@key{PDFRENDER}{StrokeColor}{%
  \begingroup
    \def\PdfRender@Color{#1}%
    \ifx\PdfRender@Color\ltx@empty
      \global\let\PdfRender@StrokeColor\ltx@empty
    \else
      \PdfRender@ColorAvailable{%
        \PdfRender@TestBox{%
          \expandafter\PdfRender@TryColor\PdfRender@Color\ltx@empty
          \PdfRender@GetStrokeColor
          \ifx\PdfRender@StrokeColor\ltx@empty
            \@PackageWarning{pdfrender}{%
              Cannot extract stroke color\MessageBreak
              from value `#1'%
            }%
          \fi
        }%
      }%
    \fi
  \endgroup
}
%    \end{macrocode}
%    \begin{macro}{\PdfRender@ColorAvailable}
%    \begin{macrocode}
\def\PdfRender@ColorAvailable{%
  \@ifundefined{set@color}{%
    \@PackageError{pdfrender}{%
      Ignoring color options, because neither\MessageBreak
      package `color' nor package `xcolor' is loaded%
    }\@ehc
    \global\let\PdfRender@ColorAvailable\ltx@gobble
  }{%
    \global\let\PdfRender@ColorAvailable\ltx@firstofone
  }%
  \PdfRender@ColorAvailable
}
%    \end{macrocode}
%    \end{macro}
%    \begin{macro}{\PdfRender@TryColor}
%    \begin{macrocode}
\def\PdfRender@TryColor{%
  \@ifnextchar[\color\PdfRender@@TryColor
}
%    \end{macrocode}
%    \end{macro}
%    \begin{macro}{\PdfRender@@TryColor}
%    \begin{macrocode}
\def\PdfRender@@TryColor#1\ltx@empty{%
  \expandafter\color\expandafter{\PdfRender@Color}%
}
%    \end{macrocode}
%    \end{macro}
%    \begin{macro}{\PdfRender@SetColor}
%    \begin{macrocode}
\def\PdfRender@SetColor{%
  \chardef\PdfRender@NeedsCurrentColor=0 %
  \ifx\PdfRender@FillColor\ltx@empty
    \ifx\PdfRender@StrokeColor\ltx@empty
    \else
      \edef\PdfRender@CurrentColor{%
        \noexpand\PdfRender@FillColor\ltx@space\PdfRender@StrokeColor
      }%
      \chardef\PdfRender@NeedsCurrentColor=1 %
    \fi
  \else
    \ifx\PdfRender@StrokeColor\ltx@empty
      \edef\PdfRender@CurrentColor{%
        \PdfRender@FillColor\ltx@space\noexpand\PdfRender@StrokeColor
      }%
      \chardef\PdfRender@NeedsCurrentColor=2 %
    \else
      \edef\current@color{%
        \PdfRender@FillColor\ltx@space\PdfRender@StrokeColor
      }%
      \set@color
    \fi
  \fi
  \ifnum\PdfRender@NeedsCurrentColor=1 %
    \PdfRender@GetFillColor
    \ifx\PdfRender@FillColor\ltx@empty
      \@PackageWarning{pdfrender}{%
        Cannot extract current fill color%
      }%
    \else
      \edef\current@color{\PdfRender@CurrentColor}%
      \set@color
    \fi
  \else
    \ifnum\PdfRender@NeedsCurrentColor=2 %
      \PdfRender@GetStrokeColor
      \ifx\PdfRender@StrokeColor\ltx@empty
        \@PackageWarning{pdfrender}{%
          Cannot extract current stroke color%
        }%
      \else
        \edef\current@color{\PdfRender@CurrentColor}%
        \set@color
      \fi
    \fi
  \fi
}
%    \end{macrocode}
%    \end{macro}
%    \begin{macro}{\PdfRender@PatternFillColor}
%    \begin{macrocode}
\edef\PdfRender@PatternFillColor{ % space
  (%
    [0-9\string\.]+ g|%
    [0-9\string\.]+ [0-9\string\.]+ [0-9\string\.]+ rg|%
    [0-9\string\.]+ [0-9\string\.]+ %
    [0-9\string\.]+ [0-9\string\.]+ k%
  ) % space
  (.*)$%
}
%    \end{macrocode}
%    \end{macro}
%    \begin{macro}{\PdfRender@PatternStrokeColor}
%    \begin{macrocode}
\edef\PdfRender@PatternStrokeColor{ % space
  (%
    [0-9\string\.]+ G|%
    [0-9\string\.]+ [0-9\string\.]+ [0-9\string\.]+ RG|%
    [0-9\string\.]+ [0-9\string\.]+ %
    [0-9\string\.]+ [0-9\string\.]+ K%
  ) % space
  (.*)$%
}
%    \end{macrocode}
%    \end{macro}
%    \begin{macro}{\PdfRender@MatchPattern}
%    \begin{macrocode}
\def\PdfRender@MatchPattern#1{%
  \ifnum\pdfmatch{\PdfRender@Pattern}{\PdfRender@String}=1 %
    \xdef#1{%
      \expandafter\strip@prefix\pdflastmatch 1%
    }%
    \edef\PdfRender@String{%
      \expandafter\strip@prefix\pdflastmatch 2%
    }%
    \ifx\PdfRender@String\ltx@empty
    \else
      \expandafter\expandafter\expandafter\PdfRender@MatchPattern
      \expandafter\expandafter\expandafter#1%
    \fi
  \fi
}
%    \end{macrocode}
%    \end{macro}
%    \begin{macro}{\PdfRender@GetFillColor}
%    \begin{macrocode}
\def\PdfRender@GetFillColor{%
  \global\let\PdfRender@FillColor\ltx@empty
  \begingroup
    \ifPdfRender@Match
      \let\PdfRender@Pattern\PdfRender@PatternFillColor
      \edef\PdfRender@String{\ltx@space\current@color\ltx@space}%
      \PdfRender@MatchPattern\PdfRender@FillColor
    \else
      \edef\current@color{\current@color\ltx@space}%
      \let\PdfRender@OP\relax
      \PdfRender@FindOp{g}0%
      \PdfRender@FindOp{G}1%
      \PdfRender@FindOp{rg}0%
      \PdfRender@FindOp{RG}1%
      \PdfRender@FindOp{k}0%
      \PdfRender@FindOp{K}1%
      \PdfRender@FilterOp 0\PdfRender@FillColor
    \fi
  \endgroup
}
%    \end{macrocode}
%    \end{macro}
%    \begin{macro}{\PdfRender@GetStrokeColor}
%    \begin{macrocode}
\def\PdfRender@GetStrokeColor{%
  \global\let\PdfRender@StrokeColor\ltx@empty
  \begingroup
    \ifPdfRender@Match
      \let\PdfRender@Pattern\PdfRender@PatternStrokeColor
      \edef\PdfRender@String{\ltx@space\current@color\ltx@space}%
      \PdfRender@MatchPattern\PdfRender@StrokeColor
    \else
      \edef\current@color{\current@color\ltx@space}%
      \let\PdfRender@OP\relax
      \PdfRender@FindOp{g}0%
      \PdfRender@FindOp{G}1%
      \PdfRender@FindOp{rg}0%
      \PdfRender@FindOp{RG}1%
      \PdfRender@FindOp{k}0%
      \PdfRender@FindOp{K}1%
      \PdfRender@FilterOp 1\PdfRender@StrokeColor
    \fi
  \endgroup
}
%    \end{macrocode}
%    \end{macro}
%    \begin{macrocode}
\ifPdfRender@Match
  \expandafter\PdfRender@AtEnd
\fi%
%    \end{macrocode}
%    \begin{macro}{\PdfRender@FindOp}
%    \begin{macrocode}
\def\PdfRender@FindOp#1#2{%
  \def\PdfRender@temp##1 #1 ##2\@nil{%
    ##1%
    \ifx\\##2\\%
      \expandafter\@gobble
    \else
      \PdfRender@OP{#1}#2%
      \expandafter\@firstofone
    \fi
    {%
      \PdfRender@temp##2\@nil
    }%
  }%
  \edef\current@color{%
    \@firstofone{\expandafter\PdfRender@temp\current@color} #1 \@nil
  }%
}
%    \end{macrocode}
%    \end{macro}
%    \begin{macro}{\PdfRender@FilterOp}
%    \begin{macrocode}
\def\PdfRender@FilterOp#1#2{%
  \expandafter\PdfRender@@FilterOp\expandafter#1\expandafter#2%
      \current@color\PdfRender@OP{}{}%
}
%    \end{macrocode}
%    \end{macro}
%    \begin{macro}{\PdfRender@@FilterOp}
%    \begin{macrocode}
\def\PdfRender@@FilterOp#1#2#3\PdfRender@OP#4#5{%
  \ifx\\#4#5\\%
  \else
    \ifnum#1=#5 %
      \xdef#2{#3 #4}%
    \fi
    \expandafter\PdfRender@@FilterOp\expandafter#1\expandafter#2%
  \fi
}
%    \end{macrocode}
%    \end{macro}
%
%    \begin{macrocode}
\PdfRender@AtEnd%
%</package>
%    \end{macrocode}
%% \section{Installation}
%
% \subsection{Download}
%
% \paragraph{Package.} This package is available on
% CTAN\footnote{\CTANpkg{pdfrender}}:
% \begin{description}
% \item[\CTAN{macros/latex/contrib/oberdiek/pdfrender.dtx}] The source file.
% \item[\CTAN{macros/latex/contrib/oberdiek/pdfrender.pdf}] Documentation.
% \end{description}
%
%
% \paragraph{Bundle.} All the packages of the bundle `oberdiek'
% are also available in a TDS compliant ZIP archive. There
% the packages are already unpacked and the documentation files
% are generated. The files and directories obey the TDS standard.
% \begin{description}
% \item[\CTANinstall{install/macros/latex/contrib/oberdiek.tds.zip}]
% \end{description}
% \emph{TDS} refers to the standard ``A Directory Structure
% for \TeX\ Files'' (\CTANpkg{tds}). Directories
% with \xfile{texmf} in their name are usually organized this way.
%
% \subsection{Bundle installation}
%
% \paragraph{Unpacking.} Unpack the \xfile{oberdiek.tds.zip} in the
% TDS tree (also known as \xfile{texmf} tree) of your choice.
% Example (linux):
% \begin{quote}
%   |unzip oberdiek.tds.zip -d ~/texmf|
% \end{quote}
%
% \subsection{Package installation}
%
% \paragraph{Unpacking.} The \xfile{.dtx} file is a self-extracting
% \docstrip\ archive. The files are extracted by running the
% \xfile{.dtx} through \plainTeX:
% \begin{quote}
%   \verb|tex pdfrender.dtx|
% \end{quote}
%
% \paragraph{TDS.} Now the different files must be moved into
% the different directories in your installation TDS tree
% (also known as \xfile{texmf} tree):
% \begin{quote}
% \def\t{^^A
% \begin{tabular}{@{}>{\ttfamily}l@{ $\rightarrow$ }>{\ttfamily}l@{}}
%   pdfrender.sty & tex/generic/oberdiek/pdfrender.sty\\
%   pdfrender.pdf & doc/latex/oberdiek/pdfrender.pdf\\
%   pdfrender.dtx & source/latex/oberdiek/pdfrender.dtx\\
% \end{tabular}^^A
% }^^A
% \sbox0{\t}^^A
% \ifdim\wd0>\linewidth
%   \begingroup
%     \advance\linewidth by\leftmargin
%     \advance\linewidth by\rightmargin
%   \edef\x{\endgroup
%     \def\noexpand\lw{\the\linewidth}^^A
%   }\x
%   \def\lwbox{^^A
%     \leavevmode
%     \hbox to \linewidth{^^A
%       \kern-\leftmargin\relax
%       \hss
%       \usebox0
%       \hss
%       \kern-\rightmargin\relax
%     }^^A
%   }^^A
%   \ifdim\wd0>\lw
%     \sbox0{\small\t}^^A
%     \ifdim\wd0>\linewidth
%       \ifdim\wd0>\lw
%         \sbox0{\footnotesize\t}^^A
%         \ifdim\wd0>\linewidth
%           \ifdim\wd0>\lw
%             \sbox0{\scriptsize\t}^^A
%             \ifdim\wd0>\linewidth
%               \ifdim\wd0>\lw
%                 \sbox0{\tiny\t}^^A
%                 \ifdim\wd0>\linewidth
%                   \lwbox
%                 \else
%                   \usebox0
%                 \fi
%               \else
%                 \lwbox
%               \fi
%             \else
%               \usebox0
%             \fi
%           \else
%             \lwbox
%           \fi
%         \else
%           \usebox0
%         \fi
%       \else
%         \lwbox
%       \fi
%     \else
%       \usebox0
%     \fi
%   \else
%     \lwbox
%   \fi
% \else
%   \usebox0
% \fi
% \end{quote}
% If you have a \xfile{docstrip.cfg} that configures and enables \docstrip's
% TDS installing feature, then some files can already be in the right
% place, see the documentation of \docstrip.
%
% \subsection{Refresh file name databases}
%
% If your \TeX~distribution
% (\TeX\,Live, \mikTeX, \dots) relies on file name databases, you must refresh
% these. For example, \TeX\,Live\ users run \verb|texhash| or
% \verb|mktexlsr|.
%
% \subsection{Some details for the interested}
%
% \paragraph{Unpacking with \LaTeX.}
% The \xfile{.dtx} chooses its action depending on the format:
% \begin{description}
% \item[\plainTeX:] Run \docstrip\ and extract the files.
% \item[\LaTeX:] Generate the documentation.
% \end{description}
% If you insist on using \LaTeX\ for \docstrip\ (really,
% \docstrip\ does not need \LaTeX), then inform the autodetect routine
% about your intention:
% \begin{quote}
%   \verb|latex \let\install=y\input{pdfrender.dtx}|
% \end{quote}
% Do not forget to quote the argument according to the demands
% of your shell.
%
% \paragraph{Generating the documentation.}
% You can use both the \xfile{.dtx} or the \xfile{.drv} to generate
% the documentation. The process can be configured by the
% configuration file \xfile{ltxdoc.cfg}. For instance, put this
% line into this file, if you want to have A4 as paper format:
% \begin{quote}
%   \verb|\PassOptionsToClass{a4paper}{article}|
% \end{quote}
% An example follows how to generate the
% documentation with pdf\LaTeX:
% \begin{quote}
%\begin{verbatim}
%pdflatex pdfrender.dtx
%makeindex -s gind.ist pdfrender.idx
%pdflatex pdfrender.dtx
%makeindex -s gind.ist pdfrender.idx
%pdflatex pdfrender.dtx
%\end{verbatim}
% \end{quote}
%
% \section{Acknowledgement}
%
% \begin{description}
% \item[Friedrich Vosberg] asked in the
%   newsgroup \textsf{de.comp.text.tex} for the font
%   outline feature \cite{news-Vosberg}.
% \item[Gaius Pupus] proposed the basic method
%   using \cs{pdfliteral} in this thread \cite{news-Pupus}.
% \item[Rolf Niepraschk] added color support \cite{news-Niepraschk}.
% \end{description}
%
% \begin{thebibliography}{9}
% \raggedright
%
% \bibitem{pdf-spec}
%   Adobe Systems Incorporated. \textit{PDF Reference --
%   Adobe Portable Document format -- Version 1.7}.
%   6th ed. 2006.
%   \textsc{URL:}\,\url{http://www.adobe.com/devnet/acrobat/pdfs/pdf_reference_1-7.pdf}.
%
% \bibitem{news-Vosberg}
%   Friedrich Vosberg, \textit{Text in Buchstabenumrissen},
%   de.comp.text.tex, 2010-01-22.
%   \textsc{URL:}\,\url{https://groups.google.com/group/de.comp.text.tex/msg/f442310ac8b2d506}.
%
% \bibitem{news-Pupus}
%   Gaius Pupus, \textit{Re: Text in Buchstabenumrissen},
%   de.comp.text.tex, 2010-01-23.
%   \textsc{URL:}\,\url{https://groups.google.com/group/de.comp.text.tex/msg/95d890d77ac47eb1}.
%
% \bibitem{news-Niepraschk}
%   Rolf Niepraschk, \textit{Re: Text in Buchstabenumrissen},
%   de.comp.text.tex, 2010-01-24.
%   \textsc{URL:}\,\url{https://groups.google.com/group/de.comp.text.tex/msg/4eb61a5879db54db}.
%
% \end{thebibliography}
%
% \begin{History}
%   \begin{Version}{2010/01/26 v1.0}
%   \item
%     The first version.
%   \end{Version}
%   \begin{Version}{2010/01/27 v1.1}
%   \item
%     Macros \cs{pdfrender} and \cs{textpdfrender} are made robust.
%   \item
%     Color extraction rewritten for the case that \cs{pdfmatch}
%     is not available. This fixes wrong color assigments in
%     case of nesting.
%   \item
%     Color extraction of case \cs{pdfmatch} is fixed
%     for the case that the color string contains several
%     fill or several stroke operations.
%   \end{Version}
%   \begin{Version}{2010/01/28 v1.2}
%   \item
%     Dependency from package \xpackage{color} is removed.
%   \item
%     Compatibility for \plainTeX\ and even ini\TeX\ added.
%   \end{Version}
%   \begin{Version}{2016/05/14 v1.3}
%   \item
%     Use package \xpackage{luatex85} for compatibility with new LuaTeX.
%   \end{Version}
%   \begin{Version}{2016/05/17 v1.4}
%   \item
%     Documentation updates.
%   \item adjust luatex85 reference so that it works in plain TeX.
%   \end{Version}
%   \begin{Version}{2018/11/01 v1.5}
%   \item
%     Remove luatex85 dependency
%   \end{Version}
%   \begin{Version}{2019/12/29 v1.6}
%   \item
%     \xpackage{iftex} package.
%   \end{Version}
% \end{History}
%
% \PrintIndex
%
% \Finale
\endinput
|
% \end{quote}
% Do not forget to quote the argument according to the demands
% of your shell.
%
% \paragraph{Generating the documentation.}
% You can use both the \xfile{.dtx} or the \xfile{.drv} to generate
% the documentation. The process can be configured by the
% configuration file \xfile{ltxdoc.cfg}. For instance, put this
% line into this file, if you want to have A4 as paper format:
% \begin{quote}
%   \verb|\PassOptionsToClass{a4paper}{article}|
% \end{quote}
% An example follows how to generate the
% documentation with pdf\LaTeX:
% \begin{quote}
%\begin{verbatim}
%pdflatex pdfrender.dtx
%makeindex -s gind.ist pdfrender.idx
%pdflatex pdfrender.dtx
%makeindex -s gind.ist pdfrender.idx
%pdflatex pdfrender.dtx
%\end{verbatim}
% \end{quote}
%
% \section{Acknowledgement}
%
% \begin{description}
% \item[Friedrich Vosberg] asked in the
%   newsgroup \textsf{de.comp.text.tex} for the font
%   outline feature \cite{news-Vosberg}.
% \item[Gaius Pupus] proposed the basic method
%   using \cs{pdfliteral} in this thread \cite{news-Pupus}.
% \item[Rolf Niepraschk] added color support \cite{news-Niepraschk}.
% \end{description}
%
% \begin{thebibliography}{9}
% \raggedright
%
% \bibitem{pdf-spec}
%   Adobe Systems Incorporated. \textit{PDF Reference --
%   Adobe Portable Document format -- Version 1.7}.
%   6th ed. 2006.
%   \textsc{URL:}\,\url{http://www.adobe.com/devnet/acrobat/pdfs/pdf_reference_1-7.pdf}.
%
% \bibitem{news-Vosberg}
%   Friedrich Vosberg, \textit{Text in Buchstabenumrissen},
%   de.comp.text.tex, 2010-01-22.
%   \textsc{URL:}\,\url{https://groups.google.com/group/de.comp.text.tex/msg/f442310ac8b2d506}.
%
% \bibitem{news-Pupus}
%   Gaius Pupus, \textit{Re: Text in Buchstabenumrissen},
%   de.comp.text.tex, 2010-01-23.
%   \textsc{URL:}\,\url{https://groups.google.com/group/de.comp.text.tex/msg/95d890d77ac47eb1}.
%
% \bibitem{news-Niepraschk}
%   Rolf Niepraschk, \textit{Re: Text in Buchstabenumrissen},
%   de.comp.text.tex, 2010-01-24.
%   \textsc{URL:}\,\url{https://groups.google.com/group/de.comp.text.tex/msg/4eb61a5879db54db}.
%
% \end{thebibliography}
%
% \begin{History}
%   \begin{Version}{2010/01/26 v1.0}
%   \item
%     The first version.
%   \end{Version}
%   \begin{Version}{2010/01/27 v1.1}
%   \item
%     Macros \cs{pdfrender} and \cs{textpdfrender} are made robust.
%   \item
%     Color extraction rewritten for the case that \cs{pdfmatch}
%     is not available. This fixes wrong color assigments in
%     case of nesting.
%   \item
%     Color extraction of case \cs{pdfmatch} is fixed
%     for the case that the color string contains several
%     fill or several stroke operations.
%   \end{Version}
%   \begin{Version}{2010/01/28 v1.2}
%   \item
%     Dependency from package \xpackage{color} is removed.
%   \item
%     Compatibility for \plainTeX\ and even ini\TeX\ added.
%   \end{Version}
%   \begin{Version}{2016/05/14 v1.3}
%   \item
%     Use package \xpackage{luatex85} for compatibility with new LuaTeX.
%   \end{Version}
%   \begin{Version}{2016/05/17 v1.4}
%   \item
%     Documentation updates.
%   \item adjust luatex85 reference so that it works in plain TeX.
%   \end{Version}
%   \begin{Version}{2018/11/01 v1.5}
%   \item
%     Remove luatex85 dependency
%   \end{Version}
%   \begin{Version}{2019/12/29 v1.6}
%   \item
%     \xpackage{iftex} package.
%   \end{Version}
% \end{History}
%
% \PrintIndex
%
% \Finale
\endinput
|
% \end{quote}
% Do not forget to quote the argument according to the demands
% of your shell.
%
% \paragraph{Generating the documentation.}
% You can use both the \xfile{.dtx} or the \xfile{.drv} to generate
% the documentation. The process can be configured by the
% configuration file \xfile{ltxdoc.cfg}. For instance, put this
% line into this file, if you want to have A4 as paper format:
% \begin{quote}
%   \verb|\PassOptionsToClass{a4paper}{article}|
% \end{quote}
% An example follows how to generate the
% documentation with pdf\LaTeX:
% \begin{quote}
%\begin{verbatim}
%pdflatex pdfrender.dtx
%makeindex -s gind.ist pdfrender.idx
%pdflatex pdfrender.dtx
%makeindex -s gind.ist pdfrender.idx
%pdflatex pdfrender.dtx
%\end{verbatim}
% \end{quote}
%
% \section{Acknowledgement}
%
% \begin{description}
% \item[Friedrich Vosberg] asked in the
%   newsgroup \textsf{de.comp.text.tex} for the font
%   outline feature \cite{news-Vosberg}.
% \item[Gaius Pupus] proposed the basic method
%   using \cs{pdfliteral} in this thread \cite{news-Pupus}.
% \item[Rolf Niepraschk] added color support \cite{news-Niepraschk}.
% \end{description}
%
% \begin{thebibliography}{9}
% \raggedright
%
% \bibitem{pdf-spec}
%   Adobe Systems Incorporated. \textit{PDF Reference --
%   Adobe Portable Document format -- Version 1.7}.
%   6th ed. 2006.
%   \textsc{URL:}\,\url{http://www.adobe.com/devnet/acrobat/pdfs/pdf_reference_1-7.pdf}.
%
% \bibitem{news-Vosberg}
%   Friedrich Vosberg, \textit{Text in Buchstabenumrissen},
%   de.comp.text.tex, 2010-01-22.
%   \textsc{URL:}\,\url{https://groups.google.com/group/de.comp.text.tex/msg/f442310ac8b2d506}.
%
% \bibitem{news-Pupus}
%   Gaius Pupus, \textit{Re: Text in Buchstabenumrissen},
%   de.comp.text.tex, 2010-01-23.
%   \textsc{URL:}\,\url{https://groups.google.com/group/de.comp.text.tex/msg/95d890d77ac47eb1}.
%
% \bibitem{news-Niepraschk}
%   Rolf Niepraschk, \textit{Re: Text in Buchstabenumrissen},
%   de.comp.text.tex, 2010-01-24.
%   \textsc{URL:}\,\url{https://groups.google.com/group/de.comp.text.tex/msg/4eb61a5879db54db}.
%
% \end{thebibliography}
%
% \begin{History}
%   \begin{Version}{2010/01/26 v1.0}
%   \item
%     The first version.
%   \end{Version}
%   \begin{Version}{2010/01/27 v1.1}
%   \item
%     Macros \cs{pdfrender} and \cs{textpdfrender} are made robust.
%   \item
%     Color extraction rewritten for the case that \cs{pdfmatch}
%     is not available. This fixes wrong color assigments in
%     case of nesting.
%   \item
%     Color extraction of case \cs{pdfmatch} is fixed
%     for the case that the color string contains several
%     fill or several stroke operations.
%   \end{Version}
%   \begin{Version}{2010/01/28 v1.2}
%   \item
%     Dependency from package \xpackage{color} is removed.
%   \item
%     Compatibility for \plainTeX\ and even ini\TeX\ added.
%   \end{Version}
%   \begin{Version}{2016/05/14 v1.3}
%   \item
%     Use package \xpackage{luatex85} for compatibility with new LuaTeX.
%   \end{Version}
%   \begin{Version}{2016/05/17 v1.4}
%   \item
%     Documentation updates.
%   \item adjust luatex85 reference so that it works in plain TeX.
%   \end{Version}
%   \begin{Version}{2018/11/01 v1.5}
%   \item
%     Remove luatex85 dependency
%   \end{Version}
%   \begin{Version}{2019/12/29 v1.6}
%   \item
%     \xpackage{iftex} package.
%   \end{Version}
% \end{History}
%
% \PrintIndex
%
% \Finale
\endinput

%        (quote the arguments according to the demands of your shell)
%
% Documentation:
%    (a) If pdfrender.drv is present:
%           latex pdfrender.drv
%    (b) Without pdfrender.drv:
%           latex pdfrender.dtx; ...
%    The class ltxdoc loads the configuration file ltxdoc.cfg
%    if available. Here you can specify further options, e.g.
%    use A4 as paper format:
%       \PassOptionsToClass{a4paper}{article}
%
%    Programm calls to get the documentation (example):
%       pdflatex pdfrender.dtx
%       makeindex -s gind.ist pdfrender.idx
%       pdflatex pdfrender.dtx
%       makeindex -s gind.ist pdfrender.idx
%       pdflatex pdfrender.dtx
%
% Installation:
%    TDS:tex/generic/oberdiek/pdfrender.sty
%    TDS:doc/latex/oberdiek/pdfrender.pdf
%    TDS:source/latex/oberdiek/pdfrender.dtx
%
%<*ignore>
\begingroup
  \catcode123=1 %
  \catcode125=2 %
  \def\x{LaTeX2e}%
\expandafter\endgroup
\ifcase 0\ifx\install y1\fi\expandafter
         \ifx\csname processbatchFile\endcsname\relax\else1\fi
         \ifx\fmtname\x\else 1\fi\relax
\else\csname fi\endcsname
%</ignore>
%<*install>
\input docstrip.tex
\Msg{************************************************************************}
\Msg{* Installation}
\Msg{* Package: pdfrender 2019/12/29 v1.6 Access to some PDF graphics parameters (HO)}
\Msg{************************************************************************}

\keepsilent
\askforoverwritefalse

\let\MetaPrefix\relax
\preamble

This is a generated file.

Project: pdfrender
Version: 2019/12/29 v1.6

Copyright (C)
   2010 Heiko Oberdiek
   2016-2019 Oberdiek Package Support Group

This work may be distributed and/or modified under the
conditions of the LaTeX Project Public License, either
version 1.3c of this license or (at your option) any later
version. This version of this license is in
   https://www.latex-project.org/lppl/lppl-1-3c.txt
and the latest version of this license is in
   https://www.latex-project.org/lppl.txt
and version 1.3 or later is part of all distributions of
LaTeX version 2005/12/01 or later.

This work has the LPPL maintenance status "maintained".

The Current Maintainers of this work are
Heiko Oberdiek and the Oberdiek Package Support Group
https://github.com/ho-tex/oberdiek/issues


The Base Interpreter refers to any `TeX-Format',
because some files are installed in TDS:tex/generic//.

This work consists of the main source file pdfrender.dtx
and the derived files
   pdfrender.sty, pdfrender.pdf, pdfrender.ins, pdfrender.drv,
   pdfrender-test1.tex, pdfrender-test2.tex,
   pdfrender-test3.tex, pdfrender-test4.tex,
   pdfrender-test5.tex.

\endpreamble
\let\MetaPrefix\DoubleperCent

\generate{%
  \file{pdfrender.ins}{\from{pdfrender.dtx}{install}}%
  \file{pdfrender.drv}{\from{pdfrender.dtx}{driver}}%
  \usedir{tex/generic/oberdiek}%
  \file{pdfrender.sty}{\from{pdfrender.dtx}{package}}%
%  \usedir{doc/latex/oberdiek/test}%
%  \file{pdfrender-test1.tex}{\from{pdfrender.dtx}{test1}}%
%  \file{pdfrender-test2.tex}{\from{pdfrender.dtx}{test2}}%
%  \file{pdfrender-test3.tex}{\from{pdfrender.dtx}{test3}}%
%  \file{pdfrender-test4.tex}{\from{pdfrender.dtx}{test4}}%
%  \file{pdfrender-test5.tex}{\from{pdfrender.dtx}{test5}}%
}

\catcode32=13\relax% active space
\let =\space%
\Msg{************************************************************************}
\Msg{*}
\Msg{* To finish the installation you have to move the following}
\Msg{* file into a directory searched by TeX:}
\Msg{*}
\Msg{*     pdfrender.sty}
\Msg{*}
\Msg{* To produce the documentation run the file `pdfrender.drv'}
\Msg{* through LaTeX.}
\Msg{*}
\Msg{* Happy TeXing!}
\Msg{*}
\Msg{************************************************************************}

\endbatchfile
%</install>
%<*ignore>
\fi
%</ignore>
%<*driver>
\NeedsTeXFormat{LaTeX2e}
\ProvidesFile{pdfrender.drv}%
  [2019/12/29 v1.6 Access to some PDF graphics parameters (HO)]%
\documentclass{ltxdoc}
\usepackage{holtxdoc}[2011/11/22]
\usepackage{longtable}
\renewcommand*{\arraystretch}{1.2}
\begin{document}
  \DocInput{pdfrender.dtx}%
\end{document}
%</driver>
% \fi
%
%
%
% \GetFileInfo{pdfrender.drv}
%
% \title{The \xpackage{pdfrender} package}
% \date{2019/12/29 v1.6}
% \author{Heiko Oberdiek\thanks
% {Please report any issues at \url{https://github.com/ho-tex/oberdiek/issues}}}
%
% \maketitle
%
% \begin{abstract}
% The PDF format has some graphics parameter like
% line width or text rendering mode. This package
% provides an interface for setting these parameters.
% \end{abstract}
%
% \tableofcontents
%
% \section{Documentation}
%
% This package \xpackage{pdfrender} defines an interface for
% PDF specific parameters that affects the rendering of
% graphics or text. The interface and its implementation
% uses the same technique as package \xpackage{color}
% for color settings. Therefore this package is loaded to
% enable \LaTeX's color interface.
%
% At different places \LaTeX\ uses \cs{normalcolor} to avoid that
% header, footer or floats are print in the current color of the
% main text. \cs{setgroup@color} is used to start a save box
% with the color that is set at box saving time.
% Package \xpackage{pdfrender} extends these macros to
% add its own hooks of its parameters. Therefore \LaTeX3 should
% generalize \LaTeXe's color interface.
%
% \subsection{Usage}
%
% In \LaTeX\ the package is loaded as normal package. Options
% are not defined for this package.
% \begin{quote}
% |\usepackage{pdfrender}|
% \end{quote}
%
% This package can also be used in \plainTeX\ and even ini\TeX:
% \begin{quote}
% |input pdfrender.sty|
% \end{quote}
%
% \subsection{Macros}
%
% \begin{declcs}{pdfrender} \M{key value list}
% \end{declcs}
% The first parameter \meta{key value list} contains a
% list of parameter settings. The key entry is the parameter name.
% The macro works like
% \cs{color} (without optional argument) for color setting.
%
% \begin{declcs}{textpdfrender} \M{key value list} \M{text}
% \end{declcs}
% In the same way as \cs{pdfrender} the first argument
% specifies the parameters that should be set. This parameter
% setting affects \meta{text} only. Basically it works
% the same way as \cs{textcolor} (without optional argument).
%
% \subsection{Parameters}
%
% The following table shows an overview for the supported
% parameters and values:
%
% \begin{longtable}{@{}l|l|l@{}}
% \bfseries Parameter & \bfseries Value & \bfseries Alias\\*
% \hline
% \endhead
% \xoption{TextRenderingMode}
%   & \texttt{0} & \texttt{Fill}\\*
%   & \texttt{1} & \texttt{Stroke}\\*
%   & \texttt{2} & \texttt{FillStroke}\\*
%   & \texttt{3} & \texttt{Invisible}\\*
%   & \texttt{4} & \texttt{FillClip}\\*
%   & \texttt{5} & \texttt{StrokeClip}\\*
%   & \texttt{6} & \texttt{FillStrokeClip}\\*
%   & \texttt{7} & \texttt{Clip}\\
% \hline
% \xoption{LineWidth}
%   & \itshape positive number, & \itshape \TeX\ dimen\\*[-.2ex]
%   & \qquad \itshape unit is bp & \\
% \hline
% \xoption{LineCapStyle}
%   & \texttt{0} & \texttt{Butt}\\*
%   & \texttt{1} & \texttt{Round}\\*
%   & \texttt{2} & \texttt{ProjectingSquare}\\
% \hline
% \xoption{LineJoinStyle}
%   & \texttt{0} & \texttt{Miter}\\*
%   & \texttt{1} & \texttt{Round}\\*
%   & \texttt{2} & \texttt{Bevel}\\
% \hline
% \xoption{MiterLimit}
%   & \multicolumn{2}{l@{}}{\itshape positive number} \\
% \hline
% \xoption{Flatness}
%   & \multicolumn{2}{l@{}}{\itshape number between 0 and 100} \\
% \hline
% \xoption{LineDashPattern}
%   & \multicolumn{2}{l@{}}{^^A
%       \itshape numbers in square brackets,^^A
%     } \\*[-.2ex]
%   & \multicolumn{2}{l@{}}{^^A
%       \qquad \itshape followed by number,^^A
%     } \\*
%   & \multicolumn{2}{l@{}}{^^A
%       \qquad \itshape units are bp^^A
%     } \\
% \hline
% \xoption{RenderingIntent}
%   & \multicolumn{2}{l@{}}{\texttt{AbsoluteColorimetric}} \\*
%   & \multicolumn{2}{l@{}}{\texttt{RelativeColorimetric}} \\*
%   & \multicolumn{2}{l@{}}{\texttt{Saturation}} \\*
%   & \multicolumn{2}{l@{}}{\texttt{Perceptual}} \\
% \hline
% \xoption{FillColor}
%   & & \itshape color specification \\
% \hline
% \xoption{StrokeColor}
%   & & \itshape color specification \\
% \end{longtable}
%
% \subsubsection{Details}
%
% The description and specification of these parameters
% are available in the PDF specification \cite{pdf-spec}.
% Therefore they are not repeated here.
%
% \begin{description}
% \item[Value:]
%   The values in the second column lists or describe
%   the values that are specified by the PDF specification.
% \item[Alias:] Instead of magic numbers the package also
%   defines some aliases that can be given as value.
%   Example: |LineCapStyle=Round| has the same effect as
%   |LineCapStyle=1|.
% \item[Number:] The term \emph{number} means an integer
%   or real number. The real number is given as plain
%   decimal number without exponent. The decimal separator is a period.
%   At least one digit must be present.
% \item[\xoption{LineWidth}:] As alias a \TeX\ dimen specification
%   can be given. This includes explicit specifications with number
%   and unit, e.g. |LineWidth=0.5pt|. Also \LaTeX\ length registers
%   may be used. If \eTeX's \cs{dimexpr} is available, then it is
%   automatically added. However package \xpackage{calc} is not
%   supported.
% \item[\xoption{FillColor}, \xoption{StrokeColor}:]
%   Package \xpackage{color} or \xpackage{xcolor} must be loaded
%   before these options can be used (since version 1.2).
%   \LaTeX's color support sets both colors at the same time
%   to the same value. However parameter \xoption{TextRenderingMode}
%   offers the value \texttt{FillStroke} that makes only sense, if
%   the two color types can be set separately.
%   If one of the options \xoption{FillColor} or \xoption{StrokeColor}
%   is specified, then also the color is set. For compatibility
%   with the \LaTeX\ color packages (\xpackage{color} or \xpackage{xcolor}),
%   always both colors must be set. Thus if one of them is not specified,
%   it is taken from the current color.
%
%   Both options \xoption{FillColor} and \xoption{StrokeColor}
%   expect a \LaTeX\ color specification as value.
%   Also the optional color model argument is supported.
%   Example:
%   \begin{quote}
%     |FillColor=yellow|,\\
%     |StrokeColor=[cmyk]{1,.5,0,0}|\\
%   \end{quote}
% \end{description}
%
% \subsection{Color stack}
%
% If the \pdfTeX\ version provides color stacks, then each
% parameter is assigned a page based color stack. The assignment
% of a stack takes place, when its parameter is set the first time.
% This avoids the use of color stacks that are not needed.
%
% \StopEventually{
% }
%
% \section{Implementation}
%    \begin{macrocode}
%<*package>
%    \end{macrocode}
%    Reload check, especially if the package is not used with \LaTeX.
%    \begin{macrocode}
\begingroup\catcode61\catcode48\catcode32=10\relax%
  \catcode13=5 % ^^M
  \endlinechar=13 %
  \catcode35=6 % #
  \catcode39=12 % '
  \catcode44=12 % ,
  \catcode45=12 % -
  \catcode46=12 % .
  \catcode58=12 % :
  \catcode64=11 % @
  \catcode123=1 % {
  \catcode125=2 % }
  \expandafter\let\expandafter\x\csname ver@pdfrender.sty\endcsname
  \ifx\x\relax % plain-TeX, first loading
  \else
    \def\empty{}%
    \ifx\x\empty % LaTeX, first loading,
      % variable is initialized, but \ProvidesPackage not yet seen
    \else
      \expandafter\ifx\csname PackageInfo\endcsname\relax
        \def\x#1#2{%
          \immediate\write-1{Package #1 Info: #2.}%
        }%
      \else
        \def\x#1#2{\PackageInfo{#1}{#2, stopped}}%
      \fi
      \x{pdfrender}{The package is already loaded}%
      \aftergroup\endinput
    \fi
  \fi
\endgroup%
%    \end{macrocode}
%    Package identification:
%    \begin{macrocode}
\begingroup\catcode61\catcode48\catcode32=10\relax%
  \catcode13=5 % ^^M
  \endlinechar=13 %
  \catcode35=6 % #
  \catcode39=12 % '
  \catcode40=12 % (
  \catcode41=12 % )
  \catcode44=12 % ,
  \catcode45=12 % -
  \catcode46=12 % .
  \catcode47=12 % /
  \catcode58=12 % :
  \catcode64=11 % @
  \catcode91=12 % [
  \catcode93=12 % ]
  \catcode123=1 % {
  \catcode125=2 % }
  \expandafter\ifx\csname ProvidesPackage\endcsname\relax
    \def\x#1#2#3[#4]{\endgroup
      \immediate\write-1{Package: #3 #4}%
      \xdef#1{#4}%
    }%
  \else
    \def\x#1#2[#3]{\endgroup
      #2[{#3}]%
      \ifx#1\@undefined
        \xdef#1{#3}%
      \fi
      \ifx#1\relax
        \xdef#1{#3}%
      \fi
    }%
  \fi
\expandafter\x\csname ver@pdfrender.sty\endcsname
\ProvidesPackage{pdfrender}%
  [2019/12/29 v1.6 Access to some PDF graphics parameters (HO)]%
%    \end{macrocode}
%
%    \begin{macrocode}
\begingroup\catcode61\catcode48\catcode32=10\relax%
  \catcode13=5 % ^^M
  \endlinechar=13 %
  \catcode123=1 % {
  \catcode125=2 % }
  \catcode64=11 % @
  \def\x{\endgroup
    \expandafter\edef\csname PdfRender@AtEnd\endcsname{%
      \endlinechar=\the\endlinechar\relax
      \catcode13=\the\catcode13\relax
      \catcode32=\the\catcode32\relax
      \catcode35=\the\catcode35\relax
      \catcode61=\the\catcode61\relax
      \catcode64=\the\catcode64\relax
      \catcode123=\the\catcode123\relax
      \catcode125=\the\catcode125\relax
    }%
  }%
\x\catcode61\catcode48\catcode32=10\relax%
\catcode13=5 % ^^M
\endlinechar=13 %
\catcode35=6 % #
\catcode64=11 % @
\catcode123=1 % {
\catcode125=2 % }
\def\TMP@EnsureCode#1#2{%
  \edef\PdfRender@AtEnd{%
    \PdfRender@AtEnd
    \catcode#1=\the\catcode#1\relax
  }%
  \catcode#1=#2\relax
}
\TMP@EnsureCode{10}{12}% ^^J
\TMP@EnsureCode{36}{3}% $
\TMP@EnsureCode{39}{12}% '
\TMP@EnsureCode{40}{12}% (
\TMP@EnsureCode{41}{12}% )
\TMP@EnsureCode{42}{12}% *
\TMP@EnsureCode{43}{12}% +
\TMP@EnsureCode{44}{12}% ,
\TMP@EnsureCode{45}{12}% -
\TMP@EnsureCode{46}{12}% .
\TMP@EnsureCode{47}{12}% /
\TMP@EnsureCode{58}{12}% :
\TMP@EnsureCode{59}{12}% ;
\TMP@EnsureCode{60}{12}% <
\TMP@EnsureCode{62}{12}% >
\TMP@EnsureCode{63}{12}% ?
\TMP@EnsureCode{91}{12}% [
\TMP@EnsureCode{93}{12}% ]
\TMP@EnsureCode{94}{7}% ^ (superscript)
\TMP@EnsureCode{96}{12}% `
\TMP@EnsureCode{124}{12}% |
%    \end{macrocode}
%
%    \begin{macrocode}
\def\PdfRender@AtEndHook{}
\expandafter\def\expandafter\PdfRender@AtEnd\expandafter{%
  \expandafter\PdfRender@AtEndHook
  \PdfRender@AtEnd
  \endinput
}
%    \end{macrocode}
%
% \subsection{Look for \pdfTeX, its mode and features}
%
%    \begin{macro}{\PdfRender@newif}
%    \begin{macrocode}
\def\PdfRender@newif#1{%
  \expandafter\edef\csname PdfRender@#1true\endcsname{%
    \let
    \expandafter\noexpand\csname ifPdfRender@#1\endcsname
    \noexpand\iftrue
  }%
  \expandafter\edef\csname PdfRender@#1false\endcsname{%
    \let
    \expandafter\noexpand\csname ifPdfRender@#1\endcsname
    \noexpand\iffalse
  }%
  \csname PdfRender@#1false\endcsname
}
%    \end{macrocode}
%    \end{macro}
%
%    \begin{macro}{\ifPdfRender@Stack}
%    \begin{macrocode}
\PdfRender@newif{Stack}
%    \end{macrocode}
%    \end{macro}
%    \begin{macro}{\ifPdfRender@Match}
%    \begin{macrocode}
\PdfRender@newif{Match}
%    \end{macrocode}
%    \end{macro}
%
%    \begin{macro}{\PdfRender@RequirePackage}
%    \begin{macrocode}
\begingroup\expandafter\expandafter\expandafter\endgroup
\expandafter\ifx\csname RequirePackage\endcsname\relax
  \def\PdfRender@RequirePackage#1[#2]{%
    \expandafter\def\expandafter\PdfRender@AtEndHook\expandafter{%
      \PdfRender@AtEndHook
      \ltx@ifpackagelater{#1}{#2}{}{%
        \@PackageWarningNoLine{pdfrender}{%
          You have requested version\MessageBreak
          `#2' of package `#1',\MessageBreak
          but only version\MessageBreak
          `\csname ver@#1.\ltx@pkgextension\endcsname'\MessageBreak
          is available%
        }%
      }%
    }%
    \input #1.sty\relax
  }%
\else
  \let\PdfRender@RequirePackage\RequirePackage
\fi
%    \end{macrocode}
%    \end{macro}
% Luatex compatibility
%    \begin{macrocode}
\ifx\pdfextension\@undefined\else
  \def\pdfcolorstackinit {\pdffeedback colorstackinit}
  \protected\def\pdfcolorstack     {\pdfextension colorstack}
  \protected\def\pdfliteral        {\pdfextension literal}
\fi
%    \end{macrocode}
%
%    \begin{macrocode}
\PdfRender@RequirePackage{iftex}[2019/11/07]
\PdfRender@RequirePackage{infwarerr}[2007/09/09]
\PdfRender@RequirePackage{ltxcmds}[2010/01/28]
%    \end{macrocode}
%
%    \begin{macrocode}
\ifpdf
  \ltx@IfUndefined{pdfcolorstackinit}{%
    \@PackageWarning{pdfrender}{%
      Missing \string\pdfcolorstackinit
    }%
  }{%
    \PdfRender@Stacktrue
  }%
  \ltx@IfUndefined{pdfmatch}{%
    \@PackageInfoNoLine{pdfrender}{%
      \string\pdfmatch\ltx@space not found. %
      Therefore the values\MessageBreak
      of some parameters are not validated%
    }%
  }{%
    \PdfRender@Matchtrue
  }%
\else
  \@PackageWarning{pdfrender}{%
    Missing pdfTeX in PDF mode%
  }%
%    \end{macrocode}
%    \begin{macrocode}
  \ltx@IfUndefined{newcommand}{%
%    \end{macrocode}
%    \begin{macro}{\pdfrender}
%    \begin{macrocode}
    \def\pdfrender#1{}%
%    \end{macrocode}
%    \end{macro}
%    \begin{macro}{\textpdfrender}
%    \begin{macrocode}
    \long\def\textpdfrender#1#2{#2}%
%    \end{macrocode}
%    \end{macro}
%    \begin{macrocode}
  }{%
%    \end{macrocode}
%    \begin{macro}{\pdfrender}
%    \begin{macrocode}
    \newcommand*{\pdfrender}[1]{}%
%    \end{macrocode}
%    \end{macro}
%    \begin{macro}{\textpdfrender}
%    \begin{macrocode}
    \newcommand{\textpdfrender}[2]{#2}%
%    \end{macrocode}
%    \end{macro}
%    \begin{macrocode}
  }%
  \expandafter\PdfRender@AtEnd
\fi%
%    \end{macrocode}
%
% \subsection{Enable color support of \LaTeX}
%
%    \begin{macrocode}
\ltx@ifpackageloaded{color}{}{%
  \def\color@setgroup{\begingroup\set@color}%
  \let\color@begingroup\begingroup
  \def\color@endgroup{\endgraf\endgroup}%
  \def\color@hbox{\hbox\bgroup\color@begingroup}%
  \def\color@vbox{\vbox\bgroup\color@begingroup}%
  \def\color@endbox{\color@endgroup\egroup}%
  \ltx@ifundefined{bgroup}{%
    \let\bgroup={\let\egroup=}%
  }{}%
  \ltx@ifundefined{endgraf}{%
    \let\endgraf=\par
  }{}%
}
%    \end{macrocode}
%
% \subsection{Hook into \cs{normalcolor}}
%
%    The problem is that packages \xpackage{color} and \xpackage{xcolor}
%    each overwrite \cs{normalcolor}. For example, after the package
%    loading order \xpackage{color}, \xpackage{pdfrender} and \xpackage{xcolor}
%    the patched version of \cs{normalcolor} is overwritten by
%    package \xpackage{xcolor}. Also using \cs{AtBeginDocument} for
%    patching is not enough. If package \xpackage{hyperref} is loaded
%    later, it might load package \xpackage{color} using \cs{AtBeginDocument}.
%    \begin{macro}{\PdfRender@NormalColorHook}
%    \begin{macrocode}
\def\PdfRender@NormalColorHook{}
%    \end{macrocode}
%    \end{macro}
%    \begin{macro}{\PdfRender@ColorSetGroupHook}
%    \begin{macrocode}
\def\PdfRender@ColorSetGroupHook{}
%    \end{macrocode}
%    \end{macro}
%    \begin{macro}{\PdfRender@TestBox}
%    \begin{macrocode}
\def\PdfRender@TestBox#1{%
  \setbox0=\color@hbox#1\color@endbox
}
%    \end{macrocode}
%    \end{macro}
%    \begin{macro}{\PdfRender@PatchNormalColor}
%    \begin{macrocode}
\def\PdfRender@PatchNormalColor{%
  \ltx@ifundefined{normalcolor}{%
    \gdef\normalcolor{\PdfRender@NormalColorHook}%
  }{%
    \begingroup
      \def\PdfRender@NormalColorHook{\let\PdfRender@temp=Y}%
      \PdfRender@TestBox{%
        \let\set@color\relax
        \normalcolor
        \ifx\PdfRender@temp Y%
        \else
          \ltx@GlobalAppendToMacro\normalcolor{%
            \PdfRender@NormalColorHook
          }%
        \fi
      }%
    \endgroup
  }%
  \ifx\@nodocument\relax
    \global\let\PdfRender@PatchNormalColor\relax
  \fi
}%
%    \end{macrocode}
%    \end{macro}
%    \begin{macro}{\PdfRender@PatchColorSetGroup}
%    \begin{macrocode}
\def\PdfRender@PatchColorSetGroup{%
  \begingroup
    \def\PdfRender@ColorSetGroupHook{\let\PdfRender@temp=Y}%
    \PdfRender@TestBox{%
      \let\set@color\relax
      \color@setgroup\color@endgroup
      \ifx\PdfRender@temp Y%
      \else
        \ltx@GlobalAppendToMacro\color@setgroup{%
          \PdfRender@ColorSetGroupHook
        }%
      \fi
    }%
  \endgroup
  \ifx\@nodocument\relax
    \global\let\PdfRender@PatchColorSetGroup\relax
  \fi
}%
%    \end{macrocode}
%    \end{macro}
%    \begin{macro}{\PdfRender@PatchColor}
%    \begin{macrocode}
\def\PdfRender@PatchColor{%
  \PdfRender@PatchNormalColor
  \PdfRender@PatchColorSetGroup
}
%    \end{macrocode}
%    \end{macro}
%    \begin{macrocode}
\PdfRender@PatchColor
\ltx@IfUndefined{AtBeginDocument}{}{%
  \AtBeginDocument{\PdfRender@PatchColor}%
}
%    \end{macrocode}
%    \cs{AfterPackage} is provided by package \xpackage{scrlfile}.
%    \begin{macrocode}
\ltx@IfUndefined{AfterPackage}{%
}{%
  \AfterPackage{color}{\PdfRender@PatchColor}%
  \AfterPackage{xcolor}{\PdfRender@PatchColor}%
  \AfterPackage{etoolbox}{%
    \AfterEndPreamble{\PdfRender@PatchColor}%
  }%
}%
%    \end{macrocode}
%    \cs{AfterEndPreamble} is provided by package \xpackage{etoolbox}.
%    \begin{macrocode}
\ltx@IfUndefined{AfterEndPreamble}{%
}{%
  \AfterEndPreamble{\PdfRender@PatchColor}%
}%
%    \end{macrocode}
%
%    \begin{macrocode}
\PdfRender@RequirePackage{kvsetkeys}[2010/01/28]
%    \end{macrocode}
%    \begin{macro}{\PdfRender@texorpdfstring}
%    \begin{macrocode}
\def\PdfRender@texorpdfstring{%
  \ltx@IfUndefined{texorpdfstring}\ltx@firstoftwo\texorpdfstring
}
%    \end{macrocode}
%    \end{macro}
%    \begin{macro}{\pdfrender}
%    \begin{macrocode}
\ltx@IfUndefined{DeclareRobustCommand}%
\ltx@firstoftwo\ltx@secondoftwo
{%
  \def\pdfrender#1%
}{%
  \newcommand{\pdfrender}{}%
  \DeclareRobustCommand*{\pdfrender}[1]%
}%
{%
  \PdfRender@texorpdfstring{%
    \PdfRender@PatchNormalColor
    \global\let\PdfRender@FillColor\ltx@empty
    \global\let\PdfRender@StrokeColor\ltx@empty
    \kvsetkeys{PDFRENDER}{#1}%
    \PdfRender@SetColor
  }{}%
}
%    \end{macrocode}
%    \end{macro}
%    \begin{macro}{\textpdfrender}
%    \begin{macrocode}
\ltx@IfUndefined{DeclareRobustCommand}%
\ltx@firstoftwo\ltx@secondoftwo
{%
  \long\def\textpdfrender#1#2%
}{%
  \newcommand{\textpdfrender}{}%
  \DeclareRobustCommand{\textpdfrender}[2]%
}%
{%
  \PdfRender@texorpdfstring{%
    \begingroup
      \pdfrender{#1}%
      #2%
    \endgroup
  }{#2}%
}
%    \end{macrocode}
%    \end{macro}
%
%    \begin{macro}{\ifPdfRender@Values}
%    \begin{macrocode}
\PdfRender@newif{Values}
%    \end{macrocode}
%    \end{macro}
%    \begin{macro}{\PdfRender@NewClassValues}
%    \begin{macrocode}
\def\PdfRender@NewClassValues#1#2#3#4{%
  \PdfRender@Valuestrue
  \PdfRender@NewClass{#1}{#2}{#3}{#4}{}%
}
%    \end{macrocode}
%    \end{macro}
%    \begin{macro}{\PdfRender@NewClass}
%    \begin{macrocode}
\def\PdfRender@NewClass#1#2#3#4#5{%
  \PdfRender@newif{Active#1}%
  \expandafter\def\csname PdfRender@Default#1\endcsname{#2}%
  \expandafter\let\csname PdfRender@Current#1\expandafter\endcsname
      \csname PdfRender@Default#1\endcsname
  \ifPdfRender@Stack
    \expandafter\edef\csname PdfRender@Init#1\endcsname{%
      \global\chardef
      \expandafter\noexpand\csname PdfRender@Stack#1\endcsname=%
          \noexpand\pdfcolorstackinit page direct{%
            \noexpand#3%
            \expandafter\noexpand\csname PdfRender@Default#1\endcsname
          }\relax
      \noexpand\@PackageInfo{pdfrender}{%
        New color stack `#1' = \noexpand\number
        \expandafter\noexpand\csname PdfRender@Stack#1\endcsname
      }%
      \gdef\expandafter\noexpand\csname PdfRender@Init#1\endcsname{}%
    }%
    \expandafter\edef\csname PdfRender@Set#1\endcsname{%
      \expandafter\noexpand\csname PdfRender@Init#1\endcsname
      \noexpand\pdfcolorstack
      \expandafter\noexpand\csname PdfRender@Stack#1\endcsname
      push{%
        #3{\expandafter\noexpand\csname PdfRender@Current#1\endcsname}%
      }%
      \noexpand\aftergroup
      \expandafter\noexpand\csname PdfRender@Reset#1\endcsname
    }%
    \expandafter\edef\csname PdfRender@Reset#1\endcsname{%
      \expandafter\noexpand\csname PdfRender@Init#1\endcsname
      \noexpand\pdfcolorstack
      \expandafter\noexpand\csname PdfRender@Stack#1\endcsname
      pop\relax
    }%
  \else
    \expandafter\edef\csname PdfRender@Set#1\endcsname{%
      \noexpand\pdfliteral direct{%
        #3{\expandafter\noexpand\csname PdfRender@Current#1\endcsname}%
      }%
      \noexpand\aftergroup
      \expandafter\noexpand\csname PdfRender@Reset#1\endcsname
    }%
    \expandafter\edef\csname PdfRender@Reset#1\endcsname{%
      \noexpand\pdfliteral direct{%
        #3{\expandafter\noexpand\csname PdfRender@Current#1\endcsname}%
      }%
    }%
  \fi
  \expandafter\edef\csname PdfRender@Normal#1\endcsname{%
    \let
    \expandafter\noexpand\csname PdfRender@Current#1\endcsname
    \expandafter\noexpand\csname PdfRender@Default#1\endcsname
    \noexpand\PdfRender@Set{#1}%
  }%
  \expandafter\ltx@GlobalAppendToMacro\expandafter\PdfRender@NormalColorHook
  \expandafter{%
    \csname PdfRender@Normal#1\endcsname
  }%
  \ltx@GlobalAppendToMacro\PdfRender@ColorSetGroupHook{%
    \PdfRender@Set{#1}%
  }%
  \ifPdfRender@Values
    \kv@parse@normalized{#4}{%
      \expandafter\let\csname PdfRender@#1@\kv@key\endcsname\kv@key
      \ifx\kv@value\relax
      \else
        \expandafter\let\csname PdfRender@#1@\kv@value\endcsname\kv@key
      \fi
      \ltx@gobbletwo
    }%
    \PdfRender@define@key{PDFRENDER}{#1}{%
      \global\csname PdfRender@Active#1true\endcsname
      \def\PdfRender@Current{##1}%
      \PdfRender@SetValidateValues{#1}%
    }%
    \PdfRender@Valuesfalse
  \else
    \PdfRender@define@key{PDFRENDER}{#1}{%
      \global\csname PdfRender@Active#1true\endcsname
      \expandafter\def\csname PdfRender@Current#1\endcsname{##1}%
      \ltx@IfUndefined{PdfRender@PostProcess#1}{%
      }{%
        \csname PdfRender@PostProcess#1\endcsname
      }%
      \PdfRender@SetValidate{#1}{#4}{#5}%
    }%
  \fi
}%
%    \end{macrocode}
%    \end{macro}
%    \begin{macro}{\PdfRender@define@key}
%    \begin{macrocode}
\ltx@IfUndefined{define@key}{%
  \def\PdfRender@define@key#1#2{%
    \expandafter\def\csname KV@#1@#2\endcsname##1%
  }%
}{%
  \let\PdfRender@define@key\define@key
}
%    \end{macrocode}
%    \end{macro}
%    \begin{macro}{\PdfRender@Set}
%    \begin{macrocode}
\def\PdfRender@Set#1{%
  \csname ifPdfRender@Active#1\endcsname
    \csname PdfRender@Set#1\expandafter\endcsname
  \fi
}
%    \end{macrocode}
%    \end{macro}
%    \begin{macro}{\PdfRender@Reset}
%    \begin{macrocode}
\def\PdfRender@Reset#1{%
  \csname ifPdfRender@Active#1\endcsname
    \csname PdfRender@Reset#1\expandafter\endcsname
  \fi
}
%    \end{macrocode}
%    \end{macro}
%    \begin{macro}{\PdfRender@ErrorInvalidValue}
%    \begin{macrocode}
\def\PdfRender@ErrorInvalidValue#1{%
  \PackageError{pdfrender}{%
    Ignoring parameter setting for `#1'\MessageBreak
    because of invalid value %
    `\csname PdfRender@Current#1\endcsname'%
  }\@ehc
  \expandafter\let\csname PdfRender@Current#1\endcsname\ltx@empty
}%
%    \end{macrocode}
%    \end{macro}
%    \begin{macro}{\PdfRender@SetValidate}
%    \begin{macrocode}
\ifPdfRender@Match
  \def\PdfRender@SetValidate#1#2#3{%
    \ifnum\pdfmatch{^(#2)$}{\csname PdfRender@Current#1\endcsname}=1 %
      \csname PdfRender@Set#1\expandafter\endcsname
    \else
      \PdfRender@ErrorInvalidValue{#1}%
    \fi
  }%
\else
  \def\PdfRender@SetValidate#1#2#3{%
    \expandafter\let\expandafter\PdfRender@Current
    \csname PdfRender@Current#1\endcsname
    #3%
    \ifx\PdfRender@Current\@empty
      \PdfRender@ErrorInvalidValue{#1}%
    \else
      \csname PdfRender@Set#1\expandafter\endcsname
    \fi
  }%
\fi
%    \end{macrocode}
%    \end{macro}
%    \begin{macro}{\PdfRender@SetValidateValues}
%    \begin{macrocode}
\def\PdfRender@SetValidateValues#1{%
  \ltx@IfUndefined{PdfRender@#1@\PdfRender@Current}{%
    \expandafter\let\csname PdfRender@Current#1\endcsname
                    \PdfRender@Current
    \PdfRender@ErrorInvalidValue{#1}%
  }{%
    \expandafter\edef\csname PdfRender@Current#1\endcsname{%
      \csname PdfRender@#1@\PdfRender@Current\endcsname
    }%
    \csname PdfRender@Set#1\endcsname
  }%
}
%    \end{macrocode}
%    \end{macro}
%    \begin{macro}{\PdfRender@OpValue}
%    \begin{macrocode}
\def\PdfRender@OpValue#1#2{#2\ltx@space#1}%
%    \end{macrocode}
%    \end{macro}
%    \begin{macro}{\PdfRender@OpName}
%    \begin{macrocode}
\def\PdfRender@OpName#1#2{/#2\ltx@space#1}%
%    \end{macrocode}
%    \end{macro}
%
% \subsection{Declare and setup parameters}
%
%    \begin{macrocode}
\PdfRender@NewClassValues{TextRenderingMode}%
                      {0}%
                      {\PdfRender@OpValue{Tr}}{%
  0=Fill,%
  1=Stroke,%
  2=FillStroke,%
  3=Invisible,%
  4=FillClip,%
  5=StrokeClip,%
  6=FillStrokeClip,%
  7=Clip,%
}%
\PdfRender@NewClass{LineWidth}{1}{\PdfRender@OpValue{w}}{%
  [0-9]+\string\.?[0-9]*|\string\.[0-9]+%
}{}%
\ltx@IfUndefined{dimexpr}{%
  \def\PdfRender@dimexpr{}%
}{%
  \let\PdfRender@dimexpr\dimexpr
}
\def\PdfRender@PostProcessLineWidth{%
  \begingroup
  \afterassignment\PdfRender@@PostProcessLineWidth
  \dimen0=\PdfRender@dimexpr\PdfRender@CurrentLineWidth bp %
  \PdfRender@let\PdfRender@relax\PdfRender@relax
}
\let\PdfRender@let\let
\let\PdfRender@relax\relax
\def\PdfRender@@PostProcessLineWidth#1\PdfRender@let{%
  \ifx\\#1\\%
    \endgroup
  \else
    \dimen0=.996264\dimen0 % 72/72.27
    \edef\x{\endgroup
      \def\noexpand\PdfRender@CurrentLineWidth{%
        \strip@pt\dimen0%
      }%
    }%
    \expandafter\x
  \fi
}
\PdfRender@NewClassValues{LineCapStyle}{0}{\PdfRender@OpValue{J}}{%
  0=Butt,%
  1=Round,%
  2=ProjectingSquare,%
}%
\PdfRender@NewClassValues{LineJoinStyle}{0}{\PdfRender@OpValue{j}}{%
  0=Miter,%
  1=Round,%
  2=Bevel,%
}%
\PdfRender@NewClass{MiterLimit}{10}{\PdfRender@OpValue{M}}{%
  [0-9]*[1-9][0-9]*\string\.?[0-9]*|%
  [0-9]*\string\.?[0-9]*[1-9][0-9]*%
}{}%
\PdfRender@NewClass{Flatness}{0}{\PdfRender@OpValue{i}}{%
  100(\string\.0*)?|[0-9][0-9](\string\.[0-9]*)?|\string\.[0-9]+%
}{}%
\PdfRender@NewClass{LineDashPattern}{[]0}{\PdfRender@OpValue{d}}{%
  \string\[%
  ( ?([0-9]+\string\.?[0-9]*|\string\.[0-9]+) ?)*%
  \string\] ?%
  ([0-9]+\string\.?[0-9]*|\string\.[0-9]+)%
}{}%
\PdfRender@NewClassValues{RenderingIntent}%
                      {RelativeColorimetric}%
                      {\PdfRender@OpName{ri}}{%
  AbsoluteColorimetric,%
  RelativeColorimetric,%
  Saturation,%
  Perceptual,%
}%
%    \end{macrocode}
%
% \subsection{Fill and stroke color support}
%
%    \begin{macrocode}
\PdfRender@define@key{PDFRENDER}{FillColor}{%
  \begingroup
    \def\PdfRender@Color{#1}%
    \ifx\PdfRender@Color\ltx@empty
      \global\let\PdfRender@FillColor\ltx@empty
    \else
      \PdfRender@ColorAvailable{%
        \PdfRender@TestBox{%
          \expandafter\PdfRender@TryColor\PdfRender@Color\ltx@empty
          \PdfRender@GetFillColor
          \ifx\PdfRender@FillColor\ltx@empty
            \@PackageWarning{pdfrender}{%
              Cannot extract fill color\MessageBreak
              from value `#1'%
            }%
          \fi
        }%
      }%
    \fi
  \endgroup
}
\PdfRender@define@key{PDFRENDER}{StrokeColor}{%
  \begingroup
    \def\PdfRender@Color{#1}%
    \ifx\PdfRender@Color\ltx@empty
      \global\let\PdfRender@StrokeColor\ltx@empty
    \else
      \PdfRender@ColorAvailable{%
        \PdfRender@TestBox{%
          \expandafter\PdfRender@TryColor\PdfRender@Color\ltx@empty
          \PdfRender@GetStrokeColor
          \ifx\PdfRender@StrokeColor\ltx@empty
            \@PackageWarning{pdfrender}{%
              Cannot extract stroke color\MessageBreak
              from value `#1'%
            }%
          \fi
        }%
      }%
    \fi
  \endgroup
}
%    \end{macrocode}
%    \begin{macro}{\PdfRender@ColorAvailable}
%    \begin{macrocode}
\def\PdfRender@ColorAvailable{%
  \@ifundefined{set@color}{%
    \@PackageError{pdfrender}{%
      Ignoring color options, because neither\MessageBreak
      package `color' nor package `xcolor' is loaded%
    }\@ehc
    \global\let\PdfRender@ColorAvailable\ltx@gobble
  }{%
    \global\let\PdfRender@ColorAvailable\ltx@firstofone
  }%
  \PdfRender@ColorAvailable
}
%    \end{macrocode}
%    \end{macro}
%    \begin{macro}{\PdfRender@TryColor}
%    \begin{macrocode}
\def\PdfRender@TryColor{%
  \@ifnextchar[\color\PdfRender@@TryColor
}
%    \end{macrocode}
%    \end{macro}
%    \begin{macro}{\PdfRender@@TryColor}
%    \begin{macrocode}
\def\PdfRender@@TryColor#1\ltx@empty{%
  \expandafter\color\expandafter{\PdfRender@Color}%
}
%    \end{macrocode}
%    \end{macro}
%    \begin{macro}{\PdfRender@SetColor}
%    \begin{macrocode}
\def\PdfRender@SetColor{%
  \chardef\PdfRender@NeedsCurrentColor=0 %
  \ifx\PdfRender@FillColor\ltx@empty
    \ifx\PdfRender@StrokeColor\ltx@empty
    \else
      \edef\PdfRender@CurrentColor{%
        \noexpand\PdfRender@FillColor\ltx@space\PdfRender@StrokeColor
      }%
      \chardef\PdfRender@NeedsCurrentColor=1 %
    \fi
  \else
    \ifx\PdfRender@StrokeColor\ltx@empty
      \edef\PdfRender@CurrentColor{%
        \PdfRender@FillColor\ltx@space\noexpand\PdfRender@StrokeColor
      }%
      \chardef\PdfRender@NeedsCurrentColor=2 %
    \else
      \edef\current@color{%
        \PdfRender@FillColor\ltx@space\PdfRender@StrokeColor
      }%
      \set@color
    \fi
  \fi
  \ifnum\PdfRender@NeedsCurrentColor=1 %
    \PdfRender@GetFillColor
    \ifx\PdfRender@FillColor\ltx@empty
      \@PackageWarning{pdfrender}{%
        Cannot extract current fill color%
      }%
    \else
      \edef\current@color{\PdfRender@CurrentColor}%
      \set@color
    \fi
  \else
    \ifnum\PdfRender@NeedsCurrentColor=2 %
      \PdfRender@GetStrokeColor
      \ifx\PdfRender@StrokeColor\ltx@empty
        \@PackageWarning{pdfrender}{%
          Cannot extract current stroke color%
        }%
      \else
        \edef\current@color{\PdfRender@CurrentColor}%
        \set@color
      \fi
    \fi
  \fi
}
%    \end{macrocode}
%    \end{macro}
%    \begin{macro}{\PdfRender@PatternFillColor}
%    \begin{macrocode}
\edef\PdfRender@PatternFillColor{ % space
  (%
    [0-9\string\.]+ g|%
    [0-9\string\.]+ [0-9\string\.]+ [0-9\string\.]+ rg|%
    [0-9\string\.]+ [0-9\string\.]+ %
    [0-9\string\.]+ [0-9\string\.]+ k%
  ) % space
  (.*)$%
}
%    \end{macrocode}
%    \end{macro}
%    \begin{macro}{\PdfRender@PatternStrokeColor}
%    \begin{macrocode}
\edef\PdfRender@PatternStrokeColor{ % space
  (%
    [0-9\string\.]+ G|%
    [0-9\string\.]+ [0-9\string\.]+ [0-9\string\.]+ RG|%
    [0-9\string\.]+ [0-9\string\.]+ %
    [0-9\string\.]+ [0-9\string\.]+ K%
  ) % space
  (.*)$%
}
%    \end{macrocode}
%    \end{macro}
%    \begin{macro}{\PdfRender@MatchPattern}
%    \begin{macrocode}
\def\PdfRender@MatchPattern#1{%
  \ifnum\pdfmatch{\PdfRender@Pattern}{\PdfRender@String}=1 %
    \xdef#1{%
      \expandafter\strip@prefix\pdflastmatch 1%
    }%
    \edef\PdfRender@String{%
      \expandafter\strip@prefix\pdflastmatch 2%
    }%
    \ifx\PdfRender@String\ltx@empty
    \else
      \expandafter\expandafter\expandafter\PdfRender@MatchPattern
      \expandafter\expandafter\expandafter#1%
    \fi
  \fi
}
%    \end{macrocode}
%    \end{macro}
%    \begin{macro}{\PdfRender@GetFillColor}
%    \begin{macrocode}
\def\PdfRender@GetFillColor{%
  \global\let\PdfRender@FillColor\ltx@empty
  \begingroup
    \ifPdfRender@Match
      \let\PdfRender@Pattern\PdfRender@PatternFillColor
      \edef\PdfRender@String{\ltx@space\current@color\ltx@space}%
      \PdfRender@MatchPattern\PdfRender@FillColor
    \else
      \edef\current@color{\current@color\ltx@space}%
      \let\PdfRender@OP\relax
      \PdfRender@FindOp{g}0%
      \PdfRender@FindOp{G}1%
      \PdfRender@FindOp{rg}0%
      \PdfRender@FindOp{RG}1%
      \PdfRender@FindOp{k}0%
      \PdfRender@FindOp{K}1%
      \PdfRender@FilterOp 0\PdfRender@FillColor
    \fi
  \endgroup
}
%    \end{macrocode}
%    \end{macro}
%    \begin{macro}{\PdfRender@GetStrokeColor}
%    \begin{macrocode}
\def\PdfRender@GetStrokeColor{%
  \global\let\PdfRender@StrokeColor\ltx@empty
  \begingroup
    \ifPdfRender@Match
      \let\PdfRender@Pattern\PdfRender@PatternStrokeColor
      \edef\PdfRender@String{\ltx@space\current@color\ltx@space}%
      \PdfRender@MatchPattern\PdfRender@StrokeColor
    \else
      \edef\current@color{\current@color\ltx@space}%
      \let\PdfRender@OP\relax
      \PdfRender@FindOp{g}0%
      \PdfRender@FindOp{G}1%
      \PdfRender@FindOp{rg}0%
      \PdfRender@FindOp{RG}1%
      \PdfRender@FindOp{k}0%
      \PdfRender@FindOp{K}1%
      \PdfRender@FilterOp 1\PdfRender@StrokeColor
    \fi
  \endgroup
}
%    \end{macrocode}
%    \end{macro}
%    \begin{macrocode}
\ifPdfRender@Match
  \expandafter\PdfRender@AtEnd
\fi%
%    \end{macrocode}
%    \begin{macro}{\PdfRender@FindOp}
%    \begin{macrocode}
\def\PdfRender@FindOp#1#2{%
  \def\PdfRender@temp##1 #1 ##2\@nil{%
    ##1%
    \ifx\\##2\\%
      \expandafter\@gobble
    \else
      \PdfRender@OP{#1}#2%
      \expandafter\@firstofone
    \fi
    {%
      \PdfRender@temp##2\@nil
    }%
  }%
  \edef\current@color{%
    \@firstofone{\expandafter\PdfRender@temp\current@color} #1 \@nil
  }%
}
%    \end{macrocode}
%    \end{macro}
%    \begin{macro}{\PdfRender@FilterOp}
%    \begin{macrocode}
\def\PdfRender@FilterOp#1#2{%
  \expandafter\PdfRender@@FilterOp\expandafter#1\expandafter#2%
      \current@color\PdfRender@OP{}{}%
}
%    \end{macrocode}
%    \end{macro}
%    \begin{macro}{\PdfRender@@FilterOp}
%    \begin{macrocode}
\def\PdfRender@@FilterOp#1#2#3\PdfRender@OP#4#5{%
  \ifx\\#4#5\\%
  \else
    \ifnum#1=#5 %
      \xdef#2{#3 #4}%
    \fi
    \expandafter\PdfRender@@FilterOp\expandafter#1\expandafter#2%
  \fi
}
%    \end{macrocode}
%    \end{macro}
%
%    \begin{macrocode}
\PdfRender@AtEnd%
%</package>
%    \end{macrocode}
%% \section{Installation}
%
% \subsection{Download}
%
% \paragraph{Package.} This package is available on
% CTAN\footnote{\CTANpkg{pdfrender}}:
% \begin{description}
% \item[\CTAN{macros/latex/contrib/oberdiek/pdfrender.dtx}] The source file.
% \item[\CTAN{macros/latex/contrib/oberdiek/pdfrender.pdf}] Documentation.
% \end{description}
%
%
% \paragraph{Bundle.} All the packages of the bundle `oberdiek'
% are also available in a TDS compliant ZIP archive. There
% the packages are already unpacked and the documentation files
% are generated. The files and directories obey the TDS standard.
% \begin{description}
% \item[\CTANinstall{install/macros/latex/contrib/oberdiek.tds.zip}]
% \end{description}
% \emph{TDS} refers to the standard ``A Directory Structure
% for \TeX\ Files'' (\CTANpkg{tds}). Directories
% with \xfile{texmf} in their name are usually organized this way.
%
% \subsection{Bundle installation}
%
% \paragraph{Unpacking.} Unpack the \xfile{oberdiek.tds.zip} in the
% TDS tree (also known as \xfile{texmf} tree) of your choice.
% Example (linux):
% \begin{quote}
%   |unzip oberdiek.tds.zip -d ~/texmf|
% \end{quote}
%
% \subsection{Package installation}
%
% \paragraph{Unpacking.} The \xfile{.dtx} file is a self-extracting
% \docstrip\ archive. The files are extracted by running the
% \xfile{.dtx} through \plainTeX:
% \begin{quote}
%   \verb|tex pdfrender.dtx|
% \end{quote}
%
% \paragraph{TDS.} Now the different files must be moved into
% the different directories in your installation TDS tree
% (also known as \xfile{texmf} tree):
% \begin{quote}
% \def\t{^^A
% \begin{tabular}{@{}>{\ttfamily}l@{ $\rightarrow$ }>{\ttfamily}l@{}}
%   pdfrender.sty & tex/generic/oberdiek/pdfrender.sty\\
%   pdfrender.pdf & doc/latex/oberdiek/pdfrender.pdf\\
%   pdfrender.dtx & source/latex/oberdiek/pdfrender.dtx\\
% \end{tabular}^^A
% }^^A
% \sbox0{\t}^^A
% \ifdim\wd0>\linewidth
%   \begingroup
%     \advance\linewidth by\leftmargin
%     \advance\linewidth by\rightmargin
%   \edef\x{\endgroup
%     \def\noexpand\lw{\the\linewidth}^^A
%   }\x
%   \def\lwbox{^^A
%     \leavevmode
%     \hbox to \linewidth{^^A
%       \kern-\leftmargin\relax
%       \hss
%       \usebox0
%       \hss
%       \kern-\rightmargin\relax
%     }^^A
%   }^^A
%   \ifdim\wd0>\lw
%     \sbox0{\small\t}^^A
%     \ifdim\wd0>\linewidth
%       \ifdim\wd0>\lw
%         \sbox0{\footnotesize\t}^^A
%         \ifdim\wd0>\linewidth
%           \ifdim\wd0>\lw
%             \sbox0{\scriptsize\t}^^A
%             \ifdim\wd0>\linewidth
%               \ifdim\wd0>\lw
%                 \sbox0{\tiny\t}^^A
%                 \ifdim\wd0>\linewidth
%                   \lwbox
%                 \else
%                   \usebox0
%                 \fi
%               \else
%                 \lwbox
%               \fi
%             \else
%               \usebox0
%             \fi
%           \else
%             \lwbox
%           \fi
%         \else
%           \usebox0
%         \fi
%       \else
%         \lwbox
%       \fi
%     \else
%       \usebox0
%     \fi
%   \else
%     \lwbox
%   \fi
% \else
%   \usebox0
% \fi
% \end{quote}
% If you have a \xfile{docstrip.cfg} that configures and enables \docstrip's
% TDS installing feature, then some files can already be in the right
% place, see the documentation of \docstrip.
%
% \subsection{Refresh file name databases}
%
% If your \TeX~distribution
% (\TeX\,Live, \mikTeX, \dots) relies on file name databases, you must refresh
% these. For example, \TeX\,Live\ users run \verb|texhash| or
% \verb|mktexlsr|.
%
% \subsection{Some details for the interested}
%
% \paragraph{Unpacking with \LaTeX.}
% The \xfile{.dtx} chooses its action depending on the format:
% \begin{description}
% \item[\plainTeX:] Run \docstrip\ and extract the files.
% \item[\LaTeX:] Generate the documentation.
% \end{description}
% If you insist on using \LaTeX\ for \docstrip\ (really,
% \docstrip\ does not need \LaTeX), then inform the autodetect routine
% about your intention:
% \begin{quote}
%   \verb|latex \let\install=y% \iffalse meta-comment
%
% File: pdfrender.dtx
% Version: 2019/12/29 v1.6
% Info: Access to some PDF graphics parameters
%
% Copyright (C)
%    2010 Heiko Oberdiek
%    2016-2019 Oberdiek Package Support Group
%    https://github.com/ho-tex/oberdiek/issues
%
% This work may be distributed and/or modified under the
% conditions of the LaTeX Project Public License, either
% version 1.3c of this license or (at your option) any later
% version. This version of this license is in
%    https://www.latex-project.org/lppl/lppl-1-3c.txt
% and the latest version of this license is in
%    https://www.latex-project.org/lppl.txt
% and version 1.3 or later is part of all distributions of
% LaTeX version 2005/12/01 or later.
%
% This work has the LPPL maintenance status "maintained".
%
% The Current Maintainers of this work are
% Heiko Oberdiek and the Oberdiek Package Support Group
% https://github.com/ho-tex/oberdiek/issues
%
% The Base Interpreter refers to any `TeX-Format',
% because some files are installed in TDS:tex/generic//.
%
% This work consists of the main source file pdfrender.dtx
% and the derived files
%    pdfrender.sty, pdfrender.pdf, pdfrender.ins, pdfrender.drv,
%    pdfrender-test1.tex, pdfrender-test2.tex,
%    pdfrender-test3.tex, pdfrender-test4.tex,
%    pdfrender-test5.tex.
%
% Distribution:
%    CTAN:macros/latex/contrib/oberdiek/pdfrender.dtx
%    CTAN:macros/latex/contrib/oberdiek/pdfrender.pdf
%
% Unpacking:
%    (a) If pdfrender.ins is present:
%           tex pdfrender.ins
%    (b) Without pdfrender.ins:
%           tex pdfrender.dtx
%    (c) If you insist on using LaTeX
%           latex \let\install=y% \iffalse meta-comment
%
% File: pdfrender.dtx
% Version: 2019/12/29 v1.6
% Info: Access to some PDF graphics parameters
%
% Copyright (C)
%    2010 Heiko Oberdiek
%    2016-2019 Oberdiek Package Support Group
%    https://github.com/ho-tex/oberdiek/issues
%
% This work may be distributed and/or modified under the
% conditions of the LaTeX Project Public License, either
% version 1.3c of this license or (at your option) any later
% version. This version of this license is in
%    https://www.latex-project.org/lppl/lppl-1-3c.txt
% and the latest version of this license is in
%    https://www.latex-project.org/lppl.txt
% and version 1.3 or later is part of all distributions of
% LaTeX version 2005/12/01 or later.
%
% This work has the LPPL maintenance status "maintained".
%
% The Current Maintainers of this work are
% Heiko Oberdiek and the Oberdiek Package Support Group
% https://github.com/ho-tex/oberdiek/issues
%
% The Base Interpreter refers to any `TeX-Format',
% because some files are installed in TDS:tex/generic//.
%
% This work consists of the main source file pdfrender.dtx
% and the derived files
%    pdfrender.sty, pdfrender.pdf, pdfrender.ins, pdfrender.drv,
%    pdfrender-test1.tex, pdfrender-test2.tex,
%    pdfrender-test3.tex, pdfrender-test4.tex,
%    pdfrender-test5.tex.
%
% Distribution:
%    CTAN:macros/latex/contrib/oberdiek/pdfrender.dtx
%    CTAN:macros/latex/contrib/oberdiek/pdfrender.pdf
%
% Unpacking:
%    (a) If pdfrender.ins is present:
%           tex pdfrender.ins
%    (b) Without pdfrender.ins:
%           tex pdfrender.dtx
%    (c) If you insist on using LaTeX
%           latex \let\install=y% \iffalse meta-comment
%
% File: pdfrender.dtx
% Version: 2019/12/29 v1.6
% Info: Access to some PDF graphics parameters
%
% Copyright (C)
%    2010 Heiko Oberdiek
%    2016-2019 Oberdiek Package Support Group
%    https://github.com/ho-tex/oberdiek/issues
%
% This work may be distributed and/or modified under the
% conditions of the LaTeX Project Public License, either
% version 1.3c of this license or (at your option) any later
% version. This version of this license is in
%    https://www.latex-project.org/lppl/lppl-1-3c.txt
% and the latest version of this license is in
%    https://www.latex-project.org/lppl.txt
% and version 1.3 or later is part of all distributions of
% LaTeX version 2005/12/01 or later.
%
% This work has the LPPL maintenance status "maintained".
%
% The Current Maintainers of this work are
% Heiko Oberdiek and the Oberdiek Package Support Group
% https://github.com/ho-tex/oberdiek/issues
%
% The Base Interpreter refers to any `TeX-Format',
% because some files are installed in TDS:tex/generic//.
%
% This work consists of the main source file pdfrender.dtx
% and the derived files
%    pdfrender.sty, pdfrender.pdf, pdfrender.ins, pdfrender.drv,
%    pdfrender-test1.tex, pdfrender-test2.tex,
%    pdfrender-test3.tex, pdfrender-test4.tex,
%    pdfrender-test5.tex.
%
% Distribution:
%    CTAN:macros/latex/contrib/oberdiek/pdfrender.dtx
%    CTAN:macros/latex/contrib/oberdiek/pdfrender.pdf
%
% Unpacking:
%    (a) If pdfrender.ins is present:
%           tex pdfrender.ins
%    (b) Without pdfrender.ins:
%           tex pdfrender.dtx
%    (c) If you insist on using LaTeX
%           latex \let\install=y\input{pdfrender.dtx}
%        (quote the arguments according to the demands of your shell)
%
% Documentation:
%    (a) If pdfrender.drv is present:
%           latex pdfrender.drv
%    (b) Without pdfrender.drv:
%           latex pdfrender.dtx; ...
%    The class ltxdoc loads the configuration file ltxdoc.cfg
%    if available. Here you can specify further options, e.g.
%    use A4 as paper format:
%       \PassOptionsToClass{a4paper}{article}
%
%    Programm calls to get the documentation (example):
%       pdflatex pdfrender.dtx
%       makeindex -s gind.ist pdfrender.idx
%       pdflatex pdfrender.dtx
%       makeindex -s gind.ist pdfrender.idx
%       pdflatex pdfrender.dtx
%
% Installation:
%    TDS:tex/generic/oberdiek/pdfrender.sty
%    TDS:doc/latex/oberdiek/pdfrender.pdf
%    TDS:source/latex/oberdiek/pdfrender.dtx
%
%<*ignore>
\begingroup
  \catcode123=1 %
  \catcode125=2 %
  \def\x{LaTeX2e}%
\expandafter\endgroup
\ifcase 0\ifx\install y1\fi\expandafter
         \ifx\csname processbatchFile\endcsname\relax\else1\fi
         \ifx\fmtname\x\else 1\fi\relax
\else\csname fi\endcsname
%</ignore>
%<*install>
\input docstrip.tex
\Msg{************************************************************************}
\Msg{* Installation}
\Msg{* Package: pdfrender 2019/12/29 v1.6 Access to some PDF graphics parameters (HO)}
\Msg{************************************************************************}

\keepsilent
\askforoverwritefalse

\let\MetaPrefix\relax
\preamble

This is a generated file.

Project: pdfrender
Version: 2019/12/29 v1.6

Copyright (C)
   2010 Heiko Oberdiek
   2016-2019 Oberdiek Package Support Group

This work may be distributed and/or modified under the
conditions of the LaTeX Project Public License, either
version 1.3c of this license or (at your option) any later
version. This version of this license is in
   https://www.latex-project.org/lppl/lppl-1-3c.txt
and the latest version of this license is in
   https://www.latex-project.org/lppl.txt
and version 1.3 or later is part of all distributions of
LaTeX version 2005/12/01 or later.

This work has the LPPL maintenance status "maintained".

The Current Maintainers of this work are
Heiko Oberdiek and the Oberdiek Package Support Group
https://github.com/ho-tex/oberdiek/issues


The Base Interpreter refers to any `TeX-Format',
because some files are installed in TDS:tex/generic//.

This work consists of the main source file pdfrender.dtx
and the derived files
   pdfrender.sty, pdfrender.pdf, pdfrender.ins, pdfrender.drv,
   pdfrender-test1.tex, pdfrender-test2.tex,
   pdfrender-test3.tex, pdfrender-test4.tex,
   pdfrender-test5.tex.

\endpreamble
\let\MetaPrefix\DoubleperCent

\generate{%
  \file{pdfrender.ins}{\from{pdfrender.dtx}{install}}%
  \file{pdfrender.drv}{\from{pdfrender.dtx}{driver}}%
  \usedir{tex/generic/oberdiek}%
  \file{pdfrender.sty}{\from{pdfrender.dtx}{package}}%
%  \usedir{doc/latex/oberdiek/test}%
%  \file{pdfrender-test1.tex}{\from{pdfrender.dtx}{test1}}%
%  \file{pdfrender-test2.tex}{\from{pdfrender.dtx}{test2}}%
%  \file{pdfrender-test3.tex}{\from{pdfrender.dtx}{test3}}%
%  \file{pdfrender-test4.tex}{\from{pdfrender.dtx}{test4}}%
%  \file{pdfrender-test5.tex}{\from{pdfrender.dtx}{test5}}%
}

\catcode32=13\relax% active space
\let =\space%
\Msg{************************************************************************}
\Msg{*}
\Msg{* To finish the installation you have to move the following}
\Msg{* file into a directory searched by TeX:}
\Msg{*}
\Msg{*     pdfrender.sty}
\Msg{*}
\Msg{* To produce the documentation run the file `pdfrender.drv'}
\Msg{* through LaTeX.}
\Msg{*}
\Msg{* Happy TeXing!}
\Msg{*}
\Msg{************************************************************************}

\endbatchfile
%</install>
%<*ignore>
\fi
%</ignore>
%<*driver>
\NeedsTeXFormat{LaTeX2e}
\ProvidesFile{pdfrender.drv}%
  [2019/12/29 v1.6 Access to some PDF graphics parameters (HO)]%
\documentclass{ltxdoc}
\usepackage{holtxdoc}[2011/11/22]
\usepackage{longtable}
\renewcommand*{\arraystretch}{1.2}
\begin{document}
  \DocInput{pdfrender.dtx}%
\end{document}
%</driver>
% \fi
%
%
%
% \GetFileInfo{pdfrender.drv}
%
% \title{The \xpackage{pdfrender} package}
% \date{2019/12/29 v1.6}
% \author{Heiko Oberdiek\thanks
% {Please report any issues at \url{https://github.com/ho-tex/oberdiek/issues}}}
%
% \maketitle
%
% \begin{abstract}
% The PDF format has some graphics parameter like
% line width or text rendering mode. This package
% provides an interface for setting these parameters.
% \end{abstract}
%
% \tableofcontents
%
% \section{Documentation}
%
% This package \xpackage{pdfrender} defines an interface for
% PDF specific parameters that affects the rendering of
% graphics or text. The interface and its implementation
% uses the same technique as package \xpackage{color}
% for color settings. Therefore this package is loaded to
% enable \LaTeX's color interface.
%
% At different places \LaTeX\ uses \cs{normalcolor} to avoid that
% header, footer or floats are print in the current color of the
% main text. \cs{setgroup@color} is used to start a save box
% with the color that is set at box saving time.
% Package \xpackage{pdfrender} extends these macros to
% add its own hooks of its parameters. Therefore \LaTeX3 should
% generalize \LaTeXe's color interface.
%
% \subsection{Usage}
%
% In \LaTeX\ the package is loaded as normal package. Options
% are not defined for this package.
% \begin{quote}
% |\usepackage{pdfrender}|
% \end{quote}
%
% This package can also be used in \plainTeX\ and even ini\TeX:
% \begin{quote}
% |input pdfrender.sty|
% \end{quote}
%
% \subsection{Macros}
%
% \begin{declcs}{pdfrender} \M{key value list}
% \end{declcs}
% The first parameter \meta{key value list} contains a
% list of parameter settings. The key entry is the parameter name.
% The macro works like
% \cs{color} (without optional argument) for color setting.
%
% \begin{declcs}{textpdfrender} \M{key value list} \M{text}
% \end{declcs}
% In the same way as \cs{pdfrender} the first argument
% specifies the parameters that should be set. This parameter
% setting affects \meta{text} only. Basically it works
% the same way as \cs{textcolor} (without optional argument).
%
% \subsection{Parameters}
%
% The following table shows an overview for the supported
% parameters and values:
%
% \begin{longtable}{@{}l|l|l@{}}
% \bfseries Parameter & \bfseries Value & \bfseries Alias\\*
% \hline
% \endhead
% \xoption{TextRenderingMode}
%   & \texttt{0} & \texttt{Fill}\\*
%   & \texttt{1} & \texttt{Stroke}\\*
%   & \texttt{2} & \texttt{FillStroke}\\*
%   & \texttt{3} & \texttt{Invisible}\\*
%   & \texttt{4} & \texttt{FillClip}\\*
%   & \texttt{5} & \texttt{StrokeClip}\\*
%   & \texttt{6} & \texttt{FillStrokeClip}\\*
%   & \texttt{7} & \texttt{Clip}\\
% \hline
% \xoption{LineWidth}
%   & \itshape positive number, & \itshape \TeX\ dimen\\*[-.2ex]
%   & \qquad \itshape unit is bp & \\
% \hline
% \xoption{LineCapStyle}
%   & \texttt{0} & \texttt{Butt}\\*
%   & \texttt{1} & \texttt{Round}\\*
%   & \texttt{2} & \texttt{ProjectingSquare}\\
% \hline
% \xoption{LineJoinStyle}
%   & \texttt{0} & \texttt{Miter}\\*
%   & \texttt{1} & \texttt{Round}\\*
%   & \texttt{2} & \texttt{Bevel}\\
% \hline
% \xoption{MiterLimit}
%   & \multicolumn{2}{l@{}}{\itshape positive number} \\
% \hline
% \xoption{Flatness}
%   & \multicolumn{2}{l@{}}{\itshape number between 0 and 100} \\
% \hline
% \xoption{LineDashPattern}
%   & \multicolumn{2}{l@{}}{^^A
%       \itshape numbers in square brackets,^^A
%     } \\*[-.2ex]
%   & \multicolumn{2}{l@{}}{^^A
%       \qquad \itshape followed by number,^^A
%     } \\*
%   & \multicolumn{2}{l@{}}{^^A
%       \qquad \itshape units are bp^^A
%     } \\
% \hline
% \xoption{RenderingIntent}
%   & \multicolumn{2}{l@{}}{\texttt{AbsoluteColorimetric}} \\*
%   & \multicolumn{2}{l@{}}{\texttt{RelativeColorimetric}} \\*
%   & \multicolumn{2}{l@{}}{\texttt{Saturation}} \\*
%   & \multicolumn{2}{l@{}}{\texttt{Perceptual}} \\
% \hline
% \xoption{FillColor}
%   & & \itshape color specification \\
% \hline
% \xoption{StrokeColor}
%   & & \itshape color specification \\
% \end{longtable}
%
% \subsubsection{Details}
%
% The description and specification of these parameters
% are available in the PDF specification \cite{pdf-spec}.
% Therefore they are not repeated here.
%
% \begin{description}
% \item[Value:]
%   The values in the second column lists or describe
%   the values that are specified by the PDF specification.
% \item[Alias:] Instead of magic numbers the package also
%   defines some aliases that can be given as value.
%   Example: |LineCapStyle=Round| has the same effect as
%   |LineCapStyle=1|.
% \item[Number:] The term \emph{number} means an integer
%   or real number. The real number is given as plain
%   decimal number without exponent. The decimal separator is a period.
%   At least one digit must be present.
% \item[\xoption{LineWidth}:] As alias a \TeX\ dimen specification
%   can be given. This includes explicit specifications with number
%   and unit, e.g. |LineWidth=0.5pt|. Also \LaTeX\ length registers
%   may be used. If \eTeX's \cs{dimexpr} is available, then it is
%   automatically added. However package \xpackage{calc} is not
%   supported.
% \item[\xoption{FillColor}, \xoption{StrokeColor}:]
%   Package \xpackage{color} or \xpackage{xcolor} must be loaded
%   before these options can be used (since version 1.2).
%   \LaTeX's color support sets both colors at the same time
%   to the same value. However parameter \xoption{TextRenderingMode}
%   offers the value \texttt{FillStroke} that makes only sense, if
%   the two color types can be set separately.
%   If one of the options \xoption{FillColor} or \xoption{StrokeColor}
%   is specified, then also the color is set. For compatibility
%   with the \LaTeX\ color packages (\xpackage{color} or \xpackage{xcolor}),
%   always both colors must be set. Thus if one of them is not specified,
%   it is taken from the current color.
%
%   Both options \xoption{FillColor} and \xoption{StrokeColor}
%   expect a \LaTeX\ color specification as value.
%   Also the optional color model argument is supported.
%   Example:
%   \begin{quote}
%     |FillColor=yellow|,\\
%     |StrokeColor=[cmyk]{1,.5,0,0}|\\
%   \end{quote}
% \end{description}
%
% \subsection{Color stack}
%
% If the \pdfTeX\ version provides color stacks, then each
% parameter is assigned a page based color stack. The assignment
% of a stack takes place, when its parameter is set the first time.
% This avoids the use of color stacks that are not needed.
%
% \StopEventually{
% }
%
% \section{Implementation}
%    \begin{macrocode}
%<*package>
%    \end{macrocode}
%    Reload check, especially if the package is not used with \LaTeX.
%    \begin{macrocode}
\begingroup\catcode61\catcode48\catcode32=10\relax%
  \catcode13=5 % ^^M
  \endlinechar=13 %
  \catcode35=6 % #
  \catcode39=12 % '
  \catcode44=12 % ,
  \catcode45=12 % -
  \catcode46=12 % .
  \catcode58=12 % :
  \catcode64=11 % @
  \catcode123=1 % {
  \catcode125=2 % }
  \expandafter\let\expandafter\x\csname ver@pdfrender.sty\endcsname
  \ifx\x\relax % plain-TeX, first loading
  \else
    \def\empty{}%
    \ifx\x\empty % LaTeX, first loading,
      % variable is initialized, but \ProvidesPackage not yet seen
    \else
      \expandafter\ifx\csname PackageInfo\endcsname\relax
        \def\x#1#2{%
          \immediate\write-1{Package #1 Info: #2.}%
        }%
      \else
        \def\x#1#2{\PackageInfo{#1}{#2, stopped}}%
      \fi
      \x{pdfrender}{The package is already loaded}%
      \aftergroup\endinput
    \fi
  \fi
\endgroup%
%    \end{macrocode}
%    Package identification:
%    \begin{macrocode}
\begingroup\catcode61\catcode48\catcode32=10\relax%
  \catcode13=5 % ^^M
  \endlinechar=13 %
  \catcode35=6 % #
  \catcode39=12 % '
  \catcode40=12 % (
  \catcode41=12 % )
  \catcode44=12 % ,
  \catcode45=12 % -
  \catcode46=12 % .
  \catcode47=12 % /
  \catcode58=12 % :
  \catcode64=11 % @
  \catcode91=12 % [
  \catcode93=12 % ]
  \catcode123=1 % {
  \catcode125=2 % }
  \expandafter\ifx\csname ProvidesPackage\endcsname\relax
    \def\x#1#2#3[#4]{\endgroup
      \immediate\write-1{Package: #3 #4}%
      \xdef#1{#4}%
    }%
  \else
    \def\x#1#2[#3]{\endgroup
      #2[{#3}]%
      \ifx#1\@undefined
        \xdef#1{#3}%
      \fi
      \ifx#1\relax
        \xdef#1{#3}%
      \fi
    }%
  \fi
\expandafter\x\csname ver@pdfrender.sty\endcsname
\ProvidesPackage{pdfrender}%
  [2019/12/29 v1.6 Access to some PDF graphics parameters (HO)]%
%    \end{macrocode}
%
%    \begin{macrocode}
\begingroup\catcode61\catcode48\catcode32=10\relax%
  \catcode13=5 % ^^M
  \endlinechar=13 %
  \catcode123=1 % {
  \catcode125=2 % }
  \catcode64=11 % @
  \def\x{\endgroup
    \expandafter\edef\csname PdfRender@AtEnd\endcsname{%
      \endlinechar=\the\endlinechar\relax
      \catcode13=\the\catcode13\relax
      \catcode32=\the\catcode32\relax
      \catcode35=\the\catcode35\relax
      \catcode61=\the\catcode61\relax
      \catcode64=\the\catcode64\relax
      \catcode123=\the\catcode123\relax
      \catcode125=\the\catcode125\relax
    }%
  }%
\x\catcode61\catcode48\catcode32=10\relax%
\catcode13=5 % ^^M
\endlinechar=13 %
\catcode35=6 % #
\catcode64=11 % @
\catcode123=1 % {
\catcode125=2 % }
\def\TMP@EnsureCode#1#2{%
  \edef\PdfRender@AtEnd{%
    \PdfRender@AtEnd
    \catcode#1=\the\catcode#1\relax
  }%
  \catcode#1=#2\relax
}
\TMP@EnsureCode{10}{12}% ^^J
\TMP@EnsureCode{36}{3}% $
\TMP@EnsureCode{39}{12}% '
\TMP@EnsureCode{40}{12}% (
\TMP@EnsureCode{41}{12}% )
\TMP@EnsureCode{42}{12}% *
\TMP@EnsureCode{43}{12}% +
\TMP@EnsureCode{44}{12}% ,
\TMP@EnsureCode{45}{12}% -
\TMP@EnsureCode{46}{12}% .
\TMP@EnsureCode{47}{12}% /
\TMP@EnsureCode{58}{12}% :
\TMP@EnsureCode{59}{12}% ;
\TMP@EnsureCode{60}{12}% <
\TMP@EnsureCode{62}{12}% >
\TMP@EnsureCode{63}{12}% ?
\TMP@EnsureCode{91}{12}% [
\TMP@EnsureCode{93}{12}% ]
\TMP@EnsureCode{94}{7}% ^ (superscript)
\TMP@EnsureCode{96}{12}% `
\TMP@EnsureCode{124}{12}% |
%    \end{macrocode}
%
%    \begin{macrocode}
\def\PdfRender@AtEndHook{}
\expandafter\def\expandafter\PdfRender@AtEnd\expandafter{%
  \expandafter\PdfRender@AtEndHook
  \PdfRender@AtEnd
  \endinput
}
%    \end{macrocode}
%
% \subsection{Look for \pdfTeX, its mode and features}
%
%    \begin{macro}{\PdfRender@newif}
%    \begin{macrocode}
\def\PdfRender@newif#1{%
  \expandafter\edef\csname PdfRender@#1true\endcsname{%
    \let
    \expandafter\noexpand\csname ifPdfRender@#1\endcsname
    \noexpand\iftrue
  }%
  \expandafter\edef\csname PdfRender@#1false\endcsname{%
    \let
    \expandafter\noexpand\csname ifPdfRender@#1\endcsname
    \noexpand\iffalse
  }%
  \csname PdfRender@#1false\endcsname
}
%    \end{macrocode}
%    \end{macro}
%
%    \begin{macro}{\ifPdfRender@Stack}
%    \begin{macrocode}
\PdfRender@newif{Stack}
%    \end{macrocode}
%    \end{macro}
%    \begin{macro}{\ifPdfRender@Match}
%    \begin{macrocode}
\PdfRender@newif{Match}
%    \end{macrocode}
%    \end{macro}
%
%    \begin{macro}{\PdfRender@RequirePackage}
%    \begin{macrocode}
\begingroup\expandafter\expandafter\expandafter\endgroup
\expandafter\ifx\csname RequirePackage\endcsname\relax
  \def\PdfRender@RequirePackage#1[#2]{%
    \expandafter\def\expandafter\PdfRender@AtEndHook\expandafter{%
      \PdfRender@AtEndHook
      \ltx@ifpackagelater{#1}{#2}{}{%
        \@PackageWarningNoLine{pdfrender}{%
          You have requested version\MessageBreak
          `#2' of package `#1',\MessageBreak
          but only version\MessageBreak
          `\csname ver@#1.\ltx@pkgextension\endcsname'\MessageBreak
          is available%
        }%
      }%
    }%
    \input #1.sty\relax
  }%
\else
  \let\PdfRender@RequirePackage\RequirePackage
\fi
%    \end{macrocode}
%    \end{macro}
% Luatex compatibility
%    \begin{macrocode}
\ifx\pdfextension\@undefined\else
  \def\pdfcolorstackinit {\pdffeedback colorstackinit}
  \protected\def\pdfcolorstack     {\pdfextension colorstack}
  \protected\def\pdfliteral        {\pdfextension literal}
\fi
%    \end{macrocode}
%
%    \begin{macrocode}
\PdfRender@RequirePackage{iftex}[2019/11/07]
\PdfRender@RequirePackage{infwarerr}[2007/09/09]
\PdfRender@RequirePackage{ltxcmds}[2010/01/28]
%    \end{macrocode}
%
%    \begin{macrocode}
\ifpdf
  \ltx@IfUndefined{pdfcolorstackinit}{%
    \@PackageWarning{pdfrender}{%
      Missing \string\pdfcolorstackinit
    }%
  }{%
    \PdfRender@Stacktrue
  }%
  \ltx@IfUndefined{pdfmatch}{%
    \@PackageInfoNoLine{pdfrender}{%
      \string\pdfmatch\ltx@space not found. %
      Therefore the values\MessageBreak
      of some parameters are not validated%
    }%
  }{%
    \PdfRender@Matchtrue
  }%
\else
  \@PackageWarning{pdfrender}{%
    Missing pdfTeX in PDF mode%
  }%
%    \end{macrocode}
%    \begin{macrocode}
  \ltx@IfUndefined{newcommand}{%
%    \end{macrocode}
%    \begin{macro}{\pdfrender}
%    \begin{macrocode}
    \def\pdfrender#1{}%
%    \end{macrocode}
%    \end{macro}
%    \begin{macro}{\textpdfrender}
%    \begin{macrocode}
    \long\def\textpdfrender#1#2{#2}%
%    \end{macrocode}
%    \end{macro}
%    \begin{macrocode}
  }{%
%    \end{macrocode}
%    \begin{macro}{\pdfrender}
%    \begin{macrocode}
    \newcommand*{\pdfrender}[1]{}%
%    \end{macrocode}
%    \end{macro}
%    \begin{macro}{\textpdfrender}
%    \begin{macrocode}
    \newcommand{\textpdfrender}[2]{#2}%
%    \end{macrocode}
%    \end{macro}
%    \begin{macrocode}
  }%
  \expandafter\PdfRender@AtEnd
\fi%
%    \end{macrocode}
%
% \subsection{Enable color support of \LaTeX}
%
%    \begin{macrocode}
\ltx@ifpackageloaded{color}{}{%
  \def\color@setgroup{\begingroup\set@color}%
  \let\color@begingroup\begingroup
  \def\color@endgroup{\endgraf\endgroup}%
  \def\color@hbox{\hbox\bgroup\color@begingroup}%
  \def\color@vbox{\vbox\bgroup\color@begingroup}%
  \def\color@endbox{\color@endgroup\egroup}%
  \ltx@ifundefined{bgroup}{%
    \let\bgroup={\let\egroup=}%
  }{}%
  \ltx@ifundefined{endgraf}{%
    \let\endgraf=\par
  }{}%
}
%    \end{macrocode}
%
% \subsection{Hook into \cs{normalcolor}}
%
%    The problem is that packages \xpackage{color} and \xpackage{xcolor}
%    each overwrite \cs{normalcolor}. For example, after the package
%    loading order \xpackage{color}, \xpackage{pdfrender} and \xpackage{xcolor}
%    the patched version of \cs{normalcolor} is overwritten by
%    package \xpackage{xcolor}. Also using \cs{AtBeginDocument} for
%    patching is not enough. If package \xpackage{hyperref} is loaded
%    later, it might load package \xpackage{color} using \cs{AtBeginDocument}.
%    \begin{macro}{\PdfRender@NormalColorHook}
%    \begin{macrocode}
\def\PdfRender@NormalColorHook{}
%    \end{macrocode}
%    \end{macro}
%    \begin{macro}{\PdfRender@ColorSetGroupHook}
%    \begin{macrocode}
\def\PdfRender@ColorSetGroupHook{}
%    \end{macrocode}
%    \end{macro}
%    \begin{macro}{\PdfRender@TestBox}
%    \begin{macrocode}
\def\PdfRender@TestBox#1{%
  \setbox0=\color@hbox#1\color@endbox
}
%    \end{macrocode}
%    \end{macro}
%    \begin{macro}{\PdfRender@PatchNormalColor}
%    \begin{macrocode}
\def\PdfRender@PatchNormalColor{%
  \ltx@ifundefined{normalcolor}{%
    \gdef\normalcolor{\PdfRender@NormalColorHook}%
  }{%
    \begingroup
      \def\PdfRender@NormalColorHook{\let\PdfRender@temp=Y}%
      \PdfRender@TestBox{%
        \let\set@color\relax
        \normalcolor
        \ifx\PdfRender@temp Y%
        \else
          \ltx@GlobalAppendToMacro\normalcolor{%
            \PdfRender@NormalColorHook
          }%
        \fi
      }%
    \endgroup
  }%
  \ifx\@nodocument\relax
    \global\let\PdfRender@PatchNormalColor\relax
  \fi
}%
%    \end{macrocode}
%    \end{macro}
%    \begin{macro}{\PdfRender@PatchColorSetGroup}
%    \begin{macrocode}
\def\PdfRender@PatchColorSetGroup{%
  \begingroup
    \def\PdfRender@ColorSetGroupHook{\let\PdfRender@temp=Y}%
    \PdfRender@TestBox{%
      \let\set@color\relax
      \color@setgroup\color@endgroup
      \ifx\PdfRender@temp Y%
      \else
        \ltx@GlobalAppendToMacro\color@setgroup{%
          \PdfRender@ColorSetGroupHook
        }%
      \fi
    }%
  \endgroup
  \ifx\@nodocument\relax
    \global\let\PdfRender@PatchColorSetGroup\relax
  \fi
}%
%    \end{macrocode}
%    \end{macro}
%    \begin{macro}{\PdfRender@PatchColor}
%    \begin{macrocode}
\def\PdfRender@PatchColor{%
  \PdfRender@PatchNormalColor
  \PdfRender@PatchColorSetGroup
}
%    \end{macrocode}
%    \end{macro}
%    \begin{macrocode}
\PdfRender@PatchColor
\ltx@IfUndefined{AtBeginDocument}{}{%
  \AtBeginDocument{\PdfRender@PatchColor}%
}
%    \end{macrocode}
%    \cs{AfterPackage} is provided by package \xpackage{scrlfile}.
%    \begin{macrocode}
\ltx@IfUndefined{AfterPackage}{%
}{%
  \AfterPackage{color}{\PdfRender@PatchColor}%
  \AfterPackage{xcolor}{\PdfRender@PatchColor}%
  \AfterPackage{etoolbox}{%
    \AfterEndPreamble{\PdfRender@PatchColor}%
  }%
}%
%    \end{macrocode}
%    \cs{AfterEndPreamble} is provided by package \xpackage{etoolbox}.
%    \begin{macrocode}
\ltx@IfUndefined{AfterEndPreamble}{%
}{%
  \AfterEndPreamble{\PdfRender@PatchColor}%
}%
%    \end{macrocode}
%
%    \begin{macrocode}
\PdfRender@RequirePackage{kvsetkeys}[2010/01/28]
%    \end{macrocode}
%    \begin{macro}{\PdfRender@texorpdfstring}
%    \begin{macrocode}
\def\PdfRender@texorpdfstring{%
  \ltx@IfUndefined{texorpdfstring}\ltx@firstoftwo\texorpdfstring
}
%    \end{macrocode}
%    \end{macro}
%    \begin{macro}{\pdfrender}
%    \begin{macrocode}
\ltx@IfUndefined{DeclareRobustCommand}%
\ltx@firstoftwo\ltx@secondoftwo
{%
  \def\pdfrender#1%
}{%
  \newcommand{\pdfrender}{}%
  \DeclareRobustCommand*{\pdfrender}[1]%
}%
{%
  \PdfRender@texorpdfstring{%
    \PdfRender@PatchNormalColor
    \global\let\PdfRender@FillColor\ltx@empty
    \global\let\PdfRender@StrokeColor\ltx@empty
    \kvsetkeys{PDFRENDER}{#1}%
    \PdfRender@SetColor
  }{}%
}
%    \end{macrocode}
%    \end{macro}
%    \begin{macro}{\textpdfrender}
%    \begin{macrocode}
\ltx@IfUndefined{DeclareRobustCommand}%
\ltx@firstoftwo\ltx@secondoftwo
{%
  \long\def\textpdfrender#1#2%
}{%
  \newcommand{\textpdfrender}{}%
  \DeclareRobustCommand{\textpdfrender}[2]%
}%
{%
  \PdfRender@texorpdfstring{%
    \begingroup
      \pdfrender{#1}%
      #2%
    \endgroup
  }{#2}%
}
%    \end{macrocode}
%    \end{macro}
%
%    \begin{macro}{\ifPdfRender@Values}
%    \begin{macrocode}
\PdfRender@newif{Values}
%    \end{macrocode}
%    \end{macro}
%    \begin{macro}{\PdfRender@NewClassValues}
%    \begin{macrocode}
\def\PdfRender@NewClassValues#1#2#3#4{%
  \PdfRender@Valuestrue
  \PdfRender@NewClass{#1}{#2}{#3}{#4}{}%
}
%    \end{macrocode}
%    \end{macro}
%    \begin{macro}{\PdfRender@NewClass}
%    \begin{macrocode}
\def\PdfRender@NewClass#1#2#3#4#5{%
  \PdfRender@newif{Active#1}%
  \expandafter\def\csname PdfRender@Default#1\endcsname{#2}%
  \expandafter\let\csname PdfRender@Current#1\expandafter\endcsname
      \csname PdfRender@Default#1\endcsname
  \ifPdfRender@Stack
    \expandafter\edef\csname PdfRender@Init#1\endcsname{%
      \global\chardef
      \expandafter\noexpand\csname PdfRender@Stack#1\endcsname=%
          \noexpand\pdfcolorstackinit page direct{%
            \noexpand#3%
            \expandafter\noexpand\csname PdfRender@Default#1\endcsname
          }\relax
      \noexpand\@PackageInfo{pdfrender}{%
        New color stack `#1' = \noexpand\number
        \expandafter\noexpand\csname PdfRender@Stack#1\endcsname
      }%
      \gdef\expandafter\noexpand\csname PdfRender@Init#1\endcsname{}%
    }%
    \expandafter\edef\csname PdfRender@Set#1\endcsname{%
      \expandafter\noexpand\csname PdfRender@Init#1\endcsname
      \noexpand\pdfcolorstack
      \expandafter\noexpand\csname PdfRender@Stack#1\endcsname
      push{%
        #3{\expandafter\noexpand\csname PdfRender@Current#1\endcsname}%
      }%
      \noexpand\aftergroup
      \expandafter\noexpand\csname PdfRender@Reset#1\endcsname
    }%
    \expandafter\edef\csname PdfRender@Reset#1\endcsname{%
      \expandafter\noexpand\csname PdfRender@Init#1\endcsname
      \noexpand\pdfcolorstack
      \expandafter\noexpand\csname PdfRender@Stack#1\endcsname
      pop\relax
    }%
  \else
    \expandafter\edef\csname PdfRender@Set#1\endcsname{%
      \noexpand\pdfliteral direct{%
        #3{\expandafter\noexpand\csname PdfRender@Current#1\endcsname}%
      }%
      \noexpand\aftergroup
      \expandafter\noexpand\csname PdfRender@Reset#1\endcsname
    }%
    \expandafter\edef\csname PdfRender@Reset#1\endcsname{%
      \noexpand\pdfliteral direct{%
        #3{\expandafter\noexpand\csname PdfRender@Current#1\endcsname}%
      }%
    }%
  \fi
  \expandafter\edef\csname PdfRender@Normal#1\endcsname{%
    \let
    \expandafter\noexpand\csname PdfRender@Current#1\endcsname
    \expandafter\noexpand\csname PdfRender@Default#1\endcsname
    \noexpand\PdfRender@Set{#1}%
  }%
  \expandafter\ltx@GlobalAppendToMacro\expandafter\PdfRender@NormalColorHook
  \expandafter{%
    \csname PdfRender@Normal#1\endcsname
  }%
  \ltx@GlobalAppendToMacro\PdfRender@ColorSetGroupHook{%
    \PdfRender@Set{#1}%
  }%
  \ifPdfRender@Values
    \kv@parse@normalized{#4}{%
      \expandafter\let\csname PdfRender@#1@\kv@key\endcsname\kv@key
      \ifx\kv@value\relax
      \else
        \expandafter\let\csname PdfRender@#1@\kv@value\endcsname\kv@key
      \fi
      \ltx@gobbletwo
    }%
    \PdfRender@define@key{PDFRENDER}{#1}{%
      \global\csname PdfRender@Active#1true\endcsname
      \def\PdfRender@Current{##1}%
      \PdfRender@SetValidateValues{#1}%
    }%
    \PdfRender@Valuesfalse
  \else
    \PdfRender@define@key{PDFRENDER}{#1}{%
      \global\csname PdfRender@Active#1true\endcsname
      \expandafter\def\csname PdfRender@Current#1\endcsname{##1}%
      \ltx@IfUndefined{PdfRender@PostProcess#1}{%
      }{%
        \csname PdfRender@PostProcess#1\endcsname
      }%
      \PdfRender@SetValidate{#1}{#4}{#5}%
    }%
  \fi
}%
%    \end{macrocode}
%    \end{macro}
%    \begin{macro}{\PdfRender@define@key}
%    \begin{macrocode}
\ltx@IfUndefined{define@key}{%
  \def\PdfRender@define@key#1#2{%
    \expandafter\def\csname KV@#1@#2\endcsname##1%
  }%
}{%
  \let\PdfRender@define@key\define@key
}
%    \end{macrocode}
%    \end{macro}
%    \begin{macro}{\PdfRender@Set}
%    \begin{macrocode}
\def\PdfRender@Set#1{%
  \csname ifPdfRender@Active#1\endcsname
    \csname PdfRender@Set#1\expandafter\endcsname
  \fi
}
%    \end{macrocode}
%    \end{macro}
%    \begin{macro}{\PdfRender@Reset}
%    \begin{macrocode}
\def\PdfRender@Reset#1{%
  \csname ifPdfRender@Active#1\endcsname
    \csname PdfRender@Reset#1\expandafter\endcsname
  \fi
}
%    \end{macrocode}
%    \end{macro}
%    \begin{macro}{\PdfRender@ErrorInvalidValue}
%    \begin{macrocode}
\def\PdfRender@ErrorInvalidValue#1{%
  \PackageError{pdfrender}{%
    Ignoring parameter setting for `#1'\MessageBreak
    because of invalid value %
    `\csname PdfRender@Current#1\endcsname'%
  }\@ehc
  \expandafter\let\csname PdfRender@Current#1\endcsname\ltx@empty
}%
%    \end{macrocode}
%    \end{macro}
%    \begin{macro}{\PdfRender@SetValidate}
%    \begin{macrocode}
\ifPdfRender@Match
  \def\PdfRender@SetValidate#1#2#3{%
    \ifnum\pdfmatch{^(#2)$}{\csname PdfRender@Current#1\endcsname}=1 %
      \csname PdfRender@Set#1\expandafter\endcsname
    \else
      \PdfRender@ErrorInvalidValue{#1}%
    \fi
  }%
\else
  \def\PdfRender@SetValidate#1#2#3{%
    \expandafter\let\expandafter\PdfRender@Current
    \csname PdfRender@Current#1\endcsname
    #3%
    \ifx\PdfRender@Current\@empty
      \PdfRender@ErrorInvalidValue{#1}%
    \else
      \csname PdfRender@Set#1\expandafter\endcsname
    \fi
  }%
\fi
%    \end{macrocode}
%    \end{macro}
%    \begin{macro}{\PdfRender@SetValidateValues}
%    \begin{macrocode}
\def\PdfRender@SetValidateValues#1{%
  \ltx@IfUndefined{PdfRender@#1@\PdfRender@Current}{%
    \expandafter\let\csname PdfRender@Current#1\endcsname
                    \PdfRender@Current
    \PdfRender@ErrorInvalidValue{#1}%
  }{%
    \expandafter\edef\csname PdfRender@Current#1\endcsname{%
      \csname PdfRender@#1@\PdfRender@Current\endcsname
    }%
    \csname PdfRender@Set#1\endcsname
  }%
}
%    \end{macrocode}
%    \end{macro}
%    \begin{macro}{\PdfRender@OpValue}
%    \begin{macrocode}
\def\PdfRender@OpValue#1#2{#2\ltx@space#1}%
%    \end{macrocode}
%    \end{macro}
%    \begin{macro}{\PdfRender@OpName}
%    \begin{macrocode}
\def\PdfRender@OpName#1#2{/#2\ltx@space#1}%
%    \end{macrocode}
%    \end{macro}
%
% \subsection{Declare and setup parameters}
%
%    \begin{macrocode}
\PdfRender@NewClassValues{TextRenderingMode}%
                      {0}%
                      {\PdfRender@OpValue{Tr}}{%
  0=Fill,%
  1=Stroke,%
  2=FillStroke,%
  3=Invisible,%
  4=FillClip,%
  5=StrokeClip,%
  6=FillStrokeClip,%
  7=Clip,%
}%
\PdfRender@NewClass{LineWidth}{1}{\PdfRender@OpValue{w}}{%
  [0-9]+\string\.?[0-9]*|\string\.[0-9]+%
}{}%
\ltx@IfUndefined{dimexpr}{%
  \def\PdfRender@dimexpr{}%
}{%
  \let\PdfRender@dimexpr\dimexpr
}
\def\PdfRender@PostProcessLineWidth{%
  \begingroup
  \afterassignment\PdfRender@@PostProcessLineWidth
  \dimen0=\PdfRender@dimexpr\PdfRender@CurrentLineWidth bp %
  \PdfRender@let\PdfRender@relax\PdfRender@relax
}
\let\PdfRender@let\let
\let\PdfRender@relax\relax
\def\PdfRender@@PostProcessLineWidth#1\PdfRender@let{%
  \ifx\\#1\\%
    \endgroup
  \else
    \dimen0=.996264\dimen0 % 72/72.27
    \edef\x{\endgroup
      \def\noexpand\PdfRender@CurrentLineWidth{%
        \strip@pt\dimen0%
      }%
    }%
    \expandafter\x
  \fi
}
\PdfRender@NewClassValues{LineCapStyle}{0}{\PdfRender@OpValue{J}}{%
  0=Butt,%
  1=Round,%
  2=ProjectingSquare,%
}%
\PdfRender@NewClassValues{LineJoinStyle}{0}{\PdfRender@OpValue{j}}{%
  0=Miter,%
  1=Round,%
  2=Bevel,%
}%
\PdfRender@NewClass{MiterLimit}{10}{\PdfRender@OpValue{M}}{%
  [0-9]*[1-9][0-9]*\string\.?[0-9]*|%
  [0-9]*\string\.?[0-9]*[1-9][0-9]*%
}{}%
\PdfRender@NewClass{Flatness}{0}{\PdfRender@OpValue{i}}{%
  100(\string\.0*)?|[0-9][0-9](\string\.[0-9]*)?|\string\.[0-9]+%
}{}%
\PdfRender@NewClass{LineDashPattern}{[]0}{\PdfRender@OpValue{d}}{%
  \string\[%
  ( ?([0-9]+\string\.?[0-9]*|\string\.[0-9]+) ?)*%
  \string\] ?%
  ([0-9]+\string\.?[0-9]*|\string\.[0-9]+)%
}{}%
\PdfRender@NewClassValues{RenderingIntent}%
                      {RelativeColorimetric}%
                      {\PdfRender@OpName{ri}}{%
  AbsoluteColorimetric,%
  RelativeColorimetric,%
  Saturation,%
  Perceptual,%
}%
%    \end{macrocode}
%
% \subsection{Fill and stroke color support}
%
%    \begin{macrocode}
\PdfRender@define@key{PDFRENDER}{FillColor}{%
  \begingroup
    \def\PdfRender@Color{#1}%
    \ifx\PdfRender@Color\ltx@empty
      \global\let\PdfRender@FillColor\ltx@empty
    \else
      \PdfRender@ColorAvailable{%
        \PdfRender@TestBox{%
          \expandafter\PdfRender@TryColor\PdfRender@Color\ltx@empty
          \PdfRender@GetFillColor
          \ifx\PdfRender@FillColor\ltx@empty
            \@PackageWarning{pdfrender}{%
              Cannot extract fill color\MessageBreak
              from value `#1'%
            }%
          \fi
        }%
      }%
    \fi
  \endgroup
}
\PdfRender@define@key{PDFRENDER}{StrokeColor}{%
  \begingroup
    \def\PdfRender@Color{#1}%
    \ifx\PdfRender@Color\ltx@empty
      \global\let\PdfRender@StrokeColor\ltx@empty
    \else
      \PdfRender@ColorAvailable{%
        \PdfRender@TestBox{%
          \expandafter\PdfRender@TryColor\PdfRender@Color\ltx@empty
          \PdfRender@GetStrokeColor
          \ifx\PdfRender@StrokeColor\ltx@empty
            \@PackageWarning{pdfrender}{%
              Cannot extract stroke color\MessageBreak
              from value `#1'%
            }%
          \fi
        }%
      }%
    \fi
  \endgroup
}
%    \end{macrocode}
%    \begin{macro}{\PdfRender@ColorAvailable}
%    \begin{macrocode}
\def\PdfRender@ColorAvailable{%
  \@ifundefined{set@color}{%
    \@PackageError{pdfrender}{%
      Ignoring color options, because neither\MessageBreak
      package `color' nor package `xcolor' is loaded%
    }\@ehc
    \global\let\PdfRender@ColorAvailable\ltx@gobble
  }{%
    \global\let\PdfRender@ColorAvailable\ltx@firstofone
  }%
  \PdfRender@ColorAvailable
}
%    \end{macrocode}
%    \end{macro}
%    \begin{macro}{\PdfRender@TryColor}
%    \begin{macrocode}
\def\PdfRender@TryColor{%
  \@ifnextchar[\color\PdfRender@@TryColor
}
%    \end{macrocode}
%    \end{macro}
%    \begin{macro}{\PdfRender@@TryColor}
%    \begin{macrocode}
\def\PdfRender@@TryColor#1\ltx@empty{%
  \expandafter\color\expandafter{\PdfRender@Color}%
}
%    \end{macrocode}
%    \end{macro}
%    \begin{macro}{\PdfRender@SetColor}
%    \begin{macrocode}
\def\PdfRender@SetColor{%
  \chardef\PdfRender@NeedsCurrentColor=0 %
  \ifx\PdfRender@FillColor\ltx@empty
    \ifx\PdfRender@StrokeColor\ltx@empty
    \else
      \edef\PdfRender@CurrentColor{%
        \noexpand\PdfRender@FillColor\ltx@space\PdfRender@StrokeColor
      }%
      \chardef\PdfRender@NeedsCurrentColor=1 %
    \fi
  \else
    \ifx\PdfRender@StrokeColor\ltx@empty
      \edef\PdfRender@CurrentColor{%
        \PdfRender@FillColor\ltx@space\noexpand\PdfRender@StrokeColor
      }%
      \chardef\PdfRender@NeedsCurrentColor=2 %
    \else
      \edef\current@color{%
        \PdfRender@FillColor\ltx@space\PdfRender@StrokeColor
      }%
      \set@color
    \fi
  \fi
  \ifnum\PdfRender@NeedsCurrentColor=1 %
    \PdfRender@GetFillColor
    \ifx\PdfRender@FillColor\ltx@empty
      \@PackageWarning{pdfrender}{%
        Cannot extract current fill color%
      }%
    \else
      \edef\current@color{\PdfRender@CurrentColor}%
      \set@color
    \fi
  \else
    \ifnum\PdfRender@NeedsCurrentColor=2 %
      \PdfRender@GetStrokeColor
      \ifx\PdfRender@StrokeColor\ltx@empty
        \@PackageWarning{pdfrender}{%
          Cannot extract current stroke color%
        }%
      \else
        \edef\current@color{\PdfRender@CurrentColor}%
        \set@color
      \fi
    \fi
  \fi
}
%    \end{macrocode}
%    \end{macro}
%    \begin{macro}{\PdfRender@PatternFillColor}
%    \begin{macrocode}
\edef\PdfRender@PatternFillColor{ % space
  (%
    [0-9\string\.]+ g|%
    [0-9\string\.]+ [0-9\string\.]+ [0-9\string\.]+ rg|%
    [0-9\string\.]+ [0-9\string\.]+ %
    [0-9\string\.]+ [0-9\string\.]+ k%
  ) % space
  (.*)$%
}
%    \end{macrocode}
%    \end{macro}
%    \begin{macro}{\PdfRender@PatternStrokeColor}
%    \begin{macrocode}
\edef\PdfRender@PatternStrokeColor{ % space
  (%
    [0-9\string\.]+ G|%
    [0-9\string\.]+ [0-9\string\.]+ [0-9\string\.]+ RG|%
    [0-9\string\.]+ [0-9\string\.]+ %
    [0-9\string\.]+ [0-9\string\.]+ K%
  ) % space
  (.*)$%
}
%    \end{macrocode}
%    \end{macro}
%    \begin{macro}{\PdfRender@MatchPattern}
%    \begin{macrocode}
\def\PdfRender@MatchPattern#1{%
  \ifnum\pdfmatch{\PdfRender@Pattern}{\PdfRender@String}=1 %
    \xdef#1{%
      \expandafter\strip@prefix\pdflastmatch 1%
    }%
    \edef\PdfRender@String{%
      \expandafter\strip@prefix\pdflastmatch 2%
    }%
    \ifx\PdfRender@String\ltx@empty
    \else
      \expandafter\expandafter\expandafter\PdfRender@MatchPattern
      \expandafter\expandafter\expandafter#1%
    \fi
  \fi
}
%    \end{macrocode}
%    \end{macro}
%    \begin{macro}{\PdfRender@GetFillColor}
%    \begin{macrocode}
\def\PdfRender@GetFillColor{%
  \global\let\PdfRender@FillColor\ltx@empty
  \begingroup
    \ifPdfRender@Match
      \let\PdfRender@Pattern\PdfRender@PatternFillColor
      \edef\PdfRender@String{\ltx@space\current@color\ltx@space}%
      \PdfRender@MatchPattern\PdfRender@FillColor
    \else
      \edef\current@color{\current@color\ltx@space}%
      \let\PdfRender@OP\relax
      \PdfRender@FindOp{g}0%
      \PdfRender@FindOp{G}1%
      \PdfRender@FindOp{rg}0%
      \PdfRender@FindOp{RG}1%
      \PdfRender@FindOp{k}0%
      \PdfRender@FindOp{K}1%
      \PdfRender@FilterOp 0\PdfRender@FillColor
    \fi
  \endgroup
}
%    \end{macrocode}
%    \end{macro}
%    \begin{macro}{\PdfRender@GetStrokeColor}
%    \begin{macrocode}
\def\PdfRender@GetStrokeColor{%
  \global\let\PdfRender@StrokeColor\ltx@empty
  \begingroup
    \ifPdfRender@Match
      \let\PdfRender@Pattern\PdfRender@PatternStrokeColor
      \edef\PdfRender@String{\ltx@space\current@color\ltx@space}%
      \PdfRender@MatchPattern\PdfRender@StrokeColor
    \else
      \edef\current@color{\current@color\ltx@space}%
      \let\PdfRender@OP\relax
      \PdfRender@FindOp{g}0%
      \PdfRender@FindOp{G}1%
      \PdfRender@FindOp{rg}0%
      \PdfRender@FindOp{RG}1%
      \PdfRender@FindOp{k}0%
      \PdfRender@FindOp{K}1%
      \PdfRender@FilterOp 1\PdfRender@StrokeColor
    \fi
  \endgroup
}
%    \end{macrocode}
%    \end{macro}
%    \begin{macrocode}
\ifPdfRender@Match
  \expandafter\PdfRender@AtEnd
\fi%
%    \end{macrocode}
%    \begin{macro}{\PdfRender@FindOp}
%    \begin{macrocode}
\def\PdfRender@FindOp#1#2{%
  \def\PdfRender@temp##1 #1 ##2\@nil{%
    ##1%
    \ifx\\##2\\%
      \expandafter\@gobble
    \else
      \PdfRender@OP{#1}#2%
      \expandafter\@firstofone
    \fi
    {%
      \PdfRender@temp##2\@nil
    }%
  }%
  \edef\current@color{%
    \@firstofone{\expandafter\PdfRender@temp\current@color} #1 \@nil
  }%
}
%    \end{macrocode}
%    \end{macro}
%    \begin{macro}{\PdfRender@FilterOp}
%    \begin{macrocode}
\def\PdfRender@FilterOp#1#2{%
  \expandafter\PdfRender@@FilterOp\expandafter#1\expandafter#2%
      \current@color\PdfRender@OP{}{}%
}
%    \end{macrocode}
%    \end{macro}
%    \begin{macro}{\PdfRender@@FilterOp}
%    \begin{macrocode}
\def\PdfRender@@FilterOp#1#2#3\PdfRender@OP#4#5{%
  \ifx\\#4#5\\%
  \else
    \ifnum#1=#5 %
      \xdef#2{#3 #4}%
    \fi
    \expandafter\PdfRender@@FilterOp\expandafter#1\expandafter#2%
  \fi
}
%    \end{macrocode}
%    \end{macro}
%
%    \begin{macrocode}
\PdfRender@AtEnd%
%</package>
%    \end{macrocode}
%% \section{Installation}
%
% \subsection{Download}
%
% \paragraph{Package.} This package is available on
% CTAN\footnote{\CTANpkg{pdfrender}}:
% \begin{description}
% \item[\CTAN{macros/latex/contrib/oberdiek/pdfrender.dtx}] The source file.
% \item[\CTAN{macros/latex/contrib/oberdiek/pdfrender.pdf}] Documentation.
% \end{description}
%
%
% \paragraph{Bundle.} All the packages of the bundle `oberdiek'
% are also available in a TDS compliant ZIP archive. There
% the packages are already unpacked and the documentation files
% are generated. The files and directories obey the TDS standard.
% \begin{description}
% \item[\CTANinstall{install/macros/latex/contrib/oberdiek.tds.zip}]
% \end{description}
% \emph{TDS} refers to the standard ``A Directory Structure
% for \TeX\ Files'' (\CTANpkg{tds}). Directories
% with \xfile{texmf} in their name are usually organized this way.
%
% \subsection{Bundle installation}
%
% \paragraph{Unpacking.} Unpack the \xfile{oberdiek.tds.zip} in the
% TDS tree (also known as \xfile{texmf} tree) of your choice.
% Example (linux):
% \begin{quote}
%   |unzip oberdiek.tds.zip -d ~/texmf|
% \end{quote}
%
% \subsection{Package installation}
%
% \paragraph{Unpacking.} The \xfile{.dtx} file is a self-extracting
% \docstrip\ archive. The files are extracted by running the
% \xfile{.dtx} through \plainTeX:
% \begin{quote}
%   \verb|tex pdfrender.dtx|
% \end{quote}
%
% \paragraph{TDS.} Now the different files must be moved into
% the different directories in your installation TDS tree
% (also known as \xfile{texmf} tree):
% \begin{quote}
% \def\t{^^A
% \begin{tabular}{@{}>{\ttfamily}l@{ $\rightarrow$ }>{\ttfamily}l@{}}
%   pdfrender.sty & tex/generic/oberdiek/pdfrender.sty\\
%   pdfrender.pdf & doc/latex/oberdiek/pdfrender.pdf\\
%   pdfrender.dtx & source/latex/oberdiek/pdfrender.dtx\\
% \end{tabular}^^A
% }^^A
% \sbox0{\t}^^A
% \ifdim\wd0>\linewidth
%   \begingroup
%     \advance\linewidth by\leftmargin
%     \advance\linewidth by\rightmargin
%   \edef\x{\endgroup
%     \def\noexpand\lw{\the\linewidth}^^A
%   }\x
%   \def\lwbox{^^A
%     \leavevmode
%     \hbox to \linewidth{^^A
%       \kern-\leftmargin\relax
%       \hss
%       \usebox0
%       \hss
%       \kern-\rightmargin\relax
%     }^^A
%   }^^A
%   \ifdim\wd0>\lw
%     \sbox0{\small\t}^^A
%     \ifdim\wd0>\linewidth
%       \ifdim\wd0>\lw
%         \sbox0{\footnotesize\t}^^A
%         \ifdim\wd0>\linewidth
%           \ifdim\wd0>\lw
%             \sbox0{\scriptsize\t}^^A
%             \ifdim\wd0>\linewidth
%               \ifdim\wd0>\lw
%                 \sbox0{\tiny\t}^^A
%                 \ifdim\wd0>\linewidth
%                   \lwbox
%                 \else
%                   \usebox0
%                 \fi
%               \else
%                 \lwbox
%               \fi
%             \else
%               \usebox0
%             \fi
%           \else
%             \lwbox
%           \fi
%         \else
%           \usebox0
%         \fi
%       \else
%         \lwbox
%       \fi
%     \else
%       \usebox0
%     \fi
%   \else
%     \lwbox
%   \fi
% \else
%   \usebox0
% \fi
% \end{quote}
% If you have a \xfile{docstrip.cfg} that configures and enables \docstrip's
% TDS installing feature, then some files can already be in the right
% place, see the documentation of \docstrip.
%
% \subsection{Refresh file name databases}
%
% If your \TeX~distribution
% (\TeX\,Live, \mikTeX, \dots) relies on file name databases, you must refresh
% these. For example, \TeX\,Live\ users run \verb|texhash| or
% \verb|mktexlsr|.
%
% \subsection{Some details for the interested}
%
% \paragraph{Unpacking with \LaTeX.}
% The \xfile{.dtx} chooses its action depending on the format:
% \begin{description}
% \item[\plainTeX:] Run \docstrip\ and extract the files.
% \item[\LaTeX:] Generate the documentation.
% \end{description}
% If you insist on using \LaTeX\ for \docstrip\ (really,
% \docstrip\ does not need \LaTeX), then inform the autodetect routine
% about your intention:
% \begin{quote}
%   \verb|latex \let\install=y\input{pdfrender.dtx}|
% \end{quote}
% Do not forget to quote the argument according to the demands
% of your shell.
%
% \paragraph{Generating the documentation.}
% You can use both the \xfile{.dtx} or the \xfile{.drv} to generate
% the documentation. The process can be configured by the
% configuration file \xfile{ltxdoc.cfg}. For instance, put this
% line into this file, if you want to have A4 as paper format:
% \begin{quote}
%   \verb|\PassOptionsToClass{a4paper}{article}|
% \end{quote}
% An example follows how to generate the
% documentation with pdf\LaTeX:
% \begin{quote}
%\begin{verbatim}
%pdflatex pdfrender.dtx
%makeindex -s gind.ist pdfrender.idx
%pdflatex pdfrender.dtx
%makeindex -s gind.ist pdfrender.idx
%pdflatex pdfrender.dtx
%\end{verbatim}
% \end{quote}
%
% \section{Acknowledgement}
%
% \begin{description}
% \item[Friedrich Vosberg] asked in the
%   newsgroup \textsf{de.comp.text.tex} for the font
%   outline feature \cite{news-Vosberg}.
% \item[Gaius Pupus] proposed the basic method
%   using \cs{pdfliteral} in this thread \cite{news-Pupus}.
% \item[Rolf Niepraschk] added color support \cite{news-Niepraschk}.
% \end{description}
%
% \begin{thebibliography}{9}
% \raggedright
%
% \bibitem{pdf-spec}
%   Adobe Systems Incorporated. \textit{PDF Reference --
%   Adobe Portable Document format -- Version 1.7}.
%   6th ed. 2006.
%   \textsc{URL:}\,\url{http://www.adobe.com/devnet/acrobat/pdfs/pdf_reference_1-7.pdf}.
%
% \bibitem{news-Vosberg}
%   Friedrich Vosberg, \textit{Text in Buchstabenumrissen},
%   de.comp.text.tex, 2010-01-22.
%   \textsc{URL:}\,\url{https://groups.google.com/group/de.comp.text.tex/msg/f442310ac8b2d506}.
%
% \bibitem{news-Pupus}
%   Gaius Pupus, \textit{Re: Text in Buchstabenumrissen},
%   de.comp.text.tex, 2010-01-23.
%   \textsc{URL:}\,\url{https://groups.google.com/group/de.comp.text.tex/msg/95d890d77ac47eb1}.
%
% \bibitem{news-Niepraschk}
%   Rolf Niepraschk, \textit{Re: Text in Buchstabenumrissen},
%   de.comp.text.tex, 2010-01-24.
%   \textsc{URL:}\,\url{https://groups.google.com/group/de.comp.text.tex/msg/4eb61a5879db54db}.
%
% \end{thebibliography}
%
% \begin{History}
%   \begin{Version}{2010/01/26 v1.0}
%   \item
%     The first version.
%   \end{Version}
%   \begin{Version}{2010/01/27 v1.1}
%   \item
%     Macros \cs{pdfrender} and \cs{textpdfrender} are made robust.
%   \item
%     Color extraction rewritten for the case that \cs{pdfmatch}
%     is not available. This fixes wrong color assigments in
%     case of nesting.
%   \item
%     Color extraction of case \cs{pdfmatch} is fixed
%     for the case that the color string contains several
%     fill or several stroke operations.
%   \end{Version}
%   \begin{Version}{2010/01/28 v1.2}
%   \item
%     Dependency from package \xpackage{color} is removed.
%   \item
%     Compatibility for \plainTeX\ and even ini\TeX\ added.
%   \end{Version}
%   \begin{Version}{2016/05/14 v1.3}
%   \item
%     Use package \xpackage{luatex85} for compatibility with new LuaTeX.
%   \end{Version}
%   \begin{Version}{2016/05/17 v1.4}
%   \item
%     Documentation updates.
%   \item adjust luatex85 reference so that it works in plain TeX.
%   \end{Version}
%   \begin{Version}{2018/11/01 v1.5}
%   \item
%     Remove luatex85 dependency
%   \end{Version}
%   \begin{Version}{2019/12/29 v1.6}
%   \item
%     \xpackage{iftex} package.
%   \end{Version}
% \end{History}
%
% \PrintIndex
%
% \Finale
\endinput

%        (quote the arguments according to the demands of your shell)
%
% Documentation:
%    (a) If pdfrender.drv is present:
%           latex pdfrender.drv
%    (b) Without pdfrender.drv:
%           latex pdfrender.dtx; ...
%    The class ltxdoc loads the configuration file ltxdoc.cfg
%    if available. Here you can specify further options, e.g.
%    use A4 as paper format:
%       \PassOptionsToClass{a4paper}{article}
%
%    Programm calls to get the documentation (example):
%       pdflatex pdfrender.dtx
%       makeindex -s gind.ist pdfrender.idx
%       pdflatex pdfrender.dtx
%       makeindex -s gind.ist pdfrender.idx
%       pdflatex pdfrender.dtx
%
% Installation:
%    TDS:tex/generic/oberdiek/pdfrender.sty
%    TDS:doc/latex/oberdiek/pdfrender.pdf
%    TDS:source/latex/oberdiek/pdfrender.dtx
%
%<*ignore>
\begingroup
  \catcode123=1 %
  \catcode125=2 %
  \def\x{LaTeX2e}%
\expandafter\endgroup
\ifcase 0\ifx\install y1\fi\expandafter
         \ifx\csname processbatchFile\endcsname\relax\else1\fi
         \ifx\fmtname\x\else 1\fi\relax
\else\csname fi\endcsname
%</ignore>
%<*install>
\input docstrip.tex
\Msg{************************************************************************}
\Msg{* Installation}
\Msg{* Package: pdfrender 2019/12/29 v1.6 Access to some PDF graphics parameters (HO)}
\Msg{************************************************************************}

\keepsilent
\askforoverwritefalse

\let\MetaPrefix\relax
\preamble

This is a generated file.

Project: pdfrender
Version: 2019/12/29 v1.6

Copyright (C)
   2010 Heiko Oberdiek
   2016-2019 Oberdiek Package Support Group

This work may be distributed and/or modified under the
conditions of the LaTeX Project Public License, either
version 1.3c of this license or (at your option) any later
version. This version of this license is in
   https://www.latex-project.org/lppl/lppl-1-3c.txt
and the latest version of this license is in
   https://www.latex-project.org/lppl.txt
and version 1.3 or later is part of all distributions of
LaTeX version 2005/12/01 or later.

This work has the LPPL maintenance status "maintained".

The Current Maintainers of this work are
Heiko Oberdiek and the Oberdiek Package Support Group
https://github.com/ho-tex/oberdiek/issues


The Base Interpreter refers to any `TeX-Format',
because some files are installed in TDS:tex/generic//.

This work consists of the main source file pdfrender.dtx
and the derived files
   pdfrender.sty, pdfrender.pdf, pdfrender.ins, pdfrender.drv,
   pdfrender-test1.tex, pdfrender-test2.tex,
   pdfrender-test3.tex, pdfrender-test4.tex,
   pdfrender-test5.tex.

\endpreamble
\let\MetaPrefix\DoubleperCent

\generate{%
  \file{pdfrender.ins}{\from{pdfrender.dtx}{install}}%
  \file{pdfrender.drv}{\from{pdfrender.dtx}{driver}}%
  \usedir{tex/generic/oberdiek}%
  \file{pdfrender.sty}{\from{pdfrender.dtx}{package}}%
%  \usedir{doc/latex/oberdiek/test}%
%  \file{pdfrender-test1.tex}{\from{pdfrender.dtx}{test1}}%
%  \file{pdfrender-test2.tex}{\from{pdfrender.dtx}{test2}}%
%  \file{pdfrender-test3.tex}{\from{pdfrender.dtx}{test3}}%
%  \file{pdfrender-test4.tex}{\from{pdfrender.dtx}{test4}}%
%  \file{pdfrender-test5.tex}{\from{pdfrender.dtx}{test5}}%
}

\catcode32=13\relax% active space
\let =\space%
\Msg{************************************************************************}
\Msg{*}
\Msg{* To finish the installation you have to move the following}
\Msg{* file into a directory searched by TeX:}
\Msg{*}
\Msg{*     pdfrender.sty}
\Msg{*}
\Msg{* To produce the documentation run the file `pdfrender.drv'}
\Msg{* through LaTeX.}
\Msg{*}
\Msg{* Happy TeXing!}
\Msg{*}
\Msg{************************************************************************}

\endbatchfile
%</install>
%<*ignore>
\fi
%</ignore>
%<*driver>
\NeedsTeXFormat{LaTeX2e}
\ProvidesFile{pdfrender.drv}%
  [2019/12/29 v1.6 Access to some PDF graphics parameters (HO)]%
\documentclass{ltxdoc}
\usepackage{holtxdoc}[2011/11/22]
\usepackage{longtable}
\renewcommand*{\arraystretch}{1.2}
\begin{document}
  \DocInput{pdfrender.dtx}%
\end{document}
%</driver>
% \fi
%
%
%
% \GetFileInfo{pdfrender.drv}
%
% \title{The \xpackage{pdfrender} package}
% \date{2019/12/29 v1.6}
% \author{Heiko Oberdiek\thanks
% {Please report any issues at \url{https://github.com/ho-tex/oberdiek/issues}}}
%
% \maketitle
%
% \begin{abstract}
% The PDF format has some graphics parameter like
% line width or text rendering mode. This package
% provides an interface for setting these parameters.
% \end{abstract}
%
% \tableofcontents
%
% \section{Documentation}
%
% This package \xpackage{pdfrender} defines an interface for
% PDF specific parameters that affects the rendering of
% graphics or text. The interface and its implementation
% uses the same technique as package \xpackage{color}
% for color settings. Therefore this package is loaded to
% enable \LaTeX's color interface.
%
% At different places \LaTeX\ uses \cs{normalcolor} to avoid that
% header, footer or floats are print in the current color of the
% main text. \cs{setgroup@color} is used to start a save box
% with the color that is set at box saving time.
% Package \xpackage{pdfrender} extends these macros to
% add its own hooks of its parameters. Therefore \LaTeX3 should
% generalize \LaTeXe's color interface.
%
% \subsection{Usage}
%
% In \LaTeX\ the package is loaded as normal package. Options
% are not defined for this package.
% \begin{quote}
% |\usepackage{pdfrender}|
% \end{quote}
%
% This package can also be used in \plainTeX\ and even ini\TeX:
% \begin{quote}
% |input pdfrender.sty|
% \end{quote}
%
% \subsection{Macros}
%
% \begin{declcs}{pdfrender} \M{key value list}
% \end{declcs}
% The first parameter \meta{key value list} contains a
% list of parameter settings. The key entry is the parameter name.
% The macro works like
% \cs{color} (without optional argument) for color setting.
%
% \begin{declcs}{textpdfrender} \M{key value list} \M{text}
% \end{declcs}
% In the same way as \cs{pdfrender} the first argument
% specifies the parameters that should be set. This parameter
% setting affects \meta{text} only. Basically it works
% the same way as \cs{textcolor} (without optional argument).
%
% \subsection{Parameters}
%
% The following table shows an overview for the supported
% parameters and values:
%
% \begin{longtable}{@{}l|l|l@{}}
% \bfseries Parameter & \bfseries Value & \bfseries Alias\\*
% \hline
% \endhead
% \xoption{TextRenderingMode}
%   & \texttt{0} & \texttt{Fill}\\*
%   & \texttt{1} & \texttt{Stroke}\\*
%   & \texttt{2} & \texttt{FillStroke}\\*
%   & \texttt{3} & \texttt{Invisible}\\*
%   & \texttt{4} & \texttt{FillClip}\\*
%   & \texttt{5} & \texttt{StrokeClip}\\*
%   & \texttt{6} & \texttt{FillStrokeClip}\\*
%   & \texttt{7} & \texttt{Clip}\\
% \hline
% \xoption{LineWidth}
%   & \itshape positive number, & \itshape \TeX\ dimen\\*[-.2ex]
%   & \qquad \itshape unit is bp & \\
% \hline
% \xoption{LineCapStyle}
%   & \texttt{0} & \texttt{Butt}\\*
%   & \texttt{1} & \texttt{Round}\\*
%   & \texttt{2} & \texttt{ProjectingSquare}\\
% \hline
% \xoption{LineJoinStyle}
%   & \texttt{0} & \texttt{Miter}\\*
%   & \texttt{1} & \texttt{Round}\\*
%   & \texttt{2} & \texttt{Bevel}\\
% \hline
% \xoption{MiterLimit}
%   & \multicolumn{2}{l@{}}{\itshape positive number} \\
% \hline
% \xoption{Flatness}
%   & \multicolumn{2}{l@{}}{\itshape number between 0 and 100} \\
% \hline
% \xoption{LineDashPattern}
%   & \multicolumn{2}{l@{}}{^^A
%       \itshape numbers in square brackets,^^A
%     } \\*[-.2ex]
%   & \multicolumn{2}{l@{}}{^^A
%       \qquad \itshape followed by number,^^A
%     } \\*
%   & \multicolumn{2}{l@{}}{^^A
%       \qquad \itshape units are bp^^A
%     } \\
% \hline
% \xoption{RenderingIntent}
%   & \multicolumn{2}{l@{}}{\texttt{AbsoluteColorimetric}} \\*
%   & \multicolumn{2}{l@{}}{\texttt{RelativeColorimetric}} \\*
%   & \multicolumn{2}{l@{}}{\texttt{Saturation}} \\*
%   & \multicolumn{2}{l@{}}{\texttt{Perceptual}} \\
% \hline
% \xoption{FillColor}
%   & & \itshape color specification \\
% \hline
% \xoption{StrokeColor}
%   & & \itshape color specification \\
% \end{longtable}
%
% \subsubsection{Details}
%
% The description and specification of these parameters
% are available in the PDF specification \cite{pdf-spec}.
% Therefore they are not repeated here.
%
% \begin{description}
% \item[Value:]
%   The values in the second column lists or describe
%   the values that are specified by the PDF specification.
% \item[Alias:] Instead of magic numbers the package also
%   defines some aliases that can be given as value.
%   Example: |LineCapStyle=Round| has the same effect as
%   |LineCapStyle=1|.
% \item[Number:] The term \emph{number} means an integer
%   or real number. The real number is given as plain
%   decimal number without exponent. The decimal separator is a period.
%   At least one digit must be present.
% \item[\xoption{LineWidth}:] As alias a \TeX\ dimen specification
%   can be given. This includes explicit specifications with number
%   and unit, e.g. |LineWidth=0.5pt|. Also \LaTeX\ length registers
%   may be used. If \eTeX's \cs{dimexpr} is available, then it is
%   automatically added. However package \xpackage{calc} is not
%   supported.
% \item[\xoption{FillColor}, \xoption{StrokeColor}:]
%   Package \xpackage{color} or \xpackage{xcolor} must be loaded
%   before these options can be used (since version 1.2).
%   \LaTeX's color support sets both colors at the same time
%   to the same value. However parameter \xoption{TextRenderingMode}
%   offers the value \texttt{FillStroke} that makes only sense, if
%   the two color types can be set separately.
%   If one of the options \xoption{FillColor} or \xoption{StrokeColor}
%   is specified, then also the color is set. For compatibility
%   with the \LaTeX\ color packages (\xpackage{color} or \xpackage{xcolor}),
%   always both colors must be set. Thus if one of them is not specified,
%   it is taken from the current color.
%
%   Both options \xoption{FillColor} and \xoption{StrokeColor}
%   expect a \LaTeX\ color specification as value.
%   Also the optional color model argument is supported.
%   Example:
%   \begin{quote}
%     |FillColor=yellow|,\\
%     |StrokeColor=[cmyk]{1,.5,0,0}|\\
%   \end{quote}
% \end{description}
%
% \subsection{Color stack}
%
% If the \pdfTeX\ version provides color stacks, then each
% parameter is assigned a page based color stack. The assignment
% of a stack takes place, when its parameter is set the first time.
% This avoids the use of color stacks that are not needed.
%
% \StopEventually{
% }
%
% \section{Implementation}
%    \begin{macrocode}
%<*package>
%    \end{macrocode}
%    Reload check, especially if the package is not used with \LaTeX.
%    \begin{macrocode}
\begingroup\catcode61\catcode48\catcode32=10\relax%
  \catcode13=5 % ^^M
  \endlinechar=13 %
  \catcode35=6 % #
  \catcode39=12 % '
  \catcode44=12 % ,
  \catcode45=12 % -
  \catcode46=12 % .
  \catcode58=12 % :
  \catcode64=11 % @
  \catcode123=1 % {
  \catcode125=2 % }
  \expandafter\let\expandafter\x\csname ver@pdfrender.sty\endcsname
  \ifx\x\relax % plain-TeX, first loading
  \else
    \def\empty{}%
    \ifx\x\empty % LaTeX, first loading,
      % variable is initialized, but \ProvidesPackage not yet seen
    \else
      \expandafter\ifx\csname PackageInfo\endcsname\relax
        \def\x#1#2{%
          \immediate\write-1{Package #1 Info: #2.}%
        }%
      \else
        \def\x#1#2{\PackageInfo{#1}{#2, stopped}}%
      \fi
      \x{pdfrender}{The package is already loaded}%
      \aftergroup\endinput
    \fi
  \fi
\endgroup%
%    \end{macrocode}
%    Package identification:
%    \begin{macrocode}
\begingroup\catcode61\catcode48\catcode32=10\relax%
  \catcode13=5 % ^^M
  \endlinechar=13 %
  \catcode35=6 % #
  \catcode39=12 % '
  \catcode40=12 % (
  \catcode41=12 % )
  \catcode44=12 % ,
  \catcode45=12 % -
  \catcode46=12 % .
  \catcode47=12 % /
  \catcode58=12 % :
  \catcode64=11 % @
  \catcode91=12 % [
  \catcode93=12 % ]
  \catcode123=1 % {
  \catcode125=2 % }
  \expandafter\ifx\csname ProvidesPackage\endcsname\relax
    \def\x#1#2#3[#4]{\endgroup
      \immediate\write-1{Package: #3 #4}%
      \xdef#1{#4}%
    }%
  \else
    \def\x#1#2[#3]{\endgroup
      #2[{#3}]%
      \ifx#1\@undefined
        \xdef#1{#3}%
      \fi
      \ifx#1\relax
        \xdef#1{#3}%
      \fi
    }%
  \fi
\expandafter\x\csname ver@pdfrender.sty\endcsname
\ProvidesPackage{pdfrender}%
  [2019/12/29 v1.6 Access to some PDF graphics parameters (HO)]%
%    \end{macrocode}
%
%    \begin{macrocode}
\begingroup\catcode61\catcode48\catcode32=10\relax%
  \catcode13=5 % ^^M
  \endlinechar=13 %
  \catcode123=1 % {
  \catcode125=2 % }
  \catcode64=11 % @
  \def\x{\endgroup
    \expandafter\edef\csname PdfRender@AtEnd\endcsname{%
      \endlinechar=\the\endlinechar\relax
      \catcode13=\the\catcode13\relax
      \catcode32=\the\catcode32\relax
      \catcode35=\the\catcode35\relax
      \catcode61=\the\catcode61\relax
      \catcode64=\the\catcode64\relax
      \catcode123=\the\catcode123\relax
      \catcode125=\the\catcode125\relax
    }%
  }%
\x\catcode61\catcode48\catcode32=10\relax%
\catcode13=5 % ^^M
\endlinechar=13 %
\catcode35=6 % #
\catcode64=11 % @
\catcode123=1 % {
\catcode125=2 % }
\def\TMP@EnsureCode#1#2{%
  \edef\PdfRender@AtEnd{%
    \PdfRender@AtEnd
    \catcode#1=\the\catcode#1\relax
  }%
  \catcode#1=#2\relax
}
\TMP@EnsureCode{10}{12}% ^^J
\TMP@EnsureCode{36}{3}% $
\TMP@EnsureCode{39}{12}% '
\TMP@EnsureCode{40}{12}% (
\TMP@EnsureCode{41}{12}% )
\TMP@EnsureCode{42}{12}% *
\TMP@EnsureCode{43}{12}% +
\TMP@EnsureCode{44}{12}% ,
\TMP@EnsureCode{45}{12}% -
\TMP@EnsureCode{46}{12}% .
\TMP@EnsureCode{47}{12}% /
\TMP@EnsureCode{58}{12}% :
\TMP@EnsureCode{59}{12}% ;
\TMP@EnsureCode{60}{12}% <
\TMP@EnsureCode{62}{12}% >
\TMP@EnsureCode{63}{12}% ?
\TMP@EnsureCode{91}{12}% [
\TMP@EnsureCode{93}{12}% ]
\TMP@EnsureCode{94}{7}% ^ (superscript)
\TMP@EnsureCode{96}{12}% `
\TMP@EnsureCode{124}{12}% |
%    \end{macrocode}
%
%    \begin{macrocode}
\def\PdfRender@AtEndHook{}
\expandafter\def\expandafter\PdfRender@AtEnd\expandafter{%
  \expandafter\PdfRender@AtEndHook
  \PdfRender@AtEnd
  \endinput
}
%    \end{macrocode}
%
% \subsection{Look for \pdfTeX, its mode and features}
%
%    \begin{macro}{\PdfRender@newif}
%    \begin{macrocode}
\def\PdfRender@newif#1{%
  \expandafter\edef\csname PdfRender@#1true\endcsname{%
    \let
    \expandafter\noexpand\csname ifPdfRender@#1\endcsname
    \noexpand\iftrue
  }%
  \expandafter\edef\csname PdfRender@#1false\endcsname{%
    \let
    \expandafter\noexpand\csname ifPdfRender@#1\endcsname
    \noexpand\iffalse
  }%
  \csname PdfRender@#1false\endcsname
}
%    \end{macrocode}
%    \end{macro}
%
%    \begin{macro}{\ifPdfRender@Stack}
%    \begin{macrocode}
\PdfRender@newif{Stack}
%    \end{macrocode}
%    \end{macro}
%    \begin{macro}{\ifPdfRender@Match}
%    \begin{macrocode}
\PdfRender@newif{Match}
%    \end{macrocode}
%    \end{macro}
%
%    \begin{macro}{\PdfRender@RequirePackage}
%    \begin{macrocode}
\begingroup\expandafter\expandafter\expandafter\endgroup
\expandafter\ifx\csname RequirePackage\endcsname\relax
  \def\PdfRender@RequirePackage#1[#2]{%
    \expandafter\def\expandafter\PdfRender@AtEndHook\expandafter{%
      \PdfRender@AtEndHook
      \ltx@ifpackagelater{#1}{#2}{}{%
        \@PackageWarningNoLine{pdfrender}{%
          You have requested version\MessageBreak
          `#2' of package `#1',\MessageBreak
          but only version\MessageBreak
          `\csname ver@#1.\ltx@pkgextension\endcsname'\MessageBreak
          is available%
        }%
      }%
    }%
    \input #1.sty\relax
  }%
\else
  \let\PdfRender@RequirePackage\RequirePackage
\fi
%    \end{macrocode}
%    \end{macro}
% Luatex compatibility
%    \begin{macrocode}
\ifx\pdfextension\@undefined\else
  \def\pdfcolorstackinit {\pdffeedback colorstackinit}
  \protected\def\pdfcolorstack     {\pdfextension colorstack}
  \protected\def\pdfliteral        {\pdfextension literal}
\fi
%    \end{macrocode}
%
%    \begin{macrocode}
\PdfRender@RequirePackage{iftex}[2019/11/07]
\PdfRender@RequirePackage{infwarerr}[2007/09/09]
\PdfRender@RequirePackage{ltxcmds}[2010/01/28]
%    \end{macrocode}
%
%    \begin{macrocode}
\ifpdf
  \ltx@IfUndefined{pdfcolorstackinit}{%
    \@PackageWarning{pdfrender}{%
      Missing \string\pdfcolorstackinit
    }%
  }{%
    \PdfRender@Stacktrue
  }%
  \ltx@IfUndefined{pdfmatch}{%
    \@PackageInfoNoLine{pdfrender}{%
      \string\pdfmatch\ltx@space not found. %
      Therefore the values\MessageBreak
      of some parameters are not validated%
    }%
  }{%
    \PdfRender@Matchtrue
  }%
\else
  \@PackageWarning{pdfrender}{%
    Missing pdfTeX in PDF mode%
  }%
%    \end{macrocode}
%    \begin{macrocode}
  \ltx@IfUndefined{newcommand}{%
%    \end{macrocode}
%    \begin{macro}{\pdfrender}
%    \begin{macrocode}
    \def\pdfrender#1{}%
%    \end{macrocode}
%    \end{macro}
%    \begin{macro}{\textpdfrender}
%    \begin{macrocode}
    \long\def\textpdfrender#1#2{#2}%
%    \end{macrocode}
%    \end{macro}
%    \begin{macrocode}
  }{%
%    \end{macrocode}
%    \begin{macro}{\pdfrender}
%    \begin{macrocode}
    \newcommand*{\pdfrender}[1]{}%
%    \end{macrocode}
%    \end{macro}
%    \begin{macro}{\textpdfrender}
%    \begin{macrocode}
    \newcommand{\textpdfrender}[2]{#2}%
%    \end{macrocode}
%    \end{macro}
%    \begin{macrocode}
  }%
  \expandafter\PdfRender@AtEnd
\fi%
%    \end{macrocode}
%
% \subsection{Enable color support of \LaTeX}
%
%    \begin{macrocode}
\ltx@ifpackageloaded{color}{}{%
  \def\color@setgroup{\begingroup\set@color}%
  \let\color@begingroup\begingroup
  \def\color@endgroup{\endgraf\endgroup}%
  \def\color@hbox{\hbox\bgroup\color@begingroup}%
  \def\color@vbox{\vbox\bgroup\color@begingroup}%
  \def\color@endbox{\color@endgroup\egroup}%
  \ltx@ifundefined{bgroup}{%
    \let\bgroup={\let\egroup=}%
  }{}%
  \ltx@ifundefined{endgraf}{%
    \let\endgraf=\par
  }{}%
}
%    \end{macrocode}
%
% \subsection{Hook into \cs{normalcolor}}
%
%    The problem is that packages \xpackage{color} and \xpackage{xcolor}
%    each overwrite \cs{normalcolor}. For example, after the package
%    loading order \xpackage{color}, \xpackage{pdfrender} and \xpackage{xcolor}
%    the patched version of \cs{normalcolor} is overwritten by
%    package \xpackage{xcolor}. Also using \cs{AtBeginDocument} for
%    patching is not enough. If package \xpackage{hyperref} is loaded
%    later, it might load package \xpackage{color} using \cs{AtBeginDocument}.
%    \begin{macro}{\PdfRender@NormalColorHook}
%    \begin{macrocode}
\def\PdfRender@NormalColorHook{}
%    \end{macrocode}
%    \end{macro}
%    \begin{macro}{\PdfRender@ColorSetGroupHook}
%    \begin{macrocode}
\def\PdfRender@ColorSetGroupHook{}
%    \end{macrocode}
%    \end{macro}
%    \begin{macro}{\PdfRender@TestBox}
%    \begin{macrocode}
\def\PdfRender@TestBox#1{%
  \setbox0=\color@hbox#1\color@endbox
}
%    \end{macrocode}
%    \end{macro}
%    \begin{macro}{\PdfRender@PatchNormalColor}
%    \begin{macrocode}
\def\PdfRender@PatchNormalColor{%
  \ltx@ifundefined{normalcolor}{%
    \gdef\normalcolor{\PdfRender@NormalColorHook}%
  }{%
    \begingroup
      \def\PdfRender@NormalColorHook{\let\PdfRender@temp=Y}%
      \PdfRender@TestBox{%
        \let\set@color\relax
        \normalcolor
        \ifx\PdfRender@temp Y%
        \else
          \ltx@GlobalAppendToMacro\normalcolor{%
            \PdfRender@NormalColorHook
          }%
        \fi
      }%
    \endgroup
  }%
  \ifx\@nodocument\relax
    \global\let\PdfRender@PatchNormalColor\relax
  \fi
}%
%    \end{macrocode}
%    \end{macro}
%    \begin{macro}{\PdfRender@PatchColorSetGroup}
%    \begin{macrocode}
\def\PdfRender@PatchColorSetGroup{%
  \begingroup
    \def\PdfRender@ColorSetGroupHook{\let\PdfRender@temp=Y}%
    \PdfRender@TestBox{%
      \let\set@color\relax
      \color@setgroup\color@endgroup
      \ifx\PdfRender@temp Y%
      \else
        \ltx@GlobalAppendToMacro\color@setgroup{%
          \PdfRender@ColorSetGroupHook
        }%
      \fi
    }%
  \endgroup
  \ifx\@nodocument\relax
    \global\let\PdfRender@PatchColorSetGroup\relax
  \fi
}%
%    \end{macrocode}
%    \end{macro}
%    \begin{macro}{\PdfRender@PatchColor}
%    \begin{macrocode}
\def\PdfRender@PatchColor{%
  \PdfRender@PatchNormalColor
  \PdfRender@PatchColorSetGroup
}
%    \end{macrocode}
%    \end{macro}
%    \begin{macrocode}
\PdfRender@PatchColor
\ltx@IfUndefined{AtBeginDocument}{}{%
  \AtBeginDocument{\PdfRender@PatchColor}%
}
%    \end{macrocode}
%    \cs{AfterPackage} is provided by package \xpackage{scrlfile}.
%    \begin{macrocode}
\ltx@IfUndefined{AfterPackage}{%
}{%
  \AfterPackage{color}{\PdfRender@PatchColor}%
  \AfterPackage{xcolor}{\PdfRender@PatchColor}%
  \AfterPackage{etoolbox}{%
    \AfterEndPreamble{\PdfRender@PatchColor}%
  }%
}%
%    \end{macrocode}
%    \cs{AfterEndPreamble} is provided by package \xpackage{etoolbox}.
%    \begin{macrocode}
\ltx@IfUndefined{AfterEndPreamble}{%
}{%
  \AfterEndPreamble{\PdfRender@PatchColor}%
}%
%    \end{macrocode}
%
%    \begin{macrocode}
\PdfRender@RequirePackage{kvsetkeys}[2010/01/28]
%    \end{macrocode}
%    \begin{macro}{\PdfRender@texorpdfstring}
%    \begin{macrocode}
\def\PdfRender@texorpdfstring{%
  \ltx@IfUndefined{texorpdfstring}\ltx@firstoftwo\texorpdfstring
}
%    \end{macrocode}
%    \end{macro}
%    \begin{macro}{\pdfrender}
%    \begin{macrocode}
\ltx@IfUndefined{DeclareRobustCommand}%
\ltx@firstoftwo\ltx@secondoftwo
{%
  \def\pdfrender#1%
}{%
  \newcommand{\pdfrender}{}%
  \DeclareRobustCommand*{\pdfrender}[1]%
}%
{%
  \PdfRender@texorpdfstring{%
    \PdfRender@PatchNormalColor
    \global\let\PdfRender@FillColor\ltx@empty
    \global\let\PdfRender@StrokeColor\ltx@empty
    \kvsetkeys{PDFRENDER}{#1}%
    \PdfRender@SetColor
  }{}%
}
%    \end{macrocode}
%    \end{macro}
%    \begin{macro}{\textpdfrender}
%    \begin{macrocode}
\ltx@IfUndefined{DeclareRobustCommand}%
\ltx@firstoftwo\ltx@secondoftwo
{%
  \long\def\textpdfrender#1#2%
}{%
  \newcommand{\textpdfrender}{}%
  \DeclareRobustCommand{\textpdfrender}[2]%
}%
{%
  \PdfRender@texorpdfstring{%
    \begingroup
      \pdfrender{#1}%
      #2%
    \endgroup
  }{#2}%
}
%    \end{macrocode}
%    \end{macro}
%
%    \begin{macro}{\ifPdfRender@Values}
%    \begin{macrocode}
\PdfRender@newif{Values}
%    \end{macrocode}
%    \end{macro}
%    \begin{macro}{\PdfRender@NewClassValues}
%    \begin{macrocode}
\def\PdfRender@NewClassValues#1#2#3#4{%
  \PdfRender@Valuestrue
  \PdfRender@NewClass{#1}{#2}{#3}{#4}{}%
}
%    \end{macrocode}
%    \end{macro}
%    \begin{macro}{\PdfRender@NewClass}
%    \begin{macrocode}
\def\PdfRender@NewClass#1#2#3#4#5{%
  \PdfRender@newif{Active#1}%
  \expandafter\def\csname PdfRender@Default#1\endcsname{#2}%
  \expandafter\let\csname PdfRender@Current#1\expandafter\endcsname
      \csname PdfRender@Default#1\endcsname
  \ifPdfRender@Stack
    \expandafter\edef\csname PdfRender@Init#1\endcsname{%
      \global\chardef
      \expandafter\noexpand\csname PdfRender@Stack#1\endcsname=%
          \noexpand\pdfcolorstackinit page direct{%
            \noexpand#3%
            \expandafter\noexpand\csname PdfRender@Default#1\endcsname
          }\relax
      \noexpand\@PackageInfo{pdfrender}{%
        New color stack `#1' = \noexpand\number
        \expandafter\noexpand\csname PdfRender@Stack#1\endcsname
      }%
      \gdef\expandafter\noexpand\csname PdfRender@Init#1\endcsname{}%
    }%
    \expandafter\edef\csname PdfRender@Set#1\endcsname{%
      \expandafter\noexpand\csname PdfRender@Init#1\endcsname
      \noexpand\pdfcolorstack
      \expandafter\noexpand\csname PdfRender@Stack#1\endcsname
      push{%
        #3{\expandafter\noexpand\csname PdfRender@Current#1\endcsname}%
      }%
      \noexpand\aftergroup
      \expandafter\noexpand\csname PdfRender@Reset#1\endcsname
    }%
    \expandafter\edef\csname PdfRender@Reset#1\endcsname{%
      \expandafter\noexpand\csname PdfRender@Init#1\endcsname
      \noexpand\pdfcolorstack
      \expandafter\noexpand\csname PdfRender@Stack#1\endcsname
      pop\relax
    }%
  \else
    \expandafter\edef\csname PdfRender@Set#1\endcsname{%
      \noexpand\pdfliteral direct{%
        #3{\expandafter\noexpand\csname PdfRender@Current#1\endcsname}%
      }%
      \noexpand\aftergroup
      \expandafter\noexpand\csname PdfRender@Reset#1\endcsname
    }%
    \expandafter\edef\csname PdfRender@Reset#1\endcsname{%
      \noexpand\pdfliteral direct{%
        #3{\expandafter\noexpand\csname PdfRender@Current#1\endcsname}%
      }%
    }%
  \fi
  \expandafter\edef\csname PdfRender@Normal#1\endcsname{%
    \let
    \expandafter\noexpand\csname PdfRender@Current#1\endcsname
    \expandafter\noexpand\csname PdfRender@Default#1\endcsname
    \noexpand\PdfRender@Set{#1}%
  }%
  \expandafter\ltx@GlobalAppendToMacro\expandafter\PdfRender@NormalColorHook
  \expandafter{%
    \csname PdfRender@Normal#1\endcsname
  }%
  \ltx@GlobalAppendToMacro\PdfRender@ColorSetGroupHook{%
    \PdfRender@Set{#1}%
  }%
  \ifPdfRender@Values
    \kv@parse@normalized{#4}{%
      \expandafter\let\csname PdfRender@#1@\kv@key\endcsname\kv@key
      \ifx\kv@value\relax
      \else
        \expandafter\let\csname PdfRender@#1@\kv@value\endcsname\kv@key
      \fi
      \ltx@gobbletwo
    }%
    \PdfRender@define@key{PDFRENDER}{#1}{%
      \global\csname PdfRender@Active#1true\endcsname
      \def\PdfRender@Current{##1}%
      \PdfRender@SetValidateValues{#1}%
    }%
    \PdfRender@Valuesfalse
  \else
    \PdfRender@define@key{PDFRENDER}{#1}{%
      \global\csname PdfRender@Active#1true\endcsname
      \expandafter\def\csname PdfRender@Current#1\endcsname{##1}%
      \ltx@IfUndefined{PdfRender@PostProcess#1}{%
      }{%
        \csname PdfRender@PostProcess#1\endcsname
      }%
      \PdfRender@SetValidate{#1}{#4}{#5}%
    }%
  \fi
}%
%    \end{macrocode}
%    \end{macro}
%    \begin{macro}{\PdfRender@define@key}
%    \begin{macrocode}
\ltx@IfUndefined{define@key}{%
  \def\PdfRender@define@key#1#2{%
    \expandafter\def\csname KV@#1@#2\endcsname##1%
  }%
}{%
  \let\PdfRender@define@key\define@key
}
%    \end{macrocode}
%    \end{macro}
%    \begin{macro}{\PdfRender@Set}
%    \begin{macrocode}
\def\PdfRender@Set#1{%
  \csname ifPdfRender@Active#1\endcsname
    \csname PdfRender@Set#1\expandafter\endcsname
  \fi
}
%    \end{macrocode}
%    \end{macro}
%    \begin{macro}{\PdfRender@Reset}
%    \begin{macrocode}
\def\PdfRender@Reset#1{%
  \csname ifPdfRender@Active#1\endcsname
    \csname PdfRender@Reset#1\expandafter\endcsname
  \fi
}
%    \end{macrocode}
%    \end{macro}
%    \begin{macro}{\PdfRender@ErrorInvalidValue}
%    \begin{macrocode}
\def\PdfRender@ErrorInvalidValue#1{%
  \PackageError{pdfrender}{%
    Ignoring parameter setting for `#1'\MessageBreak
    because of invalid value %
    `\csname PdfRender@Current#1\endcsname'%
  }\@ehc
  \expandafter\let\csname PdfRender@Current#1\endcsname\ltx@empty
}%
%    \end{macrocode}
%    \end{macro}
%    \begin{macro}{\PdfRender@SetValidate}
%    \begin{macrocode}
\ifPdfRender@Match
  \def\PdfRender@SetValidate#1#2#3{%
    \ifnum\pdfmatch{^(#2)$}{\csname PdfRender@Current#1\endcsname}=1 %
      \csname PdfRender@Set#1\expandafter\endcsname
    \else
      \PdfRender@ErrorInvalidValue{#1}%
    \fi
  }%
\else
  \def\PdfRender@SetValidate#1#2#3{%
    \expandafter\let\expandafter\PdfRender@Current
    \csname PdfRender@Current#1\endcsname
    #3%
    \ifx\PdfRender@Current\@empty
      \PdfRender@ErrorInvalidValue{#1}%
    \else
      \csname PdfRender@Set#1\expandafter\endcsname
    \fi
  }%
\fi
%    \end{macrocode}
%    \end{macro}
%    \begin{macro}{\PdfRender@SetValidateValues}
%    \begin{macrocode}
\def\PdfRender@SetValidateValues#1{%
  \ltx@IfUndefined{PdfRender@#1@\PdfRender@Current}{%
    \expandafter\let\csname PdfRender@Current#1\endcsname
                    \PdfRender@Current
    \PdfRender@ErrorInvalidValue{#1}%
  }{%
    \expandafter\edef\csname PdfRender@Current#1\endcsname{%
      \csname PdfRender@#1@\PdfRender@Current\endcsname
    }%
    \csname PdfRender@Set#1\endcsname
  }%
}
%    \end{macrocode}
%    \end{macro}
%    \begin{macro}{\PdfRender@OpValue}
%    \begin{macrocode}
\def\PdfRender@OpValue#1#2{#2\ltx@space#1}%
%    \end{macrocode}
%    \end{macro}
%    \begin{macro}{\PdfRender@OpName}
%    \begin{macrocode}
\def\PdfRender@OpName#1#2{/#2\ltx@space#1}%
%    \end{macrocode}
%    \end{macro}
%
% \subsection{Declare and setup parameters}
%
%    \begin{macrocode}
\PdfRender@NewClassValues{TextRenderingMode}%
                      {0}%
                      {\PdfRender@OpValue{Tr}}{%
  0=Fill,%
  1=Stroke,%
  2=FillStroke,%
  3=Invisible,%
  4=FillClip,%
  5=StrokeClip,%
  6=FillStrokeClip,%
  7=Clip,%
}%
\PdfRender@NewClass{LineWidth}{1}{\PdfRender@OpValue{w}}{%
  [0-9]+\string\.?[0-9]*|\string\.[0-9]+%
}{}%
\ltx@IfUndefined{dimexpr}{%
  \def\PdfRender@dimexpr{}%
}{%
  \let\PdfRender@dimexpr\dimexpr
}
\def\PdfRender@PostProcessLineWidth{%
  \begingroup
  \afterassignment\PdfRender@@PostProcessLineWidth
  \dimen0=\PdfRender@dimexpr\PdfRender@CurrentLineWidth bp %
  \PdfRender@let\PdfRender@relax\PdfRender@relax
}
\let\PdfRender@let\let
\let\PdfRender@relax\relax
\def\PdfRender@@PostProcessLineWidth#1\PdfRender@let{%
  \ifx\\#1\\%
    \endgroup
  \else
    \dimen0=.996264\dimen0 % 72/72.27
    \edef\x{\endgroup
      \def\noexpand\PdfRender@CurrentLineWidth{%
        \strip@pt\dimen0%
      }%
    }%
    \expandafter\x
  \fi
}
\PdfRender@NewClassValues{LineCapStyle}{0}{\PdfRender@OpValue{J}}{%
  0=Butt,%
  1=Round,%
  2=ProjectingSquare,%
}%
\PdfRender@NewClassValues{LineJoinStyle}{0}{\PdfRender@OpValue{j}}{%
  0=Miter,%
  1=Round,%
  2=Bevel,%
}%
\PdfRender@NewClass{MiterLimit}{10}{\PdfRender@OpValue{M}}{%
  [0-9]*[1-9][0-9]*\string\.?[0-9]*|%
  [0-9]*\string\.?[0-9]*[1-9][0-9]*%
}{}%
\PdfRender@NewClass{Flatness}{0}{\PdfRender@OpValue{i}}{%
  100(\string\.0*)?|[0-9][0-9](\string\.[0-9]*)?|\string\.[0-9]+%
}{}%
\PdfRender@NewClass{LineDashPattern}{[]0}{\PdfRender@OpValue{d}}{%
  \string\[%
  ( ?([0-9]+\string\.?[0-9]*|\string\.[0-9]+) ?)*%
  \string\] ?%
  ([0-9]+\string\.?[0-9]*|\string\.[0-9]+)%
}{}%
\PdfRender@NewClassValues{RenderingIntent}%
                      {RelativeColorimetric}%
                      {\PdfRender@OpName{ri}}{%
  AbsoluteColorimetric,%
  RelativeColorimetric,%
  Saturation,%
  Perceptual,%
}%
%    \end{macrocode}
%
% \subsection{Fill and stroke color support}
%
%    \begin{macrocode}
\PdfRender@define@key{PDFRENDER}{FillColor}{%
  \begingroup
    \def\PdfRender@Color{#1}%
    \ifx\PdfRender@Color\ltx@empty
      \global\let\PdfRender@FillColor\ltx@empty
    \else
      \PdfRender@ColorAvailable{%
        \PdfRender@TestBox{%
          \expandafter\PdfRender@TryColor\PdfRender@Color\ltx@empty
          \PdfRender@GetFillColor
          \ifx\PdfRender@FillColor\ltx@empty
            \@PackageWarning{pdfrender}{%
              Cannot extract fill color\MessageBreak
              from value `#1'%
            }%
          \fi
        }%
      }%
    \fi
  \endgroup
}
\PdfRender@define@key{PDFRENDER}{StrokeColor}{%
  \begingroup
    \def\PdfRender@Color{#1}%
    \ifx\PdfRender@Color\ltx@empty
      \global\let\PdfRender@StrokeColor\ltx@empty
    \else
      \PdfRender@ColorAvailable{%
        \PdfRender@TestBox{%
          \expandafter\PdfRender@TryColor\PdfRender@Color\ltx@empty
          \PdfRender@GetStrokeColor
          \ifx\PdfRender@StrokeColor\ltx@empty
            \@PackageWarning{pdfrender}{%
              Cannot extract stroke color\MessageBreak
              from value `#1'%
            }%
          \fi
        }%
      }%
    \fi
  \endgroup
}
%    \end{macrocode}
%    \begin{macro}{\PdfRender@ColorAvailable}
%    \begin{macrocode}
\def\PdfRender@ColorAvailable{%
  \@ifundefined{set@color}{%
    \@PackageError{pdfrender}{%
      Ignoring color options, because neither\MessageBreak
      package `color' nor package `xcolor' is loaded%
    }\@ehc
    \global\let\PdfRender@ColorAvailable\ltx@gobble
  }{%
    \global\let\PdfRender@ColorAvailable\ltx@firstofone
  }%
  \PdfRender@ColorAvailable
}
%    \end{macrocode}
%    \end{macro}
%    \begin{macro}{\PdfRender@TryColor}
%    \begin{macrocode}
\def\PdfRender@TryColor{%
  \@ifnextchar[\color\PdfRender@@TryColor
}
%    \end{macrocode}
%    \end{macro}
%    \begin{macro}{\PdfRender@@TryColor}
%    \begin{macrocode}
\def\PdfRender@@TryColor#1\ltx@empty{%
  \expandafter\color\expandafter{\PdfRender@Color}%
}
%    \end{macrocode}
%    \end{macro}
%    \begin{macro}{\PdfRender@SetColor}
%    \begin{macrocode}
\def\PdfRender@SetColor{%
  \chardef\PdfRender@NeedsCurrentColor=0 %
  \ifx\PdfRender@FillColor\ltx@empty
    \ifx\PdfRender@StrokeColor\ltx@empty
    \else
      \edef\PdfRender@CurrentColor{%
        \noexpand\PdfRender@FillColor\ltx@space\PdfRender@StrokeColor
      }%
      \chardef\PdfRender@NeedsCurrentColor=1 %
    \fi
  \else
    \ifx\PdfRender@StrokeColor\ltx@empty
      \edef\PdfRender@CurrentColor{%
        \PdfRender@FillColor\ltx@space\noexpand\PdfRender@StrokeColor
      }%
      \chardef\PdfRender@NeedsCurrentColor=2 %
    \else
      \edef\current@color{%
        \PdfRender@FillColor\ltx@space\PdfRender@StrokeColor
      }%
      \set@color
    \fi
  \fi
  \ifnum\PdfRender@NeedsCurrentColor=1 %
    \PdfRender@GetFillColor
    \ifx\PdfRender@FillColor\ltx@empty
      \@PackageWarning{pdfrender}{%
        Cannot extract current fill color%
      }%
    \else
      \edef\current@color{\PdfRender@CurrentColor}%
      \set@color
    \fi
  \else
    \ifnum\PdfRender@NeedsCurrentColor=2 %
      \PdfRender@GetStrokeColor
      \ifx\PdfRender@StrokeColor\ltx@empty
        \@PackageWarning{pdfrender}{%
          Cannot extract current stroke color%
        }%
      \else
        \edef\current@color{\PdfRender@CurrentColor}%
        \set@color
      \fi
    \fi
  \fi
}
%    \end{macrocode}
%    \end{macro}
%    \begin{macro}{\PdfRender@PatternFillColor}
%    \begin{macrocode}
\edef\PdfRender@PatternFillColor{ % space
  (%
    [0-9\string\.]+ g|%
    [0-9\string\.]+ [0-9\string\.]+ [0-9\string\.]+ rg|%
    [0-9\string\.]+ [0-9\string\.]+ %
    [0-9\string\.]+ [0-9\string\.]+ k%
  ) % space
  (.*)$%
}
%    \end{macrocode}
%    \end{macro}
%    \begin{macro}{\PdfRender@PatternStrokeColor}
%    \begin{macrocode}
\edef\PdfRender@PatternStrokeColor{ % space
  (%
    [0-9\string\.]+ G|%
    [0-9\string\.]+ [0-9\string\.]+ [0-9\string\.]+ RG|%
    [0-9\string\.]+ [0-9\string\.]+ %
    [0-9\string\.]+ [0-9\string\.]+ K%
  ) % space
  (.*)$%
}
%    \end{macrocode}
%    \end{macro}
%    \begin{macro}{\PdfRender@MatchPattern}
%    \begin{macrocode}
\def\PdfRender@MatchPattern#1{%
  \ifnum\pdfmatch{\PdfRender@Pattern}{\PdfRender@String}=1 %
    \xdef#1{%
      \expandafter\strip@prefix\pdflastmatch 1%
    }%
    \edef\PdfRender@String{%
      \expandafter\strip@prefix\pdflastmatch 2%
    }%
    \ifx\PdfRender@String\ltx@empty
    \else
      \expandafter\expandafter\expandafter\PdfRender@MatchPattern
      \expandafter\expandafter\expandafter#1%
    \fi
  \fi
}
%    \end{macrocode}
%    \end{macro}
%    \begin{macro}{\PdfRender@GetFillColor}
%    \begin{macrocode}
\def\PdfRender@GetFillColor{%
  \global\let\PdfRender@FillColor\ltx@empty
  \begingroup
    \ifPdfRender@Match
      \let\PdfRender@Pattern\PdfRender@PatternFillColor
      \edef\PdfRender@String{\ltx@space\current@color\ltx@space}%
      \PdfRender@MatchPattern\PdfRender@FillColor
    \else
      \edef\current@color{\current@color\ltx@space}%
      \let\PdfRender@OP\relax
      \PdfRender@FindOp{g}0%
      \PdfRender@FindOp{G}1%
      \PdfRender@FindOp{rg}0%
      \PdfRender@FindOp{RG}1%
      \PdfRender@FindOp{k}0%
      \PdfRender@FindOp{K}1%
      \PdfRender@FilterOp 0\PdfRender@FillColor
    \fi
  \endgroup
}
%    \end{macrocode}
%    \end{macro}
%    \begin{macro}{\PdfRender@GetStrokeColor}
%    \begin{macrocode}
\def\PdfRender@GetStrokeColor{%
  \global\let\PdfRender@StrokeColor\ltx@empty
  \begingroup
    \ifPdfRender@Match
      \let\PdfRender@Pattern\PdfRender@PatternStrokeColor
      \edef\PdfRender@String{\ltx@space\current@color\ltx@space}%
      \PdfRender@MatchPattern\PdfRender@StrokeColor
    \else
      \edef\current@color{\current@color\ltx@space}%
      \let\PdfRender@OP\relax
      \PdfRender@FindOp{g}0%
      \PdfRender@FindOp{G}1%
      \PdfRender@FindOp{rg}0%
      \PdfRender@FindOp{RG}1%
      \PdfRender@FindOp{k}0%
      \PdfRender@FindOp{K}1%
      \PdfRender@FilterOp 1\PdfRender@StrokeColor
    \fi
  \endgroup
}
%    \end{macrocode}
%    \end{macro}
%    \begin{macrocode}
\ifPdfRender@Match
  \expandafter\PdfRender@AtEnd
\fi%
%    \end{macrocode}
%    \begin{macro}{\PdfRender@FindOp}
%    \begin{macrocode}
\def\PdfRender@FindOp#1#2{%
  \def\PdfRender@temp##1 #1 ##2\@nil{%
    ##1%
    \ifx\\##2\\%
      \expandafter\@gobble
    \else
      \PdfRender@OP{#1}#2%
      \expandafter\@firstofone
    \fi
    {%
      \PdfRender@temp##2\@nil
    }%
  }%
  \edef\current@color{%
    \@firstofone{\expandafter\PdfRender@temp\current@color} #1 \@nil
  }%
}
%    \end{macrocode}
%    \end{macro}
%    \begin{macro}{\PdfRender@FilterOp}
%    \begin{macrocode}
\def\PdfRender@FilterOp#1#2{%
  \expandafter\PdfRender@@FilterOp\expandafter#1\expandafter#2%
      \current@color\PdfRender@OP{}{}%
}
%    \end{macrocode}
%    \end{macro}
%    \begin{macro}{\PdfRender@@FilterOp}
%    \begin{macrocode}
\def\PdfRender@@FilterOp#1#2#3\PdfRender@OP#4#5{%
  \ifx\\#4#5\\%
  \else
    \ifnum#1=#5 %
      \xdef#2{#3 #4}%
    \fi
    \expandafter\PdfRender@@FilterOp\expandafter#1\expandafter#2%
  \fi
}
%    \end{macrocode}
%    \end{macro}
%
%    \begin{macrocode}
\PdfRender@AtEnd%
%</package>
%    \end{macrocode}
%% \section{Installation}
%
% \subsection{Download}
%
% \paragraph{Package.} This package is available on
% CTAN\footnote{\CTANpkg{pdfrender}}:
% \begin{description}
% \item[\CTAN{macros/latex/contrib/oberdiek/pdfrender.dtx}] The source file.
% \item[\CTAN{macros/latex/contrib/oberdiek/pdfrender.pdf}] Documentation.
% \end{description}
%
%
% \paragraph{Bundle.} All the packages of the bundle `oberdiek'
% are also available in a TDS compliant ZIP archive. There
% the packages are already unpacked and the documentation files
% are generated. The files and directories obey the TDS standard.
% \begin{description}
% \item[\CTANinstall{install/macros/latex/contrib/oberdiek.tds.zip}]
% \end{description}
% \emph{TDS} refers to the standard ``A Directory Structure
% for \TeX\ Files'' (\CTANpkg{tds}). Directories
% with \xfile{texmf} in their name are usually organized this way.
%
% \subsection{Bundle installation}
%
% \paragraph{Unpacking.} Unpack the \xfile{oberdiek.tds.zip} in the
% TDS tree (also known as \xfile{texmf} tree) of your choice.
% Example (linux):
% \begin{quote}
%   |unzip oberdiek.tds.zip -d ~/texmf|
% \end{quote}
%
% \subsection{Package installation}
%
% \paragraph{Unpacking.} The \xfile{.dtx} file is a self-extracting
% \docstrip\ archive. The files are extracted by running the
% \xfile{.dtx} through \plainTeX:
% \begin{quote}
%   \verb|tex pdfrender.dtx|
% \end{quote}
%
% \paragraph{TDS.} Now the different files must be moved into
% the different directories in your installation TDS tree
% (also known as \xfile{texmf} tree):
% \begin{quote}
% \def\t{^^A
% \begin{tabular}{@{}>{\ttfamily}l@{ $\rightarrow$ }>{\ttfamily}l@{}}
%   pdfrender.sty & tex/generic/oberdiek/pdfrender.sty\\
%   pdfrender.pdf & doc/latex/oberdiek/pdfrender.pdf\\
%   pdfrender.dtx & source/latex/oberdiek/pdfrender.dtx\\
% \end{tabular}^^A
% }^^A
% \sbox0{\t}^^A
% \ifdim\wd0>\linewidth
%   \begingroup
%     \advance\linewidth by\leftmargin
%     \advance\linewidth by\rightmargin
%   \edef\x{\endgroup
%     \def\noexpand\lw{\the\linewidth}^^A
%   }\x
%   \def\lwbox{^^A
%     \leavevmode
%     \hbox to \linewidth{^^A
%       \kern-\leftmargin\relax
%       \hss
%       \usebox0
%       \hss
%       \kern-\rightmargin\relax
%     }^^A
%   }^^A
%   \ifdim\wd0>\lw
%     \sbox0{\small\t}^^A
%     \ifdim\wd0>\linewidth
%       \ifdim\wd0>\lw
%         \sbox0{\footnotesize\t}^^A
%         \ifdim\wd0>\linewidth
%           \ifdim\wd0>\lw
%             \sbox0{\scriptsize\t}^^A
%             \ifdim\wd0>\linewidth
%               \ifdim\wd0>\lw
%                 \sbox0{\tiny\t}^^A
%                 \ifdim\wd0>\linewidth
%                   \lwbox
%                 \else
%                   \usebox0
%                 \fi
%               \else
%                 \lwbox
%               \fi
%             \else
%               \usebox0
%             \fi
%           \else
%             \lwbox
%           \fi
%         \else
%           \usebox0
%         \fi
%       \else
%         \lwbox
%       \fi
%     \else
%       \usebox0
%     \fi
%   \else
%     \lwbox
%   \fi
% \else
%   \usebox0
% \fi
% \end{quote}
% If you have a \xfile{docstrip.cfg} that configures and enables \docstrip's
% TDS installing feature, then some files can already be in the right
% place, see the documentation of \docstrip.
%
% \subsection{Refresh file name databases}
%
% If your \TeX~distribution
% (\TeX\,Live, \mikTeX, \dots) relies on file name databases, you must refresh
% these. For example, \TeX\,Live\ users run \verb|texhash| or
% \verb|mktexlsr|.
%
% \subsection{Some details for the interested}
%
% \paragraph{Unpacking with \LaTeX.}
% The \xfile{.dtx} chooses its action depending on the format:
% \begin{description}
% \item[\plainTeX:] Run \docstrip\ and extract the files.
% \item[\LaTeX:] Generate the documentation.
% \end{description}
% If you insist on using \LaTeX\ for \docstrip\ (really,
% \docstrip\ does not need \LaTeX), then inform the autodetect routine
% about your intention:
% \begin{quote}
%   \verb|latex \let\install=y% \iffalse meta-comment
%
% File: pdfrender.dtx
% Version: 2019/12/29 v1.6
% Info: Access to some PDF graphics parameters
%
% Copyright (C)
%    2010 Heiko Oberdiek
%    2016-2019 Oberdiek Package Support Group
%    https://github.com/ho-tex/oberdiek/issues
%
% This work may be distributed and/or modified under the
% conditions of the LaTeX Project Public License, either
% version 1.3c of this license or (at your option) any later
% version. This version of this license is in
%    https://www.latex-project.org/lppl/lppl-1-3c.txt
% and the latest version of this license is in
%    https://www.latex-project.org/lppl.txt
% and version 1.3 or later is part of all distributions of
% LaTeX version 2005/12/01 or later.
%
% This work has the LPPL maintenance status "maintained".
%
% The Current Maintainers of this work are
% Heiko Oberdiek and the Oberdiek Package Support Group
% https://github.com/ho-tex/oberdiek/issues
%
% The Base Interpreter refers to any `TeX-Format',
% because some files are installed in TDS:tex/generic//.
%
% This work consists of the main source file pdfrender.dtx
% and the derived files
%    pdfrender.sty, pdfrender.pdf, pdfrender.ins, pdfrender.drv,
%    pdfrender-test1.tex, pdfrender-test2.tex,
%    pdfrender-test3.tex, pdfrender-test4.tex,
%    pdfrender-test5.tex.
%
% Distribution:
%    CTAN:macros/latex/contrib/oberdiek/pdfrender.dtx
%    CTAN:macros/latex/contrib/oberdiek/pdfrender.pdf
%
% Unpacking:
%    (a) If pdfrender.ins is present:
%           tex pdfrender.ins
%    (b) Without pdfrender.ins:
%           tex pdfrender.dtx
%    (c) If you insist on using LaTeX
%           latex \let\install=y\input{pdfrender.dtx}
%        (quote the arguments according to the demands of your shell)
%
% Documentation:
%    (a) If pdfrender.drv is present:
%           latex pdfrender.drv
%    (b) Without pdfrender.drv:
%           latex pdfrender.dtx; ...
%    The class ltxdoc loads the configuration file ltxdoc.cfg
%    if available. Here you can specify further options, e.g.
%    use A4 as paper format:
%       \PassOptionsToClass{a4paper}{article}
%
%    Programm calls to get the documentation (example):
%       pdflatex pdfrender.dtx
%       makeindex -s gind.ist pdfrender.idx
%       pdflatex pdfrender.dtx
%       makeindex -s gind.ist pdfrender.idx
%       pdflatex pdfrender.dtx
%
% Installation:
%    TDS:tex/generic/oberdiek/pdfrender.sty
%    TDS:doc/latex/oberdiek/pdfrender.pdf
%    TDS:source/latex/oberdiek/pdfrender.dtx
%
%<*ignore>
\begingroup
  \catcode123=1 %
  \catcode125=2 %
  \def\x{LaTeX2e}%
\expandafter\endgroup
\ifcase 0\ifx\install y1\fi\expandafter
         \ifx\csname processbatchFile\endcsname\relax\else1\fi
         \ifx\fmtname\x\else 1\fi\relax
\else\csname fi\endcsname
%</ignore>
%<*install>
\input docstrip.tex
\Msg{************************************************************************}
\Msg{* Installation}
\Msg{* Package: pdfrender 2019/12/29 v1.6 Access to some PDF graphics parameters (HO)}
\Msg{************************************************************************}

\keepsilent
\askforoverwritefalse

\let\MetaPrefix\relax
\preamble

This is a generated file.

Project: pdfrender
Version: 2019/12/29 v1.6

Copyright (C)
   2010 Heiko Oberdiek
   2016-2019 Oberdiek Package Support Group

This work may be distributed and/or modified under the
conditions of the LaTeX Project Public License, either
version 1.3c of this license or (at your option) any later
version. This version of this license is in
   https://www.latex-project.org/lppl/lppl-1-3c.txt
and the latest version of this license is in
   https://www.latex-project.org/lppl.txt
and version 1.3 or later is part of all distributions of
LaTeX version 2005/12/01 or later.

This work has the LPPL maintenance status "maintained".

The Current Maintainers of this work are
Heiko Oberdiek and the Oberdiek Package Support Group
https://github.com/ho-tex/oberdiek/issues


The Base Interpreter refers to any `TeX-Format',
because some files are installed in TDS:tex/generic//.

This work consists of the main source file pdfrender.dtx
and the derived files
   pdfrender.sty, pdfrender.pdf, pdfrender.ins, pdfrender.drv,
   pdfrender-test1.tex, pdfrender-test2.tex,
   pdfrender-test3.tex, pdfrender-test4.tex,
   pdfrender-test5.tex.

\endpreamble
\let\MetaPrefix\DoubleperCent

\generate{%
  \file{pdfrender.ins}{\from{pdfrender.dtx}{install}}%
  \file{pdfrender.drv}{\from{pdfrender.dtx}{driver}}%
  \usedir{tex/generic/oberdiek}%
  \file{pdfrender.sty}{\from{pdfrender.dtx}{package}}%
%  \usedir{doc/latex/oberdiek/test}%
%  \file{pdfrender-test1.tex}{\from{pdfrender.dtx}{test1}}%
%  \file{pdfrender-test2.tex}{\from{pdfrender.dtx}{test2}}%
%  \file{pdfrender-test3.tex}{\from{pdfrender.dtx}{test3}}%
%  \file{pdfrender-test4.tex}{\from{pdfrender.dtx}{test4}}%
%  \file{pdfrender-test5.tex}{\from{pdfrender.dtx}{test5}}%
}

\catcode32=13\relax% active space
\let =\space%
\Msg{************************************************************************}
\Msg{*}
\Msg{* To finish the installation you have to move the following}
\Msg{* file into a directory searched by TeX:}
\Msg{*}
\Msg{*     pdfrender.sty}
\Msg{*}
\Msg{* To produce the documentation run the file `pdfrender.drv'}
\Msg{* through LaTeX.}
\Msg{*}
\Msg{* Happy TeXing!}
\Msg{*}
\Msg{************************************************************************}

\endbatchfile
%</install>
%<*ignore>
\fi
%</ignore>
%<*driver>
\NeedsTeXFormat{LaTeX2e}
\ProvidesFile{pdfrender.drv}%
  [2019/12/29 v1.6 Access to some PDF graphics parameters (HO)]%
\documentclass{ltxdoc}
\usepackage{holtxdoc}[2011/11/22]
\usepackage{longtable}
\renewcommand*{\arraystretch}{1.2}
\begin{document}
  \DocInput{pdfrender.dtx}%
\end{document}
%</driver>
% \fi
%
%
%
% \GetFileInfo{pdfrender.drv}
%
% \title{The \xpackage{pdfrender} package}
% \date{2019/12/29 v1.6}
% \author{Heiko Oberdiek\thanks
% {Please report any issues at \url{https://github.com/ho-tex/oberdiek/issues}}}
%
% \maketitle
%
% \begin{abstract}
% The PDF format has some graphics parameter like
% line width or text rendering mode. This package
% provides an interface for setting these parameters.
% \end{abstract}
%
% \tableofcontents
%
% \section{Documentation}
%
% This package \xpackage{pdfrender} defines an interface for
% PDF specific parameters that affects the rendering of
% graphics or text. The interface and its implementation
% uses the same technique as package \xpackage{color}
% for color settings. Therefore this package is loaded to
% enable \LaTeX's color interface.
%
% At different places \LaTeX\ uses \cs{normalcolor} to avoid that
% header, footer or floats are print in the current color of the
% main text. \cs{setgroup@color} is used to start a save box
% with the color that is set at box saving time.
% Package \xpackage{pdfrender} extends these macros to
% add its own hooks of its parameters. Therefore \LaTeX3 should
% generalize \LaTeXe's color interface.
%
% \subsection{Usage}
%
% In \LaTeX\ the package is loaded as normal package. Options
% are not defined for this package.
% \begin{quote}
% |\usepackage{pdfrender}|
% \end{quote}
%
% This package can also be used in \plainTeX\ and even ini\TeX:
% \begin{quote}
% |input pdfrender.sty|
% \end{quote}
%
% \subsection{Macros}
%
% \begin{declcs}{pdfrender} \M{key value list}
% \end{declcs}
% The first parameter \meta{key value list} contains a
% list of parameter settings. The key entry is the parameter name.
% The macro works like
% \cs{color} (without optional argument) for color setting.
%
% \begin{declcs}{textpdfrender} \M{key value list} \M{text}
% \end{declcs}
% In the same way as \cs{pdfrender} the first argument
% specifies the parameters that should be set. This parameter
% setting affects \meta{text} only. Basically it works
% the same way as \cs{textcolor} (without optional argument).
%
% \subsection{Parameters}
%
% The following table shows an overview for the supported
% parameters and values:
%
% \begin{longtable}{@{}l|l|l@{}}
% \bfseries Parameter & \bfseries Value & \bfseries Alias\\*
% \hline
% \endhead
% \xoption{TextRenderingMode}
%   & \texttt{0} & \texttt{Fill}\\*
%   & \texttt{1} & \texttt{Stroke}\\*
%   & \texttt{2} & \texttt{FillStroke}\\*
%   & \texttt{3} & \texttt{Invisible}\\*
%   & \texttt{4} & \texttt{FillClip}\\*
%   & \texttt{5} & \texttt{StrokeClip}\\*
%   & \texttt{6} & \texttt{FillStrokeClip}\\*
%   & \texttt{7} & \texttt{Clip}\\
% \hline
% \xoption{LineWidth}
%   & \itshape positive number, & \itshape \TeX\ dimen\\*[-.2ex]
%   & \qquad \itshape unit is bp & \\
% \hline
% \xoption{LineCapStyle}
%   & \texttt{0} & \texttt{Butt}\\*
%   & \texttt{1} & \texttt{Round}\\*
%   & \texttt{2} & \texttt{ProjectingSquare}\\
% \hline
% \xoption{LineJoinStyle}
%   & \texttt{0} & \texttt{Miter}\\*
%   & \texttt{1} & \texttt{Round}\\*
%   & \texttt{2} & \texttt{Bevel}\\
% \hline
% \xoption{MiterLimit}
%   & \multicolumn{2}{l@{}}{\itshape positive number} \\
% \hline
% \xoption{Flatness}
%   & \multicolumn{2}{l@{}}{\itshape number between 0 and 100} \\
% \hline
% \xoption{LineDashPattern}
%   & \multicolumn{2}{l@{}}{^^A
%       \itshape numbers in square brackets,^^A
%     } \\*[-.2ex]
%   & \multicolumn{2}{l@{}}{^^A
%       \qquad \itshape followed by number,^^A
%     } \\*
%   & \multicolumn{2}{l@{}}{^^A
%       \qquad \itshape units are bp^^A
%     } \\
% \hline
% \xoption{RenderingIntent}
%   & \multicolumn{2}{l@{}}{\texttt{AbsoluteColorimetric}} \\*
%   & \multicolumn{2}{l@{}}{\texttt{RelativeColorimetric}} \\*
%   & \multicolumn{2}{l@{}}{\texttt{Saturation}} \\*
%   & \multicolumn{2}{l@{}}{\texttt{Perceptual}} \\
% \hline
% \xoption{FillColor}
%   & & \itshape color specification \\
% \hline
% \xoption{StrokeColor}
%   & & \itshape color specification \\
% \end{longtable}
%
% \subsubsection{Details}
%
% The description and specification of these parameters
% are available in the PDF specification \cite{pdf-spec}.
% Therefore they are not repeated here.
%
% \begin{description}
% \item[Value:]
%   The values in the second column lists or describe
%   the values that are specified by the PDF specification.
% \item[Alias:] Instead of magic numbers the package also
%   defines some aliases that can be given as value.
%   Example: |LineCapStyle=Round| has the same effect as
%   |LineCapStyle=1|.
% \item[Number:] The term \emph{number} means an integer
%   or real number. The real number is given as plain
%   decimal number without exponent. The decimal separator is a period.
%   At least one digit must be present.
% \item[\xoption{LineWidth}:] As alias a \TeX\ dimen specification
%   can be given. This includes explicit specifications with number
%   and unit, e.g. |LineWidth=0.5pt|. Also \LaTeX\ length registers
%   may be used. If \eTeX's \cs{dimexpr} is available, then it is
%   automatically added. However package \xpackage{calc} is not
%   supported.
% \item[\xoption{FillColor}, \xoption{StrokeColor}:]
%   Package \xpackage{color} or \xpackage{xcolor} must be loaded
%   before these options can be used (since version 1.2).
%   \LaTeX's color support sets both colors at the same time
%   to the same value. However parameter \xoption{TextRenderingMode}
%   offers the value \texttt{FillStroke} that makes only sense, if
%   the two color types can be set separately.
%   If one of the options \xoption{FillColor} or \xoption{StrokeColor}
%   is specified, then also the color is set. For compatibility
%   with the \LaTeX\ color packages (\xpackage{color} or \xpackage{xcolor}),
%   always both colors must be set. Thus if one of them is not specified,
%   it is taken from the current color.
%
%   Both options \xoption{FillColor} and \xoption{StrokeColor}
%   expect a \LaTeX\ color specification as value.
%   Also the optional color model argument is supported.
%   Example:
%   \begin{quote}
%     |FillColor=yellow|,\\
%     |StrokeColor=[cmyk]{1,.5,0,0}|\\
%   \end{quote}
% \end{description}
%
% \subsection{Color stack}
%
% If the \pdfTeX\ version provides color stacks, then each
% parameter is assigned a page based color stack. The assignment
% of a stack takes place, when its parameter is set the first time.
% This avoids the use of color stacks that are not needed.
%
% \StopEventually{
% }
%
% \section{Implementation}
%    \begin{macrocode}
%<*package>
%    \end{macrocode}
%    Reload check, especially if the package is not used with \LaTeX.
%    \begin{macrocode}
\begingroup\catcode61\catcode48\catcode32=10\relax%
  \catcode13=5 % ^^M
  \endlinechar=13 %
  \catcode35=6 % #
  \catcode39=12 % '
  \catcode44=12 % ,
  \catcode45=12 % -
  \catcode46=12 % .
  \catcode58=12 % :
  \catcode64=11 % @
  \catcode123=1 % {
  \catcode125=2 % }
  \expandafter\let\expandafter\x\csname ver@pdfrender.sty\endcsname
  \ifx\x\relax % plain-TeX, first loading
  \else
    \def\empty{}%
    \ifx\x\empty % LaTeX, first loading,
      % variable is initialized, but \ProvidesPackage not yet seen
    \else
      \expandafter\ifx\csname PackageInfo\endcsname\relax
        \def\x#1#2{%
          \immediate\write-1{Package #1 Info: #2.}%
        }%
      \else
        \def\x#1#2{\PackageInfo{#1}{#2, stopped}}%
      \fi
      \x{pdfrender}{The package is already loaded}%
      \aftergroup\endinput
    \fi
  \fi
\endgroup%
%    \end{macrocode}
%    Package identification:
%    \begin{macrocode}
\begingroup\catcode61\catcode48\catcode32=10\relax%
  \catcode13=5 % ^^M
  \endlinechar=13 %
  \catcode35=6 % #
  \catcode39=12 % '
  \catcode40=12 % (
  \catcode41=12 % )
  \catcode44=12 % ,
  \catcode45=12 % -
  \catcode46=12 % .
  \catcode47=12 % /
  \catcode58=12 % :
  \catcode64=11 % @
  \catcode91=12 % [
  \catcode93=12 % ]
  \catcode123=1 % {
  \catcode125=2 % }
  \expandafter\ifx\csname ProvidesPackage\endcsname\relax
    \def\x#1#2#3[#4]{\endgroup
      \immediate\write-1{Package: #3 #4}%
      \xdef#1{#4}%
    }%
  \else
    \def\x#1#2[#3]{\endgroup
      #2[{#3}]%
      \ifx#1\@undefined
        \xdef#1{#3}%
      \fi
      \ifx#1\relax
        \xdef#1{#3}%
      \fi
    }%
  \fi
\expandafter\x\csname ver@pdfrender.sty\endcsname
\ProvidesPackage{pdfrender}%
  [2019/12/29 v1.6 Access to some PDF graphics parameters (HO)]%
%    \end{macrocode}
%
%    \begin{macrocode}
\begingroup\catcode61\catcode48\catcode32=10\relax%
  \catcode13=5 % ^^M
  \endlinechar=13 %
  \catcode123=1 % {
  \catcode125=2 % }
  \catcode64=11 % @
  \def\x{\endgroup
    \expandafter\edef\csname PdfRender@AtEnd\endcsname{%
      \endlinechar=\the\endlinechar\relax
      \catcode13=\the\catcode13\relax
      \catcode32=\the\catcode32\relax
      \catcode35=\the\catcode35\relax
      \catcode61=\the\catcode61\relax
      \catcode64=\the\catcode64\relax
      \catcode123=\the\catcode123\relax
      \catcode125=\the\catcode125\relax
    }%
  }%
\x\catcode61\catcode48\catcode32=10\relax%
\catcode13=5 % ^^M
\endlinechar=13 %
\catcode35=6 % #
\catcode64=11 % @
\catcode123=1 % {
\catcode125=2 % }
\def\TMP@EnsureCode#1#2{%
  \edef\PdfRender@AtEnd{%
    \PdfRender@AtEnd
    \catcode#1=\the\catcode#1\relax
  }%
  \catcode#1=#2\relax
}
\TMP@EnsureCode{10}{12}% ^^J
\TMP@EnsureCode{36}{3}% $
\TMP@EnsureCode{39}{12}% '
\TMP@EnsureCode{40}{12}% (
\TMP@EnsureCode{41}{12}% )
\TMP@EnsureCode{42}{12}% *
\TMP@EnsureCode{43}{12}% +
\TMP@EnsureCode{44}{12}% ,
\TMP@EnsureCode{45}{12}% -
\TMP@EnsureCode{46}{12}% .
\TMP@EnsureCode{47}{12}% /
\TMP@EnsureCode{58}{12}% :
\TMP@EnsureCode{59}{12}% ;
\TMP@EnsureCode{60}{12}% <
\TMP@EnsureCode{62}{12}% >
\TMP@EnsureCode{63}{12}% ?
\TMP@EnsureCode{91}{12}% [
\TMP@EnsureCode{93}{12}% ]
\TMP@EnsureCode{94}{7}% ^ (superscript)
\TMP@EnsureCode{96}{12}% `
\TMP@EnsureCode{124}{12}% |
%    \end{macrocode}
%
%    \begin{macrocode}
\def\PdfRender@AtEndHook{}
\expandafter\def\expandafter\PdfRender@AtEnd\expandafter{%
  \expandafter\PdfRender@AtEndHook
  \PdfRender@AtEnd
  \endinput
}
%    \end{macrocode}
%
% \subsection{Look for \pdfTeX, its mode and features}
%
%    \begin{macro}{\PdfRender@newif}
%    \begin{macrocode}
\def\PdfRender@newif#1{%
  \expandafter\edef\csname PdfRender@#1true\endcsname{%
    \let
    \expandafter\noexpand\csname ifPdfRender@#1\endcsname
    \noexpand\iftrue
  }%
  \expandafter\edef\csname PdfRender@#1false\endcsname{%
    \let
    \expandafter\noexpand\csname ifPdfRender@#1\endcsname
    \noexpand\iffalse
  }%
  \csname PdfRender@#1false\endcsname
}
%    \end{macrocode}
%    \end{macro}
%
%    \begin{macro}{\ifPdfRender@Stack}
%    \begin{macrocode}
\PdfRender@newif{Stack}
%    \end{macrocode}
%    \end{macro}
%    \begin{macro}{\ifPdfRender@Match}
%    \begin{macrocode}
\PdfRender@newif{Match}
%    \end{macrocode}
%    \end{macro}
%
%    \begin{macro}{\PdfRender@RequirePackage}
%    \begin{macrocode}
\begingroup\expandafter\expandafter\expandafter\endgroup
\expandafter\ifx\csname RequirePackage\endcsname\relax
  \def\PdfRender@RequirePackage#1[#2]{%
    \expandafter\def\expandafter\PdfRender@AtEndHook\expandafter{%
      \PdfRender@AtEndHook
      \ltx@ifpackagelater{#1}{#2}{}{%
        \@PackageWarningNoLine{pdfrender}{%
          You have requested version\MessageBreak
          `#2' of package `#1',\MessageBreak
          but only version\MessageBreak
          `\csname ver@#1.\ltx@pkgextension\endcsname'\MessageBreak
          is available%
        }%
      }%
    }%
    \input #1.sty\relax
  }%
\else
  \let\PdfRender@RequirePackage\RequirePackage
\fi
%    \end{macrocode}
%    \end{macro}
% Luatex compatibility
%    \begin{macrocode}
\ifx\pdfextension\@undefined\else
  \def\pdfcolorstackinit {\pdffeedback colorstackinit}
  \protected\def\pdfcolorstack     {\pdfextension colorstack}
  \protected\def\pdfliteral        {\pdfextension literal}
\fi
%    \end{macrocode}
%
%    \begin{macrocode}
\PdfRender@RequirePackage{iftex}[2019/11/07]
\PdfRender@RequirePackage{infwarerr}[2007/09/09]
\PdfRender@RequirePackage{ltxcmds}[2010/01/28]
%    \end{macrocode}
%
%    \begin{macrocode}
\ifpdf
  \ltx@IfUndefined{pdfcolorstackinit}{%
    \@PackageWarning{pdfrender}{%
      Missing \string\pdfcolorstackinit
    }%
  }{%
    \PdfRender@Stacktrue
  }%
  \ltx@IfUndefined{pdfmatch}{%
    \@PackageInfoNoLine{pdfrender}{%
      \string\pdfmatch\ltx@space not found. %
      Therefore the values\MessageBreak
      of some parameters are not validated%
    }%
  }{%
    \PdfRender@Matchtrue
  }%
\else
  \@PackageWarning{pdfrender}{%
    Missing pdfTeX in PDF mode%
  }%
%    \end{macrocode}
%    \begin{macrocode}
  \ltx@IfUndefined{newcommand}{%
%    \end{macrocode}
%    \begin{macro}{\pdfrender}
%    \begin{macrocode}
    \def\pdfrender#1{}%
%    \end{macrocode}
%    \end{macro}
%    \begin{macro}{\textpdfrender}
%    \begin{macrocode}
    \long\def\textpdfrender#1#2{#2}%
%    \end{macrocode}
%    \end{macro}
%    \begin{macrocode}
  }{%
%    \end{macrocode}
%    \begin{macro}{\pdfrender}
%    \begin{macrocode}
    \newcommand*{\pdfrender}[1]{}%
%    \end{macrocode}
%    \end{macro}
%    \begin{macro}{\textpdfrender}
%    \begin{macrocode}
    \newcommand{\textpdfrender}[2]{#2}%
%    \end{macrocode}
%    \end{macro}
%    \begin{macrocode}
  }%
  \expandafter\PdfRender@AtEnd
\fi%
%    \end{macrocode}
%
% \subsection{Enable color support of \LaTeX}
%
%    \begin{macrocode}
\ltx@ifpackageloaded{color}{}{%
  \def\color@setgroup{\begingroup\set@color}%
  \let\color@begingroup\begingroup
  \def\color@endgroup{\endgraf\endgroup}%
  \def\color@hbox{\hbox\bgroup\color@begingroup}%
  \def\color@vbox{\vbox\bgroup\color@begingroup}%
  \def\color@endbox{\color@endgroup\egroup}%
  \ltx@ifundefined{bgroup}{%
    \let\bgroup={\let\egroup=}%
  }{}%
  \ltx@ifundefined{endgraf}{%
    \let\endgraf=\par
  }{}%
}
%    \end{macrocode}
%
% \subsection{Hook into \cs{normalcolor}}
%
%    The problem is that packages \xpackage{color} and \xpackage{xcolor}
%    each overwrite \cs{normalcolor}. For example, after the package
%    loading order \xpackage{color}, \xpackage{pdfrender} and \xpackage{xcolor}
%    the patched version of \cs{normalcolor} is overwritten by
%    package \xpackage{xcolor}. Also using \cs{AtBeginDocument} for
%    patching is not enough. If package \xpackage{hyperref} is loaded
%    later, it might load package \xpackage{color} using \cs{AtBeginDocument}.
%    \begin{macro}{\PdfRender@NormalColorHook}
%    \begin{macrocode}
\def\PdfRender@NormalColorHook{}
%    \end{macrocode}
%    \end{macro}
%    \begin{macro}{\PdfRender@ColorSetGroupHook}
%    \begin{macrocode}
\def\PdfRender@ColorSetGroupHook{}
%    \end{macrocode}
%    \end{macro}
%    \begin{macro}{\PdfRender@TestBox}
%    \begin{macrocode}
\def\PdfRender@TestBox#1{%
  \setbox0=\color@hbox#1\color@endbox
}
%    \end{macrocode}
%    \end{macro}
%    \begin{macro}{\PdfRender@PatchNormalColor}
%    \begin{macrocode}
\def\PdfRender@PatchNormalColor{%
  \ltx@ifundefined{normalcolor}{%
    \gdef\normalcolor{\PdfRender@NormalColorHook}%
  }{%
    \begingroup
      \def\PdfRender@NormalColorHook{\let\PdfRender@temp=Y}%
      \PdfRender@TestBox{%
        \let\set@color\relax
        \normalcolor
        \ifx\PdfRender@temp Y%
        \else
          \ltx@GlobalAppendToMacro\normalcolor{%
            \PdfRender@NormalColorHook
          }%
        \fi
      }%
    \endgroup
  }%
  \ifx\@nodocument\relax
    \global\let\PdfRender@PatchNormalColor\relax
  \fi
}%
%    \end{macrocode}
%    \end{macro}
%    \begin{macro}{\PdfRender@PatchColorSetGroup}
%    \begin{macrocode}
\def\PdfRender@PatchColorSetGroup{%
  \begingroup
    \def\PdfRender@ColorSetGroupHook{\let\PdfRender@temp=Y}%
    \PdfRender@TestBox{%
      \let\set@color\relax
      \color@setgroup\color@endgroup
      \ifx\PdfRender@temp Y%
      \else
        \ltx@GlobalAppendToMacro\color@setgroup{%
          \PdfRender@ColorSetGroupHook
        }%
      \fi
    }%
  \endgroup
  \ifx\@nodocument\relax
    \global\let\PdfRender@PatchColorSetGroup\relax
  \fi
}%
%    \end{macrocode}
%    \end{macro}
%    \begin{macro}{\PdfRender@PatchColor}
%    \begin{macrocode}
\def\PdfRender@PatchColor{%
  \PdfRender@PatchNormalColor
  \PdfRender@PatchColorSetGroup
}
%    \end{macrocode}
%    \end{macro}
%    \begin{macrocode}
\PdfRender@PatchColor
\ltx@IfUndefined{AtBeginDocument}{}{%
  \AtBeginDocument{\PdfRender@PatchColor}%
}
%    \end{macrocode}
%    \cs{AfterPackage} is provided by package \xpackage{scrlfile}.
%    \begin{macrocode}
\ltx@IfUndefined{AfterPackage}{%
}{%
  \AfterPackage{color}{\PdfRender@PatchColor}%
  \AfterPackage{xcolor}{\PdfRender@PatchColor}%
  \AfterPackage{etoolbox}{%
    \AfterEndPreamble{\PdfRender@PatchColor}%
  }%
}%
%    \end{macrocode}
%    \cs{AfterEndPreamble} is provided by package \xpackage{etoolbox}.
%    \begin{macrocode}
\ltx@IfUndefined{AfterEndPreamble}{%
}{%
  \AfterEndPreamble{\PdfRender@PatchColor}%
}%
%    \end{macrocode}
%
%    \begin{macrocode}
\PdfRender@RequirePackage{kvsetkeys}[2010/01/28]
%    \end{macrocode}
%    \begin{macro}{\PdfRender@texorpdfstring}
%    \begin{macrocode}
\def\PdfRender@texorpdfstring{%
  \ltx@IfUndefined{texorpdfstring}\ltx@firstoftwo\texorpdfstring
}
%    \end{macrocode}
%    \end{macro}
%    \begin{macro}{\pdfrender}
%    \begin{macrocode}
\ltx@IfUndefined{DeclareRobustCommand}%
\ltx@firstoftwo\ltx@secondoftwo
{%
  \def\pdfrender#1%
}{%
  \newcommand{\pdfrender}{}%
  \DeclareRobustCommand*{\pdfrender}[1]%
}%
{%
  \PdfRender@texorpdfstring{%
    \PdfRender@PatchNormalColor
    \global\let\PdfRender@FillColor\ltx@empty
    \global\let\PdfRender@StrokeColor\ltx@empty
    \kvsetkeys{PDFRENDER}{#1}%
    \PdfRender@SetColor
  }{}%
}
%    \end{macrocode}
%    \end{macro}
%    \begin{macro}{\textpdfrender}
%    \begin{macrocode}
\ltx@IfUndefined{DeclareRobustCommand}%
\ltx@firstoftwo\ltx@secondoftwo
{%
  \long\def\textpdfrender#1#2%
}{%
  \newcommand{\textpdfrender}{}%
  \DeclareRobustCommand{\textpdfrender}[2]%
}%
{%
  \PdfRender@texorpdfstring{%
    \begingroup
      \pdfrender{#1}%
      #2%
    \endgroup
  }{#2}%
}
%    \end{macrocode}
%    \end{macro}
%
%    \begin{macro}{\ifPdfRender@Values}
%    \begin{macrocode}
\PdfRender@newif{Values}
%    \end{macrocode}
%    \end{macro}
%    \begin{macro}{\PdfRender@NewClassValues}
%    \begin{macrocode}
\def\PdfRender@NewClassValues#1#2#3#4{%
  \PdfRender@Valuestrue
  \PdfRender@NewClass{#1}{#2}{#3}{#4}{}%
}
%    \end{macrocode}
%    \end{macro}
%    \begin{macro}{\PdfRender@NewClass}
%    \begin{macrocode}
\def\PdfRender@NewClass#1#2#3#4#5{%
  \PdfRender@newif{Active#1}%
  \expandafter\def\csname PdfRender@Default#1\endcsname{#2}%
  \expandafter\let\csname PdfRender@Current#1\expandafter\endcsname
      \csname PdfRender@Default#1\endcsname
  \ifPdfRender@Stack
    \expandafter\edef\csname PdfRender@Init#1\endcsname{%
      \global\chardef
      \expandafter\noexpand\csname PdfRender@Stack#1\endcsname=%
          \noexpand\pdfcolorstackinit page direct{%
            \noexpand#3%
            \expandafter\noexpand\csname PdfRender@Default#1\endcsname
          }\relax
      \noexpand\@PackageInfo{pdfrender}{%
        New color stack `#1' = \noexpand\number
        \expandafter\noexpand\csname PdfRender@Stack#1\endcsname
      }%
      \gdef\expandafter\noexpand\csname PdfRender@Init#1\endcsname{}%
    }%
    \expandafter\edef\csname PdfRender@Set#1\endcsname{%
      \expandafter\noexpand\csname PdfRender@Init#1\endcsname
      \noexpand\pdfcolorstack
      \expandafter\noexpand\csname PdfRender@Stack#1\endcsname
      push{%
        #3{\expandafter\noexpand\csname PdfRender@Current#1\endcsname}%
      }%
      \noexpand\aftergroup
      \expandafter\noexpand\csname PdfRender@Reset#1\endcsname
    }%
    \expandafter\edef\csname PdfRender@Reset#1\endcsname{%
      \expandafter\noexpand\csname PdfRender@Init#1\endcsname
      \noexpand\pdfcolorstack
      \expandafter\noexpand\csname PdfRender@Stack#1\endcsname
      pop\relax
    }%
  \else
    \expandafter\edef\csname PdfRender@Set#1\endcsname{%
      \noexpand\pdfliteral direct{%
        #3{\expandafter\noexpand\csname PdfRender@Current#1\endcsname}%
      }%
      \noexpand\aftergroup
      \expandafter\noexpand\csname PdfRender@Reset#1\endcsname
    }%
    \expandafter\edef\csname PdfRender@Reset#1\endcsname{%
      \noexpand\pdfliteral direct{%
        #3{\expandafter\noexpand\csname PdfRender@Current#1\endcsname}%
      }%
    }%
  \fi
  \expandafter\edef\csname PdfRender@Normal#1\endcsname{%
    \let
    \expandafter\noexpand\csname PdfRender@Current#1\endcsname
    \expandafter\noexpand\csname PdfRender@Default#1\endcsname
    \noexpand\PdfRender@Set{#1}%
  }%
  \expandafter\ltx@GlobalAppendToMacro\expandafter\PdfRender@NormalColorHook
  \expandafter{%
    \csname PdfRender@Normal#1\endcsname
  }%
  \ltx@GlobalAppendToMacro\PdfRender@ColorSetGroupHook{%
    \PdfRender@Set{#1}%
  }%
  \ifPdfRender@Values
    \kv@parse@normalized{#4}{%
      \expandafter\let\csname PdfRender@#1@\kv@key\endcsname\kv@key
      \ifx\kv@value\relax
      \else
        \expandafter\let\csname PdfRender@#1@\kv@value\endcsname\kv@key
      \fi
      \ltx@gobbletwo
    }%
    \PdfRender@define@key{PDFRENDER}{#1}{%
      \global\csname PdfRender@Active#1true\endcsname
      \def\PdfRender@Current{##1}%
      \PdfRender@SetValidateValues{#1}%
    }%
    \PdfRender@Valuesfalse
  \else
    \PdfRender@define@key{PDFRENDER}{#1}{%
      \global\csname PdfRender@Active#1true\endcsname
      \expandafter\def\csname PdfRender@Current#1\endcsname{##1}%
      \ltx@IfUndefined{PdfRender@PostProcess#1}{%
      }{%
        \csname PdfRender@PostProcess#1\endcsname
      }%
      \PdfRender@SetValidate{#1}{#4}{#5}%
    }%
  \fi
}%
%    \end{macrocode}
%    \end{macro}
%    \begin{macro}{\PdfRender@define@key}
%    \begin{macrocode}
\ltx@IfUndefined{define@key}{%
  \def\PdfRender@define@key#1#2{%
    \expandafter\def\csname KV@#1@#2\endcsname##1%
  }%
}{%
  \let\PdfRender@define@key\define@key
}
%    \end{macrocode}
%    \end{macro}
%    \begin{macro}{\PdfRender@Set}
%    \begin{macrocode}
\def\PdfRender@Set#1{%
  \csname ifPdfRender@Active#1\endcsname
    \csname PdfRender@Set#1\expandafter\endcsname
  \fi
}
%    \end{macrocode}
%    \end{macro}
%    \begin{macro}{\PdfRender@Reset}
%    \begin{macrocode}
\def\PdfRender@Reset#1{%
  \csname ifPdfRender@Active#1\endcsname
    \csname PdfRender@Reset#1\expandafter\endcsname
  \fi
}
%    \end{macrocode}
%    \end{macro}
%    \begin{macro}{\PdfRender@ErrorInvalidValue}
%    \begin{macrocode}
\def\PdfRender@ErrorInvalidValue#1{%
  \PackageError{pdfrender}{%
    Ignoring parameter setting for `#1'\MessageBreak
    because of invalid value %
    `\csname PdfRender@Current#1\endcsname'%
  }\@ehc
  \expandafter\let\csname PdfRender@Current#1\endcsname\ltx@empty
}%
%    \end{macrocode}
%    \end{macro}
%    \begin{macro}{\PdfRender@SetValidate}
%    \begin{macrocode}
\ifPdfRender@Match
  \def\PdfRender@SetValidate#1#2#3{%
    \ifnum\pdfmatch{^(#2)$}{\csname PdfRender@Current#1\endcsname}=1 %
      \csname PdfRender@Set#1\expandafter\endcsname
    \else
      \PdfRender@ErrorInvalidValue{#1}%
    \fi
  }%
\else
  \def\PdfRender@SetValidate#1#2#3{%
    \expandafter\let\expandafter\PdfRender@Current
    \csname PdfRender@Current#1\endcsname
    #3%
    \ifx\PdfRender@Current\@empty
      \PdfRender@ErrorInvalidValue{#1}%
    \else
      \csname PdfRender@Set#1\expandafter\endcsname
    \fi
  }%
\fi
%    \end{macrocode}
%    \end{macro}
%    \begin{macro}{\PdfRender@SetValidateValues}
%    \begin{macrocode}
\def\PdfRender@SetValidateValues#1{%
  \ltx@IfUndefined{PdfRender@#1@\PdfRender@Current}{%
    \expandafter\let\csname PdfRender@Current#1\endcsname
                    \PdfRender@Current
    \PdfRender@ErrorInvalidValue{#1}%
  }{%
    \expandafter\edef\csname PdfRender@Current#1\endcsname{%
      \csname PdfRender@#1@\PdfRender@Current\endcsname
    }%
    \csname PdfRender@Set#1\endcsname
  }%
}
%    \end{macrocode}
%    \end{macro}
%    \begin{macro}{\PdfRender@OpValue}
%    \begin{macrocode}
\def\PdfRender@OpValue#1#2{#2\ltx@space#1}%
%    \end{macrocode}
%    \end{macro}
%    \begin{macro}{\PdfRender@OpName}
%    \begin{macrocode}
\def\PdfRender@OpName#1#2{/#2\ltx@space#1}%
%    \end{macrocode}
%    \end{macro}
%
% \subsection{Declare and setup parameters}
%
%    \begin{macrocode}
\PdfRender@NewClassValues{TextRenderingMode}%
                      {0}%
                      {\PdfRender@OpValue{Tr}}{%
  0=Fill,%
  1=Stroke,%
  2=FillStroke,%
  3=Invisible,%
  4=FillClip,%
  5=StrokeClip,%
  6=FillStrokeClip,%
  7=Clip,%
}%
\PdfRender@NewClass{LineWidth}{1}{\PdfRender@OpValue{w}}{%
  [0-9]+\string\.?[0-9]*|\string\.[0-9]+%
}{}%
\ltx@IfUndefined{dimexpr}{%
  \def\PdfRender@dimexpr{}%
}{%
  \let\PdfRender@dimexpr\dimexpr
}
\def\PdfRender@PostProcessLineWidth{%
  \begingroup
  \afterassignment\PdfRender@@PostProcessLineWidth
  \dimen0=\PdfRender@dimexpr\PdfRender@CurrentLineWidth bp %
  \PdfRender@let\PdfRender@relax\PdfRender@relax
}
\let\PdfRender@let\let
\let\PdfRender@relax\relax
\def\PdfRender@@PostProcessLineWidth#1\PdfRender@let{%
  \ifx\\#1\\%
    \endgroup
  \else
    \dimen0=.996264\dimen0 % 72/72.27
    \edef\x{\endgroup
      \def\noexpand\PdfRender@CurrentLineWidth{%
        \strip@pt\dimen0%
      }%
    }%
    \expandafter\x
  \fi
}
\PdfRender@NewClassValues{LineCapStyle}{0}{\PdfRender@OpValue{J}}{%
  0=Butt,%
  1=Round,%
  2=ProjectingSquare,%
}%
\PdfRender@NewClassValues{LineJoinStyle}{0}{\PdfRender@OpValue{j}}{%
  0=Miter,%
  1=Round,%
  2=Bevel,%
}%
\PdfRender@NewClass{MiterLimit}{10}{\PdfRender@OpValue{M}}{%
  [0-9]*[1-9][0-9]*\string\.?[0-9]*|%
  [0-9]*\string\.?[0-9]*[1-9][0-9]*%
}{}%
\PdfRender@NewClass{Flatness}{0}{\PdfRender@OpValue{i}}{%
  100(\string\.0*)?|[0-9][0-9](\string\.[0-9]*)?|\string\.[0-9]+%
}{}%
\PdfRender@NewClass{LineDashPattern}{[]0}{\PdfRender@OpValue{d}}{%
  \string\[%
  ( ?([0-9]+\string\.?[0-9]*|\string\.[0-9]+) ?)*%
  \string\] ?%
  ([0-9]+\string\.?[0-9]*|\string\.[0-9]+)%
}{}%
\PdfRender@NewClassValues{RenderingIntent}%
                      {RelativeColorimetric}%
                      {\PdfRender@OpName{ri}}{%
  AbsoluteColorimetric,%
  RelativeColorimetric,%
  Saturation,%
  Perceptual,%
}%
%    \end{macrocode}
%
% \subsection{Fill and stroke color support}
%
%    \begin{macrocode}
\PdfRender@define@key{PDFRENDER}{FillColor}{%
  \begingroup
    \def\PdfRender@Color{#1}%
    \ifx\PdfRender@Color\ltx@empty
      \global\let\PdfRender@FillColor\ltx@empty
    \else
      \PdfRender@ColorAvailable{%
        \PdfRender@TestBox{%
          \expandafter\PdfRender@TryColor\PdfRender@Color\ltx@empty
          \PdfRender@GetFillColor
          \ifx\PdfRender@FillColor\ltx@empty
            \@PackageWarning{pdfrender}{%
              Cannot extract fill color\MessageBreak
              from value `#1'%
            }%
          \fi
        }%
      }%
    \fi
  \endgroup
}
\PdfRender@define@key{PDFRENDER}{StrokeColor}{%
  \begingroup
    \def\PdfRender@Color{#1}%
    \ifx\PdfRender@Color\ltx@empty
      \global\let\PdfRender@StrokeColor\ltx@empty
    \else
      \PdfRender@ColorAvailable{%
        \PdfRender@TestBox{%
          \expandafter\PdfRender@TryColor\PdfRender@Color\ltx@empty
          \PdfRender@GetStrokeColor
          \ifx\PdfRender@StrokeColor\ltx@empty
            \@PackageWarning{pdfrender}{%
              Cannot extract stroke color\MessageBreak
              from value `#1'%
            }%
          \fi
        }%
      }%
    \fi
  \endgroup
}
%    \end{macrocode}
%    \begin{macro}{\PdfRender@ColorAvailable}
%    \begin{macrocode}
\def\PdfRender@ColorAvailable{%
  \@ifundefined{set@color}{%
    \@PackageError{pdfrender}{%
      Ignoring color options, because neither\MessageBreak
      package `color' nor package `xcolor' is loaded%
    }\@ehc
    \global\let\PdfRender@ColorAvailable\ltx@gobble
  }{%
    \global\let\PdfRender@ColorAvailable\ltx@firstofone
  }%
  \PdfRender@ColorAvailable
}
%    \end{macrocode}
%    \end{macro}
%    \begin{macro}{\PdfRender@TryColor}
%    \begin{macrocode}
\def\PdfRender@TryColor{%
  \@ifnextchar[\color\PdfRender@@TryColor
}
%    \end{macrocode}
%    \end{macro}
%    \begin{macro}{\PdfRender@@TryColor}
%    \begin{macrocode}
\def\PdfRender@@TryColor#1\ltx@empty{%
  \expandafter\color\expandafter{\PdfRender@Color}%
}
%    \end{macrocode}
%    \end{macro}
%    \begin{macro}{\PdfRender@SetColor}
%    \begin{macrocode}
\def\PdfRender@SetColor{%
  \chardef\PdfRender@NeedsCurrentColor=0 %
  \ifx\PdfRender@FillColor\ltx@empty
    \ifx\PdfRender@StrokeColor\ltx@empty
    \else
      \edef\PdfRender@CurrentColor{%
        \noexpand\PdfRender@FillColor\ltx@space\PdfRender@StrokeColor
      }%
      \chardef\PdfRender@NeedsCurrentColor=1 %
    \fi
  \else
    \ifx\PdfRender@StrokeColor\ltx@empty
      \edef\PdfRender@CurrentColor{%
        \PdfRender@FillColor\ltx@space\noexpand\PdfRender@StrokeColor
      }%
      \chardef\PdfRender@NeedsCurrentColor=2 %
    \else
      \edef\current@color{%
        \PdfRender@FillColor\ltx@space\PdfRender@StrokeColor
      }%
      \set@color
    \fi
  \fi
  \ifnum\PdfRender@NeedsCurrentColor=1 %
    \PdfRender@GetFillColor
    \ifx\PdfRender@FillColor\ltx@empty
      \@PackageWarning{pdfrender}{%
        Cannot extract current fill color%
      }%
    \else
      \edef\current@color{\PdfRender@CurrentColor}%
      \set@color
    \fi
  \else
    \ifnum\PdfRender@NeedsCurrentColor=2 %
      \PdfRender@GetStrokeColor
      \ifx\PdfRender@StrokeColor\ltx@empty
        \@PackageWarning{pdfrender}{%
          Cannot extract current stroke color%
        }%
      \else
        \edef\current@color{\PdfRender@CurrentColor}%
        \set@color
      \fi
    \fi
  \fi
}
%    \end{macrocode}
%    \end{macro}
%    \begin{macro}{\PdfRender@PatternFillColor}
%    \begin{macrocode}
\edef\PdfRender@PatternFillColor{ % space
  (%
    [0-9\string\.]+ g|%
    [0-9\string\.]+ [0-9\string\.]+ [0-9\string\.]+ rg|%
    [0-9\string\.]+ [0-9\string\.]+ %
    [0-9\string\.]+ [0-9\string\.]+ k%
  ) % space
  (.*)$%
}
%    \end{macrocode}
%    \end{macro}
%    \begin{macro}{\PdfRender@PatternStrokeColor}
%    \begin{macrocode}
\edef\PdfRender@PatternStrokeColor{ % space
  (%
    [0-9\string\.]+ G|%
    [0-9\string\.]+ [0-9\string\.]+ [0-9\string\.]+ RG|%
    [0-9\string\.]+ [0-9\string\.]+ %
    [0-9\string\.]+ [0-9\string\.]+ K%
  ) % space
  (.*)$%
}
%    \end{macrocode}
%    \end{macro}
%    \begin{macro}{\PdfRender@MatchPattern}
%    \begin{macrocode}
\def\PdfRender@MatchPattern#1{%
  \ifnum\pdfmatch{\PdfRender@Pattern}{\PdfRender@String}=1 %
    \xdef#1{%
      \expandafter\strip@prefix\pdflastmatch 1%
    }%
    \edef\PdfRender@String{%
      \expandafter\strip@prefix\pdflastmatch 2%
    }%
    \ifx\PdfRender@String\ltx@empty
    \else
      \expandafter\expandafter\expandafter\PdfRender@MatchPattern
      \expandafter\expandafter\expandafter#1%
    \fi
  \fi
}
%    \end{macrocode}
%    \end{macro}
%    \begin{macro}{\PdfRender@GetFillColor}
%    \begin{macrocode}
\def\PdfRender@GetFillColor{%
  \global\let\PdfRender@FillColor\ltx@empty
  \begingroup
    \ifPdfRender@Match
      \let\PdfRender@Pattern\PdfRender@PatternFillColor
      \edef\PdfRender@String{\ltx@space\current@color\ltx@space}%
      \PdfRender@MatchPattern\PdfRender@FillColor
    \else
      \edef\current@color{\current@color\ltx@space}%
      \let\PdfRender@OP\relax
      \PdfRender@FindOp{g}0%
      \PdfRender@FindOp{G}1%
      \PdfRender@FindOp{rg}0%
      \PdfRender@FindOp{RG}1%
      \PdfRender@FindOp{k}0%
      \PdfRender@FindOp{K}1%
      \PdfRender@FilterOp 0\PdfRender@FillColor
    \fi
  \endgroup
}
%    \end{macrocode}
%    \end{macro}
%    \begin{macro}{\PdfRender@GetStrokeColor}
%    \begin{macrocode}
\def\PdfRender@GetStrokeColor{%
  \global\let\PdfRender@StrokeColor\ltx@empty
  \begingroup
    \ifPdfRender@Match
      \let\PdfRender@Pattern\PdfRender@PatternStrokeColor
      \edef\PdfRender@String{\ltx@space\current@color\ltx@space}%
      \PdfRender@MatchPattern\PdfRender@StrokeColor
    \else
      \edef\current@color{\current@color\ltx@space}%
      \let\PdfRender@OP\relax
      \PdfRender@FindOp{g}0%
      \PdfRender@FindOp{G}1%
      \PdfRender@FindOp{rg}0%
      \PdfRender@FindOp{RG}1%
      \PdfRender@FindOp{k}0%
      \PdfRender@FindOp{K}1%
      \PdfRender@FilterOp 1\PdfRender@StrokeColor
    \fi
  \endgroup
}
%    \end{macrocode}
%    \end{macro}
%    \begin{macrocode}
\ifPdfRender@Match
  \expandafter\PdfRender@AtEnd
\fi%
%    \end{macrocode}
%    \begin{macro}{\PdfRender@FindOp}
%    \begin{macrocode}
\def\PdfRender@FindOp#1#2{%
  \def\PdfRender@temp##1 #1 ##2\@nil{%
    ##1%
    \ifx\\##2\\%
      \expandafter\@gobble
    \else
      \PdfRender@OP{#1}#2%
      \expandafter\@firstofone
    \fi
    {%
      \PdfRender@temp##2\@nil
    }%
  }%
  \edef\current@color{%
    \@firstofone{\expandafter\PdfRender@temp\current@color} #1 \@nil
  }%
}
%    \end{macrocode}
%    \end{macro}
%    \begin{macro}{\PdfRender@FilterOp}
%    \begin{macrocode}
\def\PdfRender@FilterOp#1#2{%
  \expandafter\PdfRender@@FilterOp\expandafter#1\expandafter#2%
      \current@color\PdfRender@OP{}{}%
}
%    \end{macrocode}
%    \end{macro}
%    \begin{macro}{\PdfRender@@FilterOp}
%    \begin{macrocode}
\def\PdfRender@@FilterOp#1#2#3\PdfRender@OP#4#5{%
  \ifx\\#4#5\\%
  \else
    \ifnum#1=#5 %
      \xdef#2{#3 #4}%
    \fi
    \expandafter\PdfRender@@FilterOp\expandafter#1\expandafter#2%
  \fi
}
%    \end{macrocode}
%    \end{macro}
%
%    \begin{macrocode}
\PdfRender@AtEnd%
%</package>
%    \end{macrocode}
%% \section{Installation}
%
% \subsection{Download}
%
% \paragraph{Package.} This package is available on
% CTAN\footnote{\CTANpkg{pdfrender}}:
% \begin{description}
% \item[\CTAN{macros/latex/contrib/oberdiek/pdfrender.dtx}] The source file.
% \item[\CTAN{macros/latex/contrib/oberdiek/pdfrender.pdf}] Documentation.
% \end{description}
%
%
% \paragraph{Bundle.} All the packages of the bundle `oberdiek'
% are also available in a TDS compliant ZIP archive. There
% the packages are already unpacked and the documentation files
% are generated. The files and directories obey the TDS standard.
% \begin{description}
% \item[\CTANinstall{install/macros/latex/contrib/oberdiek.tds.zip}]
% \end{description}
% \emph{TDS} refers to the standard ``A Directory Structure
% for \TeX\ Files'' (\CTANpkg{tds}). Directories
% with \xfile{texmf} in their name are usually organized this way.
%
% \subsection{Bundle installation}
%
% \paragraph{Unpacking.} Unpack the \xfile{oberdiek.tds.zip} in the
% TDS tree (also known as \xfile{texmf} tree) of your choice.
% Example (linux):
% \begin{quote}
%   |unzip oberdiek.tds.zip -d ~/texmf|
% \end{quote}
%
% \subsection{Package installation}
%
% \paragraph{Unpacking.} The \xfile{.dtx} file is a self-extracting
% \docstrip\ archive. The files are extracted by running the
% \xfile{.dtx} through \plainTeX:
% \begin{quote}
%   \verb|tex pdfrender.dtx|
% \end{quote}
%
% \paragraph{TDS.} Now the different files must be moved into
% the different directories in your installation TDS tree
% (also known as \xfile{texmf} tree):
% \begin{quote}
% \def\t{^^A
% \begin{tabular}{@{}>{\ttfamily}l@{ $\rightarrow$ }>{\ttfamily}l@{}}
%   pdfrender.sty & tex/generic/oberdiek/pdfrender.sty\\
%   pdfrender.pdf & doc/latex/oberdiek/pdfrender.pdf\\
%   pdfrender.dtx & source/latex/oberdiek/pdfrender.dtx\\
% \end{tabular}^^A
% }^^A
% \sbox0{\t}^^A
% \ifdim\wd0>\linewidth
%   \begingroup
%     \advance\linewidth by\leftmargin
%     \advance\linewidth by\rightmargin
%   \edef\x{\endgroup
%     \def\noexpand\lw{\the\linewidth}^^A
%   }\x
%   \def\lwbox{^^A
%     \leavevmode
%     \hbox to \linewidth{^^A
%       \kern-\leftmargin\relax
%       \hss
%       \usebox0
%       \hss
%       \kern-\rightmargin\relax
%     }^^A
%   }^^A
%   \ifdim\wd0>\lw
%     \sbox0{\small\t}^^A
%     \ifdim\wd0>\linewidth
%       \ifdim\wd0>\lw
%         \sbox0{\footnotesize\t}^^A
%         \ifdim\wd0>\linewidth
%           \ifdim\wd0>\lw
%             \sbox0{\scriptsize\t}^^A
%             \ifdim\wd0>\linewidth
%               \ifdim\wd0>\lw
%                 \sbox0{\tiny\t}^^A
%                 \ifdim\wd0>\linewidth
%                   \lwbox
%                 \else
%                   \usebox0
%                 \fi
%               \else
%                 \lwbox
%               \fi
%             \else
%               \usebox0
%             \fi
%           \else
%             \lwbox
%           \fi
%         \else
%           \usebox0
%         \fi
%       \else
%         \lwbox
%       \fi
%     \else
%       \usebox0
%     \fi
%   \else
%     \lwbox
%   \fi
% \else
%   \usebox0
% \fi
% \end{quote}
% If you have a \xfile{docstrip.cfg} that configures and enables \docstrip's
% TDS installing feature, then some files can already be in the right
% place, see the documentation of \docstrip.
%
% \subsection{Refresh file name databases}
%
% If your \TeX~distribution
% (\TeX\,Live, \mikTeX, \dots) relies on file name databases, you must refresh
% these. For example, \TeX\,Live\ users run \verb|texhash| or
% \verb|mktexlsr|.
%
% \subsection{Some details for the interested}
%
% \paragraph{Unpacking with \LaTeX.}
% The \xfile{.dtx} chooses its action depending on the format:
% \begin{description}
% \item[\plainTeX:] Run \docstrip\ and extract the files.
% \item[\LaTeX:] Generate the documentation.
% \end{description}
% If you insist on using \LaTeX\ for \docstrip\ (really,
% \docstrip\ does not need \LaTeX), then inform the autodetect routine
% about your intention:
% \begin{quote}
%   \verb|latex \let\install=y\input{pdfrender.dtx}|
% \end{quote}
% Do not forget to quote the argument according to the demands
% of your shell.
%
% \paragraph{Generating the documentation.}
% You can use both the \xfile{.dtx} or the \xfile{.drv} to generate
% the documentation. The process can be configured by the
% configuration file \xfile{ltxdoc.cfg}. For instance, put this
% line into this file, if you want to have A4 as paper format:
% \begin{quote}
%   \verb|\PassOptionsToClass{a4paper}{article}|
% \end{quote}
% An example follows how to generate the
% documentation with pdf\LaTeX:
% \begin{quote}
%\begin{verbatim}
%pdflatex pdfrender.dtx
%makeindex -s gind.ist pdfrender.idx
%pdflatex pdfrender.dtx
%makeindex -s gind.ist pdfrender.idx
%pdflatex pdfrender.dtx
%\end{verbatim}
% \end{quote}
%
% \section{Acknowledgement}
%
% \begin{description}
% \item[Friedrich Vosberg] asked in the
%   newsgroup \textsf{de.comp.text.tex} for the font
%   outline feature \cite{news-Vosberg}.
% \item[Gaius Pupus] proposed the basic method
%   using \cs{pdfliteral} in this thread \cite{news-Pupus}.
% \item[Rolf Niepraschk] added color support \cite{news-Niepraschk}.
% \end{description}
%
% \begin{thebibliography}{9}
% \raggedright
%
% \bibitem{pdf-spec}
%   Adobe Systems Incorporated. \textit{PDF Reference --
%   Adobe Portable Document format -- Version 1.7}.
%   6th ed. 2006.
%   \textsc{URL:}\,\url{http://www.adobe.com/devnet/acrobat/pdfs/pdf_reference_1-7.pdf}.
%
% \bibitem{news-Vosberg}
%   Friedrich Vosberg, \textit{Text in Buchstabenumrissen},
%   de.comp.text.tex, 2010-01-22.
%   \textsc{URL:}\,\url{https://groups.google.com/group/de.comp.text.tex/msg/f442310ac8b2d506}.
%
% \bibitem{news-Pupus}
%   Gaius Pupus, \textit{Re: Text in Buchstabenumrissen},
%   de.comp.text.tex, 2010-01-23.
%   \textsc{URL:}\,\url{https://groups.google.com/group/de.comp.text.tex/msg/95d890d77ac47eb1}.
%
% \bibitem{news-Niepraschk}
%   Rolf Niepraschk, \textit{Re: Text in Buchstabenumrissen},
%   de.comp.text.tex, 2010-01-24.
%   \textsc{URL:}\,\url{https://groups.google.com/group/de.comp.text.tex/msg/4eb61a5879db54db}.
%
% \end{thebibliography}
%
% \begin{History}
%   \begin{Version}{2010/01/26 v1.0}
%   \item
%     The first version.
%   \end{Version}
%   \begin{Version}{2010/01/27 v1.1}
%   \item
%     Macros \cs{pdfrender} and \cs{textpdfrender} are made robust.
%   \item
%     Color extraction rewritten for the case that \cs{pdfmatch}
%     is not available. This fixes wrong color assigments in
%     case of nesting.
%   \item
%     Color extraction of case \cs{pdfmatch} is fixed
%     for the case that the color string contains several
%     fill or several stroke operations.
%   \end{Version}
%   \begin{Version}{2010/01/28 v1.2}
%   \item
%     Dependency from package \xpackage{color} is removed.
%   \item
%     Compatibility for \plainTeX\ and even ini\TeX\ added.
%   \end{Version}
%   \begin{Version}{2016/05/14 v1.3}
%   \item
%     Use package \xpackage{luatex85} for compatibility with new LuaTeX.
%   \end{Version}
%   \begin{Version}{2016/05/17 v1.4}
%   \item
%     Documentation updates.
%   \item adjust luatex85 reference so that it works in plain TeX.
%   \end{Version}
%   \begin{Version}{2018/11/01 v1.5}
%   \item
%     Remove luatex85 dependency
%   \end{Version}
%   \begin{Version}{2019/12/29 v1.6}
%   \item
%     \xpackage{iftex} package.
%   \end{Version}
% \end{History}
%
% \PrintIndex
%
% \Finale
\endinput
|
% \end{quote}
% Do not forget to quote the argument according to the demands
% of your shell.
%
% \paragraph{Generating the documentation.}
% You can use both the \xfile{.dtx} or the \xfile{.drv} to generate
% the documentation. The process can be configured by the
% configuration file \xfile{ltxdoc.cfg}. For instance, put this
% line into this file, if you want to have A4 as paper format:
% \begin{quote}
%   \verb|\PassOptionsToClass{a4paper}{article}|
% \end{quote}
% An example follows how to generate the
% documentation with pdf\LaTeX:
% \begin{quote}
%\begin{verbatim}
%pdflatex pdfrender.dtx
%makeindex -s gind.ist pdfrender.idx
%pdflatex pdfrender.dtx
%makeindex -s gind.ist pdfrender.idx
%pdflatex pdfrender.dtx
%\end{verbatim}
% \end{quote}
%
% \section{Acknowledgement}
%
% \begin{description}
% \item[Friedrich Vosberg] asked in the
%   newsgroup \textsf{de.comp.text.tex} for the font
%   outline feature \cite{news-Vosberg}.
% \item[Gaius Pupus] proposed the basic method
%   using \cs{pdfliteral} in this thread \cite{news-Pupus}.
% \item[Rolf Niepraschk] added color support \cite{news-Niepraschk}.
% \end{description}
%
% \begin{thebibliography}{9}
% \raggedright
%
% \bibitem{pdf-spec}
%   Adobe Systems Incorporated. \textit{PDF Reference --
%   Adobe Portable Document format -- Version 1.7}.
%   6th ed. 2006.
%   \textsc{URL:}\,\url{http://www.adobe.com/devnet/acrobat/pdfs/pdf_reference_1-7.pdf}.
%
% \bibitem{news-Vosberg}
%   Friedrich Vosberg, \textit{Text in Buchstabenumrissen},
%   de.comp.text.tex, 2010-01-22.
%   \textsc{URL:}\,\url{https://groups.google.com/group/de.comp.text.tex/msg/f442310ac8b2d506}.
%
% \bibitem{news-Pupus}
%   Gaius Pupus, \textit{Re: Text in Buchstabenumrissen},
%   de.comp.text.tex, 2010-01-23.
%   \textsc{URL:}\,\url{https://groups.google.com/group/de.comp.text.tex/msg/95d890d77ac47eb1}.
%
% \bibitem{news-Niepraschk}
%   Rolf Niepraschk, \textit{Re: Text in Buchstabenumrissen},
%   de.comp.text.tex, 2010-01-24.
%   \textsc{URL:}\,\url{https://groups.google.com/group/de.comp.text.tex/msg/4eb61a5879db54db}.
%
% \end{thebibliography}
%
% \begin{History}
%   \begin{Version}{2010/01/26 v1.0}
%   \item
%     The first version.
%   \end{Version}
%   \begin{Version}{2010/01/27 v1.1}
%   \item
%     Macros \cs{pdfrender} and \cs{textpdfrender} are made robust.
%   \item
%     Color extraction rewritten for the case that \cs{pdfmatch}
%     is not available. This fixes wrong color assigments in
%     case of nesting.
%   \item
%     Color extraction of case \cs{pdfmatch} is fixed
%     for the case that the color string contains several
%     fill or several stroke operations.
%   \end{Version}
%   \begin{Version}{2010/01/28 v1.2}
%   \item
%     Dependency from package \xpackage{color} is removed.
%   \item
%     Compatibility for \plainTeX\ and even ini\TeX\ added.
%   \end{Version}
%   \begin{Version}{2016/05/14 v1.3}
%   \item
%     Use package \xpackage{luatex85} for compatibility with new LuaTeX.
%   \end{Version}
%   \begin{Version}{2016/05/17 v1.4}
%   \item
%     Documentation updates.
%   \item adjust luatex85 reference so that it works in plain TeX.
%   \end{Version}
%   \begin{Version}{2018/11/01 v1.5}
%   \item
%     Remove luatex85 dependency
%   \end{Version}
%   \begin{Version}{2019/12/29 v1.6}
%   \item
%     \xpackage{iftex} package.
%   \end{Version}
% \end{History}
%
% \PrintIndex
%
% \Finale
\endinput

%        (quote the arguments according to the demands of your shell)
%
% Documentation:
%    (a) If pdfrender.drv is present:
%           latex pdfrender.drv
%    (b) Without pdfrender.drv:
%           latex pdfrender.dtx; ...
%    The class ltxdoc loads the configuration file ltxdoc.cfg
%    if available. Here you can specify further options, e.g.
%    use A4 as paper format:
%       \PassOptionsToClass{a4paper}{article}
%
%    Programm calls to get the documentation (example):
%       pdflatex pdfrender.dtx
%       makeindex -s gind.ist pdfrender.idx
%       pdflatex pdfrender.dtx
%       makeindex -s gind.ist pdfrender.idx
%       pdflatex pdfrender.dtx
%
% Installation:
%    TDS:tex/generic/oberdiek/pdfrender.sty
%    TDS:doc/latex/oberdiek/pdfrender.pdf
%    TDS:source/latex/oberdiek/pdfrender.dtx
%
%<*ignore>
\begingroup
  \catcode123=1 %
  \catcode125=2 %
  \def\x{LaTeX2e}%
\expandafter\endgroup
\ifcase 0\ifx\install y1\fi\expandafter
         \ifx\csname processbatchFile\endcsname\relax\else1\fi
         \ifx\fmtname\x\else 1\fi\relax
\else\csname fi\endcsname
%</ignore>
%<*install>
\input docstrip.tex
\Msg{************************************************************************}
\Msg{* Installation}
\Msg{* Package: pdfrender 2019/12/29 v1.6 Access to some PDF graphics parameters (HO)}
\Msg{************************************************************************}

\keepsilent
\askforoverwritefalse

\let\MetaPrefix\relax
\preamble

This is a generated file.

Project: pdfrender
Version: 2019/12/29 v1.6

Copyright (C)
   2010 Heiko Oberdiek
   2016-2019 Oberdiek Package Support Group

This work may be distributed and/or modified under the
conditions of the LaTeX Project Public License, either
version 1.3c of this license or (at your option) any later
version. This version of this license is in
   https://www.latex-project.org/lppl/lppl-1-3c.txt
and the latest version of this license is in
   https://www.latex-project.org/lppl.txt
and version 1.3 or later is part of all distributions of
LaTeX version 2005/12/01 or later.

This work has the LPPL maintenance status "maintained".

The Current Maintainers of this work are
Heiko Oberdiek and the Oberdiek Package Support Group
https://github.com/ho-tex/oberdiek/issues


The Base Interpreter refers to any `TeX-Format',
because some files are installed in TDS:tex/generic//.

This work consists of the main source file pdfrender.dtx
and the derived files
   pdfrender.sty, pdfrender.pdf, pdfrender.ins, pdfrender.drv,
   pdfrender-test1.tex, pdfrender-test2.tex,
   pdfrender-test3.tex, pdfrender-test4.tex,
   pdfrender-test5.tex.

\endpreamble
\let\MetaPrefix\DoubleperCent

\generate{%
  \file{pdfrender.ins}{\from{pdfrender.dtx}{install}}%
  \file{pdfrender.drv}{\from{pdfrender.dtx}{driver}}%
  \usedir{tex/generic/oberdiek}%
  \file{pdfrender.sty}{\from{pdfrender.dtx}{package}}%
%  \usedir{doc/latex/oberdiek/test}%
%  \file{pdfrender-test1.tex}{\from{pdfrender.dtx}{test1}}%
%  \file{pdfrender-test2.tex}{\from{pdfrender.dtx}{test2}}%
%  \file{pdfrender-test3.tex}{\from{pdfrender.dtx}{test3}}%
%  \file{pdfrender-test4.tex}{\from{pdfrender.dtx}{test4}}%
%  \file{pdfrender-test5.tex}{\from{pdfrender.dtx}{test5}}%
}

\catcode32=13\relax% active space
\let =\space%
\Msg{************************************************************************}
\Msg{*}
\Msg{* To finish the installation you have to move the following}
\Msg{* file into a directory searched by TeX:}
\Msg{*}
\Msg{*     pdfrender.sty}
\Msg{*}
\Msg{* To produce the documentation run the file `pdfrender.drv'}
\Msg{* through LaTeX.}
\Msg{*}
\Msg{* Happy TeXing!}
\Msg{*}
\Msg{************************************************************************}

\endbatchfile
%</install>
%<*ignore>
\fi
%</ignore>
%<*driver>
\NeedsTeXFormat{LaTeX2e}
\ProvidesFile{pdfrender.drv}%
  [2019/12/29 v1.6 Access to some PDF graphics parameters (HO)]%
\documentclass{ltxdoc}
\usepackage{holtxdoc}[2011/11/22]
\usepackage{longtable}
\renewcommand*{\arraystretch}{1.2}
\begin{document}
  \DocInput{pdfrender.dtx}%
\end{document}
%</driver>
% \fi
%
%
%
% \GetFileInfo{pdfrender.drv}
%
% \title{The \xpackage{pdfrender} package}
% \date{2019/12/29 v1.6}
% \author{Heiko Oberdiek\thanks
% {Please report any issues at \url{https://github.com/ho-tex/oberdiek/issues}}}
%
% \maketitle
%
% \begin{abstract}
% The PDF format has some graphics parameter like
% line width or text rendering mode. This package
% provides an interface for setting these parameters.
% \end{abstract}
%
% \tableofcontents
%
% \section{Documentation}
%
% This package \xpackage{pdfrender} defines an interface for
% PDF specific parameters that affects the rendering of
% graphics or text. The interface and its implementation
% uses the same technique as package \xpackage{color}
% for color settings. Therefore this package is loaded to
% enable \LaTeX's color interface.
%
% At different places \LaTeX\ uses \cs{normalcolor} to avoid that
% header, footer or floats are print in the current color of the
% main text. \cs{setgroup@color} is used to start a save box
% with the color that is set at box saving time.
% Package \xpackage{pdfrender} extends these macros to
% add its own hooks of its parameters. Therefore \LaTeX3 should
% generalize \LaTeXe's color interface.
%
% \subsection{Usage}
%
% In \LaTeX\ the package is loaded as normal package. Options
% are not defined for this package.
% \begin{quote}
% |\usepackage{pdfrender}|
% \end{quote}
%
% This package can also be used in \plainTeX\ and even ini\TeX:
% \begin{quote}
% |input pdfrender.sty|
% \end{quote}
%
% \subsection{Macros}
%
% \begin{declcs}{pdfrender} \M{key value list}
% \end{declcs}
% The first parameter \meta{key value list} contains a
% list of parameter settings. The key entry is the parameter name.
% The macro works like
% \cs{color} (without optional argument) for color setting.
%
% \begin{declcs}{textpdfrender} \M{key value list} \M{text}
% \end{declcs}
% In the same way as \cs{pdfrender} the first argument
% specifies the parameters that should be set. This parameter
% setting affects \meta{text} only. Basically it works
% the same way as \cs{textcolor} (without optional argument).
%
% \subsection{Parameters}
%
% The following table shows an overview for the supported
% parameters and values:
%
% \begin{longtable}{@{}l|l|l@{}}
% \bfseries Parameter & \bfseries Value & \bfseries Alias\\*
% \hline
% \endhead
% \xoption{TextRenderingMode}
%   & \texttt{0} & \texttt{Fill}\\*
%   & \texttt{1} & \texttt{Stroke}\\*
%   & \texttt{2} & \texttt{FillStroke}\\*
%   & \texttt{3} & \texttt{Invisible}\\*
%   & \texttt{4} & \texttt{FillClip}\\*
%   & \texttt{5} & \texttt{StrokeClip}\\*
%   & \texttt{6} & \texttt{FillStrokeClip}\\*
%   & \texttt{7} & \texttt{Clip}\\
% \hline
% \xoption{LineWidth}
%   & \itshape positive number, & \itshape \TeX\ dimen\\*[-.2ex]
%   & \qquad \itshape unit is bp & \\
% \hline
% \xoption{LineCapStyle}
%   & \texttt{0} & \texttt{Butt}\\*
%   & \texttt{1} & \texttt{Round}\\*
%   & \texttt{2} & \texttt{ProjectingSquare}\\
% \hline
% \xoption{LineJoinStyle}
%   & \texttt{0} & \texttt{Miter}\\*
%   & \texttt{1} & \texttt{Round}\\*
%   & \texttt{2} & \texttt{Bevel}\\
% \hline
% \xoption{MiterLimit}
%   & \multicolumn{2}{l@{}}{\itshape positive number} \\
% \hline
% \xoption{Flatness}
%   & \multicolumn{2}{l@{}}{\itshape number between 0 and 100} \\
% \hline
% \xoption{LineDashPattern}
%   & \multicolumn{2}{l@{}}{^^A
%       \itshape numbers in square brackets,^^A
%     } \\*[-.2ex]
%   & \multicolumn{2}{l@{}}{^^A
%       \qquad \itshape followed by number,^^A
%     } \\*
%   & \multicolumn{2}{l@{}}{^^A
%       \qquad \itshape units are bp^^A
%     } \\
% \hline
% \xoption{RenderingIntent}
%   & \multicolumn{2}{l@{}}{\texttt{AbsoluteColorimetric}} \\*
%   & \multicolumn{2}{l@{}}{\texttt{RelativeColorimetric}} \\*
%   & \multicolumn{2}{l@{}}{\texttt{Saturation}} \\*
%   & \multicolumn{2}{l@{}}{\texttt{Perceptual}} \\
% \hline
% \xoption{FillColor}
%   & & \itshape color specification \\
% \hline
% \xoption{StrokeColor}
%   & & \itshape color specification \\
% \end{longtable}
%
% \subsubsection{Details}
%
% The description and specification of these parameters
% are available in the PDF specification \cite{pdf-spec}.
% Therefore they are not repeated here.
%
% \begin{description}
% \item[Value:]
%   The values in the second column lists or describe
%   the values that are specified by the PDF specification.
% \item[Alias:] Instead of magic numbers the package also
%   defines some aliases that can be given as value.
%   Example: |LineCapStyle=Round| has the same effect as
%   |LineCapStyle=1|.
% \item[Number:] The term \emph{number} means an integer
%   or real number. The real number is given as plain
%   decimal number without exponent. The decimal separator is a period.
%   At least one digit must be present.
% \item[\xoption{LineWidth}:] As alias a \TeX\ dimen specification
%   can be given. This includes explicit specifications with number
%   and unit, e.g. |LineWidth=0.5pt|. Also \LaTeX\ length registers
%   may be used. If \eTeX's \cs{dimexpr} is available, then it is
%   automatically added. However package \xpackage{calc} is not
%   supported.
% \item[\xoption{FillColor}, \xoption{StrokeColor}:]
%   Package \xpackage{color} or \xpackage{xcolor} must be loaded
%   before these options can be used (since version 1.2).
%   \LaTeX's color support sets both colors at the same time
%   to the same value. However parameter \xoption{TextRenderingMode}
%   offers the value \texttt{FillStroke} that makes only sense, if
%   the two color types can be set separately.
%   If one of the options \xoption{FillColor} or \xoption{StrokeColor}
%   is specified, then also the color is set. For compatibility
%   with the \LaTeX\ color packages (\xpackage{color} or \xpackage{xcolor}),
%   always both colors must be set. Thus if one of them is not specified,
%   it is taken from the current color.
%
%   Both options \xoption{FillColor} and \xoption{StrokeColor}
%   expect a \LaTeX\ color specification as value.
%   Also the optional color model argument is supported.
%   Example:
%   \begin{quote}
%     |FillColor=yellow|,\\
%     |StrokeColor=[cmyk]{1,.5,0,0}|\\
%   \end{quote}
% \end{description}
%
% \subsection{Color stack}
%
% If the \pdfTeX\ version provides color stacks, then each
% parameter is assigned a page based color stack. The assignment
% of a stack takes place, when its parameter is set the first time.
% This avoids the use of color stacks that are not needed.
%
% \StopEventually{
% }
%
% \section{Implementation}
%    \begin{macrocode}
%<*package>
%    \end{macrocode}
%    Reload check, especially if the package is not used with \LaTeX.
%    \begin{macrocode}
\begingroup\catcode61\catcode48\catcode32=10\relax%
  \catcode13=5 % ^^M
  \endlinechar=13 %
  \catcode35=6 % #
  \catcode39=12 % '
  \catcode44=12 % ,
  \catcode45=12 % -
  \catcode46=12 % .
  \catcode58=12 % :
  \catcode64=11 % @
  \catcode123=1 % {
  \catcode125=2 % }
  \expandafter\let\expandafter\x\csname ver@pdfrender.sty\endcsname
  \ifx\x\relax % plain-TeX, first loading
  \else
    \def\empty{}%
    \ifx\x\empty % LaTeX, first loading,
      % variable is initialized, but \ProvidesPackage not yet seen
    \else
      \expandafter\ifx\csname PackageInfo\endcsname\relax
        \def\x#1#2{%
          \immediate\write-1{Package #1 Info: #2.}%
        }%
      \else
        \def\x#1#2{\PackageInfo{#1}{#2, stopped}}%
      \fi
      \x{pdfrender}{The package is already loaded}%
      \aftergroup\endinput
    \fi
  \fi
\endgroup%
%    \end{macrocode}
%    Package identification:
%    \begin{macrocode}
\begingroup\catcode61\catcode48\catcode32=10\relax%
  \catcode13=5 % ^^M
  \endlinechar=13 %
  \catcode35=6 % #
  \catcode39=12 % '
  \catcode40=12 % (
  \catcode41=12 % )
  \catcode44=12 % ,
  \catcode45=12 % -
  \catcode46=12 % .
  \catcode47=12 % /
  \catcode58=12 % :
  \catcode64=11 % @
  \catcode91=12 % [
  \catcode93=12 % ]
  \catcode123=1 % {
  \catcode125=2 % }
  \expandafter\ifx\csname ProvidesPackage\endcsname\relax
    \def\x#1#2#3[#4]{\endgroup
      \immediate\write-1{Package: #3 #4}%
      \xdef#1{#4}%
    }%
  \else
    \def\x#1#2[#3]{\endgroup
      #2[{#3}]%
      \ifx#1\@undefined
        \xdef#1{#3}%
      \fi
      \ifx#1\relax
        \xdef#1{#3}%
      \fi
    }%
  \fi
\expandafter\x\csname ver@pdfrender.sty\endcsname
\ProvidesPackage{pdfrender}%
  [2019/12/29 v1.6 Access to some PDF graphics parameters (HO)]%
%    \end{macrocode}
%
%    \begin{macrocode}
\begingroup\catcode61\catcode48\catcode32=10\relax%
  \catcode13=5 % ^^M
  \endlinechar=13 %
  \catcode123=1 % {
  \catcode125=2 % }
  \catcode64=11 % @
  \def\x{\endgroup
    \expandafter\edef\csname PdfRender@AtEnd\endcsname{%
      \endlinechar=\the\endlinechar\relax
      \catcode13=\the\catcode13\relax
      \catcode32=\the\catcode32\relax
      \catcode35=\the\catcode35\relax
      \catcode61=\the\catcode61\relax
      \catcode64=\the\catcode64\relax
      \catcode123=\the\catcode123\relax
      \catcode125=\the\catcode125\relax
    }%
  }%
\x\catcode61\catcode48\catcode32=10\relax%
\catcode13=5 % ^^M
\endlinechar=13 %
\catcode35=6 % #
\catcode64=11 % @
\catcode123=1 % {
\catcode125=2 % }
\def\TMP@EnsureCode#1#2{%
  \edef\PdfRender@AtEnd{%
    \PdfRender@AtEnd
    \catcode#1=\the\catcode#1\relax
  }%
  \catcode#1=#2\relax
}
\TMP@EnsureCode{10}{12}% ^^J
\TMP@EnsureCode{36}{3}% $
\TMP@EnsureCode{39}{12}% '
\TMP@EnsureCode{40}{12}% (
\TMP@EnsureCode{41}{12}% )
\TMP@EnsureCode{42}{12}% *
\TMP@EnsureCode{43}{12}% +
\TMP@EnsureCode{44}{12}% ,
\TMP@EnsureCode{45}{12}% -
\TMP@EnsureCode{46}{12}% .
\TMP@EnsureCode{47}{12}% /
\TMP@EnsureCode{58}{12}% :
\TMP@EnsureCode{59}{12}% ;
\TMP@EnsureCode{60}{12}% <
\TMP@EnsureCode{62}{12}% >
\TMP@EnsureCode{63}{12}% ?
\TMP@EnsureCode{91}{12}% [
\TMP@EnsureCode{93}{12}% ]
\TMP@EnsureCode{94}{7}% ^ (superscript)
\TMP@EnsureCode{96}{12}% `
\TMP@EnsureCode{124}{12}% |
%    \end{macrocode}
%
%    \begin{macrocode}
\def\PdfRender@AtEndHook{}
\expandafter\def\expandafter\PdfRender@AtEnd\expandafter{%
  \expandafter\PdfRender@AtEndHook
  \PdfRender@AtEnd
  \endinput
}
%    \end{macrocode}
%
% \subsection{Look for \pdfTeX, its mode and features}
%
%    \begin{macro}{\PdfRender@newif}
%    \begin{macrocode}
\def\PdfRender@newif#1{%
  \expandafter\edef\csname PdfRender@#1true\endcsname{%
    \let
    \expandafter\noexpand\csname ifPdfRender@#1\endcsname
    \noexpand\iftrue
  }%
  \expandafter\edef\csname PdfRender@#1false\endcsname{%
    \let
    \expandafter\noexpand\csname ifPdfRender@#1\endcsname
    \noexpand\iffalse
  }%
  \csname PdfRender@#1false\endcsname
}
%    \end{macrocode}
%    \end{macro}
%
%    \begin{macro}{\ifPdfRender@Stack}
%    \begin{macrocode}
\PdfRender@newif{Stack}
%    \end{macrocode}
%    \end{macro}
%    \begin{macro}{\ifPdfRender@Match}
%    \begin{macrocode}
\PdfRender@newif{Match}
%    \end{macrocode}
%    \end{macro}
%
%    \begin{macro}{\PdfRender@RequirePackage}
%    \begin{macrocode}
\begingroup\expandafter\expandafter\expandafter\endgroup
\expandafter\ifx\csname RequirePackage\endcsname\relax
  \def\PdfRender@RequirePackage#1[#2]{%
    \expandafter\def\expandafter\PdfRender@AtEndHook\expandafter{%
      \PdfRender@AtEndHook
      \ltx@ifpackagelater{#1}{#2}{}{%
        \@PackageWarningNoLine{pdfrender}{%
          You have requested version\MessageBreak
          `#2' of package `#1',\MessageBreak
          but only version\MessageBreak
          `\csname ver@#1.\ltx@pkgextension\endcsname'\MessageBreak
          is available%
        }%
      }%
    }%
    \input #1.sty\relax
  }%
\else
  \let\PdfRender@RequirePackage\RequirePackage
\fi
%    \end{macrocode}
%    \end{macro}
% Luatex compatibility
%    \begin{macrocode}
\ifx\pdfextension\@undefined\else
  \def\pdfcolorstackinit {\pdffeedback colorstackinit}
  \protected\def\pdfcolorstack     {\pdfextension colorstack}
  \protected\def\pdfliteral        {\pdfextension literal}
\fi
%    \end{macrocode}
%
%    \begin{macrocode}
\PdfRender@RequirePackage{iftex}[2019/11/07]
\PdfRender@RequirePackage{infwarerr}[2007/09/09]
\PdfRender@RequirePackage{ltxcmds}[2010/01/28]
%    \end{macrocode}
%
%    \begin{macrocode}
\ifpdf
  \ltx@IfUndefined{pdfcolorstackinit}{%
    \@PackageWarning{pdfrender}{%
      Missing \string\pdfcolorstackinit
    }%
  }{%
    \PdfRender@Stacktrue
  }%
  \ltx@IfUndefined{pdfmatch}{%
    \@PackageInfoNoLine{pdfrender}{%
      \string\pdfmatch\ltx@space not found. %
      Therefore the values\MessageBreak
      of some parameters are not validated%
    }%
  }{%
    \PdfRender@Matchtrue
  }%
\else
  \@PackageWarning{pdfrender}{%
    Missing pdfTeX in PDF mode%
  }%
%    \end{macrocode}
%    \begin{macrocode}
  \ltx@IfUndefined{newcommand}{%
%    \end{macrocode}
%    \begin{macro}{\pdfrender}
%    \begin{macrocode}
    \def\pdfrender#1{}%
%    \end{macrocode}
%    \end{macro}
%    \begin{macro}{\textpdfrender}
%    \begin{macrocode}
    \long\def\textpdfrender#1#2{#2}%
%    \end{macrocode}
%    \end{macro}
%    \begin{macrocode}
  }{%
%    \end{macrocode}
%    \begin{macro}{\pdfrender}
%    \begin{macrocode}
    \newcommand*{\pdfrender}[1]{}%
%    \end{macrocode}
%    \end{macro}
%    \begin{macro}{\textpdfrender}
%    \begin{macrocode}
    \newcommand{\textpdfrender}[2]{#2}%
%    \end{macrocode}
%    \end{macro}
%    \begin{macrocode}
  }%
  \expandafter\PdfRender@AtEnd
\fi%
%    \end{macrocode}
%
% \subsection{Enable color support of \LaTeX}
%
%    \begin{macrocode}
\ltx@ifpackageloaded{color}{}{%
  \def\color@setgroup{\begingroup\set@color}%
  \let\color@begingroup\begingroup
  \def\color@endgroup{\endgraf\endgroup}%
  \def\color@hbox{\hbox\bgroup\color@begingroup}%
  \def\color@vbox{\vbox\bgroup\color@begingroup}%
  \def\color@endbox{\color@endgroup\egroup}%
  \ltx@ifundefined{bgroup}{%
    \let\bgroup={\let\egroup=}%
  }{}%
  \ltx@ifundefined{endgraf}{%
    \let\endgraf=\par
  }{}%
}
%    \end{macrocode}
%
% \subsection{Hook into \cs{normalcolor}}
%
%    The problem is that packages \xpackage{color} and \xpackage{xcolor}
%    each overwrite \cs{normalcolor}. For example, after the package
%    loading order \xpackage{color}, \xpackage{pdfrender} and \xpackage{xcolor}
%    the patched version of \cs{normalcolor} is overwritten by
%    package \xpackage{xcolor}. Also using \cs{AtBeginDocument} for
%    patching is not enough. If package \xpackage{hyperref} is loaded
%    later, it might load package \xpackage{color} using \cs{AtBeginDocument}.
%    \begin{macro}{\PdfRender@NormalColorHook}
%    \begin{macrocode}
\def\PdfRender@NormalColorHook{}
%    \end{macrocode}
%    \end{macro}
%    \begin{macro}{\PdfRender@ColorSetGroupHook}
%    \begin{macrocode}
\def\PdfRender@ColorSetGroupHook{}
%    \end{macrocode}
%    \end{macro}
%    \begin{macro}{\PdfRender@TestBox}
%    \begin{macrocode}
\def\PdfRender@TestBox#1{%
  \setbox0=\color@hbox#1\color@endbox
}
%    \end{macrocode}
%    \end{macro}
%    \begin{macro}{\PdfRender@PatchNormalColor}
%    \begin{macrocode}
\def\PdfRender@PatchNormalColor{%
  \ltx@ifundefined{normalcolor}{%
    \gdef\normalcolor{\PdfRender@NormalColorHook}%
  }{%
    \begingroup
      \def\PdfRender@NormalColorHook{\let\PdfRender@temp=Y}%
      \PdfRender@TestBox{%
        \let\set@color\relax
        \normalcolor
        \ifx\PdfRender@temp Y%
        \else
          \ltx@GlobalAppendToMacro\normalcolor{%
            \PdfRender@NormalColorHook
          }%
        \fi
      }%
    \endgroup
  }%
  \ifx\@nodocument\relax
    \global\let\PdfRender@PatchNormalColor\relax
  \fi
}%
%    \end{macrocode}
%    \end{macro}
%    \begin{macro}{\PdfRender@PatchColorSetGroup}
%    \begin{macrocode}
\def\PdfRender@PatchColorSetGroup{%
  \begingroup
    \def\PdfRender@ColorSetGroupHook{\let\PdfRender@temp=Y}%
    \PdfRender@TestBox{%
      \let\set@color\relax
      \color@setgroup\color@endgroup
      \ifx\PdfRender@temp Y%
      \else
        \ltx@GlobalAppendToMacro\color@setgroup{%
          \PdfRender@ColorSetGroupHook
        }%
      \fi
    }%
  \endgroup
  \ifx\@nodocument\relax
    \global\let\PdfRender@PatchColorSetGroup\relax
  \fi
}%
%    \end{macrocode}
%    \end{macro}
%    \begin{macro}{\PdfRender@PatchColor}
%    \begin{macrocode}
\def\PdfRender@PatchColor{%
  \PdfRender@PatchNormalColor
  \PdfRender@PatchColorSetGroup
}
%    \end{macrocode}
%    \end{macro}
%    \begin{macrocode}
\PdfRender@PatchColor
\ltx@IfUndefined{AtBeginDocument}{}{%
  \AtBeginDocument{\PdfRender@PatchColor}%
}
%    \end{macrocode}
%    \cs{AfterPackage} is provided by package \xpackage{scrlfile}.
%    \begin{macrocode}
\ltx@IfUndefined{AfterPackage}{%
}{%
  \AfterPackage{color}{\PdfRender@PatchColor}%
  \AfterPackage{xcolor}{\PdfRender@PatchColor}%
  \AfterPackage{etoolbox}{%
    \AfterEndPreamble{\PdfRender@PatchColor}%
  }%
}%
%    \end{macrocode}
%    \cs{AfterEndPreamble} is provided by package \xpackage{etoolbox}.
%    \begin{macrocode}
\ltx@IfUndefined{AfterEndPreamble}{%
}{%
  \AfterEndPreamble{\PdfRender@PatchColor}%
}%
%    \end{macrocode}
%
%    \begin{macrocode}
\PdfRender@RequirePackage{kvsetkeys}[2010/01/28]
%    \end{macrocode}
%    \begin{macro}{\PdfRender@texorpdfstring}
%    \begin{macrocode}
\def\PdfRender@texorpdfstring{%
  \ltx@IfUndefined{texorpdfstring}\ltx@firstoftwo\texorpdfstring
}
%    \end{macrocode}
%    \end{macro}
%    \begin{macro}{\pdfrender}
%    \begin{macrocode}
\ltx@IfUndefined{DeclareRobustCommand}%
\ltx@firstoftwo\ltx@secondoftwo
{%
  \def\pdfrender#1%
}{%
  \newcommand{\pdfrender}{}%
  \DeclareRobustCommand*{\pdfrender}[1]%
}%
{%
  \PdfRender@texorpdfstring{%
    \PdfRender@PatchNormalColor
    \global\let\PdfRender@FillColor\ltx@empty
    \global\let\PdfRender@StrokeColor\ltx@empty
    \kvsetkeys{PDFRENDER}{#1}%
    \PdfRender@SetColor
  }{}%
}
%    \end{macrocode}
%    \end{macro}
%    \begin{macro}{\textpdfrender}
%    \begin{macrocode}
\ltx@IfUndefined{DeclareRobustCommand}%
\ltx@firstoftwo\ltx@secondoftwo
{%
  \long\def\textpdfrender#1#2%
}{%
  \newcommand{\textpdfrender}{}%
  \DeclareRobustCommand{\textpdfrender}[2]%
}%
{%
  \PdfRender@texorpdfstring{%
    \begingroup
      \pdfrender{#1}%
      #2%
    \endgroup
  }{#2}%
}
%    \end{macrocode}
%    \end{macro}
%
%    \begin{macro}{\ifPdfRender@Values}
%    \begin{macrocode}
\PdfRender@newif{Values}
%    \end{macrocode}
%    \end{macro}
%    \begin{macro}{\PdfRender@NewClassValues}
%    \begin{macrocode}
\def\PdfRender@NewClassValues#1#2#3#4{%
  \PdfRender@Valuestrue
  \PdfRender@NewClass{#1}{#2}{#3}{#4}{}%
}
%    \end{macrocode}
%    \end{macro}
%    \begin{macro}{\PdfRender@NewClass}
%    \begin{macrocode}
\def\PdfRender@NewClass#1#2#3#4#5{%
  \PdfRender@newif{Active#1}%
  \expandafter\def\csname PdfRender@Default#1\endcsname{#2}%
  \expandafter\let\csname PdfRender@Current#1\expandafter\endcsname
      \csname PdfRender@Default#1\endcsname
  \ifPdfRender@Stack
    \expandafter\edef\csname PdfRender@Init#1\endcsname{%
      \global\chardef
      \expandafter\noexpand\csname PdfRender@Stack#1\endcsname=%
          \noexpand\pdfcolorstackinit page direct{%
            \noexpand#3%
            \expandafter\noexpand\csname PdfRender@Default#1\endcsname
          }\relax
      \noexpand\@PackageInfo{pdfrender}{%
        New color stack `#1' = \noexpand\number
        \expandafter\noexpand\csname PdfRender@Stack#1\endcsname
      }%
      \gdef\expandafter\noexpand\csname PdfRender@Init#1\endcsname{}%
    }%
    \expandafter\edef\csname PdfRender@Set#1\endcsname{%
      \expandafter\noexpand\csname PdfRender@Init#1\endcsname
      \noexpand\pdfcolorstack
      \expandafter\noexpand\csname PdfRender@Stack#1\endcsname
      push{%
        #3{\expandafter\noexpand\csname PdfRender@Current#1\endcsname}%
      }%
      \noexpand\aftergroup
      \expandafter\noexpand\csname PdfRender@Reset#1\endcsname
    }%
    \expandafter\edef\csname PdfRender@Reset#1\endcsname{%
      \expandafter\noexpand\csname PdfRender@Init#1\endcsname
      \noexpand\pdfcolorstack
      \expandafter\noexpand\csname PdfRender@Stack#1\endcsname
      pop\relax
    }%
  \else
    \expandafter\edef\csname PdfRender@Set#1\endcsname{%
      \noexpand\pdfliteral direct{%
        #3{\expandafter\noexpand\csname PdfRender@Current#1\endcsname}%
      }%
      \noexpand\aftergroup
      \expandafter\noexpand\csname PdfRender@Reset#1\endcsname
    }%
    \expandafter\edef\csname PdfRender@Reset#1\endcsname{%
      \noexpand\pdfliteral direct{%
        #3{\expandafter\noexpand\csname PdfRender@Current#1\endcsname}%
      }%
    }%
  \fi
  \expandafter\edef\csname PdfRender@Normal#1\endcsname{%
    \let
    \expandafter\noexpand\csname PdfRender@Current#1\endcsname
    \expandafter\noexpand\csname PdfRender@Default#1\endcsname
    \noexpand\PdfRender@Set{#1}%
  }%
  \expandafter\ltx@GlobalAppendToMacro\expandafter\PdfRender@NormalColorHook
  \expandafter{%
    \csname PdfRender@Normal#1\endcsname
  }%
  \ltx@GlobalAppendToMacro\PdfRender@ColorSetGroupHook{%
    \PdfRender@Set{#1}%
  }%
  \ifPdfRender@Values
    \kv@parse@normalized{#4}{%
      \expandafter\let\csname PdfRender@#1@\kv@key\endcsname\kv@key
      \ifx\kv@value\relax
      \else
        \expandafter\let\csname PdfRender@#1@\kv@value\endcsname\kv@key
      \fi
      \ltx@gobbletwo
    }%
    \PdfRender@define@key{PDFRENDER}{#1}{%
      \global\csname PdfRender@Active#1true\endcsname
      \def\PdfRender@Current{##1}%
      \PdfRender@SetValidateValues{#1}%
    }%
    \PdfRender@Valuesfalse
  \else
    \PdfRender@define@key{PDFRENDER}{#1}{%
      \global\csname PdfRender@Active#1true\endcsname
      \expandafter\def\csname PdfRender@Current#1\endcsname{##1}%
      \ltx@IfUndefined{PdfRender@PostProcess#1}{%
      }{%
        \csname PdfRender@PostProcess#1\endcsname
      }%
      \PdfRender@SetValidate{#1}{#4}{#5}%
    }%
  \fi
}%
%    \end{macrocode}
%    \end{macro}
%    \begin{macro}{\PdfRender@define@key}
%    \begin{macrocode}
\ltx@IfUndefined{define@key}{%
  \def\PdfRender@define@key#1#2{%
    \expandafter\def\csname KV@#1@#2\endcsname##1%
  }%
}{%
  \let\PdfRender@define@key\define@key
}
%    \end{macrocode}
%    \end{macro}
%    \begin{macro}{\PdfRender@Set}
%    \begin{macrocode}
\def\PdfRender@Set#1{%
  \csname ifPdfRender@Active#1\endcsname
    \csname PdfRender@Set#1\expandafter\endcsname
  \fi
}
%    \end{macrocode}
%    \end{macro}
%    \begin{macro}{\PdfRender@Reset}
%    \begin{macrocode}
\def\PdfRender@Reset#1{%
  \csname ifPdfRender@Active#1\endcsname
    \csname PdfRender@Reset#1\expandafter\endcsname
  \fi
}
%    \end{macrocode}
%    \end{macro}
%    \begin{macro}{\PdfRender@ErrorInvalidValue}
%    \begin{macrocode}
\def\PdfRender@ErrorInvalidValue#1{%
  \PackageError{pdfrender}{%
    Ignoring parameter setting for `#1'\MessageBreak
    because of invalid value %
    `\csname PdfRender@Current#1\endcsname'%
  }\@ehc
  \expandafter\let\csname PdfRender@Current#1\endcsname\ltx@empty
}%
%    \end{macrocode}
%    \end{macro}
%    \begin{macro}{\PdfRender@SetValidate}
%    \begin{macrocode}
\ifPdfRender@Match
  \def\PdfRender@SetValidate#1#2#3{%
    \ifnum\pdfmatch{^(#2)$}{\csname PdfRender@Current#1\endcsname}=1 %
      \csname PdfRender@Set#1\expandafter\endcsname
    \else
      \PdfRender@ErrorInvalidValue{#1}%
    \fi
  }%
\else
  \def\PdfRender@SetValidate#1#2#3{%
    \expandafter\let\expandafter\PdfRender@Current
    \csname PdfRender@Current#1\endcsname
    #3%
    \ifx\PdfRender@Current\@empty
      \PdfRender@ErrorInvalidValue{#1}%
    \else
      \csname PdfRender@Set#1\expandafter\endcsname
    \fi
  }%
\fi
%    \end{macrocode}
%    \end{macro}
%    \begin{macro}{\PdfRender@SetValidateValues}
%    \begin{macrocode}
\def\PdfRender@SetValidateValues#1{%
  \ltx@IfUndefined{PdfRender@#1@\PdfRender@Current}{%
    \expandafter\let\csname PdfRender@Current#1\endcsname
                    \PdfRender@Current
    \PdfRender@ErrorInvalidValue{#1}%
  }{%
    \expandafter\edef\csname PdfRender@Current#1\endcsname{%
      \csname PdfRender@#1@\PdfRender@Current\endcsname
    }%
    \csname PdfRender@Set#1\endcsname
  }%
}
%    \end{macrocode}
%    \end{macro}
%    \begin{macro}{\PdfRender@OpValue}
%    \begin{macrocode}
\def\PdfRender@OpValue#1#2{#2\ltx@space#1}%
%    \end{macrocode}
%    \end{macro}
%    \begin{macro}{\PdfRender@OpName}
%    \begin{macrocode}
\def\PdfRender@OpName#1#2{/#2\ltx@space#1}%
%    \end{macrocode}
%    \end{macro}
%
% \subsection{Declare and setup parameters}
%
%    \begin{macrocode}
\PdfRender@NewClassValues{TextRenderingMode}%
                      {0}%
                      {\PdfRender@OpValue{Tr}}{%
  0=Fill,%
  1=Stroke,%
  2=FillStroke,%
  3=Invisible,%
  4=FillClip,%
  5=StrokeClip,%
  6=FillStrokeClip,%
  7=Clip,%
}%
\PdfRender@NewClass{LineWidth}{1}{\PdfRender@OpValue{w}}{%
  [0-9]+\string\.?[0-9]*|\string\.[0-9]+%
}{}%
\ltx@IfUndefined{dimexpr}{%
  \def\PdfRender@dimexpr{}%
}{%
  \let\PdfRender@dimexpr\dimexpr
}
\def\PdfRender@PostProcessLineWidth{%
  \begingroup
  \afterassignment\PdfRender@@PostProcessLineWidth
  \dimen0=\PdfRender@dimexpr\PdfRender@CurrentLineWidth bp %
  \PdfRender@let\PdfRender@relax\PdfRender@relax
}
\let\PdfRender@let\let
\let\PdfRender@relax\relax
\def\PdfRender@@PostProcessLineWidth#1\PdfRender@let{%
  \ifx\\#1\\%
    \endgroup
  \else
    \dimen0=.996264\dimen0 % 72/72.27
    \edef\x{\endgroup
      \def\noexpand\PdfRender@CurrentLineWidth{%
        \strip@pt\dimen0%
      }%
    }%
    \expandafter\x
  \fi
}
\PdfRender@NewClassValues{LineCapStyle}{0}{\PdfRender@OpValue{J}}{%
  0=Butt,%
  1=Round,%
  2=ProjectingSquare,%
}%
\PdfRender@NewClassValues{LineJoinStyle}{0}{\PdfRender@OpValue{j}}{%
  0=Miter,%
  1=Round,%
  2=Bevel,%
}%
\PdfRender@NewClass{MiterLimit}{10}{\PdfRender@OpValue{M}}{%
  [0-9]*[1-9][0-9]*\string\.?[0-9]*|%
  [0-9]*\string\.?[0-9]*[1-9][0-9]*%
}{}%
\PdfRender@NewClass{Flatness}{0}{\PdfRender@OpValue{i}}{%
  100(\string\.0*)?|[0-9][0-9](\string\.[0-9]*)?|\string\.[0-9]+%
}{}%
\PdfRender@NewClass{LineDashPattern}{[]0}{\PdfRender@OpValue{d}}{%
  \string\[%
  ( ?([0-9]+\string\.?[0-9]*|\string\.[0-9]+) ?)*%
  \string\] ?%
  ([0-9]+\string\.?[0-9]*|\string\.[0-9]+)%
}{}%
\PdfRender@NewClassValues{RenderingIntent}%
                      {RelativeColorimetric}%
                      {\PdfRender@OpName{ri}}{%
  AbsoluteColorimetric,%
  RelativeColorimetric,%
  Saturation,%
  Perceptual,%
}%
%    \end{macrocode}
%
% \subsection{Fill and stroke color support}
%
%    \begin{macrocode}
\PdfRender@define@key{PDFRENDER}{FillColor}{%
  \begingroup
    \def\PdfRender@Color{#1}%
    \ifx\PdfRender@Color\ltx@empty
      \global\let\PdfRender@FillColor\ltx@empty
    \else
      \PdfRender@ColorAvailable{%
        \PdfRender@TestBox{%
          \expandafter\PdfRender@TryColor\PdfRender@Color\ltx@empty
          \PdfRender@GetFillColor
          \ifx\PdfRender@FillColor\ltx@empty
            \@PackageWarning{pdfrender}{%
              Cannot extract fill color\MessageBreak
              from value `#1'%
            }%
          \fi
        }%
      }%
    \fi
  \endgroup
}
\PdfRender@define@key{PDFRENDER}{StrokeColor}{%
  \begingroup
    \def\PdfRender@Color{#1}%
    \ifx\PdfRender@Color\ltx@empty
      \global\let\PdfRender@StrokeColor\ltx@empty
    \else
      \PdfRender@ColorAvailable{%
        \PdfRender@TestBox{%
          \expandafter\PdfRender@TryColor\PdfRender@Color\ltx@empty
          \PdfRender@GetStrokeColor
          \ifx\PdfRender@StrokeColor\ltx@empty
            \@PackageWarning{pdfrender}{%
              Cannot extract stroke color\MessageBreak
              from value `#1'%
            }%
          \fi
        }%
      }%
    \fi
  \endgroup
}
%    \end{macrocode}
%    \begin{macro}{\PdfRender@ColorAvailable}
%    \begin{macrocode}
\def\PdfRender@ColorAvailable{%
  \@ifundefined{set@color}{%
    \@PackageError{pdfrender}{%
      Ignoring color options, because neither\MessageBreak
      package `color' nor package `xcolor' is loaded%
    }\@ehc
    \global\let\PdfRender@ColorAvailable\ltx@gobble
  }{%
    \global\let\PdfRender@ColorAvailable\ltx@firstofone
  }%
  \PdfRender@ColorAvailable
}
%    \end{macrocode}
%    \end{macro}
%    \begin{macro}{\PdfRender@TryColor}
%    \begin{macrocode}
\def\PdfRender@TryColor{%
  \@ifnextchar[\color\PdfRender@@TryColor
}
%    \end{macrocode}
%    \end{macro}
%    \begin{macro}{\PdfRender@@TryColor}
%    \begin{macrocode}
\def\PdfRender@@TryColor#1\ltx@empty{%
  \expandafter\color\expandafter{\PdfRender@Color}%
}
%    \end{macrocode}
%    \end{macro}
%    \begin{macro}{\PdfRender@SetColor}
%    \begin{macrocode}
\def\PdfRender@SetColor{%
  \chardef\PdfRender@NeedsCurrentColor=0 %
  \ifx\PdfRender@FillColor\ltx@empty
    \ifx\PdfRender@StrokeColor\ltx@empty
    \else
      \edef\PdfRender@CurrentColor{%
        \noexpand\PdfRender@FillColor\ltx@space\PdfRender@StrokeColor
      }%
      \chardef\PdfRender@NeedsCurrentColor=1 %
    \fi
  \else
    \ifx\PdfRender@StrokeColor\ltx@empty
      \edef\PdfRender@CurrentColor{%
        \PdfRender@FillColor\ltx@space\noexpand\PdfRender@StrokeColor
      }%
      \chardef\PdfRender@NeedsCurrentColor=2 %
    \else
      \edef\current@color{%
        \PdfRender@FillColor\ltx@space\PdfRender@StrokeColor
      }%
      \set@color
    \fi
  \fi
  \ifnum\PdfRender@NeedsCurrentColor=1 %
    \PdfRender@GetFillColor
    \ifx\PdfRender@FillColor\ltx@empty
      \@PackageWarning{pdfrender}{%
        Cannot extract current fill color%
      }%
    \else
      \edef\current@color{\PdfRender@CurrentColor}%
      \set@color
    \fi
  \else
    \ifnum\PdfRender@NeedsCurrentColor=2 %
      \PdfRender@GetStrokeColor
      \ifx\PdfRender@StrokeColor\ltx@empty
        \@PackageWarning{pdfrender}{%
          Cannot extract current stroke color%
        }%
      \else
        \edef\current@color{\PdfRender@CurrentColor}%
        \set@color
      \fi
    \fi
  \fi
}
%    \end{macrocode}
%    \end{macro}
%    \begin{macro}{\PdfRender@PatternFillColor}
%    \begin{macrocode}
\edef\PdfRender@PatternFillColor{ % space
  (%
    [0-9\string\.]+ g|%
    [0-9\string\.]+ [0-9\string\.]+ [0-9\string\.]+ rg|%
    [0-9\string\.]+ [0-9\string\.]+ %
    [0-9\string\.]+ [0-9\string\.]+ k%
  ) % space
  (.*)$%
}
%    \end{macrocode}
%    \end{macro}
%    \begin{macro}{\PdfRender@PatternStrokeColor}
%    \begin{macrocode}
\edef\PdfRender@PatternStrokeColor{ % space
  (%
    [0-9\string\.]+ G|%
    [0-9\string\.]+ [0-9\string\.]+ [0-9\string\.]+ RG|%
    [0-9\string\.]+ [0-9\string\.]+ %
    [0-9\string\.]+ [0-9\string\.]+ K%
  ) % space
  (.*)$%
}
%    \end{macrocode}
%    \end{macro}
%    \begin{macro}{\PdfRender@MatchPattern}
%    \begin{macrocode}
\def\PdfRender@MatchPattern#1{%
  \ifnum\pdfmatch{\PdfRender@Pattern}{\PdfRender@String}=1 %
    \xdef#1{%
      \expandafter\strip@prefix\pdflastmatch 1%
    }%
    \edef\PdfRender@String{%
      \expandafter\strip@prefix\pdflastmatch 2%
    }%
    \ifx\PdfRender@String\ltx@empty
    \else
      \expandafter\expandafter\expandafter\PdfRender@MatchPattern
      \expandafter\expandafter\expandafter#1%
    \fi
  \fi
}
%    \end{macrocode}
%    \end{macro}
%    \begin{macro}{\PdfRender@GetFillColor}
%    \begin{macrocode}
\def\PdfRender@GetFillColor{%
  \global\let\PdfRender@FillColor\ltx@empty
  \begingroup
    \ifPdfRender@Match
      \let\PdfRender@Pattern\PdfRender@PatternFillColor
      \edef\PdfRender@String{\ltx@space\current@color\ltx@space}%
      \PdfRender@MatchPattern\PdfRender@FillColor
    \else
      \edef\current@color{\current@color\ltx@space}%
      \let\PdfRender@OP\relax
      \PdfRender@FindOp{g}0%
      \PdfRender@FindOp{G}1%
      \PdfRender@FindOp{rg}0%
      \PdfRender@FindOp{RG}1%
      \PdfRender@FindOp{k}0%
      \PdfRender@FindOp{K}1%
      \PdfRender@FilterOp 0\PdfRender@FillColor
    \fi
  \endgroup
}
%    \end{macrocode}
%    \end{macro}
%    \begin{macro}{\PdfRender@GetStrokeColor}
%    \begin{macrocode}
\def\PdfRender@GetStrokeColor{%
  \global\let\PdfRender@StrokeColor\ltx@empty
  \begingroup
    \ifPdfRender@Match
      \let\PdfRender@Pattern\PdfRender@PatternStrokeColor
      \edef\PdfRender@String{\ltx@space\current@color\ltx@space}%
      \PdfRender@MatchPattern\PdfRender@StrokeColor
    \else
      \edef\current@color{\current@color\ltx@space}%
      \let\PdfRender@OP\relax
      \PdfRender@FindOp{g}0%
      \PdfRender@FindOp{G}1%
      \PdfRender@FindOp{rg}0%
      \PdfRender@FindOp{RG}1%
      \PdfRender@FindOp{k}0%
      \PdfRender@FindOp{K}1%
      \PdfRender@FilterOp 1\PdfRender@StrokeColor
    \fi
  \endgroup
}
%    \end{macrocode}
%    \end{macro}
%    \begin{macrocode}
\ifPdfRender@Match
  \expandafter\PdfRender@AtEnd
\fi%
%    \end{macrocode}
%    \begin{macro}{\PdfRender@FindOp}
%    \begin{macrocode}
\def\PdfRender@FindOp#1#2{%
  \def\PdfRender@temp##1 #1 ##2\@nil{%
    ##1%
    \ifx\\##2\\%
      \expandafter\@gobble
    \else
      \PdfRender@OP{#1}#2%
      \expandafter\@firstofone
    \fi
    {%
      \PdfRender@temp##2\@nil
    }%
  }%
  \edef\current@color{%
    \@firstofone{\expandafter\PdfRender@temp\current@color} #1 \@nil
  }%
}
%    \end{macrocode}
%    \end{macro}
%    \begin{macro}{\PdfRender@FilterOp}
%    \begin{macrocode}
\def\PdfRender@FilterOp#1#2{%
  \expandafter\PdfRender@@FilterOp\expandafter#1\expandafter#2%
      \current@color\PdfRender@OP{}{}%
}
%    \end{macrocode}
%    \end{macro}
%    \begin{macro}{\PdfRender@@FilterOp}
%    \begin{macrocode}
\def\PdfRender@@FilterOp#1#2#3\PdfRender@OP#4#5{%
  \ifx\\#4#5\\%
  \else
    \ifnum#1=#5 %
      \xdef#2{#3 #4}%
    \fi
    \expandafter\PdfRender@@FilterOp\expandafter#1\expandafter#2%
  \fi
}
%    \end{macrocode}
%    \end{macro}
%
%    \begin{macrocode}
\PdfRender@AtEnd%
%</package>
%    \end{macrocode}
%% \section{Installation}
%
% \subsection{Download}
%
% \paragraph{Package.} This package is available on
% CTAN\footnote{\CTANpkg{pdfrender}}:
% \begin{description}
% \item[\CTAN{macros/latex/contrib/oberdiek/pdfrender.dtx}] The source file.
% \item[\CTAN{macros/latex/contrib/oberdiek/pdfrender.pdf}] Documentation.
% \end{description}
%
%
% \paragraph{Bundle.} All the packages of the bundle `oberdiek'
% are also available in a TDS compliant ZIP archive. There
% the packages are already unpacked and the documentation files
% are generated. The files and directories obey the TDS standard.
% \begin{description}
% \item[\CTANinstall{install/macros/latex/contrib/oberdiek.tds.zip}]
% \end{description}
% \emph{TDS} refers to the standard ``A Directory Structure
% for \TeX\ Files'' (\CTANpkg{tds}). Directories
% with \xfile{texmf} in their name are usually organized this way.
%
% \subsection{Bundle installation}
%
% \paragraph{Unpacking.} Unpack the \xfile{oberdiek.tds.zip} in the
% TDS tree (also known as \xfile{texmf} tree) of your choice.
% Example (linux):
% \begin{quote}
%   |unzip oberdiek.tds.zip -d ~/texmf|
% \end{quote}
%
% \subsection{Package installation}
%
% \paragraph{Unpacking.} The \xfile{.dtx} file is a self-extracting
% \docstrip\ archive. The files are extracted by running the
% \xfile{.dtx} through \plainTeX:
% \begin{quote}
%   \verb|tex pdfrender.dtx|
% \end{quote}
%
% \paragraph{TDS.} Now the different files must be moved into
% the different directories in your installation TDS tree
% (also known as \xfile{texmf} tree):
% \begin{quote}
% \def\t{^^A
% \begin{tabular}{@{}>{\ttfamily}l@{ $\rightarrow$ }>{\ttfamily}l@{}}
%   pdfrender.sty & tex/generic/oberdiek/pdfrender.sty\\
%   pdfrender.pdf & doc/latex/oberdiek/pdfrender.pdf\\
%   pdfrender.dtx & source/latex/oberdiek/pdfrender.dtx\\
% \end{tabular}^^A
% }^^A
% \sbox0{\t}^^A
% \ifdim\wd0>\linewidth
%   \begingroup
%     \advance\linewidth by\leftmargin
%     \advance\linewidth by\rightmargin
%   \edef\x{\endgroup
%     \def\noexpand\lw{\the\linewidth}^^A
%   }\x
%   \def\lwbox{^^A
%     \leavevmode
%     \hbox to \linewidth{^^A
%       \kern-\leftmargin\relax
%       \hss
%       \usebox0
%       \hss
%       \kern-\rightmargin\relax
%     }^^A
%   }^^A
%   \ifdim\wd0>\lw
%     \sbox0{\small\t}^^A
%     \ifdim\wd0>\linewidth
%       \ifdim\wd0>\lw
%         \sbox0{\footnotesize\t}^^A
%         \ifdim\wd0>\linewidth
%           \ifdim\wd0>\lw
%             \sbox0{\scriptsize\t}^^A
%             \ifdim\wd0>\linewidth
%               \ifdim\wd0>\lw
%                 \sbox0{\tiny\t}^^A
%                 \ifdim\wd0>\linewidth
%                   \lwbox
%                 \else
%                   \usebox0
%                 \fi
%               \else
%                 \lwbox
%               \fi
%             \else
%               \usebox0
%             \fi
%           \else
%             \lwbox
%           \fi
%         \else
%           \usebox0
%         \fi
%       \else
%         \lwbox
%       \fi
%     \else
%       \usebox0
%     \fi
%   \else
%     \lwbox
%   \fi
% \else
%   \usebox0
% \fi
% \end{quote}
% If you have a \xfile{docstrip.cfg} that configures and enables \docstrip's
% TDS installing feature, then some files can already be in the right
% place, see the documentation of \docstrip.
%
% \subsection{Refresh file name databases}
%
% If your \TeX~distribution
% (\TeX\,Live, \mikTeX, \dots) relies on file name databases, you must refresh
% these. For example, \TeX\,Live\ users run \verb|texhash| or
% \verb|mktexlsr|.
%
% \subsection{Some details for the interested}
%
% \paragraph{Unpacking with \LaTeX.}
% The \xfile{.dtx} chooses its action depending on the format:
% \begin{description}
% \item[\plainTeX:] Run \docstrip\ and extract the files.
% \item[\LaTeX:] Generate the documentation.
% \end{description}
% If you insist on using \LaTeX\ for \docstrip\ (really,
% \docstrip\ does not need \LaTeX), then inform the autodetect routine
% about your intention:
% \begin{quote}
%   \verb|latex \let\install=y% \iffalse meta-comment
%
% File: pdfrender.dtx
% Version: 2019/12/29 v1.6
% Info: Access to some PDF graphics parameters
%
% Copyright (C)
%    2010 Heiko Oberdiek
%    2016-2019 Oberdiek Package Support Group
%    https://github.com/ho-tex/oberdiek/issues
%
% This work may be distributed and/or modified under the
% conditions of the LaTeX Project Public License, either
% version 1.3c of this license or (at your option) any later
% version. This version of this license is in
%    https://www.latex-project.org/lppl/lppl-1-3c.txt
% and the latest version of this license is in
%    https://www.latex-project.org/lppl.txt
% and version 1.3 or later is part of all distributions of
% LaTeX version 2005/12/01 or later.
%
% This work has the LPPL maintenance status "maintained".
%
% The Current Maintainers of this work are
% Heiko Oberdiek and the Oberdiek Package Support Group
% https://github.com/ho-tex/oberdiek/issues
%
% The Base Interpreter refers to any `TeX-Format',
% because some files are installed in TDS:tex/generic//.
%
% This work consists of the main source file pdfrender.dtx
% and the derived files
%    pdfrender.sty, pdfrender.pdf, pdfrender.ins, pdfrender.drv,
%    pdfrender-test1.tex, pdfrender-test2.tex,
%    pdfrender-test3.tex, pdfrender-test4.tex,
%    pdfrender-test5.tex.
%
% Distribution:
%    CTAN:macros/latex/contrib/oberdiek/pdfrender.dtx
%    CTAN:macros/latex/contrib/oberdiek/pdfrender.pdf
%
% Unpacking:
%    (a) If pdfrender.ins is present:
%           tex pdfrender.ins
%    (b) Without pdfrender.ins:
%           tex pdfrender.dtx
%    (c) If you insist on using LaTeX
%           latex \let\install=y% \iffalse meta-comment
%
% File: pdfrender.dtx
% Version: 2019/12/29 v1.6
% Info: Access to some PDF graphics parameters
%
% Copyright (C)
%    2010 Heiko Oberdiek
%    2016-2019 Oberdiek Package Support Group
%    https://github.com/ho-tex/oberdiek/issues
%
% This work may be distributed and/or modified under the
% conditions of the LaTeX Project Public License, either
% version 1.3c of this license or (at your option) any later
% version. This version of this license is in
%    https://www.latex-project.org/lppl/lppl-1-3c.txt
% and the latest version of this license is in
%    https://www.latex-project.org/lppl.txt
% and version 1.3 or later is part of all distributions of
% LaTeX version 2005/12/01 or later.
%
% This work has the LPPL maintenance status "maintained".
%
% The Current Maintainers of this work are
% Heiko Oberdiek and the Oberdiek Package Support Group
% https://github.com/ho-tex/oberdiek/issues
%
% The Base Interpreter refers to any `TeX-Format',
% because some files are installed in TDS:tex/generic//.
%
% This work consists of the main source file pdfrender.dtx
% and the derived files
%    pdfrender.sty, pdfrender.pdf, pdfrender.ins, pdfrender.drv,
%    pdfrender-test1.tex, pdfrender-test2.tex,
%    pdfrender-test3.tex, pdfrender-test4.tex,
%    pdfrender-test5.tex.
%
% Distribution:
%    CTAN:macros/latex/contrib/oberdiek/pdfrender.dtx
%    CTAN:macros/latex/contrib/oberdiek/pdfrender.pdf
%
% Unpacking:
%    (a) If pdfrender.ins is present:
%           tex pdfrender.ins
%    (b) Without pdfrender.ins:
%           tex pdfrender.dtx
%    (c) If you insist on using LaTeX
%           latex \let\install=y\input{pdfrender.dtx}
%        (quote the arguments according to the demands of your shell)
%
% Documentation:
%    (a) If pdfrender.drv is present:
%           latex pdfrender.drv
%    (b) Without pdfrender.drv:
%           latex pdfrender.dtx; ...
%    The class ltxdoc loads the configuration file ltxdoc.cfg
%    if available. Here you can specify further options, e.g.
%    use A4 as paper format:
%       \PassOptionsToClass{a4paper}{article}
%
%    Programm calls to get the documentation (example):
%       pdflatex pdfrender.dtx
%       makeindex -s gind.ist pdfrender.idx
%       pdflatex pdfrender.dtx
%       makeindex -s gind.ist pdfrender.idx
%       pdflatex pdfrender.dtx
%
% Installation:
%    TDS:tex/generic/oberdiek/pdfrender.sty
%    TDS:doc/latex/oberdiek/pdfrender.pdf
%    TDS:source/latex/oberdiek/pdfrender.dtx
%
%<*ignore>
\begingroup
  \catcode123=1 %
  \catcode125=2 %
  \def\x{LaTeX2e}%
\expandafter\endgroup
\ifcase 0\ifx\install y1\fi\expandafter
         \ifx\csname processbatchFile\endcsname\relax\else1\fi
         \ifx\fmtname\x\else 1\fi\relax
\else\csname fi\endcsname
%</ignore>
%<*install>
\input docstrip.tex
\Msg{************************************************************************}
\Msg{* Installation}
\Msg{* Package: pdfrender 2019/12/29 v1.6 Access to some PDF graphics parameters (HO)}
\Msg{************************************************************************}

\keepsilent
\askforoverwritefalse

\let\MetaPrefix\relax
\preamble

This is a generated file.

Project: pdfrender
Version: 2019/12/29 v1.6

Copyright (C)
   2010 Heiko Oberdiek
   2016-2019 Oberdiek Package Support Group

This work may be distributed and/or modified under the
conditions of the LaTeX Project Public License, either
version 1.3c of this license or (at your option) any later
version. This version of this license is in
   https://www.latex-project.org/lppl/lppl-1-3c.txt
and the latest version of this license is in
   https://www.latex-project.org/lppl.txt
and version 1.3 or later is part of all distributions of
LaTeX version 2005/12/01 or later.

This work has the LPPL maintenance status "maintained".

The Current Maintainers of this work are
Heiko Oberdiek and the Oberdiek Package Support Group
https://github.com/ho-tex/oberdiek/issues


The Base Interpreter refers to any `TeX-Format',
because some files are installed in TDS:tex/generic//.

This work consists of the main source file pdfrender.dtx
and the derived files
   pdfrender.sty, pdfrender.pdf, pdfrender.ins, pdfrender.drv,
   pdfrender-test1.tex, pdfrender-test2.tex,
   pdfrender-test3.tex, pdfrender-test4.tex,
   pdfrender-test5.tex.

\endpreamble
\let\MetaPrefix\DoubleperCent

\generate{%
  \file{pdfrender.ins}{\from{pdfrender.dtx}{install}}%
  \file{pdfrender.drv}{\from{pdfrender.dtx}{driver}}%
  \usedir{tex/generic/oberdiek}%
  \file{pdfrender.sty}{\from{pdfrender.dtx}{package}}%
%  \usedir{doc/latex/oberdiek/test}%
%  \file{pdfrender-test1.tex}{\from{pdfrender.dtx}{test1}}%
%  \file{pdfrender-test2.tex}{\from{pdfrender.dtx}{test2}}%
%  \file{pdfrender-test3.tex}{\from{pdfrender.dtx}{test3}}%
%  \file{pdfrender-test4.tex}{\from{pdfrender.dtx}{test4}}%
%  \file{pdfrender-test5.tex}{\from{pdfrender.dtx}{test5}}%
}

\catcode32=13\relax% active space
\let =\space%
\Msg{************************************************************************}
\Msg{*}
\Msg{* To finish the installation you have to move the following}
\Msg{* file into a directory searched by TeX:}
\Msg{*}
\Msg{*     pdfrender.sty}
\Msg{*}
\Msg{* To produce the documentation run the file `pdfrender.drv'}
\Msg{* through LaTeX.}
\Msg{*}
\Msg{* Happy TeXing!}
\Msg{*}
\Msg{************************************************************************}

\endbatchfile
%</install>
%<*ignore>
\fi
%</ignore>
%<*driver>
\NeedsTeXFormat{LaTeX2e}
\ProvidesFile{pdfrender.drv}%
  [2019/12/29 v1.6 Access to some PDF graphics parameters (HO)]%
\documentclass{ltxdoc}
\usepackage{holtxdoc}[2011/11/22]
\usepackage{longtable}
\renewcommand*{\arraystretch}{1.2}
\begin{document}
  \DocInput{pdfrender.dtx}%
\end{document}
%</driver>
% \fi
%
%
%
% \GetFileInfo{pdfrender.drv}
%
% \title{The \xpackage{pdfrender} package}
% \date{2019/12/29 v1.6}
% \author{Heiko Oberdiek\thanks
% {Please report any issues at \url{https://github.com/ho-tex/oberdiek/issues}}}
%
% \maketitle
%
% \begin{abstract}
% The PDF format has some graphics parameter like
% line width or text rendering mode. This package
% provides an interface for setting these parameters.
% \end{abstract}
%
% \tableofcontents
%
% \section{Documentation}
%
% This package \xpackage{pdfrender} defines an interface for
% PDF specific parameters that affects the rendering of
% graphics or text. The interface and its implementation
% uses the same technique as package \xpackage{color}
% for color settings. Therefore this package is loaded to
% enable \LaTeX's color interface.
%
% At different places \LaTeX\ uses \cs{normalcolor} to avoid that
% header, footer or floats are print in the current color of the
% main text. \cs{setgroup@color} is used to start a save box
% with the color that is set at box saving time.
% Package \xpackage{pdfrender} extends these macros to
% add its own hooks of its parameters. Therefore \LaTeX3 should
% generalize \LaTeXe's color interface.
%
% \subsection{Usage}
%
% In \LaTeX\ the package is loaded as normal package. Options
% are not defined for this package.
% \begin{quote}
% |\usepackage{pdfrender}|
% \end{quote}
%
% This package can also be used in \plainTeX\ and even ini\TeX:
% \begin{quote}
% |input pdfrender.sty|
% \end{quote}
%
% \subsection{Macros}
%
% \begin{declcs}{pdfrender} \M{key value list}
% \end{declcs}
% The first parameter \meta{key value list} contains a
% list of parameter settings. The key entry is the parameter name.
% The macro works like
% \cs{color} (without optional argument) for color setting.
%
% \begin{declcs}{textpdfrender} \M{key value list} \M{text}
% \end{declcs}
% In the same way as \cs{pdfrender} the first argument
% specifies the parameters that should be set. This parameter
% setting affects \meta{text} only. Basically it works
% the same way as \cs{textcolor} (without optional argument).
%
% \subsection{Parameters}
%
% The following table shows an overview for the supported
% parameters and values:
%
% \begin{longtable}{@{}l|l|l@{}}
% \bfseries Parameter & \bfseries Value & \bfseries Alias\\*
% \hline
% \endhead
% \xoption{TextRenderingMode}
%   & \texttt{0} & \texttt{Fill}\\*
%   & \texttt{1} & \texttt{Stroke}\\*
%   & \texttt{2} & \texttt{FillStroke}\\*
%   & \texttt{3} & \texttt{Invisible}\\*
%   & \texttt{4} & \texttt{FillClip}\\*
%   & \texttt{5} & \texttt{StrokeClip}\\*
%   & \texttt{6} & \texttt{FillStrokeClip}\\*
%   & \texttt{7} & \texttt{Clip}\\
% \hline
% \xoption{LineWidth}
%   & \itshape positive number, & \itshape \TeX\ dimen\\*[-.2ex]
%   & \qquad \itshape unit is bp & \\
% \hline
% \xoption{LineCapStyle}
%   & \texttt{0} & \texttt{Butt}\\*
%   & \texttt{1} & \texttt{Round}\\*
%   & \texttt{2} & \texttt{ProjectingSquare}\\
% \hline
% \xoption{LineJoinStyle}
%   & \texttt{0} & \texttt{Miter}\\*
%   & \texttt{1} & \texttt{Round}\\*
%   & \texttt{2} & \texttt{Bevel}\\
% \hline
% \xoption{MiterLimit}
%   & \multicolumn{2}{l@{}}{\itshape positive number} \\
% \hline
% \xoption{Flatness}
%   & \multicolumn{2}{l@{}}{\itshape number between 0 and 100} \\
% \hline
% \xoption{LineDashPattern}
%   & \multicolumn{2}{l@{}}{^^A
%       \itshape numbers in square brackets,^^A
%     } \\*[-.2ex]
%   & \multicolumn{2}{l@{}}{^^A
%       \qquad \itshape followed by number,^^A
%     } \\*
%   & \multicolumn{2}{l@{}}{^^A
%       \qquad \itshape units are bp^^A
%     } \\
% \hline
% \xoption{RenderingIntent}
%   & \multicolumn{2}{l@{}}{\texttt{AbsoluteColorimetric}} \\*
%   & \multicolumn{2}{l@{}}{\texttt{RelativeColorimetric}} \\*
%   & \multicolumn{2}{l@{}}{\texttt{Saturation}} \\*
%   & \multicolumn{2}{l@{}}{\texttt{Perceptual}} \\
% \hline
% \xoption{FillColor}
%   & & \itshape color specification \\
% \hline
% \xoption{StrokeColor}
%   & & \itshape color specification \\
% \end{longtable}
%
% \subsubsection{Details}
%
% The description and specification of these parameters
% are available in the PDF specification \cite{pdf-spec}.
% Therefore they are not repeated here.
%
% \begin{description}
% \item[Value:]
%   The values in the second column lists or describe
%   the values that are specified by the PDF specification.
% \item[Alias:] Instead of magic numbers the package also
%   defines some aliases that can be given as value.
%   Example: |LineCapStyle=Round| has the same effect as
%   |LineCapStyle=1|.
% \item[Number:] The term \emph{number} means an integer
%   or real number. The real number is given as plain
%   decimal number without exponent. The decimal separator is a period.
%   At least one digit must be present.
% \item[\xoption{LineWidth}:] As alias a \TeX\ dimen specification
%   can be given. This includes explicit specifications with number
%   and unit, e.g. |LineWidth=0.5pt|. Also \LaTeX\ length registers
%   may be used. If \eTeX's \cs{dimexpr} is available, then it is
%   automatically added. However package \xpackage{calc} is not
%   supported.
% \item[\xoption{FillColor}, \xoption{StrokeColor}:]
%   Package \xpackage{color} or \xpackage{xcolor} must be loaded
%   before these options can be used (since version 1.2).
%   \LaTeX's color support sets both colors at the same time
%   to the same value. However parameter \xoption{TextRenderingMode}
%   offers the value \texttt{FillStroke} that makes only sense, if
%   the two color types can be set separately.
%   If one of the options \xoption{FillColor} or \xoption{StrokeColor}
%   is specified, then also the color is set. For compatibility
%   with the \LaTeX\ color packages (\xpackage{color} or \xpackage{xcolor}),
%   always both colors must be set. Thus if one of them is not specified,
%   it is taken from the current color.
%
%   Both options \xoption{FillColor} and \xoption{StrokeColor}
%   expect a \LaTeX\ color specification as value.
%   Also the optional color model argument is supported.
%   Example:
%   \begin{quote}
%     |FillColor=yellow|,\\
%     |StrokeColor=[cmyk]{1,.5,0,0}|\\
%   \end{quote}
% \end{description}
%
% \subsection{Color stack}
%
% If the \pdfTeX\ version provides color stacks, then each
% parameter is assigned a page based color stack. The assignment
% of a stack takes place, when its parameter is set the first time.
% This avoids the use of color stacks that are not needed.
%
% \StopEventually{
% }
%
% \section{Implementation}
%    \begin{macrocode}
%<*package>
%    \end{macrocode}
%    Reload check, especially if the package is not used with \LaTeX.
%    \begin{macrocode}
\begingroup\catcode61\catcode48\catcode32=10\relax%
  \catcode13=5 % ^^M
  \endlinechar=13 %
  \catcode35=6 % #
  \catcode39=12 % '
  \catcode44=12 % ,
  \catcode45=12 % -
  \catcode46=12 % .
  \catcode58=12 % :
  \catcode64=11 % @
  \catcode123=1 % {
  \catcode125=2 % }
  \expandafter\let\expandafter\x\csname ver@pdfrender.sty\endcsname
  \ifx\x\relax % plain-TeX, first loading
  \else
    \def\empty{}%
    \ifx\x\empty % LaTeX, first loading,
      % variable is initialized, but \ProvidesPackage not yet seen
    \else
      \expandafter\ifx\csname PackageInfo\endcsname\relax
        \def\x#1#2{%
          \immediate\write-1{Package #1 Info: #2.}%
        }%
      \else
        \def\x#1#2{\PackageInfo{#1}{#2, stopped}}%
      \fi
      \x{pdfrender}{The package is already loaded}%
      \aftergroup\endinput
    \fi
  \fi
\endgroup%
%    \end{macrocode}
%    Package identification:
%    \begin{macrocode}
\begingroup\catcode61\catcode48\catcode32=10\relax%
  \catcode13=5 % ^^M
  \endlinechar=13 %
  \catcode35=6 % #
  \catcode39=12 % '
  \catcode40=12 % (
  \catcode41=12 % )
  \catcode44=12 % ,
  \catcode45=12 % -
  \catcode46=12 % .
  \catcode47=12 % /
  \catcode58=12 % :
  \catcode64=11 % @
  \catcode91=12 % [
  \catcode93=12 % ]
  \catcode123=1 % {
  \catcode125=2 % }
  \expandafter\ifx\csname ProvidesPackage\endcsname\relax
    \def\x#1#2#3[#4]{\endgroup
      \immediate\write-1{Package: #3 #4}%
      \xdef#1{#4}%
    }%
  \else
    \def\x#1#2[#3]{\endgroup
      #2[{#3}]%
      \ifx#1\@undefined
        \xdef#1{#3}%
      \fi
      \ifx#1\relax
        \xdef#1{#3}%
      \fi
    }%
  \fi
\expandafter\x\csname ver@pdfrender.sty\endcsname
\ProvidesPackage{pdfrender}%
  [2019/12/29 v1.6 Access to some PDF graphics parameters (HO)]%
%    \end{macrocode}
%
%    \begin{macrocode}
\begingroup\catcode61\catcode48\catcode32=10\relax%
  \catcode13=5 % ^^M
  \endlinechar=13 %
  \catcode123=1 % {
  \catcode125=2 % }
  \catcode64=11 % @
  \def\x{\endgroup
    \expandafter\edef\csname PdfRender@AtEnd\endcsname{%
      \endlinechar=\the\endlinechar\relax
      \catcode13=\the\catcode13\relax
      \catcode32=\the\catcode32\relax
      \catcode35=\the\catcode35\relax
      \catcode61=\the\catcode61\relax
      \catcode64=\the\catcode64\relax
      \catcode123=\the\catcode123\relax
      \catcode125=\the\catcode125\relax
    }%
  }%
\x\catcode61\catcode48\catcode32=10\relax%
\catcode13=5 % ^^M
\endlinechar=13 %
\catcode35=6 % #
\catcode64=11 % @
\catcode123=1 % {
\catcode125=2 % }
\def\TMP@EnsureCode#1#2{%
  \edef\PdfRender@AtEnd{%
    \PdfRender@AtEnd
    \catcode#1=\the\catcode#1\relax
  }%
  \catcode#1=#2\relax
}
\TMP@EnsureCode{10}{12}% ^^J
\TMP@EnsureCode{36}{3}% $
\TMP@EnsureCode{39}{12}% '
\TMP@EnsureCode{40}{12}% (
\TMP@EnsureCode{41}{12}% )
\TMP@EnsureCode{42}{12}% *
\TMP@EnsureCode{43}{12}% +
\TMP@EnsureCode{44}{12}% ,
\TMP@EnsureCode{45}{12}% -
\TMP@EnsureCode{46}{12}% .
\TMP@EnsureCode{47}{12}% /
\TMP@EnsureCode{58}{12}% :
\TMP@EnsureCode{59}{12}% ;
\TMP@EnsureCode{60}{12}% <
\TMP@EnsureCode{62}{12}% >
\TMP@EnsureCode{63}{12}% ?
\TMP@EnsureCode{91}{12}% [
\TMP@EnsureCode{93}{12}% ]
\TMP@EnsureCode{94}{7}% ^ (superscript)
\TMP@EnsureCode{96}{12}% `
\TMP@EnsureCode{124}{12}% |
%    \end{macrocode}
%
%    \begin{macrocode}
\def\PdfRender@AtEndHook{}
\expandafter\def\expandafter\PdfRender@AtEnd\expandafter{%
  \expandafter\PdfRender@AtEndHook
  \PdfRender@AtEnd
  \endinput
}
%    \end{macrocode}
%
% \subsection{Look for \pdfTeX, its mode and features}
%
%    \begin{macro}{\PdfRender@newif}
%    \begin{macrocode}
\def\PdfRender@newif#1{%
  \expandafter\edef\csname PdfRender@#1true\endcsname{%
    \let
    \expandafter\noexpand\csname ifPdfRender@#1\endcsname
    \noexpand\iftrue
  }%
  \expandafter\edef\csname PdfRender@#1false\endcsname{%
    \let
    \expandafter\noexpand\csname ifPdfRender@#1\endcsname
    \noexpand\iffalse
  }%
  \csname PdfRender@#1false\endcsname
}
%    \end{macrocode}
%    \end{macro}
%
%    \begin{macro}{\ifPdfRender@Stack}
%    \begin{macrocode}
\PdfRender@newif{Stack}
%    \end{macrocode}
%    \end{macro}
%    \begin{macro}{\ifPdfRender@Match}
%    \begin{macrocode}
\PdfRender@newif{Match}
%    \end{macrocode}
%    \end{macro}
%
%    \begin{macro}{\PdfRender@RequirePackage}
%    \begin{macrocode}
\begingroup\expandafter\expandafter\expandafter\endgroup
\expandafter\ifx\csname RequirePackage\endcsname\relax
  \def\PdfRender@RequirePackage#1[#2]{%
    \expandafter\def\expandafter\PdfRender@AtEndHook\expandafter{%
      \PdfRender@AtEndHook
      \ltx@ifpackagelater{#1}{#2}{}{%
        \@PackageWarningNoLine{pdfrender}{%
          You have requested version\MessageBreak
          `#2' of package `#1',\MessageBreak
          but only version\MessageBreak
          `\csname ver@#1.\ltx@pkgextension\endcsname'\MessageBreak
          is available%
        }%
      }%
    }%
    \input #1.sty\relax
  }%
\else
  \let\PdfRender@RequirePackage\RequirePackage
\fi
%    \end{macrocode}
%    \end{macro}
% Luatex compatibility
%    \begin{macrocode}
\ifx\pdfextension\@undefined\else
  \def\pdfcolorstackinit {\pdffeedback colorstackinit}
  \protected\def\pdfcolorstack     {\pdfextension colorstack}
  \protected\def\pdfliteral        {\pdfextension literal}
\fi
%    \end{macrocode}
%
%    \begin{macrocode}
\PdfRender@RequirePackage{iftex}[2019/11/07]
\PdfRender@RequirePackage{infwarerr}[2007/09/09]
\PdfRender@RequirePackage{ltxcmds}[2010/01/28]
%    \end{macrocode}
%
%    \begin{macrocode}
\ifpdf
  \ltx@IfUndefined{pdfcolorstackinit}{%
    \@PackageWarning{pdfrender}{%
      Missing \string\pdfcolorstackinit
    }%
  }{%
    \PdfRender@Stacktrue
  }%
  \ltx@IfUndefined{pdfmatch}{%
    \@PackageInfoNoLine{pdfrender}{%
      \string\pdfmatch\ltx@space not found. %
      Therefore the values\MessageBreak
      of some parameters are not validated%
    }%
  }{%
    \PdfRender@Matchtrue
  }%
\else
  \@PackageWarning{pdfrender}{%
    Missing pdfTeX in PDF mode%
  }%
%    \end{macrocode}
%    \begin{macrocode}
  \ltx@IfUndefined{newcommand}{%
%    \end{macrocode}
%    \begin{macro}{\pdfrender}
%    \begin{macrocode}
    \def\pdfrender#1{}%
%    \end{macrocode}
%    \end{macro}
%    \begin{macro}{\textpdfrender}
%    \begin{macrocode}
    \long\def\textpdfrender#1#2{#2}%
%    \end{macrocode}
%    \end{macro}
%    \begin{macrocode}
  }{%
%    \end{macrocode}
%    \begin{macro}{\pdfrender}
%    \begin{macrocode}
    \newcommand*{\pdfrender}[1]{}%
%    \end{macrocode}
%    \end{macro}
%    \begin{macro}{\textpdfrender}
%    \begin{macrocode}
    \newcommand{\textpdfrender}[2]{#2}%
%    \end{macrocode}
%    \end{macro}
%    \begin{macrocode}
  }%
  \expandafter\PdfRender@AtEnd
\fi%
%    \end{macrocode}
%
% \subsection{Enable color support of \LaTeX}
%
%    \begin{macrocode}
\ltx@ifpackageloaded{color}{}{%
  \def\color@setgroup{\begingroup\set@color}%
  \let\color@begingroup\begingroup
  \def\color@endgroup{\endgraf\endgroup}%
  \def\color@hbox{\hbox\bgroup\color@begingroup}%
  \def\color@vbox{\vbox\bgroup\color@begingroup}%
  \def\color@endbox{\color@endgroup\egroup}%
  \ltx@ifundefined{bgroup}{%
    \let\bgroup={\let\egroup=}%
  }{}%
  \ltx@ifundefined{endgraf}{%
    \let\endgraf=\par
  }{}%
}
%    \end{macrocode}
%
% \subsection{Hook into \cs{normalcolor}}
%
%    The problem is that packages \xpackage{color} and \xpackage{xcolor}
%    each overwrite \cs{normalcolor}. For example, after the package
%    loading order \xpackage{color}, \xpackage{pdfrender} and \xpackage{xcolor}
%    the patched version of \cs{normalcolor} is overwritten by
%    package \xpackage{xcolor}. Also using \cs{AtBeginDocument} for
%    patching is not enough. If package \xpackage{hyperref} is loaded
%    later, it might load package \xpackage{color} using \cs{AtBeginDocument}.
%    \begin{macro}{\PdfRender@NormalColorHook}
%    \begin{macrocode}
\def\PdfRender@NormalColorHook{}
%    \end{macrocode}
%    \end{macro}
%    \begin{macro}{\PdfRender@ColorSetGroupHook}
%    \begin{macrocode}
\def\PdfRender@ColorSetGroupHook{}
%    \end{macrocode}
%    \end{macro}
%    \begin{macro}{\PdfRender@TestBox}
%    \begin{macrocode}
\def\PdfRender@TestBox#1{%
  \setbox0=\color@hbox#1\color@endbox
}
%    \end{macrocode}
%    \end{macro}
%    \begin{macro}{\PdfRender@PatchNormalColor}
%    \begin{macrocode}
\def\PdfRender@PatchNormalColor{%
  \ltx@ifundefined{normalcolor}{%
    \gdef\normalcolor{\PdfRender@NormalColorHook}%
  }{%
    \begingroup
      \def\PdfRender@NormalColorHook{\let\PdfRender@temp=Y}%
      \PdfRender@TestBox{%
        \let\set@color\relax
        \normalcolor
        \ifx\PdfRender@temp Y%
        \else
          \ltx@GlobalAppendToMacro\normalcolor{%
            \PdfRender@NormalColorHook
          }%
        \fi
      }%
    \endgroup
  }%
  \ifx\@nodocument\relax
    \global\let\PdfRender@PatchNormalColor\relax
  \fi
}%
%    \end{macrocode}
%    \end{macro}
%    \begin{macro}{\PdfRender@PatchColorSetGroup}
%    \begin{macrocode}
\def\PdfRender@PatchColorSetGroup{%
  \begingroup
    \def\PdfRender@ColorSetGroupHook{\let\PdfRender@temp=Y}%
    \PdfRender@TestBox{%
      \let\set@color\relax
      \color@setgroup\color@endgroup
      \ifx\PdfRender@temp Y%
      \else
        \ltx@GlobalAppendToMacro\color@setgroup{%
          \PdfRender@ColorSetGroupHook
        }%
      \fi
    }%
  \endgroup
  \ifx\@nodocument\relax
    \global\let\PdfRender@PatchColorSetGroup\relax
  \fi
}%
%    \end{macrocode}
%    \end{macro}
%    \begin{macro}{\PdfRender@PatchColor}
%    \begin{macrocode}
\def\PdfRender@PatchColor{%
  \PdfRender@PatchNormalColor
  \PdfRender@PatchColorSetGroup
}
%    \end{macrocode}
%    \end{macro}
%    \begin{macrocode}
\PdfRender@PatchColor
\ltx@IfUndefined{AtBeginDocument}{}{%
  \AtBeginDocument{\PdfRender@PatchColor}%
}
%    \end{macrocode}
%    \cs{AfterPackage} is provided by package \xpackage{scrlfile}.
%    \begin{macrocode}
\ltx@IfUndefined{AfterPackage}{%
}{%
  \AfterPackage{color}{\PdfRender@PatchColor}%
  \AfterPackage{xcolor}{\PdfRender@PatchColor}%
  \AfterPackage{etoolbox}{%
    \AfterEndPreamble{\PdfRender@PatchColor}%
  }%
}%
%    \end{macrocode}
%    \cs{AfterEndPreamble} is provided by package \xpackage{etoolbox}.
%    \begin{macrocode}
\ltx@IfUndefined{AfterEndPreamble}{%
}{%
  \AfterEndPreamble{\PdfRender@PatchColor}%
}%
%    \end{macrocode}
%
%    \begin{macrocode}
\PdfRender@RequirePackage{kvsetkeys}[2010/01/28]
%    \end{macrocode}
%    \begin{macro}{\PdfRender@texorpdfstring}
%    \begin{macrocode}
\def\PdfRender@texorpdfstring{%
  \ltx@IfUndefined{texorpdfstring}\ltx@firstoftwo\texorpdfstring
}
%    \end{macrocode}
%    \end{macro}
%    \begin{macro}{\pdfrender}
%    \begin{macrocode}
\ltx@IfUndefined{DeclareRobustCommand}%
\ltx@firstoftwo\ltx@secondoftwo
{%
  \def\pdfrender#1%
}{%
  \newcommand{\pdfrender}{}%
  \DeclareRobustCommand*{\pdfrender}[1]%
}%
{%
  \PdfRender@texorpdfstring{%
    \PdfRender@PatchNormalColor
    \global\let\PdfRender@FillColor\ltx@empty
    \global\let\PdfRender@StrokeColor\ltx@empty
    \kvsetkeys{PDFRENDER}{#1}%
    \PdfRender@SetColor
  }{}%
}
%    \end{macrocode}
%    \end{macro}
%    \begin{macro}{\textpdfrender}
%    \begin{macrocode}
\ltx@IfUndefined{DeclareRobustCommand}%
\ltx@firstoftwo\ltx@secondoftwo
{%
  \long\def\textpdfrender#1#2%
}{%
  \newcommand{\textpdfrender}{}%
  \DeclareRobustCommand{\textpdfrender}[2]%
}%
{%
  \PdfRender@texorpdfstring{%
    \begingroup
      \pdfrender{#1}%
      #2%
    \endgroup
  }{#2}%
}
%    \end{macrocode}
%    \end{macro}
%
%    \begin{macro}{\ifPdfRender@Values}
%    \begin{macrocode}
\PdfRender@newif{Values}
%    \end{macrocode}
%    \end{macro}
%    \begin{macro}{\PdfRender@NewClassValues}
%    \begin{macrocode}
\def\PdfRender@NewClassValues#1#2#3#4{%
  \PdfRender@Valuestrue
  \PdfRender@NewClass{#1}{#2}{#3}{#4}{}%
}
%    \end{macrocode}
%    \end{macro}
%    \begin{macro}{\PdfRender@NewClass}
%    \begin{macrocode}
\def\PdfRender@NewClass#1#2#3#4#5{%
  \PdfRender@newif{Active#1}%
  \expandafter\def\csname PdfRender@Default#1\endcsname{#2}%
  \expandafter\let\csname PdfRender@Current#1\expandafter\endcsname
      \csname PdfRender@Default#1\endcsname
  \ifPdfRender@Stack
    \expandafter\edef\csname PdfRender@Init#1\endcsname{%
      \global\chardef
      \expandafter\noexpand\csname PdfRender@Stack#1\endcsname=%
          \noexpand\pdfcolorstackinit page direct{%
            \noexpand#3%
            \expandafter\noexpand\csname PdfRender@Default#1\endcsname
          }\relax
      \noexpand\@PackageInfo{pdfrender}{%
        New color stack `#1' = \noexpand\number
        \expandafter\noexpand\csname PdfRender@Stack#1\endcsname
      }%
      \gdef\expandafter\noexpand\csname PdfRender@Init#1\endcsname{}%
    }%
    \expandafter\edef\csname PdfRender@Set#1\endcsname{%
      \expandafter\noexpand\csname PdfRender@Init#1\endcsname
      \noexpand\pdfcolorstack
      \expandafter\noexpand\csname PdfRender@Stack#1\endcsname
      push{%
        #3{\expandafter\noexpand\csname PdfRender@Current#1\endcsname}%
      }%
      \noexpand\aftergroup
      \expandafter\noexpand\csname PdfRender@Reset#1\endcsname
    }%
    \expandafter\edef\csname PdfRender@Reset#1\endcsname{%
      \expandafter\noexpand\csname PdfRender@Init#1\endcsname
      \noexpand\pdfcolorstack
      \expandafter\noexpand\csname PdfRender@Stack#1\endcsname
      pop\relax
    }%
  \else
    \expandafter\edef\csname PdfRender@Set#1\endcsname{%
      \noexpand\pdfliteral direct{%
        #3{\expandafter\noexpand\csname PdfRender@Current#1\endcsname}%
      }%
      \noexpand\aftergroup
      \expandafter\noexpand\csname PdfRender@Reset#1\endcsname
    }%
    \expandafter\edef\csname PdfRender@Reset#1\endcsname{%
      \noexpand\pdfliteral direct{%
        #3{\expandafter\noexpand\csname PdfRender@Current#1\endcsname}%
      }%
    }%
  \fi
  \expandafter\edef\csname PdfRender@Normal#1\endcsname{%
    \let
    \expandafter\noexpand\csname PdfRender@Current#1\endcsname
    \expandafter\noexpand\csname PdfRender@Default#1\endcsname
    \noexpand\PdfRender@Set{#1}%
  }%
  \expandafter\ltx@GlobalAppendToMacro\expandafter\PdfRender@NormalColorHook
  \expandafter{%
    \csname PdfRender@Normal#1\endcsname
  }%
  \ltx@GlobalAppendToMacro\PdfRender@ColorSetGroupHook{%
    \PdfRender@Set{#1}%
  }%
  \ifPdfRender@Values
    \kv@parse@normalized{#4}{%
      \expandafter\let\csname PdfRender@#1@\kv@key\endcsname\kv@key
      \ifx\kv@value\relax
      \else
        \expandafter\let\csname PdfRender@#1@\kv@value\endcsname\kv@key
      \fi
      \ltx@gobbletwo
    }%
    \PdfRender@define@key{PDFRENDER}{#1}{%
      \global\csname PdfRender@Active#1true\endcsname
      \def\PdfRender@Current{##1}%
      \PdfRender@SetValidateValues{#1}%
    }%
    \PdfRender@Valuesfalse
  \else
    \PdfRender@define@key{PDFRENDER}{#1}{%
      \global\csname PdfRender@Active#1true\endcsname
      \expandafter\def\csname PdfRender@Current#1\endcsname{##1}%
      \ltx@IfUndefined{PdfRender@PostProcess#1}{%
      }{%
        \csname PdfRender@PostProcess#1\endcsname
      }%
      \PdfRender@SetValidate{#1}{#4}{#5}%
    }%
  \fi
}%
%    \end{macrocode}
%    \end{macro}
%    \begin{macro}{\PdfRender@define@key}
%    \begin{macrocode}
\ltx@IfUndefined{define@key}{%
  \def\PdfRender@define@key#1#2{%
    \expandafter\def\csname KV@#1@#2\endcsname##1%
  }%
}{%
  \let\PdfRender@define@key\define@key
}
%    \end{macrocode}
%    \end{macro}
%    \begin{macro}{\PdfRender@Set}
%    \begin{macrocode}
\def\PdfRender@Set#1{%
  \csname ifPdfRender@Active#1\endcsname
    \csname PdfRender@Set#1\expandafter\endcsname
  \fi
}
%    \end{macrocode}
%    \end{macro}
%    \begin{macro}{\PdfRender@Reset}
%    \begin{macrocode}
\def\PdfRender@Reset#1{%
  \csname ifPdfRender@Active#1\endcsname
    \csname PdfRender@Reset#1\expandafter\endcsname
  \fi
}
%    \end{macrocode}
%    \end{macro}
%    \begin{macro}{\PdfRender@ErrorInvalidValue}
%    \begin{macrocode}
\def\PdfRender@ErrorInvalidValue#1{%
  \PackageError{pdfrender}{%
    Ignoring parameter setting for `#1'\MessageBreak
    because of invalid value %
    `\csname PdfRender@Current#1\endcsname'%
  }\@ehc
  \expandafter\let\csname PdfRender@Current#1\endcsname\ltx@empty
}%
%    \end{macrocode}
%    \end{macro}
%    \begin{macro}{\PdfRender@SetValidate}
%    \begin{macrocode}
\ifPdfRender@Match
  \def\PdfRender@SetValidate#1#2#3{%
    \ifnum\pdfmatch{^(#2)$}{\csname PdfRender@Current#1\endcsname}=1 %
      \csname PdfRender@Set#1\expandafter\endcsname
    \else
      \PdfRender@ErrorInvalidValue{#1}%
    \fi
  }%
\else
  \def\PdfRender@SetValidate#1#2#3{%
    \expandafter\let\expandafter\PdfRender@Current
    \csname PdfRender@Current#1\endcsname
    #3%
    \ifx\PdfRender@Current\@empty
      \PdfRender@ErrorInvalidValue{#1}%
    \else
      \csname PdfRender@Set#1\expandafter\endcsname
    \fi
  }%
\fi
%    \end{macrocode}
%    \end{macro}
%    \begin{macro}{\PdfRender@SetValidateValues}
%    \begin{macrocode}
\def\PdfRender@SetValidateValues#1{%
  \ltx@IfUndefined{PdfRender@#1@\PdfRender@Current}{%
    \expandafter\let\csname PdfRender@Current#1\endcsname
                    \PdfRender@Current
    \PdfRender@ErrorInvalidValue{#1}%
  }{%
    \expandafter\edef\csname PdfRender@Current#1\endcsname{%
      \csname PdfRender@#1@\PdfRender@Current\endcsname
    }%
    \csname PdfRender@Set#1\endcsname
  }%
}
%    \end{macrocode}
%    \end{macro}
%    \begin{macro}{\PdfRender@OpValue}
%    \begin{macrocode}
\def\PdfRender@OpValue#1#2{#2\ltx@space#1}%
%    \end{macrocode}
%    \end{macro}
%    \begin{macro}{\PdfRender@OpName}
%    \begin{macrocode}
\def\PdfRender@OpName#1#2{/#2\ltx@space#1}%
%    \end{macrocode}
%    \end{macro}
%
% \subsection{Declare and setup parameters}
%
%    \begin{macrocode}
\PdfRender@NewClassValues{TextRenderingMode}%
                      {0}%
                      {\PdfRender@OpValue{Tr}}{%
  0=Fill,%
  1=Stroke,%
  2=FillStroke,%
  3=Invisible,%
  4=FillClip,%
  5=StrokeClip,%
  6=FillStrokeClip,%
  7=Clip,%
}%
\PdfRender@NewClass{LineWidth}{1}{\PdfRender@OpValue{w}}{%
  [0-9]+\string\.?[0-9]*|\string\.[0-9]+%
}{}%
\ltx@IfUndefined{dimexpr}{%
  \def\PdfRender@dimexpr{}%
}{%
  \let\PdfRender@dimexpr\dimexpr
}
\def\PdfRender@PostProcessLineWidth{%
  \begingroup
  \afterassignment\PdfRender@@PostProcessLineWidth
  \dimen0=\PdfRender@dimexpr\PdfRender@CurrentLineWidth bp %
  \PdfRender@let\PdfRender@relax\PdfRender@relax
}
\let\PdfRender@let\let
\let\PdfRender@relax\relax
\def\PdfRender@@PostProcessLineWidth#1\PdfRender@let{%
  \ifx\\#1\\%
    \endgroup
  \else
    \dimen0=.996264\dimen0 % 72/72.27
    \edef\x{\endgroup
      \def\noexpand\PdfRender@CurrentLineWidth{%
        \strip@pt\dimen0%
      }%
    }%
    \expandafter\x
  \fi
}
\PdfRender@NewClassValues{LineCapStyle}{0}{\PdfRender@OpValue{J}}{%
  0=Butt,%
  1=Round,%
  2=ProjectingSquare,%
}%
\PdfRender@NewClassValues{LineJoinStyle}{0}{\PdfRender@OpValue{j}}{%
  0=Miter,%
  1=Round,%
  2=Bevel,%
}%
\PdfRender@NewClass{MiterLimit}{10}{\PdfRender@OpValue{M}}{%
  [0-9]*[1-9][0-9]*\string\.?[0-9]*|%
  [0-9]*\string\.?[0-9]*[1-9][0-9]*%
}{}%
\PdfRender@NewClass{Flatness}{0}{\PdfRender@OpValue{i}}{%
  100(\string\.0*)?|[0-9][0-9](\string\.[0-9]*)?|\string\.[0-9]+%
}{}%
\PdfRender@NewClass{LineDashPattern}{[]0}{\PdfRender@OpValue{d}}{%
  \string\[%
  ( ?([0-9]+\string\.?[0-9]*|\string\.[0-9]+) ?)*%
  \string\] ?%
  ([0-9]+\string\.?[0-9]*|\string\.[0-9]+)%
}{}%
\PdfRender@NewClassValues{RenderingIntent}%
                      {RelativeColorimetric}%
                      {\PdfRender@OpName{ri}}{%
  AbsoluteColorimetric,%
  RelativeColorimetric,%
  Saturation,%
  Perceptual,%
}%
%    \end{macrocode}
%
% \subsection{Fill and stroke color support}
%
%    \begin{macrocode}
\PdfRender@define@key{PDFRENDER}{FillColor}{%
  \begingroup
    \def\PdfRender@Color{#1}%
    \ifx\PdfRender@Color\ltx@empty
      \global\let\PdfRender@FillColor\ltx@empty
    \else
      \PdfRender@ColorAvailable{%
        \PdfRender@TestBox{%
          \expandafter\PdfRender@TryColor\PdfRender@Color\ltx@empty
          \PdfRender@GetFillColor
          \ifx\PdfRender@FillColor\ltx@empty
            \@PackageWarning{pdfrender}{%
              Cannot extract fill color\MessageBreak
              from value `#1'%
            }%
          \fi
        }%
      }%
    \fi
  \endgroup
}
\PdfRender@define@key{PDFRENDER}{StrokeColor}{%
  \begingroup
    \def\PdfRender@Color{#1}%
    \ifx\PdfRender@Color\ltx@empty
      \global\let\PdfRender@StrokeColor\ltx@empty
    \else
      \PdfRender@ColorAvailable{%
        \PdfRender@TestBox{%
          \expandafter\PdfRender@TryColor\PdfRender@Color\ltx@empty
          \PdfRender@GetStrokeColor
          \ifx\PdfRender@StrokeColor\ltx@empty
            \@PackageWarning{pdfrender}{%
              Cannot extract stroke color\MessageBreak
              from value `#1'%
            }%
          \fi
        }%
      }%
    \fi
  \endgroup
}
%    \end{macrocode}
%    \begin{macro}{\PdfRender@ColorAvailable}
%    \begin{macrocode}
\def\PdfRender@ColorAvailable{%
  \@ifundefined{set@color}{%
    \@PackageError{pdfrender}{%
      Ignoring color options, because neither\MessageBreak
      package `color' nor package `xcolor' is loaded%
    }\@ehc
    \global\let\PdfRender@ColorAvailable\ltx@gobble
  }{%
    \global\let\PdfRender@ColorAvailable\ltx@firstofone
  }%
  \PdfRender@ColorAvailable
}
%    \end{macrocode}
%    \end{macro}
%    \begin{macro}{\PdfRender@TryColor}
%    \begin{macrocode}
\def\PdfRender@TryColor{%
  \@ifnextchar[\color\PdfRender@@TryColor
}
%    \end{macrocode}
%    \end{macro}
%    \begin{macro}{\PdfRender@@TryColor}
%    \begin{macrocode}
\def\PdfRender@@TryColor#1\ltx@empty{%
  \expandafter\color\expandafter{\PdfRender@Color}%
}
%    \end{macrocode}
%    \end{macro}
%    \begin{macro}{\PdfRender@SetColor}
%    \begin{macrocode}
\def\PdfRender@SetColor{%
  \chardef\PdfRender@NeedsCurrentColor=0 %
  \ifx\PdfRender@FillColor\ltx@empty
    \ifx\PdfRender@StrokeColor\ltx@empty
    \else
      \edef\PdfRender@CurrentColor{%
        \noexpand\PdfRender@FillColor\ltx@space\PdfRender@StrokeColor
      }%
      \chardef\PdfRender@NeedsCurrentColor=1 %
    \fi
  \else
    \ifx\PdfRender@StrokeColor\ltx@empty
      \edef\PdfRender@CurrentColor{%
        \PdfRender@FillColor\ltx@space\noexpand\PdfRender@StrokeColor
      }%
      \chardef\PdfRender@NeedsCurrentColor=2 %
    \else
      \edef\current@color{%
        \PdfRender@FillColor\ltx@space\PdfRender@StrokeColor
      }%
      \set@color
    \fi
  \fi
  \ifnum\PdfRender@NeedsCurrentColor=1 %
    \PdfRender@GetFillColor
    \ifx\PdfRender@FillColor\ltx@empty
      \@PackageWarning{pdfrender}{%
        Cannot extract current fill color%
      }%
    \else
      \edef\current@color{\PdfRender@CurrentColor}%
      \set@color
    \fi
  \else
    \ifnum\PdfRender@NeedsCurrentColor=2 %
      \PdfRender@GetStrokeColor
      \ifx\PdfRender@StrokeColor\ltx@empty
        \@PackageWarning{pdfrender}{%
          Cannot extract current stroke color%
        }%
      \else
        \edef\current@color{\PdfRender@CurrentColor}%
        \set@color
      \fi
    \fi
  \fi
}
%    \end{macrocode}
%    \end{macro}
%    \begin{macro}{\PdfRender@PatternFillColor}
%    \begin{macrocode}
\edef\PdfRender@PatternFillColor{ % space
  (%
    [0-9\string\.]+ g|%
    [0-9\string\.]+ [0-9\string\.]+ [0-9\string\.]+ rg|%
    [0-9\string\.]+ [0-9\string\.]+ %
    [0-9\string\.]+ [0-9\string\.]+ k%
  ) % space
  (.*)$%
}
%    \end{macrocode}
%    \end{macro}
%    \begin{macro}{\PdfRender@PatternStrokeColor}
%    \begin{macrocode}
\edef\PdfRender@PatternStrokeColor{ % space
  (%
    [0-9\string\.]+ G|%
    [0-9\string\.]+ [0-9\string\.]+ [0-9\string\.]+ RG|%
    [0-9\string\.]+ [0-9\string\.]+ %
    [0-9\string\.]+ [0-9\string\.]+ K%
  ) % space
  (.*)$%
}
%    \end{macrocode}
%    \end{macro}
%    \begin{macro}{\PdfRender@MatchPattern}
%    \begin{macrocode}
\def\PdfRender@MatchPattern#1{%
  \ifnum\pdfmatch{\PdfRender@Pattern}{\PdfRender@String}=1 %
    \xdef#1{%
      \expandafter\strip@prefix\pdflastmatch 1%
    }%
    \edef\PdfRender@String{%
      \expandafter\strip@prefix\pdflastmatch 2%
    }%
    \ifx\PdfRender@String\ltx@empty
    \else
      \expandafter\expandafter\expandafter\PdfRender@MatchPattern
      \expandafter\expandafter\expandafter#1%
    \fi
  \fi
}
%    \end{macrocode}
%    \end{macro}
%    \begin{macro}{\PdfRender@GetFillColor}
%    \begin{macrocode}
\def\PdfRender@GetFillColor{%
  \global\let\PdfRender@FillColor\ltx@empty
  \begingroup
    \ifPdfRender@Match
      \let\PdfRender@Pattern\PdfRender@PatternFillColor
      \edef\PdfRender@String{\ltx@space\current@color\ltx@space}%
      \PdfRender@MatchPattern\PdfRender@FillColor
    \else
      \edef\current@color{\current@color\ltx@space}%
      \let\PdfRender@OP\relax
      \PdfRender@FindOp{g}0%
      \PdfRender@FindOp{G}1%
      \PdfRender@FindOp{rg}0%
      \PdfRender@FindOp{RG}1%
      \PdfRender@FindOp{k}0%
      \PdfRender@FindOp{K}1%
      \PdfRender@FilterOp 0\PdfRender@FillColor
    \fi
  \endgroup
}
%    \end{macrocode}
%    \end{macro}
%    \begin{macro}{\PdfRender@GetStrokeColor}
%    \begin{macrocode}
\def\PdfRender@GetStrokeColor{%
  \global\let\PdfRender@StrokeColor\ltx@empty
  \begingroup
    \ifPdfRender@Match
      \let\PdfRender@Pattern\PdfRender@PatternStrokeColor
      \edef\PdfRender@String{\ltx@space\current@color\ltx@space}%
      \PdfRender@MatchPattern\PdfRender@StrokeColor
    \else
      \edef\current@color{\current@color\ltx@space}%
      \let\PdfRender@OP\relax
      \PdfRender@FindOp{g}0%
      \PdfRender@FindOp{G}1%
      \PdfRender@FindOp{rg}0%
      \PdfRender@FindOp{RG}1%
      \PdfRender@FindOp{k}0%
      \PdfRender@FindOp{K}1%
      \PdfRender@FilterOp 1\PdfRender@StrokeColor
    \fi
  \endgroup
}
%    \end{macrocode}
%    \end{macro}
%    \begin{macrocode}
\ifPdfRender@Match
  \expandafter\PdfRender@AtEnd
\fi%
%    \end{macrocode}
%    \begin{macro}{\PdfRender@FindOp}
%    \begin{macrocode}
\def\PdfRender@FindOp#1#2{%
  \def\PdfRender@temp##1 #1 ##2\@nil{%
    ##1%
    \ifx\\##2\\%
      \expandafter\@gobble
    \else
      \PdfRender@OP{#1}#2%
      \expandafter\@firstofone
    \fi
    {%
      \PdfRender@temp##2\@nil
    }%
  }%
  \edef\current@color{%
    \@firstofone{\expandafter\PdfRender@temp\current@color} #1 \@nil
  }%
}
%    \end{macrocode}
%    \end{macro}
%    \begin{macro}{\PdfRender@FilterOp}
%    \begin{macrocode}
\def\PdfRender@FilterOp#1#2{%
  \expandafter\PdfRender@@FilterOp\expandafter#1\expandafter#2%
      \current@color\PdfRender@OP{}{}%
}
%    \end{macrocode}
%    \end{macro}
%    \begin{macro}{\PdfRender@@FilterOp}
%    \begin{macrocode}
\def\PdfRender@@FilterOp#1#2#3\PdfRender@OP#4#5{%
  \ifx\\#4#5\\%
  \else
    \ifnum#1=#5 %
      \xdef#2{#3 #4}%
    \fi
    \expandafter\PdfRender@@FilterOp\expandafter#1\expandafter#2%
  \fi
}
%    \end{macrocode}
%    \end{macro}
%
%    \begin{macrocode}
\PdfRender@AtEnd%
%</package>
%    \end{macrocode}
%% \section{Installation}
%
% \subsection{Download}
%
% \paragraph{Package.} This package is available on
% CTAN\footnote{\CTANpkg{pdfrender}}:
% \begin{description}
% \item[\CTAN{macros/latex/contrib/oberdiek/pdfrender.dtx}] The source file.
% \item[\CTAN{macros/latex/contrib/oberdiek/pdfrender.pdf}] Documentation.
% \end{description}
%
%
% \paragraph{Bundle.} All the packages of the bundle `oberdiek'
% are also available in a TDS compliant ZIP archive. There
% the packages are already unpacked and the documentation files
% are generated. The files and directories obey the TDS standard.
% \begin{description}
% \item[\CTANinstall{install/macros/latex/contrib/oberdiek.tds.zip}]
% \end{description}
% \emph{TDS} refers to the standard ``A Directory Structure
% for \TeX\ Files'' (\CTANpkg{tds}). Directories
% with \xfile{texmf} in their name are usually organized this way.
%
% \subsection{Bundle installation}
%
% \paragraph{Unpacking.} Unpack the \xfile{oberdiek.tds.zip} in the
% TDS tree (also known as \xfile{texmf} tree) of your choice.
% Example (linux):
% \begin{quote}
%   |unzip oberdiek.tds.zip -d ~/texmf|
% \end{quote}
%
% \subsection{Package installation}
%
% \paragraph{Unpacking.} The \xfile{.dtx} file is a self-extracting
% \docstrip\ archive. The files are extracted by running the
% \xfile{.dtx} through \plainTeX:
% \begin{quote}
%   \verb|tex pdfrender.dtx|
% \end{quote}
%
% \paragraph{TDS.} Now the different files must be moved into
% the different directories in your installation TDS tree
% (also known as \xfile{texmf} tree):
% \begin{quote}
% \def\t{^^A
% \begin{tabular}{@{}>{\ttfamily}l@{ $\rightarrow$ }>{\ttfamily}l@{}}
%   pdfrender.sty & tex/generic/oberdiek/pdfrender.sty\\
%   pdfrender.pdf & doc/latex/oberdiek/pdfrender.pdf\\
%   pdfrender.dtx & source/latex/oberdiek/pdfrender.dtx\\
% \end{tabular}^^A
% }^^A
% \sbox0{\t}^^A
% \ifdim\wd0>\linewidth
%   \begingroup
%     \advance\linewidth by\leftmargin
%     \advance\linewidth by\rightmargin
%   \edef\x{\endgroup
%     \def\noexpand\lw{\the\linewidth}^^A
%   }\x
%   \def\lwbox{^^A
%     \leavevmode
%     \hbox to \linewidth{^^A
%       \kern-\leftmargin\relax
%       \hss
%       \usebox0
%       \hss
%       \kern-\rightmargin\relax
%     }^^A
%   }^^A
%   \ifdim\wd0>\lw
%     \sbox0{\small\t}^^A
%     \ifdim\wd0>\linewidth
%       \ifdim\wd0>\lw
%         \sbox0{\footnotesize\t}^^A
%         \ifdim\wd0>\linewidth
%           \ifdim\wd0>\lw
%             \sbox0{\scriptsize\t}^^A
%             \ifdim\wd0>\linewidth
%               \ifdim\wd0>\lw
%                 \sbox0{\tiny\t}^^A
%                 \ifdim\wd0>\linewidth
%                   \lwbox
%                 \else
%                   \usebox0
%                 \fi
%               \else
%                 \lwbox
%               \fi
%             \else
%               \usebox0
%             \fi
%           \else
%             \lwbox
%           \fi
%         \else
%           \usebox0
%         \fi
%       \else
%         \lwbox
%       \fi
%     \else
%       \usebox0
%     \fi
%   \else
%     \lwbox
%   \fi
% \else
%   \usebox0
% \fi
% \end{quote}
% If you have a \xfile{docstrip.cfg} that configures and enables \docstrip's
% TDS installing feature, then some files can already be in the right
% place, see the documentation of \docstrip.
%
% \subsection{Refresh file name databases}
%
% If your \TeX~distribution
% (\TeX\,Live, \mikTeX, \dots) relies on file name databases, you must refresh
% these. For example, \TeX\,Live\ users run \verb|texhash| or
% \verb|mktexlsr|.
%
% \subsection{Some details for the interested}
%
% \paragraph{Unpacking with \LaTeX.}
% The \xfile{.dtx} chooses its action depending on the format:
% \begin{description}
% \item[\plainTeX:] Run \docstrip\ and extract the files.
% \item[\LaTeX:] Generate the documentation.
% \end{description}
% If you insist on using \LaTeX\ for \docstrip\ (really,
% \docstrip\ does not need \LaTeX), then inform the autodetect routine
% about your intention:
% \begin{quote}
%   \verb|latex \let\install=y\input{pdfrender.dtx}|
% \end{quote}
% Do not forget to quote the argument according to the demands
% of your shell.
%
% \paragraph{Generating the documentation.}
% You can use both the \xfile{.dtx} or the \xfile{.drv} to generate
% the documentation. The process can be configured by the
% configuration file \xfile{ltxdoc.cfg}. For instance, put this
% line into this file, if you want to have A4 as paper format:
% \begin{quote}
%   \verb|\PassOptionsToClass{a4paper}{article}|
% \end{quote}
% An example follows how to generate the
% documentation with pdf\LaTeX:
% \begin{quote}
%\begin{verbatim}
%pdflatex pdfrender.dtx
%makeindex -s gind.ist pdfrender.idx
%pdflatex pdfrender.dtx
%makeindex -s gind.ist pdfrender.idx
%pdflatex pdfrender.dtx
%\end{verbatim}
% \end{quote}
%
% \section{Acknowledgement}
%
% \begin{description}
% \item[Friedrich Vosberg] asked in the
%   newsgroup \textsf{de.comp.text.tex} for the font
%   outline feature \cite{news-Vosberg}.
% \item[Gaius Pupus] proposed the basic method
%   using \cs{pdfliteral} in this thread \cite{news-Pupus}.
% \item[Rolf Niepraschk] added color support \cite{news-Niepraschk}.
% \end{description}
%
% \begin{thebibliography}{9}
% \raggedright
%
% \bibitem{pdf-spec}
%   Adobe Systems Incorporated. \textit{PDF Reference --
%   Adobe Portable Document format -- Version 1.7}.
%   6th ed. 2006.
%   \textsc{URL:}\,\url{http://www.adobe.com/devnet/acrobat/pdfs/pdf_reference_1-7.pdf}.
%
% \bibitem{news-Vosberg}
%   Friedrich Vosberg, \textit{Text in Buchstabenumrissen},
%   de.comp.text.tex, 2010-01-22.
%   \textsc{URL:}\,\url{https://groups.google.com/group/de.comp.text.tex/msg/f442310ac8b2d506}.
%
% \bibitem{news-Pupus}
%   Gaius Pupus, \textit{Re: Text in Buchstabenumrissen},
%   de.comp.text.tex, 2010-01-23.
%   \textsc{URL:}\,\url{https://groups.google.com/group/de.comp.text.tex/msg/95d890d77ac47eb1}.
%
% \bibitem{news-Niepraschk}
%   Rolf Niepraschk, \textit{Re: Text in Buchstabenumrissen},
%   de.comp.text.tex, 2010-01-24.
%   \textsc{URL:}\,\url{https://groups.google.com/group/de.comp.text.tex/msg/4eb61a5879db54db}.
%
% \end{thebibliography}
%
% \begin{History}
%   \begin{Version}{2010/01/26 v1.0}
%   \item
%     The first version.
%   \end{Version}
%   \begin{Version}{2010/01/27 v1.1}
%   \item
%     Macros \cs{pdfrender} and \cs{textpdfrender} are made robust.
%   \item
%     Color extraction rewritten for the case that \cs{pdfmatch}
%     is not available. This fixes wrong color assigments in
%     case of nesting.
%   \item
%     Color extraction of case \cs{pdfmatch} is fixed
%     for the case that the color string contains several
%     fill or several stroke operations.
%   \end{Version}
%   \begin{Version}{2010/01/28 v1.2}
%   \item
%     Dependency from package \xpackage{color} is removed.
%   \item
%     Compatibility for \plainTeX\ and even ini\TeX\ added.
%   \end{Version}
%   \begin{Version}{2016/05/14 v1.3}
%   \item
%     Use package \xpackage{luatex85} for compatibility with new LuaTeX.
%   \end{Version}
%   \begin{Version}{2016/05/17 v1.4}
%   \item
%     Documentation updates.
%   \item adjust luatex85 reference so that it works in plain TeX.
%   \end{Version}
%   \begin{Version}{2018/11/01 v1.5}
%   \item
%     Remove luatex85 dependency
%   \end{Version}
%   \begin{Version}{2019/12/29 v1.6}
%   \item
%     \xpackage{iftex} package.
%   \end{Version}
% \end{History}
%
% \PrintIndex
%
% \Finale
\endinput

%        (quote the arguments according to the demands of your shell)
%
% Documentation:
%    (a) If pdfrender.drv is present:
%           latex pdfrender.drv
%    (b) Without pdfrender.drv:
%           latex pdfrender.dtx; ...
%    The class ltxdoc loads the configuration file ltxdoc.cfg
%    if available. Here you can specify further options, e.g.
%    use A4 as paper format:
%       \PassOptionsToClass{a4paper}{article}
%
%    Programm calls to get the documentation (example):
%       pdflatex pdfrender.dtx
%       makeindex -s gind.ist pdfrender.idx
%       pdflatex pdfrender.dtx
%       makeindex -s gind.ist pdfrender.idx
%       pdflatex pdfrender.dtx
%
% Installation:
%    TDS:tex/generic/oberdiek/pdfrender.sty
%    TDS:doc/latex/oberdiek/pdfrender.pdf
%    TDS:source/latex/oberdiek/pdfrender.dtx
%
%<*ignore>
\begingroup
  \catcode123=1 %
  \catcode125=2 %
  \def\x{LaTeX2e}%
\expandafter\endgroup
\ifcase 0\ifx\install y1\fi\expandafter
         \ifx\csname processbatchFile\endcsname\relax\else1\fi
         \ifx\fmtname\x\else 1\fi\relax
\else\csname fi\endcsname
%</ignore>
%<*install>
\input docstrip.tex
\Msg{************************************************************************}
\Msg{* Installation}
\Msg{* Package: pdfrender 2019/12/29 v1.6 Access to some PDF graphics parameters (HO)}
\Msg{************************************************************************}

\keepsilent
\askforoverwritefalse

\let\MetaPrefix\relax
\preamble

This is a generated file.

Project: pdfrender
Version: 2019/12/29 v1.6

Copyright (C)
   2010 Heiko Oberdiek
   2016-2019 Oberdiek Package Support Group

This work may be distributed and/or modified under the
conditions of the LaTeX Project Public License, either
version 1.3c of this license or (at your option) any later
version. This version of this license is in
   https://www.latex-project.org/lppl/lppl-1-3c.txt
and the latest version of this license is in
   https://www.latex-project.org/lppl.txt
and version 1.3 or later is part of all distributions of
LaTeX version 2005/12/01 or later.

This work has the LPPL maintenance status "maintained".

The Current Maintainers of this work are
Heiko Oberdiek and the Oberdiek Package Support Group
https://github.com/ho-tex/oberdiek/issues


The Base Interpreter refers to any `TeX-Format',
because some files are installed in TDS:tex/generic//.

This work consists of the main source file pdfrender.dtx
and the derived files
   pdfrender.sty, pdfrender.pdf, pdfrender.ins, pdfrender.drv,
   pdfrender-test1.tex, pdfrender-test2.tex,
   pdfrender-test3.tex, pdfrender-test4.tex,
   pdfrender-test5.tex.

\endpreamble
\let\MetaPrefix\DoubleperCent

\generate{%
  \file{pdfrender.ins}{\from{pdfrender.dtx}{install}}%
  \file{pdfrender.drv}{\from{pdfrender.dtx}{driver}}%
  \usedir{tex/generic/oberdiek}%
  \file{pdfrender.sty}{\from{pdfrender.dtx}{package}}%
%  \usedir{doc/latex/oberdiek/test}%
%  \file{pdfrender-test1.tex}{\from{pdfrender.dtx}{test1}}%
%  \file{pdfrender-test2.tex}{\from{pdfrender.dtx}{test2}}%
%  \file{pdfrender-test3.tex}{\from{pdfrender.dtx}{test3}}%
%  \file{pdfrender-test4.tex}{\from{pdfrender.dtx}{test4}}%
%  \file{pdfrender-test5.tex}{\from{pdfrender.dtx}{test5}}%
}

\catcode32=13\relax% active space
\let =\space%
\Msg{************************************************************************}
\Msg{*}
\Msg{* To finish the installation you have to move the following}
\Msg{* file into a directory searched by TeX:}
\Msg{*}
\Msg{*     pdfrender.sty}
\Msg{*}
\Msg{* To produce the documentation run the file `pdfrender.drv'}
\Msg{* through LaTeX.}
\Msg{*}
\Msg{* Happy TeXing!}
\Msg{*}
\Msg{************************************************************************}

\endbatchfile
%</install>
%<*ignore>
\fi
%</ignore>
%<*driver>
\NeedsTeXFormat{LaTeX2e}
\ProvidesFile{pdfrender.drv}%
  [2019/12/29 v1.6 Access to some PDF graphics parameters (HO)]%
\documentclass{ltxdoc}
\usepackage{holtxdoc}[2011/11/22]
\usepackage{longtable}
\renewcommand*{\arraystretch}{1.2}
\begin{document}
  \DocInput{pdfrender.dtx}%
\end{document}
%</driver>
% \fi
%
%
%
% \GetFileInfo{pdfrender.drv}
%
% \title{The \xpackage{pdfrender} package}
% \date{2019/12/29 v1.6}
% \author{Heiko Oberdiek\thanks
% {Please report any issues at \url{https://github.com/ho-tex/oberdiek/issues}}}
%
% \maketitle
%
% \begin{abstract}
% The PDF format has some graphics parameter like
% line width or text rendering mode. This package
% provides an interface for setting these parameters.
% \end{abstract}
%
% \tableofcontents
%
% \section{Documentation}
%
% This package \xpackage{pdfrender} defines an interface for
% PDF specific parameters that affects the rendering of
% graphics or text. The interface and its implementation
% uses the same technique as package \xpackage{color}
% for color settings. Therefore this package is loaded to
% enable \LaTeX's color interface.
%
% At different places \LaTeX\ uses \cs{normalcolor} to avoid that
% header, footer or floats are print in the current color of the
% main text. \cs{setgroup@color} is used to start a save box
% with the color that is set at box saving time.
% Package \xpackage{pdfrender} extends these macros to
% add its own hooks of its parameters. Therefore \LaTeX3 should
% generalize \LaTeXe's color interface.
%
% \subsection{Usage}
%
% In \LaTeX\ the package is loaded as normal package. Options
% are not defined for this package.
% \begin{quote}
% |\usepackage{pdfrender}|
% \end{quote}
%
% This package can also be used in \plainTeX\ and even ini\TeX:
% \begin{quote}
% |input pdfrender.sty|
% \end{quote}
%
% \subsection{Macros}
%
% \begin{declcs}{pdfrender} \M{key value list}
% \end{declcs}
% The first parameter \meta{key value list} contains a
% list of parameter settings. The key entry is the parameter name.
% The macro works like
% \cs{color} (without optional argument) for color setting.
%
% \begin{declcs}{textpdfrender} \M{key value list} \M{text}
% \end{declcs}
% In the same way as \cs{pdfrender} the first argument
% specifies the parameters that should be set. This parameter
% setting affects \meta{text} only. Basically it works
% the same way as \cs{textcolor} (without optional argument).
%
% \subsection{Parameters}
%
% The following table shows an overview for the supported
% parameters and values:
%
% \begin{longtable}{@{}l|l|l@{}}
% \bfseries Parameter & \bfseries Value & \bfseries Alias\\*
% \hline
% \endhead
% \xoption{TextRenderingMode}
%   & \texttt{0} & \texttt{Fill}\\*
%   & \texttt{1} & \texttt{Stroke}\\*
%   & \texttt{2} & \texttt{FillStroke}\\*
%   & \texttt{3} & \texttt{Invisible}\\*
%   & \texttt{4} & \texttt{FillClip}\\*
%   & \texttt{5} & \texttt{StrokeClip}\\*
%   & \texttt{6} & \texttt{FillStrokeClip}\\*
%   & \texttt{7} & \texttt{Clip}\\
% \hline
% \xoption{LineWidth}
%   & \itshape positive number, & \itshape \TeX\ dimen\\*[-.2ex]
%   & \qquad \itshape unit is bp & \\
% \hline
% \xoption{LineCapStyle}
%   & \texttt{0} & \texttt{Butt}\\*
%   & \texttt{1} & \texttt{Round}\\*
%   & \texttt{2} & \texttt{ProjectingSquare}\\
% \hline
% \xoption{LineJoinStyle}
%   & \texttt{0} & \texttt{Miter}\\*
%   & \texttt{1} & \texttt{Round}\\*
%   & \texttt{2} & \texttt{Bevel}\\
% \hline
% \xoption{MiterLimit}
%   & \multicolumn{2}{l@{}}{\itshape positive number} \\
% \hline
% \xoption{Flatness}
%   & \multicolumn{2}{l@{}}{\itshape number between 0 and 100} \\
% \hline
% \xoption{LineDashPattern}
%   & \multicolumn{2}{l@{}}{^^A
%       \itshape numbers in square brackets,^^A
%     } \\*[-.2ex]
%   & \multicolumn{2}{l@{}}{^^A
%       \qquad \itshape followed by number,^^A
%     } \\*
%   & \multicolumn{2}{l@{}}{^^A
%       \qquad \itshape units are bp^^A
%     } \\
% \hline
% \xoption{RenderingIntent}
%   & \multicolumn{2}{l@{}}{\texttt{AbsoluteColorimetric}} \\*
%   & \multicolumn{2}{l@{}}{\texttt{RelativeColorimetric}} \\*
%   & \multicolumn{2}{l@{}}{\texttt{Saturation}} \\*
%   & \multicolumn{2}{l@{}}{\texttt{Perceptual}} \\
% \hline
% \xoption{FillColor}
%   & & \itshape color specification \\
% \hline
% \xoption{StrokeColor}
%   & & \itshape color specification \\
% \end{longtable}
%
% \subsubsection{Details}
%
% The description and specification of these parameters
% are available in the PDF specification \cite{pdf-spec}.
% Therefore they are not repeated here.
%
% \begin{description}
% \item[Value:]
%   The values in the second column lists or describe
%   the values that are specified by the PDF specification.
% \item[Alias:] Instead of magic numbers the package also
%   defines some aliases that can be given as value.
%   Example: |LineCapStyle=Round| has the same effect as
%   |LineCapStyle=1|.
% \item[Number:] The term \emph{number} means an integer
%   or real number. The real number is given as plain
%   decimal number without exponent. The decimal separator is a period.
%   At least one digit must be present.
% \item[\xoption{LineWidth}:] As alias a \TeX\ dimen specification
%   can be given. This includes explicit specifications with number
%   and unit, e.g. |LineWidth=0.5pt|. Also \LaTeX\ length registers
%   may be used. If \eTeX's \cs{dimexpr} is available, then it is
%   automatically added. However package \xpackage{calc} is not
%   supported.
% \item[\xoption{FillColor}, \xoption{StrokeColor}:]
%   Package \xpackage{color} or \xpackage{xcolor} must be loaded
%   before these options can be used (since version 1.2).
%   \LaTeX's color support sets both colors at the same time
%   to the same value. However parameter \xoption{TextRenderingMode}
%   offers the value \texttt{FillStroke} that makes only sense, if
%   the two color types can be set separately.
%   If one of the options \xoption{FillColor} or \xoption{StrokeColor}
%   is specified, then also the color is set. For compatibility
%   with the \LaTeX\ color packages (\xpackage{color} or \xpackage{xcolor}),
%   always both colors must be set. Thus if one of them is not specified,
%   it is taken from the current color.
%
%   Both options \xoption{FillColor} and \xoption{StrokeColor}
%   expect a \LaTeX\ color specification as value.
%   Also the optional color model argument is supported.
%   Example:
%   \begin{quote}
%     |FillColor=yellow|,\\
%     |StrokeColor=[cmyk]{1,.5,0,0}|\\
%   \end{quote}
% \end{description}
%
% \subsection{Color stack}
%
% If the \pdfTeX\ version provides color stacks, then each
% parameter is assigned a page based color stack. The assignment
% of a stack takes place, when its parameter is set the first time.
% This avoids the use of color stacks that are not needed.
%
% \StopEventually{
% }
%
% \section{Implementation}
%    \begin{macrocode}
%<*package>
%    \end{macrocode}
%    Reload check, especially if the package is not used with \LaTeX.
%    \begin{macrocode}
\begingroup\catcode61\catcode48\catcode32=10\relax%
  \catcode13=5 % ^^M
  \endlinechar=13 %
  \catcode35=6 % #
  \catcode39=12 % '
  \catcode44=12 % ,
  \catcode45=12 % -
  \catcode46=12 % .
  \catcode58=12 % :
  \catcode64=11 % @
  \catcode123=1 % {
  \catcode125=2 % }
  \expandafter\let\expandafter\x\csname ver@pdfrender.sty\endcsname
  \ifx\x\relax % plain-TeX, first loading
  \else
    \def\empty{}%
    \ifx\x\empty % LaTeX, first loading,
      % variable is initialized, but \ProvidesPackage not yet seen
    \else
      \expandafter\ifx\csname PackageInfo\endcsname\relax
        \def\x#1#2{%
          \immediate\write-1{Package #1 Info: #2.}%
        }%
      \else
        \def\x#1#2{\PackageInfo{#1}{#2, stopped}}%
      \fi
      \x{pdfrender}{The package is already loaded}%
      \aftergroup\endinput
    \fi
  \fi
\endgroup%
%    \end{macrocode}
%    Package identification:
%    \begin{macrocode}
\begingroup\catcode61\catcode48\catcode32=10\relax%
  \catcode13=5 % ^^M
  \endlinechar=13 %
  \catcode35=6 % #
  \catcode39=12 % '
  \catcode40=12 % (
  \catcode41=12 % )
  \catcode44=12 % ,
  \catcode45=12 % -
  \catcode46=12 % .
  \catcode47=12 % /
  \catcode58=12 % :
  \catcode64=11 % @
  \catcode91=12 % [
  \catcode93=12 % ]
  \catcode123=1 % {
  \catcode125=2 % }
  \expandafter\ifx\csname ProvidesPackage\endcsname\relax
    \def\x#1#2#3[#4]{\endgroup
      \immediate\write-1{Package: #3 #4}%
      \xdef#1{#4}%
    }%
  \else
    \def\x#1#2[#3]{\endgroup
      #2[{#3}]%
      \ifx#1\@undefined
        \xdef#1{#3}%
      \fi
      \ifx#1\relax
        \xdef#1{#3}%
      \fi
    }%
  \fi
\expandafter\x\csname ver@pdfrender.sty\endcsname
\ProvidesPackage{pdfrender}%
  [2019/12/29 v1.6 Access to some PDF graphics parameters (HO)]%
%    \end{macrocode}
%
%    \begin{macrocode}
\begingroup\catcode61\catcode48\catcode32=10\relax%
  \catcode13=5 % ^^M
  \endlinechar=13 %
  \catcode123=1 % {
  \catcode125=2 % }
  \catcode64=11 % @
  \def\x{\endgroup
    \expandafter\edef\csname PdfRender@AtEnd\endcsname{%
      \endlinechar=\the\endlinechar\relax
      \catcode13=\the\catcode13\relax
      \catcode32=\the\catcode32\relax
      \catcode35=\the\catcode35\relax
      \catcode61=\the\catcode61\relax
      \catcode64=\the\catcode64\relax
      \catcode123=\the\catcode123\relax
      \catcode125=\the\catcode125\relax
    }%
  }%
\x\catcode61\catcode48\catcode32=10\relax%
\catcode13=5 % ^^M
\endlinechar=13 %
\catcode35=6 % #
\catcode64=11 % @
\catcode123=1 % {
\catcode125=2 % }
\def\TMP@EnsureCode#1#2{%
  \edef\PdfRender@AtEnd{%
    \PdfRender@AtEnd
    \catcode#1=\the\catcode#1\relax
  }%
  \catcode#1=#2\relax
}
\TMP@EnsureCode{10}{12}% ^^J
\TMP@EnsureCode{36}{3}% $
\TMP@EnsureCode{39}{12}% '
\TMP@EnsureCode{40}{12}% (
\TMP@EnsureCode{41}{12}% )
\TMP@EnsureCode{42}{12}% *
\TMP@EnsureCode{43}{12}% +
\TMP@EnsureCode{44}{12}% ,
\TMP@EnsureCode{45}{12}% -
\TMP@EnsureCode{46}{12}% .
\TMP@EnsureCode{47}{12}% /
\TMP@EnsureCode{58}{12}% :
\TMP@EnsureCode{59}{12}% ;
\TMP@EnsureCode{60}{12}% <
\TMP@EnsureCode{62}{12}% >
\TMP@EnsureCode{63}{12}% ?
\TMP@EnsureCode{91}{12}% [
\TMP@EnsureCode{93}{12}% ]
\TMP@EnsureCode{94}{7}% ^ (superscript)
\TMP@EnsureCode{96}{12}% `
\TMP@EnsureCode{124}{12}% |
%    \end{macrocode}
%
%    \begin{macrocode}
\def\PdfRender@AtEndHook{}
\expandafter\def\expandafter\PdfRender@AtEnd\expandafter{%
  \expandafter\PdfRender@AtEndHook
  \PdfRender@AtEnd
  \endinput
}
%    \end{macrocode}
%
% \subsection{Look for \pdfTeX, its mode and features}
%
%    \begin{macro}{\PdfRender@newif}
%    \begin{macrocode}
\def\PdfRender@newif#1{%
  \expandafter\edef\csname PdfRender@#1true\endcsname{%
    \let
    \expandafter\noexpand\csname ifPdfRender@#1\endcsname
    \noexpand\iftrue
  }%
  \expandafter\edef\csname PdfRender@#1false\endcsname{%
    \let
    \expandafter\noexpand\csname ifPdfRender@#1\endcsname
    \noexpand\iffalse
  }%
  \csname PdfRender@#1false\endcsname
}
%    \end{macrocode}
%    \end{macro}
%
%    \begin{macro}{\ifPdfRender@Stack}
%    \begin{macrocode}
\PdfRender@newif{Stack}
%    \end{macrocode}
%    \end{macro}
%    \begin{macro}{\ifPdfRender@Match}
%    \begin{macrocode}
\PdfRender@newif{Match}
%    \end{macrocode}
%    \end{macro}
%
%    \begin{macro}{\PdfRender@RequirePackage}
%    \begin{macrocode}
\begingroup\expandafter\expandafter\expandafter\endgroup
\expandafter\ifx\csname RequirePackage\endcsname\relax
  \def\PdfRender@RequirePackage#1[#2]{%
    \expandafter\def\expandafter\PdfRender@AtEndHook\expandafter{%
      \PdfRender@AtEndHook
      \ltx@ifpackagelater{#1}{#2}{}{%
        \@PackageWarningNoLine{pdfrender}{%
          You have requested version\MessageBreak
          `#2' of package `#1',\MessageBreak
          but only version\MessageBreak
          `\csname ver@#1.\ltx@pkgextension\endcsname'\MessageBreak
          is available%
        }%
      }%
    }%
    \input #1.sty\relax
  }%
\else
  \let\PdfRender@RequirePackage\RequirePackage
\fi
%    \end{macrocode}
%    \end{macro}
% Luatex compatibility
%    \begin{macrocode}
\ifx\pdfextension\@undefined\else
  \def\pdfcolorstackinit {\pdffeedback colorstackinit}
  \protected\def\pdfcolorstack     {\pdfextension colorstack}
  \protected\def\pdfliteral        {\pdfextension literal}
\fi
%    \end{macrocode}
%
%    \begin{macrocode}
\PdfRender@RequirePackage{iftex}[2019/11/07]
\PdfRender@RequirePackage{infwarerr}[2007/09/09]
\PdfRender@RequirePackage{ltxcmds}[2010/01/28]
%    \end{macrocode}
%
%    \begin{macrocode}
\ifpdf
  \ltx@IfUndefined{pdfcolorstackinit}{%
    \@PackageWarning{pdfrender}{%
      Missing \string\pdfcolorstackinit
    }%
  }{%
    \PdfRender@Stacktrue
  }%
  \ltx@IfUndefined{pdfmatch}{%
    \@PackageInfoNoLine{pdfrender}{%
      \string\pdfmatch\ltx@space not found. %
      Therefore the values\MessageBreak
      of some parameters are not validated%
    }%
  }{%
    \PdfRender@Matchtrue
  }%
\else
  \@PackageWarning{pdfrender}{%
    Missing pdfTeX in PDF mode%
  }%
%    \end{macrocode}
%    \begin{macrocode}
  \ltx@IfUndefined{newcommand}{%
%    \end{macrocode}
%    \begin{macro}{\pdfrender}
%    \begin{macrocode}
    \def\pdfrender#1{}%
%    \end{macrocode}
%    \end{macro}
%    \begin{macro}{\textpdfrender}
%    \begin{macrocode}
    \long\def\textpdfrender#1#2{#2}%
%    \end{macrocode}
%    \end{macro}
%    \begin{macrocode}
  }{%
%    \end{macrocode}
%    \begin{macro}{\pdfrender}
%    \begin{macrocode}
    \newcommand*{\pdfrender}[1]{}%
%    \end{macrocode}
%    \end{macro}
%    \begin{macro}{\textpdfrender}
%    \begin{macrocode}
    \newcommand{\textpdfrender}[2]{#2}%
%    \end{macrocode}
%    \end{macro}
%    \begin{macrocode}
  }%
  \expandafter\PdfRender@AtEnd
\fi%
%    \end{macrocode}
%
% \subsection{Enable color support of \LaTeX}
%
%    \begin{macrocode}
\ltx@ifpackageloaded{color}{}{%
  \def\color@setgroup{\begingroup\set@color}%
  \let\color@begingroup\begingroup
  \def\color@endgroup{\endgraf\endgroup}%
  \def\color@hbox{\hbox\bgroup\color@begingroup}%
  \def\color@vbox{\vbox\bgroup\color@begingroup}%
  \def\color@endbox{\color@endgroup\egroup}%
  \ltx@ifundefined{bgroup}{%
    \let\bgroup={\let\egroup=}%
  }{}%
  \ltx@ifundefined{endgraf}{%
    \let\endgraf=\par
  }{}%
}
%    \end{macrocode}
%
% \subsection{Hook into \cs{normalcolor}}
%
%    The problem is that packages \xpackage{color} and \xpackage{xcolor}
%    each overwrite \cs{normalcolor}. For example, after the package
%    loading order \xpackage{color}, \xpackage{pdfrender} and \xpackage{xcolor}
%    the patched version of \cs{normalcolor} is overwritten by
%    package \xpackage{xcolor}. Also using \cs{AtBeginDocument} for
%    patching is not enough. If package \xpackage{hyperref} is loaded
%    later, it might load package \xpackage{color} using \cs{AtBeginDocument}.
%    \begin{macro}{\PdfRender@NormalColorHook}
%    \begin{macrocode}
\def\PdfRender@NormalColorHook{}
%    \end{macrocode}
%    \end{macro}
%    \begin{macro}{\PdfRender@ColorSetGroupHook}
%    \begin{macrocode}
\def\PdfRender@ColorSetGroupHook{}
%    \end{macrocode}
%    \end{macro}
%    \begin{macro}{\PdfRender@TestBox}
%    \begin{macrocode}
\def\PdfRender@TestBox#1{%
  \setbox0=\color@hbox#1\color@endbox
}
%    \end{macrocode}
%    \end{macro}
%    \begin{macro}{\PdfRender@PatchNormalColor}
%    \begin{macrocode}
\def\PdfRender@PatchNormalColor{%
  \ltx@ifundefined{normalcolor}{%
    \gdef\normalcolor{\PdfRender@NormalColorHook}%
  }{%
    \begingroup
      \def\PdfRender@NormalColorHook{\let\PdfRender@temp=Y}%
      \PdfRender@TestBox{%
        \let\set@color\relax
        \normalcolor
        \ifx\PdfRender@temp Y%
        \else
          \ltx@GlobalAppendToMacro\normalcolor{%
            \PdfRender@NormalColorHook
          }%
        \fi
      }%
    \endgroup
  }%
  \ifx\@nodocument\relax
    \global\let\PdfRender@PatchNormalColor\relax
  \fi
}%
%    \end{macrocode}
%    \end{macro}
%    \begin{macro}{\PdfRender@PatchColorSetGroup}
%    \begin{macrocode}
\def\PdfRender@PatchColorSetGroup{%
  \begingroup
    \def\PdfRender@ColorSetGroupHook{\let\PdfRender@temp=Y}%
    \PdfRender@TestBox{%
      \let\set@color\relax
      \color@setgroup\color@endgroup
      \ifx\PdfRender@temp Y%
      \else
        \ltx@GlobalAppendToMacro\color@setgroup{%
          \PdfRender@ColorSetGroupHook
        }%
      \fi
    }%
  \endgroup
  \ifx\@nodocument\relax
    \global\let\PdfRender@PatchColorSetGroup\relax
  \fi
}%
%    \end{macrocode}
%    \end{macro}
%    \begin{macro}{\PdfRender@PatchColor}
%    \begin{macrocode}
\def\PdfRender@PatchColor{%
  \PdfRender@PatchNormalColor
  \PdfRender@PatchColorSetGroup
}
%    \end{macrocode}
%    \end{macro}
%    \begin{macrocode}
\PdfRender@PatchColor
\ltx@IfUndefined{AtBeginDocument}{}{%
  \AtBeginDocument{\PdfRender@PatchColor}%
}
%    \end{macrocode}
%    \cs{AfterPackage} is provided by package \xpackage{scrlfile}.
%    \begin{macrocode}
\ltx@IfUndefined{AfterPackage}{%
}{%
  \AfterPackage{color}{\PdfRender@PatchColor}%
  \AfterPackage{xcolor}{\PdfRender@PatchColor}%
  \AfterPackage{etoolbox}{%
    \AfterEndPreamble{\PdfRender@PatchColor}%
  }%
}%
%    \end{macrocode}
%    \cs{AfterEndPreamble} is provided by package \xpackage{etoolbox}.
%    \begin{macrocode}
\ltx@IfUndefined{AfterEndPreamble}{%
}{%
  \AfterEndPreamble{\PdfRender@PatchColor}%
}%
%    \end{macrocode}
%
%    \begin{macrocode}
\PdfRender@RequirePackage{kvsetkeys}[2010/01/28]
%    \end{macrocode}
%    \begin{macro}{\PdfRender@texorpdfstring}
%    \begin{macrocode}
\def\PdfRender@texorpdfstring{%
  \ltx@IfUndefined{texorpdfstring}\ltx@firstoftwo\texorpdfstring
}
%    \end{macrocode}
%    \end{macro}
%    \begin{macro}{\pdfrender}
%    \begin{macrocode}
\ltx@IfUndefined{DeclareRobustCommand}%
\ltx@firstoftwo\ltx@secondoftwo
{%
  \def\pdfrender#1%
}{%
  \newcommand{\pdfrender}{}%
  \DeclareRobustCommand*{\pdfrender}[1]%
}%
{%
  \PdfRender@texorpdfstring{%
    \PdfRender@PatchNormalColor
    \global\let\PdfRender@FillColor\ltx@empty
    \global\let\PdfRender@StrokeColor\ltx@empty
    \kvsetkeys{PDFRENDER}{#1}%
    \PdfRender@SetColor
  }{}%
}
%    \end{macrocode}
%    \end{macro}
%    \begin{macro}{\textpdfrender}
%    \begin{macrocode}
\ltx@IfUndefined{DeclareRobustCommand}%
\ltx@firstoftwo\ltx@secondoftwo
{%
  \long\def\textpdfrender#1#2%
}{%
  \newcommand{\textpdfrender}{}%
  \DeclareRobustCommand{\textpdfrender}[2]%
}%
{%
  \PdfRender@texorpdfstring{%
    \begingroup
      \pdfrender{#1}%
      #2%
    \endgroup
  }{#2}%
}
%    \end{macrocode}
%    \end{macro}
%
%    \begin{macro}{\ifPdfRender@Values}
%    \begin{macrocode}
\PdfRender@newif{Values}
%    \end{macrocode}
%    \end{macro}
%    \begin{macro}{\PdfRender@NewClassValues}
%    \begin{macrocode}
\def\PdfRender@NewClassValues#1#2#3#4{%
  \PdfRender@Valuestrue
  \PdfRender@NewClass{#1}{#2}{#3}{#4}{}%
}
%    \end{macrocode}
%    \end{macro}
%    \begin{macro}{\PdfRender@NewClass}
%    \begin{macrocode}
\def\PdfRender@NewClass#1#2#3#4#5{%
  \PdfRender@newif{Active#1}%
  \expandafter\def\csname PdfRender@Default#1\endcsname{#2}%
  \expandafter\let\csname PdfRender@Current#1\expandafter\endcsname
      \csname PdfRender@Default#1\endcsname
  \ifPdfRender@Stack
    \expandafter\edef\csname PdfRender@Init#1\endcsname{%
      \global\chardef
      \expandafter\noexpand\csname PdfRender@Stack#1\endcsname=%
          \noexpand\pdfcolorstackinit page direct{%
            \noexpand#3%
            \expandafter\noexpand\csname PdfRender@Default#1\endcsname
          }\relax
      \noexpand\@PackageInfo{pdfrender}{%
        New color stack `#1' = \noexpand\number
        \expandafter\noexpand\csname PdfRender@Stack#1\endcsname
      }%
      \gdef\expandafter\noexpand\csname PdfRender@Init#1\endcsname{}%
    }%
    \expandafter\edef\csname PdfRender@Set#1\endcsname{%
      \expandafter\noexpand\csname PdfRender@Init#1\endcsname
      \noexpand\pdfcolorstack
      \expandafter\noexpand\csname PdfRender@Stack#1\endcsname
      push{%
        #3{\expandafter\noexpand\csname PdfRender@Current#1\endcsname}%
      }%
      \noexpand\aftergroup
      \expandafter\noexpand\csname PdfRender@Reset#1\endcsname
    }%
    \expandafter\edef\csname PdfRender@Reset#1\endcsname{%
      \expandafter\noexpand\csname PdfRender@Init#1\endcsname
      \noexpand\pdfcolorstack
      \expandafter\noexpand\csname PdfRender@Stack#1\endcsname
      pop\relax
    }%
  \else
    \expandafter\edef\csname PdfRender@Set#1\endcsname{%
      \noexpand\pdfliteral direct{%
        #3{\expandafter\noexpand\csname PdfRender@Current#1\endcsname}%
      }%
      \noexpand\aftergroup
      \expandafter\noexpand\csname PdfRender@Reset#1\endcsname
    }%
    \expandafter\edef\csname PdfRender@Reset#1\endcsname{%
      \noexpand\pdfliteral direct{%
        #3{\expandafter\noexpand\csname PdfRender@Current#1\endcsname}%
      }%
    }%
  \fi
  \expandafter\edef\csname PdfRender@Normal#1\endcsname{%
    \let
    \expandafter\noexpand\csname PdfRender@Current#1\endcsname
    \expandafter\noexpand\csname PdfRender@Default#1\endcsname
    \noexpand\PdfRender@Set{#1}%
  }%
  \expandafter\ltx@GlobalAppendToMacro\expandafter\PdfRender@NormalColorHook
  \expandafter{%
    \csname PdfRender@Normal#1\endcsname
  }%
  \ltx@GlobalAppendToMacro\PdfRender@ColorSetGroupHook{%
    \PdfRender@Set{#1}%
  }%
  \ifPdfRender@Values
    \kv@parse@normalized{#4}{%
      \expandafter\let\csname PdfRender@#1@\kv@key\endcsname\kv@key
      \ifx\kv@value\relax
      \else
        \expandafter\let\csname PdfRender@#1@\kv@value\endcsname\kv@key
      \fi
      \ltx@gobbletwo
    }%
    \PdfRender@define@key{PDFRENDER}{#1}{%
      \global\csname PdfRender@Active#1true\endcsname
      \def\PdfRender@Current{##1}%
      \PdfRender@SetValidateValues{#1}%
    }%
    \PdfRender@Valuesfalse
  \else
    \PdfRender@define@key{PDFRENDER}{#1}{%
      \global\csname PdfRender@Active#1true\endcsname
      \expandafter\def\csname PdfRender@Current#1\endcsname{##1}%
      \ltx@IfUndefined{PdfRender@PostProcess#1}{%
      }{%
        \csname PdfRender@PostProcess#1\endcsname
      }%
      \PdfRender@SetValidate{#1}{#4}{#5}%
    }%
  \fi
}%
%    \end{macrocode}
%    \end{macro}
%    \begin{macro}{\PdfRender@define@key}
%    \begin{macrocode}
\ltx@IfUndefined{define@key}{%
  \def\PdfRender@define@key#1#2{%
    \expandafter\def\csname KV@#1@#2\endcsname##1%
  }%
}{%
  \let\PdfRender@define@key\define@key
}
%    \end{macrocode}
%    \end{macro}
%    \begin{macro}{\PdfRender@Set}
%    \begin{macrocode}
\def\PdfRender@Set#1{%
  \csname ifPdfRender@Active#1\endcsname
    \csname PdfRender@Set#1\expandafter\endcsname
  \fi
}
%    \end{macrocode}
%    \end{macro}
%    \begin{macro}{\PdfRender@Reset}
%    \begin{macrocode}
\def\PdfRender@Reset#1{%
  \csname ifPdfRender@Active#1\endcsname
    \csname PdfRender@Reset#1\expandafter\endcsname
  \fi
}
%    \end{macrocode}
%    \end{macro}
%    \begin{macro}{\PdfRender@ErrorInvalidValue}
%    \begin{macrocode}
\def\PdfRender@ErrorInvalidValue#1{%
  \PackageError{pdfrender}{%
    Ignoring parameter setting for `#1'\MessageBreak
    because of invalid value %
    `\csname PdfRender@Current#1\endcsname'%
  }\@ehc
  \expandafter\let\csname PdfRender@Current#1\endcsname\ltx@empty
}%
%    \end{macrocode}
%    \end{macro}
%    \begin{macro}{\PdfRender@SetValidate}
%    \begin{macrocode}
\ifPdfRender@Match
  \def\PdfRender@SetValidate#1#2#3{%
    \ifnum\pdfmatch{^(#2)$}{\csname PdfRender@Current#1\endcsname}=1 %
      \csname PdfRender@Set#1\expandafter\endcsname
    \else
      \PdfRender@ErrorInvalidValue{#1}%
    \fi
  }%
\else
  \def\PdfRender@SetValidate#1#2#3{%
    \expandafter\let\expandafter\PdfRender@Current
    \csname PdfRender@Current#1\endcsname
    #3%
    \ifx\PdfRender@Current\@empty
      \PdfRender@ErrorInvalidValue{#1}%
    \else
      \csname PdfRender@Set#1\expandafter\endcsname
    \fi
  }%
\fi
%    \end{macrocode}
%    \end{macro}
%    \begin{macro}{\PdfRender@SetValidateValues}
%    \begin{macrocode}
\def\PdfRender@SetValidateValues#1{%
  \ltx@IfUndefined{PdfRender@#1@\PdfRender@Current}{%
    \expandafter\let\csname PdfRender@Current#1\endcsname
                    \PdfRender@Current
    \PdfRender@ErrorInvalidValue{#1}%
  }{%
    \expandafter\edef\csname PdfRender@Current#1\endcsname{%
      \csname PdfRender@#1@\PdfRender@Current\endcsname
    }%
    \csname PdfRender@Set#1\endcsname
  }%
}
%    \end{macrocode}
%    \end{macro}
%    \begin{macro}{\PdfRender@OpValue}
%    \begin{macrocode}
\def\PdfRender@OpValue#1#2{#2\ltx@space#1}%
%    \end{macrocode}
%    \end{macro}
%    \begin{macro}{\PdfRender@OpName}
%    \begin{macrocode}
\def\PdfRender@OpName#1#2{/#2\ltx@space#1}%
%    \end{macrocode}
%    \end{macro}
%
% \subsection{Declare and setup parameters}
%
%    \begin{macrocode}
\PdfRender@NewClassValues{TextRenderingMode}%
                      {0}%
                      {\PdfRender@OpValue{Tr}}{%
  0=Fill,%
  1=Stroke,%
  2=FillStroke,%
  3=Invisible,%
  4=FillClip,%
  5=StrokeClip,%
  6=FillStrokeClip,%
  7=Clip,%
}%
\PdfRender@NewClass{LineWidth}{1}{\PdfRender@OpValue{w}}{%
  [0-9]+\string\.?[0-9]*|\string\.[0-9]+%
}{}%
\ltx@IfUndefined{dimexpr}{%
  \def\PdfRender@dimexpr{}%
}{%
  \let\PdfRender@dimexpr\dimexpr
}
\def\PdfRender@PostProcessLineWidth{%
  \begingroup
  \afterassignment\PdfRender@@PostProcessLineWidth
  \dimen0=\PdfRender@dimexpr\PdfRender@CurrentLineWidth bp %
  \PdfRender@let\PdfRender@relax\PdfRender@relax
}
\let\PdfRender@let\let
\let\PdfRender@relax\relax
\def\PdfRender@@PostProcessLineWidth#1\PdfRender@let{%
  \ifx\\#1\\%
    \endgroup
  \else
    \dimen0=.996264\dimen0 % 72/72.27
    \edef\x{\endgroup
      \def\noexpand\PdfRender@CurrentLineWidth{%
        \strip@pt\dimen0%
      }%
    }%
    \expandafter\x
  \fi
}
\PdfRender@NewClassValues{LineCapStyle}{0}{\PdfRender@OpValue{J}}{%
  0=Butt,%
  1=Round,%
  2=ProjectingSquare,%
}%
\PdfRender@NewClassValues{LineJoinStyle}{0}{\PdfRender@OpValue{j}}{%
  0=Miter,%
  1=Round,%
  2=Bevel,%
}%
\PdfRender@NewClass{MiterLimit}{10}{\PdfRender@OpValue{M}}{%
  [0-9]*[1-9][0-9]*\string\.?[0-9]*|%
  [0-9]*\string\.?[0-9]*[1-9][0-9]*%
}{}%
\PdfRender@NewClass{Flatness}{0}{\PdfRender@OpValue{i}}{%
  100(\string\.0*)?|[0-9][0-9](\string\.[0-9]*)?|\string\.[0-9]+%
}{}%
\PdfRender@NewClass{LineDashPattern}{[]0}{\PdfRender@OpValue{d}}{%
  \string\[%
  ( ?([0-9]+\string\.?[0-9]*|\string\.[0-9]+) ?)*%
  \string\] ?%
  ([0-9]+\string\.?[0-9]*|\string\.[0-9]+)%
}{}%
\PdfRender@NewClassValues{RenderingIntent}%
                      {RelativeColorimetric}%
                      {\PdfRender@OpName{ri}}{%
  AbsoluteColorimetric,%
  RelativeColorimetric,%
  Saturation,%
  Perceptual,%
}%
%    \end{macrocode}
%
% \subsection{Fill and stroke color support}
%
%    \begin{macrocode}
\PdfRender@define@key{PDFRENDER}{FillColor}{%
  \begingroup
    \def\PdfRender@Color{#1}%
    \ifx\PdfRender@Color\ltx@empty
      \global\let\PdfRender@FillColor\ltx@empty
    \else
      \PdfRender@ColorAvailable{%
        \PdfRender@TestBox{%
          \expandafter\PdfRender@TryColor\PdfRender@Color\ltx@empty
          \PdfRender@GetFillColor
          \ifx\PdfRender@FillColor\ltx@empty
            \@PackageWarning{pdfrender}{%
              Cannot extract fill color\MessageBreak
              from value `#1'%
            }%
          \fi
        }%
      }%
    \fi
  \endgroup
}
\PdfRender@define@key{PDFRENDER}{StrokeColor}{%
  \begingroup
    \def\PdfRender@Color{#1}%
    \ifx\PdfRender@Color\ltx@empty
      \global\let\PdfRender@StrokeColor\ltx@empty
    \else
      \PdfRender@ColorAvailable{%
        \PdfRender@TestBox{%
          \expandafter\PdfRender@TryColor\PdfRender@Color\ltx@empty
          \PdfRender@GetStrokeColor
          \ifx\PdfRender@StrokeColor\ltx@empty
            \@PackageWarning{pdfrender}{%
              Cannot extract stroke color\MessageBreak
              from value `#1'%
            }%
          \fi
        }%
      }%
    \fi
  \endgroup
}
%    \end{macrocode}
%    \begin{macro}{\PdfRender@ColorAvailable}
%    \begin{macrocode}
\def\PdfRender@ColorAvailable{%
  \@ifundefined{set@color}{%
    \@PackageError{pdfrender}{%
      Ignoring color options, because neither\MessageBreak
      package `color' nor package `xcolor' is loaded%
    }\@ehc
    \global\let\PdfRender@ColorAvailable\ltx@gobble
  }{%
    \global\let\PdfRender@ColorAvailable\ltx@firstofone
  }%
  \PdfRender@ColorAvailable
}
%    \end{macrocode}
%    \end{macro}
%    \begin{macro}{\PdfRender@TryColor}
%    \begin{macrocode}
\def\PdfRender@TryColor{%
  \@ifnextchar[\color\PdfRender@@TryColor
}
%    \end{macrocode}
%    \end{macro}
%    \begin{macro}{\PdfRender@@TryColor}
%    \begin{macrocode}
\def\PdfRender@@TryColor#1\ltx@empty{%
  \expandafter\color\expandafter{\PdfRender@Color}%
}
%    \end{macrocode}
%    \end{macro}
%    \begin{macro}{\PdfRender@SetColor}
%    \begin{macrocode}
\def\PdfRender@SetColor{%
  \chardef\PdfRender@NeedsCurrentColor=0 %
  \ifx\PdfRender@FillColor\ltx@empty
    \ifx\PdfRender@StrokeColor\ltx@empty
    \else
      \edef\PdfRender@CurrentColor{%
        \noexpand\PdfRender@FillColor\ltx@space\PdfRender@StrokeColor
      }%
      \chardef\PdfRender@NeedsCurrentColor=1 %
    \fi
  \else
    \ifx\PdfRender@StrokeColor\ltx@empty
      \edef\PdfRender@CurrentColor{%
        \PdfRender@FillColor\ltx@space\noexpand\PdfRender@StrokeColor
      }%
      \chardef\PdfRender@NeedsCurrentColor=2 %
    \else
      \edef\current@color{%
        \PdfRender@FillColor\ltx@space\PdfRender@StrokeColor
      }%
      \set@color
    \fi
  \fi
  \ifnum\PdfRender@NeedsCurrentColor=1 %
    \PdfRender@GetFillColor
    \ifx\PdfRender@FillColor\ltx@empty
      \@PackageWarning{pdfrender}{%
        Cannot extract current fill color%
      }%
    \else
      \edef\current@color{\PdfRender@CurrentColor}%
      \set@color
    \fi
  \else
    \ifnum\PdfRender@NeedsCurrentColor=2 %
      \PdfRender@GetStrokeColor
      \ifx\PdfRender@StrokeColor\ltx@empty
        \@PackageWarning{pdfrender}{%
          Cannot extract current stroke color%
        }%
      \else
        \edef\current@color{\PdfRender@CurrentColor}%
        \set@color
      \fi
    \fi
  \fi
}
%    \end{macrocode}
%    \end{macro}
%    \begin{macro}{\PdfRender@PatternFillColor}
%    \begin{macrocode}
\edef\PdfRender@PatternFillColor{ % space
  (%
    [0-9\string\.]+ g|%
    [0-9\string\.]+ [0-9\string\.]+ [0-9\string\.]+ rg|%
    [0-9\string\.]+ [0-9\string\.]+ %
    [0-9\string\.]+ [0-9\string\.]+ k%
  ) % space
  (.*)$%
}
%    \end{macrocode}
%    \end{macro}
%    \begin{macro}{\PdfRender@PatternStrokeColor}
%    \begin{macrocode}
\edef\PdfRender@PatternStrokeColor{ % space
  (%
    [0-9\string\.]+ G|%
    [0-9\string\.]+ [0-9\string\.]+ [0-9\string\.]+ RG|%
    [0-9\string\.]+ [0-9\string\.]+ %
    [0-9\string\.]+ [0-9\string\.]+ K%
  ) % space
  (.*)$%
}
%    \end{macrocode}
%    \end{macro}
%    \begin{macro}{\PdfRender@MatchPattern}
%    \begin{macrocode}
\def\PdfRender@MatchPattern#1{%
  \ifnum\pdfmatch{\PdfRender@Pattern}{\PdfRender@String}=1 %
    \xdef#1{%
      \expandafter\strip@prefix\pdflastmatch 1%
    }%
    \edef\PdfRender@String{%
      \expandafter\strip@prefix\pdflastmatch 2%
    }%
    \ifx\PdfRender@String\ltx@empty
    \else
      \expandafter\expandafter\expandafter\PdfRender@MatchPattern
      \expandafter\expandafter\expandafter#1%
    \fi
  \fi
}
%    \end{macrocode}
%    \end{macro}
%    \begin{macro}{\PdfRender@GetFillColor}
%    \begin{macrocode}
\def\PdfRender@GetFillColor{%
  \global\let\PdfRender@FillColor\ltx@empty
  \begingroup
    \ifPdfRender@Match
      \let\PdfRender@Pattern\PdfRender@PatternFillColor
      \edef\PdfRender@String{\ltx@space\current@color\ltx@space}%
      \PdfRender@MatchPattern\PdfRender@FillColor
    \else
      \edef\current@color{\current@color\ltx@space}%
      \let\PdfRender@OP\relax
      \PdfRender@FindOp{g}0%
      \PdfRender@FindOp{G}1%
      \PdfRender@FindOp{rg}0%
      \PdfRender@FindOp{RG}1%
      \PdfRender@FindOp{k}0%
      \PdfRender@FindOp{K}1%
      \PdfRender@FilterOp 0\PdfRender@FillColor
    \fi
  \endgroup
}
%    \end{macrocode}
%    \end{macro}
%    \begin{macro}{\PdfRender@GetStrokeColor}
%    \begin{macrocode}
\def\PdfRender@GetStrokeColor{%
  \global\let\PdfRender@StrokeColor\ltx@empty
  \begingroup
    \ifPdfRender@Match
      \let\PdfRender@Pattern\PdfRender@PatternStrokeColor
      \edef\PdfRender@String{\ltx@space\current@color\ltx@space}%
      \PdfRender@MatchPattern\PdfRender@StrokeColor
    \else
      \edef\current@color{\current@color\ltx@space}%
      \let\PdfRender@OP\relax
      \PdfRender@FindOp{g}0%
      \PdfRender@FindOp{G}1%
      \PdfRender@FindOp{rg}0%
      \PdfRender@FindOp{RG}1%
      \PdfRender@FindOp{k}0%
      \PdfRender@FindOp{K}1%
      \PdfRender@FilterOp 1\PdfRender@StrokeColor
    \fi
  \endgroup
}
%    \end{macrocode}
%    \end{macro}
%    \begin{macrocode}
\ifPdfRender@Match
  \expandafter\PdfRender@AtEnd
\fi%
%    \end{macrocode}
%    \begin{macro}{\PdfRender@FindOp}
%    \begin{macrocode}
\def\PdfRender@FindOp#1#2{%
  \def\PdfRender@temp##1 #1 ##2\@nil{%
    ##1%
    \ifx\\##2\\%
      \expandafter\@gobble
    \else
      \PdfRender@OP{#1}#2%
      \expandafter\@firstofone
    \fi
    {%
      \PdfRender@temp##2\@nil
    }%
  }%
  \edef\current@color{%
    \@firstofone{\expandafter\PdfRender@temp\current@color} #1 \@nil
  }%
}
%    \end{macrocode}
%    \end{macro}
%    \begin{macro}{\PdfRender@FilterOp}
%    \begin{macrocode}
\def\PdfRender@FilterOp#1#2{%
  \expandafter\PdfRender@@FilterOp\expandafter#1\expandafter#2%
      \current@color\PdfRender@OP{}{}%
}
%    \end{macrocode}
%    \end{macro}
%    \begin{macro}{\PdfRender@@FilterOp}
%    \begin{macrocode}
\def\PdfRender@@FilterOp#1#2#3\PdfRender@OP#4#5{%
  \ifx\\#4#5\\%
  \else
    \ifnum#1=#5 %
      \xdef#2{#3 #4}%
    \fi
    \expandafter\PdfRender@@FilterOp\expandafter#1\expandafter#2%
  \fi
}
%    \end{macrocode}
%    \end{macro}
%
%    \begin{macrocode}
\PdfRender@AtEnd%
%</package>
%    \end{macrocode}
%% \section{Installation}
%
% \subsection{Download}
%
% \paragraph{Package.} This package is available on
% CTAN\footnote{\CTANpkg{pdfrender}}:
% \begin{description}
% \item[\CTAN{macros/latex/contrib/oberdiek/pdfrender.dtx}] The source file.
% \item[\CTAN{macros/latex/contrib/oberdiek/pdfrender.pdf}] Documentation.
% \end{description}
%
%
% \paragraph{Bundle.} All the packages of the bundle `oberdiek'
% are also available in a TDS compliant ZIP archive. There
% the packages are already unpacked and the documentation files
% are generated. The files and directories obey the TDS standard.
% \begin{description}
% \item[\CTANinstall{install/macros/latex/contrib/oberdiek.tds.zip}]
% \end{description}
% \emph{TDS} refers to the standard ``A Directory Structure
% for \TeX\ Files'' (\CTANpkg{tds}). Directories
% with \xfile{texmf} in their name are usually organized this way.
%
% \subsection{Bundle installation}
%
% \paragraph{Unpacking.} Unpack the \xfile{oberdiek.tds.zip} in the
% TDS tree (also known as \xfile{texmf} tree) of your choice.
% Example (linux):
% \begin{quote}
%   |unzip oberdiek.tds.zip -d ~/texmf|
% \end{quote}
%
% \subsection{Package installation}
%
% \paragraph{Unpacking.} The \xfile{.dtx} file is a self-extracting
% \docstrip\ archive. The files are extracted by running the
% \xfile{.dtx} through \plainTeX:
% \begin{quote}
%   \verb|tex pdfrender.dtx|
% \end{quote}
%
% \paragraph{TDS.} Now the different files must be moved into
% the different directories in your installation TDS tree
% (also known as \xfile{texmf} tree):
% \begin{quote}
% \def\t{^^A
% \begin{tabular}{@{}>{\ttfamily}l@{ $\rightarrow$ }>{\ttfamily}l@{}}
%   pdfrender.sty & tex/generic/oberdiek/pdfrender.sty\\
%   pdfrender.pdf & doc/latex/oberdiek/pdfrender.pdf\\
%   pdfrender.dtx & source/latex/oberdiek/pdfrender.dtx\\
% \end{tabular}^^A
% }^^A
% \sbox0{\t}^^A
% \ifdim\wd0>\linewidth
%   \begingroup
%     \advance\linewidth by\leftmargin
%     \advance\linewidth by\rightmargin
%   \edef\x{\endgroup
%     \def\noexpand\lw{\the\linewidth}^^A
%   }\x
%   \def\lwbox{^^A
%     \leavevmode
%     \hbox to \linewidth{^^A
%       \kern-\leftmargin\relax
%       \hss
%       \usebox0
%       \hss
%       \kern-\rightmargin\relax
%     }^^A
%   }^^A
%   \ifdim\wd0>\lw
%     \sbox0{\small\t}^^A
%     \ifdim\wd0>\linewidth
%       \ifdim\wd0>\lw
%         \sbox0{\footnotesize\t}^^A
%         \ifdim\wd0>\linewidth
%           \ifdim\wd0>\lw
%             \sbox0{\scriptsize\t}^^A
%             \ifdim\wd0>\linewidth
%               \ifdim\wd0>\lw
%                 \sbox0{\tiny\t}^^A
%                 \ifdim\wd0>\linewidth
%                   \lwbox
%                 \else
%                   \usebox0
%                 \fi
%               \else
%                 \lwbox
%               \fi
%             \else
%               \usebox0
%             \fi
%           \else
%             \lwbox
%           \fi
%         \else
%           \usebox0
%         \fi
%       \else
%         \lwbox
%       \fi
%     \else
%       \usebox0
%     \fi
%   \else
%     \lwbox
%   \fi
% \else
%   \usebox0
% \fi
% \end{quote}
% If you have a \xfile{docstrip.cfg} that configures and enables \docstrip's
% TDS installing feature, then some files can already be in the right
% place, see the documentation of \docstrip.
%
% \subsection{Refresh file name databases}
%
% If your \TeX~distribution
% (\TeX\,Live, \mikTeX, \dots) relies on file name databases, you must refresh
% these. For example, \TeX\,Live\ users run \verb|texhash| or
% \verb|mktexlsr|.
%
% \subsection{Some details for the interested}
%
% \paragraph{Unpacking with \LaTeX.}
% The \xfile{.dtx} chooses its action depending on the format:
% \begin{description}
% \item[\plainTeX:] Run \docstrip\ and extract the files.
% \item[\LaTeX:] Generate the documentation.
% \end{description}
% If you insist on using \LaTeX\ for \docstrip\ (really,
% \docstrip\ does not need \LaTeX), then inform the autodetect routine
% about your intention:
% \begin{quote}
%   \verb|latex \let\install=y% \iffalse meta-comment
%
% File: pdfrender.dtx
% Version: 2019/12/29 v1.6
% Info: Access to some PDF graphics parameters
%
% Copyright (C)
%    2010 Heiko Oberdiek
%    2016-2019 Oberdiek Package Support Group
%    https://github.com/ho-tex/oberdiek/issues
%
% This work may be distributed and/or modified under the
% conditions of the LaTeX Project Public License, either
% version 1.3c of this license or (at your option) any later
% version. This version of this license is in
%    https://www.latex-project.org/lppl/lppl-1-3c.txt
% and the latest version of this license is in
%    https://www.latex-project.org/lppl.txt
% and version 1.3 or later is part of all distributions of
% LaTeX version 2005/12/01 or later.
%
% This work has the LPPL maintenance status "maintained".
%
% The Current Maintainers of this work are
% Heiko Oberdiek and the Oberdiek Package Support Group
% https://github.com/ho-tex/oberdiek/issues
%
% The Base Interpreter refers to any `TeX-Format',
% because some files are installed in TDS:tex/generic//.
%
% This work consists of the main source file pdfrender.dtx
% and the derived files
%    pdfrender.sty, pdfrender.pdf, pdfrender.ins, pdfrender.drv,
%    pdfrender-test1.tex, pdfrender-test2.tex,
%    pdfrender-test3.tex, pdfrender-test4.tex,
%    pdfrender-test5.tex.
%
% Distribution:
%    CTAN:macros/latex/contrib/oberdiek/pdfrender.dtx
%    CTAN:macros/latex/contrib/oberdiek/pdfrender.pdf
%
% Unpacking:
%    (a) If pdfrender.ins is present:
%           tex pdfrender.ins
%    (b) Without pdfrender.ins:
%           tex pdfrender.dtx
%    (c) If you insist on using LaTeX
%           latex \let\install=y\input{pdfrender.dtx}
%        (quote the arguments according to the demands of your shell)
%
% Documentation:
%    (a) If pdfrender.drv is present:
%           latex pdfrender.drv
%    (b) Without pdfrender.drv:
%           latex pdfrender.dtx; ...
%    The class ltxdoc loads the configuration file ltxdoc.cfg
%    if available. Here you can specify further options, e.g.
%    use A4 as paper format:
%       \PassOptionsToClass{a4paper}{article}
%
%    Programm calls to get the documentation (example):
%       pdflatex pdfrender.dtx
%       makeindex -s gind.ist pdfrender.idx
%       pdflatex pdfrender.dtx
%       makeindex -s gind.ist pdfrender.idx
%       pdflatex pdfrender.dtx
%
% Installation:
%    TDS:tex/generic/oberdiek/pdfrender.sty
%    TDS:doc/latex/oberdiek/pdfrender.pdf
%    TDS:source/latex/oberdiek/pdfrender.dtx
%
%<*ignore>
\begingroup
  \catcode123=1 %
  \catcode125=2 %
  \def\x{LaTeX2e}%
\expandafter\endgroup
\ifcase 0\ifx\install y1\fi\expandafter
         \ifx\csname processbatchFile\endcsname\relax\else1\fi
         \ifx\fmtname\x\else 1\fi\relax
\else\csname fi\endcsname
%</ignore>
%<*install>
\input docstrip.tex
\Msg{************************************************************************}
\Msg{* Installation}
\Msg{* Package: pdfrender 2019/12/29 v1.6 Access to some PDF graphics parameters (HO)}
\Msg{************************************************************************}

\keepsilent
\askforoverwritefalse

\let\MetaPrefix\relax
\preamble

This is a generated file.

Project: pdfrender
Version: 2019/12/29 v1.6

Copyright (C)
   2010 Heiko Oberdiek
   2016-2019 Oberdiek Package Support Group

This work may be distributed and/or modified under the
conditions of the LaTeX Project Public License, either
version 1.3c of this license or (at your option) any later
version. This version of this license is in
   https://www.latex-project.org/lppl/lppl-1-3c.txt
and the latest version of this license is in
   https://www.latex-project.org/lppl.txt
and version 1.3 or later is part of all distributions of
LaTeX version 2005/12/01 or later.

This work has the LPPL maintenance status "maintained".

The Current Maintainers of this work are
Heiko Oberdiek and the Oberdiek Package Support Group
https://github.com/ho-tex/oberdiek/issues


The Base Interpreter refers to any `TeX-Format',
because some files are installed in TDS:tex/generic//.

This work consists of the main source file pdfrender.dtx
and the derived files
   pdfrender.sty, pdfrender.pdf, pdfrender.ins, pdfrender.drv,
   pdfrender-test1.tex, pdfrender-test2.tex,
   pdfrender-test3.tex, pdfrender-test4.tex,
   pdfrender-test5.tex.

\endpreamble
\let\MetaPrefix\DoubleperCent

\generate{%
  \file{pdfrender.ins}{\from{pdfrender.dtx}{install}}%
  \file{pdfrender.drv}{\from{pdfrender.dtx}{driver}}%
  \usedir{tex/generic/oberdiek}%
  \file{pdfrender.sty}{\from{pdfrender.dtx}{package}}%
%  \usedir{doc/latex/oberdiek/test}%
%  \file{pdfrender-test1.tex}{\from{pdfrender.dtx}{test1}}%
%  \file{pdfrender-test2.tex}{\from{pdfrender.dtx}{test2}}%
%  \file{pdfrender-test3.tex}{\from{pdfrender.dtx}{test3}}%
%  \file{pdfrender-test4.tex}{\from{pdfrender.dtx}{test4}}%
%  \file{pdfrender-test5.tex}{\from{pdfrender.dtx}{test5}}%
}

\catcode32=13\relax% active space
\let =\space%
\Msg{************************************************************************}
\Msg{*}
\Msg{* To finish the installation you have to move the following}
\Msg{* file into a directory searched by TeX:}
\Msg{*}
\Msg{*     pdfrender.sty}
\Msg{*}
\Msg{* To produce the documentation run the file `pdfrender.drv'}
\Msg{* through LaTeX.}
\Msg{*}
\Msg{* Happy TeXing!}
\Msg{*}
\Msg{************************************************************************}

\endbatchfile
%</install>
%<*ignore>
\fi
%</ignore>
%<*driver>
\NeedsTeXFormat{LaTeX2e}
\ProvidesFile{pdfrender.drv}%
  [2019/12/29 v1.6 Access to some PDF graphics parameters (HO)]%
\documentclass{ltxdoc}
\usepackage{holtxdoc}[2011/11/22]
\usepackage{longtable}
\renewcommand*{\arraystretch}{1.2}
\begin{document}
  \DocInput{pdfrender.dtx}%
\end{document}
%</driver>
% \fi
%
%
%
% \GetFileInfo{pdfrender.drv}
%
% \title{The \xpackage{pdfrender} package}
% \date{2019/12/29 v1.6}
% \author{Heiko Oberdiek\thanks
% {Please report any issues at \url{https://github.com/ho-tex/oberdiek/issues}}}
%
% \maketitle
%
% \begin{abstract}
% The PDF format has some graphics parameter like
% line width or text rendering mode. This package
% provides an interface for setting these parameters.
% \end{abstract}
%
% \tableofcontents
%
% \section{Documentation}
%
% This package \xpackage{pdfrender} defines an interface for
% PDF specific parameters that affects the rendering of
% graphics or text. The interface and its implementation
% uses the same technique as package \xpackage{color}
% for color settings. Therefore this package is loaded to
% enable \LaTeX's color interface.
%
% At different places \LaTeX\ uses \cs{normalcolor} to avoid that
% header, footer or floats are print in the current color of the
% main text. \cs{setgroup@color} is used to start a save box
% with the color that is set at box saving time.
% Package \xpackage{pdfrender} extends these macros to
% add its own hooks of its parameters. Therefore \LaTeX3 should
% generalize \LaTeXe's color interface.
%
% \subsection{Usage}
%
% In \LaTeX\ the package is loaded as normal package. Options
% are not defined for this package.
% \begin{quote}
% |\usepackage{pdfrender}|
% \end{quote}
%
% This package can also be used in \plainTeX\ and even ini\TeX:
% \begin{quote}
% |input pdfrender.sty|
% \end{quote}
%
% \subsection{Macros}
%
% \begin{declcs}{pdfrender} \M{key value list}
% \end{declcs}
% The first parameter \meta{key value list} contains a
% list of parameter settings. The key entry is the parameter name.
% The macro works like
% \cs{color} (without optional argument) for color setting.
%
% \begin{declcs}{textpdfrender} \M{key value list} \M{text}
% \end{declcs}
% In the same way as \cs{pdfrender} the first argument
% specifies the parameters that should be set. This parameter
% setting affects \meta{text} only. Basically it works
% the same way as \cs{textcolor} (without optional argument).
%
% \subsection{Parameters}
%
% The following table shows an overview for the supported
% parameters and values:
%
% \begin{longtable}{@{}l|l|l@{}}
% \bfseries Parameter & \bfseries Value & \bfseries Alias\\*
% \hline
% \endhead
% \xoption{TextRenderingMode}
%   & \texttt{0} & \texttt{Fill}\\*
%   & \texttt{1} & \texttt{Stroke}\\*
%   & \texttt{2} & \texttt{FillStroke}\\*
%   & \texttt{3} & \texttt{Invisible}\\*
%   & \texttt{4} & \texttt{FillClip}\\*
%   & \texttt{5} & \texttt{StrokeClip}\\*
%   & \texttt{6} & \texttt{FillStrokeClip}\\*
%   & \texttt{7} & \texttt{Clip}\\
% \hline
% \xoption{LineWidth}
%   & \itshape positive number, & \itshape \TeX\ dimen\\*[-.2ex]
%   & \qquad \itshape unit is bp & \\
% \hline
% \xoption{LineCapStyle}
%   & \texttt{0} & \texttt{Butt}\\*
%   & \texttt{1} & \texttt{Round}\\*
%   & \texttt{2} & \texttt{ProjectingSquare}\\
% \hline
% \xoption{LineJoinStyle}
%   & \texttt{0} & \texttt{Miter}\\*
%   & \texttt{1} & \texttt{Round}\\*
%   & \texttt{2} & \texttt{Bevel}\\
% \hline
% \xoption{MiterLimit}
%   & \multicolumn{2}{l@{}}{\itshape positive number} \\
% \hline
% \xoption{Flatness}
%   & \multicolumn{2}{l@{}}{\itshape number between 0 and 100} \\
% \hline
% \xoption{LineDashPattern}
%   & \multicolumn{2}{l@{}}{^^A
%       \itshape numbers in square brackets,^^A
%     } \\*[-.2ex]
%   & \multicolumn{2}{l@{}}{^^A
%       \qquad \itshape followed by number,^^A
%     } \\*
%   & \multicolumn{2}{l@{}}{^^A
%       \qquad \itshape units are bp^^A
%     } \\
% \hline
% \xoption{RenderingIntent}
%   & \multicolumn{2}{l@{}}{\texttt{AbsoluteColorimetric}} \\*
%   & \multicolumn{2}{l@{}}{\texttt{RelativeColorimetric}} \\*
%   & \multicolumn{2}{l@{}}{\texttt{Saturation}} \\*
%   & \multicolumn{2}{l@{}}{\texttt{Perceptual}} \\
% \hline
% \xoption{FillColor}
%   & & \itshape color specification \\
% \hline
% \xoption{StrokeColor}
%   & & \itshape color specification \\
% \end{longtable}
%
% \subsubsection{Details}
%
% The description and specification of these parameters
% are available in the PDF specification \cite{pdf-spec}.
% Therefore they are not repeated here.
%
% \begin{description}
% \item[Value:]
%   The values in the second column lists or describe
%   the values that are specified by the PDF specification.
% \item[Alias:] Instead of magic numbers the package also
%   defines some aliases that can be given as value.
%   Example: |LineCapStyle=Round| has the same effect as
%   |LineCapStyle=1|.
% \item[Number:] The term \emph{number} means an integer
%   or real number. The real number is given as plain
%   decimal number without exponent. The decimal separator is a period.
%   At least one digit must be present.
% \item[\xoption{LineWidth}:] As alias a \TeX\ dimen specification
%   can be given. This includes explicit specifications with number
%   and unit, e.g. |LineWidth=0.5pt|. Also \LaTeX\ length registers
%   may be used. If \eTeX's \cs{dimexpr} is available, then it is
%   automatically added. However package \xpackage{calc} is not
%   supported.
% \item[\xoption{FillColor}, \xoption{StrokeColor}:]
%   Package \xpackage{color} or \xpackage{xcolor} must be loaded
%   before these options can be used (since version 1.2).
%   \LaTeX's color support sets both colors at the same time
%   to the same value. However parameter \xoption{TextRenderingMode}
%   offers the value \texttt{FillStroke} that makes only sense, if
%   the two color types can be set separately.
%   If one of the options \xoption{FillColor} or \xoption{StrokeColor}
%   is specified, then also the color is set. For compatibility
%   with the \LaTeX\ color packages (\xpackage{color} or \xpackage{xcolor}),
%   always both colors must be set. Thus if one of them is not specified,
%   it is taken from the current color.
%
%   Both options \xoption{FillColor} and \xoption{StrokeColor}
%   expect a \LaTeX\ color specification as value.
%   Also the optional color model argument is supported.
%   Example:
%   \begin{quote}
%     |FillColor=yellow|,\\
%     |StrokeColor=[cmyk]{1,.5,0,0}|\\
%   \end{quote}
% \end{description}
%
% \subsection{Color stack}
%
% If the \pdfTeX\ version provides color stacks, then each
% parameter is assigned a page based color stack. The assignment
% of a stack takes place, when its parameter is set the first time.
% This avoids the use of color stacks that are not needed.
%
% \StopEventually{
% }
%
% \section{Implementation}
%    \begin{macrocode}
%<*package>
%    \end{macrocode}
%    Reload check, especially if the package is not used with \LaTeX.
%    \begin{macrocode}
\begingroup\catcode61\catcode48\catcode32=10\relax%
  \catcode13=5 % ^^M
  \endlinechar=13 %
  \catcode35=6 % #
  \catcode39=12 % '
  \catcode44=12 % ,
  \catcode45=12 % -
  \catcode46=12 % .
  \catcode58=12 % :
  \catcode64=11 % @
  \catcode123=1 % {
  \catcode125=2 % }
  \expandafter\let\expandafter\x\csname ver@pdfrender.sty\endcsname
  \ifx\x\relax % plain-TeX, first loading
  \else
    \def\empty{}%
    \ifx\x\empty % LaTeX, first loading,
      % variable is initialized, but \ProvidesPackage not yet seen
    \else
      \expandafter\ifx\csname PackageInfo\endcsname\relax
        \def\x#1#2{%
          \immediate\write-1{Package #1 Info: #2.}%
        }%
      \else
        \def\x#1#2{\PackageInfo{#1}{#2, stopped}}%
      \fi
      \x{pdfrender}{The package is already loaded}%
      \aftergroup\endinput
    \fi
  \fi
\endgroup%
%    \end{macrocode}
%    Package identification:
%    \begin{macrocode}
\begingroup\catcode61\catcode48\catcode32=10\relax%
  \catcode13=5 % ^^M
  \endlinechar=13 %
  \catcode35=6 % #
  \catcode39=12 % '
  \catcode40=12 % (
  \catcode41=12 % )
  \catcode44=12 % ,
  \catcode45=12 % -
  \catcode46=12 % .
  \catcode47=12 % /
  \catcode58=12 % :
  \catcode64=11 % @
  \catcode91=12 % [
  \catcode93=12 % ]
  \catcode123=1 % {
  \catcode125=2 % }
  \expandafter\ifx\csname ProvidesPackage\endcsname\relax
    \def\x#1#2#3[#4]{\endgroup
      \immediate\write-1{Package: #3 #4}%
      \xdef#1{#4}%
    }%
  \else
    \def\x#1#2[#3]{\endgroup
      #2[{#3}]%
      \ifx#1\@undefined
        \xdef#1{#3}%
      \fi
      \ifx#1\relax
        \xdef#1{#3}%
      \fi
    }%
  \fi
\expandafter\x\csname ver@pdfrender.sty\endcsname
\ProvidesPackage{pdfrender}%
  [2019/12/29 v1.6 Access to some PDF graphics parameters (HO)]%
%    \end{macrocode}
%
%    \begin{macrocode}
\begingroup\catcode61\catcode48\catcode32=10\relax%
  \catcode13=5 % ^^M
  \endlinechar=13 %
  \catcode123=1 % {
  \catcode125=2 % }
  \catcode64=11 % @
  \def\x{\endgroup
    \expandafter\edef\csname PdfRender@AtEnd\endcsname{%
      \endlinechar=\the\endlinechar\relax
      \catcode13=\the\catcode13\relax
      \catcode32=\the\catcode32\relax
      \catcode35=\the\catcode35\relax
      \catcode61=\the\catcode61\relax
      \catcode64=\the\catcode64\relax
      \catcode123=\the\catcode123\relax
      \catcode125=\the\catcode125\relax
    }%
  }%
\x\catcode61\catcode48\catcode32=10\relax%
\catcode13=5 % ^^M
\endlinechar=13 %
\catcode35=6 % #
\catcode64=11 % @
\catcode123=1 % {
\catcode125=2 % }
\def\TMP@EnsureCode#1#2{%
  \edef\PdfRender@AtEnd{%
    \PdfRender@AtEnd
    \catcode#1=\the\catcode#1\relax
  }%
  \catcode#1=#2\relax
}
\TMP@EnsureCode{10}{12}% ^^J
\TMP@EnsureCode{36}{3}% $
\TMP@EnsureCode{39}{12}% '
\TMP@EnsureCode{40}{12}% (
\TMP@EnsureCode{41}{12}% )
\TMP@EnsureCode{42}{12}% *
\TMP@EnsureCode{43}{12}% +
\TMP@EnsureCode{44}{12}% ,
\TMP@EnsureCode{45}{12}% -
\TMP@EnsureCode{46}{12}% .
\TMP@EnsureCode{47}{12}% /
\TMP@EnsureCode{58}{12}% :
\TMP@EnsureCode{59}{12}% ;
\TMP@EnsureCode{60}{12}% <
\TMP@EnsureCode{62}{12}% >
\TMP@EnsureCode{63}{12}% ?
\TMP@EnsureCode{91}{12}% [
\TMP@EnsureCode{93}{12}% ]
\TMP@EnsureCode{94}{7}% ^ (superscript)
\TMP@EnsureCode{96}{12}% `
\TMP@EnsureCode{124}{12}% |
%    \end{macrocode}
%
%    \begin{macrocode}
\def\PdfRender@AtEndHook{}
\expandafter\def\expandafter\PdfRender@AtEnd\expandafter{%
  \expandafter\PdfRender@AtEndHook
  \PdfRender@AtEnd
  \endinput
}
%    \end{macrocode}
%
% \subsection{Look for \pdfTeX, its mode and features}
%
%    \begin{macro}{\PdfRender@newif}
%    \begin{macrocode}
\def\PdfRender@newif#1{%
  \expandafter\edef\csname PdfRender@#1true\endcsname{%
    \let
    \expandafter\noexpand\csname ifPdfRender@#1\endcsname
    \noexpand\iftrue
  }%
  \expandafter\edef\csname PdfRender@#1false\endcsname{%
    \let
    \expandafter\noexpand\csname ifPdfRender@#1\endcsname
    \noexpand\iffalse
  }%
  \csname PdfRender@#1false\endcsname
}
%    \end{macrocode}
%    \end{macro}
%
%    \begin{macro}{\ifPdfRender@Stack}
%    \begin{macrocode}
\PdfRender@newif{Stack}
%    \end{macrocode}
%    \end{macro}
%    \begin{macro}{\ifPdfRender@Match}
%    \begin{macrocode}
\PdfRender@newif{Match}
%    \end{macrocode}
%    \end{macro}
%
%    \begin{macro}{\PdfRender@RequirePackage}
%    \begin{macrocode}
\begingroup\expandafter\expandafter\expandafter\endgroup
\expandafter\ifx\csname RequirePackage\endcsname\relax
  \def\PdfRender@RequirePackage#1[#2]{%
    \expandafter\def\expandafter\PdfRender@AtEndHook\expandafter{%
      \PdfRender@AtEndHook
      \ltx@ifpackagelater{#1}{#2}{}{%
        \@PackageWarningNoLine{pdfrender}{%
          You have requested version\MessageBreak
          `#2' of package `#1',\MessageBreak
          but only version\MessageBreak
          `\csname ver@#1.\ltx@pkgextension\endcsname'\MessageBreak
          is available%
        }%
      }%
    }%
    \input #1.sty\relax
  }%
\else
  \let\PdfRender@RequirePackage\RequirePackage
\fi
%    \end{macrocode}
%    \end{macro}
% Luatex compatibility
%    \begin{macrocode}
\ifx\pdfextension\@undefined\else
  \def\pdfcolorstackinit {\pdffeedback colorstackinit}
  \protected\def\pdfcolorstack     {\pdfextension colorstack}
  \protected\def\pdfliteral        {\pdfextension literal}
\fi
%    \end{macrocode}
%
%    \begin{macrocode}
\PdfRender@RequirePackage{iftex}[2019/11/07]
\PdfRender@RequirePackage{infwarerr}[2007/09/09]
\PdfRender@RequirePackage{ltxcmds}[2010/01/28]
%    \end{macrocode}
%
%    \begin{macrocode}
\ifpdf
  \ltx@IfUndefined{pdfcolorstackinit}{%
    \@PackageWarning{pdfrender}{%
      Missing \string\pdfcolorstackinit
    }%
  }{%
    \PdfRender@Stacktrue
  }%
  \ltx@IfUndefined{pdfmatch}{%
    \@PackageInfoNoLine{pdfrender}{%
      \string\pdfmatch\ltx@space not found. %
      Therefore the values\MessageBreak
      of some parameters are not validated%
    }%
  }{%
    \PdfRender@Matchtrue
  }%
\else
  \@PackageWarning{pdfrender}{%
    Missing pdfTeX in PDF mode%
  }%
%    \end{macrocode}
%    \begin{macrocode}
  \ltx@IfUndefined{newcommand}{%
%    \end{macrocode}
%    \begin{macro}{\pdfrender}
%    \begin{macrocode}
    \def\pdfrender#1{}%
%    \end{macrocode}
%    \end{macro}
%    \begin{macro}{\textpdfrender}
%    \begin{macrocode}
    \long\def\textpdfrender#1#2{#2}%
%    \end{macrocode}
%    \end{macro}
%    \begin{macrocode}
  }{%
%    \end{macrocode}
%    \begin{macro}{\pdfrender}
%    \begin{macrocode}
    \newcommand*{\pdfrender}[1]{}%
%    \end{macrocode}
%    \end{macro}
%    \begin{macro}{\textpdfrender}
%    \begin{macrocode}
    \newcommand{\textpdfrender}[2]{#2}%
%    \end{macrocode}
%    \end{macro}
%    \begin{macrocode}
  }%
  \expandafter\PdfRender@AtEnd
\fi%
%    \end{macrocode}
%
% \subsection{Enable color support of \LaTeX}
%
%    \begin{macrocode}
\ltx@ifpackageloaded{color}{}{%
  \def\color@setgroup{\begingroup\set@color}%
  \let\color@begingroup\begingroup
  \def\color@endgroup{\endgraf\endgroup}%
  \def\color@hbox{\hbox\bgroup\color@begingroup}%
  \def\color@vbox{\vbox\bgroup\color@begingroup}%
  \def\color@endbox{\color@endgroup\egroup}%
  \ltx@ifundefined{bgroup}{%
    \let\bgroup={\let\egroup=}%
  }{}%
  \ltx@ifundefined{endgraf}{%
    \let\endgraf=\par
  }{}%
}
%    \end{macrocode}
%
% \subsection{Hook into \cs{normalcolor}}
%
%    The problem is that packages \xpackage{color} and \xpackage{xcolor}
%    each overwrite \cs{normalcolor}. For example, after the package
%    loading order \xpackage{color}, \xpackage{pdfrender} and \xpackage{xcolor}
%    the patched version of \cs{normalcolor} is overwritten by
%    package \xpackage{xcolor}. Also using \cs{AtBeginDocument} for
%    patching is not enough. If package \xpackage{hyperref} is loaded
%    later, it might load package \xpackage{color} using \cs{AtBeginDocument}.
%    \begin{macro}{\PdfRender@NormalColorHook}
%    \begin{macrocode}
\def\PdfRender@NormalColorHook{}
%    \end{macrocode}
%    \end{macro}
%    \begin{macro}{\PdfRender@ColorSetGroupHook}
%    \begin{macrocode}
\def\PdfRender@ColorSetGroupHook{}
%    \end{macrocode}
%    \end{macro}
%    \begin{macro}{\PdfRender@TestBox}
%    \begin{macrocode}
\def\PdfRender@TestBox#1{%
  \setbox0=\color@hbox#1\color@endbox
}
%    \end{macrocode}
%    \end{macro}
%    \begin{macro}{\PdfRender@PatchNormalColor}
%    \begin{macrocode}
\def\PdfRender@PatchNormalColor{%
  \ltx@ifundefined{normalcolor}{%
    \gdef\normalcolor{\PdfRender@NormalColorHook}%
  }{%
    \begingroup
      \def\PdfRender@NormalColorHook{\let\PdfRender@temp=Y}%
      \PdfRender@TestBox{%
        \let\set@color\relax
        \normalcolor
        \ifx\PdfRender@temp Y%
        \else
          \ltx@GlobalAppendToMacro\normalcolor{%
            \PdfRender@NormalColorHook
          }%
        \fi
      }%
    \endgroup
  }%
  \ifx\@nodocument\relax
    \global\let\PdfRender@PatchNormalColor\relax
  \fi
}%
%    \end{macrocode}
%    \end{macro}
%    \begin{macro}{\PdfRender@PatchColorSetGroup}
%    \begin{macrocode}
\def\PdfRender@PatchColorSetGroup{%
  \begingroup
    \def\PdfRender@ColorSetGroupHook{\let\PdfRender@temp=Y}%
    \PdfRender@TestBox{%
      \let\set@color\relax
      \color@setgroup\color@endgroup
      \ifx\PdfRender@temp Y%
      \else
        \ltx@GlobalAppendToMacro\color@setgroup{%
          \PdfRender@ColorSetGroupHook
        }%
      \fi
    }%
  \endgroup
  \ifx\@nodocument\relax
    \global\let\PdfRender@PatchColorSetGroup\relax
  \fi
}%
%    \end{macrocode}
%    \end{macro}
%    \begin{macro}{\PdfRender@PatchColor}
%    \begin{macrocode}
\def\PdfRender@PatchColor{%
  \PdfRender@PatchNormalColor
  \PdfRender@PatchColorSetGroup
}
%    \end{macrocode}
%    \end{macro}
%    \begin{macrocode}
\PdfRender@PatchColor
\ltx@IfUndefined{AtBeginDocument}{}{%
  \AtBeginDocument{\PdfRender@PatchColor}%
}
%    \end{macrocode}
%    \cs{AfterPackage} is provided by package \xpackage{scrlfile}.
%    \begin{macrocode}
\ltx@IfUndefined{AfterPackage}{%
}{%
  \AfterPackage{color}{\PdfRender@PatchColor}%
  \AfterPackage{xcolor}{\PdfRender@PatchColor}%
  \AfterPackage{etoolbox}{%
    \AfterEndPreamble{\PdfRender@PatchColor}%
  }%
}%
%    \end{macrocode}
%    \cs{AfterEndPreamble} is provided by package \xpackage{etoolbox}.
%    \begin{macrocode}
\ltx@IfUndefined{AfterEndPreamble}{%
}{%
  \AfterEndPreamble{\PdfRender@PatchColor}%
}%
%    \end{macrocode}
%
%    \begin{macrocode}
\PdfRender@RequirePackage{kvsetkeys}[2010/01/28]
%    \end{macrocode}
%    \begin{macro}{\PdfRender@texorpdfstring}
%    \begin{macrocode}
\def\PdfRender@texorpdfstring{%
  \ltx@IfUndefined{texorpdfstring}\ltx@firstoftwo\texorpdfstring
}
%    \end{macrocode}
%    \end{macro}
%    \begin{macro}{\pdfrender}
%    \begin{macrocode}
\ltx@IfUndefined{DeclareRobustCommand}%
\ltx@firstoftwo\ltx@secondoftwo
{%
  \def\pdfrender#1%
}{%
  \newcommand{\pdfrender}{}%
  \DeclareRobustCommand*{\pdfrender}[1]%
}%
{%
  \PdfRender@texorpdfstring{%
    \PdfRender@PatchNormalColor
    \global\let\PdfRender@FillColor\ltx@empty
    \global\let\PdfRender@StrokeColor\ltx@empty
    \kvsetkeys{PDFRENDER}{#1}%
    \PdfRender@SetColor
  }{}%
}
%    \end{macrocode}
%    \end{macro}
%    \begin{macro}{\textpdfrender}
%    \begin{macrocode}
\ltx@IfUndefined{DeclareRobustCommand}%
\ltx@firstoftwo\ltx@secondoftwo
{%
  \long\def\textpdfrender#1#2%
}{%
  \newcommand{\textpdfrender}{}%
  \DeclareRobustCommand{\textpdfrender}[2]%
}%
{%
  \PdfRender@texorpdfstring{%
    \begingroup
      \pdfrender{#1}%
      #2%
    \endgroup
  }{#2}%
}
%    \end{macrocode}
%    \end{macro}
%
%    \begin{macro}{\ifPdfRender@Values}
%    \begin{macrocode}
\PdfRender@newif{Values}
%    \end{macrocode}
%    \end{macro}
%    \begin{macro}{\PdfRender@NewClassValues}
%    \begin{macrocode}
\def\PdfRender@NewClassValues#1#2#3#4{%
  \PdfRender@Valuestrue
  \PdfRender@NewClass{#1}{#2}{#3}{#4}{}%
}
%    \end{macrocode}
%    \end{macro}
%    \begin{macro}{\PdfRender@NewClass}
%    \begin{macrocode}
\def\PdfRender@NewClass#1#2#3#4#5{%
  \PdfRender@newif{Active#1}%
  \expandafter\def\csname PdfRender@Default#1\endcsname{#2}%
  \expandafter\let\csname PdfRender@Current#1\expandafter\endcsname
      \csname PdfRender@Default#1\endcsname
  \ifPdfRender@Stack
    \expandafter\edef\csname PdfRender@Init#1\endcsname{%
      \global\chardef
      \expandafter\noexpand\csname PdfRender@Stack#1\endcsname=%
          \noexpand\pdfcolorstackinit page direct{%
            \noexpand#3%
            \expandafter\noexpand\csname PdfRender@Default#1\endcsname
          }\relax
      \noexpand\@PackageInfo{pdfrender}{%
        New color stack `#1' = \noexpand\number
        \expandafter\noexpand\csname PdfRender@Stack#1\endcsname
      }%
      \gdef\expandafter\noexpand\csname PdfRender@Init#1\endcsname{}%
    }%
    \expandafter\edef\csname PdfRender@Set#1\endcsname{%
      \expandafter\noexpand\csname PdfRender@Init#1\endcsname
      \noexpand\pdfcolorstack
      \expandafter\noexpand\csname PdfRender@Stack#1\endcsname
      push{%
        #3{\expandafter\noexpand\csname PdfRender@Current#1\endcsname}%
      }%
      \noexpand\aftergroup
      \expandafter\noexpand\csname PdfRender@Reset#1\endcsname
    }%
    \expandafter\edef\csname PdfRender@Reset#1\endcsname{%
      \expandafter\noexpand\csname PdfRender@Init#1\endcsname
      \noexpand\pdfcolorstack
      \expandafter\noexpand\csname PdfRender@Stack#1\endcsname
      pop\relax
    }%
  \else
    \expandafter\edef\csname PdfRender@Set#1\endcsname{%
      \noexpand\pdfliteral direct{%
        #3{\expandafter\noexpand\csname PdfRender@Current#1\endcsname}%
      }%
      \noexpand\aftergroup
      \expandafter\noexpand\csname PdfRender@Reset#1\endcsname
    }%
    \expandafter\edef\csname PdfRender@Reset#1\endcsname{%
      \noexpand\pdfliteral direct{%
        #3{\expandafter\noexpand\csname PdfRender@Current#1\endcsname}%
      }%
    }%
  \fi
  \expandafter\edef\csname PdfRender@Normal#1\endcsname{%
    \let
    \expandafter\noexpand\csname PdfRender@Current#1\endcsname
    \expandafter\noexpand\csname PdfRender@Default#1\endcsname
    \noexpand\PdfRender@Set{#1}%
  }%
  \expandafter\ltx@GlobalAppendToMacro\expandafter\PdfRender@NormalColorHook
  \expandafter{%
    \csname PdfRender@Normal#1\endcsname
  }%
  \ltx@GlobalAppendToMacro\PdfRender@ColorSetGroupHook{%
    \PdfRender@Set{#1}%
  }%
  \ifPdfRender@Values
    \kv@parse@normalized{#4}{%
      \expandafter\let\csname PdfRender@#1@\kv@key\endcsname\kv@key
      \ifx\kv@value\relax
      \else
        \expandafter\let\csname PdfRender@#1@\kv@value\endcsname\kv@key
      \fi
      \ltx@gobbletwo
    }%
    \PdfRender@define@key{PDFRENDER}{#1}{%
      \global\csname PdfRender@Active#1true\endcsname
      \def\PdfRender@Current{##1}%
      \PdfRender@SetValidateValues{#1}%
    }%
    \PdfRender@Valuesfalse
  \else
    \PdfRender@define@key{PDFRENDER}{#1}{%
      \global\csname PdfRender@Active#1true\endcsname
      \expandafter\def\csname PdfRender@Current#1\endcsname{##1}%
      \ltx@IfUndefined{PdfRender@PostProcess#1}{%
      }{%
        \csname PdfRender@PostProcess#1\endcsname
      }%
      \PdfRender@SetValidate{#1}{#4}{#5}%
    }%
  \fi
}%
%    \end{macrocode}
%    \end{macro}
%    \begin{macro}{\PdfRender@define@key}
%    \begin{macrocode}
\ltx@IfUndefined{define@key}{%
  \def\PdfRender@define@key#1#2{%
    \expandafter\def\csname KV@#1@#2\endcsname##1%
  }%
}{%
  \let\PdfRender@define@key\define@key
}
%    \end{macrocode}
%    \end{macro}
%    \begin{macro}{\PdfRender@Set}
%    \begin{macrocode}
\def\PdfRender@Set#1{%
  \csname ifPdfRender@Active#1\endcsname
    \csname PdfRender@Set#1\expandafter\endcsname
  \fi
}
%    \end{macrocode}
%    \end{macro}
%    \begin{macro}{\PdfRender@Reset}
%    \begin{macrocode}
\def\PdfRender@Reset#1{%
  \csname ifPdfRender@Active#1\endcsname
    \csname PdfRender@Reset#1\expandafter\endcsname
  \fi
}
%    \end{macrocode}
%    \end{macro}
%    \begin{macro}{\PdfRender@ErrorInvalidValue}
%    \begin{macrocode}
\def\PdfRender@ErrorInvalidValue#1{%
  \PackageError{pdfrender}{%
    Ignoring parameter setting for `#1'\MessageBreak
    because of invalid value %
    `\csname PdfRender@Current#1\endcsname'%
  }\@ehc
  \expandafter\let\csname PdfRender@Current#1\endcsname\ltx@empty
}%
%    \end{macrocode}
%    \end{macro}
%    \begin{macro}{\PdfRender@SetValidate}
%    \begin{macrocode}
\ifPdfRender@Match
  \def\PdfRender@SetValidate#1#2#3{%
    \ifnum\pdfmatch{^(#2)$}{\csname PdfRender@Current#1\endcsname}=1 %
      \csname PdfRender@Set#1\expandafter\endcsname
    \else
      \PdfRender@ErrorInvalidValue{#1}%
    \fi
  }%
\else
  \def\PdfRender@SetValidate#1#2#3{%
    \expandafter\let\expandafter\PdfRender@Current
    \csname PdfRender@Current#1\endcsname
    #3%
    \ifx\PdfRender@Current\@empty
      \PdfRender@ErrorInvalidValue{#1}%
    \else
      \csname PdfRender@Set#1\expandafter\endcsname
    \fi
  }%
\fi
%    \end{macrocode}
%    \end{macro}
%    \begin{macro}{\PdfRender@SetValidateValues}
%    \begin{macrocode}
\def\PdfRender@SetValidateValues#1{%
  \ltx@IfUndefined{PdfRender@#1@\PdfRender@Current}{%
    \expandafter\let\csname PdfRender@Current#1\endcsname
                    \PdfRender@Current
    \PdfRender@ErrorInvalidValue{#1}%
  }{%
    \expandafter\edef\csname PdfRender@Current#1\endcsname{%
      \csname PdfRender@#1@\PdfRender@Current\endcsname
    }%
    \csname PdfRender@Set#1\endcsname
  }%
}
%    \end{macrocode}
%    \end{macro}
%    \begin{macro}{\PdfRender@OpValue}
%    \begin{macrocode}
\def\PdfRender@OpValue#1#2{#2\ltx@space#1}%
%    \end{macrocode}
%    \end{macro}
%    \begin{macro}{\PdfRender@OpName}
%    \begin{macrocode}
\def\PdfRender@OpName#1#2{/#2\ltx@space#1}%
%    \end{macrocode}
%    \end{macro}
%
% \subsection{Declare and setup parameters}
%
%    \begin{macrocode}
\PdfRender@NewClassValues{TextRenderingMode}%
                      {0}%
                      {\PdfRender@OpValue{Tr}}{%
  0=Fill,%
  1=Stroke,%
  2=FillStroke,%
  3=Invisible,%
  4=FillClip,%
  5=StrokeClip,%
  6=FillStrokeClip,%
  7=Clip,%
}%
\PdfRender@NewClass{LineWidth}{1}{\PdfRender@OpValue{w}}{%
  [0-9]+\string\.?[0-9]*|\string\.[0-9]+%
}{}%
\ltx@IfUndefined{dimexpr}{%
  \def\PdfRender@dimexpr{}%
}{%
  \let\PdfRender@dimexpr\dimexpr
}
\def\PdfRender@PostProcessLineWidth{%
  \begingroup
  \afterassignment\PdfRender@@PostProcessLineWidth
  \dimen0=\PdfRender@dimexpr\PdfRender@CurrentLineWidth bp %
  \PdfRender@let\PdfRender@relax\PdfRender@relax
}
\let\PdfRender@let\let
\let\PdfRender@relax\relax
\def\PdfRender@@PostProcessLineWidth#1\PdfRender@let{%
  \ifx\\#1\\%
    \endgroup
  \else
    \dimen0=.996264\dimen0 % 72/72.27
    \edef\x{\endgroup
      \def\noexpand\PdfRender@CurrentLineWidth{%
        \strip@pt\dimen0%
      }%
    }%
    \expandafter\x
  \fi
}
\PdfRender@NewClassValues{LineCapStyle}{0}{\PdfRender@OpValue{J}}{%
  0=Butt,%
  1=Round,%
  2=ProjectingSquare,%
}%
\PdfRender@NewClassValues{LineJoinStyle}{0}{\PdfRender@OpValue{j}}{%
  0=Miter,%
  1=Round,%
  2=Bevel,%
}%
\PdfRender@NewClass{MiterLimit}{10}{\PdfRender@OpValue{M}}{%
  [0-9]*[1-9][0-9]*\string\.?[0-9]*|%
  [0-9]*\string\.?[0-9]*[1-9][0-9]*%
}{}%
\PdfRender@NewClass{Flatness}{0}{\PdfRender@OpValue{i}}{%
  100(\string\.0*)?|[0-9][0-9](\string\.[0-9]*)?|\string\.[0-9]+%
}{}%
\PdfRender@NewClass{LineDashPattern}{[]0}{\PdfRender@OpValue{d}}{%
  \string\[%
  ( ?([0-9]+\string\.?[0-9]*|\string\.[0-9]+) ?)*%
  \string\] ?%
  ([0-9]+\string\.?[0-9]*|\string\.[0-9]+)%
}{}%
\PdfRender@NewClassValues{RenderingIntent}%
                      {RelativeColorimetric}%
                      {\PdfRender@OpName{ri}}{%
  AbsoluteColorimetric,%
  RelativeColorimetric,%
  Saturation,%
  Perceptual,%
}%
%    \end{macrocode}
%
% \subsection{Fill and stroke color support}
%
%    \begin{macrocode}
\PdfRender@define@key{PDFRENDER}{FillColor}{%
  \begingroup
    \def\PdfRender@Color{#1}%
    \ifx\PdfRender@Color\ltx@empty
      \global\let\PdfRender@FillColor\ltx@empty
    \else
      \PdfRender@ColorAvailable{%
        \PdfRender@TestBox{%
          \expandafter\PdfRender@TryColor\PdfRender@Color\ltx@empty
          \PdfRender@GetFillColor
          \ifx\PdfRender@FillColor\ltx@empty
            \@PackageWarning{pdfrender}{%
              Cannot extract fill color\MessageBreak
              from value `#1'%
            }%
          \fi
        }%
      }%
    \fi
  \endgroup
}
\PdfRender@define@key{PDFRENDER}{StrokeColor}{%
  \begingroup
    \def\PdfRender@Color{#1}%
    \ifx\PdfRender@Color\ltx@empty
      \global\let\PdfRender@StrokeColor\ltx@empty
    \else
      \PdfRender@ColorAvailable{%
        \PdfRender@TestBox{%
          \expandafter\PdfRender@TryColor\PdfRender@Color\ltx@empty
          \PdfRender@GetStrokeColor
          \ifx\PdfRender@StrokeColor\ltx@empty
            \@PackageWarning{pdfrender}{%
              Cannot extract stroke color\MessageBreak
              from value `#1'%
            }%
          \fi
        }%
      }%
    \fi
  \endgroup
}
%    \end{macrocode}
%    \begin{macro}{\PdfRender@ColorAvailable}
%    \begin{macrocode}
\def\PdfRender@ColorAvailable{%
  \@ifundefined{set@color}{%
    \@PackageError{pdfrender}{%
      Ignoring color options, because neither\MessageBreak
      package `color' nor package `xcolor' is loaded%
    }\@ehc
    \global\let\PdfRender@ColorAvailable\ltx@gobble
  }{%
    \global\let\PdfRender@ColorAvailable\ltx@firstofone
  }%
  \PdfRender@ColorAvailable
}
%    \end{macrocode}
%    \end{macro}
%    \begin{macro}{\PdfRender@TryColor}
%    \begin{macrocode}
\def\PdfRender@TryColor{%
  \@ifnextchar[\color\PdfRender@@TryColor
}
%    \end{macrocode}
%    \end{macro}
%    \begin{macro}{\PdfRender@@TryColor}
%    \begin{macrocode}
\def\PdfRender@@TryColor#1\ltx@empty{%
  \expandafter\color\expandafter{\PdfRender@Color}%
}
%    \end{macrocode}
%    \end{macro}
%    \begin{macro}{\PdfRender@SetColor}
%    \begin{macrocode}
\def\PdfRender@SetColor{%
  \chardef\PdfRender@NeedsCurrentColor=0 %
  \ifx\PdfRender@FillColor\ltx@empty
    \ifx\PdfRender@StrokeColor\ltx@empty
    \else
      \edef\PdfRender@CurrentColor{%
        \noexpand\PdfRender@FillColor\ltx@space\PdfRender@StrokeColor
      }%
      \chardef\PdfRender@NeedsCurrentColor=1 %
    \fi
  \else
    \ifx\PdfRender@StrokeColor\ltx@empty
      \edef\PdfRender@CurrentColor{%
        \PdfRender@FillColor\ltx@space\noexpand\PdfRender@StrokeColor
      }%
      \chardef\PdfRender@NeedsCurrentColor=2 %
    \else
      \edef\current@color{%
        \PdfRender@FillColor\ltx@space\PdfRender@StrokeColor
      }%
      \set@color
    \fi
  \fi
  \ifnum\PdfRender@NeedsCurrentColor=1 %
    \PdfRender@GetFillColor
    \ifx\PdfRender@FillColor\ltx@empty
      \@PackageWarning{pdfrender}{%
        Cannot extract current fill color%
      }%
    \else
      \edef\current@color{\PdfRender@CurrentColor}%
      \set@color
    \fi
  \else
    \ifnum\PdfRender@NeedsCurrentColor=2 %
      \PdfRender@GetStrokeColor
      \ifx\PdfRender@StrokeColor\ltx@empty
        \@PackageWarning{pdfrender}{%
          Cannot extract current stroke color%
        }%
      \else
        \edef\current@color{\PdfRender@CurrentColor}%
        \set@color
      \fi
    \fi
  \fi
}
%    \end{macrocode}
%    \end{macro}
%    \begin{macro}{\PdfRender@PatternFillColor}
%    \begin{macrocode}
\edef\PdfRender@PatternFillColor{ % space
  (%
    [0-9\string\.]+ g|%
    [0-9\string\.]+ [0-9\string\.]+ [0-9\string\.]+ rg|%
    [0-9\string\.]+ [0-9\string\.]+ %
    [0-9\string\.]+ [0-9\string\.]+ k%
  ) % space
  (.*)$%
}
%    \end{macrocode}
%    \end{macro}
%    \begin{macro}{\PdfRender@PatternStrokeColor}
%    \begin{macrocode}
\edef\PdfRender@PatternStrokeColor{ % space
  (%
    [0-9\string\.]+ G|%
    [0-9\string\.]+ [0-9\string\.]+ [0-9\string\.]+ RG|%
    [0-9\string\.]+ [0-9\string\.]+ %
    [0-9\string\.]+ [0-9\string\.]+ K%
  ) % space
  (.*)$%
}
%    \end{macrocode}
%    \end{macro}
%    \begin{macro}{\PdfRender@MatchPattern}
%    \begin{macrocode}
\def\PdfRender@MatchPattern#1{%
  \ifnum\pdfmatch{\PdfRender@Pattern}{\PdfRender@String}=1 %
    \xdef#1{%
      \expandafter\strip@prefix\pdflastmatch 1%
    }%
    \edef\PdfRender@String{%
      \expandafter\strip@prefix\pdflastmatch 2%
    }%
    \ifx\PdfRender@String\ltx@empty
    \else
      \expandafter\expandafter\expandafter\PdfRender@MatchPattern
      \expandafter\expandafter\expandafter#1%
    \fi
  \fi
}
%    \end{macrocode}
%    \end{macro}
%    \begin{macro}{\PdfRender@GetFillColor}
%    \begin{macrocode}
\def\PdfRender@GetFillColor{%
  \global\let\PdfRender@FillColor\ltx@empty
  \begingroup
    \ifPdfRender@Match
      \let\PdfRender@Pattern\PdfRender@PatternFillColor
      \edef\PdfRender@String{\ltx@space\current@color\ltx@space}%
      \PdfRender@MatchPattern\PdfRender@FillColor
    \else
      \edef\current@color{\current@color\ltx@space}%
      \let\PdfRender@OP\relax
      \PdfRender@FindOp{g}0%
      \PdfRender@FindOp{G}1%
      \PdfRender@FindOp{rg}0%
      \PdfRender@FindOp{RG}1%
      \PdfRender@FindOp{k}0%
      \PdfRender@FindOp{K}1%
      \PdfRender@FilterOp 0\PdfRender@FillColor
    \fi
  \endgroup
}
%    \end{macrocode}
%    \end{macro}
%    \begin{macro}{\PdfRender@GetStrokeColor}
%    \begin{macrocode}
\def\PdfRender@GetStrokeColor{%
  \global\let\PdfRender@StrokeColor\ltx@empty
  \begingroup
    \ifPdfRender@Match
      \let\PdfRender@Pattern\PdfRender@PatternStrokeColor
      \edef\PdfRender@String{\ltx@space\current@color\ltx@space}%
      \PdfRender@MatchPattern\PdfRender@StrokeColor
    \else
      \edef\current@color{\current@color\ltx@space}%
      \let\PdfRender@OP\relax
      \PdfRender@FindOp{g}0%
      \PdfRender@FindOp{G}1%
      \PdfRender@FindOp{rg}0%
      \PdfRender@FindOp{RG}1%
      \PdfRender@FindOp{k}0%
      \PdfRender@FindOp{K}1%
      \PdfRender@FilterOp 1\PdfRender@StrokeColor
    \fi
  \endgroup
}
%    \end{macrocode}
%    \end{macro}
%    \begin{macrocode}
\ifPdfRender@Match
  \expandafter\PdfRender@AtEnd
\fi%
%    \end{macrocode}
%    \begin{macro}{\PdfRender@FindOp}
%    \begin{macrocode}
\def\PdfRender@FindOp#1#2{%
  \def\PdfRender@temp##1 #1 ##2\@nil{%
    ##1%
    \ifx\\##2\\%
      \expandafter\@gobble
    \else
      \PdfRender@OP{#1}#2%
      \expandafter\@firstofone
    \fi
    {%
      \PdfRender@temp##2\@nil
    }%
  }%
  \edef\current@color{%
    \@firstofone{\expandafter\PdfRender@temp\current@color} #1 \@nil
  }%
}
%    \end{macrocode}
%    \end{macro}
%    \begin{macro}{\PdfRender@FilterOp}
%    \begin{macrocode}
\def\PdfRender@FilterOp#1#2{%
  \expandafter\PdfRender@@FilterOp\expandafter#1\expandafter#2%
      \current@color\PdfRender@OP{}{}%
}
%    \end{macrocode}
%    \end{macro}
%    \begin{macro}{\PdfRender@@FilterOp}
%    \begin{macrocode}
\def\PdfRender@@FilterOp#1#2#3\PdfRender@OP#4#5{%
  \ifx\\#4#5\\%
  \else
    \ifnum#1=#5 %
      \xdef#2{#3 #4}%
    \fi
    \expandafter\PdfRender@@FilterOp\expandafter#1\expandafter#2%
  \fi
}
%    \end{macrocode}
%    \end{macro}
%
%    \begin{macrocode}
\PdfRender@AtEnd%
%</package>
%    \end{macrocode}
%% \section{Installation}
%
% \subsection{Download}
%
% \paragraph{Package.} This package is available on
% CTAN\footnote{\CTANpkg{pdfrender}}:
% \begin{description}
% \item[\CTAN{macros/latex/contrib/oberdiek/pdfrender.dtx}] The source file.
% \item[\CTAN{macros/latex/contrib/oberdiek/pdfrender.pdf}] Documentation.
% \end{description}
%
%
% \paragraph{Bundle.} All the packages of the bundle `oberdiek'
% are also available in a TDS compliant ZIP archive. There
% the packages are already unpacked and the documentation files
% are generated. The files and directories obey the TDS standard.
% \begin{description}
% \item[\CTANinstall{install/macros/latex/contrib/oberdiek.tds.zip}]
% \end{description}
% \emph{TDS} refers to the standard ``A Directory Structure
% for \TeX\ Files'' (\CTANpkg{tds}). Directories
% with \xfile{texmf} in their name are usually organized this way.
%
% \subsection{Bundle installation}
%
% \paragraph{Unpacking.} Unpack the \xfile{oberdiek.tds.zip} in the
% TDS tree (also known as \xfile{texmf} tree) of your choice.
% Example (linux):
% \begin{quote}
%   |unzip oberdiek.tds.zip -d ~/texmf|
% \end{quote}
%
% \subsection{Package installation}
%
% \paragraph{Unpacking.} The \xfile{.dtx} file is a self-extracting
% \docstrip\ archive. The files are extracted by running the
% \xfile{.dtx} through \plainTeX:
% \begin{quote}
%   \verb|tex pdfrender.dtx|
% \end{quote}
%
% \paragraph{TDS.} Now the different files must be moved into
% the different directories in your installation TDS tree
% (also known as \xfile{texmf} tree):
% \begin{quote}
% \def\t{^^A
% \begin{tabular}{@{}>{\ttfamily}l@{ $\rightarrow$ }>{\ttfamily}l@{}}
%   pdfrender.sty & tex/generic/oberdiek/pdfrender.sty\\
%   pdfrender.pdf & doc/latex/oberdiek/pdfrender.pdf\\
%   pdfrender.dtx & source/latex/oberdiek/pdfrender.dtx\\
% \end{tabular}^^A
% }^^A
% \sbox0{\t}^^A
% \ifdim\wd0>\linewidth
%   \begingroup
%     \advance\linewidth by\leftmargin
%     \advance\linewidth by\rightmargin
%   \edef\x{\endgroup
%     \def\noexpand\lw{\the\linewidth}^^A
%   }\x
%   \def\lwbox{^^A
%     \leavevmode
%     \hbox to \linewidth{^^A
%       \kern-\leftmargin\relax
%       \hss
%       \usebox0
%       \hss
%       \kern-\rightmargin\relax
%     }^^A
%   }^^A
%   \ifdim\wd0>\lw
%     \sbox0{\small\t}^^A
%     \ifdim\wd0>\linewidth
%       \ifdim\wd0>\lw
%         \sbox0{\footnotesize\t}^^A
%         \ifdim\wd0>\linewidth
%           \ifdim\wd0>\lw
%             \sbox0{\scriptsize\t}^^A
%             \ifdim\wd0>\linewidth
%               \ifdim\wd0>\lw
%                 \sbox0{\tiny\t}^^A
%                 \ifdim\wd0>\linewidth
%                   \lwbox
%                 \else
%                   \usebox0
%                 \fi
%               \else
%                 \lwbox
%               \fi
%             \else
%               \usebox0
%             \fi
%           \else
%             \lwbox
%           \fi
%         \else
%           \usebox0
%         \fi
%       \else
%         \lwbox
%       \fi
%     \else
%       \usebox0
%     \fi
%   \else
%     \lwbox
%   \fi
% \else
%   \usebox0
% \fi
% \end{quote}
% If you have a \xfile{docstrip.cfg} that configures and enables \docstrip's
% TDS installing feature, then some files can already be in the right
% place, see the documentation of \docstrip.
%
% \subsection{Refresh file name databases}
%
% If your \TeX~distribution
% (\TeX\,Live, \mikTeX, \dots) relies on file name databases, you must refresh
% these. For example, \TeX\,Live\ users run \verb|texhash| or
% \verb|mktexlsr|.
%
% \subsection{Some details for the interested}
%
% \paragraph{Unpacking with \LaTeX.}
% The \xfile{.dtx} chooses its action depending on the format:
% \begin{description}
% \item[\plainTeX:] Run \docstrip\ and extract the files.
% \item[\LaTeX:] Generate the documentation.
% \end{description}
% If you insist on using \LaTeX\ for \docstrip\ (really,
% \docstrip\ does not need \LaTeX), then inform the autodetect routine
% about your intention:
% \begin{quote}
%   \verb|latex \let\install=y\input{pdfrender.dtx}|
% \end{quote}
% Do not forget to quote the argument according to the demands
% of your shell.
%
% \paragraph{Generating the documentation.}
% You can use both the \xfile{.dtx} or the \xfile{.drv} to generate
% the documentation. The process can be configured by the
% configuration file \xfile{ltxdoc.cfg}. For instance, put this
% line into this file, if you want to have A4 as paper format:
% \begin{quote}
%   \verb|\PassOptionsToClass{a4paper}{article}|
% \end{quote}
% An example follows how to generate the
% documentation with pdf\LaTeX:
% \begin{quote}
%\begin{verbatim}
%pdflatex pdfrender.dtx
%makeindex -s gind.ist pdfrender.idx
%pdflatex pdfrender.dtx
%makeindex -s gind.ist pdfrender.idx
%pdflatex pdfrender.dtx
%\end{verbatim}
% \end{quote}
%
% \section{Acknowledgement}
%
% \begin{description}
% \item[Friedrich Vosberg] asked in the
%   newsgroup \textsf{de.comp.text.tex} for the font
%   outline feature \cite{news-Vosberg}.
% \item[Gaius Pupus] proposed the basic method
%   using \cs{pdfliteral} in this thread \cite{news-Pupus}.
% \item[Rolf Niepraschk] added color support \cite{news-Niepraschk}.
% \end{description}
%
% \begin{thebibliography}{9}
% \raggedright
%
% \bibitem{pdf-spec}
%   Adobe Systems Incorporated. \textit{PDF Reference --
%   Adobe Portable Document format -- Version 1.7}.
%   6th ed. 2006.
%   \textsc{URL:}\,\url{http://www.adobe.com/devnet/acrobat/pdfs/pdf_reference_1-7.pdf}.
%
% \bibitem{news-Vosberg}
%   Friedrich Vosberg, \textit{Text in Buchstabenumrissen},
%   de.comp.text.tex, 2010-01-22.
%   \textsc{URL:}\,\url{https://groups.google.com/group/de.comp.text.tex/msg/f442310ac8b2d506}.
%
% \bibitem{news-Pupus}
%   Gaius Pupus, \textit{Re: Text in Buchstabenumrissen},
%   de.comp.text.tex, 2010-01-23.
%   \textsc{URL:}\,\url{https://groups.google.com/group/de.comp.text.tex/msg/95d890d77ac47eb1}.
%
% \bibitem{news-Niepraschk}
%   Rolf Niepraschk, \textit{Re: Text in Buchstabenumrissen},
%   de.comp.text.tex, 2010-01-24.
%   \textsc{URL:}\,\url{https://groups.google.com/group/de.comp.text.tex/msg/4eb61a5879db54db}.
%
% \end{thebibliography}
%
% \begin{History}
%   \begin{Version}{2010/01/26 v1.0}
%   \item
%     The first version.
%   \end{Version}
%   \begin{Version}{2010/01/27 v1.1}
%   \item
%     Macros \cs{pdfrender} and \cs{textpdfrender} are made robust.
%   \item
%     Color extraction rewritten for the case that \cs{pdfmatch}
%     is not available. This fixes wrong color assigments in
%     case of nesting.
%   \item
%     Color extraction of case \cs{pdfmatch} is fixed
%     for the case that the color string contains several
%     fill or several stroke operations.
%   \end{Version}
%   \begin{Version}{2010/01/28 v1.2}
%   \item
%     Dependency from package \xpackage{color} is removed.
%   \item
%     Compatibility for \plainTeX\ and even ini\TeX\ added.
%   \end{Version}
%   \begin{Version}{2016/05/14 v1.3}
%   \item
%     Use package \xpackage{luatex85} for compatibility with new LuaTeX.
%   \end{Version}
%   \begin{Version}{2016/05/17 v1.4}
%   \item
%     Documentation updates.
%   \item adjust luatex85 reference so that it works in plain TeX.
%   \end{Version}
%   \begin{Version}{2018/11/01 v1.5}
%   \item
%     Remove luatex85 dependency
%   \end{Version}
%   \begin{Version}{2019/12/29 v1.6}
%   \item
%     \xpackage{iftex} package.
%   \end{Version}
% \end{History}
%
% \PrintIndex
%
% \Finale
\endinput
|
% \end{quote}
% Do not forget to quote the argument according to the demands
% of your shell.
%
% \paragraph{Generating the documentation.}
% You can use both the \xfile{.dtx} or the \xfile{.drv} to generate
% the documentation. The process can be configured by the
% configuration file \xfile{ltxdoc.cfg}. For instance, put this
% line into this file, if you want to have A4 as paper format:
% \begin{quote}
%   \verb|\PassOptionsToClass{a4paper}{article}|
% \end{quote}
% An example follows how to generate the
% documentation with pdf\LaTeX:
% \begin{quote}
%\begin{verbatim}
%pdflatex pdfrender.dtx
%makeindex -s gind.ist pdfrender.idx
%pdflatex pdfrender.dtx
%makeindex -s gind.ist pdfrender.idx
%pdflatex pdfrender.dtx
%\end{verbatim}
% \end{quote}
%
% \section{Acknowledgement}
%
% \begin{description}
% \item[Friedrich Vosberg] asked in the
%   newsgroup \textsf{de.comp.text.tex} for the font
%   outline feature \cite{news-Vosberg}.
% \item[Gaius Pupus] proposed the basic method
%   using \cs{pdfliteral} in this thread \cite{news-Pupus}.
% \item[Rolf Niepraschk] added color support \cite{news-Niepraschk}.
% \end{description}
%
% \begin{thebibliography}{9}
% \raggedright
%
% \bibitem{pdf-spec}
%   Adobe Systems Incorporated. \textit{PDF Reference --
%   Adobe Portable Document format -- Version 1.7}.
%   6th ed. 2006.
%   \textsc{URL:}\,\url{http://www.adobe.com/devnet/acrobat/pdfs/pdf_reference_1-7.pdf}.
%
% \bibitem{news-Vosberg}
%   Friedrich Vosberg, \textit{Text in Buchstabenumrissen},
%   de.comp.text.tex, 2010-01-22.
%   \textsc{URL:}\,\url{https://groups.google.com/group/de.comp.text.tex/msg/f442310ac8b2d506}.
%
% \bibitem{news-Pupus}
%   Gaius Pupus, \textit{Re: Text in Buchstabenumrissen},
%   de.comp.text.tex, 2010-01-23.
%   \textsc{URL:}\,\url{https://groups.google.com/group/de.comp.text.tex/msg/95d890d77ac47eb1}.
%
% \bibitem{news-Niepraschk}
%   Rolf Niepraschk, \textit{Re: Text in Buchstabenumrissen},
%   de.comp.text.tex, 2010-01-24.
%   \textsc{URL:}\,\url{https://groups.google.com/group/de.comp.text.tex/msg/4eb61a5879db54db}.
%
% \end{thebibliography}
%
% \begin{History}
%   \begin{Version}{2010/01/26 v1.0}
%   \item
%     The first version.
%   \end{Version}
%   \begin{Version}{2010/01/27 v1.1}
%   \item
%     Macros \cs{pdfrender} and \cs{textpdfrender} are made robust.
%   \item
%     Color extraction rewritten for the case that \cs{pdfmatch}
%     is not available. This fixes wrong color assigments in
%     case of nesting.
%   \item
%     Color extraction of case \cs{pdfmatch} is fixed
%     for the case that the color string contains several
%     fill or several stroke operations.
%   \end{Version}
%   \begin{Version}{2010/01/28 v1.2}
%   \item
%     Dependency from package \xpackage{color} is removed.
%   \item
%     Compatibility for \plainTeX\ and even ini\TeX\ added.
%   \end{Version}
%   \begin{Version}{2016/05/14 v1.3}
%   \item
%     Use package \xpackage{luatex85} for compatibility with new LuaTeX.
%   \end{Version}
%   \begin{Version}{2016/05/17 v1.4}
%   \item
%     Documentation updates.
%   \item adjust luatex85 reference so that it works in plain TeX.
%   \end{Version}
%   \begin{Version}{2018/11/01 v1.5}
%   \item
%     Remove luatex85 dependency
%   \end{Version}
%   \begin{Version}{2019/12/29 v1.6}
%   \item
%     \xpackage{iftex} package.
%   \end{Version}
% \end{History}
%
% \PrintIndex
%
% \Finale
\endinput
|
% \end{quote}
% Do not forget to quote the argument according to the demands
% of your shell.
%
% \paragraph{Generating the documentation.}
% You can use both the \xfile{.dtx} or the \xfile{.drv} to generate
% the documentation. The process can be configured by the
% configuration file \xfile{ltxdoc.cfg}. For instance, put this
% line into this file, if you want to have A4 as paper format:
% \begin{quote}
%   \verb|\PassOptionsToClass{a4paper}{article}|
% \end{quote}
% An example follows how to generate the
% documentation with pdf\LaTeX:
% \begin{quote}
%\begin{verbatim}
%pdflatex pdfrender.dtx
%makeindex -s gind.ist pdfrender.idx
%pdflatex pdfrender.dtx
%makeindex -s gind.ist pdfrender.idx
%pdflatex pdfrender.dtx
%\end{verbatim}
% \end{quote}
%
% \section{Acknowledgement}
%
% \begin{description}
% \item[Friedrich Vosberg] asked in the
%   newsgroup \textsf{de.comp.text.tex} for the font
%   outline feature \cite{news-Vosberg}.
% \item[Gaius Pupus] proposed the basic method
%   using \cs{pdfliteral} in this thread \cite{news-Pupus}.
% \item[Rolf Niepraschk] added color support \cite{news-Niepraschk}.
% \end{description}
%
% \begin{thebibliography}{9}
% \raggedright
%
% \bibitem{pdf-spec}
%   Adobe Systems Incorporated. \textit{PDF Reference --
%   Adobe Portable Document format -- Version 1.7}.
%   6th ed. 2006.
%   \textsc{URL:}\,\url{http://www.adobe.com/devnet/acrobat/pdfs/pdf_reference_1-7.pdf}.
%
% \bibitem{news-Vosberg}
%   Friedrich Vosberg, \textit{Text in Buchstabenumrissen},
%   de.comp.text.tex, 2010-01-22.
%   \textsc{URL:}\,\url{https://groups.google.com/group/de.comp.text.tex/msg/f442310ac8b2d506}.
%
% \bibitem{news-Pupus}
%   Gaius Pupus, \textit{Re: Text in Buchstabenumrissen},
%   de.comp.text.tex, 2010-01-23.
%   \textsc{URL:}\,\url{https://groups.google.com/group/de.comp.text.tex/msg/95d890d77ac47eb1}.
%
% \bibitem{news-Niepraschk}
%   Rolf Niepraschk, \textit{Re: Text in Buchstabenumrissen},
%   de.comp.text.tex, 2010-01-24.
%   \textsc{URL:}\,\url{https://groups.google.com/group/de.comp.text.tex/msg/4eb61a5879db54db}.
%
% \end{thebibliography}
%
% \begin{History}
%   \begin{Version}{2010/01/26 v1.0}
%   \item
%     The first version.
%   \end{Version}
%   \begin{Version}{2010/01/27 v1.1}
%   \item
%     Macros \cs{pdfrender} and \cs{textpdfrender} are made robust.
%   \item
%     Color extraction rewritten for the case that \cs{pdfmatch}
%     is not available. This fixes wrong color assigments in
%     case of nesting.
%   \item
%     Color extraction of case \cs{pdfmatch} is fixed
%     for the case that the color string contains several
%     fill or several stroke operations.
%   \end{Version}
%   \begin{Version}{2010/01/28 v1.2}
%   \item
%     Dependency from package \xpackage{color} is removed.
%   \item
%     Compatibility for \plainTeX\ and even ini\TeX\ added.
%   \end{Version}
%   \begin{Version}{2016/05/14 v1.3}
%   \item
%     Use package \xpackage{luatex85} for compatibility with new LuaTeX.
%   \end{Version}
%   \begin{Version}{2016/05/17 v1.4}
%   \item
%     Documentation updates.
%   \item adjust luatex85 reference so that it works in plain TeX.
%   \end{Version}
%   \begin{Version}{2018/11/01 v1.5}
%   \item
%     Remove luatex85 dependency
%   \end{Version}
%   \begin{Version}{2019/12/29 v1.6}
%   \item
%     \xpackage{iftex} package.
%   \end{Version}
% \end{History}
%
% \PrintIndex
%
% \Finale
\endinput
|
% \end{quote}
% Do not forget to quote the argument according to the demands
% of your shell.
%
% \paragraph{Generating the documentation.}
% You can use both the \xfile{.dtx} or the \xfile{.drv} to generate
% the documentation. The process can be configured by the
% configuration file \xfile{ltxdoc.cfg}. For instance, put this
% line into this file, if you want to have A4 as paper format:
% \begin{quote}
%   \verb|\PassOptionsToClass{a4paper}{article}|
% \end{quote}
% An example follows how to generate the
% documentation with pdf\LaTeX:
% \begin{quote}
%\begin{verbatim}
%pdflatex pdfrender.dtx
%makeindex -s gind.ist pdfrender.idx
%pdflatex pdfrender.dtx
%makeindex -s gind.ist pdfrender.idx
%pdflatex pdfrender.dtx
%\end{verbatim}
% \end{quote}
%
% \section{Acknowledgement}
%
% \begin{description}
% \item[Friedrich Vosberg] asked in the
%   newsgroup \textsf{de.comp.text.tex} for the font
%   outline feature \cite{news-Vosberg}.
% \item[Gaius Pupus] proposed the basic method
%   using \cs{pdfliteral} in this thread \cite{news-Pupus}.
% \item[Rolf Niepraschk] added color support \cite{news-Niepraschk}.
% \end{description}
%
% \begin{thebibliography}{9}
% \raggedright
%
% \bibitem{pdf-spec}
%   Adobe Systems Incorporated. \textit{PDF Reference --
%   Adobe Portable Document format -- Version 1.7}.
%   6th ed. 2006.
%   \textsc{URL:}\,\url{http://www.adobe.com/devnet/acrobat/pdfs/pdf_reference_1-7.pdf}.
%
% \bibitem{news-Vosberg}
%   Friedrich Vosberg, \textit{Text in Buchstabenumrissen},
%   de.comp.text.tex, 2010-01-22.
%   \textsc{URL:}\,\url{https://groups.google.com/group/de.comp.text.tex/msg/f442310ac8b2d506}.
%
% \bibitem{news-Pupus}
%   Gaius Pupus, \textit{Re: Text in Buchstabenumrissen},
%   de.comp.text.tex, 2010-01-23.
%   \textsc{URL:}\,\url{https://groups.google.com/group/de.comp.text.tex/msg/95d890d77ac47eb1}.
%
% \bibitem{news-Niepraschk}
%   Rolf Niepraschk, \textit{Re: Text in Buchstabenumrissen},
%   de.comp.text.tex, 2010-01-24.
%   \textsc{URL:}\,\url{https://groups.google.com/group/de.comp.text.tex/msg/4eb61a5879db54db}.
%
% \end{thebibliography}
%
% \begin{History}
%   \begin{Version}{2010/01/26 v1.0}
%   \item
%     The first version.
%   \end{Version}
%   \begin{Version}{2010/01/27 v1.1}
%   \item
%     Macros \cs{pdfrender} and \cs{textpdfrender} are made robust.
%   \item
%     Color extraction rewritten for the case that \cs{pdfmatch}
%     is not available. This fixes wrong color assigments in
%     case of nesting.
%   \item
%     Color extraction of case \cs{pdfmatch} is fixed
%     for the case that the color string contains several
%     fill or several stroke operations.
%   \end{Version}
%   \begin{Version}{2010/01/28 v1.2}
%   \item
%     Dependency from package \xpackage{color} is removed.
%   \item
%     Compatibility for \plainTeX\ and even ini\TeX\ added.
%   \end{Version}
%   \begin{Version}{2016/05/14 v1.3}
%   \item
%     Use package \xpackage{luatex85} for compatibility with new LuaTeX.
%   \end{Version}
%   \begin{Version}{2016/05/17 v1.4}
%   \item
%     Documentation updates.
%   \item adjust luatex85 reference so that it works in plain TeX.
%   \end{Version}
%   \begin{Version}{2018/11/01 v1.5}
%   \item
%     Remove luatex85 dependency
%   \end{Version}
%   \begin{Version}{2019/12/29 v1.6}
%   \item
%     \xpackage{iftex} package.
%   \end{Version}
% \end{History}
%
% \PrintIndex
%
% \Finale
\endinput
