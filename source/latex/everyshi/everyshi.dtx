% \iffalse meta-comment
%
%%%%%%%%%%%%%%%%%%%%%%%%%%%%%%%%%%%%%%%%%%%%%%%%%%%%%%%%%%%%%%%%%%%%%%%
%
% Copyright (C) Martin Schröder, 1994–2001
% 				Marei Peischl (peiTeX)  <marei@peitex.de>, 2020
%
% This work may be distributed and/or modified under the
% conditions of the LaTeX Project Public License, either version 1.3c
% of this license or (at your option) any later version.
% The latest version of this license is in
%    http://www.latex-project.org/lppl.txt
% and version 1.3c or later is part of all distributions of LaTeX
% version 2005/12/01 or later.
%
% This work has the LPPL maintenance status `maintained'.
%
% The Current Maintainer of this work is
%   Marei Peischl <marei@peitex.de>.
%
% This work consists of the files
% 	README.md
% 	everyshi.dtx
% 	everyshi.ins
% and the derived files
% 	everyshi.sty
% 	everyshi-2001-05-15.sty
% 	everyshi.pdf
%
%%%%%%%%%%%%%%%%%%%%%%%%%%%%%%%%%%%%%%%%%%%%%%%%%%%%%%%%%%%%%%%%%%%%%%%
%
% \fi
% \iffalse
%
%<package|fallback>\NeedsTeXFormat{LaTeX2e}
%<package>\ProvidesPackage{everyshi}
%<package>         [2020/11/18 v4.00 EveryShipout Package]
%
%<*driver>
\documentclass{ltxdoc}
\usepackage{url}
\usepackage{everyshi}
\GetFileInfo{everyshi.sty}
\setcounter{IndexColumns}{2}
\EnableCrossrefs
\CodelineIndex    % Index code by line number
%\OnlyDescription  % comment out for implementation details
\setcounter{IndexColumns}{2}
\setlength{\IndexMin}{30ex}
\setlength{\columnseprule}{.4pt}
\begin{document}
   \DocInput{everyshi.dtx}
\end{document}
%</driver>
% \fi
% \CheckSum{68}
%
%
%% \CharacterTable
%% {Upper-case    \A\B\C\D\E\F\G\H\I\J\K\L\M\N\O\P\Q\R\S\T\U\V\W\X\Y\Z
%%  Lower-case    \a\b\c\d\e\f\g\h\i\j\k\l\m\n\o\p\q\r\s\t\u\v\w\x\y\z
%%  Digits        \0\1\2\3\4\5\6\7\8\9
%%  Exclamation   \!     Double quote  \"     Hash (number) \#
%%  Dollar        \$     Percent       \%     Ampersand     \&
%%  Acute accent  \'     Left paren    \(     Right paren   \)
%%  Asterisk      \*     Plus          \+     Comma         \,
%%  Minus         \-     Point         \.     Solidus       \/
%%  Colon         \:     Semicolon     \;     Less than     \<
%%  Equals        \=     Greater than  \>     Question mark \?
%%  Commercial at \@     Left bracket  \[     Backslash     \\
%%  Right bracket \]     Circumflex    \^     Underscore    \_
%%  Grave accent  \`     Left brace    \{     Vertical bar  \|
%%  Right brace   \}     Tilde         \~}
%%
%  \newcommand*{\package}[1]{\textnormal{\sffamily#1}}
%  \changes{v1.00}{1994/12/04}{New}
%  \changes{v1.02}{1994/12/07}{Name changed from \textsf{atshipou} to 
%                                \textsf{everyshi}}
%  \changes{v1.03}{1994/12/09}{Documentation improved}
%  \changes{v2.00}{1995/01/25}{Redesign}
%  \changes{v2.02}{1998/08/09}{Documentation improved}
%  \changes{v2.03}{1999/06/08}{Moved to LPPL}
%  \changes{v3.00}{2001/05/15}{\cs{AtNextShipout} added}
%  \changes{v4.00}{2020/11/18}{Mark \package{everyshi} as obsolete and use the kernel methods}
%
% ^^A -----------------------------
%
%  \changes{v2.01}{1997-03-09}{Fixed use of \cs{newline} in title.}
%  \title{\unskip
%           The obsolete \package{everyshi} package^^A
%           \thanks{^^A
%              The version umber of this file is \fileversion,
%              last revised \filedate.\protect\newline
%              The name \textsf{everyshi} is a tribute to the $8+3$
%              file-naming convention of certain ``operating
%              systems''; strictly speaking it should be 
%              \textsf{everyshipout}.}^^A
%        }
%  \author{Martin Schröder\thanks{maintained by Marei Peischl}}
%  \date{\filedate}
%  \maketitle
%
% ^^A -----------------------------
%  \renewcommand*{\abstractname}{Why you should no longer use this package:}
%  \begin{abstract}
%     This packages provides hooks into \cs{sshipout}
%     called \cs{EveryShipout} and \cs{AtNextShipout} analogous to 
%     \cs{AtBeginDocument}.
%
%	  With the introduction of the \LaTeX{} hook management this package became obsolete in 2020.
%	  Information on their usage can be found in the the corresponding documentation for
%     \package{lthooks}\cite{lthooks-doc} and \package{ltshipout}\cite{ltshipout-doc}.
%     We only provide this package to allow backwards compatibility.
%     For current versions of \LaTeX{} it's only mapping the hooks to the original \package{everyshi} macros.
%	  In case you use an older \LaTeX{} format, \package{everyshi} will automatically fallback to its' old implementation by loading \package{everyshi-2001-05-15}.
%  \end{abstract}
%
%  \pagestyle{headings}
%
% ^^A -----------------------------
%
%  \tableofcontents
%
% ^^A -----------------------------
%
%  \section{Introduction}
%
% \changes{v4.00}{2020/11/18}{Information on new kernel methods}
%  This package provides the hooks \cs{EveryShipout} and 
%  \cs{AtNextShipout} whose arguments are executed after the output 
%  routine has constructed \cs{box255}, and before \cs{shipout} is 
%  called.
%  \changes{v4.00}{2020/11/18}{Remove example applications.}
% ^^A -----------------------------
%
%  \section{Usage}
%
%  \cs{EveryShipout}\marg{code} declares
%  \mbox{$\langle$\emph{code}$\rangle$} that is saved internally
%  and executed before each \cs{shipout}.
%
%  \emph{Warning:} The \mbox{$\langle$\emph{code}$\rangle$} is saved 
%  globally; there is currently no way to remove it.
%
%  \DescribeMacro{\AtNextShipout}
%  \cs{AtNextShipout}\marg{code} declares
%  \mbox{$\langle$\emph{code}$\rangle$} that is saved internally
%  and executed just before \emph{only the next} \cs{shipout}.
%
%  The \meta{code} is executed after \cs{box255} has been constructed
%  by the output routine and can change \cs{box255}.
%  \cs{shipout} is called \emph{after} \meta{code}.
%
%  Repeated use of the commands is permitted: the code in the
%  argument is stored (and executed) in the order of their
%  declarations.
%
%  The argument of \cs{AtNextShipout} is executed \emph{after}
%  the argument of \cs{EveryShipout}.
%
% ^^A -----------------------------
%
%  \section{Options}
%
%  The package has no options.
%
% ^^A -----------------------------
%
%  \section{Required packages}
%
%  The package does not require any further packages.
%
% ^^A -----------------------------
%
%  \begin{thebibliography}{1}
%	\bibitem{lthooks-doc}
%        Frank Mittelbach.
%        \newblock The \package{ltshipout} package.
%        \newblock \url{http://mirrors.ctan.org/macros/latex/base/lthooks-doc.pdf}
%	\bibitem{ltshipout-doc}
%        Frank Mittelbach.
%        \newblock The \package{ltshipout} package.
%        \newblock \url{http://mirrors.ctan.org/macros/latex/base/ltshipout-doc.pdf}
%  \end{thebibliography}
%
%  \StopEventually{^^A
% ^^A -----------------------------
%
%  \section{Acknowledgements}
%  ^^A
%  Thanks to Ulrike Fischer for the advice during preparation of Version 4.00.
%  Version 2.0 of \package{everyshi} borrows heavily from
%  \textsf{quire.tex} of the \textsf{Midnight Macros} by Marcel R.~van 
%  der Goot (\texttt{marcel@cs.caltech.edu}).
%  The pointer to \textsf{quire} was provided by Lothar Meyer-Lerbs
%  (\texttt{\TeX{}Satz@zfn.uni-bremen.de}).
%  As usual Rebecca Stiels improved the quality of this documentation.
%
%
% \PrintChanges
%  }
% ^^A -----------------------------
%
%  \section{The implementation}
%  \changes{v4.00}{2020/11/18}{Emulate everyshi if the kernel is new enough.}
%  \raggedright
%    \begin{macrocode}
%<*package>
\providecommand\IfFormatAtLeastTF{\@ifl@t@r\fmtversion}
\IfFormatAtLeastTF{2020/10/01}{}{%%
%% This is file `everyshi-2001-05-15.sty',
%% generated with the docstrip utility.
%%
%% The original source files were:
%%
%% everyshi.dtx  (with options: `fallback,origpackage')
%% Copyright (C) 1994–2001 Martin Schröder, 2020 Marei Peischl (peiTeX)  <marei@peitex.de>
%% 
%% This work may be distributed and/or modified under the
%% conditions of the LaTeX Project Public License, either version 1.3c
%% of this license or (at your option) any later version.
%% The latest version of this license is in
%% http://www.latex-project.org/lppl.txt
%% and version 1.3c or later is part of all distributions of LaTeX
%% version 2005/12/01 or later.
%% 
%% This work has the LPPL maintenance status `maintained'.
%% 
%% The Current Maintainer of this work is
%% Marei Peischl <marei@peitex.de>.
%% 
%% This work consists of the files everyshi.dtx and everyshi.ins
%% and the derived files everyshi.sty and everyshi-2001-05-15.sty.
%% 
%%%%%%%%%%%%%%%%%%%%%%%%%%%%%%%%%%%%%%%%%%%%%%%%%%%%%%%%%%%%%%%%%%%%%%%
%%%%%%%%%%%%%%%%%%%%%%%%%%%%%%%%%%%%%%%%%%%%%%%%%%%%%%%%%%%%%%%%%%%%%%%
\NeedsTeXFormat{LaTeX2e}
%% \CharacterTable
%% {Upper-case    \A\B\C\D\E\F\G\H\I\J\K\L\M\N\O\P\Q\R\S\T\U\V\W\X\Y\Z
%%  Lower-case    \a\b\c\d\e\f\g\h\i\j\k\l\m\n\o\p\q\r\s\t\u\v\w\x\y\z
%%  Digits        \0\1\2\3\4\5\6\7\8\9
%%  Exclamation   \!     Double quote  \"     Hash (number) \#
%%  Dollar        \$     Percent       \%     Ampersand     \&
%%  Acute accent  \'     Left paren    \(     Right paren   \)
%%  Asterisk      \*     Plus          \+     Comma         \,
%%  Minus         \-     Point         \.     Solidus       \/
%%  Colon         \:     Semicolon     \;     Less than     \<
%%  Equals        \=     Greater than  \>     Question mark \?
%%  Commercial at \@     Left bracket  \[     Backslash     \\
%%  Right bracket \]     Circumflex    \^     Underscore    \_
%%  Grave accent  \`     Left brace    \{     Vertical bar  \|
%%  Right brace   \}     Tilde         \~}
%%
\ProvidesPackage{everyshi-2001-05-15}
       [2020/11/18 v4.00 EveryShipout Package (fallback mechanism)]
\newcommand{\@EveryShipout@Hook}{}
\newcommand{\@EveryShipout@AtNextHook}{}
\newcommand*{\EveryShipout}[1]
   {\g@addto@macro\@EveryShipout@Hook{#1}}
\newcommand*{\AtNextShipout}[1]
   {\g@addto@macro\@EveryShipout@AtNextHook{#1}}
\newcommand{\@EveryShipout@Shipout}{%
   \afterassignment\@EveryShipout@Test
   \global\setbox\@cclv= %
   }
\newcommand{\@EveryShipout@Test}{%
   \ifvoid\@cclv\relax
      \aftergroup\@EveryShipout@Output
   \else
      \@EveryShipout@Output
   \fi%
   }
\newcommand{\@EveryShipout@Output}{%
   \@EveryShipout@Hook%
   \@EveryShipout@AtNextHook%
      \gdef\@EveryShipout@AtNextHook{}%
   \@EveryShipout@Org@Shipout\box\@cclv%
   }
\newcommand{\@EveryShipout@Org@Shipout}{}
\newcommand*{\@EveryShipout@Init}{%
   \message{ABD: EveryShipout initializing macros}%
   \let\@EveryShipout@Org@Shipout\shipout
   \let\shipout\@EveryShipout@Shipout
   }
\AtBeginDocument{\@EveryShipout@Init}
\endinput
%%
%% End of file `everyshi-2001-05-15.sty'.
}
\IfFormatAtLeastTF{2020/10/01}{}{\endinput}
\protected \def \EveryShipout  {\AddToHook{shipout/before}}
\protected \def \AtNextShipout {\AddToHookNext{shipout/before}}
%</package>
%    \end{macrocode}
%  \subsection{The original implementation by Martin Schröder}
%  To provide compatibility for older \LaTeX{} formats we wrap the
%  original implementation of \package{everyshi} version 3.00 into
%  the fallback package \package{everyshi-2001-05-15}.
%    \begin{macrocode}
%<*fallback>
\ProvidesPackage{everyshi-2001-05-15}
       [2020/11/18 v4.00 EveryShipout Package (fallback mechanism)]
%    \end{macrocode}
%
%  \subsection{Allocations}
%
%  First we allocate the hooks
%  \begin{macro}{\@EveryShipout@Hook}
%  \changes{v2.00}{1995/01/25}{Name changed from \cs{@shipouthook} to
%                                \cs{@EveryShipout@Hook}}
%  \changes{v3.00}{2001/05/15}{\cs{newcommand}}
%  The code to be executed before \cs{shipout} is stored in
%  \cs{@EveryShipout@Hook}.
%    \begin{macrocode}
\newcommand{\@EveryShipout@Hook}{}
%    \end{macrocode}
%  \end{macro}
%
%  \begin{macro}{\@EveryShipout@AtNextHook}
%  \changes{v3.00}{2001/05/15}{new}
%  The code to be executed just before the normal \cs{shipout}
%  and \cs{@EveryShipout@EveryHook}.
%    \begin{macrocode}
\newcommand{\@EveryShipout@AtNextHook}{}
%    \end{macrocode}
%  \end{macro}
%
%
% ^^A -----------------------------
%
%  \subsection{The user-visible commands}
%  ^^A
%  \begin{macro}{\EveryShipout}
%  \begin{macro}{\AtNextShipout}
%  \changes{v1.01}{1994/12/06}{\cs{newcommand} instead of \cs{def}}
%  \changes{v1.02}{1994/12/07}{Name changed from \cs{AtShipout} to 
%                                \cs{EveryShipOut}}
%  \changes{v3.00}{2001/05/15}{\cs{AtNextShipout} added}
%  The commands are modeled after \cs{AtBeginDocument}.
%    \begin{macrocode}
\newcommand*{\EveryShipout}[1]
   {\g@addto@macro\@EveryShipout@Hook{#1}}
\newcommand*{\AtNextShipout}[1]
   {\g@addto@macro\@EveryShipout@AtNextHook{#1}}
%    \end{macrocode}
%  \end{macro}
%  \end{macro}
%
%
% ^^A -----------------------------
%
%  \subsection{Inserting the hooks}
%  ^^A
%  We want to redefine \cs{shipout} so that first \cs{box255} is
%  constructed and after that we can do something and at last shipout
%  the (possible modified) \cs{box255}.
%  Alas, this does not work in the usual way, since \cs{shipout} is
%  a \TeX{} primitive whose argument is a \meta{box}.
%  This means that simply redefining \cs{shipout} via \cs{newcommand[1]}
%  is impossible since \meta{box} can be something like \cs{box255} or
%  something like \cs{vbox\{\ldots\}}.
%  In the first case \texttt{\#1} would be \meta{\cs{box}} (without 
%  \meta{255}); in the second case it would be \meta{\cs{vbox}} (without
%  \meta{\{\ldots\}}).
%
%  The solution we use here is borrowed from \textsf{quire.tex} by
%  Marcel R.~van der Goot.
%  It is based upon \cs{afterassignment} and \cs{aftergroup}.
%
%  \begin{macro}{\@EveryShipout@Shipout}
%  \changes{v2.00}{1995/01/25}{new}
%  \cs{@EveryShipout@Shipout} is our replacement for \cs{shipout}.
%    \begin{macrocode}
\newcommand{\@EveryShipout@Shipout}{%
   \afterassignment\@EveryShipout@Test
   \global\setbox\@cclv= %
   }
%    \end{macrocode}
%  \cs{box255} is set to whatever comes after \cs{shipout}; but after
%  that assignment \cs{@EveryShipout@Test} is called.
%  \end{macro}
%
%  \begin{macro}{\@EveryShipout@Test}
%  \changes{v2.00}{1995/01/25}{new}
%  \cs{@EveryShipout@Test} determines if \cs{shipout} is called with
%  an argument like \cs{box255} or something like \cs{vbox\{\ldots\}}.
%  In the later case we delay the call of \cs{@EveryShipout@Output} 
%  (where the original \cs{shipout} is called) via \cs{aftergroup}.
%    \begin{macrocode}
\newcommand{\@EveryShipout@Test}{%
   \ifvoid\@cclv\relax
      \aftergroup\@EveryShipout@Output
   \else
      \@EveryShipout@Output
   \fi%
   }
%    \end{macrocode}
%  \end{macro}
%
%  \begin{macro}{\@EveryShipout@Output}
%  \changes{v2.00}{1995/01/25}{new}
%  \changes{v3.00}{2001/05/15}{added call to \cs{@EveryShipout@AtNextHook}}
%  \cs{@EveryShipout@Output} does the actual work.
%  First the \meta{code} accumulated via \cs{EveryShipout} and 
%  \cs{AtNextShipout} is called and then the original \cs{shipout} stored 
%  in \cs{@EveryShipout@Org@Shipout} is called to finally ship out 
%  \cs{box255}.
%    \begin{macrocode}
\newcommand{\@EveryShipout@Output}{%
   \@EveryShipout@Hook%
   \@EveryShipout@AtNextHook%
%    \end{macrocode}
%  We have to reset \cs{@EveryShipout@AtNextHook} after each use.
%    \begin{macrocode}
      \gdef\@EveryShipout@AtNextHook{}%
   \@EveryShipout@Org@Shipout\box\@cclv%
   }
%    \end{macrocode}
%  \end{macro}
%
%  \begin{macro}{\@EveryShipout@Org@Shipout}
%  \changes{v2.00}{1995/01/25}{new}
%  The original \cs{shipout} is stored in \cs{@EveryShipout@Org@Shipout}
%  by \cs{@EveryShipout@Init}.
%  Here we allocate it.
%    \begin{macrocode}
\newcommand{\@EveryShipout@Org@Shipout}{}
%    \end{macrocode}
%  \end{macro}
%
%  \begin{macro}{\@EveryShipout@Init}
%  \changes{v2.00}{1995/01/25}{new}
%  \changes{v3.00}{2001/05/15}{output message}
%  \cs{@EveryShipout@Init} stores the original \cs{shipout} in 
%  \cs{@EveryShipout@Org@Shipout} and sets \cs{shipout} to 
%  \cs{@EveryShipout@Shipout}.
%  This is done at \cs{begin\{document\}} via \cs{AtBeginDocument}.
%    \begin{macrocode}
\newcommand*{\@EveryShipout@Init}{%
   \message{ABD: EveryShipout initializing macros}%
   \let\@EveryShipout@Org@Shipout\shipout
   \let\shipout\@EveryShipout@Shipout
   }
\AtBeginDocument{\@EveryShipout@Init}
%    \end{macrocode}
%  \end{macro}
%
%    \begin{macrocode}
%</fallback>
%    \end{macrocode}
%
%
% ^^A -----------------------------
%
%  \Finale
%
