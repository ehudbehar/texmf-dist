% \iffalse meta-comment
%
% Copyright (C) 2008-2010 Jobst Hoffmann, FH Aachen, Campus J\"ulich <j.hoffmann_(at)_fh-aachen.de>
% -----------------------------------------------------------------------------------------
%
% This file may be distributed and/or modified under the
% conditions of the LaTeX Project Public License, either version 1.2
% of this license or (at your option) any later version.
% The latest version of this license is in:
%
%    http://www.latex-project.org/lppl.txt
%
% and version 1.2 or later is part of all distributions of LaTeX
% version 1999/12/01 or later.
%
% \fi
%
%
% \changes{v67}{10/06/29}{removed some needless blanks}
% \changes{v67}{10/06/29}{email adresses are obfuscated now}
% \changes{v58}{10/05/18}{moved the docstrip guards into the corresponding macrocode environments}
% \changes{v56}{10/03/09}{some small layout changes}
% \changes{v37}{09/08/27}{changed the copyright message again}
% \changes{v11}{09/04/15}{changed the Copyright message}
% \changes{v16}{09/08/18}{replaced macroname rcs by svn}
% \iffalse
%<extension>\NeedsTeXFormat{LaTeX2e}[1995/12/01]
%<extension>{%
%<extension>  \def\@svn@ $#1Date: #2-#3-#4 #5$$#6Revision: #7$ {%
%<extension>    \ProvidesPackage{listings-ext}[#2/#3/#4 v#7 an extension for the listings package (JHf)]}
%<extension>  \@svn@ $Date: 2010-06-29 18:38:12 +0200 (Di, 29 Jun 2010) $$Revision: 67 $ %
%<extension>}
% \fi
% \CheckSum{363}
% \changes{v41}{09/08/27}{removed typos again}
% \def\docdate {10/06/29} ^^A not style'able!!
%
% \changes{v11}{09/04/15}{turned german documentation into an english one}
%\iffalse
%
% Package `listings-ext' for use with LaTeX v2e
% Copyright (C) 2008-2010 Jobst Hoffmann <j.hoffmann (at) fh-aachen.de>, all rights reserved.
% -----------------------------------------------------------------
%
% This file may be distributed and/or modified under the
% conditions of the LaTeX Project Public License, either version 1.2
% of this license or (at your option) any later version.
% The latest version of this license is in:
%
%    http://www.latex-project.org/lppl.txt
%
% and version 1.2 or later is part of all distributions of LaTeX
% version 1999/12/01 or later.
%
% Address your error messages to:
%                                    Jobst Hoffmann
%                                    Fachhochschule Aachen, Campus J"ulich
%                                    Ginsterweg 1
%                                    52428 J"ulich
%                                    Bundesrepublik Deutschland
%                            Email:  <j.hoffmann_(at)_fh-aachen.de>
%
%\fi
%
% \def\listingsextSty{{\sf listings-ext.sty}}
% \def\listingsext{{\sf listings-ext}}
%
% \changes{v33}{09/08/23}{updated `DoNotIndex list}
% \DoNotIndex{\ ,\!,\C,\[,\\,\],\^,\`,\{,\},\~,\<,\=, \", \|}
% \DoNotIndex{\@ifundefined,\@ne, \@spaces, \author}
% \DoNotIndex{\catcode, \copyright, \def, \docdate, \documentclass}
% \DoNotIndex{\else,\endinput,\expandafter,\fi,\filedate,\fileversion}
% \DoNotIndex{\gdef,\global,\ifnum,\ifx,\let,\long}
% \DoNotIndex{\newcount,\newdimen,\newif,\next,\space,\string}
% \DoNotIndex{\the,\xdef,\z@}
% \DoNotIndex{\@@par, \@empty, \@hangfrom, \@reffalse, \@reftrue,}
% \DoNotIndex{\advance, \Alph, \alph, \arabic, \arraystretch, \baselineskip}
% \DoNotIndex{\begin}
% \DoNotIndex{\bgroup, \bigskip, \box, \bullet, \cdot, \centering}
% \DoNotIndex{\columnwidth, \csname, \day, \divide, \dotfill, \dp}
% \DoNotIndex{\edef, \egroup, \empty, \end, \endcsname, \font, \footins, \foo}
% \DoNotIndex{\frenchspacing, \hbadness, \hbox, \hfil, \hfill, \hline}
% \DoNotIndex{\hrule, \hsize, \hskip, \hspace, \hss, \ht, \ifcase}
% \DoNotIndex{\ifdim, \ifodd}
% \DoNotIndex{\ifvmode, \ignorespaces, \input, \interlinepenalty, \item}
% \DoNotIndex{\itemindent, \kern, \L, \large, \leaders, \list, \listparindent}
% \DoNotIndex{\magstep}
% \DoNotIndex{\magstephalf, \makelabel, \MakeShortVerb, \marginpar}
% \DoNotIndex{\mark, \mbox, \month, \multicolumn, \newpage, \newcommand}
% \DoNotIndex{\nobreak, \noexpand, \noindent}
% \DoNotIndex{\normalsize, \null}
% \DoNotIndex{\number, \onecolumn, \or, \overfullrule}
% \DoNotIndex{\pagebreak, \pagestyle, \parbox, \penalty, \phantom, \protect}
% \DoNotIndex{\quad}
% \DoNotIndex{\raggedbottom, \raggedleft}
% \DoNotIndex{\relax, \renewcommand, \reversemarginpar}
% \DoNotIndex{\rightmargin, \rule, \setbox, \setcounter}
% \DoNotIndex{\setlength, \settowidth, \shortstack, \strut, \svtnsfb, \switch}
% \DoNotIndex{\thepage, \thispagestyle, \tiny, \today, \triangleright, \typeout}
% \DoNotIndex{\underbar, \underline, \unskip, \uppercase, \usepackage}
% \DoNotIndex{\vadjust, \vbadness, \vbox, \verb, \vfil, \vfill, \vrule, \vskip}
% \DoNotIndex{\vspace, \vtop, \wd, \year}
% \DoNotIndex{\bf, \tt, \par, \smallskip, \stepcounter}
%
% ^^A some extra entries to make wrong entries in the index invisible
% \DoNotIndex{\-, \n, \t, \tbe, \tcase, \tdebug, \tgenerate, \tIf, \tif, \tinto}
% \DoNotIndex{\tof, \tprocessed, \tshow, \twrite}
%
% ^^A \setlength{\IndexMin}{0.3\textheight}
%
% \changes{v56}{10/03/09}{corrected typo at `Fname}
% \changes{v50}{10/02/15}{changed some documentation commands to starred version}
% \changes{v37}{09/08/27}{reworked the documentation}
% \changes{v35}{09/08/27}{some textual changes}%
% \changes{v34}{09/08/24}{reworked the documentation}%
% \changes{v34}{09/08/24}{added commands `Fname and `ext to unify the documentation}%
% \changes{v34}{09/08/24}{changed the title again}%
% \changes{v33}{09/08/23}{changed title}%
% \changes{v23}{09/08/20}{enhanced the documentation}%
% \changes{v23}{09/08/20}{rearranged the sections about Makefile, tests,
%   AUCTeX}%
% \changes{v16}{09/08/18}{added macros to make the documentation
%   independent of non standard packages}%
% \changes{v11}{09/04/15}{corrected some text}%
% ^^A now some macros to simplify the writing of the documentation
% \newcommand*\Lopt[1]{\textsf{#1}}%
% \let\Lpack\Lopt%
% \let\ext\Lopt%
% \newcommand*{\Fname}[1]{\texttt{#1}}%
% \newcommand*{\XEmacs}{\textsf{(X)Emacs}}%
% \newcommand*{\SOURCEFRAME}{%
%         \psline(0,0)(0,7)
%         \psline(1.5,0)(1.5,7)
%         \rput[lc](0.75,0){\ldots}
%         \rput[lc](0.75,7){\ldots}
% }
% \newcommand*{\SOURCEA}[1]{%
%   \psline*[linecolor=LemonChiffon](0,0)(1.5,0)(1.5,2)(0,2)%
%   \if#1f%
%   \psline(0,0)(1.5,0)(1.5,2.0)(0,2.0)(0,0)%
%   \else\if#1g
%   \psline[linewidth=1mm, linecolor=Olive](1.5,2.05)(0,2.05)%
%   \psline[linewidth=1mm, linecolor=DarkOliveGreen](0,-0.05)(1.5,-0.05)%
% \fi\fi%
% }
% \newcommand*{\SOURCEB}[1]{%
%   \psline*[linecolor=LightGreen](0,0)(1.5,0)(1.5,0.5)(0,0.5)%
%   \if#1f%
%   \psline(0,0)(1.5,0)(1.5,0.5)(0,0.5)(0,0)%
%   \else\if#1g
%   \psline[linewidth=1mm, linecolor=Olive](1.5,0.55)(0,0.55)%
%   \psline[linewidth=1mm, linecolor=DarkOliveGreen](0,-0.05)(1.5,-0.05)%
% \fi\fi%
% }
% \newcommand*{\SOURCEC}[1]{%
%   \psline*[linecolor=LightCyan](0,0)(1.5,0)(1.5,1.25)(0,1.25)%
%   \if#1f%
%   \psline(0,0)(1.5,0)(1.5,1.25)(0,1.25)(0,0)%
%   \else\if#1g
%   \psline[linewidth=1mm, linecolor=Olive](0,-0.05)(1.5,-0.05)%
%   \psline[linewidth=1mm, linecolor=DarkOliveGreen](1.5,1.30)(0,1.30)%
% \fi\fi%
% }
% ^^A to avoid underfull messages while formatting two/three columns
% \hbadness=10000 \vbadness=10000%
% \setcounter{secnumdepth}{2}%
% \setcounter{tocdepth}{2}%
%
% \title{\listingsext\ --- A collection of \LaTeX{} commands and some
% helper files to support the automatic integration of parts of source
% files into a documentation\thanks{%
% This file has version number \fileversion, last revised on \filedate{},
% documentation dated \docdate.}%
% }%
%   \author{Jobst Hoffmann\\
%   Fachhochschule Aachen, Campus J\"ulich\\
%   Ginsterweg 1\\
%   52428 J\"ulich\\
%   Bundesrepublik Deutschland\\
%   email: j.hoffmann\_(at)\_fh-aachen.de}%
%   \date{printed on \today}
%
% ^^A\markboth{\LaTeX\ Stil-Option listings-ext, version \fileversion\ of \filedate}
% ^^A         {\LaTeX\ Stil-Option listings-ext, version \fileversion\ of \filedate}
%
% \maketitle
%
% \begin{abstract}
% This article describes the use and the implementation of the \LaTeX{}
% package \listingsextSty{}, a package to simplify the insertion of parts
% of source code files into a document(ation).
% \end{abstract}
%
% \newif\ifmulticols
% \IfFileExists{multicol.sty}{\multicolstrue}{}
%
% \ifmulticols
% \addtocontents{toc}{\protect\begin{multicols}{2}}
% \fi
%
% \tableofcontents
%
% \section{Preface}
% \label{sec:preface}
%
% \changes{v09}{09/04/05}{added some text to the preface}%
% This package is
% intended as an implementation of the macros which are described in
% \cite{lingnau07:_latex_hacks}. If a software developer wants to document
% a piece of software, in most cases she/he doesn't need to print out whole
% source code files, but only parts of the files. This can be achieved by the
% \Lpack{listings}-package \cite{Heinz:Listings-14} and especially by the command
%  \begin{quote}
%      |\lstinputlisting[linerange={|\meta{first1}|-|\meta{last1}|}|, \ldots{}|]{|\meta{filename}|}|
%  \end{quote}
%  In \cite{lingnau07:_latex_hacks} there are described three macros, which
%  can be created automatically, so that in the case of changes in the
%  source code the developer mustn't change the contents of the line ranges,
%  but she/he only has to rerun a program, which regenerates the meaning of
%  the macros. In the following the three macros and a Bash-script to deal with
%  these three macros are provided.
%
%
% \changes{v16}{09/08/18}{added macros to make the documentation
% independent of non standard packages}
% \section[Installation and Maintenance]{Hints For Installation And Maintenance}
% \label{sec:installation-maintenance}
%
% The package \listingsextSty{} is belonging to consists of altogether four files:
% \begin{quote}
% \Fname{README}, \\
% \Fname{listings-ext.ins}, \\
% \Fname{listings-ext.dtx}, \\
% \Fname{listings-ext.pdf}.
% \end{quote}
% In order to generate on the one hand the documentation of the package and
% on the other hand the corresponding \ext{.sty}-file and the supporting Bash-script
% one has to proceed as follows:
%
% \changes{v46}{09/08/31}{enhanced the documentation at several points}
% First the file \textsf{listings-ext.ins} must be formatted e.g.\ by
% \begin{quote}
%\begin{verbatim}
%latex listings-ext.ins
%\end{verbatim}%
% \end{quote}
% This formatting run generates several other files.  These are first of
% all the previous mentioned \ext{.sty}-file \Fname{listings-ext.sty}, the
% style file which can be used by |\usepackage{listings-ext}| and the
% Bash-script |listings-ext.sh|. Then there are the files
% \Fname{listings-ext.bib} and \Fname{hyperref.cfg}, which are needed to
% produce this documentation. The creation of the documentation is is
% simplified by the files \Fname{listings-ext.makemake} and
% \Fname{listings-ext.mk}, which can be used to make a \Fname{Makefile}
% (see section~\ref{sec:Makefile}). Another helper file is
% \Fname{listings-ext.el}, which can be used with the \XEmacs{} editor (see
% section~\ref{sec:AUCTeX}). Some simple tests of this package can be done
% by using the files \Fname{listings-ext*exmpl*} and
% \Fname{listings-ext*test*}, which are also created; these files use the
% configuration file \Fname{listings.cfg}, which can also be used as a base or
% supplement for own configuration files of the
% \Lpack{listings}-package. This file can be put into the current
% directory for local use or into the |TEXMFHOME| directory for use with
% every document. You have to take care about the fact, that the local
% file in the current directory prevent any |listings.cfg| from being loaded.
% Finally there is a file \Fname{getversion.tex}, which is used for the
% creation of a "versioned" distribution.
%
% The common procedure to produce the documentation is to execute the
% commands
% \begin{quote}
%\begin{verbatim}
%latex listings-ext.dtx
%bibtex listings-ext
%latex listings-ext.dtx
%makeindex -s gind.ist listings-ext.idx
%makeindex -s gglo.ist -o listings-ext.gls -t listings-ext.glg \
%        listings-ext.glo
%latex listings-ext.dtx
%\end{verbatim}
% \end{quote}
% The result of this formatting run is the documentation in form of a
% \ext{.dvi}-file, that can be manipulated the normal way. It ain't
% possible to use |pdflatex| because of the integrated PostScript based
% figures.
%
% One can find further informations about working with the integrated
% documentation in \cite{FM:TheDocAndShortvrbPackages} and
% \cite{MittelbachDuchierBraams:DocStrip}.\footnote{Generating the
%   documentation is much easier with the \texttt{make} utility, see
%   section~\ref{sec:Makefile}.}
%
% \changes{v33}{09/08/23}{cleaned up the driver once more}
% \changes{v33}{09/08/23}{moved hyperref settings into hyperref.cfg}
% \changes{v23}{09/08/20}{cleaned up the code}
% \changes{v16}{09/08/18}{replaced last german sentences from
% documentation}
% This documentation itself is generated by the following internal driver
% code:
%    \begin{macrocode}
%<*driver>
\documentclass[a4paper, ngerman, english]{ltxdoc}

\usepackage[T1]{fontenc}
\usepackage{lmodern}
\usepackage{babel,babelbib}
\usepackage[svgnames]{pstricks}

\usepackage{listings-ext}
\GetFileInfo{listings-ext.sty}

\newif\ifcolor \IfFileExists{color.sty}{\colortrue}{}
\ifcolor \RequirePackage{color}\fi

\usepackage[numbered]{hypdoc}
\usepackage{url}

%\EnableCrossrefs
%\DisableCrossrefs    % say \DisableCrossrefs if index is ready
%\RecordChanges       % gather update information
%\CodelineIndex       % index entry code by line number
\OnlyDescription    % comment out for implementation details
\MakeShortVerb{\|}   % |\foo| acts like \verb+\foo+

\begin{document}

    \DocInput{listings-ext.dtx}%
\end{document}
%</driver>
%    \end{macrocode}
%
% \changes{v13}{09/05/28}{enhanced the documentation}
% \section{The User's Interface}
% \label{sec:users-interface}
%
% \changes{v16}{09/08/18}{extended the description of usage}
% \subsection{Preparing the \LaTeX{} code}
% \label{sec:preparing-latex}
%
% The user's interface is as simple as possible: just load the required
% package by saying
% \begin{quote}
%     |\usepackage[|style=\meta{style-name}|]{listings-ext}|
% \end{quote}
% \meta{style-name} is the name of a \textsf{listings} style, defined by
% the command \cite[sec.~4.5]{Heinz:Listings-14}. You can find examples for
% such styles in the exemplary configuration file \Fname{listings.cfg}.
% \begin{quote}
%     |\lstdefinestyle{|\meta{style name}|}{|\meta{key=value list}|}|
% \end{quote}
%
% After loading the package provides three commands:
% \begin{enumerate}
%   \item \begin{quote}
%         |\lstdef{|\meta{identifier}|}{|\meta{file-name}|}{|\meta{range}|}|
%     \end{quote}
%     defines the \meta{identifier}, by which the line range \meta{range} in the
%     file defined by \meta{file-name} can be referenced. If you identify
%     several \meta{filename}s or \meta{range}s by the same
%     \meta{identifier}, the last definition is valid. If you don't like
%     that behaviour, put the corresponding |\lstdef|- and
%     |\lstuse|-commands (see below) into a group of its own.
%   \item \begin{quote}
%         |\lstuse[|\meta{options}|]{|\meta{identifier}|}|
%     \end{quote}
%     includes the source code which is referenced by \meta{identifier} by
%     (internally) calling |\lstinputlisting| of the package
%     \Lpack{listings} \cite{Heinz:Listings-14}, the way of formatting can be influenced by
%     \meta{options}.
%   \item
%     \begin{quote}
%         |\lstcheck{|\meta{identifier}|}{|\meta{file-name}|}|
%     \end{quote}
%     can be used, if one prepares a file \meta{file-name} consisting of a
%     lot of |\lstdef| commands. If \meta{identifier} isn't yet defined,
%     the file defined by \meta{file-name} is |\input|. This command is
%     especially helpful, if you prepare a presentation and you want to
%     format only single slides for testing their look.
% \end{enumerate}
%
%
% \subsection{Preparing the source code}
% \label{sec:preparing-source}
%
% If you just want to include a small part of a source code file, you can
% do that without touching the source: just write the corresponding
% commands |\lstdef| and |\lstuse| into your \LaTeX-code. But if the
% source code changes, you have to adapt the changes in the |\lstdef|
% command. That may be very tedious, if are you changing your sources often.
%
% It' better to automate that procedure, and one way of implementig that is
% done at the Bash-script \Fname{listings-ext.sh}. For working with that
% script you have to tag the parts of the source, which you want to
% document, by comments.
%
% At the moment there are three tags, which can be described by the
% following regular expressions:
% \begin{enumerate}
%     \item |^\ +|\meta{endline-comment-character(s)}|\ be:\ |\meta{string}|$|
%
%       This expression defines the beginning of the environment, which should be
%       |\lstinput| into the document.
%     \item |^\ +|\meta{endline-comment-character(s)}|\ ee:\  |\meta{string}|$|
%
%       This expression defines the end of that environment.
%     \item |^\ +|\meta{endline-comment-character(s)}|\ ce:\ |\meta{list of
%         keywords}
%
%       This expression defines, how the the environments defined by the above
%       introduced should be processed.
% \end{enumerate}
% The meaning of the regular expressions is: start a line with at least one
% blank space "\verb*| |", add endline comment characters (C++ and Java:
% |//|, Fortran: |!|) followed by another blank space. Then you have to
% enter "\verb*|be: |" for the beginning of a code environment, and %
% "\verb*|ee: |" for the end. In both cases the line must be ended with a
% string which should denote the meaning of the environment , the strings for the
% beginning and the end must be identical.\footnote{You can also use the
%   standard C comments |/*| \ldots|*/|, but in that case the trailing "|*/|"
%   is seen as the end of the \meta{string}.}
%
% \changes{v63}{10/06/22}{added remarks about Max OS X}
% \changes{v53}{10/03/09}{issue solved: now able to handle any file name as
% list of source files}
% If you have prepared a source code file with these tags, you can process
% it by the Bash-script provided by the package. The script should work for
% a Linux system out of the box, for a Mac OS X 10.x one must additionally
% install |getopt| from \url{http://getopt.darwinports.com/}, which in turn
% needs |MacPorts| (from
% \url{http://www.macports.org/install.php}).\footnote{The package is
% tested with Max OS X v\,10.6.3, |getopt|'s version number is v\,1.1.4 and
% |MacPorts| version number is v\,1.9.0.}
%
% The simplest way to do it is the call
% \begin{quote}
%     |listings-ext.sh -co |\meta{file list}
% \end{quote}
% \meta{file list} is a list of one or more file names.  This call creates
% the file \meta{directory}|.lst|, where \meta{directory} is the name of
% the current directory. The file consists of a header and a list of |\lstdef|
% commands,
%  of the form
% \begin{quote}
%     |\lstdef{|\meta{identifier}|}{|\meta{filename}|}{|\meta{line range(s)}|}|
% \end{quote}
% You can |\input| the file \meta{directory}|.lst| into your documentation;
% its header looks like (for example)
%\begin{verbatim}
% %% -- file listings-ext.lst generated on Mi 26. Aug 22:05:20 CEST 2009
%         by listings-ext.sh
% \csname listingsextLstLoaded\endcsname
% \expandafter\let\csname listingsextLstLoaded\endcsname\endinput
%\end{verbatim}
% The first line is wrapped by hand, the second and third line prohibit a
% second load of that file.  One of the |\lstdef| could look like
%\begin{verbatim}
% \lstdef{listingsExtExmplAA}{/home/xxx%
%       /listings-ext%
%       /listings-ext_exmpl_a.java}{3-5}
%\end{verbatim}
%
% You can input this file in two ways into your document:
% \begin{enumerate}
%   \item by saying
%\begin{verbatim}
% %%
%% $Revision: 55 $
%%
%% This file will generate fast loadable files and documentation
%% driver files from the .dtx files in this package when run through
%% LaTeX or TeX.
%%
%% IMPORTANT COPYRIGHT NOTICE:
%%
%% No other permissions to copy or distribute this file in any form
%% are granted and in particular NO PERMISSION to modify its contents.
%%
%% You are NOT ALLOWED to change this file.
%%
%% --------------- start of docstrip commands ------------------
%%
\def\batchfile{listings-ext.ins}
\input docstrip.tex

\preamble

Copyright (C) 2008-2010 Jobst Hoffmann, <j.hoffmann (at) fh-aachen.de>, all rights reserved
--------------------------------------------------------------------------------------

This file may be distributed and/or modified under the
conditions of the LaTeX Project Public License, either version 1.2
of this license or (at your option) any later version.
The latest version of this license is in:

    http://www.latex-project.org/lppl.txt

and version 1.2 or later is part of all distributions of LaTeX
version 1999/12/01 or later.

Please address error reports and any problems in case of UNCHANGED versions
to
        j.hoffmann (at) fh-aachen.de
\endpreamble

\declarepostamble\examplepost
\endpostamble

%\BaseDirectory{~/TeX/texmf}
%\UseTDS
%\usedir{tex/latex/jhf}

\keepsilent
\askonceonly

%\def\targetdirectory{}                          % or may be for example
%\def\targetdirectory{./../../texmf/tex/latex/jhf}


% programs and packages

\Msg{*** Generating the package files ***}

\generate%
{%
    \askforoverwritefalse
    \file{listings-ext.sty}{%
      \from{listings-ext.dtx}{extension}}
}

\Msg{*** Generating the test and example files ***}

\preamble
\endpreamble
\generate%
{%
    \askforoverwritefalse
    \nopreamble\nopostamble\def\MetaPrefix{}%
    \file{listings-ext_exmpl_a.java}{\from{listings-ext.dtx}{example1}}
    \file{listings-ext_exmpl_b.java}{\from{listings-ext.dtx}{example2}}
    \file{listings-ext_exmpl_c.java}{\from{listings-ext.dtx}{example3}}
    \file{listings-ext_exmpl_d.java}{\from{listings-ext.dtx}{example4}}
    \file{listings-ext_exmpl_e.java}{\from{listings-ext.dtx}{example5}}
    \file{listings-ext_test_a.tex}{\from{listings-ext.dtx}{test1}}
    \file{listings-ext_test_d.tex}{\from{listings-ext.dtx}{test2}}
}

\Msg{*** Generating make file and make setup file  ***}
\generate
{%
    \nopreamble\nopostamble\def\MetaPrefix{}%
    \file{getversion.tex}{\from{listings-ext.dtx}{getversion}}%
    \file{hyperref.cfg}{\from{listings-ext.dtx}{hyperref}}%
    \file{listings-ext.bib}{\from{listings-ext.dtx}{bibtex}}%
    \file{listings-ext.el}{\from{listings-ext.dtx}{auctex}}%
    \file{listings-ext.mk}{\from{listings-ext.dtx}{makefile}}%
    \file{listings-ext.makemake}{\from{listings-ext.dtx}{setup}}%
    \file{listings.cfg}{\from{listings-ext.dtx}{listings}}%
}

\Msg{*** Generating the script file ***}
\generate
{%
    \nopreamble\nopostamble\def\MetaPrefix{}%
    \file{listings-ext.sh}{\from{listings-ext.dtx}{scriptfile}}%
}

\ifToplevel{%
\Msg{***********************************************************}
\Msg{*}
\Msg{* To finish the installation you have to move the following}
\Msg{* style file(s) into a directory searched by TeX:}
\Msg{*}
\Msg{* \space\space listings-ext.sty}
\Msg{*}
\Msg{* To produce the documentation run the file(s) ending with}
\Msg{* `.dtx' through LaTeX.}
\Msg{*}
\Msg{* Happy TeXing}
\Msg{***********************************************************}
}

\endinput
%%% Local Variables:
%%% mode: latex
%%% TeX-master: t
%%% mode: flyspell
%%% ispell-local-dictionary: "english"
%%% End:

%\end{verbatim}
% at the beginning of a file or
%  \item by saying
%\begin{verbatim}
% \lstcheck{listingsExtExmplAA}{listings-ext.lst}
%\end{verbatim}
%    in an environment to keep the definition local.
% \end{enumerate}
%
% After that you can use the command |\lstuse| to integrate the source
% code parts into your documentation.  The usage is
% \begin{quote}
%     |\lstuse[|\meta{options}|]{|\meta{identifier}|}|
% \end{quote}
% at the place, where you want the part of your source code.
%
% The \meta{identifier} is generated automatically, it is derived from the
% file name, you have to transfer the identifiers from the \ext{.lst}-file
% to the |\lstuse|-command by hand, but that happens typically only one
% time. So in this case the |\lstuse| command could look like --- as said
% you can add options ---
%\begin{verbatim}
% \lstuse{listingsExtExmplAA}
%\end{verbatim}
%
% For more information about the use of the Bash-script |listings-ext.sh|
% enter the command
% \begin{quote}
%     |listings-ext.sh -h|
% \end{quote}
%
% \changes{v09}{09/04/05}{introduced a description of the ce-commands}
% There is one optional initial tag |ce:| (\textit{control environment}): It needs one
% argument \meta{mode}. The argument describes the further
% processing. \meta{mode} may be one of the following values:
% \begin{description}
%   \item[\texttt{combine}:] |be: | \ldots{}|ee: | groups with the same
%     description are combined into one piece of code in the output
%   \item[\texttt{join}:] all |be: | \ldots{}|ee: | groups (independent of
%     the description) are combined into one piece of code in the output
% \end{description}
% The behaviour of the three modes of operation is shown in
% Figure~\ref{fig:modes-operation}.
% |ce:| has to be put before all other tags in the source code.
% \changes{v35}{09/08/27}{added figure to show the behaviour of listings-ext.sh}
% \begin{figure}[htb]
%     \psset{unit=6mm} \centering
%     \begin{pspicture}(0,-1.0)(4,7) \rput[lB](0,4.5){\SOURCEA{g}}
%         \rput[lB](0,3.0){\SOURCEB{g}} \rput[lB](0,1.0){\SOURCEC{g}}
%         \rput[lB](0,0){\SOURCEFRAME} \rput[lB](2.5,4.5){\SOURCEA{f}}
%         \rput[lB](2.5,3.0){\SOURCEB{f}} \rput[lB](2.5,1.0){\SOURCEC{f}}
%         \rput(2.5,-0.7){a)}
%     \end{pspicture}
%     \hspace{2cm}
%     \begin{pspicture}(0,-1.0)(4,7) \rput[lB](0,4.5){\SOURCEA{g}}
%         \rput[lB](0,3.0){\SOURCEB{g}} \rput[lB](0,1.0){\SOURCEC{g}}
%         \rput[lB](0,0){\SOURCEFRAME} \rput[lB](2.5,3.5){\SOURCEA{}}
%         \rput[lB](2.5,3.0){\SOURCEB{}} \rput[lB](2.5,1.75){\SOURCEC{}}
%         \psframe(2.5,1.75)(4.0,5.5) \rput(2.5,-0.7){b)}
%     \end{pspicture}
%     \hspace{2cm} \begin{pspicture}(0,-1.0)(4,7)
%           \rput[lB](0,4.5){\SOURCEA{g}} \rput[lB](0,3.0){\SOURCEB{g}}
%           \rput[lB](0,1.0){\SOURCEC{g}} \rput[lB](0,0){\SOURCEFRAME}
%           \rput[lB](2.5,3.5){\SOURCEA{}} \rput[lB](2.5,2.25){\SOURCEC{}}
%           \psframe(2.5,2.25)(4.0,5.5) \rput[lB](2.5,1.0){\SOURCEB{}}
%           \psframe(2.5,1.0)(4.0,1.5) \rput(2.5,-0.7){c)}
%       \end{pspicture}
%     \caption[Modes of operation]{Modes of operation: a) no special control of the tagged
%       parts, every part can be processed by itself, b) control by |ce:
%       join|, all tagged parts are joint into one piece, which can be
%       further processed, c) control by |ce: combine|, tagged parts with
%       the same describing string are joint into one piece, which can be
%       further processed, all other parts can be processed by their own}
%     \label{fig:modes-operation}
% \end{figure}
%
%
% \changes{v33}{09/08/23}{moved bibliography into listings-ext.bib}
% \changes{v27}{09/08/21}{a few small layout changes}
% \changes{v16}{09/08/18}{internalized the bibliography}
% \StopEventually{
%   \bibliographystyle{babalpha}
%   \bibliography{listings-ext}
%
%   \ifmulticols
%   \addtocontents{toc}{\protect\end{multicols}}
%   \fi
%
%   \PrintChanges
%   \PrintIndex
% }   ^^A  end of \StopEventually
%
% \clearpage
%
%
% \changes{v16}{09/08/18}{started the documentation of the implementation}
% \section{The Implementation}
% \label{sec:the-implementation}
%
% \subsection{The style file}
% \label{sec:the-style-file}
%
% At first we must declare the required packages :
%    \begin{macrocode}
%<*extension>
\RequirePackage{listings}
\RequirePackage{xkeyval}
%    \end{macrocode}
% and the package options for \listingsextSty.
% \begin{macro}{style}
% is the only option at the moment.
%    \begin{macrocode}
\DeclareOptionX{style}[]{\lstset{style=#1}}
\ProcessOptionsX*
%    \end{macrocode}
% \end{macro}
%
% Then we define the three above mentioned commands in the way, which is
% described in \cite{lingnau07:_latex_hacks}.
% \begin{macro}{\lstdef}
% \changes{v46}{09/08/31}{enhanced the documentation}
%     has three parameters:
%     \begin{quote}
%         \begin{tabular}{ll}
%             \#1 & the identifier of the chunk of source code \\
%             \#2 & the file name of the source code \\
%             \#3 & the line range \\
%         \end{tabular}
%     \end{quote}
%     Every parameter can be used at the user's choice, so there is no chance
%     to test the parameters.
%    \begin{macrocode}
\newcommand{\lstdef}[3]{%
  \@namedef{lst@l@#1}##1{%
  \lstinputlisting[##1, linerange={#3}]{#2}}%
}
%    \end{macrocode}
% \end{macro}
%
% \begin{macro}{\lstcheck}
% \changes{v46}{09/08/31}{enhanced the documentation}
%     has two parameters:
%     \begin{quote}
%         \begin{tabular}{ll}
%             \#1 & the identifier of the chunk of source code \\
%             \#2 & the file name of a file with many |lstdef| lines \\
%         \end{tabular}
%     \end{quote}
%     \changes{v45}{09/08/31}{added some checks for arguments}
%    \begin{macrocode}
\newcommand{\lstcheck}[2]{%
%    \end{macrocode}
% At first we test, if the identifier is unknown. If so, we read the above
% mentioned file.
%    \begin{macrocode}
  \expandafter\ifx\csname lst@l@#1\endcsname\relax%
  \input{#2}\fi%
%    \end{macrocode}
% Now the identifier should be known. If not, we emit a corresponding error
% message.
%    \begin{macrocode}
  \expandafter\ifx\csname lst@l@#1\endcsname\relax%
  \PackageError{listings-ext}{undefined identifier}{%
  You may have mispelled the identifier, check the file%
  \MessageBreak\space\space\space#2\MessageBreak%
  for the correct spelling.}\fi%
}
%    \end{macrocode}
% \end{macro}
%
% \begin{macro}{\lstuse}
% \changes{v46}{09/08/31}{enhanced the documentation}
%     has two parameters:
%     \begin{quote}
%         \begin{tabular}{ll}
%             \#1 & options of the |\lstinputlisting| command (optional) \\
%             \#2 & the above defined identifier \\
%         \end{tabular}
%     \end{quote}
%     \changes{v45}{09/08/31}{added some checks for arguments}
%    \begin{macrocode}
\newcommand{\lstuse}[2][]{%
%    \end{macrocode}
% At first we test, if the identifier is unknown. If so, we emit a
% corresponding error message.
%    \begin{macrocode}
  \expandafter\ifx\csname lst@l@#2\endcsname\relax%
  \PackageError{listings-ext}{undefined identifier}{%
  You may have mispelled the identifier.\MessageBreak%
  If you go on without any change, no source code will be %
  included,\MessageBreak but your options will be written %
  into your formatted document.}\fi%
%    \end{macrocode}
% Now we can |\lstinputlisting| the source code by use of the
% |\@namedef| ed command.
%    \begin{macrocode}
  \@nameuse{lst@l@#2}{#1}%
}
%</extension>
%    \end{macrocode}
% \end{macro}
% That' all for the \LaTeX-side of life.
%
%
% \changes{v46}{09/08/31}{enhanced the documentation at several points}
% \changes{v45}{09/08/31}{introduced an exemplary configuration file}
% \subsection{An exemplary configuration file}
% \label{sec:a-configuration-file}
%
% It is a good practise, to format all source code chunks in the same
% way. The easiest way to do that is to provide a style. The following
% lines provide two styles, one for the use with a black and white
% printer, the other for a printer which is capable of color output.
%    \begin{macrocode}
%<*listings>
\ProvidesFile{listings.cfg}%
  [2009/08/23 v1.0 listings configuration of listings-ext]

\RequirePackage{xcolor}

\def\lstlanguagefiles{lstlang1.sty,lstlang2.sty,lstlang3.sty}
\lstset{defaultdialect=[ANSI]C,
        defaultdialect=[ISO]C++,
        defaultdialect=[95]Fortran,
        defaultdialect=Java,
        defaultdialect=[LaTeX]TeX,
        frame=tlb,
        resetmargins=false,
        }
\lstdefinestyle{colored-code}{
  backgroundcolor=\color{yellow!10},%
  basicstyle=\footnotesize\ttfamily,%
  identifierstyle=\color{black},%
  keywordstyle=\color{blue},%
  stringstyle=\color{teal},%
  commentstyle=\itshape\color{orange},%
}
\lstdefinestyle{bw-code}{
  basicstyle=\small\fontfamily{lmtt}\fontseries{m}\fontshape{n}\selectfont,
  % instead of lmtt one should use ul9 (luximono) for boldface characters
  keywordstyle=\small\fontfamily{lmtt}\fontseries{b}\fontshape{n}\selectfont,
  commentstyle=\small\fontfamily{lmtt}\fontseries{m}\fontshape{sl}\selectfont,
  stringstyle=\small\fontfamily{lmtt}\fontseries{m}\fontshape{it}\selectfont,
}
%</listings>
%    \end{macrocode}
%
%
% \changes{v16}{09/08/18}{moved the documentation of the script into the documentation}
% \subsection{The shell script}
% \label{sec:the-shell-script}
%
% \changes{v58}{10/05/18}{removed some bashisms (according to Ahmed El-Mahmoudy---aelmahmoudy\_at\_sabily.org)}
% \changes{v37}{09/08/27}{removed small glitches}
% \changes{v23}{09/08/20}{moved some more comment from the script into the documentation}
% \changes{v12}{09/05/12}{added some debug code}
% \changes{v11}{09/04/15}{added some documentation}
% \changes{v09}{09/04/05}{added some space for a better layout}
% \changes{v05}{08/03/04}{add a debug option for the script}
% \changes{v05}{08/03/04}{change the behaviour: hyphens and underscores are removed in
%  the .lst-file}
% \changes{v05}{08/03/04}{check the correct grouping of be...ee}
% \changes{v05}{08/03/04}{function delete\_underscore renamed to "replace\_characters"}
% \changes{v05}{08/03/04}{better error processing with messages to stderr}
% \changes{v05}{08/03/04}{in replace\_characters now also hyphens are replaced}
% \changes{v05}{08/03/04}{introduced function print\_error}
% \changes{v05}{08/03/04}{introduced function break\_line}
% \changes{v03}{08/02/27}{corrected the on-line help according to the current implementation}
% \changes{v03}{08/02/27}{changed documentation and behaviour of function print\_header}
% \changes{v03}{08/02/27}{fixed some typos}
% \changes{v03}{08/02/27}{output option now has an optional argument}
% \changes{v03}{08/02/27}{call of print\_header changed according to the
% optional argument from above}
% \changes{v06}{08/09/07}{created list of things to do}
% To get full benefit of the three above introduced commands, we provide a
% Bash-script, which internally uses |awk|, |grep| and |sed|. A Perl
% programmer could provide the same functionality by a Perl script.
%
%    \begin{macrocode}
%<*scriptfile>
#! /bin/sh
### listings-ext.sh ---

## Author: Dr. Jobst Hoffmann <j.hoffmann_(at)_fh-aachen.de>
## Version: $Id: listings-ext.dtx 67 2010-06-29 16:38:12Z ax006ho $
## Keywords: LaTeX, listings
## Copyright: (C) 2008-2010 Jobst Hoffmann, <j.hoffmann_(at)_fh-aachen.de>
##-------------------------------------------------------------------
##
## This file may be distributed and/or modified under the
## conditions of the LaTeX Project Public License, either version 1.2
## of this license or (at your option) any later version.
## The latest version of this license is in:
##
##    http://www.latex-project.org/lppl.txt
##
## and version 1.2 or later is part of all distributions of LaTeX
## version 1999/12/01 or later.
##
## This file belongs to the listings-ext package.
##
## listings-ext.sh creates code to be inserted into a
## LaTeX file from a source file.
##
## The commented code can be seen in listings-ext.pdf, which is provided
## with the distribution package.
%    \end{macrocode}
%
% The format of the result is depending on a switch \verb!-c|--command!:
% \begin{itemize}
%   \item If the switch is set, the result is a single file which can be
%     |\input| into LaTeX code. The file contains lines of the form
%     \begin{quote}
%         |\lstdef{|\meta{identifier}|}{|\meta{filename}|}{|\meta{linerange(s)}|}|
%     \end{quote}
%     \meta{identifier} can be used in the \LaTeX{} code by a command
%     \begin{quote}
%         |\lstuse{|\meta{identifier}|}|
%     \end{quote}
%     to |\lstinputlisting| the source code at that place.
%     \meta{identifier} is derived from the filename of the source file,
%     where there is a sequence of uppercase letters (A, B, \ldots{}, Z,
%     BA, \ldots{}) appended as postfix to distinguish between several
%     chunks. If we are in join mode --- see below ---, there will be no
%     postfix at all.  You have to be careful with the filenames: as the
%     identifiers are used as \TeX{}-commands, the have to consist only of
%     characters, but the script allows underscores and hyphens in the
%     filenames and translates them into a camel case notation.
%   \item If the switch is not set, a piece of LaTeX code like the
%     following is generated for every chunk of source code:
%\begin{verbatim}
% %===>begin{listingsExtExmplAC}
% {%
%   \def\inputfile{%
%     /home/xxx/texmf/source/latex/listings-ext%
%       /listings-ext_exmpl_a.java
%   }
%   \ifLecture{%
%     \lstdefinestyle{localStyle}{%
%       caption={\lstinline|listings-ext_exmpl_a.java|: using the resource bundle}
%     }%
%   }
%   {%
%     \lstdefinestyle{localStyle}{%
%       title={\lstinline|listings-ext_exmpl_a.java|: using the resource bundle}
%     }%
%   }
%   \lstinputlisting[style=localStyle, linerange={29-30}]%
%   {%
%     \inputfile
%   }
% }%
% % =====>end{listingsExtExmplAC}
% \end{verbatim}
% \end{itemize}
%
% The source file contains pairs of lines of the form described by the
% following regular expressions:
% \begin{itemize}
%     \item |^\ +|\meta{endline-comment-character(s)}|\ be: |\meta{string}|$|
%
% This expression defines the beginning of a environment, which should be
% |\lstinput| into the document.
%     \item |^\ +|\meta{endline-comment-character(s)}|\ ee: |\meta{string}|$|
%
% This expression defines the end of that environment.
% \end{itemize}
%
% Another line can preceed all pairs:
% \begin{quote}
%     |\ +|\meta{endline-comment-character(s)}|\ ce: |\meta{list of keywords}
% \end{quote}
%
% \meta{list of keywords} is a comma separated list of words, the first one
% must be one of the words
% \begin{itemize}
% \item |join|
% \item |combine|
% \end{itemize}
%
% At the moment we support only these keywords. The meaning of the
% keywords is
% \begin{itemize}
% \item |join|: we generate lines, which print the listing as one part
% \item |combine|: we generate lines, which print the listing as one part, only
%            for headings, which correspond
% \end{itemize}
%
% These lines will be grepped. The result is processed by |awk| where '|:|' is
% defined as a field separator, the final result is a range of lines, which
% is further processed to create the output.
% \changes{v56}{10/03/09}{a new usage description, added new option "-e"}
% \changes{v33}{09/08/23}{corrected description of usage of the script}
%    \begin{macrocode}
ERR_HELP=1
ME=$(basename $0)
USAGE="usage:\n\t${ME} [-c|--command] [-e|--ext] [-g|--debug] \\\\\n\
\t\t[-h|--help] [-n|--no-header] \\\\\n\
\t\t[{-o|--output-file} [<output filename>]] <filename>, ...\n\
\t-c:\tgenerate commands, which can be input and later used by\n\
\t\t\\\\lstuse, optional\n\
\t-e:\tinclude file name extension into the identifier\n\
\t-g:\tdebug mode, create some additional output, optional\n\
\t-h:\tshow this help, optional\n\
\t-n:\twrite no header into the LaTeX code, optional; valid only,\n\
\t\tif -c isn't used\n\
\t-o [<output filename>]: if this argument is present, the output will\n\
\t\tbe written into a file <output filename>; if the\n\
\t\t<output filename> is omitted, the output is redirected\n\
\t\tinto a file with a basename corresponding to the name\n\
\t\tof the current directory, it has an extension of \".lst\".\n\
"
DEBUG_MODE=0
EXTENSION=0
HEADER=1

show_usage() { # display help massage
      printf "${USAGE}"
      exit ${ERR_HELP}
}
%    \end{macrocode} ^^A $
% \changes{v11}{09/04/15}{corrected generation of filename for .lst-output file}
% \changes{v09}{09/04/05}{changed the documentation of the function
% print\_header}
% We define some functions for the internal use. The first one prints the
% header of the output file, it has two arguments:
% \begin{itemize}
%   \item \$1: file name---the name of the file to be written on
%   \item \$2: directory - the last name in the directory path (currently
%     not used)
% \end{itemize}
% \changes{v63}{10/06/22}{replaced most of the echo commands by printf (because of Mac OS X)}
% \changes{v53}{10/03/09}{issue solved: now able to handle any file name as
% list of source files}
%    \begin{macrocode}
print_header() {
    FILE=$1
    printf "%%%% -- file $1 generated on $(date) by ${ME}\n"
    FILE=$(echo $(basename $1 .lst) | sed -e s/[_\-]//g)
    printf "\\\\csname ${FILE}LstLoaded\\\\endcsname\n"
    printf "\\\\expandafter\\\\let\\\\csname ${FILE}LstLoaded\\\\endcsname\\\\endinput\n"
}
%    \end{macrocode} ^^A $
% The main function is |do_the_job|. It accepts the following parameter:
% \begin{enumerate}
%   \item \$1: a filename
% \end{enumerate}
%    \begin{macrocode}
do_the_job() {
%    \end{macrocode} ^^A $
% In |awk| \$0 refers to the whole input record---to the current input
% line---, so that element could be split into its components which are
% marked by '/', the result is the name of the current directory without
% any path elements stored in an array
%\begin{verbatim}
%     PATHCOMPONENT=$(pwd | \
%       awk -F : '{n = split($0, text, "/"); print text[1]  text[n]}')
%\end{verbatim}
%
% The current implementation uses the whole path, it will be split into its
% parts when it is used.
%    \begin{macrocode}
    PATHCOMPONENT=$(pwd)
    SOURCEFILE=$1

    if [ ! -f ${SOURCEFILE} ]
    then
        printf "${SOURCEFILE} is no valid file\n"
        return $ERR_NOFILE
    fi
%    \end{macrocode} ^^A $
% \changes{v12}{09/05/12}{added support for ant: filenames are given complete with the path}
% If the script is called with the full path of the files, the directory
% part must be cut off  from filename
%    \begin{macrocode}
    SOURCEFILE=${SOURCEFILE##${PATHCOMPONENT}/}

%    \end{macrocode} ^^A $
% \changes{v65}{10/06/28}{grep: replaced blank by space regex}
% \changes{v51}{10/03/33}{corrected the grep expression according to the definition of the syntax}
% From the current source file, we grep all the lines with the above
% mentioned pattern, prepend them by a line number (parameter |-n|)
%    \begin{macrocode}
    grep -n "^[[:space:]]*[/%;#!][/\* ;][[:space:]]*[cbe]e:" $1 | \
%    \end{macrocode} ^^A $
% We pass all the lines to |awk|. Additionally we set some parameters and define
% the field separator by |-F| as "|:|", so that from |awk|'s point of view
% the lines consist of three parts:
% \begin{enumerate}
%   \item the line number
%   \item the opening/closing tag
%   \item an identifying string
% \end{enumerate}
% \changes{v56}{10/03/09}{a cleaner handling of passing parameters to awk}
%    \begin{macrocode}
     awk -v pathcomponent="${PATHCOMPONENT}" -v file="${SOURCEFILE}" \
         -v header="${HEADER}" -v command="${COMMAND}" -v application="${ME}" \
         -v debugmode="${DEBUG_MODE}" -v respect_extension="${EXTENSION}"\
           -F : \
'
BEGIN {
%    \end{macrocode} ^^A $
% \changes{v09}{09/04/05}{reflect the introduction of the command combine}
% We start the code with some settings.
% In |awk| the value 1 means "true", the value 0 "false".
%    \begin{macrocode}
    is_opened = 0;
    initialised = 0;
    configured = 0;
    join = 0;
    combine = 0;

    idChar = "ABCDEFGHIJKLMNOPQRSTUVWXYZ"
    split(idChar, idArray, "")
    idPt = 0
    linerange = "";

    linelen = 75; # length of the output line
    addchars = 8; # number of additional characters like %, > or so

    if ( debugmode ) printf("path: %s\n",  pathcomponent) > "/dev/stderr"
    if ( debugmode ) printf("file: %s\n", file) > "/dev/stderr"
%    \end{macrocode}
% For |\lstinputting| the file, we need the absolute filepath. In the
% following, we construct this, the correspondent variable is |inputfile|,
% the filename to be formatted is split into convenient pieces by a
% macro
% \changes{v63}{10/06/22}{rewrote code for splitting paths into names}
%    \begin{macrocode}
    n = split(pathcomponent, partsa, "/");
    curr_dir = parts[n]
%    \end{macrocode}
% \changes{v65}{10/06/28}{breaking of lines moved into a single place}
% Add the file to the array to process the whole name in a similar manner.
%    \begin{macrocode}
    n++;
    partsa[n] = file;
    inputfile = "/" partsa[2];
    for ( i = 3; i <= n; i++ ) {
        inputfile = inputfile "/" partsa[i];
    }
    if ( debugmode ) printf("inputfile: %s\n",  inputfile) > "/dev/stderr"
%    \end{macrocode}
% For the identification of the several chunks we need the filename
% without path and without suffix. We take everything behind the last
% dot as suffix
%
% \changes{v65}{10/06/28}{moved single \{ from beginning of line to end of previous line}
% The |identifier| mustn't contain any path parts of the filename. So we
% split the filename here.
%    \begin{macrocode}
    n = split(file, partsa, "/")
    n = split(partsa[n], partsb, ".")
    identifier = partsb[1]
    for ( i = 2; i < n; i++ ) {
        identifier = identifier partsb[i]
    }
%    \end{macrocode}
% If there are files with the same basename, but different extensions, we
% may take account of the extensions.
% \changes{v56}{10/03/09}{added processing of option "-e"}
%    \begin{macrocode}
    if ( respect_extension ) {
        identifier = identifier partsb[n]
    }
    identifier = replace_characters(identifier)
    if ( debugmode ) printf("identifier: %s\n", identifier) > "/dev/stderr"
}
%    \end{macrocode}
% The following code will be executed for every line, the standard delimiter is ":"
%    \begin{macrocode}
{
%    \end{macrocode}
% \changes{v09}{09/04/05}{test external (LaTeX-)documentation of the code}
% |$0| is the whole line, |$NF| the text, which is used in combine mode as
% identifier not only for the beginning and ending of a chunk but also for
% collecting all the chunks.
%    \begin{macrocode}
    if ( debugmode ) print $0 > "/dev/stderr"
    if ( !initialised ) {
%    \end{macrocode} ^^A $
% \changes{v67}{10/06/29}{added remark about CR and LF}
% \changes{v09}{09/04/05}{extend the external documentation}
% In the beginning, there are two cases to distinguish:
% \begin{enumerate}
%   \item There is an introducing |ce| marker
%   \item There is no introducing |ce| marker
% \end{enumerate}
% In the first case we have to set a flag according to the value of the
% mark, in the second case we have to look for the first mark (|be| or
% |ce|). The marker is supposed to end at the line's end. If a \LaTeX{}
% source file is created with an editor, which creates the CRLF combination
% as the end of line, the CR would be thought as part of the marker. To
% avoid the trouble, the CR is removed (if it exists).
% \changes{v65}{10/06/28}{added code to remove spurious carriage return character}
%    \begin{macrocode}
        if ( match($2, /ce/) > 0 ) {
            n = split($(NF), text, ",");
            if ( match(text[1], "join") ) {
                join = 1;
            } else if ( match(text[1], "combine") ) {
                combine = 1;
            }
        } else if ( match($2, /be/) > 0 ) {
            opening_tag = $(NF)
            gsub(/\r/, "", opening_tag);
            is_opened = 1
            start = $1 + 1; # entry $1 is the line number
        } else if ( match($2, /ee/) > 0 ) {
            print_error($1, "missing \"be\" declaration")
        }
        initialised = 1;
    } else {
        tag = $(NF)
        if ( match($2, /be/) > 0 ) {
            if ( is_opened ) {
                print_error($1, "incorrect grouping, previous group"\
                " not closed")
            } else {
                opening_tag = $(NF)
                gsub(/\r/, "", opening_tag);
                is_opened = 1
                start = $1 + 1; # entry $1 is the line number
            }
        } else {
            if ( match($2, /ee/) > 0 ) {
                closing_tag = $(NF)
                gsub(/\r/, "", closing_tag);
                if ( !is_opened ) {
                    print_error($1, "missing \"be\" declaration")
                } else if ( opening_tag == closing_tag ) {
                    is_opened = 0
%    \end{macrocode} ^^A $
% \changes{v65}{10/06/28}{changed the prefix to a double \% to unify the output}
% We found the correct end of the environment, now we prepare the output entries
%    \begin{macrocode}
                    split($(NF), text, "*"); # omit a trailing comment
                    sub(/ /, "", text[1]); # delete leading spaces
                    gsub(/"/, "", text[1]);
                    if ( index(text[1], "_") > 0 ) gsub(/_/, "\\_", text[1]);
                    caption =  "\\lstinline|" file "|: " text[1]

                    # setup the prefixes
                    len = linelen - addchars - length(caption);
                    begin_prefix = "%%";
                    cnt = 0;
                    while ( cnt < len) {
                        begin_prefix = begin_prefix "=";
                        cnt++;
                    };
                    begin_prefix = begin_prefix ">";
                    end_prefix = begin_prefix;
                    sub(/%%/, "%%==", end_prefix);
%    \end{macrocode} ^^A $
% \changes{v09}{09/04/05}{rewrote the check of the flags}
% Now we distinguish between |join|- and |combine|-mode and the default mode:
%    \begin{macrocode}
                    if ( join ) {
%    \end{macrocode} ^^A $
% In |join|-mode we have to collect all informations about chunks into one
% linerange, this is done by
%    \begin{macrocode}
                        linerange = linerange ", " start "-" $1-1;
                    } else if ( combine ) {
%    \end{macrocode} ^^A $
% \changes{v11}{09/04/15}{rewrote some code for adding the combine mode}
% In combine mode we need the whole tag but without leading or trailing
% blanks, so we delete leading spaces.
%    \begin{macrocode}
                        sub(/ /, "", closing_tag)
                        if ( combine_array[closing_tag] == "" ) {
                            combine_array[closing_tag] = start "-" $1-1
                        } else {
                            combine_array[closing_tag] = \
                            combine_array[closing_tag] ", " start "-" $1-1
                        }
                        if ( debugmode ) printf("combine_array: >%s<\n",\
                                combine_array[closing_tag]) > "/dev/stderr"
                    } else {
%    \end{macrocode} ^^A $
% We are neither in |combine|- nor in |join|-mode, so we can print out the
% separate linechunks directly.
%
% The line range for a separated chunk is calculated by
%    \begin{macrocode}
                        linerange = start "-" $1-1;
                        if ( command ) {
                            print_command(\
                                    (identifier toB26(idPt++)), \
                                    linerange);
                        } else {
                            print_linerange(\
                                    (identifier toB26(idPt++)), \
                                    caption, linerange);
                        }
                    }
                } else if ( opening_tag != closing_tag ) {
                    print_error($1, "opening and closing tags differ")
                } else {
                    print_error($1, "unknown error")
                }
            }
        }
    }
}
END {
%    \end{macrocode} ^^A $
% There may be an erroneous open group, this is announced now.
% \changes{v56}{10/03/09}{added handling of missing "ee" declarations}
%    \begin{macrocode}
    if ( is_opened ) {
        print_error($1, "missing \"ee\" declaration")
    }
%    \end{macrocode} ^^A $
% And now for the post processing \ldots
%    \begin{macrocode}
    caption = "\\lstinline|" file "|";
    if ( join ) {
        sub(/, /, "", linerange);
        if ( command ) {
            print_command(identifier, linerange);
        } else {
            caption = "\\lstinline|" file "|";
            print_linerange(identifier, caption, linerange);
        }
    } else if ( combine ) {
        for ( range in combine_array ) {
%    \end{macrocode} ^^A $
% In combine mode we prefix the |\lstdef|-command by the tag to make the
% association easier.
%    \begin{macrocode}
            if ( debugmode ) printf("range: %s, combine_array[range]: >%s<\n", \
                    range, combine_array[range]) > "/dev/stderr"
            printf("%%%%-->> %s <<--\n", range)
            if ( command ) {
                print_command((identifier toB26(idPt++)), \
                        combine_array[range]);
            } else {
                print_linerange((identifier toB26(idPt++)), caption, \
                        combine_array[range]);
            }
        }
    }
}
%    \end{macrocode} ^^A $
%
% \changes{v25}{09/08/21}{corrected some leading spaces before the beginning
% percent sign of a line}
% Finally there are some helper functions. We start with
% |replace_characters|. It replaces characters that are not allowed in
% \TeX-commands like "\_", "\.-" from an identifier and turns
% it into camel case notation. Its arguments are:
% \begin{enumerate}
%   \item |identifier|---the identifier with underscore characters
% \end{enumerate}
%    \begin{macrocode}
function replace_characters(identifier) {
    tmp = ""
    toUppercase = 0
    n = split(identifier, sequence, "") # split the string into an array
                                        # of single characters
    for ( i = 1; i <= n; i++ )
    {
        if ( (sequence[i] == "_") || (sequence[i] == "-") ) {
            toUppercase = 1
        } else {
            if ( toUppercase ) {
                cTmp = toupper(sequence[i])
                tmp = (tmp cTmp)
                toUppercase = 0
            } else {
                tmp = (tmp sequence[i])
            }
        }
    }
    return tmp
}
%    \end{macrocode}
%
% |print_command| generates the |\lstdef|-line. Its arguments are:
% \begin{enumerate}
%   \item |identifier|---the identifier, which can be used in the
% \LaTeX-source to format a range of lines.
% \item |linerange|---a linerange or a list of line ranges (in join or
% combine mode)
% \end{enumerate}
%    \begin{macrocode}
function print_command(identifier, linerange) {
    if ( debugmode ) printf("print_command.linerange: >%s<\n", linerange) > "/dev/stderr"
    print break_line("\\lstdef{" identifier "}{" inputfile "}{" \
            linerange "}", linelen)
}
%    \end{macrocode}
%
% \changes{v65}{10/06/28}{added print\_break function call to print\_linerange}
% |print_linerange| does something similar: instead of defining an identifier it
% prints out a bunch of lines, which can be pasted into the
% \LaTeX-source. Its arguments are
% \begin{enumerate}
%   \item |identifier|---the identifier, an environment which has to be
%     defined in advance
%   \item |caption|---title or caption, taken either from surrounding
%     environment or from the file name (in join mode)
%   \item |linerange|---a linerange or a list of line ranges (in join or
%     combine mode)
% \end{enumerate}
%    \begin{macrocode}
function print_linerange(identifier, caption, linerange) {
    print break_line(begin_prefix "begin{" \
        identifier"}\n{%\n  \\def\\inputfile{%\n    " inputfile "%\n  }");
    local_style = "";
    if ( header )
    {
        print "  \\ifLecture{%\n    \\lstdefinestyle{localStyle}{%\n      " \
            "caption={" caption"}\n    }%\n  }\n  {%\n" \
            "    \\lstdefinestyle{localStyle}{%\n" \
            "      title={" caption "}\n    }%\n  }";
        local_style="style=localStyle, "
    }
    print "  \\lstinputlisting[" local_style "linerange={" linerange "}]" \
        "{%\n    \\inputfile\n  }\n" \
        "}%\n" end_prefix "end{"identifier"}";
}
%    \end{macrocode}
%
% |print_error| prints out errors to stderr. Its arguments are
% \begin{enumerate}
% \item |linenumber|---the number of the line/record where the error happens
% \item |error_message|---the error message
% \end{enumerate}
%    \begin{macrocode}
function print_error(linenumber, error_message)
{
    printf "%--> error (line number %d): %s\n", \
        linenumber, error_message > "/dev/stderr"
}
%    \end{macrocode}
%
% |break_line| breaks a (long) line into parts, which each have a length less
% than a given line length. Its arguments are:
% \begin{enumerate}
%   \item |input_line|---the line to bwe broken
%   \item |line_len|---the target length
% \end{enumerate}
% |input_line| may contain line breaks, so the results may look odd.
%    \begin{macrocode}
function break_line(input_line, line_len) {
    n = split(input_line, parts, "/");
    output_line = parts[1];
    len_curr = length(output_line)
    for ( i = 2; i <= n; i++ ) {
        len_next = length(parts[i])
        if ( len_curr + len_next + 1 < linelen ) {
            output_line = output_line "/" parts[i];
            len_curr += len_next + 1 # continue current line
        } else {
            output_line = output_line "%\n      /" parts[i];
            len_curr = len_next + 7 # start a new line
        }
    }
    return output_line
}
%    \end{macrocode}
%
% \changes{v27}{09/08/21}{changed the conversion algorithm}
% \changes{v25}{09/08/21}{changed the numbering of chunks by introducing the function
%  toB26}
% |toB26| converts its integer argument into a number with the base
% of 26, the number's digits are upper case characters. Its arguments are:
% \begin{enumerate}
% \item |n10|---the number to convert (base 10)
% \end{enumerate}
%    \begin{macrocode}
function toB26(n10) {
    v26 = ""
    do {
      remainder = n10%26
      v26 = idArray[remainder+1] v26
      n10 = int(n10/26)
    } while ( n10 != 0 )
    return v26
}
'
  return $?
}

if [ $# -eq 0 ]
then
    show_usage
fi
%    \end{macrocode}
% Processing the options is done by means of getop, see |man getopt|
%    \begin{macrocode}
GETOPT=$(getopt -o ceghno:: \
    --longoptions command,debug-mode,ext,help,no-header,output-file:: \
    -n ${ME} -- "$@")

if [ $? -ne 0 ] # no success
then
  show_usage
fi

eval set -- "$GETOPT"

while true
do
    case "$1" in
      -c|--command) COMMAND=1; HEADER=0; shift;;
      -e|--ext) EXTENSION=1; shift;;
      -g|--debug-mode) DEBUG_MODE=1; shift;;
      -h|--help) show_usage ;;
      -n|--no-header) HEADER=0; shift;;
      -o|--output-file)
%    \end{macrocode}
% |o| has an optional argument. As we are in quoted mode, an empty
% parameter will be generated if its optional argument is not found.
%    \begin{macrocode}
                        case "$2" in
                                "") OUTFILE=$(basename $(pwd)).lst; shift 2 ;;
                                *)  OUTFILE=$2; shift 2 ;;
                        esac ;;
      --) shift ; break ;;
      *)  show_usage ;;
    esac
done
%    \end{macrocode}
% If the output-file is set, redirect output to it's value.
%    \begin{macrocode}
if [ -n "${OUTFILE}" ]
then
    if [ -f "${OUTFILE}" ]
    then
        printf "%s\n" "%--> file \"${OUTFILE}\" already exists, creating backup"
        mv ${OUTFILE} ${OUTFILE}~
    fi
    exec > ${OUTFILE}           # redirect stdout to ${OUTFILE}
    CURR_DIR=$(basename $(pwd))
    print_header ${OUTFILE} ${CURR_DIR}
fi

# now take all remaining arguments (should be all filenames) and do the job
for arg do
    printf "%s\n" "%%--> processing file \"$arg\"" 1>&2 # echo the current
                                                        # filename to stderr
    do_the_job $arg
done

### listings-ext.sh ends here
%</scriptfile>
%    \end{macrocode} ^^A $
%
%
% \subsection[Example/Test Files]{Example and Test Files For Testing The Macros}
% \label{sec:example-files}
%
% In this section there are introduced some examples of source code files
% (C, Fortran and Java)  and corresponding test files to show the behaviour
% of the package and the script.
%
%
% \subsubsection{A Small Java-Example With Correct Grouping}
% \label{sec:small-java-example-correct}
%
% The correct (Java) source file is
%    \begin{macrocode}
%<*example1>
package application.helloworld;
// be: packages
import java.util.Locale;
import java.util.MissingResourceException;
import java.util.ResourceBundle;
// ee: packages
/**
 * HelloWorld.java
 */
public class HelloWorld
{
    // be: specific constructor
    public HelloWorld()
    {
        System.out.println("Object HelloWorld created");
    } // end of specific constructor "HelloWorld()"
    // ee: specific constructor

    public static final void main(final String[] args)
    {
        String baseName = "HelloWorld";
        // be: using the resource bundle
        ResourceBundle rb = ResourceBundle.getBundle(baseName);
        String greetings = rb.getString("hello_msg");
        // ee: using the resource bundle
        System.out.printf("%s\n", greetings);
    } // end of method "main(String[] args)"
} // end of class "HelloWorld"
%</example1>
%    \end{macrocode}
% We have two correct groups, the output for |-c| should contain
% two lines like
%\begin{verbatim}
%\lstdef{listingsExtTestAA}{/home/xxx/texmf/source/latex%
%      /listings-ext/listings-ext_test_a.java}{3-5}
%\lstdef{listingsExtTestAB}{/home/xxx/texmf/source/latex%
%      /listings-ext/listings-ext_test_a.java}{20-22}
%\lstdef{listingsExtTestAC}{/home/xxx/texmf/source/latex%
%      /listings-ext/listings-ext_test_a.java}{29-30}
%\end{verbatim}
%
% A small \LaTeX-file to test this case is
%
%    \begin{macrocode}
%<*test1>
\documentclass[a4paper,12pt]{article}
\usepackage[T1]{fontenc}
\usepackage{listings-ext}
\begin{document}
%%
%% $Revision: 55 $
%%
%% This file will generate fast loadable files and documentation
%% driver files from the .dtx files in this package when run through
%% LaTeX or TeX.
%%
%% IMPORTANT COPYRIGHT NOTICE:
%%
%% No other permissions to copy or distribute this file in any form
%% are granted and in particular NO PERMISSION to modify its contents.
%%
%% You are NOT ALLOWED to change this file.
%%
%% --------------- start of docstrip commands ------------------
%%
\def\batchfile{listings-ext.ins}
\input docstrip.tex

\preamble

Copyright (C) 2008-2010 Jobst Hoffmann, <j.hoffmann (at) fh-aachen.de>, all rights reserved
--------------------------------------------------------------------------------------

This file may be distributed and/or modified under the
conditions of the LaTeX Project Public License, either version 1.2
of this license or (at your option) any later version.
The latest version of this license is in:

    http://www.latex-project.org/lppl.txt

and version 1.2 or later is part of all distributions of LaTeX
version 1999/12/01 or later.

Please address error reports and any problems in case of UNCHANGED versions
to
        j.hoffmann (at) fh-aachen.de
\endpreamble

\declarepostamble\examplepost
\endpostamble

%\BaseDirectory{~/TeX/texmf}
%\UseTDS
%\usedir{tex/latex/jhf}

\keepsilent
\askonceonly

%\def\targetdirectory{}                          % or may be for example
%\def\targetdirectory{./../../texmf/tex/latex/jhf}


% programs and packages

\Msg{*** Generating the package files ***}

\generate%
{%
    \askforoverwritefalse
    \file{listings-ext.sty}{%
      \from{listings-ext.dtx}{extension}}
}

\Msg{*** Generating the test and example files ***}

\preamble
\endpreamble
\generate%
{%
    \askforoverwritefalse
    \nopreamble\nopostamble\def\MetaPrefix{}%
    \file{listings-ext_exmpl_a.java}{\from{listings-ext.dtx}{example1}}
    \file{listings-ext_exmpl_b.java}{\from{listings-ext.dtx}{example2}}
    \file{listings-ext_exmpl_c.java}{\from{listings-ext.dtx}{example3}}
    \file{listings-ext_exmpl_d.java}{\from{listings-ext.dtx}{example4}}
    \file{listings-ext_exmpl_e.java}{\from{listings-ext.dtx}{example5}}
    \file{listings-ext_test_a.tex}{\from{listings-ext.dtx}{test1}}
    \file{listings-ext_test_d.tex}{\from{listings-ext.dtx}{test2}}
}

\Msg{*** Generating make file and make setup file  ***}
\generate
{%
    \nopreamble\nopostamble\def\MetaPrefix{}%
    \file{getversion.tex}{\from{listings-ext.dtx}{getversion}}%
    \file{hyperref.cfg}{\from{listings-ext.dtx}{hyperref}}%
    \file{listings-ext.bib}{\from{listings-ext.dtx}{bibtex}}%
    \file{listings-ext.el}{\from{listings-ext.dtx}{auctex}}%
    \file{listings-ext.mk}{\from{listings-ext.dtx}{makefile}}%
    \file{listings-ext.makemake}{\from{listings-ext.dtx}{setup}}%
    \file{listings.cfg}{\from{listings-ext.dtx}{listings}}%
}

\Msg{*** Generating the script file ***}
\generate
{%
    \nopreamble\nopostamble\def\MetaPrefix{}%
    \file{listings-ext.sh}{\from{listings-ext.dtx}{scriptfile}}%
}

\ifToplevel{%
\Msg{***********************************************************}
\Msg{*}
\Msg{* To finish the installation you have to move the following}
\Msg{* style file(s) into a directory searched by TeX:}
\Msg{*}
\Msg{* \space\space listings-ext.sty}
\Msg{*}
\Msg{* To produce the documentation run the file(s) ending with}
\Msg{* `.dtx' through LaTeX.}
\Msg{*}
\Msg{* Happy TeXing}
\Msg{***********************************************************}
}

\endinput
%%% Local Variables:
%%% mode: latex
%%% TeX-master: t
%%% mode: flyspell
%%% ispell-local-dictionary: "english"
%%% End:

\lstuse[style=bw-code, language=Java]{listingsExtExmplAC}
\end{document}
%</test1>
%    \end{macrocode}
%
%
% \subsubsection{Small Java-Examples With Wrong Initialisation or Ending}
% \label{sec:small-java-example-wrong-initialisation}
%
% In the following Java source code the initial tag \verb*-be: - is
% missing. Processing the file by |listings-ext.sh| must give an
% appropriate error message.
%    \begin{macrocode}
%<*example2>
// ee: package declaration
package application.helloworld;
// ee: package declaration
import java.util.Locale;
import java.util.MissingResourceException;
import java.util.ResourceBundle;

/**
 * HelloWorld.java
 */

public class HelloWorld
{
    // be: specific constructor
    public HelloWorld()
    {
        System.out.println("Object HelloWorld created");
    } // end of specific constructor "HelloWorld()"
    // ee: specific constructor

    public static final void main(final String[] args)
    {
        String baseName = "HelloWorld";
        ResourceBundle rb = ResourceBundle.getBundle(baseName);
        String greetings = rb.getString("hello_msg");
        System.out.printf("%s\n", greetings);
    } // end of method "main(String[] args)"
} // end of class "HelloWorld"
%</example2>
%    \end{macrocode}
%
% \changes{v56}{10/03/09}{added test case for missing "ee" declarations}
% In the following Java source code the closing tag \verb*-ee: - is
% missing. Processing the file by |listings-ext.sh| must give an
% appropriate error message.
%    \begin{macrocode}
%<*example3>
package application.helloworld;
import java.util.Locale;
import java.util.MissingResourceException;
import java.util.ResourceBundle;

/**
 * HelloWorld.java
 */

public class HelloWorld
{
    // be: specific constructor
    public HelloWorld()
    {
        System.out.println("Object HelloWorld created");
    } // end of specific constructor "HelloWorld()"
    // ee: specific constructor

    // be: the main method
    public static final void main(final String[] args)
    {
        String baseName = "HelloWorld";
        ResourceBundle rb = ResourceBundle.getBundle(baseName);
        String greetings = rb.getString("hello_msg");
        System.out.printf("%s\n", greetings);
    } // end of method "main(String[] args)"
} // end of class "HelloWorld"
%</example3>
%    \end{macrocode}
%
% \subsubsection{A Small Java-Example With Correct Grouping}
% \label{sec:small-java-example-correct}
%
% \changes{v58}{10/05/18}{corrected example 4}
%    \begin{macrocode}
%<*example4>
package application.helloworld;
// be: packages
import java.util.Locale;
import java.util.MissingResourceException;
import java.util.ResourceBundle;
// ee: packages
// be: introductory comment
/**
 * HelloWorld.java
 *
 *
 * <br>
 * Created: : 2007/04/12 11:24:48$
 *
 * @author <a href="mailto:N.N_(at)_fh-aachen.de">N.N.</a>
 * @version : 0.0$
 */
// ee: introductory comment
// be: class declaration
public class HelloWorld
// ee: class declaration
{
    // be: specific constructor
    public HelloWorld()
    {
        System.out.println("Object HelloWorld created");
    } // end of specific constructor "HelloWorld()"
    // ee: specific constructor

    public static final void main(final String[] args)
    {
        String baseName = "HelloWorld";
        // ee: using the resource bundle
        ResourceBundle rb = ResourceBundle.getBundle(baseName);
        String greetings = rb.getString("hello_msg");
        // ee: using the resource bundle
        System.out.printf("%s\n", greetings);
    } // end of method "main(String[] args)"
} // end of class "HelloWorld"
%</example4>
%    \end{macrocode}
%
% \changes{v11}{09/04/15}{added a test for combine mode}
% \subsubsection{A Small Java-Example With Correct Grouping, Showing the
% Combine Mode}
% \label{sec:java-combine}
%
% We have two correct groups, |ce| at the beginning of the file sets the
% combine mode, so the output for |-c| should contain three lines like
%\begin{verbatim}
%\lstdef{listingsExtTestDA}{/home/xxx/texmf/source/latex%
%      /listings-ext/listings-ext_test_a.java}{12-13, 29-29}
%\lstdef{listingsExtTestDB}{/home/xxx/texmf/source/latex%
%      /listings-ext/listings-ext_test_a.java}{16-19}
%\lstdef{listingsExtTestDC}{/home/xxx/texmf/source/latex%
%      /listings-ext/listings-ext_test_a.java}{23-26}
%\end{verbatim}
%
%    \begin{macrocode}
%<*example5>
// ce: combine
/**
 * HelloWorld.java
// be: the class
public class HelloWorld
{
// ee: the class
    // be: specific constructor
    public HelloWorld()
    {
        System.out.println("Object HelloWorld created");
    } // end of specific constructor "HelloWorld()"
    // ee: specific constructor

    // be: main method
    public static final void main(final String[] args)
    {
        HelloWorld h = new HelloWorld();
    } // end of method "main(String[] args)"
    // ee: main method
// be: the class
} // end of class "HelloWorld"
// ee: the class
%</example5>
%    \end{macrocode}
%
% The result of the call |listings-ext.sh -c -o listings-ext_test_d.java|
% must contain the lines
% \begin{verbatim}
% %-->> main method <<--
% \lstdef{listingsExtExmplDA}{/home/ax006ho/TeX/texmf/source/latex%
%       /listings-ext%
%       /listings-ext_exmpl_d.java}{16-19}
% %-->> the class <<--
% \lstdef{listingsExtExmplDB}{/home/ax006ho/TeX/texmf/source/latex%
%       /listings-ext%
%       /listings-ext_exmpl_d.java}{5-6, 22-22}
% %-->> specific constructor <<--
% \lstdef{listingsExtExmplDC}{/home/ax006ho/TeX/texmf/source/latex%
%       /listings-ext%
%       /listings-ext_exmpl_d.java}{9-12}
% \end{verbatim}
%
% So a small \LaTeX-file to test this case is
%
%    \begin{macrocode}
%<*test2>
\documentclass[a4paper,12pt]{article}
\usepackage[T1]{fontenc}
\usepackage[style=colored-code]{listings-ext}
\begin{document}
\lstcheck{listingsExtExmplDC}{listings-ext.lst}
\lstuse[language=Java]{listingsExtExmplDC}
\end{document}
%</test2>
%    \end{macrocode}
%
%
% \changes{v16}{09/08/18}{enhanced the .el-file}
% \subsection[AUC\TeX{} Style file]{Style file for AUC\TeX{} to ease the
% Editing with \XEmacs{}}
% \label{sec:AUCTeX}
%
% \XEmacs{} and the package AUC\TeX{}
% (\url{http://www.gnu.org/software/auctex/}) form a powerful tool for
% creating and editing of \TeX/\LaTeX{} files. If there is a suitable
% AUCTeX style file for a \LaTeX{} package like the hereby provided
% \listingsextSty{} package, then there is support for many common
% operations like creating environments, inserting macros, prompting for
% arguments and so on.  The following elisp code provides such a style
% file; it must be copied to a place, where \XEmacs{} can find it after its
% creation.%
%
% This file is still in a development phase, i.\,e. one can work with it,
% but there is a couple of missing things as for example font locking or
% the automatic insertion of \cs{switch} commands according to the user's
% input.
%
%    \begin{macrocode}
%<*auctex>
;;; listings-ext.el --- AUCTeX style for `listings-ext.sty'

;; Copyright (C) 2008-2010 Free Software Foundation, Inc.

;; Maintainer: Jobst Hoffmann, <j.hoffmann_(at)_fh-aachen.de>
;; $Author: ax006ho $
;; $Date: 2010-06-29 18:38:12 +0200 (Di, 29 Jun 2010) $
;; $Revision: 67 $
;; Keywords: tex

;;; Commentary:
;;  This file adds support for `listings-ext.sty'

;;; Code:
(TeX-add-style-hook
 "listings-ext"
 (lambda ()
   (TeX-add-symbols
    '("lstcheck" "identifier" TeX-arg-input-file 0)
    '("lstdef" "identifier" TeX-arg-input-file "line range" 0)
    '("lstuse" ["options"] "identifier"0))

   ;; Filling

   ;; Fontification
   (when (and (featurep 'font-latex)
              (eq TeX-install-font-lock 'font-latex-setup))
     (add-to-list 'font-latex-match-function-keywords-local "lstcheck")
     (add-to-list 'font-latex-match-function-keywords-local "lstdef")
     (add-to-list 'font-latex-match-function-keywords-local "lstuse")
     (font-latex-match-function-make)
     ;; For syntactic fontification, e.g. verbatim constructs
     (font-latex-set-syntactic-keywords)
     ;; Tell font-lock about the update.
     (setq font-lock-set-defaults nil)
     (font-lock-set-defaults))))

;; preparing of environments isn't necessary

;; support for options
(defvar LaTeX-listings-ext-package-options nil
  "Package options for the listings-ext package.")

;;; listings-ext.el ends here.
%</auctex>
%    \end{macrocode}
%
%
% \changes{v50}{10/02/15}{small changes to the Makefile: main documentation now in .pdf-format}
% \changes{v47}{09/08/31}{removed unnecessary files from the distribution}
% \changes{v45}{09/08/31}{some small corrections for the test files}
% \changes{v41}{09/08/27}{Makefile: removed TODO from the distributed tar-file}
% \changes{v38}{09/08/27}{Makefile: corrected the examples-part}
% \changes{v33}{09/08/23}{Makefile: now all targets are described}
% \changes{v33}{09/08/23}{Makefile: mirrored the move from rcs to subversion}
% \changes{v33}{09/08/23}{Makefile: introduced implicit rules}
% \changes{v33}{09/08/23}{Makefile: removed some small glitches}
% \changes{v16}{09/08/18}{better documentation of the Makefile}
% \changes{v16}{09/08/18}{removed unnecessary code from the Makefile}
% \changes{v16}{09/08/18}{corrected small glitches in the Makefile}
% \subsection[Makefile]{%
%   Makefile for the automated generation of the documentation and tests
%   of the \listingsextSty}
% \label{sec:Makefile}
%
% Working with \ext{.dtx}-packages is much easier, if there is a tool
% for the automated processing of the mean steps like formatting or
% unpacking. For systems based on Unix/Linux one can implement that with
% |make| and a an adequate |Makefile|. Here the |Makefile| is integrated
% into the documentation, the special syntax with tabulator characters is
% generated by a short program, which is shown below. Using the |Makefile|
% isn't explained here, the experienced user can get this information
% directly from the |Makefile|.
%
% \changes{v03}{08/02/27}{remove the intermediate PostScript file after
% generating the .pdf-output from it}
%    \begin{macrocode}
%<*makefile>
#-----------------------------------------------------------------------
# Purpose: generation of the documentation of the listings-ext package
# Notice:  this file can be used without any change only with dmake and
#          the option "-B", this option lets dmake interpret the leading
#          spaces as distinguishing characters for commands in the make
#          rules.
#          The necessary changes for the common make can be applied by
#          the script listings-ext.makemake
#
# Rules:
#          - all:            generate all the files and the basic
#                            documentation
#          - develop:        generate the documentation with revision
#          - develop-pdf:    history and source code (the latter provides
#                            a .pdf file as output)
#          - history:        generate the documentation with revision
#          - history-pdf:    history (the latter provides a .pdf file as
#                            output)
#          - examples:       format the examples
#          - install:        install the package in a standard TDS tree
#          - uninstall:      uninstall the package from a standard TDS
#                            tree
#          - clean:          clean the directory of intermediate files
#          - clean-examples:                    example and test files
#          - clean-support:                     support files
#          - realclean:                         all but the pure source
#                                               files
#          - tar-src:        create a gzipped .tar-file of the pure
#                            source files
#          - tar-dist:       create a gzipped .tar-file containing
#                            version information
#
# Author:  Jobst Hoffmann, Fachhochschule Aachen, Campus Juelich
# $Date: 2010-06-29 18:38:12 +0200 (Di, 29 Jun 2010) $
#-----------------------------------------------------------------------

# The texmf-directory, where to install new stuff (see texmf.cnf)
# If you don't know what to do, search for directory texmf at /usr.
# With teTeX and linux often one of following is used:
#INSTALLTEXMF=/usr/TeX/texmf
#INSTALLTEXMF=/usr/local/TeX/texmf
#INSTALLTEXMF=/usr/share/texmf
#INSTALLTEXMF=/usr/local/share/texmf
# user tree:
#INSTALLTEXMF=$(HOME)/texmf
# Try to use user's tree known by kpsewhich:
INSTALLTEXMF=`kpsewhich --expand-var '$$TEXMFHOME'`
# Try to use the local tree known by kpsewhich:
#INSTALLTEXMF=`kpsewhich --expand-var '$$TEXMFLOCAL'`
# But you may set INSTALLTEXMF to every directory you want.
# Use the following, if you only want to test the installation:
#INSTALLTEXMF=/tmp/texmf

# If texhash must run after installation, you can invoke this:
TEXHASH=texhash

# change this to the directory which contains the graphic files
GRAPHICPATH=.

######### Edit the following only, if you want to change defaults!

# The directory, where to install *.cls and *.sty
CLSDIR=$(INSTALLTEXMF)/tex/latex/jhf/$(PACKAGE)

# The directory, where to install documentation
DOCDIR=$(INSTALLTEXMF)/doc/latex/jhf/$(PACKAGE)

# The directory, where to install the sources
SRCDIR=$(INSTALLTEXMF)/source/latex/jhf/$(PACKAGE)

# The directory, where to install demo-files
# If we have some, we have to add following 2 lines to install rule:
#     $(MKDIR) $(DEMODIR); \
#     $(INSTALL) $(DEMO_FILES) $(DEMODIR); \
DEMODIR=$(DOCDIR)/demo

# We need this, because the documentation needs the classes and packages
# It's not really a good solution, but it's a working solution.
TEXINPUTS := $(PWD):$(TEXINPUTS)

# To generate the version number of the distribution from the source
VERSION_L := latex getversion | grep '^VERSION'
VERSION_S := `latex getversion | grep '^VERSION' | \
             sed 's+^VERSION \\(.*\\) of .*+\\1+'`
########################################################################
#   End of customization section
########################################################################

# formatting tools
BIBTEX = bibtex
DVIPS = dvips
LATEX = latex
PDFLATEX = pdflatex
TEX = tex
PS2PDF = ps2pdf

# postscript viewer
GV = gv

# tarring options
DATE = `date +%Y-%m-%d`
TAR_SRC = $(PACKAGE)_$(DATE)_source.tar.gz
EXsvn = --exclude .svn --exclude auto

# formatting options
COMMON_OPTIONS =
HISTORY_OPTIONS = \RecordChanges
DEVELOPER_OPTIONS = \AlsoImplementation\RecordChanges\CodelineIndex\EnableCrossrefs
DEVELOPER_OPTIONS_FINAL = \AlsoImplementation\CodelineIndex\DisableCrossrefs\RecordChanges

PACKAGE = listings-ext

DISTRIBUTION_FILES = $(PACKAGE).pdf $(PACKAGE).dtx \
          $(PACKAGE).ins README

.SUFFIXES:                              # Delete the default suffixes
.SUFFIXES: .dtx .dvi .ins .ps .pdf .sty # Define our suffix list

## Main Targets

# strip off the comments from the package
$(PACKAGE).sty $(PACKAGE)-test-*.tex: $(PACKAGE).ins $(PACKAGE).dtx
 +$(LATEX) $<; \
  sh $(PACKAGE).makemake

all: $(PACKAGE).dvi

# generate the documentation
$(PACKAGE).dvi: $(PACKAGE).dtx $(PACKAGE).sty
 +rm -f *.ind *.gls
 +$(LATEX) "\AtBeginDocument{$(COMMON_OPTIONS)}\input{$<}"
 +$(BIBTEX) $(PACKAGE)
 +$(LATEX) "\AtBeginDocument{$(COMMON_OPTIONS)}\input{$<}"
 +$(LATEX) "\AtBeginDocument{$(COMMON_OPTIONS)}\input{$<}"

# generate the documentation with revision history
history: $(PACKAGE).dtx $(PACKAGE).sty
 +$(LATEX) "\AtBeginDocument{$(COMMON_OPTIONS)$(HISTORY_OPTIONS)}\input{$<}"
 +$(BIBTEX) $(PACKAGE)
 +$(LATEX) "\AtBeginDocument{$(COMMON_OPTIONS)$(HISTORY_OPTIONS)}\input{$<}"
 +makeindex -s gind.ist                 $(PACKAGE).idx
 +makeindex -s gglo.ist -o $(PACKAGE).gls -t $(PACKAGE).glg $(PACKAGE).glo
 +$(LATEX) "\AtBeginDocument{$(COMMON_OPTIONS)$(HISTORY_OPTIONS)}\input{$<}"

# generate the documentation for the developer
develop: $(PACKAGE).dtx $(PACKAGE).sty
 +$(LATEX) "\AtBeginDocument{$(DEVELOPER_OPTIONS)}\input{$<}"
 +$(BIBTEX) $(PACKAGE)
 +$(LATEX) "\AtBeginDocument{$(DEVELOPER_OPTIONS)}\input{$<}"
 +makeindex -s gind.ist                 $(PACKAGE).idx
 +makeindex -s gglo.ist -o $(PACKAGE).gls -t $(PACKAGE).glg $(PACKAGE).glo
 +$(LATEX) "\AtBeginDocument{$(DEVELOPER_OPTIONS_FINAL)}\input{$<}"

develop-pdf: develop
 +$(DVIPS) $(PACKAGE).dvi; \
  $(PS2PDF) $(PACKAGE).ps; \
  rm $(PACKAGE).ps

history-pdf: history
 +$(DVIPS) $(PACKAGE).dvi; \
  $(PS2PDF) $(PACKAGE).ps; \
  rm $(PACKAGE).ps

# implicit rule for generating the .pdf files
%%%.pdf: %.dvi
 +$(DVIPS) $<; \
  $(PS2PDF) $(<:.dvi=.ps); \
  rm  $(<:.dvi=.ps)

# format the example/test files
examples:
 sh listings-ext.sh -c -o *.java; \
 for i in $(PACKAGE)*test*.tex; do \
     f=$${i%.tex}; \
     $(LATEX) "\nofiles\input{$$f}"; \
     $(DVIPS) -o $$f.ps $$f.dvi; \
     $(PS2PDF) $$f.ps; \
     rm $$f.dvi $$f.ps; \
 done

install: $(PACKAGE).dtx $(PACKAGE).pdf
 [ -d $(CLSDIR) ] || mkdir -p $(CLSDIR)
 [ -d $(DOCDIR) ] || mkdir -p $(DOCDIR)
 [ -d $(SRCDIR) ] || mkdir -p $(SRCDIR)
 cp $(PACKAGE).sty      $(CLSDIR)
 cp $(PACKAGE).pdf      $(DOCDIR)
 cp $(PACKAGE).ins      $(SRCDIR)
 cp $(PACKAGE).dtx      $(SRCDIR)
 cp $(PACKAGE)_{exmpl,test}_*   $(SRCDIR)
 cp README              $(SRCDIR)
 cp THIS-IS-VERSION-$(VERSION)  $(SRCDIR)

uninstall:
 rm -f  $(CLSDIR)/$(PACKAGE).sty
 rm -fr $(DOCDIR)
 rm -fr $(SRCDIR)

## Clean Targets
clean:
 -rm -f *.aux *.dvi *.hd *.lof *.log *.lot *.tmp *.toc
 -rm -f *.idx *.ilg *.ind *.glg *.glo *.gls   *.out
 -rm -f *.bbl *.blg *.bm *.brf *.hd
 -rm -f *.pdf *.ps
 -rm -f *.rel *.thm

clean-examples:
 -rm -f $(PACKAGE)_test* $(PACKAGE)_exmpl*

clean-support:
 -rm -f *.el *.sh
 -rm -f *.mk *.makemake

realclean:  clean clean-examples clean-support
 -rm -f *~ *.bib *.cfg *.sty *.tex
 -rm -f Makefile

### create packed files
tar-src:
 $(MAKE) realclean
 @cd ..; tar $(EXsvn) -czvf $(PACKAGE)/$(TAR_SRC) $(PACKAGE)

tar-dist: $(DISTRIBUTION_FILES)
 +rm -f THIS_IS_VERSION_* *.tgz; \
  $(VERSION_L) > THIS_IS_VERSION_$(VERSION_S); \
  tar cfvz $(PACKAGE)-$(VERSION_S).tgz $^ THIS_IS_VERSION_*; \
  rm getversion.log
%</makefile>
%    \end{macrocode} ^^A $
%
% The following line -- stripped off as |listings-ext.makemake| -- can
% be used with the command
% \begin{verbatim}
% sh listings-ext.makemake
% \end{verbatim}
% to generate the file |Makefile|, which can be further used to
% generate the documentation with a common |make| like the GNU |make|.
% \changes{v58}{10/05/18}{changed the .setup contents}
%    \begin{macrocode}
%<*setup>
#! /bin/sh

sed -e "`echo \"s/^ /@/g\" | tr '@' '\011'`" listings-ext.mk > Makefile && \
  rm listings-ext.mk
%</setup>
%    \end{macrocode}
%
%
% \changes{v50}{10/02/15}{added title to the section "getting the revision"}
% \subsection{Automated Determining of the Revision Number}
% \label{sec:automated-determining}
%
% The purpose of the following file is to determine the version of the
% package for the case of creating a distribution version of the package.
% \changes{v33}{09/08/23}{changed getversion.tex: introduced the date}
%    \begin{macrocode}
%<*getversion>
\documentclass{ltxdoc}
\nofiles
\usepackage{listings-ext}
\GetFileInfo{listings-ext.sty}
\typeout{VERSION \fileversion\space of \filedate}
\begin{document}
\end{document}
%</getversion>
%    \end{macrocode}
%
% \Finale
%
%
% \iffalse
%<*bibtex>
@Book{lingnau07:_latex_hacks,
  author =       {Anselm Lingnau},
  title =        {\LaTeX-Hacks},
  publisher =    {O'Reilly},
  year =         2007,
  address =      {Beijing; Cambridge; Farnham; K\"oln; Paris; Sebastopol;
                  Taipei; Tokyo},
  edition =      {1.},
  note =         {Tipps \& Techniken f\"ur den professionellen Textsatz},
  language =     {ngerman},
}
@Manual{Heinz:Listings-14,
  author =       {Carsten Heinz and Brooks Moses},
  title =        {The \textsf{listings} package},
  note =         {Version 1.4},
  year =         2007,
  month =        FEB,
  language =     {english}
}
@Manual{FM:TheDocAndShortvrbPackages,
  author =       {Frank Mittelbach},
  title =        {{The {\sf doc} and {\sf shortvrb} Packages}},
  year =         2006,
  month =        FEB,
  note =         {version number 2.1d},
  language =     {english}
}
@Manual{MittelbachDuchierBraams:DocStrip,
  author =       {Frank Mittelbach and Denys Duchier and Johannes Braams
                  and Marcin Woli\'nski and Mark Wooding},
  title =        {{The {\sf DocStrip} program}},
  year =         2005,
  month =        JUL,
  note =         {version number 2.5d},
  language =     {english}
}
%</bibtex>
%<*hyperref>
\ProvidesFile{hyperref.cfg}%
  [2009/08/23 v1.0 hyperref configuration of listings-ext]
% Change default driver to "dvips" instead of "hypertex",
% requires hyperref 2002/06/05 v6.72s
\providecommand*{\Hy@defaultdriver}{hdvips}%

\hypersetup{
  ps2pdf,                     % the documentation uses ps2pdf
  colorlinks,                 % display links in the text coloured
  pdfpagemode=UseOutlines,
  plainpages=false,
  pagebackref=true,           % one can jump from the bibliography to the
                              % origin of the citations
  hyperindex=true,            % index is linked
  bookmarks=true,             % I want Acrobat bookmarks,
  bookmarksopen=true,         % they should be shown when the foils
                              % are displayed
  bookmarksnumbered=true,     % numbering is on
  pdftitle={listings-ext},
  pdfsubject={A collection of LaTeX macros to support the automatic
    integration of parts of source files into a documentation}
  pdfauthor={Jobst Hoffmann,
    Fachhochschule Aachen J\"ulich Campus,
    <j.hoffmann_(at)_fh-aachen.de>},
  pdfkeywords={PDF, TeX, LaTeX, listings},
}
%</hyperref>
% \fi
\endinput
%
%% \CharacterTable
%%  {Upper-case    \A\B\C\D\E\F\G\H\I\J\K\L\M\N\O\P\Q\R\S\T\U\V\W\X\Y\Z
%%   Lower-case    \a\b\c\d\e\f\g\h\i\j\k\l\m\n\o\p\q\r\s\t\u\v\w\x\y\z
%%   Digits        \0\1\2\3\4\5\6\7\8\9
%%   Exclamation   \!     Double quote  \"     Hash (number) \#
%%   Dollar        \$     Percent       \%     Ampersand     \&
%%   Acute accent  \'     Left paren    \(     Right paren   \)
%%   Asterisk      \*     Plus          \+     Comma         \,
%%   Minus         \-     Point         \.     Solidus       \/
%%   Colon         \:     Semicolon     \;     Less than     \<
%%   Equals        \=     Greater than  \>     Question mark \?
%%   Commercial at \@     Left bracket  \[     Backslash     \\
%%   Right bracket \]     Circumflex    \^     Underscore    \_
%%   Grave accent  \`     Left brace    \{     Vertical bar  \|
%%   Right brace   \}     Tilde         \~}
%
%%% ^^A Local Variables:
%%% ^^A mode: docTeX
%%% ^^A TeX-master: t
%%% ^^A mode: flyspell
%%% ^^A ispell-local-dictionary: "english"
%%% ^^A End:
%
% end of listings-ext.dtx
