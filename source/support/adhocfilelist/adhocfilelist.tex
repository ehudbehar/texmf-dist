\ProvidesFile{adhocfilelist.tex}[2012/11/19 documenting adhocfilelist (UL)] 
\head \charset{ISO-8859-1} %%% {utf-8} 
  \texrobots
  \title{List LaTeX file infos according to command line}
%   \stylesheet{all}{plain}                     %% tried 2012/05/06
  \simplebody 
    \lineanc{top-of-page} %%% \textopofpage
\EXECUTE{\UseBlogLigs}
\EXECUTE{\BlogInterceptHash}
\EXECUTE{\def\bodytitle{\heading1} 
         % \def\section{\heading2}
%          \def\labelsection#1{\nextview{#1}\endgraf\lineanc{#1}\heading2} 
%          \def\labelsection#1{\lineanc{#1}\heading2} 
         \def\labelsection#1#2{\heading2{\lineanc{#1}#2\ancref{#1}{.}}} 
         \def\subsection{\heading3} 
         \def\fakeitem{\bullet\enspace}
%          \def\strongcs#1{\strong{\cs{#1}}} 
         \def\strongcode#1{\strong{\code{#1}}}
         \def\strongmetaopt#1#2{\strong{\code{-#1}~\metacode{#2}}}
         \def\metamandarg#1{\code\{\metacode{#1}\code\}}
         \def\metaoptarg#1{\code[\metacode{#1}\code]}
         \def\optitem#1{\ditem{\code{-#1}}}
         \def\optmetaitem#1#2{\ditem{\optmeta{#1}{#2}}}
         \def\optmeta#1#2{\code{-#1}~\metacode{#2}}
         \def\metacode#1{\code{\metavar{#1}}}
%          \let\CPR\ctanpkgref 
%          \def\pkgnamefmt{\spanstyle{font-family:monospace}}
         \let\pkgnamefmt\textsf
         \let\pkg\pkgnamefmt
         \let\file\pkgnamefmt
         \def\Find{\pkgnamefmt{\Wikienref{find}}}
         \def\find{\pkgnamefmt{find}}
         \def\ext#1{\code{.#1}}
}
% \EXECUTE{\BlogInterceptHash}        %% upfinfo.sh instead
% \EXECUTE{\RequirePackage{monofill}} %% upfinfo.sh instead
% \EXECUTE{\let\pkgnamefmt\code}
\bodytitle{\code{adhocfilelist} ---\vspace{6} Listing \LaTeX\ Source File Infos\\
           According to Command Line}

% \small[\,\UseVersionOf{finfinfo.tex}~as~of~\UseDateOf{finfinfo.tex}\,]\endsmall
%% <- TODO \relax 2012/04/09
  <p>   %%% \vspace{24}
\begin{small}
[\,\ancrefslist{{{abstract}{abstract}}
                {{over}{overview}}
                {{required}{requirements}}
                {{install}{installing}}
                {{related}{related}}
                {{back}{background}}
                {{custom}{customize}}
                {{idea}{idea/thanks}}}\,]\quad 
[\,\textit{\fileref{\htmljob}{reload}}\,]
\end{small}

\labelsection{abstract}{Abstract}
\file{adhocfilelist.sh} is a shell script with command line options 
and parameters that displays \metavar{info} for files \metavar{file} 
on screen that (i)~are indicated on the command line and 
(ii)~contain \cs{ProvidesFile}\metamandarg{file}\metaoptarg{info}, 
\cs{ProvidesPackage}\metamandarg{file}\metaoptarg{info}, 
or \cs{ProvidesClass}\metamandarg{file}\metaoptarg{info}. 
There is one line for each file, formatted about as with \LaTeX's 
\cs{listfiles}. This formatting can be varied, and the output may 
additionally be saved in a plain text file. The files \metavar{file}
are either (a)~listed explicitly, separated by commas, or 
(b)~they are gathered by the Unix \Find\ utility. 
For the latter, \pkg{adhocfilelist} provides a somewhat simplified 
interface, especially for restricting the list of files 
to those that have been modified "today" or a few days ago.
So you/I can check whether version info was updated correctly, 
and it may allow \ctanpkgref{filedate} \dqtd{consistency checks} 
even with \CtanPkgRef{xetex}{\XeTeX} (where \cs{pdffilemoddate} 
is unavailable).
  <p>
The package may thus be considered an extension of the 
\ctanpkgref{latexfileversion} (Harald \ctanpkgauref{harders}{Harders}) 
and \ctanpkgref{typeoutfileinfo} packages 
that display \metavar{info} for a \emph{single} file \metavar{file}.
Moreover, it is a somewhat simplified interface to the packages 
\ctanpkgref{myfilist}, \ctanpkgref{longnamefilelist}, 
\ctanpkgref{nicefilelist}, and \ctanpkgref{filedate}.

\labelsection{over}{Overview of command line contents}
The general structure of a command line using \pkg{adhocfilelist} is 
\begin{quote}
  \ancref{call}{\metavar{call}} 
  \ancref{opts}{\metavar{options}} 
  \ancref{files}{\metavar{files}}
\end{quote}
The three parts are as follows:
\begin{description}
  \ditem{\metavar{call}\lineanc{call}}
    This is either
    \begin{itemize}                                             %% 2012/10/17
      \item \metavar{path}\file{adhocfilelist.sh} 
            where \metavar{path} is "./" or something, 
            depending on where you place the file \file{adhocfilelist.sh}, 
            or on your use of \Wikienref{symbolic link}s, 
            the environment variable \code{\Wikiendisambref{PATH}{variable}}, 
            or
      \item an \code{\Wikiendisambref{alias}{command}} for 
            \metavar{path}\file{adhocfilelist.sh}.   %% was <path> 2012/10/17
    \end{itemize}
  \ditem{\metavar{options}\lineanc{opts}}
    \EXECUTE{\newcommand*{\fmode}{\strong{(f)}}}
    Some options "switch into \strong{\Find\ mode}", 
    affecting the interpretation of \strong{\metavar{files}}, marked by "\fmode". 
    Their description may be inaccurate here, 
    understanding the details may require some knowledge of \find\ 
    (see "\ancref{back}{background}" below).
    Some options have one parameter, some have none.
    \begin{description}
      \optitem{0} \fmode\
        List files modified "\strong{today}" only (unless ...)
      \optmetaitem{a}{integer} \fmode\
        List files modified \metavar{integer} days \strong{ago} only 
        (unless \metavar{integer} starts with "\code{-}" or "\code{+}" ...)
      \optitem{c} \fmode\ 
        Compare date according to \cs{Provides...} content with modification 
        date using \ctanpkgref{filedate} additionally.
          <p>
        \TODO: The restriction to \find\ (v0.7) is temporary, removing it is 
        straightforward, but I may have to postpone this for a while.
      \optitem{f} \fmode\
        Add \metavar{files} to other \strong{\find} criteria, 
        \strong{replacing} internal default settings 
        (as opposed to \code{-g}). 
        For allowing certain file name extensions such as \ext{tex}, 
        the star must be escaped and \metavar{files} best 
        has outer single quotes, such as
        \begin{quote}
         \EXECUTE{\MakeOther\'}
          \code{'-L ( -name \cs{*}.tex -o -name \cs{*}.sty )'}
          %% <- follow -> L 2012/11/19
        \end{quote}
      \optmetaitem{F}{find-file} \fmode\
        Add content of the \strong{file} \metavar{find-file} to other \strong{\find} criteria. 
        As compared with \code{-f}, stars must be escaped, 
        while other quoting seems not to be needed in \metavar{files}. 
      \optitem{g} \fmode\
        Add \metavar{files} to other \strong{\find} criteria, 
        among which some hopefully "\strong{generally useful}" 
        are added from within \file{adhocfilelist.sh}.
        As to quoting/escaping in \metavar{files}, 
        this is as with \code{-f}.
      \optitem{h}
        Display an overview (brief version of present one, "\strong{help}"), 
        and ignore everything else of the command line.
      \optmetaitem{i}{tex-file}
        Prepend ("\strong{input}") contents of the file \metavar{tex-file} 
        to the temporary \ext{tex} file that \file{adhocfilelist.sh}
        generates internally\pardash to get more individual settings 
        than \file{adhocfilelist.sh}'s internal defaults 
        (which are \strong{replaced} by the content of \metavar{tex-file})
        and than the options \code{-l} and \code{-n} provide.
      \optmetaitem{l}{integer}
        Format list by the \strong{\ctanpkgref{longnamefilelist}} package 
        where \metavar{integer} is the maximum length of base 
        \Wikienref{filename}s in the list (about as ...)
      \optmetaitem{n}{filename}
        Format list by the \strong{\ctanpkgref{nicefilelist}} package 
        where \metavar{filename} is the longest base \Wikienref{filename} (or a template).
        %% 2012/10/15:
        If base filenames do \emph{not} have more than eight characters, 
        you can use "\strongcode{-n~.}" as a shorthand for 
        "\optmeta{n}{filename}" with a \metavar{filename} of eight characters.
      \optmetaitem{o}{txt-file}
        Save the displayed file info list ("\strong{output}")
        as plain text file \metavar{txt-file} also.
      \optitem{x}
        Use \code{xelatex} rather than \code{latex} (v0.7).\enpardash The 
        idea for this has been that the script should provide 
        "date consistency" checks (\strong{option~\code{-c}}) 
        \endqtd{even with \CtanPkgRef{xetex}{\XeTeX} where \cs{pdffilemoddate}
        is not available.} However, the script has used \code{latex}
        before v0.7 which will run \CtanPkgRef{pdftex}{\pdfTeX} usually, 
        even when the user prefers \emph{typesetting} with \XeTeX,
        and anyway it does not rely on \cs{pdffilemoddate}. 
        The real problem with \ctanpkgref{filedate} and \XeTeX\ is that 
        a reliable consistency check does not work 
        \emph{while typesetting with \XeTeX}. But \pkg{filedate} can be 
        used for "\TeX\ scripts" too, i.e., \emph{without} typesetting, 
        and that is what the present package does. 
        The present package does \emph{not} provide consistency checks 
        \emph{while typesetting with \XeTeX}. 
        It should not be make a difference whether the "\TeX\ script"
        that \pkg{adhocfilelist} generates is run with \XeTeX\ or with \pdfTeX.
        So the only "special \XeTeX\ support" from here probably 
        will be the present \strongcode{-x} in case \code{latex} 
        \emph{does not work}, maybe because \pdfTeX\ or anything implementing
        the \code{latex} command is missing on the user's (unusual) installation.
    \end{description}
    Options can \emph{not} be "contracted" like "\code{-0f}" 
    (too difficult with processing [parts of] \find\ 
     expressions\pardash\code{-name} ...).
  \ditem{\metavar{files}\lineanc{files}}
    Whenever an option "switches into \Find\ mode", 
    \metavar{files} is interpreted as part of a command line for the 
    \find\ utility. Otherwise, \metavar{files} is expected to be a 
    comma-separated list of filenames (for \LaTeX\ source files),
    whose extension \ext{tex} (if so) may be omitted\pardash unless \code{-h}~...
\end{description}

\labelsection{required}{Requirements}
\begin{description}
  \ditem{System}
    The package will work on a \Wikienref{Unix-like} system only, 
    but I am not expert enough to judge what additional restrictions hold. 
    As I could not make work everything that should work according to 
    documentation, I expect the other way round that what works on 
    my installation may not work at other Unix-like installations. 
    If something doesn't work at your installation, please tell me via 
    \urlref{contact-ednotes.sty.de.vu}.
      <p>
    Actually the package was developed with the \Wikienref{GNU bash}
    shell and the 
    \wikienref{GNU Find Utilities}{\acro{GNU} \find\ utility} on 
    \Wikiendisambref{Ubuntu}{operating system}~10.04. 
      <p>
    My impression is in fact 
    (looking at \Wikienref{The Open Group}'s 
     \httpref{pubs.opengroup.org/onlinepubs/9699919799/utilities/find.html}{man page} 
     for \find)
    that in order to use the \strong{\find} options 
    \strongcode{-0}, \strongcode{-a}, \strongcode{-f}, \strongcode{-F},
    and \strongcode{-g}, the \strong{\acro{GNU}} version of \find\ is required, 
    because action \strong{-printf} is used and essential (difficult without). 
  \ditem{\LaTeX}
    The package requires that the \ctanpkgref{fileinfo} bundle is installed. 
    In order to use the \strong{\code{-l}} option, 
    \ctanpkgref{longnamefilelist} must be installed additionally. 
    In order to use the \strong{\code{-n}} option, 
    \ctanpkgref{nicefilelist} must be installed additionally. 
    In order to use the \strong{\code{-c}} option, 
    \ctanpkgref{filedate} must be installed additionally. 
\end{description}

\labelsection{install}{Installing}
No more advice than implied by the section on "\ancref{call}{\metavar{call}}"
can be given right now.

\labelsection{related}{Related packages}
The file 
\mirrorctanref{info/latexfileinfo-pkgs/latexfileinfo_pkgs.htm}{%
               \file{latexfileinfo_pkgs.htm}}
describes related packages available on 
% \httpref{tug.ctan.org}{\acro{CTAN}} 
\acro{\Wikiref{CTAN}}
in some detail. 
The most obvious ones have been mentioned above already. 
% It might be added that Martin \ctanpkgref{scharrer}{Scharrer}'s
% \ctanpkgref{filemod} probably could be used as an alternative to 
% \code{\Wikiref{find}} in the future\pardash I have not studied 
% it sufficiently yet (and may not be able soon).
  <p>
The \ancref{back}{next section} indicates details that may be 
improve understanding the \LaTeX-related command line options 
\code{-i}, \code{-l}, and \code{-n}. 

\labelsection{back}{Background}
For understanding the options \strongcode{-0}, \strongcode{-a}, 
\strongcode{-f}, \strongcode{-F}, and \strongcode{-g}, 
knowing something about the \Find\ utility              %% was The 2012/10/17
and the present \ancref{back-find}{interface} to it may be helpful. 
The options \strongcode{-i}, \strongcode{-l}, \strongcode{-n},
and \strongcode{-o} refer to the \ctanpkgref{myfilist} package 
and its enhancements by the packages \ctanpkgref{longnamefilelist} 
and \ctanpkgref{nicefilelist}, and it may be helpful to know 
what a "\ancref{my-script}{\pkg{myfilist} script}" is.
\begin{description}
  \ditem{\find\lineanc{back-find}}
    We are describing the \strong{interface} to \Find. 
    To learn about the contents of a \find\ command line, 
    the \wikienref{find}{Wikipedia article} may be a good start, 
    next enter "\code{man find}" in the terminal, finally see the 
    \httpref{gnu.org/software/findutils/manual/find.html}{\acro{GNU} manual}. 
      <p>
    \file{adhocfilelist.sh} forms the following \find\ command line 
    in order to get a comma-separated list of files:
    \begin{quote}
      \code{find} 
      \ancref{back-f-prefix}{\metacode{prefix}}
      \ancref{back-f-files} {\metacode{files}}
      \ancref{back-f-day}   {\metacode{day}} 
      \code{-printf ,\%f}
    \end{quote}
    These parts are as follows:
    \begin{description}
      \ditem{\metavar{prefix}\lineanc{back-f-prefix}}
        is
        \begin{quote}
          \code{-L -maxdepth 1 ( -name \cs{*}.tex -o }\metacode{more}\code{ )}
          %% <- follow -> L 2012/11/19
        \end{quote}
        where \metavar{more} in addition to files with extension \ext{tex} allows 
        those with extensions \ext{sty}, \ext{cfg}, \ext{cls}, \ext{dtx}, 
        \ext{def}, and \ext{fd}\pardash \strong{unless} with \strong{options}
        \begin{description}
          \optitem{f}
            where \metavar{prefix} is empty
          \optmetaitem{F}{find-file}
            where \metavar{prefix} is the content of the file \metavar{find-file}.
        \end{description}
        Option \strongcode{-g} is not required with options 
        \strongcode{-0} and \strong{-a} that switch into "\strong{\find\ mode}"
        too without deleting \metavar{prefix}, it is just a way to switch 
        into \find\ mode otherwise without deleting \metavar{prefix}.
      \ditem{\metavar{files}\lineanc{back-f-files}}
        is as in the \ancref{files}{overview}.
      \ditem{\metavar{day}\lineanc{back-f-day}}
        is empty, unless \strongcode{-0} sets it to 
        \begin{quote}
          \code{-daystart~-mtime 0}
        \end{quote}
        or \strongmetaopt{a}{integer} sets it to 
        \begin{quote}
          \code{-daystart~-mtime~}\metacode{integer}
        \end{quote}
        It follows that 
        \begin{itemize}
          \item \strongcode{-0} has the same effect as 
                (is just a shorthand for) \strongcode{-a~0}
          \item if \metavar{integer} is "\code{-\metavar{digits}}",
                files modified \emph{since} \metavar{digits} days ago 
                are allowed
        \end{itemize}
        \emdash cf.~\acro{GNU} documentation on 
        \httpref{gnu.org/software/findutils/manual/html_node/%
                 find_html/Age-Ranges.html}{age ranges.}
          <p>
        A \strong{tricky} little thing here is that \metavar{day}
        may be \emph{expected} to be a \emph{restriction}, 
        an \emph{additional} criterion that a \LaTeX\ source file 
        must meet to be listed. However, if \metavar{files} ends on 
        \strongcode{-o} for "or", \metavar{day} becomes an 
        "additional possibility" for being listed ...
          <p>
        \TODO: This is different with \strong{option \code{-c}} 
        (v0.7) whose functionality also may change soon.
    \end{description}
  \ditem{\pkg{myfilist}\lineanc{my-script}}
    The \LaTeX\ package \ctanpkgref{myfilist} provides some commands 
    to control the input of \LaTeX's \cs{listfiles} so that its 
    output is not a list of files used for typesetting, 
    but a list of arbitrary files that is determined by commands 
    like 
    \begin{quote}
      \cs{ReadFileInfos}\metamandarg{comma-separated-list-of-filenames} 
    \end{quote}
    Essentially this is the command which the list of filenames that 
    was \metavar{files} or that is produced by \find\ is passed to.
    The list is generated by running a \ext{tex} file that contains 
    these commands. Such a file has been called a 
    "\strong{\pkg{myfilist} script}" here. 
    More precisely, such a script is (mostly) structured as follows:
    \begin{quote}
      \ancref{back-load-set}{\metavar{loading-settings}}\\
      \ancref{back-coll-infos}{\metavar{collecting-infos}}\\
      \ancref{back-write}{\metavar{writing-to-screen+file}}
    \end{quote}
    These parts are as follows:
    \begin{description}
      \ditem{\metavar{loading-settings}\lineanc{back-load-set}} 
        at least issues \cs{RequirePackage}\metamandarg{myfilist}. 
        By default, \pkg{adhocfilelist.sh} produces (roughly)
        \begin{quote}
          \cs{RequirePackage}\cb{myfilist}\cs{EmptyFileList}
          \metacode{adhoc-adjust}
        \end{quote}
        for \metavar{loading-settings} where \metavar{adhoc-adjust} 
        adds technical details needed for use with \pkg{adhocfilelist}.
        (Final warnings are suppressed by a command \cs{NoBottomLines}, 
         this might be changed in the future.)
          <p>
        It has been a disadvantage of \LaTeX's \cs{listfiles} functionality 
        that the resulting plain text file list looked good with base filenames 
        only that had up to eight characters. \ctanpkgref{longnamefilelist}
        made up for this disadvantage, and \ctanpkgref{nicefilelist}
        has additional refinements of aligning the list, proposed by 
        Martin \ctanpkgauref{muench-hm}{M�nch}. To use them with 
        \pkg{myfilist}, they are loaded in \metavar{loading-settings}
        as well, and additional settings for column widths may be added there too. 
        This is what \strong{options} \strongcode{-l} and \strongcode{-n} do.
        Besides in their own \acro{PDF} documentation (that you find by following the 
        links), the packages are also briefly described in that file 
        \mirrorctanref{info/latexfileinfo-pkgs/latexfileinfo_pkgs.htm}{%
               \file{latexfileinfo_pkgs.htm}}
        to be read by a web browser.
          <p>
        However, more refined settings may be needed that cannot 
        be controlled by \pkg{adhocfilelist}'s interface so far. 
        The \strong{option} \strongmetaopt{i}{tex-file}
        has the effect that \metavar{loading-settings} becomes
        \begin{quote}
          \metavar{tex-file-content}
          \metavar{adhoc-adjust}
        \end{quote}
        where \metavar{tex-file-content} is the content of the 
        (\ext{tex} file) \metavar{tex-file} (before \cs{endinput}).
      \ditem{\metavar{collecting-infos}\lineanc{back-coll-infos}}
        essentially is that 
        \begin{quote}
          \cs{ReadFileInfos}\metamandarg{comma-separated-list-of-filenames} 
        \end{quote}
        mentioned above.
      \ditem{\metavar{writing-to-screen+file}\lineanc{back-write}}
        essentially is \cs{ListFileInfos} or 
        \begin{quote}
          \cs{ListFileInfos}\metaoptarg{txt-file}
        \end{quote}
        With \pkg{adhocfilelist}, that "\metaoptarg{txt-file}"
        is inserted on \strong{option} \strongmetaopt{o}{txt-file}.
    \end{description}
    However, \pkg{myfilist} has been extended at the occasion 
    of preparing \pkg{adhocfilelist} so that the sections 
    \metavar{collecting-infos} and \metavar{writing-to-screen+file}
    may "collapse" into a single command starting with 
    \cs{ReadListFileInfos}.
\end{description}

\labelsection{custom}{General customization}
In general, options \strongmetaopt{F}{find-file} and 
\strongmetaopt{i}{tex-file} will be needed
to deal with certain directories \metavar{dir}. 
Two ways come to my mind how to simplify this situation. 
\begin{enumerate}
  \item \metavar{find-file} and \metavar{tex-file} may be 
        in the same directory \metavar{dir}, and the latter may es well 
        contain an executable (\code{\Wikienref{chmod}}) 
        script file \metavar{script-file} 
        with the following content:
        \begin{quote}
          \ancref{call}{\metacode{call}} 
          \code{-F~}\metacode{find-file}
          \code{-i~}\metacode{tex-file}
          \metacode{more}
          \code{\string$@}
        \end{quote}
        Here \metavar{call} is as in the \ancref{call}{overview}, 
        and \metavar{more} may be empty or something else that you 
        want to have fixed, such as \strongmetaopt{o}{txt-file}.
          <p>
        \strong{Then} you can \strong{vary} additional options/parameters 
        \metavar{add} by 
        \begin{quote}
          \code{./}\metacode{script-file} \metacode{add}
        \end{quote}
  \item If you have \strong{several directories} \metavar{dir} as before 
        of a certain "type", %%% \metavar{type},           %% rm. 2012/10/17
        such that they can \strong{share} the settings in 
        \metavar{find-file} and \metavar{tex-file}, 
        you can place the latter in some separate ("central") 
        directory \metavar{center} and define an \code{\Wikienref{alias}}
        as follows:
        \begin{quote}
         \EXECUTE{\MakeOther\'}
          \code{alias~}\metacode{name}\code{='}\ancref{call}{\metacode{call}}
                       \code{-F~}\metacode{center}\code/\metacode{find-file}
                       \code{-i~}\metacode{center}\code/\metacode{tex-file} 
                       \metacode{more}\code{'}
        \end{quote}
        (Note that there is no "\code{\string$@}". 
         "\metacode{dir}\code/\metacode{tex-file}" could be replaced by 
         mere "\metacode{tex-file}" if that \metavar{tex-file} 
         is installed like a \TeX~package.)
          <p>
        \strong{Then} you can \strong{vary} additional options/parameters 
        \metavar{add} by 
        \begin{quote}
          \metacode{name} \metacode{add}
        \end{quote}
\end{enumerate}

\labelsection{idea}{The idea/motivation; acknowledgements}
One day (most probably 2012-09-27), 
I wanted to check the \cs{ProvideFile} contents of \emph{two}
files I knew I had edited that day. 
On 2012-09-26, I had thought about simplifying \ctanpkgref{typeoutfileinfo} 
by using a single command line without any piping. 
I realized that \LaTeX~s \cs{typeout} could be replaced by 
\cs{ReadFileInfos} from \ctanpkgref{myfilist}.
Therefore "\Wikienref{ad hoc}".
  <p>
Maybe the same day later, 
I realized that I could no longer remember which files I changed that day. 
In July I had discovered \code{\string$(find ...)}. Now I thought that this 
could be used here. "Ad hoc" is less appropriate here, and first I thought 
that there will be different script files for explicitly specified files 
than for applying \find. 
  <p>
Only then I learnt real shell programming by googling, kind of chance, 
and most helpfully from Chapter~3 of J�rgen Wolfs 
\httpref{openbook.galileocomputing.de/shell_programmierung}{OpenBook}
from \Wikideref{Galileo Press} 
(\code{\Wikienref{getopts}}\pardash bad here, however)
and Prof. J�rgen Plates 
\httpref{netzmafia.de/skripten/unix/unix8.html}{lecture notes}
("Optionen ermitteln"\pardash "by hand").
  <p>
Then quoting/escaping for filename extensions with \Find\ became difficult ...
Quotation marks and "positional parameters" with the shell are very 
difficult/\strong{painful}\pardash\TeX\ is more straightforward ...
On the other hand it is nice that the entire command line with \emph{all} the 
"parameters" is gobbled.

\hrule
\rightpar{\textit{%
    Last~revised~\isotoday\ \copyright~\webdesignref{contact.html}{Uwe L�ck}\\
%     \vspace{6} 
    (using \CtanPkgRef{morehype}{blog.sty})\\   %% ) 2012/10/17
%     \vspace{6}
    License: \httpref{www.latex-project.org/lppl/}{LPPL~1.3c} or later, author-maintained.}}
\entotopofpage
\vspace{240}
\finish

