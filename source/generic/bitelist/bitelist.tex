\ProvidesFile{bitelist.tex}[2012/03/29 documenting bitelist.sty]
\title{\textsf{\huge bitelist.sty                   %% \huge 2012/03/19
       }\\---\\``Splitting" a List at a List Inside
       \\in \TeX's Mouth\thanks{This
       document describes version
       \textcolor{blue}{\UseVersionOf{\jobname.sty}}
       of \textsf{\jobname.sty} as of \UseDateOf{\jobname.sty}.}}
% \listfiles
{ \RequirePackage{makedoc} \ProcessLineMessage{}
  \MakeJobDoc{17}
  {\SectionLevelTwoParseInput}  }
\documentclass[fleqn]{article}%% TODO paper dimensions!?
\ProvidesFile{makedoc.cfg}[{2013/03/25 documentation settings}] 
%%
\author{Uwe L\"uck\thanks{%
        \url{http://contact-ednotes.sty.de.vu}}}
%%
%% 'hyperref':
\RequirePackage{ifpdf}
\usepackage[%
  \ifpdf
%     bookmarks=false,                  %% 2010/12/22
%     bookmarksnumbered,
    bookmarksopen,                      %% 2011/01/24!?
    bookmarksopenlevel=2,               %% 2011/01/23
%     pdfpagemode=UseNone,
%     pdfstartpage=10,
    pdfstartview=FitH,                  %% 2012/11/26 again
%     pdfstartview=0 0 100,             %% 2011/08/22
%     pdfstartview={XYZ null null 1},   %% 2011/08/25
%     pdfstartview={XYZ null null null},%% 2011/08/25
%     pdfstartview={XYZ null null .5},    %% 2011/08/26
%     pdffitwindow=true,          %% 2011/08/22
    citebordercolor={ .6 1    .6},
    filebordercolor={1    .6 1},
    linkbordercolor={1    .9  .7},
     urlbordercolor={ .7 1   1},   %% playing 2011/01/24
  \else
    draft
  \fi
]{hyperref}
\hypersetup{% 
    pdfauthor={Uwe L\374ck}% 
}
%% metadata, |\MDkeywords{<text>}|, |\MDkeywordsstring|:
%% %% 2011/08/22:
\makeatletter
  \newcommand*{\MDkeywords}[1]{%
    \gdef\MDkeywordsstring{#1}%
    \hypersetup{pdfkeywords=\MDkeywordsstring}%% TODO!?
  }
  \@onlypreamble\MDkeywords
%% |\MDaddtoabstract{<par-head>}|, `:' added:
  \newcommand*{\MDaddtoabstract}[1]{%           %% 2012/05/10
    \par\smallskip\noindent
    \strong{#1:}\quad\ignorespaces}
%% \pagebreak[2]
%% |\printMDkeywords|:
  \newcommand*{\printMDkeywords}{%
    \MDaddtoabstract{Keywords}%
    \MDkeywordsstring 
%     \global\let\MDkeywordsstring\relax    %% `%' 2012/11/12
  }
%% The previous definitions mainly are useful with a variant 
%% |\begin{MDabstract}| of \LaTeX's `{abstract}' environment:
  \newenvironment{MDabstract}
                 {\abstract\noindent
                  \hspace{1sp}%% for niceverb
                  \ignorespaces}
                 {\@ifundefined{MDkeywordsstring}%
                               {}%
                               {\printMDkeywords}%
                  \global\let\MDabstract\relax    %% 2012/11/12
                  \global\let\endMDabstract\relax %% 2012/11/12
                  \endabstract}
%% |\[MD]docnewline| 2012/11/12 from `readprov.tex':
  \newcommand*{\MDdocnewline}{\leavevmode\@normalcr[\topsep]}
%% <- `\leavevmode' for use with `\paragraph'.
%%    Sometimes needs to be preceded by a space.
%% 
%% |\MDfinaldatechecks[<tex-script>]| with \ctanpkgref{filedate}:
  \newcommand*{\MDfinaldatechecks}[1][fdatechk]{%
    \AtEndDocument{%
%       \clearpage %% 2013/03/25 no avail -- with `filedate'!
      \def\@pkgextension{sty}%
      \def\NeedsTeXFormat##1[##2]{}%
      \noNiceVerb                       %% 2013/03/22
      \input{#1}%
    }}
  \@onlypreamble\MDfinaldatechecks
\makeatother
%% Use other packages:
\RequirePackage{niceverb}[2011/01/24] 
\RequirePackage{readprov}               %% 2010/12/08
\RequirePackage{hypertoc}               %% 2011/01/23
\RequirePackage{texlinks}               %% 2011/01/24
\RequirePackage{relsize}                %% 2011/06/27
\RequirePackage{color}                  %% 2011/08/06
\RequirePackage{lmodern}                %% 2012/10/29
\RequirePackage{filedate}               %% 2012/11/12
\RequirePackage{filesdo}                %% 2013/03/22 
%% \pagebreak[3]
%% Logical markup:\qquad  |\strong{<chars>}|, |\meta{<chars>}|, 
%% |\acro{<chars>}|, |\pkg{<chars>}|, 
%% |\code{<chars>}|, |\file{<chars>}|:{\sloppy\par}
\makeatletter
  \def\do#1#2{\@ifdefinable#1{\let#1#2}}%% 2012/07/13
  \do\strong\textbf \do\file\texttt \do\acro\textsmaller 
  %% <- wrong tests before 2012/07/13
  \do\meta\textit   \do \pkg\textsf \do\code\texttt
  \ifpdf
    \pdfstringdefDisableCommands{%
        \let\acro\textrm 
        \let\file\textrm                            %% 2011/11/09
        \let\code\textrm                            %% 2011/11/20
        \let\pkg \textrm                            %% 2012/03/23
    }
  \fi
  %% TODO 2011/07/22 -> `htlogml.sty'
\makeatother
%% |\qtdcode{<text>}|: 2012/10/24:
    \newcommand*{\qtdcode}[1]{`\code{#1}'} 
%% |\pkgtitle{<package-name>}{<caption>}| 
\newcommand*{\pkgtitle}[2]{%            %% 2012/07/13
    \global\let\pkgtitle\relax
    \pkg{\huge #1}\\---\\#2\thanks{This 
       document describes version 
       \textcolor{blue}{\UseVersionOf{\jobname.sty}} 
       of \textsf{\jobname.sty} as of \UseDateOf{\jobname.sty}.}}
%% TODO: %% |\TODO| bad with `mdoccorr.cfg'
\newcommand*{\TODO}{\textcolor{blue}{\acro{TODO}}}  %% 2012/11/06
%% `\MDsampleinput[{<file>}' was added 2012/11/06. 
%% Problems with `myfilist.tex' were due to 'parskip.sty'
%% there. On 2012/11/12, we change the former simple macro to a 
%% much more complex
%% |\MDsamplecodeinput[<add-hfuss>]{<file>}| 
\newcommand*{\MDsamplecodeinput}[2][]{%
    \begingroup
        \vskip\bigskipamount \hrule
        \nobreak\vskip-\parskip 
%         \nobreak\vskip\medskipamount
%% Previous mistake (same below) due to manual change 
%% of `\topsep' in the file `myfilist.tex' (2012/11/30).
        \ifx\\#1\\\else
            \hfuzz=\textwidth \advance\hfuzz#1\relax
        \fi
        \noNiceVerb \verbatiminput{#2}%
%         \nobreak\vskip\medskipamount 
        \hrule \vskip-\parskip 
        \bigskip %%% \bigbreak
%% `\bigbreak' made much larger space in `myfilist.tex'.
    \endgroup
}
%% |\ctanpkgdref{<pkg-id>}| adds the printed link to 
%% `ctan.org/pkg' as a footnote. There is a little space 
%% for coloured link borders:
\newcommand*{\ctanpkgdref}[1]{%
    \ctanpkgref{#1}\,\urlfoot{CtanPkgRef}{#1}}
\errorcontextlines=4
\pagestyle{headings}

\endinput 
 %% shared formatting settings
% \ReadPackageInfos{bitelist}
\usepackage{bitelist}
\sloppy
\MDkeywords{macro programming, text filtering, substrings}
\begin{document}
\maketitle
\begin{MDabstract}
'bitelist.sty' provides commands for ``splitting" a token list 
at the first occurrence of a contained token list 
(i.e., for given $\sigma$, $\tau$, 
 return $\beta$ and shortest $\alpha$ s.t.\ $\tau=\alpha\sigma\beta$). 
 As opposed to other packages providing similar features, 
\ (\textit{i})\enspace the method uses \TeX's mechanism of reading 
delimited macro parameters;
\ (\textit{ii})\enspace the splitting macros work by pure expansion, 
without assignments, provided the macro doing the search has been 
defined before processing (e.g., a file);
\ (\textit{iii})\enspace instead of using one macro for a ``substring" 
test and another one to replace the ``substring"---which includes 
extracting corresponding prefix and suffix---, 
the \emph{same} macro that detects the occurrence returns 
the split;
\ (\textit{iv})\enspace 
\httpref{ctan.org/pkg/e-tex}{$\varepsilon$\hbox{-}\TeX} is not required.
\ (And \LaTeX\ is not required.)

This improves the author's \CtanPkgRef{fifinddo}{fifinddo.sty} 
(v0.51---and may once be used there). An elaborated approach 
(additionally to a simpler one) is provided that does not loose 
outer braces of prefix/suffix.

``Substring" detection and ``string" replacement are (implicitly) included 
with respect to certain representations of characters by tokens.
Counting occurrences and ``global" replacement could be achieved 
by applying the operation to earlier results, etc.---so 
this approach seems to be ``fundamental" for a certain larger 
set of list analysis tasks.

The documentation aims to prove the correctness of the methods 
with mathematical rigour.
\par\smallskip\noindent
\strong{Related packages:}\quad 
\ctanpkgref{datatool}, \ctanpkgref{stringstrings}, \ctanpkgref{ted}, 
\ctanpkgref{texapi}, \ctanpkgref{xstring}
\end{MDabstract}
\newpage
\tableofcontents

\section{Task, Background Reasoning, and Usage}
\subsection{The Task Quite Precisely}
\label{sec:task}

Perhaps I should not have written ``splitting" before, 
see Section~\ref{sec:name} why I did so though. 
Actually: 

At first we are dealing with token lists $\tau$ and $\sigma$ 
without braces 
(unless their category code has been changed appropriately)
that can be stored as macros without parameter or in token list registers. 
We want to find out whether $\tau$ contains $\sigma$ (``as a subword") 
in the sense that there are such token lists $\alpha$ and $\beta$ that 
$\tau$ is composed as $\alpha\sigma\beta$, i.e.,
\[\tau=\alpha\sigma\beta\]
and in this case 
we want to get $\alpha$ and $\beta$ of this kind with 
$\alpha$ being the \emph{shortest} possible. 
I.e., if there are such $\gamma$ and $\delta$ that $\tau$
is composed as $\gamma\sigma\delta$, $\alpha$ must be contained 
as a ``prefix" in $\gamma$, 
i.e., $\gamma$ is composed as $\alpha\eta$ for some token list $\eta$. 
The token lists $\alpha$, $\beta$, $\gamma$, $\delta$, $\eta$, 
$\sigma$, and $\tau$ are allowed to be empty throughout.

The task will be extended for some braces in Section~\ref{sec:braces}.

\subsection{Idea of Solution}

\TeX's mechanism of expanding macros (\TeX book Chapter~20)
at least has a built-in mechanism to return such $\alpha$ and $\beta$
\emph{provided} $\tau$ contains $\sigma$. Define
\[`\def<cmd>#1'\sigma`#2'\theta`{<replace-def>}'\]
where $\theta$ must be a token list (maybe of a single token)
that won't occur in $\tau$.\footnote{I am still following others in confusing 
    source code and tokens. I have better ideas, but must expand on them 
    elsewhere. Writing `&\def' rather indicates that it is source code, 
    then $\sigma$ etc. should be replaced by strings that are converted 
    into tokens $\sigma$ etc. 
    <cmd> sometimes is a \emph{string} starting with an escape character, 
    or it is an active character; but sometimes it rather is an ``active" 
    \emph{token} converted from such an escape string or an active character.}
This is a \strong{limitation} of the approach: 
It works for sets of such $\tau$ only that do not contain 
any of a small set of tokens or combinations of them.
('bitelist' will use `\BiteSep', `\BiteStop', and `\BiteCrit', 
 or any other three that can be chosen.)

On the other hand, \TeX's \emph{category codes} 
(\TeX book Chapter~7) can ensure this quite well. 
E.g., we may assume that input ``letters" always have category code 11
(or 12, or one of them), and for $\theta$ we can choose letters 
with \emph{different} category codes such as 3.
Without such tricks, you may often assume that nobody will input 
certain ``silly" commands such as `\BiteStop'. 
(But it may become difficult when you use a package for 
 replacement macros for generating its own documentation \dots)

With a <cmd> as defined above, \TeX\ will
\[\mbox{expand\quad}
    `<cmd>'\tau\theta
  \quad\mbox{to}\quad
    <replace>,\]
where <replace> will be the result of replacing 
\ (a)\enspace all occurrences of `#1' in <replace-def> by $\alpha$ as wanted and
\ (b)\enspace all occurrences of `#2' in <replace-def> by $\beta$ as wanted.
\
I.e., <cmd> returns $\alpha$ as its first argument and $\beta$ as its second argument.
The reason is that <cmd>'s first parameter is delimited by $\sigma$ and the second one by 
$\theta$ in the sense of The~\TeX book p.~203.
Our requirement to get the \emph{shortest} $\alpha$ for the composition of $\tau$ as 
$\alpha\sigma\beta$ is met because \TeX\ indeed looks for the \emph{first} occurrence of 
$\sigma$ at the right of <cmd>. 


\subsection{When We Don't Know \dots}
When $\sigma$ does \emph{not} occur in $\tau$ and we present $\tau\theta$ to <cmd> as 
before, \TeX\ will throw an error saying 
``Use of <cmd> doesn't match its definition."
When the purpose is ``substring detection" only, without returning $\beta$, 
many packages have solved the problem by issuing something like
\[`<cmd>'\tau\sigma\theta\]
Then (still provided $\theta$ does not occurr in $\tau$) 
<cmd>'s second argument is empty \emph{exactly} if $\sigma$ occurs in $\tau$.
This method has, e.g., been employed in \LaTeX's internal &\in@ mechanism 
(e.g., for dealing with package options) and by the \ctanpkgref{substr} package.
\ctanpkgref{datatool} has used the latter's substring test (for $\sigma$)
before calling a macro for replacing 
($\sigma$ by another token list, perhaps thinking of character tokens).

This way you get the wanted $\alpha$ as the first macro argument immediately indeed. 
An obstacle for getting $\beta$ is that <cmd>'s \emph{second} argument now contains 
an occurrence of $\sigma$ that is not an occurrence in $\tau$. 
In \CtanPkgRef{fifinddo}{fifinddo.sty} I didn't have a better idea than using 
another macro to remove the ``dummy text" from the second argument.
I considered it an advantage as compared with 'datatool' that 
\emph{one} macro could do this for \emph{all} replacement jobs, 
while 'datatool' uses \emph{two} macros with $\sigma$ as a delimiter 
for each $\sigma$ to be replaced.

But still, 'fifinddo' has used \emph{two} macros for each replacement, 
the extra one being for presenting $\tau$ to <cmd>, using a job identifier. 
This could be improved within 'fifinddo', but I could never afford 
to take the time for this.

\subsection{The Trick}
\label{sec:trick}

The solution presented here is not very ingenious, 
many students would have found it in an exercise for a math course.
My personal approach was looking at &\GetFileInfo from \LaTeX's 
\ctanpkgref{doc} package. There they try to get \emph{two} occurrences 
of a space token this way:\footnote{We are undoubling the hash marks 
                                    inside the definition text of 
                                    &\GetFileInfo.}
\[`\def\@tempb#1 #2 #3\relax#4\relax{%'\]
and &\@tempb is called as 
\[`\@tempb'\tau`\relax? ? \relax\relax'\]
or with $\tau=<list>$
\[`\@tempb<list>\relax? ? \relax\relax'\]
The final &\relax may not be removed, but for 'doc' it doesn't harm. 
It harms for \emph{me} when I don't want to have a `\relax' in a `.log' file list.
`\empty' would be better, however \dots

The idea is to use a \emph{three}-parameter macro for that \emph{single} occurrence 
of $\sigma$. We introduce a 
``dummy separator" $\zeta$ (or <sep>, `\BiteSep') 
between $\tau$ and the ``dummy text" and a 
``criterion" $\rho$ ($=<crit>$, `\BiteCrit') 
for determining occurrence of $\sigma$ ($=<find>$) in $\tau$ ($=<list>$).
Neither $\zeta$ nor $\rho$ must occur in $\tau$.
We will have definitions about as
\[`\def<cmd>#1'\sigma`#2'\zeta`#3'\theta`{<replace-def>}'\]
or
\[`\def<cmd>#1<find>#2<sep>#3<stop>{<replace-def>}'\]
and $\tau$ will be presented with context
\[`<cmd>'\tau\zeta\sigma\rho\zeta\theta
  \quad\mbox{or}\quad
  <cmd><list><sep><find><crit><sep><stop>
  \]
This ensures that <cmd> finds its parameter delimiters $\sigma$, $\zeta$, 
and $\theta$, in this order. $\sigma$ occurs in $\tau$ exactly if the second 
argument of <cmd> is $\rho$, and in this case the first occurrence 
of the second parameter delimiter $\zeta$ delimits $\tau$. 
Then <cmd>'s first argument is $\alpha$, and the second one is $\beta$, 
as wanted.

<cmd>'s \emph{third} parameter is delimited by the final $\theta$ (`\BiteStop'). 
When $\sigma$ occurs in $\tau$, <cmd>'s third argument starts after the first 
of the two $\zeta$, so it is $\sigma\rho\zeta$. 
It is just ignored, this way <cmd> removes all the ``dummy" material 
after $\tau$. When $\sigma$ does \emph{not} occur in $\tau$, 
we ignore all of its arguments, and the macro that invoked <cmd> 
must decide what to do next, e.g., keeping $\tau$ elsewhere 
for presenting it to another parsing macro resembling <cmd>.


\subsection{Installing and Calling}
The file 'bitelist.sty' is provided ready, installation only requires
putting it somewhere where \TeX\ finds it
(which may need updating the filename data
 base).\urlfoot{ukfaqref}{inst-wlcf}           %% corr. 2011/02/08

Below the `\documentclass' line(s) and above `\begin{document}',
you load 'bitelist.sty' (as usually) by
\begin{verbatim}
  \usepackage{bitelist}
\end{verbatim}
between the `\documentclass' line and `\begin{document}'; 
or by 
\begin{verbatim}
  \RequirePackage{bitelist}
\end{verbatim}
within a package file, or above or without the `\documentclass' line.
Moreover, the package should work \emph{without} \LaTeX\ and may be 
loaded by 
\begin{verbatim}
  \input bitelist.sty
\end{verbatim}
Actually, using the package for macro programming requires understanding 
of pp.~20f.\ of The~\TeX book. On the other hand, the package may be loaded
(without the user noticing it) automatically by a different package that 
uses programming tools from the present package.

\section{Implementation Part I}
\subsection{Package File Header (Legalize)}
\def\filename{bitelist}                     \def\filedate{2012/03/29} 
\def\fileversion{v0.1} \def\fileinfo{split lists in TeX's mouth (UL)}
%% Copyright (C) 2012 Uwe Lueck,
%% http://www.contact-ednotes.sty.de.vu
%% -- author-maintained in the sense of LPPL below --
%%
%% This file can be redistributed and/or modified under
%% the terms of the LaTeX Project Public License; either
%% version 1.3c of the License, or any later version.
%% The latest version of this license is in
%%     http://www.latex-project.org/lppl.txt
%% There is NO WARRANTY - this rather is somewhat experimental.
%%
%% Please report bugs, problems, and suggestions via
%%
%%   http://www.contact-ednotes.sty.de.vu
%%
%% === Proceeding without \LaTeX ===
%% Some tricks from Bernd Raichle's \CtanPkgRef{ngerman}{ngerman.sty}---I
%% need \LaTeX's `\Provides'\-`Package' for \ctanpkgref{fileinfo}, 
%% my package version tools. With 'readprov.sty', it issues `\endinput', 
%% close conditional before:
\begingroup\expandafter\expandafter\expandafter\endgroup
\expandafter\ifx\csname ProvidesPackage\endcsname\relax \else
    \edef\fileinfo{\noexpand\ProvidesPackage{\filename}%
        [\filedate\space \fileversion\space \fileinfo]}
    \expandafter\fileinfo
\fi
\chardef\atcode=\catcode`\@
\catcode`\@=11 % \makeatletter
%% Providing \LaTeX's `\@firstoftwo' and `\@secondoftwo': 
\long\def\@firstoftwo #1#2{#1}
\long\def\@secondoftwo#1#2{#2}
%% 
%% === Basic Parsing (No Braces) ===
%%
%% |\BiteMake{<def>}{<cmd>}{<find>}| provides the parameter text 
%% (\TeX book p.~203) for defining (by <def>) a macro <cmd> that will 
%% search for <find>:
\def\BiteMake#1#2#3{#1#2##1#3##2\BiteSep##3\BiteStop}
%% With |\BiteFindByIn{<find>}{<cmd>}{<list>}|, 
%% you can use a <cmd> (perhaps defined by &\BiteMake) 
%% in order to search <find> in <list>. 
%% This is expandable as promised:
\def\BiteFindByIn#1#2#3{%
    #2#3\BiteSep#1\BiteCrit\BiteSep\BiteStop}
%% Preparing a possible &\edef as <def>:
\let\BiteSep\relax  \let\BiteStop\relax
%% And this is important in any case for correct testing of 
%% occurrence:\footnote{The idea for the ``funny `Q'" 
%%                      is from the \ctanpkgref{ifmtarg} package.}
\catcode`\Q=7 \let\BiteCrit=Q \catcode`\Q=11
%% Perhaps you could increase safety of tests by using something similar to the funny `Q' 
%% for &\BiteSep and &\BiteStop.
%% %% 2012/03/28:
%% However, this would additionally require reimplementation of 
%% the macros for keeping braces (Section~\ref{sec:braces}) using `\edef'.
%% % It appears to me, however, that expandable tests like this one 
%% % never are perfectly safe; you only can say that it is safe with a 
%% % source meeting certain conditions. \ctanpkgref{fifinddo} originally 
%% % was made for ``plain text," to be read from files without assigning 
%% % \TeX's special category codes. \emph{Here} we assume that the source 
%% % (text in \cs{Provides.\empty..} arguments) will never contain such a 
%% % ``funny `Q'".
%%
%% === Simple Conditionals ===
%% By |\BiteMakeIfOnly{<def>}{<cmd>}{<find>}|, you can make a command <cmd>
%% that with
%% \[|\BiteFindByIn{<find>}{<cmd>}{<list>}{<yes>}{<no>}|\]
%% chooses <yes> if <find> occurs in <list> and <no> otherwise.
\def\BiteMakeIfOnly#1#2#3{\BiteMake{#1}{#2}{#3}{\BiteIfCrit{##2}}}
%% |\BiteIfCrit{<suffix>}{<yes>}{<no>}| 
%% is the basic test for occurrence of <find> in <list>:
\def\BiteIfCrit#1{\ifx\BiteCrit#1\expandafter\@secondoftwo
%% If <cmd>'s second argument---same as &\BiteIfCrit's first argument---is 
%% empty, &\BiteCrit is compared with &\expandafter, so <yes> is chosen.
%% That is correct, it happens when <find> is a suffix of <list>.
                           \else \expandafter\@firstoftwo \fi }
%%
%% === Passing Results Completely---No Braces ===
%% So the previous `\BiteMakeIfOnly' generates pure tests on occurrence, 
%% giving away information about prefix and suffix. 
%% It may be considered a didactical step fostering understanding of the following. 
%% % \medskip\noindent
%% % ** Generic Fundamental Splitter---No Braces **
%% %% Wrong:
%% % With the above `\BiteMakeIfOnly', the user can choose on her own
%% % information about the composition of <list> to use. 
%% % Perhaps it is not easy to understand. 
%% % `\BiteMakeIfOnlySplit{<def>}{<cmd>}{<find>}' by contrast, 
%% % should satisfy ``all needs" by providing \emph{all} the information 
%% % about splitting a <list>. It may also be a template for using 
%% % `\BiteMakeIfOnly'. 
%% When, by contrast \[|\BiteMakeIf{<def>}{<cmd>}{<find>}|\]
%% has been issued, a later 
%% $$|\BiteFindByIn{<find>}{<cmd>}{<list>}{<list>}{<yes>}{<no>}|\eqno(*)$$
%% will expand to 
%% \[`<yes>{<prefix>}{<find>}{<suffix>}'\]
%% if <list> is composed as <prefix><find><suffix>
%% and <prefix> is the shortest $\alpha$ such that there is some $\beta$
%% with $<list>=\alpha<find>\beta$. Otherwise, $(*)$ will expand to
%% \[`<no>{<list>}'\]
%% This gives all the information available. 
%% For actual applications, it may be too much, and the macro programmer 
%% may do something in between of `\BiteMakeIfOnly' and `\BiteMakeIf':
\def\BiteMakeIf#1#2#3{%
    \BiteMake{#1}{#2}{#3}##4##5##6{%
%% In the replacement text, we first do the same as with `\BiteMakeIfOnly':
      \BiteIfCrit{##2}%
%% What follows is new. <cmd>'s third argument is ignored. 
%% The fourth keeps the original <list>. 
%% <yes> is <cmd>'s fifth and <no> is its sixth argument.
      {##5{##1}{#3}{##2}}%  %% if #3 in ##4
      {##6{##4}}%           %% otherwise
    }%
}
%% In $(*)$, <list> has been doubled. That was no mistake. 
%% It is due to a shortcoming of `\BiteFindByIn'.
%% With
%% \[|\BiteFindByInIn{<find>}{<cmd>}{<list>}{<yes>}{<no>}|\]
%% you get the same result as with $(*)$:
\def\BiteFindByInIn#1#2#3{\BiteFindByIn{#1}{#2}{#3}{#3}}
%% TODO not sure about command names yet
%%
%% == Example Applications == 
%% === Splitting at Space ===
%% \label{sec:space}
%% This work actually arose from modifying `\GetFileInfo'
%% as provided by \LaTeX's \ctanpkgref{doc} package
%% so that it would deal reasonably with ``incomplete" file info---for
%% the \ctanpkgref{nicefilelist} package. 
%% `\GetFileInfo' works best when the file info contains 
%% at least \emph{two} blank spaces. But how many are there indeed?---And 
%% I wanted to do it \emph{expandably:} while `\GetFileInfo' issues 
%% \emph{definitions} of `\filedate', `\fileversion', and `\fileinfo', 
%% date, version, and info should be passed as \emph{macro arguments}.
%% \medbreak\noindent
%% |\BiteIfSpace| tries splitting at the next blank space passes results:
\BiteMake{\def}{\BiteIfSpace}{ }#4#5#6{%
    \BiteIfCrit{#2}{#5{#1}{#2}}{#6{#4}}}
%% The difference to the `\BiteMakeIf' construction is that we do not 
%% pass <find>, the space---it's not essential. 
%% (TODO names may change ...)
%%
%% Now \[|\BiteFindByInIn{ }{\BiteIfSpace}{<list>}{<yes>}{<no>}|\]
%% will pass prefix/suffix to <yes> or <list> to <no>.
%% If this is needed frequently, here is a shorthand 
%% |\BiteGetNextWord{<list>}{<yes>}{<no>}|:
\def\BiteGetNextWord{\BiteFindByInIn{ }\BiteIfSpace}
%% See a test in `bitedemo.tex' (Section~\ref{sec:demo}).
%% 
%% === Splitting at Comma ===
%% ... left as an exercise to the reader ...
%%
%% == Keeping Braces: Reasoning ==
%% \label{sec:braces}
%% Now we want to generalize task (Section~\ref{sec:task})
%% and solution (Section~\ref{sec:trick}) for the case that
%% $\tau=<list>$ has (balanced) braces 
%% (with category codes for argument delimiters), 
%% while $\sigma=<find>$ still has not (does not work with our method). 
%% So with $\tau=\alpha\sigma\beta$, 
%% $\alpha$ (``prefix") or $\beta$ (``suffix") or both 
%% may contain braces. But we consider another restriction: 
%% braces must be balanced in $\alpha$ and in $\beta$, 
%% we don't try parsing inside braces 
%% (as opposed to the search for asterisks in Appendix~D 
%%  of The~\TeX book).
%%
%% According to \TeX book p.~204, when a macro <cmd> finds an argument 
%% formed as `{<tokens>}', in <cmd>'s replacement text only <tokens>
%% is used, i.e., outer braces are removed. 
%% So when $\alpha=`{<tokens>}'$, a parser <cmd> as defined by our 
%% methods above will return <tokens> instead of `{<tokens>}'---likewise 
%% for $\beta$. We are now trying to \emph{keep} outer braces in prefix/suffix
%% by a more elaborate method.
%% 
%% The idea is to present $\tau=<list>$ with context\footnote{Perhaps 
%%      I am confusing `&\empty' and the token list containing just `&\empty' here?}
%% \[`<cmd>\empty<list><stop><sep><find><crit><sep><stop>'\]
%% or in the notation of Section~\ref{sec:trick}
%% \[`<cmd>\empty'\tau\theta\zeta\sigma\rho\zeta\theta\]
%% Then, if <find> occurs in <list>, we must remove the `\empty'
%% from the prefix that we get with the earlier method (easy)
%% and <stop> from the suffix (tricky, similar problem recurs).
%% Using old $\theta$ for a new purpose works here because 
%% <cmd> will look for $\theta$ only when it has found $\zeta$ before.
%%
%% Mere testing for occurrence is not affected. 
%% \[`\BiteMakeIfOnly' \quad \mbox{and} \quad `\BiteFindByIn'\] 
%% still can be used. 
%% We provide an improved version of 
%% \[`\BiteMakeIf'   \quad (`\BiteMakeIfBraces')\] and of
%% \[`\BiteFindInIn' \quad (`\BiteFindInBraces').\]
%%
%% == Implementation Part II    ==
%% === Keeping Braces           ===
%% \[|\BiteFindByInBraces{<find>}{<cmd>}{<list>}{<yes>}{<no>}|\] 
%% varies `\BiteFindByInIn' according to the previous:
\def\BiteFindByInBraces#1#2#3{%
    #2\empty#3\BiteStop\BiteSep#1\BiteCrit\BiteSep\BiteStop{#3}}
%% Such a <cmd> can be made by |\BiteMakeIfBraces{<def>}{<cmd>}{<find>}|:
\def\BiteMakeIfBraces#1#2#3{%
    \BiteMake{#1}{#2}{#3}##4##5##6{%
      \BiteIfCrit{##2}%
%% <no> works as before. For <yes>, first the `\empty' in the prefix
%% is expanded for vanishing. 
%% `\BiteTidyI' and `\BiteTidyII' continue tidying.
      {\expandafter \BiteTidyI                      %% if #3 in ##4
            \expandafter{##1}%                      %% prefix
%% Another `\empty' avoids that removal of `\BiteStop' in suffix 
%% by `\BiteTideII' removes outer braces:
                        {\BiteTidyII\empty##2}%     %% suffix
                        {#3}%                       %% find
                        {##5}}%                     %% yes
      {##6{##4}}%                                   %% otherwise
    }%
}
%% |\BiteTidyI{<prefix>}{<suffix>}| \ first expands `\BiteTidyII'
%% for removing `\BiteStop' in <suffix>. 
%% `\empty' from `\BiteFindByInBraces' remains and is expanded next 
%% for vanishing. Finally, `\BiteTidied' reorders arguments
%% for operation of <yes>:
\def\BiteTidyI#1#2{%
    \expandafter\expandafter\expandafter \BiteTidied 
        \expandafter\expandafter\expandafter {#2}{#1}}
\def\BiteTidyII#1\BiteStop{#1}
\def\BiteTidied#1#2#3#4{#4{#2}{#3}{#1}}
%% 
%% === Leaving the Package File ===
\catcode`\@=\atcode
\endinput
%%
%% === VERSION HISTORY ===

v0.1   2012/03/26   started
       2012/03/27   continued, restructured
       2012/03/28   continued, separate sections for "Mere Occurrence" 
                    vs. ...; keeping braces, \BiteIfCrit
       2012/03/29   proceeding without LaTeX corrected, restructured


\section{Examples/Tests}
\label{sec:demo}
You should find a separate file `bitedemo.tex' 
with examples. It may be run separately with `tex' 
(Plain \TeX)---demonstrating that 'bitelist' is ``\strong{generic}", 
then finish by entering `\bye'. 
With ```latex bitedemo.tex'", end the job by entering `\stop'.
\strong{Expandability} is demonstrated by the `\BiteFind' commands 
running with `\typeout'.
\medskip
\noNiceVerb
\hrule
\verbatiminput{bitedemo.tex}
\hrule
\useNiceVerb

\section{The Package's Name}
\label{sec:name}

This package deals with \TeX's expansion mechanism. 
In Knuth's metaphor, this is \TeX's mouth. 
I am not entirely sure, I have never understood it, 
or I have understood it only for a few days or hours. 
However, the package deals with ``Lists in \TeX's Mouth" 
as described in Alan Jeffrey's 1990 
\tugbartref{tb11-2/tb28jeffrey}{\acro{TUG}boat paper} 
(Volume~11, No.~2, pp.~237--245).\foothttpurlref{% 
    tug.org/TUGboat/tb11-2/tb28jeffrey.pdf} 

``Splitting" in title and abstract is an attempt to describe 
the package brief{}ly without speaking Mathematicalese. 
It roughly refers to certain \Wikienref{string functions} 
in various programming languages\foothttpurlref{%
    en.wikipedia.org/wiki/String\string_functions\string#split}
with ```split'"  in their name.
However, there strings are splitted at separators such as commas. 
I am thinking here that a comma is a certain string ```,'", 
and this can be generalized to ``splitting" at any substring. 
With \TeX, the analogues are (a)~the token with the character code 
of the comma and category code 12, or the token list consisting of this 
single token,---and (b)~other lists of tokens~\dots

Anyway, calling a triple $(\alpha,\sigma,\beta)$ of token lists 
such that $\tau=\alpha\sigma\beta$ a ``split" of $\tau$ 
is not necessarily a bad idea.
Moreover, the blank space example (Section~\ref{sec:space})
is very close to the original idea of splitting at separators, 
a blank space is about as common as a separator as the comma is.

Finally, according to \urlhttpref{en.wiktionary.org}, 
the Proto-Indo-European origin of
\httpref{en.wiktionary.org/wiki/bite}{``to bite"}
just means ``to split."\foothttpurlref{en.wiktionary.org/wiki/bite\string#Etymology}
So in \TeX's mouth, splitting and biting is the same.


\end{document}

VERSION HISTORY

2012/03/26  for v0.1    started 
2012/03/27              pages of motivation etc.
2012/03/28              abstract: "mathematical rigour"; 
                        \section{Implementation}, \section{Task, ...}; 
                        \newpage, \LaTeX\; reference to sec:braces; 
                        "Examples/Tests" halfway; "Package's"; 
                        LaTeX not required, ...
2012/03/29              "Implementation Part I", label sec:demo; 
                        keywords etc. 
