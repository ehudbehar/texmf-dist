% $Id: wripro.tex 414 2018-06-12 23:04:41Z karl $
% Used in tex4ht.sty.  Not installed in runtime.
% 
% Copyright (C)) 2009-2010 TeX Users Group
% Copyright (C) 1996-2009, 2018 Eitan M. Gurari
%
% This work may be distributed and/or modified under the
% conditions of the LaTeX Project Public License, either
% version 1.3c of this license or (at your option) any
% later version. The latest version of this license is in
%   http://www.latex-project.org/lppl.txt
% and version 1.3c or later is part of all distributions
% of LaTeX version 2005/12/01 or later.
%
% This work has the LPPL maintenance status "maintained".
%
% The Current Maintainer of this work
% is the TeX4ht Project <http://tug.org/tex4ht>.
% 
% If you modify this program, changing the 
% version identification would be appreciated.

%\openin15= wripro.tex
%\ifeof15 \closein15 \else \closein15 \input wripro.tex \fi



\Section{Root Point}

\<html TeX4ht protex\><<<
\ifx\html:addr\:UnDeF
   |<html common to TeX4ht and protex|>
   \Htmltrue
\fi
>>>

\ifalprotex

\<html+nohtml protex\><<<
|<new ifs|>
|<html+nohtml common to TeX4ht and protex|>
|<nohtml common to TeX4ht and protex|>
>>>

\fi

\<non-html utilities\><<<
|<nohtml common to TeX4ht and protex|>
>>>


\SubSection{Directory for Files}

\<html common to TeX4ht and protex\><<<
\writesixteen{--- needs --- tex4ht \jobname\space ---}%
>>>




\<html common to TeX4ht and protexNO\><<<
\ifx\HDir\:UnDef \let\HDir=\empty \fi
\let\:HDir=\HDir
\immediate\write16{--- needs --- TeX4ht \ifx\HDir\empty
               \else-d \HDir\space\space\fi \jobname\space ---}%
>>>


\SubSection{Identifiers and Counters}


\<new ifs\><<<
\ifx \Htmltrue\:UnDef
   \def\:temp{\csname newif\endcsname}
   \expandafter\:temp \csname ifHtml\endcsname  \Htmlfalse
\fi
>>>


\<html common to TeX4ht and protex\><<<
\ifx\tmp:cnt\:UnDeF    \csname newcount\endcsname\tmp:cnt \fi
\def\g:advance#1{\bgroup \def\:temp{#1}%
                 \tmp:cnt=#1\afterassignment\:gplus \advance\tmp:cnt}
\def\:gplus{\expandafter\xdef\:temp{\the\tmp:cnt}\egroup}
\def\html:lbl{1}
\def\html:addr{\xdef\last:haddr{\file:id-\html:lbl}%
     \g:advance\html:lbl by 1\relax }
>>>

\<html common to TeX4ht and protex\><<<
\ifx  \file:id\:UnDef
   \gdef\file:id{|<first file id|>}
   \global\let\maxfile:id=\file:id
\fi
>>>

\<first file id\><<<
1>>>



\SubSection{Relax Right Margin}

\<html common to TeX4ht and protex\><<<
\hbadness=10000      \vbadness=10000
\:CheckOption{fussy}  \if:Option  \else
   \hfuzz=99in          \vfuzz=\hfuzz
\fi
\hyphenpenalty=1000  \exhyphenpenalty=1000
\def\html:rightskip{\rightskip = 0pt plus 0.5\hsize  minus0.5\hsize }
\html:rightskip
>>>


%%%%%%%%%%%%%%%%%%%%%%%%%%%%%%%%%%%%%
\SubSection{Foot and Head Lines}
%%%%%%%%%%%%%%%%%%%%%%%%%%%%%%%%%%%%%

% \<html common to TeX4ht and protex\><<<

\<html protex\><<<
\ifx \footline\:UnDef \else
   \footline={\hfil}   \headline={\hfil}
\fi
>>>


\Section{Html Commmands From Users}

The \''\let\ht:special=\special' is for protection
against redefinitions of the \''\special', as is the case in french.sty.


\<html common to TeX4ht and protex\><<<
\ifx \ht:special\:UnDef
   \let\ht:special=\special
\fi
\def\HCode#1{\ifx \EndPicture\:UnDef
   \relax\ifvmode\leavevmode\fi\ht:special{t4ht=#1}\fi}
\let\:HCode=\HCode
\let\Hbrakets\empty
\def\NewLineChar{\bgroup \catcode`\^=7 \:NewLineChar}
\def\:NewLineChar#1{\egroup \def\:newlnch{ #1}}
\NewLineChar{^^J}
>>>


\<nohtml common to TeX4ht and protex\><<<
\def\HCode#1{}
>>>

The \`+\HCode+' starts a new paragraph if introduced in vertical mode.
The \`+\leavevmode+' is included for proper spacing in dviht.



\Section{Cross References within Html}

   - NCSA Mosaic can't handle the
 end-of-anchor {\tt</...>} tag if it is not on one line.
For instance, in {\tt
<A>...<A\vfil\break>},
   Mosaic paints everything blue until the next {\tt </A> }it finds that is
 entirely on one line.

\<html+nohtml common to TeX4ht and protex\><<<
\def\EndLink{\ifx \empty:lnk\empty \HCode{\Hbrakets</\tag:A>}\fi}
\def\:HRef{\ifx [\:temp \expandafter\::HRef
           \else \expandafter\:::HRef \fi }
\def\HT:tag{\ifx -\let:val   \expandafter\H:Tag
            \else            \expandafter\HTa:g\fi }
\def\HTa:g{\H:Tag-}
\def\Link{\let\:attr\empty \futurelet\let:val\HRefT:ag}
\def\HRefT:ag{\ifx -\let:val \expandafter\H:RefTag
              \else          \expandafter\HRefTa:g\fi   }
\def\HRefTa:g{\H:RefTag-}
\def\H:RefTag#1{\futurelet\:temp\:HRefTag}
\def\:HRefTag{\ifx [\:temp \expandafter\::HRefTag
              \else \expandafter\:::HRefTag \fi }
>>>


A \`'\Link-{..}{b}' asks that a \''\Tag' will not be produced for `b'.
A \`'\Link[\empty]{b}{..}' asks that a \''\Ref' will not be produced for
`a'. The same is true for any other non-empty replacement to \''\empty'
A `b' for which we neither have a \''\Tag' and a \''\Ref' provides
a memory save in `strings out of', `string characters out of', and
`multiletter control sequences' (see log files).

\<html common to TeX4ht and protex\><<<
\def\::HRef[#1]{\get:attr{#1}\::hRef\:::HRef}
\def\::HRefTag[#1]{\get:attr{#1}\::hRefTag\:::HRefTag}
\def\get:attr#1{\edef\:attr{\noexpand\get:atr #1 |<par del|>}\:attr}
\def\get:atr#1 #2|<par del|>#3#4{\def\:attr{\space#2}%
   \def\:te:mpa{#1}\ifx \:te:mpa\empty \let\:te:mpa#4\else
   \def\:te:mpa{#3[#1]}\fi \:te:mpa}
>>>


\<html common to TeX4ht and protex\><<<
\def\H:Tag#1#2{\:TagHTag{#2}\HCode{\Hbrakets<\tag:A\:newlnch
   \NAME:"#2"\empty:lnk>}}
\def\::hRef[#1]#2{%
   \HCode{\Hbrakets<\tag:A\:newlnch \if\relax#1#2\relax \NOHREF:{}\else
             \HREF:"#1\if\relax#1\relax \else\:sharp #2\fi"\fi
           \:attr \empty:lnk>}}
\def\:::HRef#1{%
   \HCode{\Hbrakets<\tag:A\:newlnch \if\relax#1\relax \NOHREF:{}\else
               \HREF:"\get:hfile{#1}\:sharp #1"\fi \:attr \empty:lnk>}}
>>>



\<\><<<
\def\::hRef[#1]#2{\def\:te:mp{#1}\def\:te:mpa{#1#2}%
   \HCode{<\tag:A\:newlnch \ifx\:te:mpa\empty \NOHREF: \else
             \HREF:"#1\ifx\:te:mp\empty \else\:sharp #2\fi"\fi
           \:attr \empty:lnk>}}
\def\:::HRef#1{\def\:te:mpa{#1}%
   \HCode{<\tag:A\:newlnch \ifx\:te:mpa\empty \NOHREF: \else
               \HREF:"\get:hfile{#1}\:sharp #1"\fi \:attr \empty:lnk>}}
>>>




\<html common to TeX4ht and protex\><<<
\def\::hRefTag[#1]#2#3{%
   \if\relax#3\relax\else\:TagHTag{#3}\fi
   \HCode{\Hbrakets<\tag:A\:newlnch \if\relax#1#2\relax \NOHREF:{#3}\else
                 \HREF:"#1\if\relax#2\relax \else\:sharp #2\fi"\fi
               \if\relax#3\relax\else\space \NAME:"#3"\fi
             \:attr \empty:lnk>}}
\def\:::HRefTag#1#2{%
   \if\relax#2\relax\else\:TagHTag{#2}\fi
   \HCode{\Hbrakets<\tag:A\:newlnch \if\relax#1\relax \NOHREF:{#2}%
                 \else\HREF:"\get:hfile{#1}\:sharp #1"\fi
               \if\relax#2\relax\else\space  \NAME:"#2"\fi
             \:attr \empty:lnk>}}
\let\empty:lnk=\empty
\def\NOHREF#1{\space}
>>>

\<\><<<
\def\::hRefTag[#1]#2#3{\def\:te:mp{#2}\def\:te:mpa{#1#2}\def\:te:mpb{#3}%
   \ifx\:te:mpb\empty\else\:TagHTag{#3}\fi
   \HCode{<\tag:A\:newlnch \ifx\:te:mpa\empty \NOHREF: \else
                 \HREF:"#1\ifx\:te:mp\empty \else\:sharp #2\fi"\fi
               \ifx\:te:mpb\empty\else\space \NAME:"#3"\fi
             \:attr \empty:lnk>}}
\def\:::HRefTag#1#2{\def\:te:mp{#1}\def\:te:mpa{#2}%
   \ifx\:te:mpa\empty\else\:TagHTag{#2}\fi
   \HCode{<\tag:A\:newlnch \ifx\:te:mp\empty \NOHREF:
                 \else\HREF:"\get:hfile{#1}\:sharp #1"\fi
               \ifx\:te:mpa\empty\else\space  \NAME:"#2"\fi
             \:attr \empty:lnk>}}
>>>







\''\def\NOHREF:#1{\ifx \hrEF:\HREF: \space NOHREF \fi}' is not part of html!

\<html protex\><<<
\def\HREF:{ href=}
\def\tag:A{a}
\def\NAME:{ name=}
>>>


% \let\hrEF:=\HREF:


We don't use \`'^^J' to break lines in \''\write' because user may
change it, and because I had unexplained problem with it in emtex.

\<html common to TeX4ht and protex\><<<
\def\:TagHTag#1{\ifx -\let:val\else \Tag{|<HTag tag|>#1}{\file:id}\fi}
>>>

\<nohtml common to TeX4ht and protex\><<<
\def\H:Tag#1#2{}
\def\::HRef[#1]#2{}
\def\:::HRef#1{}
\def\::HRefTag[#1]#2#3{}
\def\:::HRefTag#1#2{}
\def\:TagHTag#1{}
>>>




\Section{Html Addresses}

\SubSection{Retrieving for HREF}

\<html common to TeX4ht and protex\><<<
\def\get:href{\expandafter\get::href}
\def\get::href#1-#2-{\ifnum #1=\file:id\space         \else
      |<get html file name|>\fi
   \:sharp |<make html addr|>}
>>>



\<html common to TeX4ht and protexNO\><<<
\def\get:href{\expandafter\get::href}
\def\get::href#1-#2-{\ifnum #1=\file:id\space         \else
   \:HDir |<get html file name|>\fi
   \:sharp |<make html addr|>}
>>>

\SubSection{Retrieving for NAME}

\<html common to TeX4ht and protex\><<<
\def\get:htag{\expandafter\get::htag}
\def\get::htag#1-#2-{|<make html addr|>}
>>>


\SubSection{Generating}

\<make html addr\><<<
\make:addr{#2}>>>

\<html protex\><<<
\def\make:addr#1{%
   \ifnum\clearcode:id>0 \romannumeral\clearcode:id Q\else PrTx\fi
   |<tail for file name|>#1}
\ifx\clearcode:id\:UnDef
     \def\clearcode::id{0}  \def\clearcode:id{0}
\fi
>>>

We had before \`'\space' after \`'#1'. It was not a problem because
\`'|<tail...' was a \''\roman..' that absorbed the space. But
now with no one absorbing it, it became a problem.


\<html protexNO\><<<
\def\make:addr#1{%
   \ifnum\clearcode:id>0 \romannumeral\clearcode:id Q\fi
   |<tail for file name|>#1\space}
\ifx\clearcode:id\:UnDef
     \def\clearcode::id{0}  \def\clearcode:id{0}
\fi
>>>




\<tail for file name\><<<
>>>

Originally, we had \`'\romannumeral' for tail, but it created named
that are too long, and cause memory overflow.


The condition \`'\ifnum\clearcode:id>0' is set to hold only for
references to code fragments.

Romannumeral might be better tags then just numerals because they
might be less likely to be generater manually.

\SubSection{File Names}


\`=\Ref= and \`=\Tag= are mechanisms to exchange information. The
following can be used to connect locations. (Note, however, that
\`=\Tag= may appear in more than one location sending a sequences of tags
separated by comma. In such cases, we arbitrarily go for the first
choise.)


\<init XrefFile\><<<
\ifx \aXrefFile\:UnDef \let\aXrefFile=\empty \fi
>>>


\<html common to TeX4ht and protex\><<<
|<init XrefFile|>
\def\get:hfile#1{%
   \expandafter\ifx\csname
               |<tag of Tag|>|<HTag tag|>#1\endcsname\relax
      \get@hfile{#1}%
   \else
      \expandafter\expandafter\expandafter\get::hfile
          \csname |<tag of Tag|>|<HTag tag|>#1\endcsname,//%
   \fi  }
\def\get::hfile#1,#2//{%
   \ifnum \file:id=0#1  \else
       |<get file name for tag|>%
   \fi }
\ifx\get@hfile\:Undef \let\get@hfile=\:gobble \fi
>>>


\<get file name for tag\><<<
\expandafter\ifx
   \csname |<tag of Tag|>|<auto file tag|>#1%
                 |<Tag/Ref: file-id -> file-name|>\endcsname\relax \else
      \:LikeRef{|<auto file tag|>#1|<Tag/Ref: file-id -> file-name|>}%
\fi >>>


\<get file name for tagNO\><<<
\expandafter\ifx
   \csname |<tag of Tag|>|<auto file tag|>#1%
                 |<Tag/Ref: file-id -> file-name|>\endcsname\relax \else
      \:HDir \:LikeRef{|<auto file tag|>#1|<Tag/Ref: file-id -> file-name|>}%
\fi >>>



\<get html file name\><<<
\:LikeRef{|<auto file tag|>#1|<Tag/Ref: file-id -> file-name|>}>>>


The following alternative can act as basis for retrieving all
the file names, instead of just the first one.

\<COMMENT\><<<
\def\get:hfile#1{%
   \expandafter\ifx\csname
                cw:|<HTag tag|>#1\endcsname\relax
      \get@hfile{#1}%
   \else
     \expandafter\expandafter\expandafter\get::hfile
          \csname  cw:|<HTag tag|>#1\endcsname,\empty end:Links,/
   \fi  }
\def\end:Links#1//{}

\def\get::hfile#1,#2//{%
\expandafter \ifx \csname #1\endcsname\end:Links
 \expandafter\end:Links
\else
  \ifnum \file:id=0#1 \FileName\else
       \expandafter\ifx
   \csname  cw:|<auto file tag|>#1%
                 \empty F-\endcsname\relax \else
      \:LikeRef{|<auto file tag|>#1\empty F-}%
\fi %
   \fi
\expandafter\get::hfileA
\fi #2//}

\def\get::hfileA#1{%
\ifx #1\space
   \expandafter\get::hfile\else
\ifx #1\empty \else \space\fi%
 \expandafter\get::hfile\expandafter#1\fi}
>>>




\Section{Other}


Dviht produce new line only after nonempty lines. The following
can be used to force non empty lines.

\<html common to TeX4ht and protex\><<<
|%\def\HComment#1{\def\html:src{#1}}
\HComment{} |%      \let\html:src=\empty
\catcode`\#=12   \def\:sharp{#}   \catcode`\#=6
>>>




\<html protex\><<<
\ifx \html:par\:UnDef \def\html:par{\HCode{<p>}} \fi
\def\html:invisible{<!--x-->\string&\#x00A0;}
>>>

The \`'<!--x-->&nbsp;' is needed for blank lines in \`'<PRE>'
within netscape. Opposite order of comment and non-breaking-space will
not work, and the same if one of the entities is sliminates. On the
other other, mosaic show the nobreaking space as strin. Well, ...

\<nohtml common to TeX4ht and protex\><<<
|%\def\HComment#1{}|%
>>>

\`'\HComment' seems to have no real value, unless we want to
trace the input for debugging.


\SubSection{Tags}

\FreeCode\<tag of Tag\>  % to ensure agains duplication

\<tag of Tag\><<<
 cw:>>>

\<HTag tag\><<<
|<auto tag|>Q\aXrefFile >>>



\<Tag/Ref: file-id -> file-name\><<<
\empty F->>>

\<auto tag\><<<
)>>>

\<auto file tag\><<<
|<auto tag|>F\aXrefFile >>>


\FreeCode\<par del\>

\<par del\><<<
!*?: >>>
