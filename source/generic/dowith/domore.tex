\ProvidesFile{domore.tex}[2015/11/14 documenting domore files]
\title{\pkg{domore}\\---\\Some More Commands for Lists of Tokens}
% \listfiles
{ \RequirePackage{makedoc} \ProcessLineMessage{}
  \renewcommand\mdSectionLevelOne  {\string\subsection}
  \renewcommand\mdSectionLevelTwo  {\string\subsubsection}
  \MainDocParser{\SectionLevelTwoParseInput}
  \HeaderLines{17} 
  \MakeSingleDoc{domore.sty}
  \MakeSingleDoc{reversdo.sty} 
  \MakeSingleDoc{domodes.sty} 
}
\documentclass[fleqn]{article}%% TODO paper dimensions!?
\ProvidesFile{makedoc.cfg}[{2013/03/25 documentation settings}] 
%%
\author{Uwe L\"uck\thanks{%
        \url{http://contact-ednotes.sty.de.vu}}}
%%
%% 'hyperref':
\RequirePackage{ifpdf}
\usepackage[%
  \ifpdf
%     bookmarks=false,                  %% 2010/12/22
%     bookmarksnumbered,
    bookmarksopen,                      %% 2011/01/24!?
    bookmarksopenlevel=2,               %% 2011/01/23
%     pdfpagemode=UseNone,
%     pdfstartpage=10,
    pdfstartview=FitH,                  %% 2012/11/26 again
%     pdfstartview=0 0 100,             %% 2011/08/22
%     pdfstartview={XYZ null null 1},   %% 2011/08/25
%     pdfstartview={XYZ null null null},%% 2011/08/25
%     pdfstartview={XYZ null null .5},    %% 2011/08/26
%     pdffitwindow=true,          %% 2011/08/22
    citebordercolor={ .6 1    .6},
    filebordercolor={1    .6 1},
    linkbordercolor={1    .9  .7},
     urlbordercolor={ .7 1   1},   %% playing 2011/01/24
  \else
    draft
  \fi
]{hyperref}
\hypersetup{% 
    pdfauthor={Uwe L\374ck}% 
}
%% metadata, |\MDkeywords{<text>}|, |\MDkeywordsstring|:
%% %% 2011/08/22:
\makeatletter
  \newcommand*{\MDkeywords}[1]{%
    \gdef\MDkeywordsstring{#1}%
    \hypersetup{pdfkeywords=\MDkeywordsstring}%% TODO!?
  }
  \@onlypreamble\MDkeywords
%% |\MDaddtoabstract{<par-head>}|, `:' added:
  \newcommand*{\MDaddtoabstract}[1]{%           %% 2012/05/10
    \par\smallskip\noindent
    \strong{#1:}\quad\ignorespaces}
%% \pagebreak[2]
%% |\printMDkeywords|:
  \newcommand*{\printMDkeywords}{%
    \MDaddtoabstract{Keywords}%
    \MDkeywordsstring 
%     \global\let\MDkeywordsstring\relax    %% `%' 2012/11/12
  }
%% The previous definitions mainly are useful with a variant 
%% |\begin{MDabstract}| of \LaTeX's `{abstract}' environment:
  \newenvironment{MDabstract}
                 {\abstract\noindent
                  \hspace{1sp}%% for niceverb
                  \ignorespaces}
                 {\@ifundefined{MDkeywordsstring}%
                               {}%
                               {\printMDkeywords}%
                  \global\let\MDabstract\relax    %% 2012/11/12
                  \global\let\endMDabstract\relax %% 2012/11/12
                  \endabstract}
%% |\[MD]docnewline| 2012/11/12 from `readprov.tex':
  \newcommand*{\MDdocnewline}{\leavevmode\@normalcr[\topsep]}
%% <- `\leavevmode' for use with `\paragraph'.
%%    Sometimes needs to be preceded by a space.
%% 
%% |\MDfinaldatechecks[<tex-script>]| with \ctanpkgref{filedate}:
  \newcommand*{\MDfinaldatechecks}[1][fdatechk]{%
    \AtEndDocument{%
%       \clearpage %% 2013/03/25 no avail -- with `filedate'!
      \def\@pkgextension{sty}%
      \def\NeedsTeXFormat##1[##2]{}%
      \noNiceVerb                       %% 2013/03/22
      \input{#1}%
    }}
  \@onlypreamble\MDfinaldatechecks
\makeatother
%% Use other packages:
\RequirePackage{niceverb}[2011/01/24] 
\RequirePackage{readprov}               %% 2010/12/08
\RequirePackage{hypertoc}               %% 2011/01/23
\RequirePackage{texlinks}               %% 2011/01/24
\RequirePackage{relsize}                %% 2011/06/27
\RequirePackage{color}                  %% 2011/08/06
\RequirePackage{lmodern}                %% 2012/10/29
\RequirePackage{filedate}               %% 2012/11/12
\RequirePackage{filesdo}                %% 2013/03/22 
%% \pagebreak[3]
%% Logical markup:\qquad  |\strong{<chars>}|, |\meta{<chars>}|, 
%% |\acro{<chars>}|, |\pkg{<chars>}|, 
%% |\code{<chars>}|, |\file{<chars>}|:{\sloppy\par}
\makeatletter
  \def\do#1#2{\@ifdefinable#1{\let#1#2}}%% 2012/07/13
  \do\strong\textbf \do\file\texttt \do\acro\textsmaller 
  %% <- wrong tests before 2012/07/13
  \do\meta\textit   \do \pkg\textsf \do\code\texttt
  \ifpdf
    \pdfstringdefDisableCommands{%
        \let\acro\textrm 
        \let\file\textrm                            %% 2011/11/09
        \let\code\textrm                            %% 2011/11/20
        \let\pkg \textrm                            %% 2012/03/23
    }
  \fi
  %% TODO 2011/07/22 -> `htlogml.sty'
\makeatother
%% |\qtdcode{<text>}|: 2012/10/24:
    \newcommand*{\qtdcode}[1]{`\code{#1}'} 
%% |\pkgtitle{<package-name>}{<caption>}| 
\newcommand*{\pkgtitle}[2]{%            %% 2012/07/13
    \global\let\pkgtitle\relax
    \pkg{\huge #1}\\---\\#2\thanks{This 
       document describes version 
       \textcolor{blue}{\UseVersionOf{\jobname.sty}} 
       of \textsf{\jobname.sty} as of \UseDateOf{\jobname.sty}.}}
%% TODO: %% |\TODO| bad with `mdoccorr.cfg'
\newcommand*{\TODO}{\textcolor{blue}{\acro{TODO}}}  %% 2012/11/06
%% `\MDsampleinput[{<file>}' was added 2012/11/06. 
%% Problems with `myfilist.tex' were due to 'parskip.sty'
%% there. On 2012/11/12, we change the former simple macro to a 
%% much more complex
%% |\MDsamplecodeinput[<add-hfuss>]{<file>}| 
\newcommand*{\MDsamplecodeinput}[2][]{%
    \begingroup
        \vskip\bigskipamount \hrule
        \nobreak\vskip-\parskip 
%         \nobreak\vskip\medskipamount
%% Previous mistake (same below) due to manual change 
%% of `\topsep' in the file `myfilist.tex' (2012/11/30).
        \ifx\\#1\\\else
            \hfuzz=\textwidth \advance\hfuzz#1\relax
        \fi
        \noNiceVerb \verbatiminput{#2}%
%         \nobreak\vskip\medskipamount 
        \hrule \vskip-\parskip 
        \bigskip %%% \bigbreak
%% `\bigbreak' made much larger space in `myfilist.tex'.
    \endgroup
}
%% |\ctanpkgdref{<pkg-id>}| adds the printed link to 
%% `ctan.org/pkg' as a footnote. There is a little space 
%% for coloured link borders:
\newcommand*{\ctanpkgdref}[1]{%
    \ctanpkgref{#1}\,\urlfoot{CtanPkgRef}{#1}}
\errorcontextlines=4
\pagestyle{headings}

\endinput 
 %% shared formatting settings
\MDkeywords{Macro programming, programming structures, 
            loops, list macros}
\newcommand*{\headersec}{%
    \subsection{Package File Header---\pkg{plainpkg} and Legalese}}
\usepackage{catchdq} \catchdqs                  %% 2012/11/19
\AddToMacro{\mdStartPackageCode}{\MakeOther\"}  %% 2012/11/19
       %% something moved to makedoc.cfg from here 2013/03/21
\newif\ifmultmore                               %% 2012/11/19
%%% \multmoretrue %% 2015/11/14 TODO domodes/revers... with plainpkg!
\MDfinaldatechecks
\sloppy
\begin{document}
\maketitle
\begin{MDabstract}
\ifmultmore
  This document describes packages that do similar things 
  as the 'dowith' package or extend it.
\else
  'domore.sty' is a package that enhances 'dowith.sty''s 
  `\DoWith' (without assignments)
  and `\setdo' commands for applying something 
  (e.g., `\do') to each item of an "arglist".
  Each item may consist of two or more arguments for a macro, 
  and some "separator" material may be inserted between the 
  applications to items. A practiced application has been 
  generating inline lists of links that are separated 
  by \qtdcode{~\string|~}.
  'domore.sty' is (to some extent) format-independent 
  by means of the \ctanpkgref{plainpkg} and 'stacklet' packages. 
\fi
% \MDaddtoabstract{Required Packages}
% \ctanpkgref{plainpkg}, 'stacklet'
\MDaddtoabstract{Related Packages} cf. `dowith.pdf'.
\end{MDabstract}
\tableofcontents

\edef\domore{\ifmultmore\noexpand\MetaVar domore>\else domore\fi}
% \section{Shared Features of Usage}
\ifmultmore
  \section{Shared Features of Usage}
  All the packages described in this document 
  are "\pkg{plainpkg} packages" 
\else
  \section{Making 'domore.sty' Available}
  The 'domore' package is a "\pkg{plainpkg} package"
\fi
in the sense of the 
\ctanpkgdref{plainpkg}
documentation that exhibits details of what is summarized here. 
Therefore:
\begin{itemize}
  \item \ifmultmore All of them require \else It is required \fi
        that \TeX\ finds `plainpkg.tex' 
        as well as `stackrel.sty' from the 
        \ctanpkgdref{catcodes} bundle.
  \item In order to load \file{\domore.sty}%%%
        \ifmultmore\ (where \domore\ is `domore', `domodes', or `reversdo')\fi, 
        type
    \begin{description}
        \ifmultmore\cmdboxitem|\usepackage{<domore>}|
               \else\cmdboxitem|\usepackage{domore}|      \fi
        \ within a \LaTeX\ document 
        preamble, \ 
        \ifmultmore\cmdboxitem|\RequirePackage{<domore>}| 
               \else\cmdboxitem|\RequirePackage{domore}|  \fi
        \ in a "\pkg{plainpkg} package", or \ 
        \ifmultmore\cmdboxitem|\input <domore>.sty| 
               \else\cmdboxitem|\input domore.sty|        \fi
    \end{description}
        $\dots$ \ or perhaps `\input{<domore>.sty}'? 
\end{itemize}

\section{Remark on the Style of Code Documentation}
In `dowith.pdf', the documentation of the 'dowith' package,
in the section about "\TeX's tokens," I have tried to explain 
the difference between \TeX\ input code and the tokens that 
arise from it. In order to really understand what packages 
in the 'dowith' bundle do, one should think of the behaviour 
of the \emph{tokens}. For convenience however, I may rather 
fall back into the usual confusion here. After reading the 
documentation `dowith.pdf' of `dowith.sty', you may be able 
to guess successfully what is meant below.

% \section{Overview of Packages Described in \file{\jobname.pdf}}
\ifmultmore
  \section{Overview of Packages Described in \file{\jobname.pdf}}
  \label{sec:over}
  The present document describes the packages provided by the 
  \ctanpkgref{dowith} bundle apart from 'dowith.sty' itself 
  for applying something to each item from some list. 
  \begin{enumerate}
    \item 
\else
  \section{Overview of Commands}
\fi
        \strong{\pkg{domore.sty}} provides a more powerful version 
        of 'dowith.sty''s 
        \[|\DoWith{<repeat>}<args>\StopDoing|\]
        acting on an "arglist" <args>
        where <repeat> may be more complex than with 'dowith.sty'. 
        Based on this, another variant |\DoWithMore| of `\DoWith' 
        is provided where <repeat> may be a macro with more than 
        one argument. With \LaTeX\ e.g., <repeat> may be |\do| 
        defined by \[|\setdo[<digit>]<opt>{<replace>}|\] 
        an extension of 'dowith.sty''s `\setdo'.
        Further, 
        \[|\DoSeparateWith{<repeat>}{<sep>}<args>\StopDoing|\]
        inserts "separator material" <sep> between the applications 
        of <repeat> to the items in <args>. Another 
        |\DoSeparateWithMore| combines the features of the two 
        previous macros. I have used this with 'blog.sty' from 
        the \ctanpkgref{morehype} bundle for generating inline 
        lists of links, separated by something like \qtdcode{~\string|~}, 
        in \acro{HTML} documents. 

        As auxiliaries, variants |\@firstsecondoftwo| and 
        |\@secondfirstoftwo| of \LaTeX's `\@firstofone' are introduced.
\ifmultmore 
  \end{enumerate}

  \section{'domore.sty'}
  An overview of what 'domore.sty' provides has been given in 
  Section~\ref{sec:over}. 
  For details, see the comments to the package's code below.
\else

  For details, see the comments to the package's code below.
  \section{Contents of 'domore.sty'}
\fi
\headersec
\ProvidesFile{domore.tex}[2015/11/14 documenting domore files]
\title{\pkg{domore}\\---\\Some More Commands for Lists of Tokens}
% \listfiles
{ \RequirePackage{makedoc} \ProcessLineMessage{}
  \renewcommand\mdSectionLevelOne  {\string\subsection}
  \renewcommand\mdSectionLevelTwo  {\string\subsubsection}
  \MainDocParser{\SectionLevelTwoParseInput}
  \HeaderLines{17} 
  \MakeSingleDoc{domore.sty}
  \MakeSingleDoc{reversdo.sty} 
  \MakeSingleDoc{domodes.sty} 
}
\documentclass[fleqn]{article}%% TODO paper dimensions!?
\ProvidesFile{makedoc.cfg}[{2013/03/25 documentation settings}] 
%%
\author{Uwe L\"uck\thanks{%
        \url{http://contact-ednotes.sty.de.vu}}}
%%
%% 'hyperref':
\RequirePackage{ifpdf}
\usepackage[%
  \ifpdf
%     bookmarks=false,                  %% 2010/12/22
%     bookmarksnumbered,
    bookmarksopen,                      %% 2011/01/24!?
    bookmarksopenlevel=2,               %% 2011/01/23
%     pdfpagemode=UseNone,
%     pdfstartpage=10,
    pdfstartview=FitH,                  %% 2012/11/26 again
%     pdfstartview=0 0 100,             %% 2011/08/22
%     pdfstartview={XYZ null null 1},   %% 2011/08/25
%     pdfstartview={XYZ null null null},%% 2011/08/25
%     pdfstartview={XYZ null null .5},    %% 2011/08/26
%     pdffitwindow=true,          %% 2011/08/22
    citebordercolor={ .6 1    .6},
    filebordercolor={1    .6 1},
    linkbordercolor={1    .9  .7},
     urlbordercolor={ .7 1   1},   %% playing 2011/01/24
  \else
    draft
  \fi
]{hyperref}
\hypersetup{% 
    pdfauthor={Uwe L\374ck}% 
}
%% metadata, |\MDkeywords{<text>}|, |\MDkeywordsstring|:
%% %% 2011/08/22:
\makeatletter
  \newcommand*{\MDkeywords}[1]{%
    \gdef\MDkeywordsstring{#1}%
    \hypersetup{pdfkeywords=\MDkeywordsstring}%% TODO!?
  }
  \@onlypreamble\MDkeywords
%% |\MDaddtoabstract{<par-head>}|, `:' added:
  \newcommand*{\MDaddtoabstract}[1]{%           %% 2012/05/10
    \par\smallskip\noindent
    \strong{#1:}\quad\ignorespaces}
%% \pagebreak[2]
%% |\printMDkeywords|:
  \newcommand*{\printMDkeywords}{%
    \MDaddtoabstract{Keywords}%
    \MDkeywordsstring 
%     \global\let\MDkeywordsstring\relax    %% `%' 2012/11/12
  }
%% The previous definitions mainly are useful with a variant 
%% |\begin{MDabstract}| of \LaTeX's `{abstract}' environment:
  \newenvironment{MDabstract}
                 {\abstract\noindent
                  \hspace{1sp}%% for niceverb
                  \ignorespaces}
                 {\@ifundefined{MDkeywordsstring}%
                               {}%
                               {\printMDkeywords}%
                  \global\let\MDabstract\relax    %% 2012/11/12
                  \global\let\endMDabstract\relax %% 2012/11/12
                  \endabstract}
%% |\[MD]docnewline| 2012/11/12 from `readprov.tex':
  \newcommand*{\MDdocnewline}{\leavevmode\@normalcr[\topsep]}
%% <- `\leavevmode' for use with `\paragraph'.
%%    Sometimes needs to be preceded by a space.
%% 
%% |\MDfinaldatechecks[<tex-script>]| with \ctanpkgref{filedate}:
  \newcommand*{\MDfinaldatechecks}[1][fdatechk]{%
    \AtEndDocument{%
%       \clearpage %% 2013/03/25 no avail -- with `filedate'!
      \def\@pkgextension{sty}%
      \def\NeedsTeXFormat##1[##2]{}%
      \noNiceVerb                       %% 2013/03/22
      \input{#1}%
    }}
  \@onlypreamble\MDfinaldatechecks
\makeatother
%% Use other packages:
\RequirePackage{niceverb}[2011/01/24] 
\RequirePackage{readprov}               %% 2010/12/08
\RequirePackage{hypertoc}               %% 2011/01/23
\RequirePackage{texlinks}               %% 2011/01/24
\RequirePackage{relsize}                %% 2011/06/27
\RequirePackage{color}                  %% 2011/08/06
\RequirePackage{lmodern}                %% 2012/10/29
\RequirePackage{filedate}               %% 2012/11/12
\RequirePackage{filesdo}                %% 2013/03/22 
%% \pagebreak[3]
%% Logical markup:\qquad  |\strong{<chars>}|, |\meta{<chars>}|, 
%% |\acro{<chars>}|, |\pkg{<chars>}|, 
%% |\code{<chars>}|, |\file{<chars>}|:{\sloppy\par}
\makeatletter
  \def\do#1#2{\@ifdefinable#1{\let#1#2}}%% 2012/07/13
  \do\strong\textbf \do\file\texttt \do\acro\textsmaller 
  %% <- wrong tests before 2012/07/13
  \do\meta\textit   \do \pkg\textsf \do\code\texttt
  \ifpdf
    \pdfstringdefDisableCommands{%
        \let\acro\textrm 
        \let\file\textrm                            %% 2011/11/09
        \let\code\textrm                            %% 2011/11/20
        \let\pkg \textrm                            %% 2012/03/23
    }
  \fi
  %% TODO 2011/07/22 -> `htlogml.sty'
\makeatother
%% |\qtdcode{<text>}|: 2012/10/24:
    \newcommand*{\qtdcode}[1]{`\code{#1}'} 
%% |\pkgtitle{<package-name>}{<caption>}| 
\newcommand*{\pkgtitle}[2]{%            %% 2012/07/13
    \global\let\pkgtitle\relax
    \pkg{\huge #1}\\---\\#2\thanks{This 
       document describes version 
       \textcolor{blue}{\UseVersionOf{\jobname.sty}} 
       of \textsf{\jobname.sty} as of \UseDateOf{\jobname.sty}.}}
%% TODO: %% |\TODO| bad with `mdoccorr.cfg'
\newcommand*{\TODO}{\textcolor{blue}{\acro{TODO}}}  %% 2012/11/06
%% `\MDsampleinput[{<file>}' was added 2012/11/06. 
%% Problems with `myfilist.tex' were due to 'parskip.sty'
%% there. On 2012/11/12, we change the former simple macro to a 
%% much more complex
%% |\MDsamplecodeinput[<add-hfuss>]{<file>}| 
\newcommand*{\MDsamplecodeinput}[2][]{%
    \begingroup
        \vskip\bigskipamount \hrule
        \nobreak\vskip-\parskip 
%         \nobreak\vskip\medskipamount
%% Previous mistake (same below) due to manual change 
%% of `\topsep' in the file `myfilist.tex' (2012/11/30).
        \ifx\\#1\\\else
            \hfuzz=\textwidth \advance\hfuzz#1\relax
        \fi
        \noNiceVerb \verbatiminput{#2}%
%         \nobreak\vskip\medskipamount 
        \hrule \vskip-\parskip 
        \bigskip %%% \bigbreak
%% `\bigbreak' made much larger space in `myfilist.tex'.
    \endgroup
}
%% |\ctanpkgdref{<pkg-id>}| adds the printed link to 
%% `ctan.org/pkg' as a footnote. There is a little space 
%% for coloured link borders:
\newcommand*{\ctanpkgdref}[1]{%
    \ctanpkgref{#1}\,\urlfoot{CtanPkgRef}{#1}}
\errorcontextlines=4
\pagestyle{headings}

\endinput 
 %% shared formatting settings
\MDkeywords{Macro programming, programming structures, 
            loops, list macros}
\newcommand*{\headersec}{%
    \subsection{Package File Header---\pkg{plainpkg} and Legalese}}
\usepackage{catchdq} \catchdqs                  %% 2012/11/19
\AddToMacro{\mdStartPackageCode}{\MakeOther\"}  %% 2012/11/19
       %% something moved to makedoc.cfg from here 2013/03/21
\newif\ifmultmore                               %% 2012/11/19
%%% \multmoretrue %% 2015/11/14 TODO domodes/revers... with plainpkg!
\MDfinaldatechecks
\sloppy
\begin{document}
\maketitle
\begin{MDabstract}
\ifmultmore
  This document describes packages that do similar things 
  as the 'dowith' package or extend it.
\else
  'domore.sty' is a package that enhances 'dowith.sty''s 
  `\DoWith' (without assignments)
  and `\setdo' commands for applying something 
  (e.g., `\do') to each item of an "arglist".
  Each item may consist of two or more arguments for a macro, 
  and some "separator" material may be inserted between the 
  applications to items. A practiced application has been 
  generating inline lists of links that are separated 
  by \qtdcode{~\string|~}.
  'domore.sty' is (to some extent) format-independent 
  by means of the \ctanpkgref{plainpkg} and 'stacklet' packages. 
\fi
% \MDaddtoabstract{Required Packages}
% \ctanpkgref{plainpkg}, 'stacklet'
\MDaddtoabstract{Related Packages} cf. `dowith.pdf'.
\end{MDabstract}
\tableofcontents

\edef\domore{\ifmultmore\noexpand\MetaVar domore>\else domore\fi}
% \section{Shared Features of Usage}
\ifmultmore
  \section{Shared Features of Usage}
  All the packages described in this document 
  are "\pkg{plainpkg} packages" 
\else
  \section{Making 'domore.sty' Available}
  The 'domore' package is a "\pkg{plainpkg} package"
\fi
in the sense of the 
\ctanpkgdref{plainpkg}
documentation that exhibits details of what is summarized here. 
Therefore:
\begin{itemize}
  \item \ifmultmore All of them require \else It is required \fi
        that \TeX\ finds `plainpkg.tex' 
        as well as `stackrel.sty' from the 
        \ctanpkgdref{catcodes} bundle.
  \item In order to load \file{\domore.sty}%%%
        \ifmultmore\ (where \domore\ is `domore', `domodes', or `reversdo')\fi, 
        type
    \begin{description}
        \ifmultmore\cmdboxitem|\usepackage{<domore>}|
               \else\cmdboxitem|\usepackage{domore}|      \fi
        \ within a \LaTeX\ document 
        preamble, \ 
        \ifmultmore\cmdboxitem|\RequirePackage{<domore>}| 
               \else\cmdboxitem|\RequirePackage{domore}|  \fi
        \ in a "\pkg{plainpkg} package", or \ 
        \ifmultmore\cmdboxitem|\input <domore>.sty| 
               \else\cmdboxitem|\input domore.sty|        \fi
    \end{description}
        $\dots$ \ or perhaps `\input{<domore>.sty}'? 
\end{itemize}

\section{Remark on the Style of Code Documentation}
In `dowith.pdf', the documentation of the 'dowith' package,
in the section about "\TeX's tokens," I have tried to explain 
the difference between \TeX\ input code and the tokens that 
arise from it. In order to really understand what packages 
in the 'dowith' bundle do, one should think of the behaviour 
of the \emph{tokens}. For convenience however, I may rather 
fall back into the usual confusion here. After reading the 
documentation `dowith.pdf' of `dowith.sty', you may be able 
to guess successfully what is meant below.

% \section{Overview of Packages Described in \file{\jobname.pdf}}
\ifmultmore
  \section{Overview of Packages Described in \file{\jobname.pdf}}
  \label{sec:over}
  The present document describes the packages provided by the 
  \ctanpkgref{dowith} bundle apart from 'dowith.sty' itself 
  for applying something to each item from some list. 
  \begin{enumerate}
    \item 
\else
  \section{Overview of Commands}
\fi
        \strong{\pkg{domore.sty}} provides a more powerful version 
        of 'dowith.sty''s 
        \[|\DoWith{<repeat>}<args>\StopDoing|\]
        acting on an "arglist" <args>
        where <repeat> may be more complex than with 'dowith.sty'. 
        Based on this, another variant |\DoWithMore| of `\DoWith' 
        is provided where <repeat> may be a macro with more than 
        one argument. With \LaTeX\ e.g., <repeat> may be |\do| 
        defined by \[|\setdo[<digit>]<opt>{<replace>}|\] 
        an extension of 'dowith.sty''s `\setdo'.
        Further, 
        \[|\DoSeparateWith{<repeat>}{<sep>}<args>\StopDoing|\]
        inserts "separator material" <sep> between the applications 
        of <repeat> to the items in <args>. Another 
        |\DoSeparateWithMore| combines the features of the two 
        previous macros. I have used this with 'blog.sty' from 
        the \ctanpkgref{morehype} bundle for generating inline 
        lists of links, separated by something like \qtdcode{~\string|~}, 
        in \acro{HTML} documents. 

        As auxiliaries, variants |\@firstsecondoftwo| and 
        |\@secondfirstoftwo| of \LaTeX's `\@firstofone' are introduced.
\ifmultmore 
  \end{enumerate}

  \section{'domore.sty'}
  An overview of what 'domore.sty' provides has been given in 
  Section~\ref{sec:over}. 
  For details, see the comments to the package's code below.
\else

  For details, see the comments to the package's code below.
  \section{Contents of 'domore.sty'}
\fi
\headersec
\ProvidesFile{domore.tex}[2015/11/14 documenting domore files]
\title{\pkg{domore}\\---\\Some More Commands for Lists of Tokens}
% \listfiles
{ \RequirePackage{makedoc} \ProcessLineMessage{}
  \renewcommand\mdSectionLevelOne  {\string\subsection}
  \renewcommand\mdSectionLevelTwo  {\string\subsubsection}
  \MainDocParser{\SectionLevelTwoParseInput}
  \HeaderLines{17} 
  \MakeSingleDoc{domore.sty}
  \MakeSingleDoc{reversdo.sty} 
  \MakeSingleDoc{domodes.sty} 
}
\documentclass[fleqn]{article}%% TODO paper dimensions!?
\ProvidesFile{makedoc.cfg}[{2013/03/25 documentation settings}] 
%%
\author{Uwe L\"uck\thanks{%
        \url{http://contact-ednotes.sty.de.vu}}}
%%
%% 'hyperref':
\RequirePackage{ifpdf}
\usepackage[%
  \ifpdf
%     bookmarks=false,                  %% 2010/12/22
%     bookmarksnumbered,
    bookmarksopen,                      %% 2011/01/24!?
    bookmarksopenlevel=2,               %% 2011/01/23
%     pdfpagemode=UseNone,
%     pdfstartpage=10,
    pdfstartview=FitH,                  %% 2012/11/26 again
%     pdfstartview=0 0 100,             %% 2011/08/22
%     pdfstartview={XYZ null null 1},   %% 2011/08/25
%     pdfstartview={XYZ null null null},%% 2011/08/25
%     pdfstartview={XYZ null null .5},    %% 2011/08/26
%     pdffitwindow=true,          %% 2011/08/22
    citebordercolor={ .6 1    .6},
    filebordercolor={1    .6 1},
    linkbordercolor={1    .9  .7},
     urlbordercolor={ .7 1   1},   %% playing 2011/01/24
  \else
    draft
  \fi
]{hyperref}
\hypersetup{% 
    pdfauthor={Uwe L\374ck}% 
}
%% metadata, |\MDkeywords{<text>}|, |\MDkeywordsstring|:
%% %% 2011/08/22:
\makeatletter
  \newcommand*{\MDkeywords}[1]{%
    \gdef\MDkeywordsstring{#1}%
    \hypersetup{pdfkeywords=\MDkeywordsstring}%% TODO!?
  }
  \@onlypreamble\MDkeywords
%% |\MDaddtoabstract{<par-head>}|, `:' added:
  \newcommand*{\MDaddtoabstract}[1]{%           %% 2012/05/10
    \par\smallskip\noindent
    \strong{#1:}\quad\ignorespaces}
%% \pagebreak[2]
%% |\printMDkeywords|:
  \newcommand*{\printMDkeywords}{%
    \MDaddtoabstract{Keywords}%
    \MDkeywordsstring 
%     \global\let\MDkeywordsstring\relax    %% `%' 2012/11/12
  }
%% The previous definitions mainly are useful with a variant 
%% |\begin{MDabstract}| of \LaTeX's `{abstract}' environment:
  \newenvironment{MDabstract}
                 {\abstract\noindent
                  \hspace{1sp}%% for niceverb
                  \ignorespaces}
                 {\@ifundefined{MDkeywordsstring}%
                               {}%
                               {\printMDkeywords}%
                  \global\let\MDabstract\relax    %% 2012/11/12
                  \global\let\endMDabstract\relax %% 2012/11/12
                  \endabstract}
%% |\[MD]docnewline| 2012/11/12 from `readprov.tex':
  \newcommand*{\MDdocnewline}{\leavevmode\@normalcr[\topsep]}
%% <- `\leavevmode' for use with `\paragraph'.
%%    Sometimes needs to be preceded by a space.
%% 
%% |\MDfinaldatechecks[<tex-script>]| with \ctanpkgref{filedate}:
  \newcommand*{\MDfinaldatechecks}[1][fdatechk]{%
    \AtEndDocument{%
%       \clearpage %% 2013/03/25 no avail -- with `filedate'!
      \def\@pkgextension{sty}%
      \def\NeedsTeXFormat##1[##2]{}%
      \noNiceVerb                       %% 2013/03/22
      \input{#1}%
    }}
  \@onlypreamble\MDfinaldatechecks
\makeatother
%% Use other packages:
\RequirePackage{niceverb}[2011/01/24] 
\RequirePackage{readprov}               %% 2010/12/08
\RequirePackage{hypertoc}               %% 2011/01/23
\RequirePackage{texlinks}               %% 2011/01/24
\RequirePackage{relsize}                %% 2011/06/27
\RequirePackage{color}                  %% 2011/08/06
\RequirePackage{lmodern}                %% 2012/10/29
\RequirePackage{filedate}               %% 2012/11/12
\RequirePackage{filesdo}                %% 2013/03/22 
%% \pagebreak[3]
%% Logical markup:\qquad  |\strong{<chars>}|, |\meta{<chars>}|, 
%% |\acro{<chars>}|, |\pkg{<chars>}|, 
%% |\code{<chars>}|, |\file{<chars>}|:{\sloppy\par}
\makeatletter
  \def\do#1#2{\@ifdefinable#1{\let#1#2}}%% 2012/07/13
  \do\strong\textbf \do\file\texttt \do\acro\textsmaller 
  %% <- wrong tests before 2012/07/13
  \do\meta\textit   \do \pkg\textsf \do\code\texttt
  \ifpdf
    \pdfstringdefDisableCommands{%
        \let\acro\textrm 
        \let\file\textrm                            %% 2011/11/09
        \let\code\textrm                            %% 2011/11/20
        \let\pkg \textrm                            %% 2012/03/23
    }
  \fi
  %% TODO 2011/07/22 -> `htlogml.sty'
\makeatother
%% |\qtdcode{<text>}|: 2012/10/24:
    \newcommand*{\qtdcode}[1]{`\code{#1}'} 
%% |\pkgtitle{<package-name>}{<caption>}| 
\newcommand*{\pkgtitle}[2]{%            %% 2012/07/13
    \global\let\pkgtitle\relax
    \pkg{\huge #1}\\---\\#2\thanks{This 
       document describes version 
       \textcolor{blue}{\UseVersionOf{\jobname.sty}} 
       of \textsf{\jobname.sty} as of \UseDateOf{\jobname.sty}.}}
%% TODO: %% |\TODO| bad with `mdoccorr.cfg'
\newcommand*{\TODO}{\textcolor{blue}{\acro{TODO}}}  %% 2012/11/06
%% `\MDsampleinput[{<file>}' was added 2012/11/06. 
%% Problems with `myfilist.tex' were due to 'parskip.sty'
%% there. On 2012/11/12, we change the former simple macro to a 
%% much more complex
%% |\MDsamplecodeinput[<add-hfuss>]{<file>}| 
\newcommand*{\MDsamplecodeinput}[2][]{%
    \begingroup
        \vskip\bigskipamount \hrule
        \nobreak\vskip-\parskip 
%         \nobreak\vskip\medskipamount
%% Previous mistake (same below) due to manual change 
%% of `\topsep' in the file `myfilist.tex' (2012/11/30).
        \ifx\\#1\\\else
            \hfuzz=\textwidth \advance\hfuzz#1\relax
        \fi
        \noNiceVerb \verbatiminput{#2}%
%         \nobreak\vskip\medskipamount 
        \hrule \vskip-\parskip 
        \bigskip %%% \bigbreak
%% `\bigbreak' made much larger space in `myfilist.tex'.
    \endgroup
}
%% |\ctanpkgdref{<pkg-id>}| adds the printed link to 
%% `ctan.org/pkg' as a footnote. There is a little space 
%% for coloured link borders:
\newcommand*{\ctanpkgdref}[1]{%
    \ctanpkgref{#1}\,\urlfoot{CtanPkgRef}{#1}}
\errorcontextlines=4
\pagestyle{headings}

\endinput 
 %% shared formatting settings
\MDkeywords{Macro programming, programming structures, 
            loops, list macros}
\newcommand*{\headersec}{%
    \subsection{Package File Header---\pkg{plainpkg} and Legalese}}
\usepackage{catchdq} \catchdqs                  %% 2012/11/19
\AddToMacro{\mdStartPackageCode}{\MakeOther\"}  %% 2012/11/19
       %% something moved to makedoc.cfg from here 2013/03/21
\newif\ifmultmore                               %% 2012/11/19
%%% \multmoretrue %% 2015/11/14 TODO domodes/revers... with plainpkg!
\MDfinaldatechecks
\sloppy
\begin{document}
\maketitle
\begin{MDabstract}
\ifmultmore
  This document describes packages that do similar things 
  as the 'dowith' package or extend it.
\else
  'domore.sty' is a package that enhances 'dowith.sty''s 
  `\DoWith' (without assignments)
  and `\setdo' commands for applying something 
  (e.g., `\do') to each item of an "arglist".
  Each item may consist of two or more arguments for a macro, 
  and some "separator" material may be inserted between the 
  applications to items. A practiced application has been 
  generating inline lists of links that are separated 
  by \qtdcode{~\string|~}.
  'domore.sty' is (to some extent) format-independent 
  by means of the \ctanpkgref{plainpkg} and 'stacklet' packages. 
\fi
% \MDaddtoabstract{Required Packages}
% \ctanpkgref{plainpkg}, 'stacklet'
\MDaddtoabstract{Related Packages} cf. `dowith.pdf'.
\end{MDabstract}
\tableofcontents

\edef\domore{\ifmultmore\noexpand\MetaVar domore>\else domore\fi}
% \section{Shared Features of Usage}
\ifmultmore
  \section{Shared Features of Usage}
  All the packages described in this document 
  are "\pkg{plainpkg} packages" 
\else
  \section{Making 'domore.sty' Available}
  The 'domore' package is a "\pkg{plainpkg} package"
\fi
in the sense of the 
\ctanpkgdref{plainpkg}
documentation that exhibits details of what is summarized here. 
Therefore:
\begin{itemize}
  \item \ifmultmore All of them require \else It is required \fi
        that \TeX\ finds `plainpkg.tex' 
        as well as `stackrel.sty' from the 
        \ctanpkgdref{catcodes} bundle.
  \item In order to load \file{\domore.sty}%%%
        \ifmultmore\ (where \domore\ is `domore', `domodes', or `reversdo')\fi, 
        type
    \begin{description}
        \ifmultmore\cmdboxitem|\usepackage{<domore>}|
               \else\cmdboxitem|\usepackage{domore}|      \fi
        \ within a \LaTeX\ document 
        preamble, \ 
        \ifmultmore\cmdboxitem|\RequirePackage{<domore>}| 
               \else\cmdboxitem|\RequirePackage{domore}|  \fi
        \ in a "\pkg{plainpkg} package", or \ 
        \ifmultmore\cmdboxitem|\input <domore>.sty| 
               \else\cmdboxitem|\input domore.sty|        \fi
    \end{description}
        $\dots$ \ or perhaps `\input{<domore>.sty}'? 
\end{itemize}

\section{Remark on the Style of Code Documentation}
In `dowith.pdf', the documentation of the 'dowith' package,
in the section about "\TeX's tokens," I have tried to explain 
the difference between \TeX\ input code and the tokens that 
arise from it. In order to really understand what packages 
in the 'dowith' bundle do, one should think of the behaviour 
of the \emph{tokens}. For convenience however, I may rather 
fall back into the usual confusion here. After reading the 
documentation `dowith.pdf' of `dowith.sty', you may be able 
to guess successfully what is meant below.

% \section{Overview of Packages Described in \file{\jobname.pdf}}
\ifmultmore
  \section{Overview of Packages Described in \file{\jobname.pdf}}
  \label{sec:over}
  The present document describes the packages provided by the 
  \ctanpkgref{dowith} bundle apart from 'dowith.sty' itself 
  for applying something to each item from some list. 
  \begin{enumerate}
    \item 
\else
  \section{Overview of Commands}
\fi
        \strong{\pkg{domore.sty}} provides a more powerful version 
        of 'dowith.sty''s 
        \[|\DoWith{<repeat>}<args>\StopDoing|\]
        acting on an "arglist" <args>
        where <repeat> may be more complex than with 'dowith.sty'. 
        Based on this, another variant |\DoWithMore| of `\DoWith' 
        is provided where <repeat> may be a macro with more than 
        one argument. With \LaTeX\ e.g., <repeat> may be |\do| 
        defined by \[|\setdo[<digit>]<opt>{<replace>}|\] 
        an extension of 'dowith.sty''s `\setdo'.
        Further, 
        \[|\DoSeparateWith{<repeat>}{<sep>}<args>\StopDoing|\]
        inserts "separator material" <sep> between the applications 
        of <repeat> to the items in <args>. Another 
        |\DoSeparateWithMore| combines the features of the two 
        previous macros. I have used this with 'blog.sty' from 
        the \ctanpkgref{morehype} bundle for generating inline 
        lists of links, separated by something like \qtdcode{~\string|~}, 
        in \acro{HTML} documents. 

        As auxiliaries, variants |\@firstsecondoftwo| and 
        |\@secondfirstoftwo| of \LaTeX's `\@firstofone' are introduced.
\ifmultmore 
  \end{enumerate}

  \section{'domore.sty'}
  An overview of what 'domore.sty' provides has been given in 
  Section~\ref{sec:over}. 
  For details, see the comments to the package's code below.
\else

  For details, see the comments to the package's code below.
  \section{Contents of 'domore.sty'}
\fi
\headersec
\ProvidesFile{domore.tex}[2015/11/14 documenting domore files]
\title{\pkg{domore}\\---\\Some More Commands for Lists of Tokens}
% \listfiles
{ \RequirePackage{makedoc} \ProcessLineMessage{}
  \renewcommand\mdSectionLevelOne  {\string\subsection}
  \renewcommand\mdSectionLevelTwo  {\string\subsubsection}
  \MainDocParser{\SectionLevelTwoParseInput}
  \HeaderLines{17} 
  \MakeSingleDoc{domore.sty}
  \MakeSingleDoc{reversdo.sty} 
  \MakeSingleDoc{domodes.sty} 
}
\documentclass[fleqn]{article}%% TODO paper dimensions!?
\input{makedoc.cfg} %% shared formatting settings
\MDkeywords{Macro programming, programming structures, 
            loops, list macros}
\newcommand*{\headersec}{%
    \subsection{Package File Header---\pkg{plainpkg} and Legalese}}
\usepackage{catchdq} \catchdqs                  %% 2012/11/19
\AddToMacro{\mdStartPackageCode}{\MakeOther\"}  %% 2012/11/19
       %% something moved to makedoc.cfg from here 2013/03/21
\newif\ifmultmore                               %% 2012/11/19
%%% \multmoretrue %% 2015/11/14 TODO domodes/revers... with plainpkg!
\MDfinaldatechecks
\sloppy
\begin{document}
\maketitle
\begin{MDabstract}
\ifmultmore
  This document describes packages that do similar things 
  as the 'dowith' package or extend it.
\else
  'domore.sty' is a package that enhances 'dowith.sty''s 
  `\DoWith' (without assignments)
  and `\setdo' commands for applying something 
  (e.g., `\do') to each item of an "arglist".
  Each item may consist of two or more arguments for a macro, 
  and some "separator" material may be inserted between the 
  applications to items. A practiced application has been 
  generating inline lists of links that are separated 
  by \qtdcode{~\string|~}.
  'domore.sty' is (to some extent) format-independent 
  by means of the \ctanpkgref{plainpkg} and 'stacklet' packages. 
\fi
% \MDaddtoabstract{Required Packages}
% \ctanpkgref{plainpkg}, 'stacklet'
\MDaddtoabstract{Related Packages} cf. `dowith.pdf'.
\end{MDabstract}
\tableofcontents

\edef\domore{\ifmultmore\noexpand\MetaVar domore>\else domore\fi}
% \section{Shared Features of Usage}
\ifmultmore
  \section{Shared Features of Usage}
  All the packages described in this document 
  are "\pkg{plainpkg} packages" 
\else
  \section{Making 'domore.sty' Available}
  The 'domore' package is a "\pkg{plainpkg} package"
\fi
in the sense of the 
\ctanpkgdref{plainpkg}
documentation that exhibits details of what is summarized here. 
Therefore:
\begin{itemize}
  \item \ifmultmore All of them require \else It is required \fi
        that \TeX\ finds `plainpkg.tex' 
        as well as `stackrel.sty' from the 
        \ctanpkgdref{catcodes} bundle.
  \item In order to load \file{\domore.sty}%%%
        \ifmultmore\ (where \domore\ is `domore', `domodes', or `reversdo')\fi, 
        type
    \begin{description}
        \ifmultmore\cmdboxitem|\usepackage{<domore>}|
               \else\cmdboxitem|\usepackage{domore}|      \fi
        \ within a \LaTeX\ document 
        preamble, \ 
        \ifmultmore\cmdboxitem|\RequirePackage{<domore>}| 
               \else\cmdboxitem|\RequirePackage{domore}|  \fi
        \ in a "\pkg{plainpkg} package", or \ 
        \ifmultmore\cmdboxitem|\input <domore>.sty| 
               \else\cmdboxitem|\input domore.sty|        \fi
    \end{description}
        $\dots$ \ or perhaps `\input{<domore>.sty}'? 
\end{itemize}

\section{Remark on the Style of Code Documentation}
In `dowith.pdf', the documentation of the 'dowith' package,
in the section about "\TeX's tokens," I have tried to explain 
the difference between \TeX\ input code and the tokens that 
arise from it. In order to really understand what packages 
in the 'dowith' bundle do, one should think of the behaviour 
of the \emph{tokens}. For convenience however, I may rather 
fall back into the usual confusion here. After reading the 
documentation `dowith.pdf' of `dowith.sty', you may be able 
to guess successfully what is meant below.

% \section{Overview of Packages Described in \file{\jobname.pdf}}
\ifmultmore
  \section{Overview of Packages Described in \file{\jobname.pdf}}
  \label{sec:over}
  The present document describes the packages provided by the 
  \ctanpkgref{dowith} bundle apart from 'dowith.sty' itself 
  for applying something to each item from some list. 
  \begin{enumerate}
    \item 
\else
  \section{Overview of Commands}
\fi
        \strong{\pkg{domore.sty}} provides a more powerful version 
        of 'dowith.sty''s 
        \[|\DoWith{<repeat>}<args>\StopDoing|\]
        acting on an "arglist" <args>
        where <repeat> may be more complex than with 'dowith.sty'. 
        Based on this, another variant |\DoWithMore| of `\DoWith' 
        is provided where <repeat> may be a macro with more than 
        one argument. With \LaTeX\ e.g., <repeat> may be |\do| 
        defined by \[|\setdo[<digit>]<opt>{<replace>}|\] 
        an extension of 'dowith.sty''s `\setdo'.
        Further, 
        \[|\DoSeparateWith{<repeat>}{<sep>}<args>\StopDoing|\]
        inserts "separator material" <sep> between the applications 
        of <repeat> to the items in <args>. Another 
        |\DoSeparateWithMore| combines the features of the two 
        previous macros. I have used this with 'blog.sty' from 
        the \ctanpkgref{morehype} bundle for generating inline 
        lists of links, separated by something like \qtdcode{~\string|~}, 
        in \acro{HTML} documents. 

        As auxiliaries, variants |\@firstsecondoftwo| and 
        |\@secondfirstoftwo| of \LaTeX's `\@firstofone' are introduced.
\ifmultmore 
  \end{enumerate}

  \section{'domore.sty'}
  An overview of what 'domore.sty' provides has been given in 
  Section~\ref{sec:over}. 
  For details, see the comments to the package's code below.
\else

  For details, see the comments to the package's code below.
  \section{Contents of 'domore.sty'}
\fi
\headersec
\input{domore.doc}

\MakeOther\"
\ifmultmore
\section{'domodes.sty'}
See Section~\ref{sec:domodes-cmds} for the commands provided.
\headersec
\input{domodes.doc}

\section{'reversdo.sty'}
See Section~\ref{sec:reversdo-cmds} for the commands provided.
\headersec
\input{reversdo.doc}

\fi
\end{document}

VERSION HISTORY

2012/10/23    for v0.2   started
2012/11/05ff. for v0.3
2013/03/21f.  for v0.31  \MDfinaldatecheck


\MakeOther\"
\ifmultmore
\section{'domodes.sty'}
See Section~\ref{sec:domodes-cmds} for the commands provided.
\headersec
\input{domodes.doc}

\section{'reversdo.sty'}
See Section~\ref{sec:reversdo-cmds} for the commands provided.
\headersec
\input{reversdo.doc}

\fi
\end{document}

VERSION HISTORY

2012/10/23    for v0.2   started
2012/11/05ff. for v0.3
2013/03/21f.  for v0.31  \MDfinaldatecheck


\MakeOther\"
\ifmultmore
\section{'domodes.sty'}
See Section~\ref{sec:domodes-cmds} for the commands provided.
\headersec
\input{domodes.doc}

\section{'reversdo.sty'}
See Section~\ref{sec:reversdo-cmds} for the commands provided.
\headersec
\input{reversdo.doc}

\fi
\end{document}

VERSION HISTORY

2012/10/23    for v0.2   started
2012/11/05ff. for v0.3
2013/03/21f.  for v0.31  \MDfinaldatecheck


\MakeOther\"
\ifmultmore
\section{'domodes.sty'}
See Section~\ref{sec:domodes-cmds} for the commands provided.
\headersec
\input{domodes.doc}

\section{'reversdo.sty'}
See Section~\ref{sec:reversdo-cmds} for the commands provided.
\headersec
\input{reversdo.doc}

\fi
\end{document}

VERSION HISTORY

2012/10/23    for v0.2   started
2012/11/05ff. for v0.3
2013/03/21f.  for v0.31  \MDfinaldatecheck
