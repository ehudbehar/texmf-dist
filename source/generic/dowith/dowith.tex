\ProvidesFile{dowith.tex}[2015/11/14 documenting dowith.sty]
\title{%%%\kern-\baselineskip
       \textsf{\huge dowith.sty}\\---\\Apply Command to
%        Elements of Lists without Separators\,---\,%%% 2012/05/14
%        and without Iterator\thanks{This
       %% 2012/05/15, "in" 2012/05/18:
       Each Item \\ in a List of Arguments in ``\TeX's Gullet"\thanks{This
       document describes version
       \textcolor{blue}{\UseVersionOf{\jobname.sty}}
       of \textsf{\jobname.sty} as of \UseDateOf{\jobname.sty}.}}
{ \RequirePackage{makedoc} \ProcessLineMessage{}
  \renewcommand*\mdSectionLevelOne{\string\subsection}
  \renewcommand*\mdSectionLevelTwo{\string\subsubsection}
  \MakeJobDoc{18}%% 2012/11/05 for v0.3
  {\SectionLevelTwoParseInput}  }
% \RequirePackage[ir]{inputtrc}
\documentclass[fleqn]{article}%% TODO paper dimensions!?
\ProvidesFile{makedoc.cfg}[{2013/03/25 documentation settings}] 
%%
\author{Uwe L\"uck\thanks{%
        \url{http://contact-ednotes.sty.de.vu}}}
%%
%% 'hyperref':
\RequirePackage{ifpdf}
\usepackage[%
  \ifpdf
%     bookmarks=false,                  %% 2010/12/22
%     bookmarksnumbered,
    bookmarksopen,                      %% 2011/01/24!?
    bookmarksopenlevel=2,               %% 2011/01/23
%     pdfpagemode=UseNone,
%     pdfstartpage=10,
    pdfstartview=FitH,                  %% 2012/11/26 again
%     pdfstartview=0 0 100,             %% 2011/08/22
%     pdfstartview={XYZ null null 1},   %% 2011/08/25
%     pdfstartview={XYZ null null null},%% 2011/08/25
%     pdfstartview={XYZ null null .5},    %% 2011/08/26
%     pdffitwindow=true,          %% 2011/08/22
    citebordercolor={ .6 1    .6},
    filebordercolor={1    .6 1},
    linkbordercolor={1    .9  .7},
     urlbordercolor={ .7 1   1},   %% playing 2011/01/24
  \else
    draft
  \fi
]{hyperref}
\hypersetup{% 
    pdfauthor={Uwe L\374ck}% 
}
%% metadata, |\MDkeywords{<text>}|, |\MDkeywordsstring|:
%% %% 2011/08/22:
\makeatletter
  \newcommand*{\MDkeywords}[1]{%
    \gdef\MDkeywordsstring{#1}%
    \hypersetup{pdfkeywords=\MDkeywordsstring}%% TODO!?
  }
  \@onlypreamble\MDkeywords
%% |\MDaddtoabstract{<par-head>}|, `:' added:
  \newcommand*{\MDaddtoabstract}[1]{%           %% 2012/05/10
    \par\smallskip\noindent
    \strong{#1:}\quad\ignorespaces}
%% \pagebreak[2]
%% |\printMDkeywords|:
  \newcommand*{\printMDkeywords}{%
    \MDaddtoabstract{Keywords}%
    \MDkeywordsstring 
%     \global\let\MDkeywordsstring\relax    %% `%' 2012/11/12
  }
%% The previous definitions mainly are useful with a variant 
%% |\begin{MDabstract}| of \LaTeX's `{abstract}' environment:
  \newenvironment{MDabstract}
                 {\abstract\noindent
                  \hspace{1sp}%% for niceverb
                  \ignorespaces}
                 {\@ifundefined{MDkeywordsstring}%
                               {}%
                               {\printMDkeywords}%
                  \global\let\MDabstract\relax    %% 2012/11/12
                  \global\let\endMDabstract\relax %% 2012/11/12
                  \endabstract}
%% |\[MD]docnewline| 2012/11/12 from `readprov.tex':
  \newcommand*{\MDdocnewline}{\leavevmode\@normalcr[\topsep]}
%% <- `\leavevmode' for use with `\paragraph'.
%%    Sometimes needs to be preceded by a space.
%% 
%% |\MDfinaldatechecks[<tex-script>]| with \ctanpkgref{filedate}:
  \newcommand*{\MDfinaldatechecks}[1][fdatechk]{%
    \AtEndDocument{%
%       \clearpage %% 2013/03/25 no avail -- with `filedate'!
      \def\@pkgextension{sty}%
      \def\NeedsTeXFormat##1[##2]{}%
      \noNiceVerb                       %% 2013/03/22
      \input{#1}%
    }}
  \@onlypreamble\MDfinaldatechecks
\makeatother
%% Use other packages:
\RequirePackage{niceverb}[2011/01/24] 
\RequirePackage{readprov}               %% 2010/12/08
\RequirePackage{hypertoc}               %% 2011/01/23
\RequirePackage{texlinks}               %% 2011/01/24
\RequirePackage{relsize}                %% 2011/06/27
\RequirePackage{color}                  %% 2011/08/06
\RequirePackage{lmodern}                %% 2012/10/29
\RequirePackage{filedate}               %% 2012/11/12
\RequirePackage{filesdo}                %% 2013/03/22 
%% \pagebreak[3]
%% Logical markup:\qquad  |\strong{<chars>}|, |\meta{<chars>}|, 
%% |\acro{<chars>}|, |\pkg{<chars>}|, 
%% |\code{<chars>}|, |\file{<chars>}|:{\sloppy\par}
\makeatletter
  \def\do#1#2{\@ifdefinable#1{\let#1#2}}%% 2012/07/13
  \do\strong\textbf \do\file\texttt \do\acro\textsmaller 
  %% <- wrong tests before 2012/07/13
  \do\meta\textit   \do \pkg\textsf \do\code\texttt
  \ifpdf
    \pdfstringdefDisableCommands{%
        \let\acro\textrm 
        \let\file\textrm                            %% 2011/11/09
        \let\code\textrm                            %% 2011/11/20
        \let\pkg \textrm                            %% 2012/03/23
    }
  \fi
  %% TODO 2011/07/22 -> `htlogml.sty'
\makeatother
%% |\qtdcode{<text>}|: 2012/10/24:
    \newcommand*{\qtdcode}[1]{`\code{#1}'} 
%% |\pkgtitle{<package-name>}{<caption>}| 
\newcommand*{\pkgtitle}[2]{%            %% 2012/07/13
    \global\let\pkgtitle\relax
    \pkg{\huge #1}\\---\\#2\thanks{This 
       document describes version 
       \textcolor{blue}{\UseVersionOf{\jobname.sty}} 
       of \textsf{\jobname.sty} as of \UseDateOf{\jobname.sty}.}}
%% TODO: %% |\TODO| bad with `mdoccorr.cfg'
\newcommand*{\TODO}{\textcolor{blue}{\acro{TODO}}}  %% 2012/11/06
%% `\MDsampleinput[{<file>}' was added 2012/11/06. 
%% Problems with `myfilist.tex' were due to 'parskip.sty'
%% there. On 2012/11/12, we change the former simple macro to a 
%% much more complex
%% |\MDsamplecodeinput[<add-hfuss>]{<file>}| 
\newcommand*{\MDsamplecodeinput}[2][]{%
    \begingroup
        \vskip\bigskipamount \hrule
        \nobreak\vskip-\parskip 
%         \nobreak\vskip\medskipamount
%% Previous mistake (same below) due to manual change 
%% of `\topsep' in the file `myfilist.tex' (2012/11/30).
        \ifx\\#1\\\else
            \hfuzz=\textwidth \advance\hfuzz#1\relax
        \fi
        \noNiceVerb \verbatiminput{#2}%
%         \nobreak\vskip\medskipamount 
        \hrule \vskip-\parskip 
        \bigskip %%% \bigbreak
%% `\bigbreak' made much larger space in `myfilist.tex'.
    \endgroup
}
%% |\ctanpkgdref{<pkg-id>}| adds the printed link to 
%% `ctan.org/pkg' as a footnote. There is a little space 
%% for coloured link borders:
\newcommand*{\ctanpkgdref}[1]{%
    \ctanpkgref{#1}\,\urlfoot{CtanPkgRef}{#1}}
\errorcontextlines=4
\pagestyle{headings}

\endinput 
    %% shared formatting settings
\ifpdf\else \errhelp{hyperref draft bad with {equation}}
            \errmessage{run this with pdflatex only}\fi
% \ReadPackageInfos{dowith}
\usepackage{dowith}                                         %% 2012/05/17b
\sloppy
\MDkeywords{programming structures;                         %% 2012/05/14b
            macro programming, loops, list macros}
\hypersetup{%
    pdftitle=dowith.sty handles lists without separators,
    pdfsubject=documenting dowith.sty
}%% 2011/08/22 
\usepackage{fixltx2e}                       %% \textsubscript 2012/05/17b
%% <- TODO with `lmodern'?
% \newcommand*{\ctanpkgdref}[1]{%                             %% 2012/11/19
%     \ctanpkgref{#1}\,\foothttpurlref{ctan.org/pkg/#1}}
%% <- in `makedoc.cfg' 2015/11/14
\makeatletter                                   %% etc. 2012/05/17b
%% TODO `actcodes'!? 2012/11/02
  \newcommand*{\GetOtherChar}[2]{% 
    \@ifdefinable#1{%
        \edef#1{\expandafter\@gobble\string#2}}}
\makeatother
\GetOtherChar\codeLB\{ 
\GetOtherChar\codeRB\} 
% \GetOtherChar\codeSP\ %       %% ???
{\MakeOther\ \xdef\codeSP{ }}
\newcommand*{\codelb}{\code\codeLB}
\newcommand*{\coderb}{\code\codeRB}
\newcommand*{\codesp}{\code\codeSP}
\newcommand*{\codebd}[1]{\code{\codeLB#1\codeRB}}       %% 2012/05/19
%% rm. 2012/05/17b:
% \DeclareRobustCommand*{\code}[1]{%
%   \texttt{%
%     \let\{\codeLB \let\}\codeRB \let\ \codeSP
%     #1}}
% \newcommand*{\chtok}[2]{\ensuremath{\code{#1}\sb{#2}}}
%% <- 2012/05/17b ->
\newcommand*{\chtok}[2]{\code{#1}\textsubscript{#2}} 
\newcommand*{\lbtok}{\chtok\codeLB{1}}
\newcommand*{\rbtok}{\chtok\codeRB{2}}
\newcommand*{\lttok}[1]{\chtok{#1}{11}}
\newcommand*{\sptok}{\chtok\codeSP{10}}
\providecommand*{\TTb}{\meta{The\nolinebreak[3] \TeX book}}
\providecommand*{\TTbp}{\TTb\nolinebreak[3] p}
\newenvironment*{smallpar}
                {\medskip\par\begingroup\footnotesize}
                {\par\endgroup\medskip}
\newcommand*{\NTOK}[1]{\textsf{ntok}(\code{#1})}
\newcommand*{\ntok}[1]{\textrm{?}\code{#1}}
\newenvironment*{example}[1]
                {\trivlist\item
                 (\ulroman{#1})}
                {\endtrivlist}
\providecommand*{\ulroman}[1]{\meta{\romannumeral #1 }}
\newcommand*{\inlineitem}[1]{\ (\ulroman{#1})\enspace
             \ignorespaces}
\newcommand*{\pdots}{~.\kern\fontdimen3\font
                      .\kern\fontdimen3\font. }
\providecommand*{\Chi}{\mathrm{X}}
\renewcommand*{\httpprefix}{\theHTTPprefix}
\begin{document}
\maketitle
\begin{MDabstract}
This package provides macros for applying a 
%% mod. 2012/05/15:
% ``command" <cmd> to all items of a list $<arg-1><arg-2>\dots<arg-$n$>$
``command" to all items in a ``list of possible macro arguments," %% v0.21a
% in ``\TeX's mouth," 
% such as `\DoWithAllIn{<cmd>}{<list-macro>}', 
and also for extending and reducing macros storing such lists.
``Brace groups" are single items of such lists, as opposed to 
token lists. Iteration is implemented within \TeX's expansion 
processor, so works within `\write' as with 'blog.sty'.
 %%% ---pleading for mathematical rigour in \TeX ology!)    %% rm. 2012/05/17b
% Applications in mind belonged to \LaTeX, but the package should work 
% with other formats as well.
Loop and list macros in other packages are discussed in the documentation.
% %% 2012/05/09:
% There is an emphasis on expandability 
% %% mod./add. 2012/05/15:
% (no \wikienref{Iterator}{iterator;} 
%  essential within \cs{write} as with \CtanPkgRef{morehype}{blog.sty}),
% without relying on \CtanPkgRef{e-tex}{$\varepsilon$-\TeX}.
%% 2012/05/18:
% Iteration is implemented within ``\TeX's mouth," so works within 
% `\write' as with \CtanPkgRef{morehype}{blog.sty}. 
There is no need for \CtanPkgRef{e-tex}{$\varepsilon$-\TeX} to which 
some of them refer.

The package is ``generic," i.e., should also work with Plain \TeX\ 
or even other formats, relying on the \ctanpkgref{plainpkg} package 
for some minimal \LaTeX-like behaviour.     %% \ 2012/11/27

\MDaddtoabstract{Related packages}
\let\pkg\ctanpkgref     \pkg{catoptions},
\pkg{etextools}, \pkg{etoolbox}, \pkg{forarray}, \pkg{forloop}, 
\pkg{loops}, \pkg{multido}, \pkg{moredefs}, \pkg{lmake}, 
\pkg{texapi}, \pkg{xfor}, \pkg{xspace}
\end{MDabstract}
\tableofcontents

%   \newpage
\section{Usage and Features}                    %% restructured 2012/05/16

\subsection{Installing and Calling}
The file 'dowith.sty' is provided ready, installation only requires
putting it somewhere where \TeX\ finds it
(which may need updating the filename data
 base).\urlfoot{ukfaqref}{inst-wlcf}           %% corr. 2011/02/08
The packages \ctanpkgdref{plainpkg}
and 'stacklet' (\ctanpkgref{catcodes})\urlfoot{ctanpkgref}{catcodes} 
must be installed as well.

As to calling (loading): 'dowith' is a ``\pkg{plainpkg} package" 
in the sense of the \ctanpkgref{plainpkg}
documentation that you may consult for details.
So roughly, 
\begin{itemize}
  \item load it by \ |\usepackage{dowith}| \ if you can, 
  \item otherwise by \ |\RequirePackage{dowith}| \\
        (perhaps from within another ``\pkg{plainpkg} package"), 
  \item or by \ |\input dowith.sty| 
  \item or even by \ |\def\filename{dowith}     \def\fileinfo{simple list loop (UL)}
\def\filedate{2012/11/05} \def\fileversion{v0.3}
%%
%% Copyright (C) 2011 2012 Uwe Lueck,
%% http://www.contact-ednotes.sty.de.vu
%% -- author-maintained in the sense of LPPL below --
%%
%% This file can be redistributed and/or modified under
%% the terms of the LaTeX Project Public License; either
%% version 1.3c of the License, or any later version.
%% The latest version of this license is in
%%     http://www.latex-project.org/lppl.txt
%% We did our best to help you, but there is NO WARRANTY.
%%
%% Please report bugs, problems, and suggestions via
%%
%%   http://www.contact-ednotes.sty.de.vu
%%
%% == Proceeding without \LaTeX ==
%% v0.3 mainly replaces imitating the 
%% \CtanPkgRef{german}{german.sty} approach to genericity 
%% by \ctanpkgref{plainpkg}:
\input plainpkg
\ProvidesPackage{\filename}[\filedate\space 
                 \fileversion\space \fileinfo]
\PushCatMakeLetterAt
%% If \LaTeX\ is not present ... 
\ifltx \else
%% ... an old version of its `\in@' is introduced. 
%% It is bad as a subword test 
%% (false positive cases, cf. \CtanPkgRef{nicetext}{fifinddo} 
%%  documentation), but 'dowith' will check for single tokens only. 
%% If \LaTeX\ \emph{is} present, 
%% on the other hand, `\ifin@' is recognized while skipping 
%% `false' parts of conditionals, without being matched 
%% by some `\fi' before the next `\else', so I hide it by `\csname':
  \expandafter\newif\csname ifin@\endcsname
  \def\in@#1#2{%
    \def\in@@##1#1##2##3\in@@{%
      \ifx\in@##2\in@false\else\in@true\fi}%
    \in@@#2#1\in@\in@@}
\fi
%% 
%% == Applying a Command ==                             %% 2011/11/07
%% \label{sec:apply}
%% === Core ===
%% \label{sec:core}
%% |\DoWith{<cmd>}<list>\StopDoing| applies <cmd> to all elements 
%% of <list>. An element of <list> (after tokenizing) 
%% may be either a single token or a group `{<balanced>}'.
\def\DoWith#1#2{%
    \ifx\StopDoing#2\empty
%% The previous `\empty' (replacing `%') is a bug fix as of v0.22 
%% (June 2012),                                             %% 2012/11/05
%% while in my extension draft I already had it in January 2012. 
%% It allows ``empty" arglist items \qtd{\lbtok\rbtok}. 
%% Before v0.22, such an empty brace group would have resulted 
%% in comparing `\StopDoing' with `\else', so \qtd{\lbtok\rbtok}
%% would have had the same effect as `\StopDoing', the token text 
%% after `\else' until `\fi' would have been skipped. 
%% Instead, the user may have a reason to allow empty arguments\slash
%% brace groups.
        \else#1{#2}\expandafter\DoWith\expandafter#1\fi}
%% |\StopDoing| delimits the list: 
\let\StopDoing\DoWith
%% ... something arbitrary that is not expected to occur in a list.
%% With \[`\let\StopDoing*'\] instead, the star would end lists.
%%
%% |\DoWithAllOf{<cmd>}{<list>}| works like 
%% \[`\DoWith{<cmd>}<list>\StopDoing':\]
\def\DoWithAllOf#1#2{\DoWith#1#2\StopDoing}
%% 
%% === `&\do' being the Command ===
%% \label{sec:do}
%% When the <list> is worked at a single time in the \TeX\ run 
%% where assignments are possible, instead of introducing a new 
%% macro name for <cmd> you can use `\do' for <cmd> as a 
%% ``temporary" macro and define it right before 
%% \[`\DoWith{\do}<list>\StopDoing'\] 
%% However, we provide \[|\DoDoWith{<cmd>}<list>\StopDoing|\] 
%% as a substitute for the former line that at least saves one token. 
%% For the definition of `\do', we provide
%% |\setdo{<def-text>}|.
%% It works similarly to \[`\renewcommand{\do}[1]{<def-text>}',\] 
%% so <def-text> should contain a `#1': 
\def\setdo{\long\def\do##1}
%% With |\letdo<cmd>| that is provided next where <cmd> is defined 
%% elsewhere, you could type 
%% \[`\letdo<cmd>\DoDoWith<list>\StopDoing'\]
%% It seems to me, however, that you better type 
%% \[`\dowith<cmd><list>\StopDoing'\] instead. 
%% So I provide `\letdo' although I consider it useless here. 
%% It is provided somewhat for the sake of ``completeness," 
%% thinking that it might be useful at other occasions such as 
%% preceding `\dospecials'.
\def\letdo{\let\do}
%% |\DoDoWith| has been described above:
\def\DoDoWith{\DoWith\do}
%% By analogy to `\DoWithAllOf', we provide 
%% |\DoDoWithAllOf{<list>}|:
\def\DoDoWithAllOf{\DoWithAllOf\do}
%%
%% === Expand List Macro ===
%% The former facilities may be quite useless
%% as such a <list> will not be typed at a single place in the 
%% source code, rather the items to run <cmd> on may be collected 
%% occasionally when some routines run. The elements may be collected 
%% in a macro <list-macro> expanding to <list>. 
%% So we provide \[|\DoWithAllIn{<cmd>}{<list-macro>}|\]
%% (or `\DoWithAllIn<cmd><list-macro>').
%% There is no need to type `\StopDoing' here:
\def\DoWithAllIn#1#2{%
    \expandafter\DoWith\expandafter#1#2\StopDoing}
%% |\DoDoWithAllIn{<list-macro>}| saves a backslash or token 
%% for `\do' as above in Sec.~\ref{sec:do}:
\def\DoDoWithAllIn{\DoWithAllIn\do}
%% 
%% 
%% == Handling List Macros ==
%% === Initializing ===
%% Here is some advanced `\let<cmd>\empty', perhaps a little 
%% irrelevant for practical purposes. Both
%% \[|\InitializeListMacro{<list-macro>}|\] and 
%% \[|\ReInitializeListMacro{<list-macro>}|\] attempt to ``empty" 
%% <list-macro>, and when we don't believe that \LaTeX\ has been 
%% loaded, both do the same indeed. Otherwise the first one 
%% complains when <list-macro> seems to have been used earlier 
%% while the second complains when <list-macro> seems \emph{not} to 
%% have been used before:
\ifltx                                                  %% v0.3
  \def\InitializeListMacro#1{\@ifdefinable#1{\let#1\empty}}
  \def\ReInitializeListMacro#1{%
    \edef\@tempa{\expandafter\@gobble\string#1}%
    \expandafter\@ifundefined\expandafter{\@tempa}%
        {\@latex@error{\noexpand#1undefined}\@ehc}%
        {\let#1\empty}}
\else
  \def\InitializeListMacro#1{\let#1\empty} %% not \@empty 2011/11/07
  \let\ReInitializeListMacro\InitializeListMacro
\fi
%% |\ToListMacroAdd{<list-macro>}{<cmd-or>}|
%% appends <cmd-or> to the replacement token list of <list-macro>. 
%% <cmd-or> may either be tokenized into a single token,
%% or it is some `{<balanced>}'.
\def\ToListMacroAdd#1#2{\DefExpandStart#1{#1#2}}
\def\DefExpandStart#1{\expandafter\def\expandafter#1\expandafter}
%% 
%% === Testing for Occurrence of a Token ===
%% |\TestListMacroForToken{<list-macro>}{<cmd>}|
%% sets `\in@true' when <cmd> occurs in <list-macro> 
%% and sets `\in@false' otherwise:                      %% 2011/11/07
\def\TestListMacroForToken#1#2{%
    \expandafter \in@ \expandafter #2\expandafter{#1}}
%% Indeed I removed an earlier `\IfTokenInListMacro', 
%% now it's a kind of compromise between having a shorthand macro 
%% below and a generalization for users of the package.
%% 
%% === Adding and Removing ===
%% |\FromTokenListMacroRemove{<list-macro>}{<cmd>}| 
%% removes the token corresponding to <cmd> from the list 
%% stored in <list-macro> 
%% (our parsing method does not work with braces):      %% 2011/11/07
\def\FromTokenListMacroRemove#1#2{%
%% I am not happy about defining \emph{two} parser macros, 
%% but for now ...
    \TestListMacroForToken#1#2%
    \ifin@
      \def\RemoveThisToken##1#2{##1}%
      \expandafter \DefExpandStart
        \expandafter #1\expandafter {%
          \expandafter\RemoveThisToken #1}% 
%% TODO warning otherwise?
   \fi}
%% %% 2011/11/07:
%% ... but this only removes a single occurrence ... 
%% \[|\InTokenListMacroProvide{<list-macro>}{<cmd>}|\]
%% avoids multiple entries of a token 
%% by \emph{not} adding anything when <cmd>
%% already occurs in <list-macro> (again, this does not work with 
%% braces, try `\in@{{}}{{}}').
\def\InTokenListMacroProvide#1#2{%
    \TestListMacroForToken#1#2%
    \ifin@ \else                    %% TODO warning?
        \ToListMacroAdd#1#2%
    \fi}
%%
%%
%% == Leaving and History ==
\PopLetterCatAt                                         %% v0.3
\endinput

VERSION HISTORY 
v0.1    2011/06/23/28  stored separately
v0.2    2011/11/02     simpler, documented
        2011/11/03     corrected \if/\else for init
        2011/11/07     \TestListMacroForToken, \InListMacroProvide; 
                       doc.: \pagebreak, structure
        2011/11/19     modified LaTeX supplements
v0.21   2012/05/14     fix for "generic" and `typeoutfileinfo': 
                       @ before ...!
v0.21a  2012/05/19     \labels sec:apply, sec:core; \pagebreak?
v0.22   2012/06/04     allow {} items
v0.3    2012/11/05     updating copyright, using `plainpkg', 
                       rewording documentation there
|~\dots
\end{itemize}

% \section{Example}

% \section{Discussion}                                      %% 2012/05/16
% \subsection{What It Seems to Do \dots}
% The 'dowith' package provides tools to simplify \TeX\ macro programming. 
% Understanding it \emph{really} may require understanding certain passages of 
% \TTb, such as pp.~38f. It may even require overcoming 
% terrible confusions in \TTb.
% Let this be a last try at understanding the package without understanding \TeX: 
% It allows you to abbreviate 
% \[<cmd><arg-1><cmd><arg-2>\dots<cmd><arg-$n$>\]
% by 
% \[`\DoWith<cmd><arg-1><arg-2>\dots<arg-$n$>\StopDoing'\]
% or by 
% \[`\DoWithAllOf<cmd>{<arg-1><arg-2>\dots<arg-$n$>}'\]
%
%   \pagebreak                                                %% 2012/05/19
\subsection{What It Does With What Lists}
% \subsection{What It Actually Does \dots}
\label{sec:lists-intuit}

The single commands that the package provides are described 
in the \hyperref[sec:implement]{implementation section} below. 
What follows here is some general background about how the 
commands work.
%% <- moved down from previous section, mod. 2012/05/19

The term \qtd{list} may refer to various things and need clarification here. 

First of all, we are not referring to \LaTeX\ `list' environments 
such as `enumerate' or `itemize'; 
neither to ``\acro{TODO}" lists of what needs to be done soon. 

Rather, 'dowith' allows you to abbreviate 
\[<cmd><arg-1><cmd><arg-2>\dots<cmd><arg-$n$>\]
by 
\[`\DoWith<cmd><arg-1><arg-2>'\dots<arg-$n$>`\StopDoing'\]
or by 
\[`\DoWithAllOf<cmd>{<arg-1><arg-2>'\dots<arg-$n$>`}'\]
With small $n$, one may doubt whether this really is an abbreviation~\dots;
anyway, 
\[<arg-1><arg-2>\dots<arg-$n$>\]
was an attempt to refer to the kind of lists we are dealing with. 
\[<arg-1>, <art-2>, \dots, <arg-$n$>\]
are the ``items" of the list. 
The question is: what counts as an item?

We might say that `aa' is a list of \emph{two} items, 
<arg-1> being `a' and <arg-2> being `a', too.

When we do \emph{three} keystrokes to get `a a' instead of `aa', 
we still have \emph{two} items, 
<arg-1> being `a' and <arg-2> being `a' too. 
Strange, isn't it?

Also, when in `aa' we replace the first `a' by a backslash, `\', 
we get `\a', and this is a list of a \emph{single} item, 
$<arg-1>=`\a'$~\dots

You shouldn't believe these stories of mine entirely. 
What I am alluding to is that the \emph{``items"} 'dowith' is about 
are determined in terms of \emph{\TeX's tokens}, and the relation 
between the ``characters you type" and \emph{\TeX's tokens} 
is not entirely straightforward. 

\subsection{The Notion of Arglists for \LaTeX\ Users}
%% <- 2012/05/17b ->
% \subsection{The Notion of Arglist}
\label{sec:arglists-intuit}
Still, it may suffice to clarify what counts as an <arg-$i$> 
without speaking of \emph{tokens} explicitly: It is simply 
what a \emph{one-parameter macro} 
(where the parameter is \emph{not delimited} in terms of 
 \TTbp p.~203f.)
can take as an \emph{argument.}

The lists 'dowith' is about then are lists \emph{of possible arguments}
in the previous sense---let me call them \emph{``arglists."}\footnote{Not 
    to be confused with German \httpref{de.wiktionary.org/wiki/Arglist}{Arglist.}}
The single \emph{items} of such lists are those single possible arguments. 
They become \emph{actual} arguments beginning from the leftmost \emph{possible} one 
when 'dowith' presents them to that <cmd> mentioned earlier---where 
<cmd> \emph{should} be a one-parameter macro (or some \TeX\ primitive 
parsing arguments similarly).

The reader perhaps has an \emph{intuitive} understanding of 
what can be an argument of a one-parameter macro. 
A \emph{strict \LaTeX} user may think that such an argument <arg-$i$>
just has form `{'<ark-$i$>`}', i.e., $<arg-$i$>=`{'<ark-$i$>`}'$
for some <ark-$i$>. Such arguments are also called \emph{``brace groups".}
(\LaTeX's \emph{optional} arguments `[<extra>]' 
 do not count as possible arguments here, they are not macro arguments 
 in the sense of \TTb.)
In this restricted \LaTeX\ sense, arglists consist of brace groups 
\[`{<ark-1>}{<ark-2>}'\dots`{'<ark-$n$>`}',\]
and each single brace group is an \emph{item} of it. 

The \emph{\TeX\ macro writer}, by contrast, knows that a macro argument 
doesn't need outer braces. In an intuitive sense, a single 
``command" can be a macro argument, too. ``Command" may be understood 
as ``control sequence" (starting with a backslash), 
but some authors also have considered single \emph{characters} 
(character \emph{tokens}?) ``commands." Blank spaces, by contrast, 
are ignored when a macro looks for its argument.    %% 2012/05/17b

%% useless, just name items 2012/05/17b:
% We arrive at an ``intuitive recursive definition" of ``arglist:" \
% (\meta{i})\enspace The empty list is an arglist; it doesn't have an item 
% for 'dowith'. \ (\meta{ii})\enspace If <list> is an arglist, then 
% (a)~`{<ark>}<list>' is an arglist whose first item for 'dowith' is `{<ark>}';
% (b)~`<cs><list>' with a ``control sequence" <cs>
%     is an arglist whose first item is `<cs>' 
%     (for the command <cmd> to which 'dowith' presents <cs>); and
% (c)~`<char><list>' with a non-blank character <char>
%     is an arglist whose first item is `<char>'. \
% (\meta{iii})\enspace Nothing else is an arglist.
% 
% But keep in mind that this ``intuitive" understanding essentially is wrong.

\subsection{Anatomy of \TeX}
The documentation of v0.22 as of 2012-06-04 said that the package 
is about ``lists in \TeX's mouth." However, this was very wrong. 
I believed it following Alan Jeffrey's paper ``Lists in \TeX's Mouth",\footnote{%
    Alan Jeffrey: \tugbartref{tb11-2/tb28jeffrey}{``Lists in \TeX's Mouth,"} 
    \acro{TUG}boat Vol.~11 (1990), No.~2, pp.~237--245), 
    \urlhttpref{tug.org/TUGboat/tb11-2/tb28jeffrey.pdf}.}
in whose Section~2 you read:
\begin{quote}\it
  \TeX's programming facilities come in two forms---there 
  are \TeX's \emph{macros} which are expanded in its
  mouth, and some additional \emph{assignment} operations
  like \cs{def} which take place in the stomach.
\end{quote}
The macros that Jeffrey lists and describes in that article can be 
obtained as a \acro{CTAN} package 
\ctanpkgref{lambda-lists}.\footnote{\urlhttpref{ctan.org/pkg/lambda-lists}} 
If you follow the link given here (in the footnote), 
you currently (2012-11-03) read about this package:
\begin{quote}\it
    These list-processing macros avoid the reassignments 
    employed in the macros shown in Appendix D of the TeXbook: 
    all the manipulations take place in what Knuth is pleased to 
    call ``TeX's mouth".
\end{quote}
But Knuth doesn't. On page 267 of \TTb, you read:
\begin{quote}\it
  Chapter 7 has described the process by which input files
  are converted to lists of tokens in \TeX's ``mouth,"
  and Chapter 20 explained how expandable tokens are converted
  to unexpandable ones in \TeX's ``gullet" by a process 
  similar to regurgitation.
\end{quote}
I.e., the ``mouth" is \TeX's ``tokenizer," the inner part of what 
van Eijkhout calls \TeX's ``input processor" on, e.g., p.~15 of his 
\meta{\TeX~by Topic}.\footnote{It is available as a \acro{CTAN} package 
    \ctanpkgref{texbytopic} at \urlhttpref{ctan.org/pkg/texbytopic}.}
The exact rules the tokenizer follows are described on pp.~46f.\ 
(Chapter~8!) of \TTb.
\emph{Macro expansion} takes place in \TeX's \emph{``gullet"}, 
which van Eijkhout calls \TeX's ``expansion processor" (p.~16). 
Abrahams, Hargreaves, and Berry follow Knuth's terminology on
pp.~16 and 46f.\ of their \meta{\TeX~for the Impatient}.\footnote{It 
    is available as \acro{CTAN} package \ctanpkgref{impatient}, 
    \urlhttpref{ctan.org/pkg/impatient}.} 
% Only Jeffreys and my wrong old documentation say this were 
% ``\TeX's mouth"---and I made the same mistake with the 
% \ctanpkgref{bitelist} package.
%% <- 2012/11/19 ->

\TeX's gullet has been called ``\TeX's mouth" also 
in the documentation of my \ctanpkgdref{bitelist} package and 
in the documentation of the package \ctanpkgdref{bibleref-mouth}.

Moreover, I should have clarified that Jeffrey's paper deals with 
``lists" in some general, rather abstract sense, different from 
the kind of lists the present documentation tries to characterize 
as the objects for 'dowith'.

\subsection{\TeX's Tokens}
\label{sec:toks}

% The \emph{\TeX\ macro writer} understanding \TeX\ properly 
% does not really think of arglists. \TeX perts instead think of 
% What \TeX nically matters is
% what happens in ``\TeX's mouth,"\footnote{Cf.~\TTbp.~46.} 
% as some authors have suggested a metaphor,\footnote{%
The 'dowith' package is a tool that affects the order of tokens in 
\TeX's gullet.

\begin{smallpar}
The ``characters you type" enter ``\TeX's mouth" line by line, 
in a slightly modified appearance. Each line forms a \emph{string}.
\TeX\ takes initial substrings away from it and turns them 
into \emph{tokens} that are appended to the right of \TeX's 
\emph{expansion buffer} (``gullet").

%% Removing 2012/11/04:
% More formally, \TeX\ has a \emph{character buffer.}
% It forms a single token from an initial segment of the buffer content---unless 
% there is a special situation with blank spaces or something pathological.
% When an \emph{escape character}, as the backslash usually is one, 
% has been noticed recently (that isn't followed by another one immediately), 
% the character buffer may need to be feeded from more outside, 
%% <- especially wrong, 2012/11/04
% until it contains enough material to form a token from. 

There are \emph{two kinds of tokens} here: \emph{named} tokens 
and \emph{character} tokens. ``Named" tokens usually are referred
to as ``control sequence tokens" or just ``control sequences"---I~really 
want to avoid those horrible confusions from \TTb.
There never are any ``parameter tokens" in \TeX's gullet 
(perhaps unless one considers a one-step macro expansion 
 a two-or-more-step procedure).
%  merged 2012/11/04
%% Moving down 2012/11/04
The character(s) \emph{after} the escape character until 
some delimiting character form a \emph{string} that is the 
\emph{name} of the token that is formed---a \emph{named} 
token, as I am saying.
% What has been used to form a token is removed from the character buffer.
%% 2012/11/04:
\emph{Character} tokens are formed by removing a character from the 
beginning of the character buffer and appending it to the token buffer 
paired with its \emph{category} code.

For every \emph{string of characters}, there is exactly one 
(possible) \emph{named token} whose name the string is.\footnote{``Possible" 
    refers to the fact that \TeX\ does not store named tokens anywhere 
    before they appear in its gullet, maybe apart from ``primitive" 
    tokens that have a ``pre-assigned meaning" when a \TeX\ run     %% \ 2012/11/04
    starts.---What is more bad with my claim is that 
    the \TeX\ program by design cannot extend its memory arbitrarily---even 
    not using the ``cloud''---, so it doesn't support tokens 
    whose name lengths are above a certain limit.}
It is so common (starting from \TTb) to denote the token whose name 
is <string> by \lq\verb+\+<string>\rq. For instance, the token whose 
name is `input' is denoted by \qtd{&\input}. On the other hand, 
on page~7 of \TTb\ \qtd{&\input} is a ``string of characters."
With this notation, it is already difficult to explain what 
the \LaTeX\ command `\DeclareRobustCommand' does or what the difference 
between a starred \LaTeX\ command and a starred \LaTeX\ environment is.\footnote{%
    A reader knowing \LaTeX\ only thinks that \qtd{\code{&\\\codesp}} 
    is the result of typing a double backslash and a space 
    and that \qtd{\cs{equation*}} is the ``command" \cs{equation} 
    followed by a `*'.} 
\TTb\ makes it worse by saying on page~39: 
\meta{``A control sequence is considered to be a single object 
        that is no longer composed of a sequence of symbols."}
So ``it depends" whether \qtd{&\input} is a string of characters or not---it \emph{is 
before} tokenization, but \emph{no longer} afterwards. 
So if you have two computers and start a \TeX\ run on each of them 
with a little difference in time, there will be a moment where 
\qtd{&\input} is a string on the one computer but not on the other? 
This 
appears to me %%% is %% 2012/11/04
like saying \meta{``When we apply the square root function to 
the number 4, the number 4 will no longer be the number 4, 
it will be the number 2 instead."}
\end{smallpar}

\TTb\ does offer an alternative notation for named tokens: ``boxing;" 
so the token whose name is `input' can be denoted by the rather 
``graphical" notation \qtd{|input|} (used only exceptionally).\footnote{%
    The box notation is introduced on page~38 without explanation, 
    as if it explained something.} 
\emph{I}~would suggest something like 
\qtd{\NTOK{input}} for clarity and \qtd{\ntok{input}} 
for brevity.\footnote{I am suggesting the question mark for named tokens 
 since \TeX\ ``must look up the current definition" of a named token 
 according to \TTbp.~39, while the meaning of character tokens rather 
 is ``fixed," at least according to \TTbp.~39. However, 
    \emph{active}-character such as .&~ are in the same situation 
    as named tokens as to this respect. The dot notation may be fine for them, 
    though.}

\begin{smallpar}
%% rm. 2012/11/04, cf. above
% \emph{Character tokens} get into \TeX's mouth by tokenization 
% when characters begin the buffer content while \emph{not} scanning 
% a name for a named token. A single character then is removed from the 
% character buffer, and a token storing its character code and current 
% category code is pushed into \TeX's mouth.

Named tokens may get into \TeX's gullet by ``tokenization" as described above, 
i.e., they are drawn from the character buffer. But they also can 
appear in \TeX's gullet ``from within," 
by the manipulation inside \TeX's gullet. 

More formally, those manipulations are called ``expansion,"
and \TeX's gullet can be conceived of as a \emph{token buffer} 
that is feeded to the right (or end) by tokenization from the character buffer. 
Expansion means that certain tokens in the token buffer are substituted 
by other ones. This way tokens may get into \TeX's gullet that 
emerged from tokenization a ``long time ago", maybe in a previous 
run that created the \emph{format} (\TeX's variant \code{INITEX}); 
or tokens may appear by some hardwired expansion function. 

However, \emph{named tokens} may get into \TeX's gullet 
also by \emph{expansion}, never having been drawn by tokenization 
and not being hardwired. This happens by the `\csname' 
% name      %%  rm. 2012/07/25
construct. 
The input \emph{code} may contain 
\[`\csname tupni\endcsname'\]
This may be converted into 7 tokens entering \TeX's gullet, 
the first one being \NTOK{csname}, the last one 
\NTOK{endcsname}, and five character tokens in between. 
Due to some \emph{function} (which I would denote as *\code{csname})
originally associated with the token \NTOK{csname}, 
those seven tokens then are replaced by \NTOK{tupni}, 
the named token whose name is `tupni'. It is not required that 
the \TeX\ program knows about a token \NTOK{tupni}, 
neither anybody must type \qtd{&\tupni} in any file.\footnote{These 
    considerations may not be essential here, 
    rather a draft for a paper. Using 'dowith', 
    one better just thinks of the arglist items one actually lists.}
\end{smallpar}

\subsection{Arglists vs.\ Lists of Tokens---Example}
% \subsection{Arglists and 'dowith' \TeX nically}
% In the \TeX nical sense, I think of arglists \emph{and their items}
% as follows. Arglists are lists (or sequences) of tokens. 
% What is somewhat difficult is that the \emph{items} of a token list 
% usually are \emph{tokens.} Especially, the curly braces in the code 
% you type usually are converted into certain \emph{character tokens}
% that are single items of the resulting token list.\footnote{I discover 
%     this conceptual puzzle 2012-05-16.} 
% 
% The conceptual trap here may have resulted from denoting lists 
% in a \emph{\Wikienref{juxtaposition}} notation. 
% In Section~\ref{sec:lists-intuit}, I have written `aa' for a 
% ``list" of \emph{two} items. ``List" is rather a \emph{computer science} 
% term, its mathematical counterpart rather is the notion of (finite) 
% \emph{sequences.} The usual \emph{mathematical} notation for a finite sequence 
% writes list items surrounded by \emph{\wikienref{bracket}{brackets}}
% (round---\emph{\Wikienref{parentheses}}---seems to be more common than square, 
%  also \wikienref{angle brackets}{chevrons} are used).
% 
% So the \emph{string} `aa' can be written more clearly---mathematically---as 
% $(`a',`a')$. The trap with `\a' is that it could be \emph{either} the \emph{string} 
% $(`\',`a')$ \emph{or} the one-item list $(`\a')$ of strings. 
% But even in the latter case, I urge not to consider it a \emph{\TeX\ token}. 
% Rather, I consider `\a' a mistaken way of referring to the 
% named token \NTOK{a} whose name is the string `a' (or $(`a')$).
% (Section~\ref{sec:toks}). 
% 
% However, the conceptual trap about arglists and token lists 
% (there must have been some ``Arglist"!\@) lurks on another level, 
% as follows. Recall \TTb's notation of \chtok{<char>}{<cat>}
% for the \emph{character token} that \TeX's tokenizer forms 
% from <char> in the character buffer when <char>'s category code is <cat>.
% Usually, the \emph{character} `a' is converted into the 
% \emph{character token} \lttok{a}, `{' is converted into \lbtok, 
% and `}' is converted into \rbtok. 
% 
% We are turning to some \strong{examples} and \strong{counterexamples}. 
% Let us see what confusions occurred in the ``intuitive" view on 
% arglists in Section~\ref{sec:arglists-intuit}.
% 
% % \begin{example}{1} It 
% First, it was bad in Section~\ref{sec:arglists-intuit}
% to think that `aa' is a two-item arglist. It was confused with something 
% like $\lttok{a}\lttok{a}$. The latter looks like a token list---or is it an arglist? 
% Both? Anyway, it is juxtaposition notation applied to tokens, 
% mathematically it is $(\lttok{a},\lttok{a})$, so~\dots
% % \end{example}
% 
% Now let us reconsider the ``intuitive recursive definition" of arglist. 
% Or let us look at a recursive definition of \emph{token} list.
% \inlineitem{1} There is nothing wrong with saying the the empty list is 
% a token list, the same holds for arglists. 
% \inlineitem{2} a.~When we attach an arbitrary token to a \emph{token} list 
% (at the left), the result is another token list---fine. 
% There are only certain difficulties with ``handling" special token lists 
% such as $\lbtok\lttok{a} = (\lbtok,\lttok{a})$. 
% And the latter is \emph{not} an \emph{arglist!} \
% b.~When we attach a \emph{named} token to an arglist (at the left), 
% the result \emph{is} an \emph{arglist}.\footnote{This is 
%     a \emph{conjecture} only right now---2012-05-16---in which I strongly believe. 
%     Likewise later.} 
% c.~When we attach a \emph{letter} token \lttok{<char>} to an arglist 
% (at the left), the result \emph{is} an \emph{arglist}.
% d.~When we talked about ``brace groups," they seemed to be \emph{strings} 
% of characters. Instead, I would like to suggest that a brace group 
% is an \emph{arglist}\pdots
% When \lbtok\ is attached to the left of an arglist and \rbtok\ to the right, 
% the result is an arglist---this is what I would call a ``brace group"!
% e.~The \Wikienref{concatenation} of two arglists is an arglist. 
% 
% \begin{smallpar}
% The above notion of ``attaching" a token to a token list or an arglist 
% % should be clarified, but instead of a general definition in terms of 
% % ``words" of formal languages, examples may suffice here and now.\footnote{2012-05-16.}
% should be clarified. Attaching an item $j$ at the left of a list $\lambda$ 
% is the same as concatenating the one-item list $(j)$ with $\lambda$\pdots 
% but the English \wikienref{append}{Wikipedia} seems to explain concatenation by 
% \emph{appending}. As we \emph{remove} items one-by-one from the \emph{left} 
% (beginning) of a list, I prefer the inverse \emph{prepending} items as basic operation 
% for building lists---cf.\ \Wikienref{CAR and CDR}.
% \end{smallpar}
% 
% We won't complete a formal (recursive) definition of arglist here and 
% now.\footnote{2012-05-16.} Just observe that ``brace groups" 
% make the difference between token lists and arglists. 
% Recall that an ``item" of an arglist ``operationally" is defined 
% as something that a one-parameter macro removes. 
% Such a macro removes certain single tokens 
% (space tokens not among them)---and entire ``brace groups"! 
% Especially, consider \[\lbtok\rbtok\]%%%.
% Actually, this is another ambiguous notation. 
% If it refers to $(\lbtok,\rbtok)$, it is a \emph{token} list, 
% not an arglist. If it refers to $(\lbtok\rbtok)$---it should 
% better refer to $((\lbtok,\rbtok))$, which is a 
% \emph{one-item arglist} whose only item is the former two-item token list!
% 
%% <- 2012/05/17b ->
Let us reconsider the examples from Sections~\ref{sec:lists-intuit}
and~\ref{sec:arglists-intuit}, and pack them into a single example. 
If you type a file line
\begin{equation}
  `a a\a{a}'
\end{equation}
(\emph{eight} keystrokes),
it should usually be converted into this \emph{seven}-item list 
of 
%%% (five) %% rm. 2015/11/14 -- ???
tokens:
\begin{equation}
  \label{eq:toks}
  \lttok{a}\;\sptok\;\lttok{a}\;\NTOK{a}\;\lbtok\;\lttok{a}\;\rbtok
\end{equation}
---with notation from Section~\ref{sec:toks} and
\TTb's notation \chtok{<char>}{<cat>} 
for the \emph{character token} that \TeX's tokenizer forms 
from <char> in the character buffer when <char>'s category code is <cat>.

It turns out that the token list in \ref{eq:toks} 
provides an arglist of \emph{four} items: The token \lttok{a}
at the first and third place, the named token \NTOK{a}, and the entire 
token list $\lbtok\lttok{a}\rbtok$ as a single item---a ``brace group."
The space token is ignored.\footnote{\TTbp.~201: ``\TeX\ doesn't use 
single spaces as undelimited arguments."}

You can try this after `\renewcommand{\a}{A}'\footnote{Otherwise 
    \cs{a} is a one-parameter macro that breaks 'dowith''s control.}
with 'dowith':
\begin{equation}
  \label{eq:sample-code}
  `\DoWith\typein a a\a{a}\StopDoing'
\end{equation}
Then \LaTeX\ shows `a', `a', `A' from `\a', and another `a' from within 
the braces---`\typein' (as any macro with arguments) removes them.

% \show\a
% \a{a}
\renewcommand*{\a}{A}
% {\MakeNormalHere\# \newcommand*{\TypeOut}[1]{\typein{#1}}
%  \global\let\TypeOut\TypeOut}
% \let\TypeOut\typein
% \let\TypeOut\typeout
% \DoWith\TypeOut a a\a{a}\StopDoing
% \DoWith\typein a a\a{a}\StopDoing
% { %%% \tracingmacros=1 \tracingonline=1
%  \MakeNormalHere\# \AssignCatCodeTo{2}\]
%  \DoWith\typein #a\a{a]\StopDoing}

I have avoided saying \ref{eq:toks} \emph{were} an arglist of 4 items. 
The mathematical basic way of writing lists---understood as finite
\emph{\wikienref{sequence}{sequences}}---as ``commma-separated lists" 
within \wikienref{bracket}{brackets} may clarify the difference 
(that the \Wikienref{juxtaposition} notation tends to conceal). 
The \emph{token} list is 
\begin{equation}
  \label{eq:toks,}
  (\lttok{a}, \sptok, \lttok{a}, \NTOK{a}, \lbtok, \lttok{a}, \rbtok)
\end{equation}
while the list of macro arguments is
\begin{equation}
  \label{eq:args}
  (\lttok{a}, \lttok{a}, \NTOK{a}, (\lbtok, \lttok{a}, \rbtok)).
\end{equation}
\ref{eq:toks} or \ref{eq:toks,} simply is \emph{not} an arglist 
(since neither \lbtok\ nor \rbtok\ can be a macro argument), 
and the arglist \ref{eq:args} ``provided" by the list of tokens 
is \emph{not} a list of \emph{tokens}---its final item is a 
three-item list of tokens, and a token cannot be a list of 
two or more tokens itself(\emph{!?}).

\subsection{Another Notation and the Example's Steps}
\label{sec:steps}                                           %% 2012/05/20
\begin{smallpar}
To write token lists easier and hopefully easier to read, 
I would suggest writing \qtd{.<char>} for the character token 
that the tokenizer ``usually" forms from character <char>, i.e., 
adding the \emph{standard} category code as in \TTb\ (page~37).
Then \ref{eq:toks} would read\footnote{See Section~\ref{sec:toks} for the question mark.}
\begin{equation}
  .&a\,.\codesp\,.&a\,\ntok{a}\,.\codelb\,.&a\,.\coderb
\end{equation}
and the corresponding arglist is
\begin{equation}
  (.&a,.&a,\ntok{a},(.\codelb\,.&a\,.\coderb))
\end{equation}
In ``retrospect," the result of tokenizing \ref{eq:toks} should be
\begin{equation}
  \label{eq:retro}
  \ntok{DoWith}\,\ntok{typein}\,
  .&a\,.\codesp\,.&a\,\ntok{a}\,.\codelb\,.&a\,.\coderb\,
  \ntok{StopDoing}
\end{equation}
and the intention is that it works like 
\begin{equation}
  %% added \,s 2012/06/07:
  \ntok{typein}\,.\codelb\,.&a\,.\coderb\,
  \ntok{typein}\,.\codelb\,.&a\,.\coderb\,
  \ntok{typein}\,.\codelb \ntok{a}\,.\coderb\,
  \ntok{typein}\,.\codelb\,.&a\,.\coderb
\end{equation}
(The definition of `\DoWith' in Section~\ref{sec:core} indeed adds surrounding braces, 
 if missing.)
However, \TeX\ rather tries to work with as few tokens ahead as possible.
When it finds \ntok{DoWith} and the latter's meaning is the one intended 
by 'dowith', it first looks for nothing more than the two arguments 
required by our definition of `\DoWith'. A few moments later, 
the token buffer's content will just be\footnote{If you use 
    `&\DoWithAllOf&\typein\codeLB a\codeSP a&\a\codeLB a\codeRB\codeRB'
    %% <- \cs bad 2015/11/14
    instead, the entire token sequence \ref{eq:retro} will appear in the 
    token buffer ``at once."}
\begin{equation}
  \ntok{typein}.\codelb.&a.\coderb\,\ntok{expandafter}\,\ntok{DoWith}\,
  \ntok{expandafter}\,\ntok{typein}\,\ntok{fi}
\end{equation}
Next $\ntok{typein}.\codelb.&a.\coderb$ is expanded according to the code for 
`\typein' in \file{latex.ltx}. Some unexpandable tokens will emerge 
and be moved into the ``instruction buffer,"                %% cmd -> instr 2012/11/04
and you should get a screen
message with `a' and a prompt. When you have entered something, 
the remaining \ntok{expandafter} tokens and the \ntok{fi} will be 
removed from the character buffer, and it contains only
\begin{equation}
  \ntok{DoWith}\,\ntok{typein}
\end{equation}
Another token is ordered from the tokenizer to provide a second 
argument for expanding \ntok{DoWith}. The token .\codesp\ comes in, 
but that doesn't serve as a macro argument. It is removed, and the 
next token is .&a. The same story as before happens, until the 
named token \ntok{a} is found\pdots
\end{smallpar}

\subsection{Summary of Possible Arglist Items}
\begin{smallpar}
For $0\leq i\leq 15$, let $\Chi_i$ be the set of character tokens 
of category code $i$. $\Chi_1$ is the set of tokens working like 
$\lbtok$, and $\Chi_2$ is the set of tokens working like $\rbtok$.

Let $E$ be the set $\{3,4,6,7,8,11,12,13\}$. 
These numbers are the category codes for 
\meta{math},   \meta{align}, \meta{parameter}, \meta{super}, \meta{sub}, 
\meta{letter}, \meta{other}, \meta{active} respectively. 
Let $\Chi_E$ be the set of character tokens of category code in $E$
(so $\Chi_E=\bigcup_{i\in E}\Chi_i$).

Let $\circ$ be the \emph{concatenation} operation among token lists.\footnote{%
    %% 2012/05/18
    \TODO: Define for representations by maps, or:              %% \ 2012/11/05
    ``Concatenation is about as basic as natural numbers and is 
      understood in terms of axioms rather than by a definition.''---See 
        notes from 2011 (even with attempts with \Wikienref{Category theory})
%         the \wikienref{Sequence}{English} 
        the English Wikipedia for 
        \wikienref{Sequence}{sequences}---\wikideref{Folge (Mathematik)}{German} 
        article too much restricted to maps.}

The following kinds of token lists form a single arglist item, 
i.e., can serve as an argument for an undelimited parameter:
\begin{enumerate}
  \item a \emph{named} token, or the single-token list consisting of it, 
        if you prefer that;
  \item a \emph{character} token from $\Chi_E$ or the list consisting of it; 
  \item a \emph{brace group.} 
        That is a token list meeting the following conditions:
        \inlineitem{1} its \emph{first} token is in $\Chi_1$,
        \inlineitem{2} its \emph{last} token is in $\Chi_2$,
        \inlineitem{3} it has as many occurrences of tokens from $\Chi_1$ as from $\Chi_2$,
        \inlineitem{4} if it is split as $\lambda\circ\rho$, there are not 
            more $\Chi_2$ occurrences in $\lambda$ than $\Chi_1$ occurrences in $\rho$
            (``don't close before opening").
\end{enumerate}
The second claim can be checked with
\begin{equation}
  `\DoWith\typein$#^_a1~\StopDoing'
\end{equation}
%  \begingroup
%   \def\a{A}\MakeNormalHere\# 
%   \DoWith\typein$# ^_a1~\StopDoing
%   \DoWith\typein#1\StopDoing
%  \endgroup
as to what works. 
(The claim is not affected by one or two surprises.)\footnote{Moreover, 
    `&\DoWith&\typein&#1&\StopDoing'              %% \cs bad 2015/11/14
    tells something about ``parameter tokens."} 
Characters with different category codes 
either are not converted into a character token\footnote{\TTbp.~47.} 
or are not accepted as macro arguments. The latter applies to ``brace" tokens 
in $\Chi_1$, $\Chi_2$ and to the single space token \sptok.

As to \emph{``brace groups"}, the third and fourth condition above 
are intended to say that what is between the two outer tokens 
is $\langle$balanced text$\rangle$ in the sense of \TTbp p.~275f. and~385;
i.e., for two tokens $a$, $b$ and a token list $\beta$, 
$(a)\circ\beta\circ(b)$ is a brace group exactly if $a$ is from $\Chi_1$,
$b$ is from $\Chi_2$, and $\beta$ is $\langle$balanced text$\rangle$. 
The conditions are more formal than what I can find in \TTb, 
but still they don't give me an idea of all possibities. 
This should be improved by the following recursive definition:

\begin{trivlist}\item 
B1.~The empty list is balanced text. 
B2.~For any token $t$ not in $\Chi_1$ or $\Chi_2$, 
    the single-item token list $(t)$ is balanced text. 
    (Such a token is either a \emph{named} token or a 
     \emph{character} token from $\Chi_E$ or \emph{the space token}~\sptok.)
B3.~If $\alpha$ and $\beta$ are balanced texts, 
    then $\alpha\circ\beta$ is balanced text.
B4.~If $\beta$ is balanced text, 
    $a$ is from $\Chi_1$, and $b$ is from $\Chi_2$, 
    then $(a)\circ\beta\circ(b)$ is balanced text.
    (This is a brace group, and the only way of getting a brace group.)
B5.~Nothing else is balanced text.
\end{trivlist}

In other words, a token list is a brace group if and only if 
it is balanced text and starts with a token from $\Chi_1$
and ends with a token from $\Chi_2$.\footnote{Again, this may be  more 
    of a draft for a paper, or notes for it, than package documentation.}
\end{smallpar}

\subsection{Summary: ``Commands" Usable with 'dowith'}
In the \hyperref[sec:apply]{implementation section,} 
you learn about 
\[`\DoWith<cmd>',\quad `\DoWithAllOf<cmd>',\quad 
    \mbox{and}\quad  `\DoWithAllIn<cmd>'.\] 
(\LaTeX\ users may type `{<cmd>}' instead.)
What <cmd>s are allowed?

\begin{enumerate}
  \item All \strong{one-parameter macros} <cmd> work this way, 
        unless there are programming mistakes outside 'dowith' 
        (also thinking of arguments that take over control 
         from 'dowith' commands before the argument list is finished). 
  \item \strong{Other one-parameter} ``commands" <cmd> such as
        \TeX\ \strong{primitives} may work---you must think of 
        the fact that surrounding \emph{braces} are added.\footnote{\TODO: %% \ 2012/11/05
            in the future, variants not adding braces could be added.}
        So the \strong{primitives} `\hbox' and `\vbox' work, 
        for instance. `\show' is an example that doesn't work at all, 
        it takes the single starting brace token and then confuses 
        `\DoWith'.
  \item Some <cmd>s taking \strong{no argument} may make sense, e.g., 
        for getting 
          \begin{enumerate}
            \DoWithAllOf{\item}{{apples,}{pears,}{peaches}}
          \end{enumerate}
        from 
        \begin{quote}
          &\begin{enumerate}\\
          \null\code{~~}&\DoWithAllOf{&\item}{%
            \codebd{apples,}\codebd{pears,}\codebd{peaches}}\\
          &\end{enumerate}
        \end{quote}
        Recall that `\item' at most takes an \emph{optional} argument.
  \item <cmd> must \strong{not take more than one} parameter. 
        %% see def. \DoWith
        A different package will support multi-parameter macros.
\end{enumerate}

% Rather, ``list" is a term from \Wikienref{computer science} here. 
% It corresponds to the notion of ``sequence" in mathematics 
% and to ``word" with formal languages.
% 
% Especially, we have (\meta{i})~lists of \emph{characters} that 
% the \emph{\TeX} program recieves from files and 
% (\meta{ii})~lists of so-called \emph{tokens} 
% that the \TeX\ program forms from the incoming list of characters 
% and that it works on.
% 
% There are \emph{more} kinds of lists that \TeX\ works on. 
% Here we are dealing with ``\TeX's gullet."\footnote{Cf.\ documentation 
%     of the \ctanpkgref{bitelist} package.}
% \TeX's mouth processes tokens formed from character input. 
% It may turn a list `<toks-a><toks_b><toks-c>' of tokens 
% into a list `<toks-a><toks-B><toks-c>', i.e., 
% replace <toks-b> by <toks-B>---by so-called ``expansion". 
% The result may be subject to expansion as well. 
% When nothing is left to be expanded, results are passed to another, 
% more interior subprocessor of \TeX. This one is deeper than \TeX's mouth, 
% we are not concerned with that here.
% 
% Not \emph{all} of those tokens in ``\TeX's mouth" are formed 
% (``directly") from input characters. 
% Rather, some \cs{csname} function may form \emph{new} tokens 
% from other tokens in \TeX's mouth---and place them there again. 
% This way characters `\csname a\endcsname' you type may just work like 
% `\a'.
% 
% It is difficult to tell on the \emph{character level} what the 
% present package does. The relation between incoming characters 
% and resulting tokens can hardly be explained by a single sentence or so.
% However, <cmd> here refers to a command <macro> with a single 
% undelimited macro. More precisely, a certain character sequence 
% (``string") in the code you type, will be converted into a TODO
% 
% \subsection{Separators}
% TODO
% 
% \subsection{Iterators and \TeX's Mouth}
% TODO
% 
\section{Similar Commands in other Packages}
%% <- section again 2012/05/17b
\subsection{``Heavy" Packages}                      %% add. heading 2012/05/17b
The \CtanPkgRef{etex}{$\varepsilon$-TeX}-related    %% e->... 2012/05/09
packages \ctanpkgref{etextools} 
(Florent Chervet), \ctanpkgref{etoolbox} (Philipp Lehman), 
and \ctanpkgref{texapi} (Paul Isambert) seem to include and 
(very much) extend the functionality of 'dowith'. 
Also the `\ForEach'\texttt{\dots\unkern\@} macros of            %% \@ 2012/06/03
\ctanpkgref{forarray} (Christian Schr\"oppel) seem to extend 
the present `\DoWith'\texttt{\dots\unkern\@} commands.          %% \@ 2012/06/03
Moreover, Ahmed Musa describes such commands as                 %% 2012/06/03
``Parsing \qtd{tsv} lists" in documenting his 
\ctanpkgref{catoptions} package.
\ctanpkgref{moredefs} (Matt Swift) provides list handling commands 
like the few that are here.\footnote{\ctanpkgref{arrayjobx}%    %% % 2012/11/05
    provides somewhat ``exotic" handling of ``lists".}          %% 2012/05/10
%% 2012/11/05:
---In October 2012, Ahmed Musa's \ctanpkgref{loops} appeared on 
\acro{CTAN}, offering loops of several ``categories" about as those 
that are listed below, very elaborate.% 
%% /2012/11/05
% (I~do not want to load that much.)
%% <- 2012/11/04 ->
---I do not want to load that much. I need and only need something 
excessively simple, very few lines of code, as presented in 
Section~\ref{sec:implement}. The next sections somewhat point 
out single features of loop constructs that I do not want to have.

\subsection{Separators}                             %% add. heading 2012/05/17b
Regarding \LaTeX\ macros in `latex.ltx', the basic macro `\DoWith' of 
the present package resembles |\@tfor| very much, which likewise deals 
with lists without separators. By contrast, \LaTeX's |\@for| deals with 
\emph{comma-separated} lists (such as lists of package options). 
With comma-separated lists, a ``string" of characters counts as 
an item when it is delimited by commas, or by a comma and the 
list ``border," or spaces may be used as separators additionally. 
However, when \LaTeX\ analyzes such lists (in ``\TeX's gullet"), 
it uses representations by \emph{character tokens} of them.

%% moved here 2012/05/17b:
% %% 2011/11/10:
% Also Heiko Oberdiek's \ctanpkgref{zref} deals with ``lists" of 
% ``properties" of ``entities," 
%% 2011/11/11: comma separated!

%% 2012/05/09f.:
The more recent \ctanpkgref{lmake} 
(Shengjun Pan)                                      %% 2012/05/18
provides a key-value syntax for printing lists of 
complex mathematical expressions easily (using some assignments) 
as well as defining commands according to a pattern from a list.
Those lists are comma-separated.

\subsection{``For" Loops vs.\ ``Foreach" Loops}     %% mv. up 2012/05/20
What about \ctanpkgref{forloop} (Nick Setzer), 
\ctanpkgref{multido} (Timothy Van Zandt, Rolf Nie\-praksch, Herbert 
Vo\ss), and \ctanpkgref{xfor} (Nicola Talbot)? 

'xfor' is just a reimplementation of `\@for'. 
'forloop' and 'multido' are more close to ``real \qtd{for} loops"
(cf.\ \wikienref{for loop}{\meta{Wikipedia}}).
Loops of the latter kind go through a certain set as well, 
but such sets rather consist of \emph{numbers} and are exhausted 
by incrementing (or also decrementing) variables 
\wikienref{Loop counter}{(counters).}
This is essentially not needed 
(neither helpful)                                       %% 2012/05/20
when a list literally is 
\emph{enumerated}---such loops are distinguished as 
\wikienref{Foreach loop}{``foreach loops."}             %% 2012/05/19

\subsection{Iterators}                              %% add. heading 2012/05/17b
So `\DoWith' and |\@tfor| rather provide ``foreach" loops. 
A major difference between them is that the latter uses a 
\wikienref{For loop}{``loop variable"} or 
%%% rather                                              %% rm. 2012/05/20
\wikienref{Iterator}{``iterator"} to which the elements 
of the list are assigned. 
`\DoWith<cmd>' does not use such a loop variable 
or such assignments and thus is ``expandable" at least when <cmd> 
(and the elements, depending on <cmd>) are expandable.
On the other hand, `\@tfor' applies some procedure to the list 
elements without needing a \emph{name} for the procedure 
(or a \emph{macro} storing the procedure). 
%
I wondered whether behind \LaTeX's 
`\@tfor' (and `\@for') there was an ``ideological" consideration 
such as ``A loop must have a loop variable!"\pdots
% However, avoiding usage of a macro name 
% (to store the ``loop body" code)                        %% 2012/05/18
% and a macro parameter 
% (to incorporate the list item into the body code)       %% 2012/05/18
% may have been a good reason.
%% <- 2012/05/19

%% 2012/05/20:
Hopefully more clearly on ``loop variable" vs.\ our approach: 
In order to run
\[<code-before><item><code-after>\]
on each <item> of a <list>, \emph{we here}
% \[
\begin{equation}
  \mbox{define} \quad `\do' \quad \mbox{as} \quad 
  `#1'\;\to\;`<code-before>#1<code-after>'
\end{equation}
% \]
and then run `\do{<item>}' for each <item> in <list>,\footnote{Cf.~description 
    of procedure in terms of tokens in Section~\ref{sec:steps}.} 
\begin{equation}
  \mbox{always replacing}\quad  `\do{<item>}'\quad \mbox{by}\quad  `\do{<item>}\do'. 
\end{equation}
(`\do' is only an example command that 'dowith' supports especially.)
% The ``received" approach from ``usual" programming languages, 
% \file{latex.ltx}, and its followers is
In \file{latex.ltx} instead, we find things like
% \[
\begin{equation}
  \label{eq:tfor}
  `\@tfor\@tmp:=<list>\do{<code-before>\@tmp<code-after>}'
\end{equation}
% \]
where `\@tmp' is a \emph{macro} that is set to be <item> at each 
iteration of the loop, by 
\begin{equation}
  `\def\@tmp{<item>}'
\end{equation}
within `\@tforloop'. After that,
\begin{equation}
  \label{eq:code-tmp-code}
  `<code-before>\@tmp<code-after>'
\end{equation}
from \ref{eq:tfor} is run.---\ref{eq:code-tmp-code} like 
\ref{eq:tfor} is stored in a larger macro. `\do' in \ref{eq:tfor}
does not act as a macro, it just delimits a macro parameter 
in order give a feeling of some familiar programming structure. 
This organisation of macros is fine when the loop body code 
is only used by the containing macro, while the 'dowith' approach 
to store the ``loop body" in an own macro has been useful when 
the loop body code also is used for different purposes 
or when it has been introduced before I thought of using it 
in a loop.
% \[\mbox{defining}\;`\@tmp'\;\mbox{as}\;<item>\quad \mbox{and run} \quad 
%   `<code-before>\@tmp<code-after>'\]
% for each <item> in <list>.

Note that this only was an example. In general, <item> may appear 
more than once in the ``loop body."

``Expandability" by \emph{avoiding} something iterating `\def\@tmp{<item>}'
and doing iteration in \TeX's gullet (`\do' or so must have been defined earlier)
is essential especially within \cs{write}. 
Assignments do not work there. A major motivation for developing 
'dowith' developed with the \ctanpkgref{blog} package 
that \cs{write}s \acro{HTML} code.
Assignments happen in ``\TeX's stomach."                    %% 2012/11/04
That place might be called the 
``instruction buffer" to which                              %% cmd -> instr 2012/11/04
the ``expansion processor" moves items from the incoming token buffer 
that cannot be expanded (any more).

\subsection{Separator Macros}                       %% add. heading 2012/05/17b
Commands like `\DoWith' also could save tokens thinking of list macros 
(in \LaTeX/`latex.ltx') that use a \emph{separator macro} 
which may be used as a \emph{command} to be applied to the list 
elements. One example is 
`\dospecials' that already is in Plain \TeX\ and expands to 
\[`\do\ \do\\\do\{\do\}\do\$\do\&\do\#\do\^\do\_\do\%\do\~'\]
% \begin{verbatim}
% `\do\ \do\\\do\{\do\}\do\$\do\&\do\#\do\^\do\_\do\%\do\~'
% \end{verbatim}
An important application of `\dospecials' is temporarily 
switching off the ``special" functionality of the ``elements" 
in `\dospecials'. With \LaTeX, this may happen thus:
\[`\let\do\@makeother\dospecials'\]
% \begin{verbatim}
% `\let\do\@makeother
% \end{verbatim}
With 'dowith', you can do the same with a shorter variant 
`\specials' of `\dospecials', defined by 
\[`\def\specials{\ \\\{\}\$\&\#\^\_\%\~}'\]
and then 
\[`\DoWithAllIn\@makeother\specials'\]
`latex.ltx' uses `\@elt' instead of `\do' for its own list macros.

\subsection{Ye Olde \cs{loop}}                      %% add. heading 2012/05/17b
%% added 2011/11/03:
There also is |\loop<loop-body>\repeat| in Plain \TeX\ and a 
refined\footnote{Using Kabelschacht's suggestion, cf.\ Section~\ref{sec:kabel}} 
version of it in `latex.ltx'. It is \emph{not} expandable 
since it starts with an assignment for `\body' (Plain \TeX) or 
`\iterate' (`latex.ltx'), 
%% add. 2012/05/20:
and then some assignments are needed to stop the loop, 
such as incrementing or decrementing a \emph{counter.} 
As to the programming structure, it is very simple and general, 
I think any kind of loop can be implemented by this
(apart from nested loops). E.g., I realize\footnote{2012-05-20}
that even a ``foreach" loop could be implemented by managing 
a list macro, e.g., using \LaTeX's internal `\@next'.
%% rm. 2012/05/20:
% that you cannot probably
% what kind of loops it addresses. However, the applications 
% I have seen have been ``for" or (rather) ``while" loops. 
% ``While" loops can ``emulate" ``for" and ``foreach" loops 
% by having the ``incrementation" method or the ``enumeration" method in 
% their body. This is quite obvious for ``for" loops, not quite so for 
% ``foreach" loops; which for practical application (in my view) means 
% that neither \LaTeX/\TeX's `\loop' macro nor in general ``while" loops 
% is/are very helpful for implementing ``foreach" loops, 
% as rather `\DoWith' and similar constructions are.
% The reason for this is (as it seems to me) is that you 
% (a human being) can much more easily enumerate (``list") 
% the items of a list (you have in mind) than define the \emph{method} 
% that (allegedly) is behind your enumeration. \ \meta{Example:} 
% \[`\DoWithAllOf{\printsamplearea}{\red\green\blue}'\]
% ---\emph{how} (according to what ``method"?) did you ``proceed" from 
% `\red' to `\green' and from `\green' to `\blue'?

\subsection{Without Iterator and Separators}        %% add. heading 2012/05/17b
%% 2011/11/07:
In \LaTeX's \ctanpkgref{tools} bundle, \ctanpkgref{xspace}
%% add. 2012/05/18:
was developed in the nineties by David Carlisle. 
It had a rather fixed exception list implemented by a deeply
nested conditional. In 2004 Morton H\o gholm joined, 
and now 'xspace' has a list macro 
`\@xspace@exceptions@tlp'                           %% was xpspace 2012/05/17b
without separators. 
It is handled like here, except that it ``breaks" the loop 
when an item is found that applies.
%% add. 2012/05/17bf.:
After the ``next" token is stored by the usual \cs{futurelet}, 
the exception list is searched without using an iterator. 
Addition and removal commands are provided as well. 

%   \pagebreak                                                %% 2012/05/17b
% \newpage                                                    %% 2012/05/19
\section{Implementation}                                    %% 2012/05/10
\label{sec:implement}
\subsection{Package File Header (Legalese)}                 %% sub 2012/05/10
\ProvidesFile{dowith.tex}[2015/11/14 documenting dowith.sty]
\title{%%%\kern-\baselineskip
       \textsf{\huge dowith.sty}\\---\\Apply Command to
%        Elements of Lists without Separators\,---\,%%% 2012/05/14
%        and without Iterator\thanks{This
       %% 2012/05/15, "in" 2012/05/18:
       Each Item \\ in a List of Arguments in ``\TeX's Gullet"\thanks{This
       document describes version
       \textcolor{blue}{\UseVersionOf{\jobname.sty}}
       of \textsf{\jobname.sty} as of \UseDateOf{\jobname.sty}.}}
{ \RequirePackage{makedoc} \ProcessLineMessage{}
  \renewcommand*\mdSectionLevelOne{\string\subsection}
  \renewcommand*\mdSectionLevelTwo{\string\subsubsection}
  \MakeJobDoc{18}%% 2012/11/05 for v0.3
  {\SectionLevelTwoParseInput}  }
% \RequirePackage[ir]{inputtrc}
\documentclass[fleqn]{article}%% TODO paper dimensions!?
\ProvidesFile{makedoc.cfg}[{2013/03/25 documentation settings}] 
%%
\author{Uwe L\"uck\thanks{%
        \url{http://contact-ednotes.sty.de.vu}}}
%%
%% 'hyperref':
\RequirePackage{ifpdf}
\usepackage[%
  \ifpdf
%     bookmarks=false,                  %% 2010/12/22
%     bookmarksnumbered,
    bookmarksopen,                      %% 2011/01/24!?
    bookmarksopenlevel=2,               %% 2011/01/23
%     pdfpagemode=UseNone,
%     pdfstartpage=10,
    pdfstartview=FitH,                  %% 2012/11/26 again
%     pdfstartview=0 0 100,             %% 2011/08/22
%     pdfstartview={XYZ null null 1},   %% 2011/08/25
%     pdfstartview={XYZ null null null},%% 2011/08/25
%     pdfstartview={XYZ null null .5},    %% 2011/08/26
%     pdffitwindow=true,          %% 2011/08/22
    citebordercolor={ .6 1    .6},
    filebordercolor={1    .6 1},
    linkbordercolor={1    .9  .7},
     urlbordercolor={ .7 1   1},   %% playing 2011/01/24
  \else
    draft
  \fi
]{hyperref}
\hypersetup{% 
    pdfauthor={Uwe L\374ck}% 
}
%% metadata, |\MDkeywords{<text>}|, |\MDkeywordsstring|:
%% %% 2011/08/22:
\makeatletter
  \newcommand*{\MDkeywords}[1]{%
    \gdef\MDkeywordsstring{#1}%
    \hypersetup{pdfkeywords=\MDkeywordsstring}%% TODO!?
  }
  \@onlypreamble\MDkeywords
%% |\MDaddtoabstract{<par-head>}|, `:' added:
  \newcommand*{\MDaddtoabstract}[1]{%           %% 2012/05/10
    \par\smallskip\noindent
    \strong{#1:}\quad\ignorespaces}
%% \pagebreak[2]
%% |\printMDkeywords|:
  \newcommand*{\printMDkeywords}{%
    \MDaddtoabstract{Keywords}%
    \MDkeywordsstring 
%     \global\let\MDkeywordsstring\relax    %% `%' 2012/11/12
  }
%% The previous definitions mainly are useful with a variant 
%% |\begin{MDabstract}| of \LaTeX's `{abstract}' environment:
  \newenvironment{MDabstract}
                 {\abstract\noindent
                  \hspace{1sp}%% for niceverb
                  \ignorespaces}
                 {\@ifundefined{MDkeywordsstring}%
                               {}%
                               {\printMDkeywords}%
                  \global\let\MDabstract\relax    %% 2012/11/12
                  \global\let\endMDabstract\relax %% 2012/11/12
                  \endabstract}
%% |\[MD]docnewline| 2012/11/12 from `readprov.tex':
  \newcommand*{\MDdocnewline}{\leavevmode\@normalcr[\topsep]}
%% <- `\leavevmode' for use with `\paragraph'.
%%    Sometimes needs to be preceded by a space.
%% 
%% |\MDfinaldatechecks[<tex-script>]| with \ctanpkgref{filedate}:
  \newcommand*{\MDfinaldatechecks}[1][fdatechk]{%
    \AtEndDocument{%
%       \clearpage %% 2013/03/25 no avail -- with `filedate'!
      \def\@pkgextension{sty}%
      \def\NeedsTeXFormat##1[##2]{}%
      \noNiceVerb                       %% 2013/03/22
      \input{#1}%
    }}
  \@onlypreamble\MDfinaldatechecks
\makeatother
%% Use other packages:
\RequirePackage{niceverb}[2011/01/24] 
\RequirePackage{readprov}               %% 2010/12/08
\RequirePackage{hypertoc}               %% 2011/01/23
\RequirePackage{texlinks}               %% 2011/01/24
\RequirePackage{relsize}                %% 2011/06/27
\RequirePackage{color}                  %% 2011/08/06
\RequirePackage{lmodern}                %% 2012/10/29
\RequirePackage{filedate}               %% 2012/11/12
\RequirePackage{filesdo}                %% 2013/03/22 
%% \pagebreak[3]
%% Logical markup:\qquad  |\strong{<chars>}|, |\meta{<chars>}|, 
%% |\acro{<chars>}|, |\pkg{<chars>}|, 
%% |\code{<chars>}|, |\file{<chars>}|:{\sloppy\par}
\makeatletter
  \def\do#1#2{\@ifdefinable#1{\let#1#2}}%% 2012/07/13
  \do\strong\textbf \do\file\texttt \do\acro\textsmaller 
  %% <- wrong tests before 2012/07/13
  \do\meta\textit   \do \pkg\textsf \do\code\texttt
  \ifpdf
    \pdfstringdefDisableCommands{%
        \let\acro\textrm 
        \let\file\textrm                            %% 2011/11/09
        \let\code\textrm                            %% 2011/11/20
        \let\pkg \textrm                            %% 2012/03/23
    }
  \fi
  %% TODO 2011/07/22 -> `htlogml.sty'
\makeatother
%% |\qtdcode{<text>}|: 2012/10/24:
    \newcommand*{\qtdcode}[1]{`\code{#1}'} 
%% |\pkgtitle{<package-name>}{<caption>}| 
\newcommand*{\pkgtitle}[2]{%            %% 2012/07/13
    \global\let\pkgtitle\relax
    \pkg{\huge #1}\\---\\#2\thanks{This 
       document describes version 
       \textcolor{blue}{\UseVersionOf{\jobname.sty}} 
       of \textsf{\jobname.sty} as of \UseDateOf{\jobname.sty}.}}
%% TODO: %% |\TODO| bad with `mdoccorr.cfg'
\newcommand*{\TODO}{\textcolor{blue}{\acro{TODO}}}  %% 2012/11/06
%% `\MDsampleinput[{<file>}' was added 2012/11/06. 
%% Problems with `myfilist.tex' were due to 'parskip.sty'
%% there. On 2012/11/12, we change the former simple macro to a 
%% much more complex
%% |\MDsamplecodeinput[<add-hfuss>]{<file>}| 
\newcommand*{\MDsamplecodeinput}[2][]{%
    \begingroup
        \vskip\bigskipamount \hrule
        \nobreak\vskip-\parskip 
%         \nobreak\vskip\medskipamount
%% Previous mistake (same below) due to manual change 
%% of `\topsep' in the file `myfilist.tex' (2012/11/30).
        \ifx\\#1\\\else
            \hfuzz=\textwidth \advance\hfuzz#1\relax
        \fi
        \noNiceVerb \verbatiminput{#2}%
%         \nobreak\vskip\medskipamount 
        \hrule \vskip-\parskip 
        \bigskip %%% \bigbreak
%% `\bigbreak' made much larger space in `myfilist.tex'.
    \endgroup
}
%% |\ctanpkgdref{<pkg-id>}| adds the printed link to 
%% `ctan.org/pkg' as a footnote. There is a little space 
%% for coloured link borders:
\newcommand*{\ctanpkgdref}[1]{%
    \ctanpkgref{#1}\,\urlfoot{CtanPkgRef}{#1}}
\errorcontextlines=4
\pagestyle{headings}

\endinput 
    %% shared formatting settings
\ifpdf\else \errhelp{hyperref draft bad with {equation}}
            \errmessage{run this with pdflatex only}\fi
% \ReadPackageInfos{dowith}
\usepackage{dowith}                                         %% 2012/05/17b
\sloppy
\MDkeywords{programming structures;                         %% 2012/05/14b
            macro programming, loops, list macros}
\hypersetup{%
    pdftitle=dowith.sty handles lists without separators,
    pdfsubject=documenting dowith.sty
}%% 2011/08/22 
\usepackage{fixltx2e}                       %% \textsubscript 2012/05/17b
%% <- TODO with `lmodern'?
% \newcommand*{\ctanpkgdref}[1]{%                             %% 2012/11/19
%     \ctanpkgref{#1}\,\foothttpurlref{ctan.org/pkg/#1}}
%% <- in `makedoc.cfg' 2015/11/14
\makeatletter                                   %% etc. 2012/05/17b
%% TODO `actcodes'!? 2012/11/02
  \newcommand*{\GetOtherChar}[2]{% 
    \@ifdefinable#1{%
        \edef#1{\expandafter\@gobble\string#2}}}
\makeatother
\GetOtherChar\codeLB\{ 
\GetOtherChar\codeRB\} 
% \GetOtherChar\codeSP\ %       %% ???
{\MakeOther\ \xdef\codeSP{ }}
\newcommand*{\codelb}{\code\codeLB}
\newcommand*{\coderb}{\code\codeRB}
\newcommand*{\codesp}{\code\codeSP}
\newcommand*{\codebd}[1]{\code{\codeLB#1\codeRB}}       %% 2012/05/19
%% rm. 2012/05/17b:
% \DeclareRobustCommand*{\code}[1]{%
%   \texttt{%
%     \let\{\codeLB \let\}\codeRB \let\ \codeSP
%     #1}}
% \newcommand*{\chtok}[2]{\ensuremath{\code{#1}\sb{#2}}}
%% <- 2012/05/17b ->
\newcommand*{\chtok}[2]{\code{#1}\textsubscript{#2}} 
\newcommand*{\lbtok}{\chtok\codeLB{1}}
\newcommand*{\rbtok}{\chtok\codeRB{2}}
\newcommand*{\lttok}[1]{\chtok{#1}{11}}
\newcommand*{\sptok}{\chtok\codeSP{10}}
\providecommand*{\TTb}{\meta{The\nolinebreak[3] \TeX book}}
\providecommand*{\TTbp}{\TTb\nolinebreak[3] p}
\newenvironment*{smallpar}
                {\medskip\par\begingroup\footnotesize}
                {\par\endgroup\medskip}
\newcommand*{\NTOK}[1]{\textsf{ntok}(\code{#1})}
\newcommand*{\ntok}[1]{\textrm{?}\code{#1}}
\newenvironment*{example}[1]
                {\trivlist\item
                 (\ulroman{#1})}
                {\endtrivlist}
\providecommand*{\ulroman}[1]{\meta{\romannumeral #1 }}
\newcommand*{\inlineitem}[1]{\ (\ulroman{#1})\enspace
             \ignorespaces}
\newcommand*{\pdots}{~.\kern\fontdimen3\font
                      .\kern\fontdimen3\font. }
\providecommand*{\Chi}{\mathrm{X}}
\renewcommand*{\httpprefix}{\theHTTPprefix}
\begin{document}
\maketitle
\begin{MDabstract}
This package provides macros for applying a 
%% mod. 2012/05/15:
% ``command" <cmd> to all items of a list $<arg-1><arg-2>\dots<arg-$n$>$
``command" to all items in a ``list of possible macro arguments," %% v0.21a
% in ``\TeX's mouth," 
% such as `\DoWithAllIn{<cmd>}{<list-macro>}', 
and also for extending and reducing macros storing such lists.
``Brace groups" are single items of such lists, as opposed to 
token lists. Iteration is implemented within \TeX's expansion 
processor, so works within `\write' as with 'blog.sty'.
 %%% ---pleading for mathematical rigour in \TeX ology!)    %% rm. 2012/05/17b
% Applications in mind belonged to \LaTeX, but the package should work 
% with other formats as well.
Loop and list macros in other packages are discussed in the documentation.
% %% 2012/05/09:
% There is an emphasis on expandability 
% %% mod./add. 2012/05/15:
% (no \wikienref{Iterator}{iterator;} 
%  essential within \cs{write} as with \CtanPkgRef{morehype}{blog.sty}),
% without relying on \CtanPkgRef{e-tex}{$\varepsilon$-\TeX}.
%% 2012/05/18:
% Iteration is implemented within ``\TeX's mouth," so works within 
% `\write' as with \CtanPkgRef{morehype}{blog.sty}. 
There is no need for \CtanPkgRef{e-tex}{$\varepsilon$-\TeX} to which 
some of them refer.

The package is ``generic," i.e., should also work with Plain \TeX\ 
or even other formats, relying on the \ctanpkgref{plainpkg} package 
for some minimal \LaTeX-like behaviour.     %% \ 2012/11/27

\MDaddtoabstract{Related packages}
\let\pkg\ctanpkgref     \pkg{catoptions},
\pkg{etextools}, \pkg{etoolbox}, \pkg{forarray}, \pkg{forloop}, 
\pkg{loops}, \pkg{multido}, \pkg{moredefs}, \pkg{lmake}, 
\pkg{texapi}, \pkg{xfor}, \pkg{xspace}
\end{MDabstract}
\tableofcontents

%   \newpage
\section{Usage and Features}                    %% restructured 2012/05/16

\subsection{Installing and Calling}
The file 'dowith.sty' is provided ready, installation only requires
putting it somewhere where \TeX\ finds it
(which may need updating the filename data
 base).\urlfoot{ukfaqref}{inst-wlcf}           %% corr. 2011/02/08
The packages \ctanpkgdref{plainpkg}
and 'stacklet' (\ctanpkgref{catcodes})\urlfoot{ctanpkgref}{catcodes} 
must be installed as well.

As to calling (loading): 'dowith' is a ``\pkg{plainpkg} package" 
in the sense of the \ctanpkgref{plainpkg}
documentation that you may consult for details.
So roughly, 
\begin{itemize}
  \item load it by \ |\usepackage{dowith}| \ if you can, 
  \item otherwise by \ |\RequirePackage{dowith}| \\
        (perhaps from within another ``\pkg{plainpkg} package"), 
  \item or by \ |\input dowith.sty| 
  \item or even by \ |\def\filename{dowith}     \def\fileinfo{simple list loop (UL)}
\def\filedate{2012/11/05} \def\fileversion{v0.3}
%%
%% Copyright (C) 2011 2012 Uwe Lueck,
%% http://www.contact-ednotes.sty.de.vu
%% -- author-maintained in the sense of LPPL below --
%%
%% This file can be redistributed and/or modified under
%% the terms of the LaTeX Project Public License; either
%% version 1.3c of the License, or any later version.
%% The latest version of this license is in
%%     http://www.latex-project.org/lppl.txt
%% We did our best to help you, but there is NO WARRANTY.
%%
%% Please report bugs, problems, and suggestions via
%%
%%   http://www.contact-ednotes.sty.de.vu
%%
%% == Proceeding without \LaTeX ==
%% v0.3 mainly replaces imitating the 
%% \CtanPkgRef{german}{german.sty} approach to genericity 
%% by \ctanpkgref{plainpkg}:
\input plainpkg
\ProvidesPackage{\filename}[\filedate\space 
                 \fileversion\space \fileinfo]
\PushCatMakeLetterAt
%% If \LaTeX\ is not present ... 
\ifltx \else
%% ... an old version of its `\in@' is introduced. 
%% It is bad as a subword test 
%% (false positive cases, cf. \CtanPkgRef{nicetext}{fifinddo} 
%%  documentation), but 'dowith' will check for single tokens only. 
%% If \LaTeX\ \emph{is} present, 
%% on the other hand, `\ifin@' is recognized while skipping 
%% `false' parts of conditionals, without being matched 
%% by some `\fi' before the next `\else', so I hide it by `\csname':
  \expandafter\newif\csname ifin@\endcsname
  \def\in@#1#2{%
    \def\in@@##1#1##2##3\in@@{%
      \ifx\in@##2\in@false\else\in@true\fi}%
    \in@@#2#1\in@\in@@}
\fi
%% 
%% == Applying a Command ==                             %% 2011/11/07
%% \label{sec:apply}
%% === Core ===
%% \label{sec:core}
%% |\DoWith{<cmd>}<list>\StopDoing| applies <cmd> to all elements 
%% of <list>. An element of <list> (after tokenizing) 
%% may be either a single token or a group `{<balanced>}'.
\def\DoWith#1#2{%
    \ifx\StopDoing#2\empty
%% The previous `\empty' (replacing `%') is a bug fix as of v0.22 
%% (June 2012),                                             %% 2012/11/05
%% while in my extension draft I already had it in January 2012. 
%% It allows ``empty" arglist items \qtd{\lbtok\rbtok}. 
%% Before v0.22, such an empty brace group would have resulted 
%% in comparing `\StopDoing' with `\else', so \qtd{\lbtok\rbtok}
%% would have had the same effect as `\StopDoing', the token text 
%% after `\else' until `\fi' would have been skipped. 
%% Instead, the user may have a reason to allow empty arguments\slash
%% brace groups.
        \else#1{#2}\expandafter\DoWith\expandafter#1\fi}
%% |\StopDoing| delimits the list: 
\let\StopDoing\DoWith
%% ... something arbitrary that is not expected to occur in a list.
%% With \[`\let\StopDoing*'\] instead, the star would end lists.
%%
%% |\DoWithAllOf{<cmd>}{<list>}| works like 
%% \[`\DoWith{<cmd>}<list>\StopDoing':\]
\def\DoWithAllOf#1#2{\DoWith#1#2\StopDoing}
%% 
%% === `&\do' being the Command ===
%% \label{sec:do}
%% When the <list> is worked at a single time in the \TeX\ run 
%% where assignments are possible, instead of introducing a new 
%% macro name for <cmd> you can use `\do' for <cmd> as a 
%% ``temporary" macro and define it right before 
%% \[`\DoWith{\do}<list>\StopDoing'\] 
%% However, we provide \[|\DoDoWith{<cmd>}<list>\StopDoing|\] 
%% as a substitute for the former line that at least saves one token. 
%% For the definition of `\do', we provide
%% |\setdo{<def-text>}|.
%% It works similarly to \[`\renewcommand{\do}[1]{<def-text>}',\] 
%% so <def-text> should contain a `#1': 
\def\setdo{\long\def\do##1}
%% With |\letdo<cmd>| that is provided next where <cmd> is defined 
%% elsewhere, you could type 
%% \[`\letdo<cmd>\DoDoWith<list>\StopDoing'\]
%% It seems to me, however, that you better type 
%% \[`\dowith<cmd><list>\StopDoing'\] instead. 
%% So I provide `\letdo' although I consider it useless here. 
%% It is provided somewhat for the sake of ``completeness," 
%% thinking that it might be useful at other occasions such as 
%% preceding `\dospecials'.
\def\letdo{\let\do}
%% |\DoDoWith| has been described above:
\def\DoDoWith{\DoWith\do}
%% By analogy to `\DoWithAllOf', we provide 
%% |\DoDoWithAllOf{<list>}|:
\def\DoDoWithAllOf{\DoWithAllOf\do}
%%
%% === Expand List Macro ===
%% The former facilities may be quite useless
%% as such a <list> will not be typed at a single place in the 
%% source code, rather the items to run <cmd> on may be collected 
%% occasionally when some routines run. The elements may be collected 
%% in a macro <list-macro> expanding to <list>. 
%% So we provide \[|\DoWithAllIn{<cmd>}{<list-macro>}|\]
%% (or `\DoWithAllIn<cmd><list-macro>').
%% There is no need to type `\StopDoing' here:
\def\DoWithAllIn#1#2{%
    \expandafter\DoWith\expandafter#1#2\StopDoing}
%% |\DoDoWithAllIn{<list-macro>}| saves a backslash or token 
%% for `\do' as above in Sec.~\ref{sec:do}:
\def\DoDoWithAllIn{\DoWithAllIn\do}
%% 
%% 
%% == Handling List Macros ==
%% === Initializing ===
%% Here is some advanced `\let<cmd>\empty', perhaps a little 
%% irrelevant for practical purposes. Both
%% \[|\InitializeListMacro{<list-macro>}|\] and 
%% \[|\ReInitializeListMacro{<list-macro>}|\] attempt to ``empty" 
%% <list-macro>, and when we don't believe that \LaTeX\ has been 
%% loaded, both do the same indeed. Otherwise the first one 
%% complains when <list-macro> seems to have been used earlier 
%% while the second complains when <list-macro> seems \emph{not} to 
%% have been used before:
\ifltx                                                  %% v0.3
  \def\InitializeListMacro#1{\@ifdefinable#1{\let#1\empty}}
  \def\ReInitializeListMacro#1{%
    \edef\@tempa{\expandafter\@gobble\string#1}%
    \expandafter\@ifundefined\expandafter{\@tempa}%
        {\@latex@error{\noexpand#1undefined}\@ehc}%
        {\let#1\empty}}
\else
  \def\InitializeListMacro#1{\let#1\empty} %% not \@empty 2011/11/07
  \let\ReInitializeListMacro\InitializeListMacro
\fi
%% |\ToListMacroAdd{<list-macro>}{<cmd-or>}|
%% appends <cmd-or> to the replacement token list of <list-macro>. 
%% <cmd-or> may either be tokenized into a single token,
%% or it is some `{<balanced>}'.
\def\ToListMacroAdd#1#2{\DefExpandStart#1{#1#2}}
\def\DefExpandStart#1{\expandafter\def\expandafter#1\expandafter}
%% 
%% === Testing for Occurrence of a Token ===
%% |\TestListMacroForToken{<list-macro>}{<cmd>}|
%% sets `\in@true' when <cmd> occurs in <list-macro> 
%% and sets `\in@false' otherwise:                      %% 2011/11/07
\def\TestListMacroForToken#1#2{%
    \expandafter \in@ \expandafter #2\expandafter{#1}}
%% Indeed I removed an earlier `\IfTokenInListMacro', 
%% now it's a kind of compromise between having a shorthand macro 
%% below and a generalization for users of the package.
%% 
%% === Adding and Removing ===
%% |\FromTokenListMacroRemove{<list-macro>}{<cmd>}| 
%% removes the token corresponding to <cmd> from the list 
%% stored in <list-macro> 
%% (our parsing method does not work with braces):      %% 2011/11/07
\def\FromTokenListMacroRemove#1#2{%
%% I am not happy about defining \emph{two} parser macros, 
%% but for now ...
    \TestListMacroForToken#1#2%
    \ifin@
      \def\RemoveThisToken##1#2{##1}%
      \expandafter \DefExpandStart
        \expandafter #1\expandafter {%
          \expandafter\RemoveThisToken #1}% 
%% TODO warning otherwise?
   \fi}
%% %% 2011/11/07:
%% ... but this only removes a single occurrence ... 
%% \[|\InTokenListMacroProvide{<list-macro>}{<cmd>}|\]
%% avoids multiple entries of a token 
%% by \emph{not} adding anything when <cmd>
%% already occurs in <list-macro> (again, this does not work with 
%% braces, try `\in@{{}}{{}}').
\def\InTokenListMacroProvide#1#2{%
    \TestListMacroForToken#1#2%
    \ifin@ \else                    %% TODO warning?
        \ToListMacroAdd#1#2%
    \fi}
%%
%%
%% == Leaving and History ==
\PopLetterCatAt                                         %% v0.3
\endinput

VERSION HISTORY 
v0.1    2011/06/23/28  stored separately
v0.2    2011/11/02     simpler, documented
        2011/11/03     corrected \if/\else for init
        2011/11/07     \TestListMacroForToken, \InListMacroProvide; 
                       doc.: \pagebreak, structure
        2011/11/19     modified LaTeX supplements
v0.21   2012/05/14     fix for "generic" and `typeoutfileinfo': 
                       @ before ...!
v0.21a  2012/05/19     \labels sec:apply, sec:core; \pagebreak?
v0.22   2012/06/04     allow {} items
v0.3    2012/11/05     updating copyright, using `plainpkg', 
                       rewording documentation there
|~\dots
\end{itemize}

% \section{Example}

% \section{Discussion}                                      %% 2012/05/16
% \subsection{What It Seems to Do \dots}
% The 'dowith' package provides tools to simplify \TeX\ macro programming. 
% Understanding it \emph{really} may require understanding certain passages of 
% \TTb, such as pp.~38f. It may even require overcoming 
% terrible confusions in \TTb.
% Let this be a last try at understanding the package without understanding \TeX: 
% It allows you to abbreviate 
% \[<cmd><arg-1><cmd><arg-2>\dots<cmd><arg-$n$>\]
% by 
% \[`\DoWith<cmd><arg-1><arg-2>\dots<arg-$n$>\StopDoing'\]
% or by 
% \[`\DoWithAllOf<cmd>{<arg-1><arg-2>\dots<arg-$n$>}'\]
%
%   \pagebreak                                                %% 2012/05/19
\subsection{What It Does With What Lists}
% \subsection{What It Actually Does \dots}
\label{sec:lists-intuit}

The single commands that the package provides are described 
in the \hyperref[sec:implement]{implementation section} below. 
What follows here is some general background about how the 
commands work.
%% <- moved down from previous section, mod. 2012/05/19

The term \qtd{list} may refer to various things and need clarification here. 

First of all, we are not referring to \LaTeX\ `list' environments 
such as `enumerate' or `itemize'; 
neither to ``\acro{TODO}" lists of what needs to be done soon. 

Rather, 'dowith' allows you to abbreviate 
\[<cmd><arg-1><cmd><arg-2>\dots<cmd><arg-$n$>\]
by 
\[`\DoWith<cmd><arg-1><arg-2>'\dots<arg-$n$>`\StopDoing'\]
or by 
\[`\DoWithAllOf<cmd>{<arg-1><arg-2>'\dots<arg-$n$>`}'\]
With small $n$, one may doubt whether this really is an abbreviation~\dots;
anyway, 
\[<arg-1><arg-2>\dots<arg-$n$>\]
was an attempt to refer to the kind of lists we are dealing with. 
\[<arg-1>, <art-2>, \dots, <arg-$n$>\]
are the ``items" of the list. 
The question is: what counts as an item?

We might say that `aa' is a list of \emph{two} items, 
<arg-1> being `a' and <arg-2> being `a', too.

When we do \emph{three} keystrokes to get `a a' instead of `aa', 
we still have \emph{two} items, 
<arg-1> being `a' and <arg-2> being `a' too. 
Strange, isn't it?

Also, when in `aa' we replace the first `a' by a backslash, `\', 
we get `\a', and this is a list of a \emph{single} item, 
$<arg-1>=`\a'$~\dots

You shouldn't believe these stories of mine entirely. 
What I am alluding to is that the \emph{``items"} 'dowith' is about 
are determined in terms of \emph{\TeX's tokens}, and the relation 
between the ``characters you type" and \emph{\TeX's tokens} 
is not entirely straightforward. 

\subsection{The Notion of Arglists for \LaTeX\ Users}
%% <- 2012/05/17b ->
% \subsection{The Notion of Arglist}
\label{sec:arglists-intuit}
Still, it may suffice to clarify what counts as an <arg-$i$> 
without speaking of \emph{tokens} explicitly: It is simply 
what a \emph{one-parameter macro} 
(where the parameter is \emph{not delimited} in terms of 
 \TTbp p.~203f.)
can take as an \emph{argument.}

The lists 'dowith' is about then are lists \emph{of possible arguments}
in the previous sense---let me call them \emph{``arglists."}\footnote{Not 
    to be confused with German \httpref{de.wiktionary.org/wiki/Arglist}{Arglist.}}
The single \emph{items} of such lists are those single possible arguments. 
They become \emph{actual} arguments beginning from the leftmost \emph{possible} one 
when 'dowith' presents them to that <cmd> mentioned earlier---where 
<cmd> \emph{should} be a one-parameter macro (or some \TeX\ primitive 
parsing arguments similarly).

The reader perhaps has an \emph{intuitive} understanding of 
what can be an argument of a one-parameter macro. 
A \emph{strict \LaTeX} user may think that such an argument <arg-$i$>
just has form `{'<ark-$i$>`}', i.e., $<arg-$i$>=`{'<ark-$i$>`}'$
for some <ark-$i$>. Such arguments are also called \emph{``brace groups".}
(\LaTeX's \emph{optional} arguments `[<extra>]' 
 do not count as possible arguments here, they are not macro arguments 
 in the sense of \TTb.)
In this restricted \LaTeX\ sense, arglists consist of brace groups 
\[`{<ark-1>}{<ark-2>}'\dots`{'<ark-$n$>`}',\]
and each single brace group is an \emph{item} of it. 

The \emph{\TeX\ macro writer}, by contrast, knows that a macro argument 
doesn't need outer braces. In an intuitive sense, a single 
``command" can be a macro argument, too. ``Command" may be understood 
as ``control sequence" (starting with a backslash), 
but some authors also have considered single \emph{characters} 
(character \emph{tokens}?) ``commands." Blank spaces, by contrast, 
are ignored when a macro looks for its argument.    %% 2012/05/17b

%% useless, just name items 2012/05/17b:
% We arrive at an ``intuitive recursive definition" of ``arglist:" \
% (\meta{i})\enspace The empty list is an arglist; it doesn't have an item 
% for 'dowith'. \ (\meta{ii})\enspace If <list> is an arglist, then 
% (a)~`{<ark>}<list>' is an arglist whose first item for 'dowith' is `{<ark>}';
% (b)~`<cs><list>' with a ``control sequence" <cs>
%     is an arglist whose first item is `<cs>' 
%     (for the command <cmd> to which 'dowith' presents <cs>); and
% (c)~`<char><list>' with a non-blank character <char>
%     is an arglist whose first item is `<char>'. \
% (\meta{iii})\enspace Nothing else is an arglist.
% 
% But keep in mind that this ``intuitive" understanding essentially is wrong.

\subsection{Anatomy of \TeX}
The documentation of v0.22 as of 2012-06-04 said that the package 
is about ``lists in \TeX's mouth." However, this was very wrong. 
I believed it following Alan Jeffrey's paper ``Lists in \TeX's Mouth",\footnote{%
    Alan Jeffrey: \tugbartref{tb11-2/tb28jeffrey}{``Lists in \TeX's Mouth,"} 
    \acro{TUG}boat Vol.~11 (1990), No.~2, pp.~237--245), 
    \urlhttpref{tug.org/TUGboat/tb11-2/tb28jeffrey.pdf}.}
in whose Section~2 you read:
\begin{quote}\it
  \TeX's programming facilities come in two forms---there 
  are \TeX's \emph{macros} which are expanded in its
  mouth, and some additional \emph{assignment} operations
  like \cs{def} which take place in the stomach.
\end{quote}
The macros that Jeffrey lists and describes in that article can be 
obtained as a \acro{CTAN} package 
\ctanpkgref{lambda-lists}.\footnote{\urlhttpref{ctan.org/pkg/lambda-lists}} 
If you follow the link given here (in the footnote), 
you currently (2012-11-03) read about this package:
\begin{quote}\it
    These list-processing macros avoid the reassignments 
    employed in the macros shown in Appendix D of the TeXbook: 
    all the manipulations take place in what Knuth is pleased to 
    call ``TeX's mouth".
\end{quote}
But Knuth doesn't. On page 267 of \TTb, you read:
\begin{quote}\it
  Chapter 7 has described the process by which input files
  are converted to lists of tokens in \TeX's ``mouth,"
  and Chapter 20 explained how expandable tokens are converted
  to unexpandable ones in \TeX's ``gullet" by a process 
  similar to regurgitation.
\end{quote}
I.e., the ``mouth" is \TeX's ``tokenizer," the inner part of what 
van Eijkhout calls \TeX's ``input processor" on, e.g., p.~15 of his 
\meta{\TeX~by Topic}.\footnote{It is available as a \acro{CTAN} package 
    \ctanpkgref{texbytopic} at \urlhttpref{ctan.org/pkg/texbytopic}.}
The exact rules the tokenizer follows are described on pp.~46f.\ 
(Chapter~8!) of \TTb.
\emph{Macro expansion} takes place in \TeX's \emph{``gullet"}, 
which van Eijkhout calls \TeX's ``expansion processor" (p.~16). 
Abrahams, Hargreaves, and Berry follow Knuth's terminology on
pp.~16 and 46f.\ of their \meta{\TeX~for the Impatient}.\footnote{It 
    is available as \acro{CTAN} package \ctanpkgref{impatient}, 
    \urlhttpref{ctan.org/pkg/impatient}.} 
% Only Jeffreys and my wrong old documentation say this were 
% ``\TeX's mouth"---and I made the same mistake with the 
% \ctanpkgref{bitelist} package.
%% <- 2012/11/19 ->

\TeX's gullet has been called ``\TeX's mouth" also 
in the documentation of my \ctanpkgdref{bitelist} package and 
in the documentation of the package \ctanpkgdref{bibleref-mouth}.

Moreover, I should have clarified that Jeffrey's paper deals with 
``lists" in some general, rather abstract sense, different from 
the kind of lists the present documentation tries to characterize 
as the objects for 'dowith'.

\subsection{\TeX's Tokens}
\label{sec:toks}

% The \emph{\TeX\ macro writer} understanding \TeX\ properly 
% does not really think of arglists. \TeX perts instead think of 
% What \TeX nically matters is
% what happens in ``\TeX's mouth,"\footnote{Cf.~\TTbp.~46.} 
% as some authors have suggested a metaphor,\footnote{%
The 'dowith' package is a tool that affects the order of tokens in 
\TeX's gullet.

\begin{smallpar}
The ``characters you type" enter ``\TeX's mouth" line by line, 
in a slightly modified appearance. Each line forms a \emph{string}.
\TeX\ takes initial substrings away from it and turns them 
into \emph{tokens} that are appended to the right of \TeX's 
\emph{expansion buffer} (``gullet").

%% Removing 2012/11/04:
% More formally, \TeX\ has a \emph{character buffer.}
% It forms a single token from an initial segment of the buffer content---unless 
% there is a special situation with blank spaces or something pathological.
% When an \emph{escape character}, as the backslash usually is one, 
% has been noticed recently (that isn't followed by another one immediately), 
% the character buffer may need to be feeded from more outside, 
%% <- especially wrong, 2012/11/04
% until it contains enough material to form a token from. 

There are \emph{two kinds of tokens} here: \emph{named} tokens 
and \emph{character} tokens. ``Named" tokens usually are referred
to as ``control sequence tokens" or just ``control sequences"---I~really 
want to avoid those horrible confusions from \TTb.
There never are any ``parameter tokens" in \TeX's gullet 
(perhaps unless one considers a one-step macro expansion 
 a two-or-more-step procedure).
%  merged 2012/11/04
%% Moving down 2012/11/04
The character(s) \emph{after} the escape character until 
some delimiting character form a \emph{string} that is the 
\emph{name} of the token that is formed---a \emph{named} 
token, as I am saying.
% What has been used to form a token is removed from the character buffer.
%% 2012/11/04:
\emph{Character} tokens are formed by removing a character from the 
beginning of the character buffer and appending it to the token buffer 
paired with its \emph{category} code.

For every \emph{string of characters}, there is exactly one 
(possible) \emph{named token} whose name the string is.\footnote{``Possible" 
    refers to the fact that \TeX\ does not store named tokens anywhere 
    before they appear in its gullet, maybe apart from ``primitive" 
    tokens that have a ``pre-assigned meaning" when a \TeX\ run     %% \ 2012/11/04
    starts.---What is more bad with my claim is that 
    the \TeX\ program by design cannot extend its memory arbitrarily---even 
    not using the ``cloud''---, so it doesn't support tokens 
    whose name lengths are above a certain limit.}
It is so common (starting from \TTb) to denote the token whose name 
is <string> by \lq\verb+\+<string>\rq. For instance, the token whose 
name is `input' is denoted by \qtd{&\input}. On the other hand, 
on page~7 of \TTb\ \qtd{&\input} is a ``string of characters."
With this notation, it is already difficult to explain what 
the \LaTeX\ command `\DeclareRobustCommand' does or what the difference 
between a starred \LaTeX\ command and a starred \LaTeX\ environment is.\footnote{%
    A reader knowing \LaTeX\ only thinks that \qtd{\code{&\\\codesp}} 
    is the result of typing a double backslash and a space 
    and that \qtd{\cs{equation*}} is the ``command" \cs{equation} 
    followed by a `*'.} 
\TTb\ makes it worse by saying on page~39: 
\meta{``A control sequence is considered to be a single object 
        that is no longer composed of a sequence of symbols."}
So ``it depends" whether \qtd{&\input} is a string of characters or not---it \emph{is 
before} tokenization, but \emph{no longer} afterwards. 
So if you have two computers and start a \TeX\ run on each of them 
with a little difference in time, there will be a moment where 
\qtd{&\input} is a string on the one computer but not on the other? 
This 
appears to me %%% is %% 2012/11/04
like saying \meta{``When we apply the square root function to 
the number 4, the number 4 will no longer be the number 4, 
it will be the number 2 instead."}
\end{smallpar}

\TTb\ does offer an alternative notation for named tokens: ``boxing;" 
so the token whose name is `input' can be denoted by the rather 
``graphical" notation \qtd{|input|} (used only exceptionally).\footnote{%
    The box notation is introduced on page~38 without explanation, 
    as if it explained something.} 
\emph{I}~would suggest something like 
\qtd{\NTOK{input}} for clarity and \qtd{\ntok{input}} 
for brevity.\footnote{I am suggesting the question mark for named tokens 
 since \TeX\ ``must look up the current definition" of a named token 
 according to \TTbp.~39, while the meaning of character tokens rather 
 is ``fixed," at least according to \TTbp.~39. However, 
    \emph{active}-character such as .&~ are in the same situation 
    as named tokens as to this respect. The dot notation may be fine for them, 
    though.}

\begin{smallpar}
%% rm. 2012/11/04, cf. above
% \emph{Character tokens} get into \TeX's mouth by tokenization 
% when characters begin the buffer content while \emph{not} scanning 
% a name for a named token. A single character then is removed from the 
% character buffer, and a token storing its character code and current 
% category code is pushed into \TeX's mouth.

Named tokens may get into \TeX's gullet by ``tokenization" as described above, 
i.e., they are drawn from the character buffer. But they also can 
appear in \TeX's gullet ``from within," 
by the manipulation inside \TeX's gullet. 

More formally, those manipulations are called ``expansion,"
and \TeX's gullet can be conceived of as a \emph{token buffer} 
that is feeded to the right (or end) by tokenization from the character buffer. 
Expansion means that certain tokens in the token buffer are substituted 
by other ones. This way tokens may get into \TeX's gullet that 
emerged from tokenization a ``long time ago", maybe in a previous 
run that created the \emph{format} (\TeX's variant \code{INITEX}); 
or tokens may appear by some hardwired expansion function. 

However, \emph{named tokens} may get into \TeX's gullet 
also by \emph{expansion}, never having been drawn by tokenization 
and not being hardwired. This happens by the `\csname' 
% name      %%  rm. 2012/07/25
construct. 
The input \emph{code} may contain 
\[`\csname tupni\endcsname'\]
This may be converted into 7 tokens entering \TeX's gullet, 
the first one being \NTOK{csname}, the last one 
\NTOK{endcsname}, and five character tokens in between. 
Due to some \emph{function} (which I would denote as *\code{csname})
originally associated with the token \NTOK{csname}, 
those seven tokens then are replaced by \NTOK{tupni}, 
the named token whose name is `tupni'. It is not required that 
the \TeX\ program knows about a token \NTOK{tupni}, 
neither anybody must type \qtd{&\tupni} in any file.\footnote{These 
    considerations may not be essential here, 
    rather a draft for a paper. Using 'dowith', 
    one better just thinks of the arglist items one actually lists.}
\end{smallpar}

\subsection{Arglists vs.\ Lists of Tokens---Example}
% \subsection{Arglists and 'dowith' \TeX nically}
% In the \TeX nical sense, I think of arglists \emph{and their items}
% as follows. Arglists are lists (or sequences) of tokens. 
% What is somewhat difficult is that the \emph{items} of a token list 
% usually are \emph{tokens.} Especially, the curly braces in the code 
% you type usually are converted into certain \emph{character tokens}
% that are single items of the resulting token list.\footnote{I discover 
%     this conceptual puzzle 2012-05-16.} 
% 
% The conceptual trap here may have resulted from denoting lists 
% in a \emph{\Wikienref{juxtaposition}} notation. 
% In Section~\ref{sec:lists-intuit}, I have written `aa' for a 
% ``list" of \emph{two} items. ``List" is rather a \emph{computer science} 
% term, its mathematical counterpart rather is the notion of (finite) 
% \emph{sequences.} The usual \emph{mathematical} notation for a finite sequence 
% writes list items surrounded by \emph{\wikienref{bracket}{brackets}}
% (round---\emph{\Wikienref{parentheses}}---seems to be more common than square, 
%  also \wikienref{angle brackets}{chevrons} are used).
% 
% So the \emph{string} `aa' can be written more clearly---mathematically---as 
% $(`a',`a')$. The trap with `\a' is that it could be \emph{either} the \emph{string} 
% $(`\',`a')$ \emph{or} the one-item list $(`\a')$ of strings. 
% But even in the latter case, I urge not to consider it a \emph{\TeX\ token}. 
% Rather, I consider `\a' a mistaken way of referring to the 
% named token \NTOK{a} whose name is the string `a' (or $(`a')$).
% (Section~\ref{sec:toks}). 
% 
% However, the conceptual trap about arglists and token lists 
% (there must have been some ``Arglist"!\@) lurks on another level, 
% as follows. Recall \TTb's notation of \chtok{<char>}{<cat>}
% for the \emph{character token} that \TeX's tokenizer forms 
% from <char> in the character buffer when <char>'s category code is <cat>.
% Usually, the \emph{character} `a' is converted into the 
% \emph{character token} \lttok{a}, `{' is converted into \lbtok, 
% and `}' is converted into \rbtok. 
% 
% We are turning to some \strong{examples} and \strong{counterexamples}. 
% Let us see what confusions occurred in the ``intuitive" view on 
% arglists in Section~\ref{sec:arglists-intuit}.
% 
% % \begin{example}{1} It 
% First, it was bad in Section~\ref{sec:arglists-intuit}
% to think that `aa' is a two-item arglist. It was confused with something 
% like $\lttok{a}\lttok{a}$. The latter looks like a token list---or is it an arglist? 
% Both? Anyway, it is juxtaposition notation applied to tokens, 
% mathematically it is $(\lttok{a},\lttok{a})$, so~\dots
% % \end{example}
% 
% Now let us reconsider the ``intuitive recursive definition" of arglist. 
% Or let us look at a recursive definition of \emph{token} list.
% \inlineitem{1} There is nothing wrong with saying the the empty list is 
% a token list, the same holds for arglists. 
% \inlineitem{2} a.~When we attach an arbitrary token to a \emph{token} list 
% (at the left), the result is another token list---fine. 
% There are only certain difficulties with ``handling" special token lists 
% such as $\lbtok\lttok{a} = (\lbtok,\lttok{a})$. 
% And the latter is \emph{not} an \emph{arglist!} \
% b.~When we attach a \emph{named} token to an arglist (at the left), 
% the result \emph{is} an \emph{arglist}.\footnote{This is 
%     a \emph{conjecture} only right now---2012-05-16---in which I strongly believe. 
%     Likewise later.} 
% c.~When we attach a \emph{letter} token \lttok{<char>} to an arglist 
% (at the left), the result \emph{is} an \emph{arglist}.
% d.~When we talked about ``brace groups," they seemed to be \emph{strings} 
% of characters. Instead, I would like to suggest that a brace group 
% is an \emph{arglist}\pdots
% When \lbtok\ is attached to the left of an arglist and \rbtok\ to the right, 
% the result is an arglist---this is what I would call a ``brace group"!
% e.~The \Wikienref{concatenation} of two arglists is an arglist. 
% 
% \begin{smallpar}
% The above notion of ``attaching" a token to a token list or an arglist 
% % should be clarified, but instead of a general definition in terms of 
% % ``words" of formal languages, examples may suffice here and now.\footnote{2012-05-16.}
% should be clarified. Attaching an item $j$ at the left of a list $\lambda$ 
% is the same as concatenating the one-item list $(j)$ with $\lambda$\pdots 
% but the English \wikienref{append}{Wikipedia} seems to explain concatenation by 
% \emph{appending}. As we \emph{remove} items one-by-one from the \emph{left} 
% (beginning) of a list, I prefer the inverse \emph{prepending} items as basic operation 
% for building lists---cf.\ \Wikienref{CAR and CDR}.
% \end{smallpar}
% 
% We won't complete a formal (recursive) definition of arglist here and 
% now.\footnote{2012-05-16.} Just observe that ``brace groups" 
% make the difference between token lists and arglists. 
% Recall that an ``item" of an arglist ``operationally" is defined 
% as something that a one-parameter macro removes. 
% Such a macro removes certain single tokens 
% (space tokens not among them)---and entire ``brace groups"! 
% Especially, consider \[\lbtok\rbtok\]%%%.
% Actually, this is another ambiguous notation. 
% If it refers to $(\lbtok,\rbtok)$, it is a \emph{token} list, 
% not an arglist. If it refers to $(\lbtok\rbtok)$---it should 
% better refer to $((\lbtok,\rbtok))$, which is a 
% \emph{one-item arglist} whose only item is the former two-item token list!
% 
%% <- 2012/05/17b ->
Let us reconsider the examples from Sections~\ref{sec:lists-intuit}
and~\ref{sec:arglists-intuit}, and pack them into a single example. 
If you type a file line
\begin{equation}
  `a a\a{a}'
\end{equation}
(\emph{eight} keystrokes),
it should usually be converted into this \emph{seven}-item list 
of 
%%% (five) %% rm. 2015/11/14 -- ???
tokens:
\begin{equation}
  \label{eq:toks}
  \lttok{a}\;\sptok\;\lttok{a}\;\NTOK{a}\;\lbtok\;\lttok{a}\;\rbtok
\end{equation}
---with notation from Section~\ref{sec:toks} and
\TTb's notation \chtok{<char>}{<cat>} 
for the \emph{character token} that \TeX's tokenizer forms 
from <char> in the character buffer when <char>'s category code is <cat>.

It turns out that the token list in \ref{eq:toks} 
provides an arglist of \emph{four} items: The token \lttok{a}
at the first and third place, the named token \NTOK{a}, and the entire 
token list $\lbtok\lttok{a}\rbtok$ as a single item---a ``brace group."
The space token is ignored.\footnote{\TTbp.~201: ``\TeX\ doesn't use 
single spaces as undelimited arguments."}

You can try this after `\renewcommand{\a}{A}'\footnote{Otherwise 
    \cs{a} is a one-parameter macro that breaks 'dowith''s control.}
with 'dowith':
\begin{equation}
  \label{eq:sample-code}
  `\DoWith\typein a a\a{a}\StopDoing'
\end{equation}
Then \LaTeX\ shows `a', `a', `A' from `\a', and another `a' from within 
the braces---`\typein' (as any macro with arguments) removes them.

% \show\a
% \a{a}
\renewcommand*{\a}{A}
% {\MakeNormalHere\# \newcommand*{\TypeOut}[1]{\typein{#1}}
%  \global\let\TypeOut\TypeOut}
% \let\TypeOut\typein
% \let\TypeOut\typeout
% \DoWith\TypeOut a a\a{a}\StopDoing
% \DoWith\typein a a\a{a}\StopDoing
% { %%% \tracingmacros=1 \tracingonline=1
%  \MakeNormalHere\# \AssignCatCodeTo{2}\]
%  \DoWith\typein #a\a{a]\StopDoing}

I have avoided saying \ref{eq:toks} \emph{were} an arglist of 4 items. 
The mathematical basic way of writing lists---understood as finite
\emph{\wikienref{sequence}{sequences}}---as ``commma-separated lists" 
within \wikienref{bracket}{brackets} may clarify the difference 
(that the \Wikienref{juxtaposition} notation tends to conceal). 
The \emph{token} list is 
\begin{equation}
  \label{eq:toks,}
  (\lttok{a}, \sptok, \lttok{a}, \NTOK{a}, \lbtok, \lttok{a}, \rbtok)
\end{equation}
while the list of macro arguments is
\begin{equation}
  \label{eq:args}
  (\lttok{a}, \lttok{a}, \NTOK{a}, (\lbtok, \lttok{a}, \rbtok)).
\end{equation}
\ref{eq:toks} or \ref{eq:toks,} simply is \emph{not} an arglist 
(since neither \lbtok\ nor \rbtok\ can be a macro argument), 
and the arglist \ref{eq:args} ``provided" by the list of tokens 
is \emph{not} a list of \emph{tokens}---its final item is a 
three-item list of tokens, and a token cannot be a list of 
two or more tokens itself(\emph{!?}).

\subsection{Another Notation and the Example's Steps}
\label{sec:steps}                                           %% 2012/05/20
\begin{smallpar}
To write token lists easier and hopefully easier to read, 
I would suggest writing \qtd{.<char>} for the character token 
that the tokenizer ``usually" forms from character <char>, i.e., 
adding the \emph{standard} category code as in \TTb\ (page~37).
Then \ref{eq:toks} would read\footnote{See Section~\ref{sec:toks} for the question mark.}
\begin{equation}
  .&a\,.\codesp\,.&a\,\ntok{a}\,.\codelb\,.&a\,.\coderb
\end{equation}
and the corresponding arglist is
\begin{equation}
  (.&a,.&a,\ntok{a},(.\codelb\,.&a\,.\coderb))
\end{equation}
In ``retrospect," the result of tokenizing \ref{eq:toks} should be
\begin{equation}
  \label{eq:retro}
  \ntok{DoWith}\,\ntok{typein}\,
  .&a\,.\codesp\,.&a\,\ntok{a}\,.\codelb\,.&a\,.\coderb\,
  \ntok{StopDoing}
\end{equation}
and the intention is that it works like 
\begin{equation}
  %% added \,s 2012/06/07:
  \ntok{typein}\,.\codelb\,.&a\,.\coderb\,
  \ntok{typein}\,.\codelb\,.&a\,.\coderb\,
  \ntok{typein}\,.\codelb \ntok{a}\,.\coderb\,
  \ntok{typein}\,.\codelb\,.&a\,.\coderb
\end{equation}
(The definition of `\DoWith' in Section~\ref{sec:core} indeed adds surrounding braces, 
 if missing.)
However, \TeX\ rather tries to work with as few tokens ahead as possible.
When it finds \ntok{DoWith} and the latter's meaning is the one intended 
by 'dowith', it first looks for nothing more than the two arguments 
required by our definition of `\DoWith'. A few moments later, 
the token buffer's content will just be\footnote{If you use 
    `&\DoWithAllOf&\typein\codeLB a\codeSP a&\a\codeLB a\codeRB\codeRB'
    %% <- \cs bad 2015/11/14
    instead, the entire token sequence \ref{eq:retro} will appear in the 
    token buffer ``at once."}
\begin{equation}
  \ntok{typein}.\codelb.&a.\coderb\,\ntok{expandafter}\,\ntok{DoWith}\,
  \ntok{expandafter}\,\ntok{typein}\,\ntok{fi}
\end{equation}
Next $\ntok{typein}.\codelb.&a.\coderb$ is expanded according to the code for 
`\typein' in \file{latex.ltx}. Some unexpandable tokens will emerge 
and be moved into the ``instruction buffer,"                %% cmd -> instr 2012/11/04
and you should get a screen
message with `a' and a prompt. When you have entered something, 
the remaining \ntok{expandafter} tokens and the \ntok{fi} will be 
removed from the character buffer, and it contains only
\begin{equation}
  \ntok{DoWith}\,\ntok{typein}
\end{equation}
Another token is ordered from the tokenizer to provide a second 
argument for expanding \ntok{DoWith}. The token .\codesp\ comes in, 
but that doesn't serve as a macro argument. It is removed, and the 
next token is .&a. The same story as before happens, until the 
named token \ntok{a} is found\pdots
\end{smallpar}

\subsection{Summary of Possible Arglist Items}
\begin{smallpar}
For $0\leq i\leq 15$, let $\Chi_i$ be the set of character tokens 
of category code $i$. $\Chi_1$ is the set of tokens working like 
$\lbtok$, and $\Chi_2$ is the set of tokens working like $\rbtok$.

Let $E$ be the set $\{3,4,6,7,8,11,12,13\}$. 
These numbers are the category codes for 
\meta{math},   \meta{align}, \meta{parameter}, \meta{super}, \meta{sub}, 
\meta{letter}, \meta{other}, \meta{active} respectively. 
Let $\Chi_E$ be the set of character tokens of category code in $E$
(so $\Chi_E=\bigcup_{i\in E}\Chi_i$).

Let $\circ$ be the \emph{concatenation} operation among token lists.\footnote{%
    %% 2012/05/18
    \TODO: Define for representations by maps, or:              %% \ 2012/11/05
    ``Concatenation is about as basic as natural numbers and is 
      understood in terms of axioms rather than by a definition.''---See 
        notes from 2011 (even with attempts with \Wikienref{Category theory})
%         the \wikienref{Sequence}{English} 
        the English Wikipedia for 
        \wikienref{Sequence}{sequences}---\wikideref{Folge (Mathematik)}{German} 
        article too much restricted to maps.}

The following kinds of token lists form a single arglist item, 
i.e., can serve as an argument for an undelimited parameter:
\begin{enumerate}
  \item a \emph{named} token, or the single-token list consisting of it, 
        if you prefer that;
  \item a \emph{character} token from $\Chi_E$ or the list consisting of it; 
  \item a \emph{brace group.} 
        That is a token list meeting the following conditions:
        \inlineitem{1} its \emph{first} token is in $\Chi_1$,
        \inlineitem{2} its \emph{last} token is in $\Chi_2$,
        \inlineitem{3} it has as many occurrences of tokens from $\Chi_1$ as from $\Chi_2$,
        \inlineitem{4} if it is split as $\lambda\circ\rho$, there are not 
            more $\Chi_2$ occurrences in $\lambda$ than $\Chi_1$ occurrences in $\rho$
            (``don't close before opening").
\end{enumerate}
The second claim can be checked with
\begin{equation}
  `\DoWith\typein$#^_a1~\StopDoing'
\end{equation}
%  \begingroup
%   \def\a{A}\MakeNormalHere\# 
%   \DoWith\typein$# ^_a1~\StopDoing
%   \DoWith\typein#1\StopDoing
%  \endgroup
as to what works. 
(The claim is not affected by one or two surprises.)\footnote{Moreover, 
    `&\DoWith&\typein&#1&\StopDoing'              %% \cs bad 2015/11/14
    tells something about ``parameter tokens."} 
Characters with different category codes 
either are not converted into a character token\footnote{\TTbp.~47.} 
or are not accepted as macro arguments. The latter applies to ``brace" tokens 
in $\Chi_1$, $\Chi_2$ and to the single space token \sptok.

As to \emph{``brace groups"}, the third and fourth condition above 
are intended to say that what is between the two outer tokens 
is $\langle$balanced text$\rangle$ in the sense of \TTbp p.~275f. and~385;
i.e., for two tokens $a$, $b$ and a token list $\beta$, 
$(a)\circ\beta\circ(b)$ is a brace group exactly if $a$ is from $\Chi_1$,
$b$ is from $\Chi_2$, and $\beta$ is $\langle$balanced text$\rangle$. 
The conditions are more formal than what I can find in \TTb, 
but still they don't give me an idea of all possibities. 
This should be improved by the following recursive definition:

\begin{trivlist}\item 
B1.~The empty list is balanced text. 
B2.~For any token $t$ not in $\Chi_1$ or $\Chi_2$, 
    the single-item token list $(t)$ is balanced text. 
    (Such a token is either a \emph{named} token or a 
     \emph{character} token from $\Chi_E$ or \emph{the space token}~\sptok.)
B3.~If $\alpha$ and $\beta$ are balanced texts, 
    then $\alpha\circ\beta$ is balanced text.
B4.~If $\beta$ is balanced text, 
    $a$ is from $\Chi_1$, and $b$ is from $\Chi_2$, 
    then $(a)\circ\beta\circ(b)$ is balanced text.
    (This is a brace group, and the only way of getting a brace group.)
B5.~Nothing else is balanced text.
\end{trivlist}

In other words, a token list is a brace group if and only if 
it is balanced text and starts with a token from $\Chi_1$
and ends with a token from $\Chi_2$.\footnote{Again, this may be  more 
    of a draft for a paper, or notes for it, than package documentation.}
\end{smallpar}

\subsection{Summary: ``Commands" Usable with 'dowith'}
In the \hyperref[sec:apply]{implementation section,} 
you learn about 
\[`\DoWith<cmd>',\quad `\DoWithAllOf<cmd>',\quad 
    \mbox{and}\quad  `\DoWithAllIn<cmd>'.\] 
(\LaTeX\ users may type `{<cmd>}' instead.)
What <cmd>s are allowed?

\begin{enumerate}
  \item All \strong{one-parameter macros} <cmd> work this way, 
        unless there are programming mistakes outside 'dowith' 
        (also thinking of arguments that take over control 
         from 'dowith' commands before the argument list is finished). 
  \item \strong{Other one-parameter} ``commands" <cmd> such as
        \TeX\ \strong{primitives} may work---you must think of 
        the fact that surrounding \emph{braces} are added.\footnote{\TODO: %% \ 2012/11/05
            in the future, variants not adding braces could be added.}
        So the \strong{primitives} `\hbox' and `\vbox' work, 
        for instance. `\show' is an example that doesn't work at all, 
        it takes the single starting brace token and then confuses 
        `\DoWith'.
  \item Some <cmd>s taking \strong{no argument} may make sense, e.g., 
        for getting 
          \begin{enumerate}
            \DoWithAllOf{\item}{{apples,}{pears,}{peaches}}
          \end{enumerate}
        from 
        \begin{quote}
          &\begin{enumerate}\\
          \null\code{~~}&\DoWithAllOf{&\item}{%
            \codebd{apples,}\codebd{pears,}\codebd{peaches}}\\
          &\end{enumerate}
        \end{quote}
        Recall that `\item' at most takes an \emph{optional} argument.
  \item <cmd> must \strong{not take more than one} parameter. 
        %% see def. \DoWith
        A different package will support multi-parameter macros.
\end{enumerate}

% Rather, ``list" is a term from \Wikienref{computer science} here. 
% It corresponds to the notion of ``sequence" in mathematics 
% and to ``word" with formal languages.
% 
% Especially, we have (\meta{i})~lists of \emph{characters} that 
% the \emph{\TeX} program recieves from files and 
% (\meta{ii})~lists of so-called \emph{tokens} 
% that the \TeX\ program forms from the incoming list of characters 
% and that it works on.
% 
% There are \emph{more} kinds of lists that \TeX\ works on. 
% Here we are dealing with ``\TeX's gullet."\footnote{Cf.\ documentation 
%     of the \ctanpkgref{bitelist} package.}
% \TeX's mouth processes tokens formed from character input. 
% It may turn a list `<toks-a><toks_b><toks-c>' of tokens 
% into a list `<toks-a><toks-B><toks-c>', i.e., 
% replace <toks-b> by <toks-B>---by so-called ``expansion". 
% The result may be subject to expansion as well. 
% When nothing is left to be expanded, results are passed to another, 
% more interior subprocessor of \TeX. This one is deeper than \TeX's mouth, 
% we are not concerned with that here.
% 
% Not \emph{all} of those tokens in ``\TeX's mouth" are formed 
% (``directly") from input characters. 
% Rather, some \cs{csname} function may form \emph{new} tokens 
% from other tokens in \TeX's mouth---and place them there again. 
% This way characters `\csname a\endcsname' you type may just work like 
% `\a'.
% 
% It is difficult to tell on the \emph{character level} what the 
% present package does. The relation between incoming characters 
% and resulting tokens can hardly be explained by a single sentence or so.
% However, <cmd> here refers to a command <macro> with a single 
% undelimited macro. More precisely, a certain character sequence 
% (``string") in the code you type, will be converted into a TODO
% 
% \subsection{Separators}
% TODO
% 
% \subsection{Iterators and \TeX's Mouth}
% TODO
% 
\section{Similar Commands in other Packages}
%% <- section again 2012/05/17b
\subsection{``Heavy" Packages}                      %% add. heading 2012/05/17b
The \CtanPkgRef{etex}{$\varepsilon$-TeX}-related    %% e->... 2012/05/09
packages \ctanpkgref{etextools} 
(Florent Chervet), \ctanpkgref{etoolbox} (Philipp Lehman), 
and \ctanpkgref{texapi} (Paul Isambert) seem to include and 
(very much) extend the functionality of 'dowith'. 
Also the `\ForEach'\texttt{\dots\unkern\@} macros of            %% \@ 2012/06/03
\ctanpkgref{forarray} (Christian Schr\"oppel) seem to extend 
the present `\DoWith'\texttt{\dots\unkern\@} commands.          %% \@ 2012/06/03
Moreover, Ahmed Musa describes such commands as                 %% 2012/06/03
``Parsing \qtd{tsv} lists" in documenting his 
\ctanpkgref{catoptions} package.
\ctanpkgref{moredefs} (Matt Swift) provides list handling commands 
like the few that are here.\footnote{\ctanpkgref{arrayjobx}%    %% % 2012/11/05
    provides somewhat ``exotic" handling of ``lists".}          %% 2012/05/10
%% 2012/11/05:
---In October 2012, Ahmed Musa's \ctanpkgref{loops} appeared on 
\acro{CTAN}, offering loops of several ``categories" about as those 
that are listed below, very elaborate.% 
%% /2012/11/05
% (I~do not want to load that much.)
%% <- 2012/11/04 ->
---I do not want to load that much. I need and only need something 
excessively simple, very few lines of code, as presented in 
Section~\ref{sec:implement}. The next sections somewhat point 
out single features of loop constructs that I do not want to have.

\subsection{Separators}                             %% add. heading 2012/05/17b
Regarding \LaTeX\ macros in `latex.ltx', the basic macro `\DoWith' of 
the present package resembles |\@tfor| very much, which likewise deals 
with lists without separators. By contrast, \LaTeX's |\@for| deals with 
\emph{comma-separated} lists (such as lists of package options). 
With comma-separated lists, a ``string" of characters counts as 
an item when it is delimited by commas, or by a comma and the 
list ``border," or spaces may be used as separators additionally. 
However, when \LaTeX\ analyzes such lists (in ``\TeX's gullet"), 
it uses representations by \emph{character tokens} of them.

%% moved here 2012/05/17b:
% %% 2011/11/10:
% Also Heiko Oberdiek's \ctanpkgref{zref} deals with ``lists" of 
% ``properties" of ``entities," 
%% 2011/11/11: comma separated!

%% 2012/05/09f.:
The more recent \ctanpkgref{lmake} 
(Shengjun Pan)                                      %% 2012/05/18
provides a key-value syntax for printing lists of 
complex mathematical expressions easily (using some assignments) 
as well as defining commands according to a pattern from a list.
Those lists are comma-separated.

\subsection{``For" Loops vs.\ ``Foreach" Loops}     %% mv. up 2012/05/20
What about \ctanpkgref{forloop} (Nick Setzer), 
\ctanpkgref{multido} (Timothy Van Zandt, Rolf Nie\-praksch, Herbert 
Vo\ss), and \ctanpkgref{xfor} (Nicola Talbot)? 

'xfor' is just a reimplementation of `\@for'. 
'forloop' and 'multido' are more close to ``real \qtd{for} loops"
(cf.\ \wikienref{for loop}{\meta{Wikipedia}}).
Loops of the latter kind go through a certain set as well, 
but such sets rather consist of \emph{numbers} and are exhausted 
by incrementing (or also decrementing) variables 
\wikienref{Loop counter}{(counters).}
This is essentially not needed 
(neither helpful)                                       %% 2012/05/20
when a list literally is 
\emph{enumerated}---such loops are distinguished as 
\wikienref{Foreach loop}{``foreach loops."}             %% 2012/05/19

\subsection{Iterators}                              %% add. heading 2012/05/17b
So `\DoWith' and |\@tfor| rather provide ``foreach" loops. 
A major difference between them is that the latter uses a 
\wikienref{For loop}{``loop variable"} or 
%%% rather                                              %% rm. 2012/05/20
\wikienref{Iterator}{``iterator"} to which the elements 
of the list are assigned. 
`\DoWith<cmd>' does not use such a loop variable 
or such assignments and thus is ``expandable" at least when <cmd> 
(and the elements, depending on <cmd>) are expandable.
On the other hand, `\@tfor' applies some procedure to the list 
elements without needing a \emph{name} for the procedure 
(or a \emph{macro} storing the procedure). 
%
I wondered whether behind \LaTeX's 
`\@tfor' (and `\@for') there was an ``ideological" consideration 
such as ``A loop must have a loop variable!"\pdots
% However, avoiding usage of a macro name 
% (to store the ``loop body" code)                        %% 2012/05/18
% and a macro parameter 
% (to incorporate the list item into the body code)       %% 2012/05/18
% may have been a good reason.
%% <- 2012/05/19

%% 2012/05/20:
Hopefully more clearly on ``loop variable" vs.\ our approach: 
In order to run
\[<code-before><item><code-after>\]
on each <item> of a <list>, \emph{we here}
% \[
\begin{equation}
  \mbox{define} \quad `\do' \quad \mbox{as} \quad 
  `#1'\;\to\;`<code-before>#1<code-after>'
\end{equation}
% \]
and then run `\do{<item>}' for each <item> in <list>,\footnote{Cf.~description 
    of procedure in terms of tokens in Section~\ref{sec:steps}.} 
\begin{equation}
  \mbox{always replacing}\quad  `\do{<item>}'\quad \mbox{by}\quad  `\do{<item>}\do'. 
\end{equation}
(`\do' is only an example command that 'dowith' supports especially.)
% The ``received" approach from ``usual" programming languages, 
% \file{latex.ltx}, and its followers is
In \file{latex.ltx} instead, we find things like
% \[
\begin{equation}
  \label{eq:tfor}
  `\@tfor\@tmp:=<list>\do{<code-before>\@tmp<code-after>}'
\end{equation}
% \]
where `\@tmp' is a \emph{macro} that is set to be <item> at each 
iteration of the loop, by 
\begin{equation}
  `\def\@tmp{<item>}'
\end{equation}
within `\@tforloop'. After that,
\begin{equation}
  \label{eq:code-tmp-code}
  `<code-before>\@tmp<code-after>'
\end{equation}
from \ref{eq:tfor} is run.---\ref{eq:code-tmp-code} like 
\ref{eq:tfor} is stored in a larger macro. `\do' in \ref{eq:tfor}
does not act as a macro, it just delimits a macro parameter 
in order give a feeling of some familiar programming structure. 
This organisation of macros is fine when the loop body code 
is only used by the containing macro, while the 'dowith' approach 
to store the ``loop body" in an own macro has been useful when 
the loop body code also is used for different purposes 
or when it has been introduced before I thought of using it 
in a loop.
% \[\mbox{defining}\;`\@tmp'\;\mbox{as}\;<item>\quad \mbox{and run} \quad 
%   `<code-before>\@tmp<code-after>'\]
% for each <item> in <list>.

Note that this only was an example. In general, <item> may appear 
more than once in the ``loop body."

``Expandability" by \emph{avoiding} something iterating `\def\@tmp{<item>}'
and doing iteration in \TeX's gullet (`\do' or so must have been defined earlier)
is essential especially within \cs{write}. 
Assignments do not work there. A major motivation for developing 
'dowith' developed with the \ctanpkgref{blog} package 
that \cs{write}s \acro{HTML} code.
Assignments happen in ``\TeX's stomach."                    %% 2012/11/04
That place might be called the 
``instruction buffer" to which                              %% cmd -> instr 2012/11/04
the ``expansion processor" moves items from the incoming token buffer 
that cannot be expanded (any more).

\subsection{Separator Macros}                       %% add. heading 2012/05/17b
Commands like `\DoWith' also could save tokens thinking of list macros 
(in \LaTeX/`latex.ltx') that use a \emph{separator macro} 
which may be used as a \emph{command} to be applied to the list 
elements. One example is 
`\dospecials' that already is in Plain \TeX\ and expands to 
\[`\do\ \do\\\do\{\do\}\do\$\do\&\do\#\do\^\do\_\do\%\do\~'\]
% \begin{verbatim}
% `\do\ \do\\\do\{\do\}\do\$\do\&\do\#\do\^\do\_\do\%\do\~'
% \end{verbatim}
An important application of `\dospecials' is temporarily 
switching off the ``special" functionality of the ``elements" 
in `\dospecials'. With \LaTeX, this may happen thus:
\[`\let\do\@makeother\dospecials'\]
% \begin{verbatim}
% `\let\do\@makeother
% \end{verbatim}
With 'dowith', you can do the same with a shorter variant 
`\specials' of `\dospecials', defined by 
\[`\def\specials{\ \\\{\}\$\&\#\^\_\%\~}'\]
and then 
\[`\DoWithAllIn\@makeother\specials'\]
`latex.ltx' uses `\@elt' instead of `\do' for its own list macros.

\subsection{Ye Olde \cs{loop}}                      %% add. heading 2012/05/17b
%% added 2011/11/03:
There also is |\loop<loop-body>\repeat| in Plain \TeX\ and a 
refined\footnote{Using Kabelschacht's suggestion, cf.\ Section~\ref{sec:kabel}} 
version of it in `latex.ltx'. It is \emph{not} expandable 
since it starts with an assignment for `\body' (Plain \TeX) or 
`\iterate' (`latex.ltx'), 
%% add. 2012/05/20:
and then some assignments are needed to stop the loop, 
such as incrementing or decrementing a \emph{counter.} 
As to the programming structure, it is very simple and general, 
I think any kind of loop can be implemented by this
(apart from nested loops). E.g., I realize\footnote{2012-05-20}
that even a ``foreach" loop could be implemented by managing 
a list macro, e.g., using \LaTeX's internal `\@next'.
%% rm. 2012/05/20:
% that you cannot probably
% what kind of loops it addresses. However, the applications 
% I have seen have been ``for" or (rather) ``while" loops. 
% ``While" loops can ``emulate" ``for" and ``foreach" loops 
% by having the ``incrementation" method or the ``enumeration" method in 
% their body. This is quite obvious for ``for" loops, not quite so for 
% ``foreach" loops; which for practical application (in my view) means 
% that neither \LaTeX/\TeX's `\loop' macro nor in general ``while" loops 
% is/are very helpful for implementing ``foreach" loops, 
% as rather `\DoWith' and similar constructions are.
% The reason for this is (as it seems to me) is that you 
% (a human being) can much more easily enumerate (``list") 
% the items of a list (you have in mind) than define the \emph{method} 
% that (allegedly) is behind your enumeration. \ \meta{Example:} 
% \[`\DoWithAllOf{\printsamplearea}{\red\green\blue}'\]
% ---\emph{how} (according to what ``method"?) did you ``proceed" from 
% `\red' to `\green' and from `\green' to `\blue'?

\subsection{Without Iterator and Separators}        %% add. heading 2012/05/17b
%% 2011/11/07:
In \LaTeX's \ctanpkgref{tools} bundle, \ctanpkgref{xspace}
%% add. 2012/05/18:
was developed in the nineties by David Carlisle. 
It had a rather fixed exception list implemented by a deeply
nested conditional. In 2004 Morton H\o gholm joined, 
and now 'xspace' has a list macro 
`\@xspace@exceptions@tlp'                           %% was xpspace 2012/05/17b
without separators. 
It is handled like here, except that it ``breaks" the loop 
when an item is found that applies.
%% add. 2012/05/17bf.:
After the ``next" token is stored by the usual \cs{futurelet}, 
the exception list is searched without using an iterator. 
Addition and removal commands are provided as well. 

%   \pagebreak                                                %% 2012/05/17b
% \newpage                                                    %% 2012/05/19
\section{Implementation}                                    %% 2012/05/10
\label{sec:implement}
\subsection{Package File Header (Legalese)}                 %% sub 2012/05/10
\ProvidesFile{dowith.tex}[2015/11/14 documenting dowith.sty]
\title{%%%\kern-\baselineskip
       \textsf{\huge dowith.sty}\\---\\Apply Command to
%        Elements of Lists without Separators\,---\,%%% 2012/05/14
%        and without Iterator\thanks{This
       %% 2012/05/15, "in" 2012/05/18:
       Each Item \\ in a List of Arguments in ``\TeX's Gullet"\thanks{This
       document describes version
       \textcolor{blue}{\UseVersionOf{\jobname.sty}}
       of \textsf{\jobname.sty} as of \UseDateOf{\jobname.sty}.}}
{ \RequirePackage{makedoc} \ProcessLineMessage{}
  \renewcommand*\mdSectionLevelOne{\string\subsection}
  \renewcommand*\mdSectionLevelTwo{\string\subsubsection}
  \MakeJobDoc{18}%% 2012/11/05 for v0.3
  {\SectionLevelTwoParseInput}  }
% \RequirePackage[ir]{inputtrc}
\documentclass[fleqn]{article}%% TODO paper dimensions!?
\ProvidesFile{makedoc.cfg}[{2013/03/25 documentation settings}] 
%%
\author{Uwe L\"uck\thanks{%
        \url{http://contact-ednotes.sty.de.vu}}}
%%
%% 'hyperref':
\RequirePackage{ifpdf}
\usepackage[%
  \ifpdf
%     bookmarks=false,                  %% 2010/12/22
%     bookmarksnumbered,
    bookmarksopen,                      %% 2011/01/24!?
    bookmarksopenlevel=2,               %% 2011/01/23
%     pdfpagemode=UseNone,
%     pdfstartpage=10,
    pdfstartview=FitH,                  %% 2012/11/26 again
%     pdfstartview=0 0 100,             %% 2011/08/22
%     pdfstartview={XYZ null null 1},   %% 2011/08/25
%     pdfstartview={XYZ null null null},%% 2011/08/25
%     pdfstartview={XYZ null null .5},    %% 2011/08/26
%     pdffitwindow=true,          %% 2011/08/22
    citebordercolor={ .6 1    .6},
    filebordercolor={1    .6 1},
    linkbordercolor={1    .9  .7},
     urlbordercolor={ .7 1   1},   %% playing 2011/01/24
  \else
    draft
  \fi
]{hyperref}
\hypersetup{% 
    pdfauthor={Uwe L\374ck}% 
}
%% metadata, |\MDkeywords{<text>}|, |\MDkeywordsstring|:
%% %% 2011/08/22:
\makeatletter
  \newcommand*{\MDkeywords}[1]{%
    \gdef\MDkeywordsstring{#1}%
    \hypersetup{pdfkeywords=\MDkeywordsstring}%% TODO!?
  }
  \@onlypreamble\MDkeywords
%% |\MDaddtoabstract{<par-head>}|, `:' added:
  \newcommand*{\MDaddtoabstract}[1]{%           %% 2012/05/10
    \par\smallskip\noindent
    \strong{#1:}\quad\ignorespaces}
%% \pagebreak[2]
%% |\printMDkeywords|:
  \newcommand*{\printMDkeywords}{%
    \MDaddtoabstract{Keywords}%
    \MDkeywordsstring 
%     \global\let\MDkeywordsstring\relax    %% `%' 2012/11/12
  }
%% The previous definitions mainly are useful with a variant 
%% |\begin{MDabstract}| of \LaTeX's `{abstract}' environment:
  \newenvironment{MDabstract}
                 {\abstract\noindent
                  \hspace{1sp}%% for niceverb
                  \ignorespaces}
                 {\@ifundefined{MDkeywordsstring}%
                               {}%
                               {\printMDkeywords}%
                  \global\let\MDabstract\relax    %% 2012/11/12
                  \global\let\endMDabstract\relax %% 2012/11/12
                  \endabstract}
%% |\[MD]docnewline| 2012/11/12 from `readprov.tex':
  \newcommand*{\MDdocnewline}{\leavevmode\@normalcr[\topsep]}
%% <- `\leavevmode' for use with `\paragraph'.
%%    Sometimes needs to be preceded by a space.
%% 
%% |\MDfinaldatechecks[<tex-script>]| with \ctanpkgref{filedate}:
  \newcommand*{\MDfinaldatechecks}[1][fdatechk]{%
    \AtEndDocument{%
%       \clearpage %% 2013/03/25 no avail -- with `filedate'!
      \def\@pkgextension{sty}%
      \def\NeedsTeXFormat##1[##2]{}%
      \noNiceVerb                       %% 2013/03/22
      \input{#1}%
    }}
  \@onlypreamble\MDfinaldatechecks
\makeatother
%% Use other packages:
\RequirePackage{niceverb}[2011/01/24] 
\RequirePackage{readprov}               %% 2010/12/08
\RequirePackage{hypertoc}               %% 2011/01/23
\RequirePackage{texlinks}               %% 2011/01/24
\RequirePackage{relsize}                %% 2011/06/27
\RequirePackage{color}                  %% 2011/08/06
\RequirePackage{lmodern}                %% 2012/10/29
\RequirePackage{filedate}               %% 2012/11/12
\RequirePackage{filesdo}                %% 2013/03/22 
%% \pagebreak[3]
%% Logical markup:\qquad  |\strong{<chars>}|, |\meta{<chars>}|, 
%% |\acro{<chars>}|, |\pkg{<chars>}|, 
%% |\code{<chars>}|, |\file{<chars>}|:{\sloppy\par}
\makeatletter
  \def\do#1#2{\@ifdefinable#1{\let#1#2}}%% 2012/07/13
  \do\strong\textbf \do\file\texttt \do\acro\textsmaller 
  %% <- wrong tests before 2012/07/13
  \do\meta\textit   \do \pkg\textsf \do\code\texttt
  \ifpdf
    \pdfstringdefDisableCommands{%
        \let\acro\textrm 
        \let\file\textrm                            %% 2011/11/09
        \let\code\textrm                            %% 2011/11/20
        \let\pkg \textrm                            %% 2012/03/23
    }
  \fi
  %% TODO 2011/07/22 -> `htlogml.sty'
\makeatother
%% |\qtdcode{<text>}|: 2012/10/24:
    \newcommand*{\qtdcode}[1]{`\code{#1}'} 
%% |\pkgtitle{<package-name>}{<caption>}| 
\newcommand*{\pkgtitle}[2]{%            %% 2012/07/13
    \global\let\pkgtitle\relax
    \pkg{\huge #1}\\---\\#2\thanks{This 
       document describes version 
       \textcolor{blue}{\UseVersionOf{\jobname.sty}} 
       of \textsf{\jobname.sty} as of \UseDateOf{\jobname.sty}.}}
%% TODO: %% |\TODO| bad with `mdoccorr.cfg'
\newcommand*{\TODO}{\textcolor{blue}{\acro{TODO}}}  %% 2012/11/06
%% `\MDsampleinput[{<file>}' was added 2012/11/06. 
%% Problems with `myfilist.tex' were due to 'parskip.sty'
%% there. On 2012/11/12, we change the former simple macro to a 
%% much more complex
%% |\MDsamplecodeinput[<add-hfuss>]{<file>}| 
\newcommand*{\MDsamplecodeinput}[2][]{%
    \begingroup
        \vskip\bigskipamount \hrule
        \nobreak\vskip-\parskip 
%         \nobreak\vskip\medskipamount
%% Previous mistake (same below) due to manual change 
%% of `\topsep' in the file `myfilist.tex' (2012/11/30).
        \ifx\\#1\\\else
            \hfuzz=\textwidth \advance\hfuzz#1\relax
        \fi
        \noNiceVerb \verbatiminput{#2}%
%         \nobreak\vskip\medskipamount 
        \hrule \vskip-\parskip 
        \bigskip %%% \bigbreak
%% `\bigbreak' made much larger space in `myfilist.tex'.
    \endgroup
}
%% |\ctanpkgdref{<pkg-id>}| adds the printed link to 
%% `ctan.org/pkg' as a footnote. There is a little space 
%% for coloured link borders:
\newcommand*{\ctanpkgdref}[1]{%
    \ctanpkgref{#1}\,\urlfoot{CtanPkgRef}{#1}}
\errorcontextlines=4
\pagestyle{headings}

\endinput 
    %% shared formatting settings
\ifpdf\else \errhelp{hyperref draft bad with {equation}}
            \errmessage{run this with pdflatex only}\fi
% \ReadPackageInfos{dowith}
\usepackage{dowith}                                         %% 2012/05/17b
\sloppy
\MDkeywords{programming structures;                         %% 2012/05/14b
            macro programming, loops, list macros}
\hypersetup{%
    pdftitle=dowith.sty handles lists without separators,
    pdfsubject=documenting dowith.sty
}%% 2011/08/22 
\usepackage{fixltx2e}                       %% \textsubscript 2012/05/17b
%% <- TODO with `lmodern'?
% \newcommand*{\ctanpkgdref}[1]{%                             %% 2012/11/19
%     \ctanpkgref{#1}\,\foothttpurlref{ctan.org/pkg/#1}}
%% <- in `makedoc.cfg' 2015/11/14
\makeatletter                                   %% etc. 2012/05/17b
%% TODO `actcodes'!? 2012/11/02
  \newcommand*{\GetOtherChar}[2]{% 
    \@ifdefinable#1{%
        \edef#1{\expandafter\@gobble\string#2}}}
\makeatother
\GetOtherChar\codeLB\{ 
\GetOtherChar\codeRB\} 
% \GetOtherChar\codeSP\ %       %% ???
{\MakeOther\ \xdef\codeSP{ }}
\newcommand*{\codelb}{\code\codeLB}
\newcommand*{\coderb}{\code\codeRB}
\newcommand*{\codesp}{\code\codeSP}
\newcommand*{\codebd}[1]{\code{\codeLB#1\codeRB}}       %% 2012/05/19
%% rm. 2012/05/17b:
% \DeclareRobustCommand*{\code}[1]{%
%   \texttt{%
%     \let\{\codeLB \let\}\codeRB \let\ \codeSP
%     #1}}
% \newcommand*{\chtok}[2]{\ensuremath{\code{#1}\sb{#2}}}
%% <- 2012/05/17b ->
\newcommand*{\chtok}[2]{\code{#1}\textsubscript{#2}} 
\newcommand*{\lbtok}{\chtok\codeLB{1}}
\newcommand*{\rbtok}{\chtok\codeRB{2}}
\newcommand*{\lttok}[1]{\chtok{#1}{11}}
\newcommand*{\sptok}{\chtok\codeSP{10}}
\providecommand*{\TTb}{\meta{The\nolinebreak[3] \TeX book}}
\providecommand*{\TTbp}{\TTb\nolinebreak[3] p}
\newenvironment*{smallpar}
                {\medskip\par\begingroup\footnotesize}
                {\par\endgroup\medskip}
\newcommand*{\NTOK}[1]{\textsf{ntok}(\code{#1})}
\newcommand*{\ntok}[1]{\textrm{?}\code{#1}}
\newenvironment*{example}[1]
                {\trivlist\item
                 (\ulroman{#1})}
                {\endtrivlist}
\providecommand*{\ulroman}[1]{\meta{\romannumeral #1 }}
\newcommand*{\inlineitem}[1]{\ (\ulroman{#1})\enspace
             \ignorespaces}
\newcommand*{\pdots}{~.\kern\fontdimen3\font
                      .\kern\fontdimen3\font. }
\providecommand*{\Chi}{\mathrm{X}}
\renewcommand*{\httpprefix}{\theHTTPprefix}
\begin{document}
\maketitle
\begin{MDabstract}
This package provides macros for applying a 
%% mod. 2012/05/15:
% ``command" <cmd> to all items of a list $<arg-1><arg-2>\dots<arg-$n$>$
``command" to all items in a ``list of possible macro arguments," %% v0.21a
% in ``\TeX's mouth," 
% such as `\DoWithAllIn{<cmd>}{<list-macro>}', 
and also for extending and reducing macros storing such lists.
``Brace groups" are single items of such lists, as opposed to 
token lists. Iteration is implemented within \TeX's expansion 
processor, so works within `\write' as with 'blog.sty'.
 %%% ---pleading for mathematical rigour in \TeX ology!)    %% rm. 2012/05/17b
% Applications in mind belonged to \LaTeX, but the package should work 
% with other formats as well.
Loop and list macros in other packages are discussed in the documentation.
% %% 2012/05/09:
% There is an emphasis on expandability 
% %% mod./add. 2012/05/15:
% (no \wikienref{Iterator}{iterator;} 
%  essential within \cs{write} as with \CtanPkgRef{morehype}{blog.sty}),
% without relying on \CtanPkgRef{e-tex}{$\varepsilon$-\TeX}.
%% 2012/05/18:
% Iteration is implemented within ``\TeX's mouth," so works within 
% `\write' as with \CtanPkgRef{morehype}{blog.sty}. 
There is no need for \CtanPkgRef{e-tex}{$\varepsilon$-\TeX} to which 
some of them refer.

The package is ``generic," i.e., should also work with Plain \TeX\ 
or even other formats, relying on the \ctanpkgref{plainpkg} package 
for some minimal \LaTeX-like behaviour.     %% \ 2012/11/27

\MDaddtoabstract{Related packages}
\let\pkg\ctanpkgref     \pkg{catoptions},
\pkg{etextools}, \pkg{etoolbox}, \pkg{forarray}, \pkg{forloop}, 
\pkg{loops}, \pkg{multido}, \pkg{moredefs}, \pkg{lmake}, 
\pkg{texapi}, \pkg{xfor}, \pkg{xspace}
\end{MDabstract}
\tableofcontents

%   \newpage
\section{Usage and Features}                    %% restructured 2012/05/16

\subsection{Installing and Calling}
The file 'dowith.sty' is provided ready, installation only requires
putting it somewhere where \TeX\ finds it
(which may need updating the filename data
 base).\urlfoot{ukfaqref}{inst-wlcf}           %% corr. 2011/02/08
The packages \ctanpkgdref{plainpkg}
and 'stacklet' (\ctanpkgref{catcodes})\urlfoot{ctanpkgref}{catcodes} 
must be installed as well.

As to calling (loading): 'dowith' is a ``\pkg{plainpkg} package" 
in the sense of the \ctanpkgref{plainpkg}
documentation that you may consult for details.
So roughly, 
\begin{itemize}
  \item load it by \ |\usepackage{dowith}| \ if you can, 
  \item otherwise by \ |\RequirePackage{dowith}| \\
        (perhaps from within another ``\pkg{plainpkg} package"), 
  \item or by \ |\input dowith.sty| 
  \item or even by \ |\def\filename{dowith}     \def\fileinfo{simple list loop (UL)}
\def\filedate{2012/11/05} \def\fileversion{v0.3}
%%
%% Copyright (C) 2011 2012 Uwe Lueck,
%% http://www.contact-ednotes.sty.de.vu
%% -- author-maintained in the sense of LPPL below --
%%
%% This file can be redistributed and/or modified under
%% the terms of the LaTeX Project Public License; either
%% version 1.3c of the License, or any later version.
%% The latest version of this license is in
%%     http://www.latex-project.org/lppl.txt
%% We did our best to help you, but there is NO WARRANTY.
%%
%% Please report bugs, problems, and suggestions via
%%
%%   http://www.contact-ednotes.sty.de.vu
%%
%% == Proceeding without \LaTeX ==
%% v0.3 mainly replaces imitating the 
%% \CtanPkgRef{german}{german.sty} approach to genericity 
%% by \ctanpkgref{plainpkg}:
\input plainpkg
\ProvidesPackage{\filename}[\filedate\space 
                 \fileversion\space \fileinfo]
\PushCatMakeLetterAt
%% If \LaTeX\ is not present ... 
\ifltx \else
%% ... an old version of its `\in@' is introduced. 
%% It is bad as a subword test 
%% (false positive cases, cf. \CtanPkgRef{nicetext}{fifinddo} 
%%  documentation), but 'dowith' will check for single tokens only. 
%% If \LaTeX\ \emph{is} present, 
%% on the other hand, `\ifin@' is recognized while skipping 
%% `false' parts of conditionals, without being matched 
%% by some `\fi' before the next `\else', so I hide it by `\csname':
  \expandafter\newif\csname ifin@\endcsname
  \def\in@#1#2{%
    \def\in@@##1#1##2##3\in@@{%
      \ifx\in@##2\in@false\else\in@true\fi}%
    \in@@#2#1\in@\in@@}
\fi
%% 
%% == Applying a Command ==                             %% 2011/11/07
%% \label{sec:apply}
%% === Core ===
%% \label{sec:core}
%% |\DoWith{<cmd>}<list>\StopDoing| applies <cmd> to all elements 
%% of <list>. An element of <list> (after tokenizing) 
%% may be either a single token or a group `{<balanced>}'.
\def\DoWith#1#2{%
    \ifx\StopDoing#2\empty
%% The previous `\empty' (replacing `%') is a bug fix as of v0.22 
%% (June 2012),                                             %% 2012/11/05
%% while in my extension draft I already had it in January 2012. 
%% It allows ``empty" arglist items \qtd{\lbtok\rbtok}. 
%% Before v0.22, such an empty brace group would have resulted 
%% in comparing `\StopDoing' with `\else', so \qtd{\lbtok\rbtok}
%% would have had the same effect as `\StopDoing', the token text 
%% after `\else' until `\fi' would have been skipped. 
%% Instead, the user may have a reason to allow empty arguments\slash
%% brace groups.
        \else#1{#2}\expandafter\DoWith\expandafter#1\fi}
%% |\StopDoing| delimits the list: 
\let\StopDoing\DoWith
%% ... something arbitrary that is not expected to occur in a list.
%% With \[`\let\StopDoing*'\] instead, the star would end lists.
%%
%% |\DoWithAllOf{<cmd>}{<list>}| works like 
%% \[`\DoWith{<cmd>}<list>\StopDoing':\]
\def\DoWithAllOf#1#2{\DoWith#1#2\StopDoing}
%% 
%% === `&\do' being the Command ===
%% \label{sec:do}
%% When the <list> is worked at a single time in the \TeX\ run 
%% where assignments are possible, instead of introducing a new 
%% macro name for <cmd> you can use `\do' for <cmd> as a 
%% ``temporary" macro and define it right before 
%% \[`\DoWith{\do}<list>\StopDoing'\] 
%% However, we provide \[|\DoDoWith{<cmd>}<list>\StopDoing|\] 
%% as a substitute for the former line that at least saves one token. 
%% For the definition of `\do', we provide
%% |\setdo{<def-text>}|.
%% It works similarly to \[`\renewcommand{\do}[1]{<def-text>}',\] 
%% so <def-text> should contain a `#1': 
\def\setdo{\long\def\do##1}
%% With |\letdo<cmd>| that is provided next where <cmd> is defined 
%% elsewhere, you could type 
%% \[`\letdo<cmd>\DoDoWith<list>\StopDoing'\]
%% It seems to me, however, that you better type 
%% \[`\dowith<cmd><list>\StopDoing'\] instead. 
%% So I provide `\letdo' although I consider it useless here. 
%% It is provided somewhat for the sake of ``completeness," 
%% thinking that it might be useful at other occasions such as 
%% preceding `\dospecials'.
\def\letdo{\let\do}
%% |\DoDoWith| has been described above:
\def\DoDoWith{\DoWith\do}
%% By analogy to `\DoWithAllOf', we provide 
%% |\DoDoWithAllOf{<list>}|:
\def\DoDoWithAllOf{\DoWithAllOf\do}
%%
%% === Expand List Macro ===
%% The former facilities may be quite useless
%% as such a <list> will not be typed at a single place in the 
%% source code, rather the items to run <cmd> on may be collected 
%% occasionally when some routines run. The elements may be collected 
%% in a macro <list-macro> expanding to <list>. 
%% So we provide \[|\DoWithAllIn{<cmd>}{<list-macro>}|\]
%% (or `\DoWithAllIn<cmd><list-macro>').
%% There is no need to type `\StopDoing' here:
\def\DoWithAllIn#1#2{%
    \expandafter\DoWith\expandafter#1#2\StopDoing}
%% |\DoDoWithAllIn{<list-macro>}| saves a backslash or token 
%% for `\do' as above in Sec.~\ref{sec:do}:
\def\DoDoWithAllIn{\DoWithAllIn\do}
%% 
%% 
%% == Handling List Macros ==
%% === Initializing ===
%% Here is some advanced `\let<cmd>\empty', perhaps a little 
%% irrelevant for practical purposes. Both
%% \[|\InitializeListMacro{<list-macro>}|\] and 
%% \[|\ReInitializeListMacro{<list-macro>}|\] attempt to ``empty" 
%% <list-macro>, and when we don't believe that \LaTeX\ has been 
%% loaded, both do the same indeed. Otherwise the first one 
%% complains when <list-macro> seems to have been used earlier 
%% while the second complains when <list-macro> seems \emph{not} to 
%% have been used before:
\ifltx                                                  %% v0.3
  \def\InitializeListMacro#1{\@ifdefinable#1{\let#1\empty}}
  \def\ReInitializeListMacro#1{%
    \edef\@tempa{\expandafter\@gobble\string#1}%
    \expandafter\@ifundefined\expandafter{\@tempa}%
        {\@latex@error{\noexpand#1undefined}\@ehc}%
        {\let#1\empty}}
\else
  \def\InitializeListMacro#1{\let#1\empty} %% not \@empty 2011/11/07
  \let\ReInitializeListMacro\InitializeListMacro
\fi
%% |\ToListMacroAdd{<list-macro>}{<cmd-or>}|
%% appends <cmd-or> to the replacement token list of <list-macro>. 
%% <cmd-or> may either be tokenized into a single token,
%% or it is some `{<balanced>}'.
\def\ToListMacroAdd#1#2{\DefExpandStart#1{#1#2}}
\def\DefExpandStart#1{\expandafter\def\expandafter#1\expandafter}
%% 
%% === Testing for Occurrence of a Token ===
%% |\TestListMacroForToken{<list-macro>}{<cmd>}|
%% sets `\in@true' when <cmd> occurs in <list-macro> 
%% and sets `\in@false' otherwise:                      %% 2011/11/07
\def\TestListMacroForToken#1#2{%
    \expandafter \in@ \expandafter #2\expandafter{#1}}
%% Indeed I removed an earlier `\IfTokenInListMacro', 
%% now it's a kind of compromise between having a shorthand macro 
%% below and a generalization for users of the package.
%% 
%% === Adding and Removing ===
%% |\FromTokenListMacroRemove{<list-macro>}{<cmd>}| 
%% removes the token corresponding to <cmd> from the list 
%% stored in <list-macro> 
%% (our parsing method does not work with braces):      %% 2011/11/07
\def\FromTokenListMacroRemove#1#2{%
%% I am not happy about defining \emph{two} parser macros, 
%% but for now ...
    \TestListMacroForToken#1#2%
    \ifin@
      \def\RemoveThisToken##1#2{##1}%
      \expandafter \DefExpandStart
        \expandafter #1\expandafter {%
          \expandafter\RemoveThisToken #1}% 
%% TODO warning otherwise?
   \fi}
%% %% 2011/11/07:
%% ... but this only removes a single occurrence ... 
%% \[|\InTokenListMacroProvide{<list-macro>}{<cmd>}|\]
%% avoids multiple entries of a token 
%% by \emph{not} adding anything when <cmd>
%% already occurs in <list-macro> (again, this does not work with 
%% braces, try `\in@{{}}{{}}').
\def\InTokenListMacroProvide#1#2{%
    \TestListMacroForToken#1#2%
    \ifin@ \else                    %% TODO warning?
        \ToListMacroAdd#1#2%
    \fi}
%%
%%
%% == Leaving and History ==
\PopLetterCatAt                                         %% v0.3
\endinput

VERSION HISTORY 
v0.1    2011/06/23/28  stored separately
v0.2    2011/11/02     simpler, documented
        2011/11/03     corrected \if/\else for init
        2011/11/07     \TestListMacroForToken, \InListMacroProvide; 
                       doc.: \pagebreak, structure
        2011/11/19     modified LaTeX supplements
v0.21   2012/05/14     fix for "generic" and `typeoutfileinfo': 
                       @ before ...!
v0.21a  2012/05/19     \labels sec:apply, sec:core; \pagebreak?
v0.22   2012/06/04     allow {} items
v0.3    2012/11/05     updating copyright, using `plainpkg', 
                       rewording documentation there
|~\dots
\end{itemize}

% \section{Example}

% \section{Discussion}                                      %% 2012/05/16
% \subsection{What It Seems to Do \dots}
% The 'dowith' package provides tools to simplify \TeX\ macro programming. 
% Understanding it \emph{really} may require understanding certain passages of 
% \TTb, such as pp.~38f. It may even require overcoming 
% terrible confusions in \TTb.
% Let this be a last try at understanding the package without understanding \TeX: 
% It allows you to abbreviate 
% \[<cmd><arg-1><cmd><arg-2>\dots<cmd><arg-$n$>\]
% by 
% \[`\DoWith<cmd><arg-1><arg-2>\dots<arg-$n$>\StopDoing'\]
% or by 
% \[`\DoWithAllOf<cmd>{<arg-1><arg-2>\dots<arg-$n$>}'\]
%
%   \pagebreak                                                %% 2012/05/19
\subsection{What It Does With What Lists}
% \subsection{What It Actually Does \dots}
\label{sec:lists-intuit}

The single commands that the package provides are described 
in the \hyperref[sec:implement]{implementation section} below. 
What follows here is some general background about how the 
commands work.
%% <- moved down from previous section, mod. 2012/05/19

The term \qtd{list} may refer to various things and need clarification here. 

First of all, we are not referring to \LaTeX\ `list' environments 
such as `enumerate' or `itemize'; 
neither to ``\acro{TODO}" lists of what needs to be done soon. 

Rather, 'dowith' allows you to abbreviate 
\[<cmd><arg-1><cmd><arg-2>\dots<cmd><arg-$n$>\]
by 
\[`\DoWith<cmd><arg-1><arg-2>'\dots<arg-$n$>`\StopDoing'\]
or by 
\[`\DoWithAllOf<cmd>{<arg-1><arg-2>'\dots<arg-$n$>`}'\]
With small $n$, one may doubt whether this really is an abbreviation~\dots;
anyway, 
\[<arg-1><arg-2>\dots<arg-$n$>\]
was an attempt to refer to the kind of lists we are dealing with. 
\[<arg-1>, <art-2>, \dots, <arg-$n$>\]
are the ``items" of the list. 
The question is: what counts as an item?

We might say that `aa' is a list of \emph{two} items, 
<arg-1> being `a' and <arg-2> being `a', too.

When we do \emph{three} keystrokes to get `a a' instead of `aa', 
we still have \emph{two} items, 
<arg-1> being `a' and <arg-2> being `a' too. 
Strange, isn't it?

Also, when in `aa' we replace the first `a' by a backslash, `\', 
we get `\a', and this is a list of a \emph{single} item, 
$<arg-1>=`\a'$~\dots

You shouldn't believe these stories of mine entirely. 
What I am alluding to is that the \emph{``items"} 'dowith' is about 
are determined in terms of \emph{\TeX's tokens}, and the relation 
between the ``characters you type" and \emph{\TeX's tokens} 
is not entirely straightforward. 

\subsection{The Notion of Arglists for \LaTeX\ Users}
%% <- 2012/05/17b ->
% \subsection{The Notion of Arglist}
\label{sec:arglists-intuit}
Still, it may suffice to clarify what counts as an <arg-$i$> 
without speaking of \emph{tokens} explicitly: It is simply 
what a \emph{one-parameter macro} 
(where the parameter is \emph{not delimited} in terms of 
 \TTbp p.~203f.)
can take as an \emph{argument.}

The lists 'dowith' is about then are lists \emph{of possible arguments}
in the previous sense---let me call them \emph{``arglists."}\footnote{Not 
    to be confused with German \httpref{de.wiktionary.org/wiki/Arglist}{Arglist.}}
The single \emph{items} of such lists are those single possible arguments. 
They become \emph{actual} arguments beginning from the leftmost \emph{possible} one 
when 'dowith' presents them to that <cmd> mentioned earlier---where 
<cmd> \emph{should} be a one-parameter macro (or some \TeX\ primitive 
parsing arguments similarly).

The reader perhaps has an \emph{intuitive} understanding of 
what can be an argument of a one-parameter macro. 
A \emph{strict \LaTeX} user may think that such an argument <arg-$i$>
just has form `{'<ark-$i$>`}', i.e., $<arg-$i$>=`{'<ark-$i$>`}'$
for some <ark-$i$>. Such arguments are also called \emph{``brace groups".}
(\LaTeX's \emph{optional} arguments `[<extra>]' 
 do not count as possible arguments here, they are not macro arguments 
 in the sense of \TTb.)
In this restricted \LaTeX\ sense, arglists consist of brace groups 
\[`{<ark-1>}{<ark-2>}'\dots`{'<ark-$n$>`}',\]
and each single brace group is an \emph{item} of it. 

The \emph{\TeX\ macro writer}, by contrast, knows that a macro argument 
doesn't need outer braces. In an intuitive sense, a single 
``command" can be a macro argument, too. ``Command" may be understood 
as ``control sequence" (starting with a backslash), 
but some authors also have considered single \emph{characters} 
(character \emph{tokens}?) ``commands." Blank spaces, by contrast, 
are ignored when a macro looks for its argument.    %% 2012/05/17b

%% useless, just name items 2012/05/17b:
% We arrive at an ``intuitive recursive definition" of ``arglist:" \
% (\meta{i})\enspace The empty list is an arglist; it doesn't have an item 
% for 'dowith'. \ (\meta{ii})\enspace If <list> is an arglist, then 
% (a)~`{<ark>}<list>' is an arglist whose first item for 'dowith' is `{<ark>}';
% (b)~`<cs><list>' with a ``control sequence" <cs>
%     is an arglist whose first item is `<cs>' 
%     (for the command <cmd> to which 'dowith' presents <cs>); and
% (c)~`<char><list>' with a non-blank character <char>
%     is an arglist whose first item is `<char>'. \
% (\meta{iii})\enspace Nothing else is an arglist.
% 
% But keep in mind that this ``intuitive" understanding essentially is wrong.

\subsection{Anatomy of \TeX}
The documentation of v0.22 as of 2012-06-04 said that the package 
is about ``lists in \TeX's mouth." However, this was very wrong. 
I believed it following Alan Jeffrey's paper ``Lists in \TeX's Mouth",\footnote{%
    Alan Jeffrey: \tugbartref{tb11-2/tb28jeffrey}{``Lists in \TeX's Mouth,"} 
    \acro{TUG}boat Vol.~11 (1990), No.~2, pp.~237--245), 
    \urlhttpref{tug.org/TUGboat/tb11-2/tb28jeffrey.pdf}.}
in whose Section~2 you read:
\begin{quote}\it
  \TeX's programming facilities come in two forms---there 
  are \TeX's \emph{macros} which are expanded in its
  mouth, and some additional \emph{assignment} operations
  like \cs{def} which take place in the stomach.
\end{quote}
The macros that Jeffrey lists and describes in that article can be 
obtained as a \acro{CTAN} package 
\ctanpkgref{lambda-lists}.\footnote{\urlhttpref{ctan.org/pkg/lambda-lists}} 
If you follow the link given here (in the footnote), 
you currently (2012-11-03) read about this package:
\begin{quote}\it
    These list-processing macros avoid the reassignments 
    employed in the macros shown in Appendix D of the TeXbook: 
    all the manipulations take place in what Knuth is pleased to 
    call ``TeX's mouth".
\end{quote}
But Knuth doesn't. On page 267 of \TTb, you read:
\begin{quote}\it
  Chapter 7 has described the process by which input files
  are converted to lists of tokens in \TeX's ``mouth,"
  and Chapter 20 explained how expandable tokens are converted
  to unexpandable ones in \TeX's ``gullet" by a process 
  similar to regurgitation.
\end{quote}
I.e., the ``mouth" is \TeX's ``tokenizer," the inner part of what 
van Eijkhout calls \TeX's ``input processor" on, e.g., p.~15 of his 
\meta{\TeX~by Topic}.\footnote{It is available as a \acro{CTAN} package 
    \ctanpkgref{texbytopic} at \urlhttpref{ctan.org/pkg/texbytopic}.}
The exact rules the tokenizer follows are described on pp.~46f.\ 
(Chapter~8!) of \TTb.
\emph{Macro expansion} takes place in \TeX's \emph{``gullet"}, 
which van Eijkhout calls \TeX's ``expansion processor" (p.~16). 
Abrahams, Hargreaves, and Berry follow Knuth's terminology on
pp.~16 and 46f.\ of their \meta{\TeX~for the Impatient}.\footnote{It 
    is available as \acro{CTAN} package \ctanpkgref{impatient}, 
    \urlhttpref{ctan.org/pkg/impatient}.} 
% Only Jeffreys and my wrong old documentation say this were 
% ``\TeX's mouth"---and I made the same mistake with the 
% \ctanpkgref{bitelist} package.
%% <- 2012/11/19 ->

\TeX's gullet has been called ``\TeX's mouth" also 
in the documentation of my \ctanpkgdref{bitelist} package and 
in the documentation of the package \ctanpkgdref{bibleref-mouth}.

Moreover, I should have clarified that Jeffrey's paper deals with 
``lists" in some general, rather abstract sense, different from 
the kind of lists the present documentation tries to characterize 
as the objects for 'dowith'.

\subsection{\TeX's Tokens}
\label{sec:toks}

% The \emph{\TeX\ macro writer} understanding \TeX\ properly 
% does not really think of arglists. \TeX perts instead think of 
% What \TeX nically matters is
% what happens in ``\TeX's mouth,"\footnote{Cf.~\TTbp.~46.} 
% as some authors have suggested a metaphor,\footnote{%
The 'dowith' package is a tool that affects the order of tokens in 
\TeX's gullet.

\begin{smallpar}
The ``characters you type" enter ``\TeX's mouth" line by line, 
in a slightly modified appearance. Each line forms a \emph{string}.
\TeX\ takes initial substrings away from it and turns them 
into \emph{tokens} that are appended to the right of \TeX's 
\emph{expansion buffer} (``gullet").

%% Removing 2012/11/04:
% More formally, \TeX\ has a \emph{character buffer.}
% It forms a single token from an initial segment of the buffer content---unless 
% there is a special situation with blank spaces or something pathological.
% When an \emph{escape character}, as the backslash usually is one, 
% has been noticed recently (that isn't followed by another one immediately), 
% the character buffer may need to be feeded from more outside, 
%% <- especially wrong, 2012/11/04
% until it contains enough material to form a token from. 

There are \emph{two kinds of tokens} here: \emph{named} tokens 
and \emph{character} tokens. ``Named" tokens usually are referred
to as ``control sequence tokens" or just ``control sequences"---I~really 
want to avoid those horrible confusions from \TTb.
There never are any ``parameter tokens" in \TeX's gullet 
(perhaps unless one considers a one-step macro expansion 
 a two-or-more-step procedure).
%  merged 2012/11/04
%% Moving down 2012/11/04
The character(s) \emph{after} the escape character until 
some delimiting character form a \emph{string} that is the 
\emph{name} of the token that is formed---a \emph{named} 
token, as I am saying.
% What has been used to form a token is removed from the character buffer.
%% 2012/11/04:
\emph{Character} tokens are formed by removing a character from the 
beginning of the character buffer and appending it to the token buffer 
paired with its \emph{category} code.

For every \emph{string of characters}, there is exactly one 
(possible) \emph{named token} whose name the string is.\footnote{``Possible" 
    refers to the fact that \TeX\ does not store named tokens anywhere 
    before they appear in its gullet, maybe apart from ``primitive" 
    tokens that have a ``pre-assigned meaning" when a \TeX\ run     %% \ 2012/11/04
    starts.---What is more bad with my claim is that 
    the \TeX\ program by design cannot extend its memory arbitrarily---even 
    not using the ``cloud''---, so it doesn't support tokens 
    whose name lengths are above a certain limit.}
It is so common (starting from \TTb) to denote the token whose name 
is <string> by \lq\verb+\+<string>\rq. For instance, the token whose 
name is `input' is denoted by \qtd{&\input}. On the other hand, 
on page~7 of \TTb\ \qtd{&\input} is a ``string of characters."
With this notation, it is already difficult to explain what 
the \LaTeX\ command `\DeclareRobustCommand' does or what the difference 
between a starred \LaTeX\ command and a starred \LaTeX\ environment is.\footnote{%
    A reader knowing \LaTeX\ only thinks that \qtd{\code{&\\\codesp}} 
    is the result of typing a double backslash and a space 
    and that \qtd{\cs{equation*}} is the ``command" \cs{equation} 
    followed by a `*'.} 
\TTb\ makes it worse by saying on page~39: 
\meta{``A control sequence is considered to be a single object 
        that is no longer composed of a sequence of symbols."}
So ``it depends" whether \qtd{&\input} is a string of characters or not---it \emph{is 
before} tokenization, but \emph{no longer} afterwards. 
So if you have two computers and start a \TeX\ run on each of them 
with a little difference in time, there will be a moment where 
\qtd{&\input} is a string on the one computer but not on the other? 
This 
appears to me %%% is %% 2012/11/04
like saying \meta{``When we apply the square root function to 
the number 4, the number 4 will no longer be the number 4, 
it will be the number 2 instead."}
\end{smallpar}

\TTb\ does offer an alternative notation for named tokens: ``boxing;" 
so the token whose name is `input' can be denoted by the rather 
``graphical" notation \qtd{|input|} (used only exceptionally).\footnote{%
    The box notation is introduced on page~38 without explanation, 
    as if it explained something.} 
\emph{I}~would suggest something like 
\qtd{\NTOK{input}} for clarity and \qtd{\ntok{input}} 
for brevity.\footnote{I am suggesting the question mark for named tokens 
 since \TeX\ ``must look up the current definition" of a named token 
 according to \TTbp.~39, while the meaning of character tokens rather 
 is ``fixed," at least according to \TTbp.~39. However, 
    \emph{active}-character such as .&~ are in the same situation 
    as named tokens as to this respect. The dot notation may be fine for them, 
    though.}

\begin{smallpar}
%% rm. 2012/11/04, cf. above
% \emph{Character tokens} get into \TeX's mouth by tokenization 
% when characters begin the buffer content while \emph{not} scanning 
% a name for a named token. A single character then is removed from the 
% character buffer, and a token storing its character code and current 
% category code is pushed into \TeX's mouth.

Named tokens may get into \TeX's gullet by ``tokenization" as described above, 
i.e., they are drawn from the character buffer. But they also can 
appear in \TeX's gullet ``from within," 
by the manipulation inside \TeX's gullet. 

More formally, those manipulations are called ``expansion,"
and \TeX's gullet can be conceived of as a \emph{token buffer} 
that is feeded to the right (or end) by tokenization from the character buffer. 
Expansion means that certain tokens in the token buffer are substituted 
by other ones. This way tokens may get into \TeX's gullet that 
emerged from tokenization a ``long time ago", maybe in a previous 
run that created the \emph{format} (\TeX's variant \code{INITEX}); 
or tokens may appear by some hardwired expansion function. 

However, \emph{named tokens} may get into \TeX's gullet 
also by \emph{expansion}, never having been drawn by tokenization 
and not being hardwired. This happens by the `\csname' 
% name      %%  rm. 2012/07/25
construct. 
The input \emph{code} may contain 
\[`\csname tupni\endcsname'\]
This may be converted into 7 tokens entering \TeX's gullet, 
the first one being \NTOK{csname}, the last one 
\NTOK{endcsname}, and five character tokens in between. 
Due to some \emph{function} (which I would denote as *\code{csname})
originally associated with the token \NTOK{csname}, 
those seven tokens then are replaced by \NTOK{tupni}, 
the named token whose name is `tupni'. It is not required that 
the \TeX\ program knows about a token \NTOK{tupni}, 
neither anybody must type \qtd{&\tupni} in any file.\footnote{These 
    considerations may not be essential here, 
    rather a draft for a paper. Using 'dowith', 
    one better just thinks of the arglist items one actually lists.}
\end{smallpar}

\subsection{Arglists vs.\ Lists of Tokens---Example}
% \subsection{Arglists and 'dowith' \TeX nically}
% In the \TeX nical sense, I think of arglists \emph{and their items}
% as follows. Arglists are lists (or sequences) of tokens. 
% What is somewhat difficult is that the \emph{items} of a token list 
% usually are \emph{tokens.} Especially, the curly braces in the code 
% you type usually are converted into certain \emph{character tokens}
% that are single items of the resulting token list.\footnote{I discover 
%     this conceptual puzzle 2012-05-16.} 
% 
% The conceptual trap here may have resulted from denoting lists 
% in a \emph{\Wikienref{juxtaposition}} notation. 
% In Section~\ref{sec:lists-intuit}, I have written `aa' for a 
% ``list" of \emph{two} items. ``List" is rather a \emph{computer science} 
% term, its mathematical counterpart rather is the notion of (finite) 
% \emph{sequences.} The usual \emph{mathematical} notation for a finite sequence 
% writes list items surrounded by \emph{\wikienref{bracket}{brackets}}
% (round---\emph{\Wikienref{parentheses}}---seems to be more common than square, 
%  also \wikienref{angle brackets}{chevrons} are used).
% 
% So the \emph{string} `aa' can be written more clearly---mathematically---as 
% $(`a',`a')$. The trap with `\a' is that it could be \emph{either} the \emph{string} 
% $(`\',`a')$ \emph{or} the one-item list $(`\a')$ of strings. 
% But even in the latter case, I urge not to consider it a \emph{\TeX\ token}. 
% Rather, I consider `\a' a mistaken way of referring to the 
% named token \NTOK{a} whose name is the string `a' (or $(`a')$).
% (Section~\ref{sec:toks}). 
% 
% However, the conceptual trap about arglists and token lists 
% (there must have been some ``Arglist"!\@) lurks on another level, 
% as follows. Recall \TTb's notation of \chtok{<char>}{<cat>}
% for the \emph{character token} that \TeX's tokenizer forms 
% from <char> in the character buffer when <char>'s category code is <cat>.
% Usually, the \emph{character} `a' is converted into the 
% \emph{character token} \lttok{a}, `{' is converted into \lbtok, 
% and `}' is converted into \rbtok. 
% 
% We are turning to some \strong{examples} and \strong{counterexamples}. 
% Let us see what confusions occurred in the ``intuitive" view on 
% arglists in Section~\ref{sec:arglists-intuit}.
% 
% % \begin{example}{1} It 
% First, it was bad in Section~\ref{sec:arglists-intuit}
% to think that `aa' is a two-item arglist. It was confused with something 
% like $\lttok{a}\lttok{a}$. The latter looks like a token list---or is it an arglist? 
% Both? Anyway, it is juxtaposition notation applied to tokens, 
% mathematically it is $(\lttok{a},\lttok{a})$, so~\dots
% % \end{example}
% 
% Now let us reconsider the ``intuitive recursive definition" of arglist. 
% Or let us look at a recursive definition of \emph{token} list.
% \inlineitem{1} There is nothing wrong with saying the the empty list is 
% a token list, the same holds for arglists. 
% \inlineitem{2} a.~When we attach an arbitrary token to a \emph{token} list 
% (at the left), the result is another token list---fine. 
% There are only certain difficulties with ``handling" special token lists 
% such as $\lbtok\lttok{a} = (\lbtok,\lttok{a})$. 
% And the latter is \emph{not} an \emph{arglist!} \
% b.~When we attach a \emph{named} token to an arglist (at the left), 
% the result \emph{is} an \emph{arglist}.\footnote{This is 
%     a \emph{conjecture} only right now---2012-05-16---in which I strongly believe. 
%     Likewise later.} 
% c.~When we attach a \emph{letter} token \lttok{<char>} to an arglist 
% (at the left), the result \emph{is} an \emph{arglist}.
% d.~When we talked about ``brace groups," they seemed to be \emph{strings} 
% of characters. Instead, I would like to suggest that a brace group 
% is an \emph{arglist}\pdots
% When \lbtok\ is attached to the left of an arglist and \rbtok\ to the right, 
% the result is an arglist---this is what I would call a ``brace group"!
% e.~The \Wikienref{concatenation} of two arglists is an arglist. 
% 
% \begin{smallpar}
% The above notion of ``attaching" a token to a token list or an arglist 
% % should be clarified, but instead of a general definition in terms of 
% % ``words" of formal languages, examples may suffice here and now.\footnote{2012-05-16.}
% should be clarified. Attaching an item $j$ at the left of a list $\lambda$ 
% is the same as concatenating the one-item list $(j)$ with $\lambda$\pdots 
% but the English \wikienref{append}{Wikipedia} seems to explain concatenation by 
% \emph{appending}. As we \emph{remove} items one-by-one from the \emph{left} 
% (beginning) of a list, I prefer the inverse \emph{prepending} items as basic operation 
% for building lists---cf.\ \Wikienref{CAR and CDR}.
% \end{smallpar}
% 
% We won't complete a formal (recursive) definition of arglist here and 
% now.\footnote{2012-05-16.} Just observe that ``brace groups" 
% make the difference between token lists and arglists. 
% Recall that an ``item" of an arglist ``operationally" is defined 
% as something that a one-parameter macro removes. 
% Such a macro removes certain single tokens 
% (space tokens not among them)---and entire ``brace groups"! 
% Especially, consider \[\lbtok\rbtok\]%%%.
% Actually, this is another ambiguous notation. 
% If it refers to $(\lbtok,\rbtok)$, it is a \emph{token} list, 
% not an arglist. If it refers to $(\lbtok\rbtok)$---it should 
% better refer to $((\lbtok,\rbtok))$, which is a 
% \emph{one-item arglist} whose only item is the former two-item token list!
% 
%% <- 2012/05/17b ->
Let us reconsider the examples from Sections~\ref{sec:lists-intuit}
and~\ref{sec:arglists-intuit}, and pack them into a single example. 
If you type a file line
\begin{equation}
  `a a\a{a}'
\end{equation}
(\emph{eight} keystrokes),
it should usually be converted into this \emph{seven}-item list 
of 
%%% (five) %% rm. 2015/11/14 -- ???
tokens:
\begin{equation}
  \label{eq:toks}
  \lttok{a}\;\sptok\;\lttok{a}\;\NTOK{a}\;\lbtok\;\lttok{a}\;\rbtok
\end{equation}
---with notation from Section~\ref{sec:toks} and
\TTb's notation \chtok{<char>}{<cat>} 
for the \emph{character token} that \TeX's tokenizer forms 
from <char> in the character buffer when <char>'s category code is <cat>.

It turns out that the token list in \ref{eq:toks} 
provides an arglist of \emph{four} items: The token \lttok{a}
at the first and third place, the named token \NTOK{a}, and the entire 
token list $\lbtok\lttok{a}\rbtok$ as a single item---a ``brace group."
The space token is ignored.\footnote{\TTbp.~201: ``\TeX\ doesn't use 
single spaces as undelimited arguments."}

You can try this after `\renewcommand{\a}{A}'\footnote{Otherwise 
    \cs{a} is a one-parameter macro that breaks 'dowith''s control.}
with 'dowith':
\begin{equation}
  \label{eq:sample-code}
  `\DoWith\typein a a\a{a}\StopDoing'
\end{equation}
Then \LaTeX\ shows `a', `a', `A' from `\a', and another `a' from within 
the braces---`\typein' (as any macro with arguments) removes them.

% \show\a
% \a{a}
\renewcommand*{\a}{A}
% {\MakeNormalHere\# \newcommand*{\TypeOut}[1]{\typein{#1}}
%  \global\let\TypeOut\TypeOut}
% \let\TypeOut\typein
% \let\TypeOut\typeout
% \DoWith\TypeOut a a\a{a}\StopDoing
% \DoWith\typein a a\a{a}\StopDoing
% { %%% \tracingmacros=1 \tracingonline=1
%  \MakeNormalHere\# \AssignCatCodeTo{2}\]
%  \DoWith\typein #a\a{a]\StopDoing}

I have avoided saying \ref{eq:toks} \emph{were} an arglist of 4 items. 
The mathematical basic way of writing lists---understood as finite
\emph{\wikienref{sequence}{sequences}}---as ``commma-separated lists" 
within \wikienref{bracket}{brackets} may clarify the difference 
(that the \Wikienref{juxtaposition} notation tends to conceal). 
The \emph{token} list is 
\begin{equation}
  \label{eq:toks,}
  (\lttok{a}, \sptok, \lttok{a}, \NTOK{a}, \lbtok, \lttok{a}, \rbtok)
\end{equation}
while the list of macro arguments is
\begin{equation}
  \label{eq:args}
  (\lttok{a}, \lttok{a}, \NTOK{a}, (\lbtok, \lttok{a}, \rbtok)).
\end{equation}
\ref{eq:toks} or \ref{eq:toks,} simply is \emph{not} an arglist 
(since neither \lbtok\ nor \rbtok\ can be a macro argument), 
and the arglist \ref{eq:args} ``provided" by the list of tokens 
is \emph{not} a list of \emph{tokens}---its final item is a 
three-item list of tokens, and a token cannot be a list of 
two or more tokens itself(\emph{!?}).

\subsection{Another Notation and the Example's Steps}
\label{sec:steps}                                           %% 2012/05/20
\begin{smallpar}
To write token lists easier and hopefully easier to read, 
I would suggest writing \qtd{.<char>} for the character token 
that the tokenizer ``usually" forms from character <char>, i.e., 
adding the \emph{standard} category code as in \TTb\ (page~37).
Then \ref{eq:toks} would read\footnote{See Section~\ref{sec:toks} for the question mark.}
\begin{equation}
  .&a\,.\codesp\,.&a\,\ntok{a}\,.\codelb\,.&a\,.\coderb
\end{equation}
and the corresponding arglist is
\begin{equation}
  (.&a,.&a,\ntok{a},(.\codelb\,.&a\,.\coderb))
\end{equation}
In ``retrospect," the result of tokenizing \ref{eq:toks} should be
\begin{equation}
  \label{eq:retro}
  \ntok{DoWith}\,\ntok{typein}\,
  .&a\,.\codesp\,.&a\,\ntok{a}\,.\codelb\,.&a\,.\coderb\,
  \ntok{StopDoing}
\end{equation}
and the intention is that it works like 
\begin{equation}
  %% added \,s 2012/06/07:
  \ntok{typein}\,.\codelb\,.&a\,.\coderb\,
  \ntok{typein}\,.\codelb\,.&a\,.\coderb\,
  \ntok{typein}\,.\codelb \ntok{a}\,.\coderb\,
  \ntok{typein}\,.\codelb\,.&a\,.\coderb
\end{equation}
(The definition of `\DoWith' in Section~\ref{sec:core} indeed adds surrounding braces, 
 if missing.)
However, \TeX\ rather tries to work with as few tokens ahead as possible.
When it finds \ntok{DoWith} and the latter's meaning is the one intended 
by 'dowith', it first looks for nothing more than the two arguments 
required by our definition of `\DoWith'. A few moments later, 
the token buffer's content will just be\footnote{If you use 
    `&\DoWithAllOf&\typein\codeLB a\codeSP a&\a\codeLB a\codeRB\codeRB'
    %% <- \cs bad 2015/11/14
    instead, the entire token sequence \ref{eq:retro} will appear in the 
    token buffer ``at once."}
\begin{equation}
  \ntok{typein}.\codelb.&a.\coderb\,\ntok{expandafter}\,\ntok{DoWith}\,
  \ntok{expandafter}\,\ntok{typein}\,\ntok{fi}
\end{equation}
Next $\ntok{typein}.\codelb.&a.\coderb$ is expanded according to the code for 
`\typein' in \file{latex.ltx}. Some unexpandable tokens will emerge 
and be moved into the ``instruction buffer,"                %% cmd -> instr 2012/11/04
and you should get a screen
message with `a' and a prompt. When you have entered something, 
the remaining \ntok{expandafter} tokens and the \ntok{fi} will be 
removed from the character buffer, and it contains only
\begin{equation}
  \ntok{DoWith}\,\ntok{typein}
\end{equation}
Another token is ordered from the tokenizer to provide a second 
argument for expanding \ntok{DoWith}. The token .\codesp\ comes in, 
but that doesn't serve as a macro argument. It is removed, and the 
next token is .&a. The same story as before happens, until the 
named token \ntok{a} is found\pdots
\end{smallpar}

\subsection{Summary of Possible Arglist Items}
\begin{smallpar}
For $0\leq i\leq 15$, let $\Chi_i$ be the set of character tokens 
of category code $i$. $\Chi_1$ is the set of tokens working like 
$\lbtok$, and $\Chi_2$ is the set of tokens working like $\rbtok$.

Let $E$ be the set $\{3,4,6,7,8,11,12,13\}$. 
These numbers are the category codes for 
\meta{math},   \meta{align}, \meta{parameter}, \meta{super}, \meta{sub}, 
\meta{letter}, \meta{other}, \meta{active} respectively. 
Let $\Chi_E$ be the set of character tokens of category code in $E$
(so $\Chi_E=\bigcup_{i\in E}\Chi_i$).

Let $\circ$ be the \emph{concatenation} operation among token lists.\footnote{%
    %% 2012/05/18
    \TODO: Define for representations by maps, or:              %% \ 2012/11/05
    ``Concatenation is about as basic as natural numbers and is 
      understood in terms of axioms rather than by a definition.''---See 
        notes from 2011 (even with attempts with \Wikienref{Category theory})
%         the \wikienref{Sequence}{English} 
        the English Wikipedia for 
        \wikienref{Sequence}{sequences}---\wikideref{Folge (Mathematik)}{German} 
        article too much restricted to maps.}

The following kinds of token lists form a single arglist item, 
i.e., can serve as an argument for an undelimited parameter:
\begin{enumerate}
  \item a \emph{named} token, or the single-token list consisting of it, 
        if you prefer that;
  \item a \emph{character} token from $\Chi_E$ or the list consisting of it; 
  \item a \emph{brace group.} 
        That is a token list meeting the following conditions:
        \inlineitem{1} its \emph{first} token is in $\Chi_1$,
        \inlineitem{2} its \emph{last} token is in $\Chi_2$,
        \inlineitem{3} it has as many occurrences of tokens from $\Chi_1$ as from $\Chi_2$,
        \inlineitem{4} if it is split as $\lambda\circ\rho$, there are not 
            more $\Chi_2$ occurrences in $\lambda$ than $\Chi_1$ occurrences in $\rho$
            (``don't close before opening").
\end{enumerate}
The second claim can be checked with
\begin{equation}
  `\DoWith\typein$#^_a1~\StopDoing'
\end{equation}
%  \begingroup
%   \def\a{A}\MakeNormalHere\# 
%   \DoWith\typein$# ^_a1~\StopDoing
%   \DoWith\typein#1\StopDoing
%  \endgroup
as to what works. 
(The claim is not affected by one or two surprises.)\footnote{Moreover, 
    `&\DoWith&\typein&#1&\StopDoing'              %% \cs bad 2015/11/14
    tells something about ``parameter tokens."} 
Characters with different category codes 
either are not converted into a character token\footnote{\TTbp.~47.} 
or are not accepted as macro arguments. The latter applies to ``brace" tokens 
in $\Chi_1$, $\Chi_2$ and to the single space token \sptok.

As to \emph{``brace groups"}, the third and fourth condition above 
are intended to say that what is between the two outer tokens 
is $\langle$balanced text$\rangle$ in the sense of \TTbp p.~275f. and~385;
i.e., for two tokens $a$, $b$ and a token list $\beta$, 
$(a)\circ\beta\circ(b)$ is a brace group exactly if $a$ is from $\Chi_1$,
$b$ is from $\Chi_2$, and $\beta$ is $\langle$balanced text$\rangle$. 
The conditions are more formal than what I can find in \TTb, 
but still they don't give me an idea of all possibities. 
This should be improved by the following recursive definition:

\begin{trivlist}\item 
B1.~The empty list is balanced text. 
B2.~For any token $t$ not in $\Chi_1$ or $\Chi_2$, 
    the single-item token list $(t)$ is balanced text. 
    (Such a token is either a \emph{named} token or a 
     \emph{character} token from $\Chi_E$ or \emph{the space token}~\sptok.)
B3.~If $\alpha$ and $\beta$ are balanced texts, 
    then $\alpha\circ\beta$ is balanced text.
B4.~If $\beta$ is balanced text, 
    $a$ is from $\Chi_1$, and $b$ is from $\Chi_2$, 
    then $(a)\circ\beta\circ(b)$ is balanced text.
    (This is a brace group, and the only way of getting a brace group.)
B5.~Nothing else is balanced text.
\end{trivlist}

In other words, a token list is a brace group if and only if 
it is balanced text and starts with a token from $\Chi_1$
and ends with a token from $\Chi_2$.\footnote{Again, this may be  more 
    of a draft for a paper, or notes for it, than package documentation.}
\end{smallpar}

\subsection{Summary: ``Commands" Usable with 'dowith'}
In the \hyperref[sec:apply]{implementation section,} 
you learn about 
\[`\DoWith<cmd>',\quad `\DoWithAllOf<cmd>',\quad 
    \mbox{and}\quad  `\DoWithAllIn<cmd>'.\] 
(\LaTeX\ users may type `{<cmd>}' instead.)
What <cmd>s are allowed?

\begin{enumerate}
  \item All \strong{one-parameter macros} <cmd> work this way, 
        unless there are programming mistakes outside 'dowith' 
        (also thinking of arguments that take over control 
         from 'dowith' commands before the argument list is finished). 
  \item \strong{Other one-parameter} ``commands" <cmd> such as
        \TeX\ \strong{primitives} may work---you must think of 
        the fact that surrounding \emph{braces} are added.\footnote{\TODO: %% \ 2012/11/05
            in the future, variants not adding braces could be added.}
        So the \strong{primitives} `\hbox' and `\vbox' work, 
        for instance. `\show' is an example that doesn't work at all, 
        it takes the single starting brace token and then confuses 
        `\DoWith'.
  \item Some <cmd>s taking \strong{no argument} may make sense, e.g., 
        for getting 
          \begin{enumerate}
            \DoWithAllOf{\item}{{apples,}{pears,}{peaches}}
          \end{enumerate}
        from 
        \begin{quote}
          &\begin{enumerate}\\
          \null\code{~~}&\DoWithAllOf{&\item}{%
            \codebd{apples,}\codebd{pears,}\codebd{peaches}}\\
          &\end{enumerate}
        \end{quote}
        Recall that `\item' at most takes an \emph{optional} argument.
  \item <cmd> must \strong{not take more than one} parameter. 
        %% see def. \DoWith
        A different package will support multi-parameter macros.
\end{enumerate}

% Rather, ``list" is a term from \Wikienref{computer science} here. 
% It corresponds to the notion of ``sequence" in mathematics 
% and to ``word" with formal languages.
% 
% Especially, we have (\meta{i})~lists of \emph{characters} that 
% the \emph{\TeX} program recieves from files and 
% (\meta{ii})~lists of so-called \emph{tokens} 
% that the \TeX\ program forms from the incoming list of characters 
% and that it works on.
% 
% There are \emph{more} kinds of lists that \TeX\ works on. 
% Here we are dealing with ``\TeX's gullet."\footnote{Cf.\ documentation 
%     of the \ctanpkgref{bitelist} package.}
% \TeX's mouth processes tokens formed from character input. 
% It may turn a list `<toks-a><toks_b><toks-c>' of tokens 
% into a list `<toks-a><toks-B><toks-c>', i.e., 
% replace <toks-b> by <toks-B>---by so-called ``expansion". 
% The result may be subject to expansion as well. 
% When nothing is left to be expanded, results are passed to another, 
% more interior subprocessor of \TeX. This one is deeper than \TeX's mouth, 
% we are not concerned with that here.
% 
% Not \emph{all} of those tokens in ``\TeX's mouth" are formed 
% (``directly") from input characters. 
% Rather, some \cs{csname} function may form \emph{new} tokens 
% from other tokens in \TeX's mouth---and place them there again. 
% This way characters `\csname a\endcsname' you type may just work like 
% `\a'.
% 
% It is difficult to tell on the \emph{character level} what the 
% present package does. The relation between incoming characters 
% and resulting tokens can hardly be explained by a single sentence or so.
% However, <cmd> here refers to a command <macro> with a single 
% undelimited macro. More precisely, a certain character sequence 
% (``string") in the code you type, will be converted into a TODO
% 
% \subsection{Separators}
% TODO
% 
% \subsection{Iterators and \TeX's Mouth}
% TODO
% 
\section{Similar Commands in other Packages}
%% <- section again 2012/05/17b
\subsection{``Heavy" Packages}                      %% add. heading 2012/05/17b
The \CtanPkgRef{etex}{$\varepsilon$-TeX}-related    %% e->... 2012/05/09
packages \ctanpkgref{etextools} 
(Florent Chervet), \ctanpkgref{etoolbox} (Philipp Lehman), 
and \ctanpkgref{texapi} (Paul Isambert) seem to include and 
(very much) extend the functionality of 'dowith'. 
Also the `\ForEach'\texttt{\dots\unkern\@} macros of            %% \@ 2012/06/03
\ctanpkgref{forarray} (Christian Schr\"oppel) seem to extend 
the present `\DoWith'\texttt{\dots\unkern\@} commands.          %% \@ 2012/06/03
Moreover, Ahmed Musa describes such commands as                 %% 2012/06/03
``Parsing \qtd{tsv} lists" in documenting his 
\ctanpkgref{catoptions} package.
\ctanpkgref{moredefs} (Matt Swift) provides list handling commands 
like the few that are here.\footnote{\ctanpkgref{arrayjobx}%    %% % 2012/11/05
    provides somewhat ``exotic" handling of ``lists".}          %% 2012/05/10
%% 2012/11/05:
---In October 2012, Ahmed Musa's \ctanpkgref{loops} appeared on 
\acro{CTAN}, offering loops of several ``categories" about as those 
that are listed below, very elaborate.% 
%% /2012/11/05
% (I~do not want to load that much.)
%% <- 2012/11/04 ->
---I do not want to load that much. I need and only need something 
excessively simple, very few lines of code, as presented in 
Section~\ref{sec:implement}. The next sections somewhat point 
out single features of loop constructs that I do not want to have.

\subsection{Separators}                             %% add. heading 2012/05/17b
Regarding \LaTeX\ macros in `latex.ltx', the basic macro `\DoWith' of 
the present package resembles |\@tfor| very much, which likewise deals 
with lists without separators. By contrast, \LaTeX's |\@for| deals with 
\emph{comma-separated} lists (such as lists of package options). 
With comma-separated lists, a ``string" of characters counts as 
an item when it is delimited by commas, or by a comma and the 
list ``border," or spaces may be used as separators additionally. 
However, when \LaTeX\ analyzes such lists (in ``\TeX's gullet"), 
it uses representations by \emph{character tokens} of them.

%% moved here 2012/05/17b:
% %% 2011/11/10:
% Also Heiko Oberdiek's \ctanpkgref{zref} deals with ``lists" of 
% ``properties" of ``entities," 
%% 2011/11/11: comma separated!

%% 2012/05/09f.:
The more recent \ctanpkgref{lmake} 
(Shengjun Pan)                                      %% 2012/05/18
provides a key-value syntax for printing lists of 
complex mathematical expressions easily (using some assignments) 
as well as defining commands according to a pattern from a list.
Those lists are comma-separated.

\subsection{``For" Loops vs.\ ``Foreach" Loops}     %% mv. up 2012/05/20
What about \ctanpkgref{forloop} (Nick Setzer), 
\ctanpkgref{multido} (Timothy Van Zandt, Rolf Nie\-praksch, Herbert 
Vo\ss), and \ctanpkgref{xfor} (Nicola Talbot)? 

'xfor' is just a reimplementation of `\@for'. 
'forloop' and 'multido' are more close to ``real \qtd{for} loops"
(cf.\ \wikienref{for loop}{\meta{Wikipedia}}).
Loops of the latter kind go through a certain set as well, 
but such sets rather consist of \emph{numbers} and are exhausted 
by incrementing (or also decrementing) variables 
\wikienref{Loop counter}{(counters).}
This is essentially not needed 
(neither helpful)                                       %% 2012/05/20
when a list literally is 
\emph{enumerated}---such loops are distinguished as 
\wikienref{Foreach loop}{``foreach loops."}             %% 2012/05/19

\subsection{Iterators}                              %% add. heading 2012/05/17b
So `\DoWith' and |\@tfor| rather provide ``foreach" loops. 
A major difference between them is that the latter uses a 
\wikienref{For loop}{``loop variable"} or 
%%% rather                                              %% rm. 2012/05/20
\wikienref{Iterator}{``iterator"} to which the elements 
of the list are assigned. 
`\DoWith<cmd>' does not use such a loop variable 
or such assignments and thus is ``expandable" at least when <cmd> 
(and the elements, depending on <cmd>) are expandable.
On the other hand, `\@tfor' applies some procedure to the list 
elements without needing a \emph{name} for the procedure 
(or a \emph{macro} storing the procedure). 
%
I wondered whether behind \LaTeX's 
`\@tfor' (and `\@for') there was an ``ideological" consideration 
such as ``A loop must have a loop variable!"\pdots
% However, avoiding usage of a macro name 
% (to store the ``loop body" code)                        %% 2012/05/18
% and a macro parameter 
% (to incorporate the list item into the body code)       %% 2012/05/18
% may have been a good reason.
%% <- 2012/05/19

%% 2012/05/20:
Hopefully more clearly on ``loop variable" vs.\ our approach: 
In order to run
\[<code-before><item><code-after>\]
on each <item> of a <list>, \emph{we here}
% \[
\begin{equation}
  \mbox{define} \quad `\do' \quad \mbox{as} \quad 
  `#1'\;\to\;`<code-before>#1<code-after>'
\end{equation}
% \]
and then run `\do{<item>}' for each <item> in <list>,\footnote{Cf.~description 
    of procedure in terms of tokens in Section~\ref{sec:steps}.} 
\begin{equation}
  \mbox{always replacing}\quad  `\do{<item>}'\quad \mbox{by}\quad  `\do{<item>}\do'. 
\end{equation}
(`\do' is only an example command that 'dowith' supports especially.)
% The ``received" approach from ``usual" programming languages, 
% \file{latex.ltx}, and its followers is
In \file{latex.ltx} instead, we find things like
% \[
\begin{equation}
  \label{eq:tfor}
  `\@tfor\@tmp:=<list>\do{<code-before>\@tmp<code-after>}'
\end{equation}
% \]
where `\@tmp' is a \emph{macro} that is set to be <item> at each 
iteration of the loop, by 
\begin{equation}
  `\def\@tmp{<item>}'
\end{equation}
within `\@tforloop'. After that,
\begin{equation}
  \label{eq:code-tmp-code}
  `<code-before>\@tmp<code-after>'
\end{equation}
from \ref{eq:tfor} is run.---\ref{eq:code-tmp-code} like 
\ref{eq:tfor} is stored in a larger macro. `\do' in \ref{eq:tfor}
does not act as a macro, it just delimits a macro parameter 
in order give a feeling of some familiar programming structure. 
This organisation of macros is fine when the loop body code 
is only used by the containing macro, while the 'dowith' approach 
to store the ``loop body" in an own macro has been useful when 
the loop body code also is used for different purposes 
or when it has been introduced before I thought of using it 
in a loop.
% \[\mbox{defining}\;`\@tmp'\;\mbox{as}\;<item>\quad \mbox{and run} \quad 
%   `<code-before>\@tmp<code-after>'\]
% for each <item> in <list>.

Note that this only was an example. In general, <item> may appear 
more than once in the ``loop body."

``Expandability" by \emph{avoiding} something iterating `\def\@tmp{<item>}'
and doing iteration in \TeX's gullet (`\do' or so must have been defined earlier)
is essential especially within \cs{write}. 
Assignments do not work there. A major motivation for developing 
'dowith' developed with the \ctanpkgref{blog} package 
that \cs{write}s \acro{HTML} code.
Assignments happen in ``\TeX's stomach."                    %% 2012/11/04
That place might be called the 
``instruction buffer" to which                              %% cmd -> instr 2012/11/04
the ``expansion processor" moves items from the incoming token buffer 
that cannot be expanded (any more).

\subsection{Separator Macros}                       %% add. heading 2012/05/17b
Commands like `\DoWith' also could save tokens thinking of list macros 
(in \LaTeX/`latex.ltx') that use a \emph{separator macro} 
which may be used as a \emph{command} to be applied to the list 
elements. One example is 
`\dospecials' that already is in Plain \TeX\ and expands to 
\[`\do\ \do\\\do\{\do\}\do\$\do\&\do\#\do\^\do\_\do\%\do\~'\]
% \begin{verbatim}
% `\do\ \do\\\do\{\do\}\do\$\do\&\do\#\do\^\do\_\do\%\do\~'
% \end{verbatim}
An important application of `\dospecials' is temporarily 
switching off the ``special" functionality of the ``elements" 
in `\dospecials'. With \LaTeX, this may happen thus:
\[`\let\do\@makeother\dospecials'\]
% \begin{verbatim}
% `\let\do\@makeother
% \end{verbatim}
With 'dowith', you can do the same with a shorter variant 
`\specials' of `\dospecials', defined by 
\[`\def\specials{\ \\\{\}\$\&\#\^\_\%\~}'\]
and then 
\[`\DoWithAllIn\@makeother\specials'\]
`latex.ltx' uses `\@elt' instead of `\do' for its own list macros.

\subsection{Ye Olde \cs{loop}}                      %% add. heading 2012/05/17b
%% added 2011/11/03:
There also is |\loop<loop-body>\repeat| in Plain \TeX\ and a 
refined\footnote{Using Kabelschacht's suggestion, cf.\ Section~\ref{sec:kabel}} 
version of it in `latex.ltx'. It is \emph{not} expandable 
since it starts with an assignment for `\body' (Plain \TeX) or 
`\iterate' (`latex.ltx'), 
%% add. 2012/05/20:
and then some assignments are needed to stop the loop, 
such as incrementing or decrementing a \emph{counter.} 
As to the programming structure, it is very simple and general, 
I think any kind of loop can be implemented by this
(apart from nested loops). E.g., I realize\footnote{2012-05-20}
that even a ``foreach" loop could be implemented by managing 
a list macro, e.g., using \LaTeX's internal `\@next'.
%% rm. 2012/05/20:
% that you cannot probably
% what kind of loops it addresses. However, the applications 
% I have seen have been ``for" or (rather) ``while" loops. 
% ``While" loops can ``emulate" ``for" and ``foreach" loops 
% by having the ``incrementation" method or the ``enumeration" method in 
% their body. This is quite obvious for ``for" loops, not quite so for 
% ``foreach" loops; which for practical application (in my view) means 
% that neither \LaTeX/\TeX's `\loop' macro nor in general ``while" loops 
% is/are very helpful for implementing ``foreach" loops, 
% as rather `\DoWith' and similar constructions are.
% The reason for this is (as it seems to me) is that you 
% (a human being) can much more easily enumerate (``list") 
% the items of a list (you have in mind) than define the \emph{method} 
% that (allegedly) is behind your enumeration. \ \meta{Example:} 
% \[`\DoWithAllOf{\printsamplearea}{\red\green\blue}'\]
% ---\emph{how} (according to what ``method"?) did you ``proceed" from 
% `\red' to `\green' and from `\green' to `\blue'?

\subsection{Without Iterator and Separators}        %% add. heading 2012/05/17b
%% 2011/11/07:
In \LaTeX's \ctanpkgref{tools} bundle, \ctanpkgref{xspace}
%% add. 2012/05/18:
was developed in the nineties by David Carlisle. 
It had a rather fixed exception list implemented by a deeply
nested conditional. In 2004 Morton H\o gholm joined, 
and now 'xspace' has a list macro 
`\@xspace@exceptions@tlp'                           %% was xpspace 2012/05/17b
without separators. 
It is handled like here, except that it ``breaks" the loop 
when an item is found that applies.
%% add. 2012/05/17bf.:
After the ``next" token is stored by the usual \cs{futurelet}, 
the exception list is searched without using an iterator. 
Addition and removal commands are provided as well. 

%   \pagebreak                                                %% 2012/05/17b
% \newpage                                                    %% 2012/05/19
\section{Implementation}                                    %% 2012/05/10
\label{sec:implement}
\subsection{Package File Header (Legalese)}                 %% sub 2012/05/10
\ProvidesFile{dowith.tex}[2015/11/14 documenting dowith.sty]
\title{%%%\kern-\baselineskip
       \textsf{\huge dowith.sty}\\---\\Apply Command to
%        Elements of Lists without Separators\,---\,%%% 2012/05/14
%        and without Iterator\thanks{This
       %% 2012/05/15, "in" 2012/05/18:
       Each Item \\ in a List of Arguments in ``\TeX's Gullet"\thanks{This
       document describes version
       \textcolor{blue}{\UseVersionOf{\jobname.sty}}
       of \textsf{\jobname.sty} as of \UseDateOf{\jobname.sty}.}}
{ \RequirePackage{makedoc} \ProcessLineMessage{}
  \renewcommand*\mdSectionLevelOne{\string\subsection}
  \renewcommand*\mdSectionLevelTwo{\string\subsubsection}
  \MakeJobDoc{18}%% 2012/11/05 for v0.3
  {\SectionLevelTwoParseInput}  }
% \RequirePackage[ir]{inputtrc}
\documentclass[fleqn]{article}%% TODO paper dimensions!?
\input{makedoc.cfg}    %% shared formatting settings
\ifpdf\else \errhelp{hyperref draft bad with {equation}}
            \errmessage{run this with pdflatex only}\fi
% \ReadPackageInfos{dowith}
\usepackage{dowith}                                         %% 2012/05/17b
\sloppy
\MDkeywords{programming structures;                         %% 2012/05/14b
            macro programming, loops, list macros}
\hypersetup{%
    pdftitle=dowith.sty handles lists without separators,
    pdfsubject=documenting dowith.sty
}%% 2011/08/22 
\usepackage{fixltx2e}                       %% \textsubscript 2012/05/17b
%% <- TODO with `lmodern'?
% \newcommand*{\ctanpkgdref}[1]{%                             %% 2012/11/19
%     \ctanpkgref{#1}\,\foothttpurlref{ctan.org/pkg/#1}}
%% <- in `makedoc.cfg' 2015/11/14
\makeatletter                                   %% etc. 2012/05/17b
%% TODO `actcodes'!? 2012/11/02
  \newcommand*{\GetOtherChar}[2]{% 
    \@ifdefinable#1{%
        \edef#1{\expandafter\@gobble\string#2}}}
\makeatother
\GetOtherChar\codeLB\{ 
\GetOtherChar\codeRB\} 
% \GetOtherChar\codeSP\ %       %% ???
{\MakeOther\ \xdef\codeSP{ }}
\newcommand*{\codelb}{\code\codeLB}
\newcommand*{\coderb}{\code\codeRB}
\newcommand*{\codesp}{\code\codeSP}
\newcommand*{\codebd}[1]{\code{\codeLB#1\codeRB}}       %% 2012/05/19
%% rm. 2012/05/17b:
% \DeclareRobustCommand*{\code}[1]{%
%   \texttt{%
%     \let\{\codeLB \let\}\codeRB \let\ \codeSP
%     #1}}
% \newcommand*{\chtok}[2]{\ensuremath{\code{#1}\sb{#2}}}
%% <- 2012/05/17b ->
\newcommand*{\chtok}[2]{\code{#1}\textsubscript{#2}} 
\newcommand*{\lbtok}{\chtok\codeLB{1}}
\newcommand*{\rbtok}{\chtok\codeRB{2}}
\newcommand*{\lttok}[1]{\chtok{#1}{11}}
\newcommand*{\sptok}{\chtok\codeSP{10}}
\providecommand*{\TTb}{\meta{The\nolinebreak[3] \TeX book}}
\providecommand*{\TTbp}{\TTb\nolinebreak[3] p}
\newenvironment*{smallpar}
                {\medskip\par\begingroup\footnotesize}
                {\par\endgroup\medskip}
\newcommand*{\NTOK}[1]{\textsf{ntok}(\code{#1})}
\newcommand*{\ntok}[1]{\textrm{?}\code{#1}}
\newenvironment*{example}[1]
                {\trivlist\item
                 (\ulroman{#1})}
                {\endtrivlist}
\providecommand*{\ulroman}[1]{\meta{\romannumeral #1 }}
\newcommand*{\inlineitem}[1]{\ (\ulroman{#1})\enspace
             \ignorespaces}
\newcommand*{\pdots}{~.\kern\fontdimen3\font
                      .\kern\fontdimen3\font. }
\providecommand*{\Chi}{\mathrm{X}}
\renewcommand*{\httpprefix}{\theHTTPprefix}
\begin{document}
\maketitle
\begin{MDabstract}
This package provides macros for applying a 
%% mod. 2012/05/15:
% ``command" <cmd> to all items of a list $<arg-1><arg-2>\dots<arg-$n$>$
``command" to all items in a ``list of possible macro arguments," %% v0.21a
% in ``\TeX's mouth," 
% such as `\DoWithAllIn{<cmd>}{<list-macro>}', 
and also for extending and reducing macros storing such lists.
``Brace groups" are single items of such lists, as opposed to 
token lists. Iteration is implemented within \TeX's expansion 
processor, so works within `\write' as with 'blog.sty'.
 %%% ---pleading for mathematical rigour in \TeX ology!)    %% rm. 2012/05/17b
% Applications in mind belonged to \LaTeX, but the package should work 
% with other formats as well.
Loop and list macros in other packages are discussed in the documentation.
% %% 2012/05/09:
% There is an emphasis on expandability 
% %% mod./add. 2012/05/15:
% (no \wikienref{Iterator}{iterator;} 
%  essential within \cs{write} as with \CtanPkgRef{morehype}{blog.sty}),
% without relying on \CtanPkgRef{e-tex}{$\varepsilon$-\TeX}.
%% 2012/05/18:
% Iteration is implemented within ``\TeX's mouth," so works within 
% `\write' as with \CtanPkgRef{morehype}{blog.sty}. 
There is no need for \CtanPkgRef{e-tex}{$\varepsilon$-\TeX} to which 
some of them refer.

The package is ``generic," i.e., should also work with Plain \TeX\ 
or even other formats, relying on the \ctanpkgref{plainpkg} package 
for some minimal \LaTeX-like behaviour.     %% \ 2012/11/27

\MDaddtoabstract{Related packages}
\let\pkg\ctanpkgref     \pkg{catoptions},
\pkg{etextools}, \pkg{etoolbox}, \pkg{forarray}, \pkg{forloop}, 
\pkg{loops}, \pkg{multido}, \pkg{moredefs}, \pkg{lmake}, 
\pkg{texapi}, \pkg{xfor}, \pkg{xspace}
\end{MDabstract}
\tableofcontents

%   \newpage
\section{Usage and Features}                    %% restructured 2012/05/16

\subsection{Installing and Calling}
The file 'dowith.sty' is provided ready, installation only requires
putting it somewhere where \TeX\ finds it
(which may need updating the filename data
 base).\urlfoot{ukfaqref}{inst-wlcf}           %% corr. 2011/02/08
The packages \ctanpkgdref{plainpkg}
and 'stacklet' (\ctanpkgref{catcodes})\urlfoot{ctanpkgref}{catcodes} 
must be installed as well.

As to calling (loading): 'dowith' is a ``\pkg{plainpkg} package" 
in the sense of the \ctanpkgref{plainpkg}
documentation that you may consult for details.
So roughly, 
\begin{itemize}
  \item load it by \ |\usepackage{dowith}| \ if you can, 
  \item otherwise by \ |\RequirePackage{dowith}| \\
        (perhaps from within another ``\pkg{plainpkg} package"), 
  \item or by \ |\input dowith.sty| 
  \item or even by \ |\input{dowith.sty}|~\dots
\end{itemize}

% \section{Example}

% \section{Discussion}                                      %% 2012/05/16
% \subsection{What It Seems to Do \dots}
% The 'dowith' package provides tools to simplify \TeX\ macro programming. 
% Understanding it \emph{really} may require understanding certain passages of 
% \TTb, such as pp.~38f. It may even require overcoming 
% terrible confusions in \TTb.
% Let this be a last try at understanding the package without understanding \TeX: 
% It allows you to abbreviate 
% \[<cmd><arg-1><cmd><arg-2>\dots<cmd><arg-$n$>\]
% by 
% \[`\DoWith<cmd><arg-1><arg-2>\dots<arg-$n$>\StopDoing'\]
% or by 
% \[`\DoWithAllOf<cmd>{<arg-1><arg-2>\dots<arg-$n$>}'\]
%
%   \pagebreak                                                %% 2012/05/19
\subsection{What It Does With What Lists}
% \subsection{What It Actually Does \dots}
\label{sec:lists-intuit}

The single commands that the package provides are described 
in the \hyperref[sec:implement]{implementation section} below. 
What follows here is some general background about how the 
commands work.
%% <- moved down from previous section, mod. 2012/05/19

The term \qtd{list} may refer to various things and need clarification here. 

First of all, we are not referring to \LaTeX\ `list' environments 
such as `enumerate' or `itemize'; 
neither to ``\acro{TODO}" lists of what needs to be done soon. 

Rather, 'dowith' allows you to abbreviate 
\[<cmd><arg-1><cmd><arg-2>\dots<cmd><arg-$n$>\]
by 
\[`\DoWith<cmd><arg-1><arg-2>'\dots<arg-$n$>`\StopDoing'\]
or by 
\[`\DoWithAllOf<cmd>{<arg-1><arg-2>'\dots<arg-$n$>`}'\]
With small $n$, one may doubt whether this really is an abbreviation~\dots;
anyway, 
\[<arg-1><arg-2>\dots<arg-$n$>\]
was an attempt to refer to the kind of lists we are dealing with. 
\[<arg-1>, <art-2>, \dots, <arg-$n$>\]
are the ``items" of the list. 
The question is: what counts as an item?

We might say that `aa' is a list of \emph{two} items, 
<arg-1> being `a' and <arg-2> being `a', too.

When we do \emph{three} keystrokes to get `a a' instead of `aa', 
we still have \emph{two} items, 
<arg-1> being `a' and <arg-2> being `a' too. 
Strange, isn't it?

Also, when in `aa' we replace the first `a' by a backslash, `\', 
we get `\a', and this is a list of a \emph{single} item, 
$<arg-1>=`\a'$~\dots

You shouldn't believe these stories of mine entirely. 
What I am alluding to is that the \emph{``items"} 'dowith' is about 
are determined in terms of \emph{\TeX's tokens}, and the relation 
between the ``characters you type" and \emph{\TeX's tokens} 
is not entirely straightforward. 

\subsection{The Notion of Arglists for \LaTeX\ Users}
%% <- 2012/05/17b ->
% \subsection{The Notion of Arglist}
\label{sec:arglists-intuit}
Still, it may suffice to clarify what counts as an <arg-$i$> 
without speaking of \emph{tokens} explicitly: It is simply 
what a \emph{one-parameter macro} 
(where the parameter is \emph{not delimited} in terms of 
 \TTbp p.~203f.)
can take as an \emph{argument.}

The lists 'dowith' is about then are lists \emph{of possible arguments}
in the previous sense---let me call them \emph{``arglists."}\footnote{Not 
    to be confused with German \httpref{de.wiktionary.org/wiki/Arglist}{Arglist.}}
The single \emph{items} of such lists are those single possible arguments. 
They become \emph{actual} arguments beginning from the leftmost \emph{possible} one 
when 'dowith' presents them to that <cmd> mentioned earlier---where 
<cmd> \emph{should} be a one-parameter macro (or some \TeX\ primitive 
parsing arguments similarly).

The reader perhaps has an \emph{intuitive} understanding of 
what can be an argument of a one-parameter macro. 
A \emph{strict \LaTeX} user may think that such an argument <arg-$i$>
just has form `{'<ark-$i$>`}', i.e., $<arg-$i$>=`{'<ark-$i$>`}'$
for some <ark-$i$>. Such arguments are also called \emph{``brace groups".}
(\LaTeX's \emph{optional} arguments `[<extra>]' 
 do not count as possible arguments here, they are not macro arguments 
 in the sense of \TTb.)
In this restricted \LaTeX\ sense, arglists consist of brace groups 
\[`{<ark-1>}{<ark-2>}'\dots`{'<ark-$n$>`}',\]
and each single brace group is an \emph{item} of it. 

The \emph{\TeX\ macro writer}, by contrast, knows that a macro argument 
doesn't need outer braces. In an intuitive sense, a single 
``command" can be a macro argument, too. ``Command" may be understood 
as ``control sequence" (starting with a backslash), 
but some authors also have considered single \emph{characters} 
(character \emph{tokens}?) ``commands." Blank spaces, by contrast, 
are ignored when a macro looks for its argument.    %% 2012/05/17b

%% useless, just name items 2012/05/17b:
% We arrive at an ``intuitive recursive definition" of ``arglist:" \
% (\meta{i})\enspace The empty list is an arglist; it doesn't have an item 
% for 'dowith'. \ (\meta{ii})\enspace If <list> is an arglist, then 
% (a)~`{<ark>}<list>' is an arglist whose first item for 'dowith' is `{<ark>}';
% (b)~`<cs><list>' with a ``control sequence" <cs>
%     is an arglist whose first item is `<cs>' 
%     (for the command <cmd> to which 'dowith' presents <cs>); and
% (c)~`<char><list>' with a non-blank character <char>
%     is an arglist whose first item is `<char>'. \
% (\meta{iii})\enspace Nothing else is an arglist.
% 
% But keep in mind that this ``intuitive" understanding essentially is wrong.

\subsection{Anatomy of \TeX}
The documentation of v0.22 as of 2012-06-04 said that the package 
is about ``lists in \TeX's mouth." However, this was very wrong. 
I believed it following Alan Jeffrey's paper ``Lists in \TeX's Mouth",\footnote{%
    Alan Jeffrey: \tugbartref{tb11-2/tb28jeffrey}{``Lists in \TeX's Mouth,"} 
    \acro{TUG}boat Vol.~11 (1990), No.~2, pp.~237--245), 
    \urlhttpref{tug.org/TUGboat/tb11-2/tb28jeffrey.pdf}.}
in whose Section~2 you read:
\begin{quote}\it
  \TeX's programming facilities come in two forms---there 
  are \TeX's \emph{macros} which are expanded in its
  mouth, and some additional \emph{assignment} operations
  like \cs{def} which take place in the stomach.
\end{quote}
The macros that Jeffrey lists and describes in that article can be 
obtained as a \acro{CTAN} package 
\ctanpkgref{lambda-lists}.\footnote{\urlhttpref{ctan.org/pkg/lambda-lists}} 
If you follow the link given here (in the footnote), 
you currently (2012-11-03) read about this package:
\begin{quote}\it
    These list-processing macros avoid the reassignments 
    employed in the macros shown in Appendix D of the TeXbook: 
    all the manipulations take place in what Knuth is pleased to 
    call ``TeX's mouth".
\end{quote}
But Knuth doesn't. On page 267 of \TTb, you read:
\begin{quote}\it
  Chapter 7 has described the process by which input files
  are converted to lists of tokens in \TeX's ``mouth,"
  and Chapter 20 explained how expandable tokens are converted
  to unexpandable ones in \TeX's ``gullet" by a process 
  similar to regurgitation.
\end{quote}
I.e., the ``mouth" is \TeX's ``tokenizer," the inner part of what 
van Eijkhout calls \TeX's ``input processor" on, e.g., p.~15 of his 
\meta{\TeX~by Topic}.\footnote{It is available as a \acro{CTAN} package 
    \ctanpkgref{texbytopic} at \urlhttpref{ctan.org/pkg/texbytopic}.}
The exact rules the tokenizer follows are described on pp.~46f.\ 
(Chapter~8!) of \TTb.
\emph{Macro expansion} takes place in \TeX's \emph{``gullet"}, 
which van Eijkhout calls \TeX's ``expansion processor" (p.~16). 
Abrahams, Hargreaves, and Berry follow Knuth's terminology on
pp.~16 and 46f.\ of their \meta{\TeX~for the Impatient}.\footnote{It 
    is available as \acro{CTAN} package \ctanpkgref{impatient}, 
    \urlhttpref{ctan.org/pkg/impatient}.} 
% Only Jeffreys and my wrong old documentation say this were 
% ``\TeX's mouth"---and I made the same mistake with the 
% \ctanpkgref{bitelist} package.
%% <- 2012/11/19 ->

\TeX's gullet has been called ``\TeX's mouth" also 
in the documentation of my \ctanpkgdref{bitelist} package and 
in the documentation of the package \ctanpkgdref{bibleref-mouth}.

Moreover, I should have clarified that Jeffrey's paper deals with 
``lists" in some general, rather abstract sense, different from 
the kind of lists the present documentation tries to characterize 
as the objects for 'dowith'.

\subsection{\TeX's Tokens}
\label{sec:toks}

% The \emph{\TeX\ macro writer} understanding \TeX\ properly 
% does not really think of arglists. \TeX perts instead think of 
% What \TeX nically matters is
% what happens in ``\TeX's mouth,"\footnote{Cf.~\TTbp.~46.} 
% as some authors have suggested a metaphor,\footnote{%
The 'dowith' package is a tool that affects the order of tokens in 
\TeX's gullet.

\begin{smallpar}
The ``characters you type" enter ``\TeX's mouth" line by line, 
in a slightly modified appearance. Each line forms a \emph{string}.
\TeX\ takes initial substrings away from it and turns them 
into \emph{tokens} that are appended to the right of \TeX's 
\emph{expansion buffer} (``gullet").

%% Removing 2012/11/04:
% More formally, \TeX\ has a \emph{character buffer.}
% It forms a single token from an initial segment of the buffer content---unless 
% there is a special situation with blank spaces or something pathological.
% When an \emph{escape character}, as the backslash usually is one, 
% has been noticed recently (that isn't followed by another one immediately), 
% the character buffer may need to be feeded from more outside, 
%% <- especially wrong, 2012/11/04
% until it contains enough material to form a token from. 

There are \emph{two kinds of tokens} here: \emph{named} tokens 
and \emph{character} tokens. ``Named" tokens usually are referred
to as ``control sequence tokens" or just ``control sequences"---I~really 
want to avoid those horrible confusions from \TTb.
There never are any ``parameter tokens" in \TeX's gullet 
(perhaps unless one considers a one-step macro expansion 
 a two-or-more-step procedure).
%  merged 2012/11/04
%% Moving down 2012/11/04
The character(s) \emph{after} the escape character until 
some delimiting character form a \emph{string} that is the 
\emph{name} of the token that is formed---a \emph{named} 
token, as I am saying.
% What has been used to form a token is removed from the character buffer.
%% 2012/11/04:
\emph{Character} tokens are formed by removing a character from the 
beginning of the character buffer and appending it to the token buffer 
paired with its \emph{category} code.

For every \emph{string of characters}, there is exactly one 
(possible) \emph{named token} whose name the string is.\footnote{``Possible" 
    refers to the fact that \TeX\ does not store named tokens anywhere 
    before they appear in its gullet, maybe apart from ``primitive" 
    tokens that have a ``pre-assigned meaning" when a \TeX\ run     %% \ 2012/11/04
    starts.---What is more bad with my claim is that 
    the \TeX\ program by design cannot extend its memory arbitrarily---even 
    not using the ``cloud''---, so it doesn't support tokens 
    whose name lengths are above a certain limit.}
It is so common (starting from \TTb) to denote the token whose name 
is <string> by \lq\verb+\+<string>\rq. For instance, the token whose 
name is `input' is denoted by \qtd{&\input}. On the other hand, 
on page~7 of \TTb\ \qtd{&\input} is a ``string of characters."
With this notation, it is already difficult to explain what 
the \LaTeX\ command `\DeclareRobustCommand' does or what the difference 
between a starred \LaTeX\ command and a starred \LaTeX\ environment is.\footnote{%
    A reader knowing \LaTeX\ only thinks that \qtd{\code{&\\\codesp}} 
    is the result of typing a double backslash and a space 
    and that \qtd{\cs{equation*}} is the ``command" \cs{equation} 
    followed by a `*'.} 
\TTb\ makes it worse by saying on page~39: 
\meta{``A control sequence is considered to be a single object 
        that is no longer composed of a sequence of symbols."}
So ``it depends" whether \qtd{&\input} is a string of characters or not---it \emph{is 
before} tokenization, but \emph{no longer} afterwards. 
So if you have two computers and start a \TeX\ run on each of them 
with a little difference in time, there will be a moment where 
\qtd{&\input} is a string on the one computer but not on the other? 
This 
appears to me %%% is %% 2012/11/04
like saying \meta{``When we apply the square root function to 
the number 4, the number 4 will no longer be the number 4, 
it will be the number 2 instead."}
\end{smallpar}

\TTb\ does offer an alternative notation for named tokens: ``boxing;" 
so the token whose name is `input' can be denoted by the rather 
``graphical" notation \qtd{|input|} (used only exceptionally).\footnote{%
    The box notation is introduced on page~38 without explanation, 
    as if it explained something.} 
\emph{I}~would suggest something like 
\qtd{\NTOK{input}} for clarity and \qtd{\ntok{input}} 
for brevity.\footnote{I am suggesting the question mark for named tokens 
 since \TeX\ ``must look up the current definition" of a named token 
 according to \TTbp.~39, while the meaning of character tokens rather 
 is ``fixed," at least according to \TTbp.~39. However, 
    \emph{active}-character such as .&~ are in the same situation 
    as named tokens as to this respect. The dot notation may be fine for them, 
    though.}

\begin{smallpar}
%% rm. 2012/11/04, cf. above
% \emph{Character tokens} get into \TeX's mouth by tokenization 
% when characters begin the buffer content while \emph{not} scanning 
% a name for a named token. A single character then is removed from the 
% character buffer, and a token storing its character code and current 
% category code is pushed into \TeX's mouth.

Named tokens may get into \TeX's gullet by ``tokenization" as described above, 
i.e., they are drawn from the character buffer. But they also can 
appear in \TeX's gullet ``from within," 
by the manipulation inside \TeX's gullet. 

More formally, those manipulations are called ``expansion,"
and \TeX's gullet can be conceived of as a \emph{token buffer} 
that is feeded to the right (or end) by tokenization from the character buffer. 
Expansion means that certain tokens in the token buffer are substituted 
by other ones. This way tokens may get into \TeX's gullet that 
emerged from tokenization a ``long time ago", maybe in a previous 
run that created the \emph{format} (\TeX's variant \code{INITEX}); 
or tokens may appear by some hardwired expansion function. 

However, \emph{named tokens} may get into \TeX's gullet 
also by \emph{expansion}, never having been drawn by tokenization 
and not being hardwired. This happens by the `\csname' 
% name      %%  rm. 2012/07/25
construct. 
The input \emph{code} may contain 
\[`\csname tupni\endcsname'\]
This may be converted into 7 tokens entering \TeX's gullet, 
the first one being \NTOK{csname}, the last one 
\NTOK{endcsname}, and five character tokens in between. 
Due to some \emph{function} (which I would denote as *\code{csname})
originally associated with the token \NTOK{csname}, 
those seven tokens then are replaced by \NTOK{tupni}, 
the named token whose name is `tupni'. It is not required that 
the \TeX\ program knows about a token \NTOK{tupni}, 
neither anybody must type \qtd{&\tupni} in any file.\footnote{These 
    considerations may not be essential here, 
    rather a draft for a paper. Using 'dowith', 
    one better just thinks of the arglist items one actually lists.}
\end{smallpar}

\subsection{Arglists vs.\ Lists of Tokens---Example}
% \subsection{Arglists and 'dowith' \TeX nically}
% In the \TeX nical sense, I think of arglists \emph{and their items}
% as follows. Arglists are lists (or sequences) of tokens. 
% What is somewhat difficult is that the \emph{items} of a token list 
% usually are \emph{tokens.} Especially, the curly braces in the code 
% you type usually are converted into certain \emph{character tokens}
% that are single items of the resulting token list.\footnote{I discover 
%     this conceptual puzzle 2012-05-16.} 
% 
% The conceptual trap here may have resulted from denoting lists 
% in a \emph{\Wikienref{juxtaposition}} notation. 
% In Section~\ref{sec:lists-intuit}, I have written `aa' for a 
% ``list" of \emph{two} items. ``List" is rather a \emph{computer science} 
% term, its mathematical counterpart rather is the notion of (finite) 
% \emph{sequences.} The usual \emph{mathematical} notation for a finite sequence 
% writes list items surrounded by \emph{\wikienref{bracket}{brackets}}
% (round---\emph{\Wikienref{parentheses}}---seems to be more common than square, 
%  also \wikienref{angle brackets}{chevrons} are used).
% 
% So the \emph{string} `aa' can be written more clearly---mathematically---as 
% $(`a',`a')$. The trap with `\a' is that it could be \emph{either} the \emph{string} 
% $(`\',`a')$ \emph{or} the one-item list $(`\a')$ of strings. 
% But even in the latter case, I urge not to consider it a \emph{\TeX\ token}. 
% Rather, I consider `\a' a mistaken way of referring to the 
% named token \NTOK{a} whose name is the string `a' (or $(`a')$).
% (Section~\ref{sec:toks}). 
% 
% However, the conceptual trap about arglists and token lists 
% (there must have been some ``Arglist"!\@) lurks on another level, 
% as follows. Recall \TTb's notation of \chtok{<char>}{<cat>}
% for the \emph{character token} that \TeX's tokenizer forms 
% from <char> in the character buffer when <char>'s category code is <cat>.
% Usually, the \emph{character} `a' is converted into the 
% \emph{character token} \lttok{a}, `{' is converted into \lbtok, 
% and `}' is converted into \rbtok. 
% 
% We are turning to some \strong{examples} and \strong{counterexamples}. 
% Let us see what confusions occurred in the ``intuitive" view on 
% arglists in Section~\ref{sec:arglists-intuit}.
% 
% % \begin{example}{1} It 
% First, it was bad in Section~\ref{sec:arglists-intuit}
% to think that `aa' is a two-item arglist. It was confused with something 
% like $\lttok{a}\lttok{a}$. The latter looks like a token list---or is it an arglist? 
% Both? Anyway, it is juxtaposition notation applied to tokens, 
% mathematically it is $(\lttok{a},\lttok{a})$, so~\dots
% % \end{example}
% 
% Now let us reconsider the ``intuitive recursive definition" of arglist. 
% Or let us look at a recursive definition of \emph{token} list.
% \inlineitem{1} There is nothing wrong with saying the the empty list is 
% a token list, the same holds for arglists. 
% \inlineitem{2} a.~When we attach an arbitrary token to a \emph{token} list 
% (at the left), the result is another token list---fine. 
% There are only certain difficulties with ``handling" special token lists 
% such as $\lbtok\lttok{a} = (\lbtok,\lttok{a})$. 
% And the latter is \emph{not} an \emph{arglist!} \
% b.~When we attach a \emph{named} token to an arglist (at the left), 
% the result \emph{is} an \emph{arglist}.\footnote{This is 
%     a \emph{conjecture} only right now---2012-05-16---in which I strongly believe. 
%     Likewise later.} 
% c.~When we attach a \emph{letter} token \lttok{<char>} to an arglist 
% (at the left), the result \emph{is} an \emph{arglist}.
% d.~When we talked about ``brace groups," they seemed to be \emph{strings} 
% of characters. Instead, I would like to suggest that a brace group 
% is an \emph{arglist}\pdots
% When \lbtok\ is attached to the left of an arglist and \rbtok\ to the right, 
% the result is an arglist---this is what I would call a ``brace group"!
% e.~The \Wikienref{concatenation} of two arglists is an arglist. 
% 
% \begin{smallpar}
% The above notion of ``attaching" a token to a token list or an arglist 
% % should be clarified, but instead of a general definition in terms of 
% % ``words" of formal languages, examples may suffice here and now.\footnote{2012-05-16.}
% should be clarified. Attaching an item $j$ at the left of a list $\lambda$ 
% is the same as concatenating the one-item list $(j)$ with $\lambda$\pdots 
% but the English \wikienref{append}{Wikipedia} seems to explain concatenation by 
% \emph{appending}. As we \emph{remove} items one-by-one from the \emph{left} 
% (beginning) of a list, I prefer the inverse \emph{prepending} items as basic operation 
% for building lists---cf.\ \Wikienref{CAR and CDR}.
% \end{smallpar}
% 
% We won't complete a formal (recursive) definition of arglist here and 
% now.\footnote{2012-05-16.} Just observe that ``brace groups" 
% make the difference between token lists and arglists. 
% Recall that an ``item" of an arglist ``operationally" is defined 
% as something that a one-parameter macro removes. 
% Such a macro removes certain single tokens 
% (space tokens not among them)---and entire ``brace groups"! 
% Especially, consider \[\lbtok\rbtok\]%%%.
% Actually, this is another ambiguous notation. 
% If it refers to $(\lbtok,\rbtok)$, it is a \emph{token} list, 
% not an arglist. If it refers to $(\lbtok\rbtok)$---it should 
% better refer to $((\lbtok,\rbtok))$, which is a 
% \emph{one-item arglist} whose only item is the former two-item token list!
% 
%% <- 2012/05/17b ->
Let us reconsider the examples from Sections~\ref{sec:lists-intuit}
and~\ref{sec:arglists-intuit}, and pack them into a single example. 
If you type a file line
\begin{equation}
  `a a\a{a}'
\end{equation}
(\emph{eight} keystrokes),
it should usually be converted into this \emph{seven}-item list 
of 
%%% (five) %% rm. 2015/11/14 -- ???
tokens:
\begin{equation}
  \label{eq:toks}
  \lttok{a}\;\sptok\;\lttok{a}\;\NTOK{a}\;\lbtok\;\lttok{a}\;\rbtok
\end{equation}
---with notation from Section~\ref{sec:toks} and
\TTb's notation \chtok{<char>}{<cat>} 
for the \emph{character token} that \TeX's tokenizer forms 
from <char> in the character buffer when <char>'s category code is <cat>.

It turns out that the token list in \ref{eq:toks} 
provides an arglist of \emph{four} items: The token \lttok{a}
at the first and third place, the named token \NTOK{a}, and the entire 
token list $\lbtok\lttok{a}\rbtok$ as a single item---a ``brace group."
The space token is ignored.\footnote{\TTbp.~201: ``\TeX\ doesn't use 
single spaces as undelimited arguments."}

You can try this after `\renewcommand{\a}{A}'\footnote{Otherwise 
    \cs{a} is a one-parameter macro that breaks 'dowith''s control.}
with 'dowith':
\begin{equation}
  \label{eq:sample-code}
  `\DoWith\typein a a\a{a}\StopDoing'
\end{equation}
Then \LaTeX\ shows `a', `a', `A' from `\a', and another `a' from within 
the braces---`\typein' (as any macro with arguments) removes them.

% \show\a
% \a{a}
\renewcommand*{\a}{A}
% {\MakeNormalHere\# \newcommand*{\TypeOut}[1]{\typein{#1}}
%  \global\let\TypeOut\TypeOut}
% \let\TypeOut\typein
% \let\TypeOut\typeout
% \DoWith\TypeOut a a\a{a}\StopDoing
% \DoWith\typein a a\a{a}\StopDoing
% { %%% \tracingmacros=1 \tracingonline=1
%  \MakeNormalHere\# \AssignCatCodeTo{2}\]
%  \DoWith\typein #a\a{a]\StopDoing}

I have avoided saying \ref{eq:toks} \emph{were} an arglist of 4 items. 
The mathematical basic way of writing lists---understood as finite
\emph{\wikienref{sequence}{sequences}}---as ``commma-separated lists" 
within \wikienref{bracket}{brackets} may clarify the difference 
(that the \Wikienref{juxtaposition} notation tends to conceal). 
The \emph{token} list is 
\begin{equation}
  \label{eq:toks,}
  (\lttok{a}, \sptok, \lttok{a}, \NTOK{a}, \lbtok, \lttok{a}, \rbtok)
\end{equation}
while the list of macro arguments is
\begin{equation}
  \label{eq:args}
  (\lttok{a}, \lttok{a}, \NTOK{a}, (\lbtok, \lttok{a}, \rbtok)).
\end{equation}
\ref{eq:toks} or \ref{eq:toks,} simply is \emph{not} an arglist 
(since neither \lbtok\ nor \rbtok\ can be a macro argument), 
and the arglist \ref{eq:args} ``provided" by the list of tokens 
is \emph{not} a list of \emph{tokens}---its final item is a 
three-item list of tokens, and a token cannot be a list of 
two or more tokens itself(\emph{!?}).

\subsection{Another Notation and the Example's Steps}
\label{sec:steps}                                           %% 2012/05/20
\begin{smallpar}
To write token lists easier and hopefully easier to read, 
I would suggest writing \qtd{.<char>} for the character token 
that the tokenizer ``usually" forms from character <char>, i.e., 
adding the \emph{standard} category code as in \TTb\ (page~37).
Then \ref{eq:toks} would read\footnote{See Section~\ref{sec:toks} for the question mark.}
\begin{equation}
  .&a\,.\codesp\,.&a\,\ntok{a}\,.\codelb\,.&a\,.\coderb
\end{equation}
and the corresponding arglist is
\begin{equation}
  (.&a,.&a,\ntok{a},(.\codelb\,.&a\,.\coderb))
\end{equation}
In ``retrospect," the result of tokenizing \ref{eq:toks} should be
\begin{equation}
  \label{eq:retro}
  \ntok{DoWith}\,\ntok{typein}\,
  .&a\,.\codesp\,.&a\,\ntok{a}\,.\codelb\,.&a\,.\coderb\,
  \ntok{StopDoing}
\end{equation}
and the intention is that it works like 
\begin{equation}
  %% added \,s 2012/06/07:
  \ntok{typein}\,.\codelb\,.&a\,.\coderb\,
  \ntok{typein}\,.\codelb\,.&a\,.\coderb\,
  \ntok{typein}\,.\codelb \ntok{a}\,.\coderb\,
  \ntok{typein}\,.\codelb\,.&a\,.\coderb
\end{equation}
(The definition of `\DoWith' in Section~\ref{sec:core} indeed adds surrounding braces, 
 if missing.)
However, \TeX\ rather tries to work with as few tokens ahead as possible.
When it finds \ntok{DoWith} and the latter's meaning is the one intended 
by 'dowith', it first looks for nothing more than the two arguments 
required by our definition of `\DoWith'. A few moments later, 
the token buffer's content will just be\footnote{If you use 
    `&\DoWithAllOf&\typein\codeLB a\codeSP a&\a\codeLB a\codeRB\codeRB'
    %% <- \cs bad 2015/11/14
    instead, the entire token sequence \ref{eq:retro} will appear in the 
    token buffer ``at once."}
\begin{equation}
  \ntok{typein}.\codelb.&a.\coderb\,\ntok{expandafter}\,\ntok{DoWith}\,
  \ntok{expandafter}\,\ntok{typein}\,\ntok{fi}
\end{equation}
Next $\ntok{typein}.\codelb.&a.\coderb$ is expanded according to the code for 
`\typein' in \file{latex.ltx}. Some unexpandable tokens will emerge 
and be moved into the ``instruction buffer,"                %% cmd -> instr 2012/11/04
and you should get a screen
message with `a' and a prompt. When you have entered something, 
the remaining \ntok{expandafter} tokens and the \ntok{fi} will be 
removed from the character buffer, and it contains only
\begin{equation}
  \ntok{DoWith}\,\ntok{typein}
\end{equation}
Another token is ordered from the tokenizer to provide a second 
argument for expanding \ntok{DoWith}. The token .\codesp\ comes in, 
but that doesn't serve as a macro argument. It is removed, and the 
next token is .&a. The same story as before happens, until the 
named token \ntok{a} is found\pdots
\end{smallpar}

\subsection{Summary of Possible Arglist Items}
\begin{smallpar}
For $0\leq i\leq 15$, let $\Chi_i$ be the set of character tokens 
of category code $i$. $\Chi_1$ is the set of tokens working like 
$\lbtok$, and $\Chi_2$ is the set of tokens working like $\rbtok$.

Let $E$ be the set $\{3,4,6,7,8,11,12,13\}$. 
These numbers are the category codes for 
\meta{math},   \meta{align}, \meta{parameter}, \meta{super}, \meta{sub}, 
\meta{letter}, \meta{other}, \meta{active} respectively. 
Let $\Chi_E$ be the set of character tokens of category code in $E$
(so $\Chi_E=\bigcup_{i\in E}\Chi_i$).

Let $\circ$ be the \emph{concatenation} operation among token lists.\footnote{%
    %% 2012/05/18
    \TODO: Define for representations by maps, or:              %% \ 2012/11/05
    ``Concatenation is about as basic as natural numbers and is 
      understood in terms of axioms rather than by a definition.''---See 
        notes from 2011 (even with attempts with \Wikienref{Category theory})
%         the \wikienref{Sequence}{English} 
        the English Wikipedia for 
        \wikienref{Sequence}{sequences}---\wikideref{Folge (Mathematik)}{German} 
        article too much restricted to maps.}

The following kinds of token lists form a single arglist item, 
i.e., can serve as an argument for an undelimited parameter:
\begin{enumerate}
  \item a \emph{named} token, or the single-token list consisting of it, 
        if you prefer that;
  \item a \emph{character} token from $\Chi_E$ or the list consisting of it; 
  \item a \emph{brace group.} 
        That is a token list meeting the following conditions:
        \inlineitem{1} its \emph{first} token is in $\Chi_1$,
        \inlineitem{2} its \emph{last} token is in $\Chi_2$,
        \inlineitem{3} it has as many occurrences of tokens from $\Chi_1$ as from $\Chi_2$,
        \inlineitem{4} if it is split as $\lambda\circ\rho$, there are not 
            more $\Chi_2$ occurrences in $\lambda$ than $\Chi_1$ occurrences in $\rho$
            (``don't close before opening").
\end{enumerate}
The second claim can be checked with
\begin{equation}
  `\DoWith\typein$#^_a1~\StopDoing'
\end{equation}
%  \begingroup
%   \def\a{A}\MakeNormalHere\# 
%   \DoWith\typein$# ^_a1~\StopDoing
%   \DoWith\typein#1\StopDoing
%  \endgroup
as to what works. 
(The claim is not affected by one or two surprises.)\footnote{Moreover, 
    `&\DoWith&\typein&#1&\StopDoing'              %% \cs bad 2015/11/14
    tells something about ``parameter tokens."} 
Characters with different category codes 
either are not converted into a character token\footnote{\TTbp.~47.} 
or are not accepted as macro arguments. The latter applies to ``brace" tokens 
in $\Chi_1$, $\Chi_2$ and to the single space token \sptok.

As to \emph{``brace groups"}, the third and fourth condition above 
are intended to say that what is between the two outer tokens 
is $\langle$balanced text$\rangle$ in the sense of \TTbp p.~275f. and~385;
i.e., for two tokens $a$, $b$ and a token list $\beta$, 
$(a)\circ\beta\circ(b)$ is a brace group exactly if $a$ is from $\Chi_1$,
$b$ is from $\Chi_2$, and $\beta$ is $\langle$balanced text$\rangle$. 
The conditions are more formal than what I can find in \TTb, 
but still they don't give me an idea of all possibities. 
This should be improved by the following recursive definition:

\begin{trivlist}\item 
B1.~The empty list is balanced text. 
B2.~For any token $t$ not in $\Chi_1$ or $\Chi_2$, 
    the single-item token list $(t)$ is balanced text. 
    (Such a token is either a \emph{named} token or a 
     \emph{character} token from $\Chi_E$ or \emph{the space token}~\sptok.)
B3.~If $\alpha$ and $\beta$ are balanced texts, 
    then $\alpha\circ\beta$ is balanced text.
B4.~If $\beta$ is balanced text, 
    $a$ is from $\Chi_1$, and $b$ is from $\Chi_2$, 
    then $(a)\circ\beta\circ(b)$ is balanced text.
    (This is a brace group, and the only way of getting a brace group.)
B5.~Nothing else is balanced text.
\end{trivlist}

In other words, a token list is a brace group if and only if 
it is balanced text and starts with a token from $\Chi_1$
and ends with a token from $\Chi_2$.\footnote{Again, this may be  more 
    of a draft for a paper, or notes for it, than package documentation.}
\end{smallpar}

\subsection{Summary: ``Commands" Usable with 'dowith'}
In the \hyperref[sec:apply]{implementation section,} 
you learn about 
\[`\DoWith<cmd>',\quad `\DoWithAllOf<cmd>',\quad 
    \mbox{and}\quad  `\DoWithAllIn<cmd>'.\] 
(\LaTeX\ users may type `{<cmd>}' instead.)
What <cmd>s are allowed?

\begin{enumerate}
  \item All \strong{one-parameter macros} <cmd> work this way, 
        unless there are programming mistakes outside 'dowith' 
        (also thinking of arguments that take over control 
         from 'dowith' commands before the argument list is finished). 
  \item \strong{Other one-parameter} ``commands" <cmd> such as
        \TeX\ \strong{primitives} may work---you must think of 
        the fact that surrounding \emph{braces} are added.\footnote{\TODO: %% \ 2012/11/05
            in the future, variants not adding braces could be added.}
        So the \strong{primitives} `\hbox' and `\vbox' work, 
        for instance. `\show' is an example that doesn't work at all, 
        it takes the single starting brace token and then confuses 
        `\DoWith'.
  \item Some <cmd>s taking \strong{no argument} may make sense, e.g., 
        for getting 
          \begin{enumerate}
            \DoWithAllOf{\item}{{apples,}{pears,}{peaches}}
          \end{enumerate}
        from 
        \begin{quote}
          &\begin{enumerate}\\
          \null\code{~~}&\DoWithAllOf{&\item}{%
            \codebd{apples,}\codebd{pears,}\codebd{peaches}}\\
          &\end{enumerate}
        \end{quote}
        Recall that `\item' at most takes an \emph{optional} argument.
  \item <cmd> must \strong{not take more than one} parameter. 
        %% see def. \DoWith
        A different package will support multi-parameter macros.
\end{enumerate}

% Rather, ``list" is a term from \Wikienref{computer science} here. 
% It corresponds to the notion of ``sequence" in mathematics 
% and to ``word" with formal languages.
% 
% Especially, we have (\meta{i})~lists of \emph{characters} that 
% the \emph{\TeX} program recieves from files and 
% (\meta{ii})~lists of so-called \emph{tokens} 
% that the \TeX\ program forms from the incoming list of characters 
% and that it works on.
% 
% There are \emph{more} kinds of lists that \TeX\ works on. 
% Here we are dealing with ``\TeX's gullet."\footnote{Cf.\ documentation 
%     of the \ctanpkgref{bitelist} package.}
% \TeX's mouth processes tokens formed from character input. 
% It may turn a list `<toks-a><toks_b><toks-c>' of tokens 
% into a list `<toks-a><toks-B><toks-c>', i.e., 
% replace <toks-b> by <toks-B>---by so-called ``expansion". 
% The result may be subject to expansion as well. 
% When nothing is left to be expanded, results are passed to another, 
% more interior subprocessor of \TeX. This one is deeper than \TeX's mouth, 
% we are not concerned with that here.
% 
% Not \emph{all} of those tokens in ``\TeX's mouth" are formed 
% (``directly") from input characters. 
% Rather, some \cs{csname} function may form \emph{new} tokens 
% from other tokens in \TeX's mouth---and place them there again. 
% This way characters `\csname a\endcsname' you type may just work like 
% `\a'.
% 
% It is difficult to tell on the \emph{character level} what the 
% present package does. The relation between incoming characters 
% and resulting tokens can hardly be explained by a single sentence or so.
% However, <cmd> here refers to a command <macro> with a single 
% undelimited macro. More precisely, a certain character sequence 
% (``string") in the code you type, will be converted into a TODO
% 
% \subsection{Separators}
% TODO
% 
% \subsection{Iterators and \TeX's Mouth}
% TODO
% 
\section{Similar Commands in other Packages}
%% <- section again 2012/05/17b
\subsection{``Heavy" Packages}                      %% add. heading 2012/05/17b
The \CtanPkgRef{etex}{$\varepsilon$-TeX}-related    %% e->... 2012/05/09
packages \ctanpkgref{etextools} 
(Florent Chervet), \ctanpkgref{etoolbox} (Philipp Lehman), 
and \ctanpkgref{texapi} (Paul Isambert) seem to include and 
(very much) extend the functionality of 'dowith'. 
Also the `\ForEach'\texttt{\dots\unkern\@} macros of            %% \@ 2012/06/03
\ctanpkgref{forarray} (Christian Schr\"oppel) seem to extend 
the present `\DoWith'\texttt{\dots\unkern\@} commands.          %% \@ 2012/06/03
Moreover, Ahmed Musa describes such commands as                 %% 2012/06/03
``Parsing \qtd{tsv} lists" in documenting his 
\ctanpkgref{catoptions} package.
\ctanpkgref{moredefs} (Matt Swift) provides list handling commands 
like the few that are here.\footnote{\ctanpkgref{arrayjobx}%    %% % 2012/11/05
    provides somewhat ``exotic" handling of ``lists".}          %% 2012/05/10
%% 2012/11/05:
---In October 2012, Ahmed Musa's \ctanpkgref{loops} appeared on 
\acro{CTAN}, offering loops of several ``categories" about as those 
that are listed below, very elaborate.% 
%% /2012/11/05
% (I~do not want to load that much.)
%% <- 2012/11/04 ->
---I do not want to load that much. I need and only need something 
excessively simple, very few lines of code, as presented in 
Section~\ref{sec:implement}. The next sections somewhat point 
out single features of loop constructs that I do not want to have.

\subsection{Separators}                             %% add. heading 2012/05/17b
Regarding \LaTeX\ macros in `latex.ltx', the basic macro `\DoWith' of 
the present package resembles |\@tfor| very much, which likewise deals 
with lists without separators. By contrast, \LaTeX's |\@for| deals with 
\emph{comma-separated} lists (such as lists of package options). 
With comma-separated lists, a ``string" of characters counts as 
an item when it is delimited by commas, or by a comma and the 
list ``border," or spaces may be used as separators additionally. 
However, when \LaTeX\ analyzes such lists (in ``\TeX's gullet"), 
it uses representations by \emph{character tokens} of them.

%% moved here 2012/05/17b:
% %% 2011/11/10:
% Also Heiko Oberdiek's \ctanpkgref{zref} deals with ``lists" of 
% ``properties" of ``entities," 
%% 2011/11/11: comma separated!

%% 2012/05/09f.:
The more recent \ctanpkgref{lmake} 
(Shengjun Pan)                                      %% 2012/05/18
provides a key-value syntax for printing lists of 
complex mathematical expressions easily (using some assignments) 
as well as defining commands according to a pattern from a list.
Those lists are comma-separated.

\subsection{``For" Loops vs.\ ``Foreach" Loops}     %% mv. up 2012/05/20
What about \ctanpkgref{forloop} (Nick Setzer), 
\ctanpkgref{multido} (Timothy Van Zandt, Rolf Nie\-praksch, Herbert 
Vo\ss), and \ctanpkgref{xfor} (Nicola Talbot)? 

'xfor' is just a reimplementation of `\@for'. 
'forloop' and 'multido' are more close to ``real \qtd{for} loops"
(cf.\ \wikienref{for loop}{\meta{Wikipedia}}).
Loops of the latter kind go through a certain set as well, 
but such sets rather consist of \emph{numbers} and are exhausted 
by incrementing (or also decrementing) variables 
\wikienref{Loop counter}{(counters).}
This is essentially not needed 
(neither helpful)                                       %% 2012/05/20
when a list literally is 
\emph{enumerated}---such loops are distinguished as 
\wikienref{Foreach loop}{``foreach loops."}             %% 2012/05/19

\subsection{Iterators}                              %% add. heading 2012/05/17b
So `\DoWith' and |\@tfor| rather provide ``foreach" loops. 
A major difference between them is that the latter uses a 
\wikienref{For loop}{``loop variable"} or 
%%% rather                                              %% rm. 2012/05/20
\wikienref{Iterator}{``iterator"} to which the elements 
of the list are assigned. 
`\DoWith<cmd>' does not use such a loop variable 
or such assignments and thus is ``expandable" at least when <cmd> 
(and the elements, depending on <cmd>) are expandable.
On the other hand, `\@tfor' applies some procedure to the list 
elements without needing a \emph{name} for the procedure 
(or a \emph{macro} storing the procedure). 
%
I wondered whether behind \LaTeX's 
`\@tfor' (and `\@for') there was an ``ideological" consideration 
such as ``A loop must have a loop variable!"\pdots
% However, avoiding usage of a macro name 
% (to store the ``loop body" code)                        %% 2012/05/18
% and a macro parameter 
% (to incorporate the list item into the body code)       %% 2012/05/18
% may have been a good reason.
%% <- 2012/05/19

%% 2012/05/20:
Hopefully more clearly on ``loop variable" vs.\ our approach: 
In order to run
\[<code-before><item><code-after>\]
on each <item> of a <list>, \emph{we here}
% \[
\begin{equation}
  \mbox{define} \quad `\do' \quad \mbox{as} \quad 
  `#1'\;\to\;`<code-before>#1<code-after>'
\end{equation}
% \]
and then run `\do{<item>}' for each <item> in <list>,\footnote{Cf.~description 
    of procedure in terms of tokens in Section~\ref{sec:steps}.} 
\begin{equation}
  \mbox{always replacing}\quad  `\do{<item>}'\quad \mbox{by}\quad  `\do{<item>}\do'. 
\end{equation}
(`\do' is only an example command that 'dowith' supports especially.)
% The ``received" approach from ``usual" programming languages, 
% \file{latex.ltx}, and its followers is
In \file{latex.ltx} instead, we find things like
% \[
\begin{equation}
  \label{eq:tfor}
  `\@tfor\@tmp:=<list>\do{<code-before>\@tmp<code-after>}'
\end{equation}
% \]
where `\@tmp' is a \emph{macro} that is set to be <item> at each 
iteration of the loop, by 
\begin{equation}
  `\def\@tmp{<item>}'
\end{equation}
within `\@tforloop'. After that,
\begin{equation}
  \label{eq:code-tmp-code}
  `<code-before>\@tmp<code-after>'
\end{equation}
from \ref{eq:tfor} is run.---\ref{eq:code-tmp-code} like 
\ref{eq:tfor} is stored in a larger macro. `\do' in \ref{eq:tfor}
does not act as a macro, it just delimits a macro parameter 
in order give a feeling of some familiar programming structure. 
This organisation of macros is fine when the loop body code 
is only used by the containing macro, while the 'dowith' approach 
to store the ``loop body" in an own macro has been useful when 
the loop body code also is used for different purposes 
or when it has been introduced before I thought of using it 
in a loop.
% \[\mbox{defining}\;`\@tmp'\;\mbox{as}\;<item>\quad \mbox{and run} \quad 
%   `<code-before>\@tmp<code-after>'\]
% for each <item> in <list>.

Note that this only was an example. In general, <item> may appear 
more than once in the ``loop body."

``Expandability" by \emph{avoiding} something iterating `\def\@tmp{<item>}'
and doing iteration in \TeX's gullet (`\do' or so must have been defined earlier)
is essential especially within \cs{write}. 
Assignments do not work there. A major motivation for developing 
'dowith' developed with the \ctanpkgref{blog} package 
that \cs{write}s \acro{HTML} code.
Assignments happen in ``\TeX's stomach."                    %% 2012/11/04
That place might be called the 
``instruction buffer" to which                              %% cmd -> instr 2012/11/04
the ``expansion processor" moves items from the incoming token buffer 
that cannot be expanded (any more).

\subsection{Separator Macros}                       %% add. heading 2012/05/17b
Commands like `\DoWith' also could save tokens thinking of list macros 
(in \LaTeX/`latex.ltx') that use a \emph{separator macro} 
which may be used as a \emph{command} to be applied to the list 
elements. One example is 
`\dospecials' that already is in Plain \TeX\ and expands to 
\[`\do\ \do\\\do\{\do\}\do\$\do\&\do\#\do\^\do\_\do\%\do\~'\]
% \begin{verbatim}
% `\do\ \do\\\do\{\do\}\do\$\do\&\do\#\do\^\do\_\do\%\do\~'
% \end{verbatim}
An important application of `\dospecials' is temporarily 
switching off the ``special" functionality of the ``elements" 
in `\dospecials'. With \LaTeX, this may happen thus:
\[`\let\do\@makeother\dospecials'\]
% \begin{verbatim}
% `\let\do\@makeother
% \end{verbatim}
With 'dowith', you can do the same with a shorter variant 
`\specials' of `\dospecials', defined by 
\[`\def\specials{\ \\\{\}\$\&\#\^\_\%\~}'\]
and then 
\[`\DoWithAllIn\@makeother\specials'\]
`latex.ltx' uses `\@elt' instead of `\do' for its own list macros.

\subsection{Ye Olde \cs{loop}}                      %% add. heading 2012/05/17b
%% added 2011/11/03:
There also is |\loop<loop-body>\repeat| in Plain \TeX\ and a 
refined\footnote{Using Kabelschacht's suggestion, cf.\ Section~\ref{sec:kabel}} 
version of it in `latex.ltx'. It is \emph{not} expandable 
since it starts with an assignment for `\body' (Plain \TeX) or 
`\iterate' (`latex.ltx'), 
%% add. 2012/05/20:
and then some assignments are needed to stop the loop, 
such as incrementing or decrementing a \emph{counter.} 
As to the programming structure, it is very simple and general, 
I think any kind of loop can be implemented by this
(apart from nested loops). E.g., I realize\footnote{2012-05-20}
that even a ``foreach" loop could be implemented by managing 
a list macro, e.g., using \LaTeX's internal `\@next'.
%% rm. 2012/05/20:
% that you cannot probably
% what kind of loops it addresses. However, the applications 
% I have seen have been ``for" or (rather) ``while" loops. 
% ``While" loops can ``emulate" ``for" and ``foreach" loops 
% by having the ``incrementation" method or the ``enumeration" method in 
% their body. This is quite obvious for ``for" loops, not quite so for 
% ``foreach" loops; which for practical application (in my view) means 
% that neither \LaTeX/\TeX's `\loop' macro nor in general ``while" loops 
% is/are very helpful for implementing ``foreach" loops, 
% as rather `\DoWith' and similar constructions are.
% The reason for this is (as it seems to me) is that you 
% (a human being) can much more easily enumerate (``list") 
% the items of a list (you have in mind) than define the \emph{method} 
% that (allegedly) is behind your enumeration. \ \meta{Example:} 
% \[`\DoWithAllOf{\printsamplearea}{\red\green\blue}'\]
% ---\emph{how} (according to what ``method"?) did you ``proceed" from 
% `\red' to `\green' and from `\green' to `\blue'?

\subsection{Without Iterator and Separators}        %% add. heading 2012/05/17b
%% 2011/11/07:
In \LaTeX's \ctanpkgref{tools} bundle, \ctanpkgref{xspace}
%% add. 2012/05/18:
was developed in the nineties by David Carlisle. 
It had a rather fixed exception list implemented by a deeply
nested conditional. In 2004 Morton H\o gholm joined, 
and now 'xspace' has a list macro 
`\@xspace@exceptions@tlp'                           %% was xpspace 2012/05/17b
without separators. 
It is handled like here, except that it ``breaks" the loop 
when an item is found that applies.
%% add. 2012/05/17bf.:
After the ``next" token is stored by the usual \cs{futurelet}, 
the exception list is searched without using an iterator. 
Addition and removal commands are provided as well. 

%   \pagebreak                                                %% 2012/05/17b
% \newpage                                                    %% 2012/05/19
\section{Implementation}                                    %% 2012/05/10
\label{sec:implement}
\subsection{Package File Header (Legalese)}                 %% sub 2012/05/10
\input{dowith.doc}

\newpage                                                        %% 2012/06/04
\section{Ack.: 25 Years of Kabel\-schacht's \cs{expandafter}}   %% 2012/05/20
\label{sec:kabel}
The essential idea of 'dowith' and `\DoWith' is 
\[`\if<code>\expandafter<one-token>\fi'\]
%     Alan Jeffrey: \tugbartref{tb11-2/tb28jeffrey}{``Lists in \TeX's Mouth,"} 
%     \acro{TUG}boat Vol.~11 (1990), No.~2, pp.~237--245), 
%     \urlhttpref{tug.org/TUGboat/tb11-2/tb28jeffrey.pdf}.}
It was described by \textsc{Alois Kabelschacht} as 
\tugbartref{tb08-2/tb18kabel}{``&\expandafter\ vs.\ &\let\ and &\def\ 
    in Conditionals and a Generalization of PLAIN's &\loop"}
in \acro{TUG}boat Vol.~8 (1987), No.~2, pp.~184f.\
(a little more than one column).\foothttpurlref{tug.org/TUGboat/%
    tb08-2/tb18kabel.pdf}
See some German biographical notes on Kabelschacht in the 
\wikideref{Benutzer:RolteVolte/Alois_Kabelschacht}{German Wikipedia.}\foothttpurlref{%% 
    de.wikipedia.org/wiki/Benutzer:RolteVolte/Alois\string_Kabelschacht}
It seems to me that Knuth didn't note this application of `\expandafter' 
in \TTb.\footnote{%                                             %% 2012/05/22
    However, the paper \meta{\qtd{uses the fact that the expansion of both 
    \cs{else ... }\cs{fi} and \cs{fi} is empty.}} In \TTb\ I only find 
    %% &\else and &\fi failed
    \meta{\qtd{The ``expansion" of a conditional is empty}} on page~213.}
It was then applied in many macros of \file{latex.ltx}, 
cf.~\file{source2e.pdf}.

\end{document}

VERSION HISTORY

2011/11/02  for v0.1    very first
2011/11/03              discussing \loop; \pagebreak
2011/11/07              xspace
2011/11/10f.            zref added/removed
2012/05/09  for v0.2    $\varepsilon$-\TeX
2012/05/10              "iterator", \MDkeywords, \hypersetup,
                        Legalize -> Legalese, "Related packages", 
                        tighter sectioning
2012/05/14  for v0.21   spurious space in title fixed
2012/05/14b     r0.21a  another keyword
2012/05/15              abstract: why expandable
2012/05/16              discussion much extended; 
                        stored separately before reworking
2012/05/16b     r0.21b  reworking ... many mistakes!
2012/05/17              updating date, was 2012/05/14 before!
                        and from 2012/05/14b onwards it should 
                        have been r0.21a; storing again, 
                        renaming dir.s ...
2012/05/17bf.   r0.21c  reducing text ...
2012/05/19      r0.21d  braces and commands applicable
2012/05/20              corrections, clarifications, reorder; 
                        especially examples for `dowith' vs. 
                        `\@tfor'; remarks about `\loop' corr.:
                        how to `foreach' ...; Kabelschacht
2012/05/22      r0.21e  Kabelschacht vs. TeXbook
2012/06/03      r0.21f  cf. `catoptions'; \dots\unkern\@, vs.\ 
2012/06/07      r0.22a  added \,s in "?typein ..."
2012/07/25      r0.3    "\csname name construct"?
2012/11/02ff.           gullet, plainpkg, more modifications
2012/11/19              mod. on "mouth" (bibleref-mouth)
2012/11/27      r0.3a   plainpkg
2015/11/14              rm. "(five)" & \ctanpkgdref, code in two 
                        footnotes, don't use hyperref draft


\newpage                                                        %% 2012/06/04
\section{Ack.: 25 Years of Kabel\-schacht's \cs{expandafter}}   %% 2012/05/20
\label{sec:kabel}
The essential idea of 'dowith' and `\DoWith' is 
\[`\if<code>\expandafter<one-token>\fi'\]
%     Alan Jeffrey: \tugbartref{tb11-2/tb28jeffrey}{``Lists in \TeX's Mouth,"} 
%     \acro{TUG}boat Vol.~11 (1990), No.~2, pp.~237--245), 
%     \urlhttpref{tug.org/TUGboat/tb11-2/tb28jeffrey.pdf}.}
It was described by \textsc{Alois Kabelschacht} as 
\tugbartref{tb08-2/tb18kabel}{``&\expandafter\ vs.\ &\let\ and &\def\ 
    in Conditionals and a Generalization of PLAIN's &\loop"}
in \acro{TUG}boat Vol.~8 (1987), No.~2, pp.~184f.\
(a little more than one column).\foothttpurlref{tug.org/TUGboat/%
    tb08-2/tb18kabel.pdf}
See some German biographical notes on Kabelschacht in the 
\wikideref{Benutzer:RolteVolte/Alois_Kabelschacht}{German Wikipedia.}\foothttpurlref{%% 
    de.wikipedia.org/wiki/Benutzer:RolteVolte/Alois\string_Kabelschacht}
It seems to me that Knuth didn't note this application of `\expandafter' 
in \TTb.\footnote{%                                             %% 2012/05/22
    However, the paper \meta{\qtd{uses the fact that the expansion of both 
    \cs{else ... }\cs{fi} and \cs{fi} is empty.}} In \TTb\ I only find 
    %% &\else and &\fi failed
    \meta{\qtd{The ``expansion" of a conditional is empty}} on page~213.}
It was then applied in many macros of \file{latex.ltx}, 
cf.~\file{source2e.pdf}.

\end{document}

VERSION HISTORY

2011/11/02  for v0.1    very first
2011/11/03              discussing \loop; \pagebreak
2011/11/07              xspace
2011/11/10f.            zref added/removed
2012/05/09  for v0.2    $\varepsilon$-\TeX
2012/05/10              "iterator", \MDkeywords, \hypersetup,
                        Legalize -> Legalese, "Related packages", 
                        tighter sectioning
2012/05/14  for v0.21   spurious space in title fixed
2012/05/14b     r0.21a  another keyword
2012/05/15              abstract: why expandable
2012/05/16              discussion much extended; 
                        stored separately before reworking
2012/05/16b     r0.21b  reworking ... many mistakes!
2012/05/17              updating date, was 2012/05/14 before!
                        and from 2012/05/14b onwards it should 
                        have been r0.21a; storing again, 
                        renaming dir.s ...
2012/05/17bf.   r0.21c  reducing text ...
2012/05/19      r0.21d  braces and commands applicable
2012/05/20              corrections, clarifications, reorder; 
                        especially examples for `dowith' vs. 
                        `\@tfor'; remarks about `\loop' corr.:
                        how to `foreach' ...; Kabelschacht
2012/05/22      r0.21e  Kabelschacht vs. TeXbook
2012/06/03      r0.21f  cf. `catoptions'; \dots\unkern\@, vs.\ 
2012/06/07      r0.22a  added \,s in "?typein ..."
2012/07/25      r0.3    "\csname name construct"?
2012/11/02ff.           gullet, plainpkg, more modifications
2012/11/19              mod. on "mouth" (bibleref-mouth)
2012/11/27      r0.3a   plainpkg
2015/11/14              rm. "(five)" & \ctanpkgdref, code in two 
                        footnotes, don't use hyperref draft


\newpage                                                        %% 2012/06/04
\section{Ack.: 25 Years of Kabel\-schacht's \cs{expandafter}}   %% 2012/05/20
\label{sec:kabel}
The essential idea of 'dowith' and `\DoWith' is 
\[`\if<code>\expandafter<one-token>\fi'\]
%     Alan Jeffrey: \tugbartref{tb11-2/tb28jeffrey}{``Lists in \TeX's Mouth,"} 
%     \acro{TUG}boat Vol.~11 (1990), No.~2, pp.~237--245), 
%     \urlhttpref{tug.org/TUGboat/tb11-2/tb28jeffrey.pdf}.}
It was described by \textsc{Alois Kabelschacht} as 
\tugbartref{tb08-2/tb18kabel}{``&\expandafter\ vs.\ &\let\ and &\def\ 
    in Conditionals and a Generalization of PLAIN's &\loop"}
in \acro{TUG}boat Vol.~8 (1987), No.~2, pp.~184f.\
(a little more than one column).\foothttpurlref{tug.org/TUGboat/%
    tb08-2/tb18kabel.pdf}
See some German biographical notes on Kabelschacht in the 
\wikideref{Benutzer:RolteVolte/Alois_Kabelschacht}{German Wikipedia.}\foothttpurlref{%% 
    de.wikipedia.org/wiki/Benutzer:RolteVolte/Alois\string_Kabelschacht}
It seems to me that Knuth didn't note this application of `\expandafter' 
in \TTb.\footnote{%                                             %% 2012/05/22
    However, the paper \meta{\qtd{uses the fact that the expansion of both 
    \cs{else ... }\cs{fi} and \cs{fi} is empty.}} In \TTb\ I only find 
    %% &\else and &\fi failed
    \meta{\qtd{The ``expansion" of a conditional is empty}} on page~213.}
It was then applied in many macros of \file{latex.ltx}, 
cf.~\file{source2e.pdf}.

\end{document}

VERSION HISTORY

2011/11/02  for v0.1    very first
2011/11/03              discussing \loop; \pagebreak
2011/11/07              xspace
2011/11/10f.            zref added/removed
2012/05/09  for v0.2    $\varepsilon$-\TeX
2012/05/10              "iterator", \MDkeywords, \hypersetup,
                        Legalize -> Legalese, "Related packages", 
                        tighter sectioning
2012/05/14  for v0.21   spurious space in title fixed
2012/05/14b     r0.21a  another keyword
2012/05/15              abstract: why expandable
2012/05/16              discussion much extended; 
                        stored separately before reworking
2012/05/16b     r0.21b  reworking ... many mistakes!
2012/05/17              updating date, was 2012/05/14 before!
                        and from 2012/05/14b onwards it should 
                        have been r0.21a; storing again, 
                        renaming dir.s ...
2012/05/17bf.   r0.21c  reducing text ...
2012/05/19      r0.21d  braces and commands applicable
2012/05/20              corrections, clarifications, reorder; 
                        especially examples for `dowith' vs. 
                        `\@tfor'; remarks about `\loop' corr.:
                        how to `foreach' ...; Kabelschacht
2012/05/22      r0.21e  Kabelschacht vs. TeXbook
2012/06/03      r0.21f  cf. `catoptions'; \dots\unkern\@, vs.\ 
2012/06/07      r0.22a  added \,s in "?typein ..."
2012/07/25      r0.3    "\csname name construct"?
2012/11/02ff.           gullet, plainpkg, more modifications
2012/11/19              mod. on "mouth" (bibleref-mouth)
2012/11/27      r0.3a   plainpkg
2015/11/14              rm. "(five)" & \ctanpkgdref, code in two 
                        footnotes, don't use hyperref draft


\newpage                                                        %% 2012/06/04
\section{Ack.: 25 Years of Kabel\-schacht's \cs{expandafter}}   %% 2012/05/20
\label{sec:kabel}
The essential idea of 'dowith' and `\DoWith' is 
\[`\if<code>\expandafter<one-token>\fi'\]
%     Alan Jeffrey: \tugbartref{tb11-2/tb28jeffrey}{``Lists in \TeX's Mouth,"} 
%     \acro{TUG}boat Vol.~11 (1990), No.~2, pp.~237--245), 
%     \urlhttpref{tug.org/TUGboat/tb11-2/tb28jeffrey.pdf}.}
It was described by \textsc{Alois Kabelschacht} as 
\tugbartref{tb08-2/tb18kabel}{``&\expandafter\ vs.\ &\let\ and &\def\ 
    in Conditionals and a Generalization of PLAIN's &\loop"}
in \acro{TUG}boat Vol.~8 (1987), No.~2, pp.~184f.\
(a little more than one column).\foothttpurlref{tug.org/TUGboat/%
    tb08-2/tb18kabel.pdf}
See some German biographical notes on Kabelschacht in the 
\wikideref{Benutzer:RolteVolte/Alois_Kabelschacht}{German Wikipedia.}\foothttpurlref{%% 
    de.wikipedia.org/wiki/Benutzer:RolteVolte/Alois\string_Kabelschacht}
It seems to me that Knuth didn't note this application of `\expandafter' 
in \TTb.\footnote{%                                             %% 2012/05/22
    However, the paper \meta{\qtd{uses the fact that the expansion of both 
    \cs{else ... }\cs{fi} and \cs{fi} is empty.}} In \TTb\ I only find 
    %% &\else and &\fi failed
    \meta{\qtd{The ``expansion" of a conditional is empty}} on page~213.}
It was then applied in many macros of \file{latex.ltx}, 
cf.~\file{source2e.pdf}.

\end{document}

VERSION HISTORY

2011/11/02  for v0.1    very first
2011/11/03              discussing \loop; \pagebreak
2011/11/07              xspace
2011/11/10f.            zref added/removed
2012/05/09  for v0.2    $\varepsilon$-\TeX
2012/05/10              "iterator", \MDkeywords, \hypersetup,
                        Legalize -> Legalese, "Related packages", 
                        tighter sectioning
2012/05/14  for v0.21   spurious space in title fixed
2012/05/14b     r0.21a  another keyword
2012/05/15              abstract: why expandable
2012/05/16              discussion much extended; 
                        stored separately before reworking
2012/05/16b     r0.21b  reworking ... many mistakes!
2012/05/17              updating date, was 2012/05/14 before!
                        and from 2012/05/14b onwards it should 
                        have been r0.21a; storing again, 
                        renaming dir.s ...
2012/05/17bf.   r0.21c  reducing text ...
2012/05/19      r0.21d  braces and commands applicable
2012/05/20              corrections, clarifications, reorder; 
                        especially examples for `dowith' vs. 
                        `\@tfor'; remarks about `\loop' corr.:
                        how to `foreach' ...; Kabelschacht
2012/05/22      r0.21e  Kabelschacht vs. TeXbook
2012/06/03      r0.21f  cf. `catoptions'; \dots\unkern\@, vs.\ 
2012/06/07      r0.22a  added \,s in "?typein ..."
2012/07/25      r0.3    "\csname name construct"?
2012/11/02ff.           gullet, plainpkg, more modifications
2012/11/19              mod. on "mouth" (bibleref-mouth)
2012/11/27      r0.3a   plainpkg
2015/11/14              rm. "(five)" & \ctanpkgdref, code in two 
                        footnotes, don't use hyperref draft
