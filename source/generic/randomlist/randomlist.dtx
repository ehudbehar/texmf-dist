% \iffalse
%
%% randomlist.dtx
%% Copyleft 2013-2017 J.-C. Charpentier & C. Tellechea
%
%% Packages `randomlist' to use with (La)eTeX
%% Copyleft (L) 2013-2017 Jean-C\^ome Charpentier & Christian Tellechea.
% \fi
%
%% \CharacterTable
%%  {Upper-case    \A\B\C\D\E\F\G\H\I\J\K\L\M\N\O\P\Q\R\S\T\U\V\W\X\Y\Z
%%   Lower-case    \a\b\c\d\e\f\g\h\i\j\k\l\m\n\o\p\q\r\s\t\u\v\w\x\y\z
%%   Digits        \0\1\2\3\4\5\6\7\8\9
%%   Exclamation   \!     Double quote  \"     Hash (number) \#
%%   Dollar        \$     Percent       \%     Ampersand     \&
%%   Acute accent  \'     Left paren    \(     Right paren   \)
%%   Asterisk      \*     Plus          \+     Comma         \,
%%   Minus         \-     Point         \.     Solidus       \/
%%   Colon         \:     Semicolon     \;     Less than     \<
%%   Equals        \=     Greater than  \>     Question mark \?
%%   Commercial at \@     Left bracket  \[     Backslash     \\
%%   Right bracket \]     Circumflex    \^     Underscore    \_
%%   Grave accent  \`     Left brace    \{     Vertical bar  \|
%%   Right brace   \}     Tilde         \~}
%%
% \iffalse
%<*driver>
\documentclass[a4paper, 12pt]{ltxdoc}
\errorcontextlines=10

\makeatletter
\renewenvironment{theglossary}{%
  \glossary@prologue
  \GlossaryParms \let\item\@idxitem \ignorespaces}%
{}
\makeatother
\GlossaryPrologue{%
  \newpage
  \section*{Change history}
  \markboth{{Change history}}{{Change history}}%
}
\DisableCrossrefs
%\EnableCrossrefs
\CodelineIndex
\RecordChanges
%\OnlyDescription
\usepackage[T1]{fontenc}
\usepackage[utf8]{inputenc}
\usepackage{kpfonts}
\usepackage[a4paper, left=2cm, right=2cm, top=2cm, bottom=2cm]{geometry}
\usepackage{array}
\usepackage{fvrb-ex}
\fvset{frame=lines,xrightmargin=0.5\linewidth}
\usepackage{makeidx}
\usepackage{randomlist}
\usepackage[USenglish]{babel}
\usepackage{hyperref}

\pagestyle{plain}
\makeindex
\title{%
  Package \package{randomlist}\\Tools for data base, table and random
  writting-reading}
\author{%
  Jean-Côme Charpentier\thanks{\ttfamily
    jean-come.charpentier@wanadoo.fr}
  \and
  Christian Tellechea\thanks{\ttfamily unbonpetit@openmailbox.org}
}
\newcommand*\pgm[1]{\texttt{#1}}
\newcommand*\file[1]{\texttt{#1}}
\newcommand*\package[1]{\texttt{#1}}
\newcommand*\class[1]{\texttt{#1}}
\newcommand*\option[1]{\texttt{#1}}
\newcommand*\key[1]{\texttt{#1}}
\newcommand*\Key[1]{\texttt{#1}\index{#1=\texttt{#1}}}
\newcommand*\environ[1]{\texttt{#1}}
\newcommand*\Environ[1]{\texttt{#1}\index{#1=\texttt{#1}}}
\makeatletter
\edef\quotechar{\string!}
\edef\actualchar{\string=}
\edef\verbatimchar{\string+}
\def\SpecialPageIndex#1{%
  \immediate\write\@indexfile{%
    \string\indexentry{\expandafter\@gobble\string#1\actualchar
    \string\verb\quotechar*\verbatimchar\string#1\verbatimchar|hyperpage}%
    {\number\c@page}%
  }%
}
\makeatother
\newcommand*\Cmd[1]{\texttt{\string#1}\SpecialPageIndex{#1}}

\begin{document}
\maketitle
\newpage
\setcounter{tocdepth}{3}
\tableofcontents
\newpage
\section{Overview}
The main aim of package \package{randomlist} is to work on list,
especially with random operation. The hidden aim is to build
personnal collection of exercices with different data for each
pupils. In order to build such exercices, some features about
databases are necessary.

In ``randomlist'', the word ``List'' must be understound with two
meanings:
\begin{itemize}
\item itemize and enumerate \LaTeX{} environments;
\item list as in computer science.
\end{itemize}
In fact, lists as in computer are not really lists: they are
arrays. Some commands allow to deal with these data structures as
queues, other commands as stacks and another commands as arrays.

\section{Array, queue, stack, or list?}
The package give the name ``list'' to the main data structure. First,
we have to declare a new list with command \Cmd{\NewList}. There is
nothing special about this command. It has a mandatory argument: the
name of the list.

Nearly any name is possible. However, don't use hyphen and number at
the end of the name. For instance \texttt{mylist-1} isn't a good idea
(\texttt{mylist*1} is a good one). Don't use fragile commands and
special characters. However you can use commands inside list names.

\subsection{Create, erase and show list}
You can't create an already existing list. For instance, the code:
\begin{Verbatim}[xrightmargin=0pt,xleftmargin=0.5\linewidth]
  \NewList{MyList}
  \NewList{MyList}
\end{Verbatim}
give the error message:
\begin{Verbatim}[frame=none]
  ! Package randomlist Error: List MyList already exists.
\end{Verbatim}
If you want erase a list, use the command \Cmd{\ClearList}.

You can't create a list \key{namelist} if the macro \cmd{\namelist}
exists. For instance the code:
\begin{Verbatim}[xrightmargin=0pt,xleftmargin=0.5\linewidth]
  \NewList{def}
\end{Verbatim}
give the error message:
\begin{Verbatim}[frame=none]
  ! Package randomlist Error: Command \def already exists.
\end{Verbatim}

\medskip
As you can see, the source is between horizontal rules and flush
right. The result is flush left. When the result isn't an error
message, source and result are side by side.

For the next commands we must be able to see the state of list. The
package \package{randomlist} offers the \Cmd{\ShowList} command which
allows to see the whole list. When a list is just created, it is empty,
so the \cmd{\ShowList} command shows it like that (source is typeseted
at the right side and result is showed at the left side):\par\medskip

\begin{SideBySideExample}
  \NewList{MyList}
  \ShowList{MyList}
\end{SideBySideExample}
\par\medskip

\subsection{Writing and reading in a list}
Once a list is created, you can write values and, after that, read
values.

In fact, these lists can behaves like queues, stacks, or arrays
according to the used command. There are four kinds of command:
\begin{itemize}
\item Insert;
\item Extract;
\item Set;
\item Get.
\end{itemize}
\goodbreak
Each one has four variants to reach some position in the list:
\begin{itemize}
\item First;
\item Last;
\item Index (without prefix);
\item Random.
\end{itemize}
Thus we have the commands:
\begin{center}
  \begin{tabular}{*{4}{>{\ttfamily\textbackslash}l}}
    InsertFirstItem & ExtractFirstItem & SetFirstItem & GetFirstItem
    \\
    InsertLastItem & ExtractLastItem & SetLastItem & GetLastItem
    \\
    InsertItem & ExtractItem & SetItem & GetItem
    \\
    InsertRandomItem & ExtractRandomItem & SetRandomItem & GetRandomItem
  \end{tabular}
\end{center}

Each one has also a ``List'' variant which acts on several items. That
is, we have these commands: \cmd{\InsertList}, \cmd{\ExtractList},
\cmd{\SetList}, and \cmd{\GetList}. There is also a command
\Cmd{\CopyList} which is a shortcut for a special \cmd{\SetList}.

Finally, we have \cmd{\ShiftList} which is somewhere quiet special: it
creates empty items or destroy items by shifting inside a list. This
macro is rather for internal operation but you can use it. The syntax is:
\begin{Verbatim}[xrightmargin=0pt,xleftmargin=0.5\linewidth]
  \ShiftList{<list>}{<start>}{<nb>}
\end{Verbatim}

When \texttt{<nb>} is positive, then items from index \texttt{<start>}
are right shifted. That is, index $n$ becomes index $n+\mathtt{nb}$
and index from \texttt{<start>} to
$\text{\texttt{<start>}}+\text{\texttt{<nb>}}-1$ are empty.

When \texttt{<nb>} is negative, then items starting from the end of
the list are left shifted and \texttt{<nb>} items (from index
\texttt{start}) disappear.

Usually, you don't need \Cmd{\ShiftList}: the other commands,
especially \cmd{\ExtractList} and \Cmd{\InsertList} are enough.

\subsubsection{Insert commands}
\paragraph[\string\InsertFirstItem]{\cmd{\InsertFirstItem}}
\Cmd{\InsertFirstItem} writes a value at the beginning of a list and
shift other values to the end of the list. This command has two
arguments: the list name and the value to insert. For
example:\par\medskip 

\begin{SideBySideExample}
  \NewList{MyList}
  \InsertFirstItem{MyList}{First value}
  \ShowList{MyList}
  \InsertFirstItem{MyList}{Second value}
  \ShowList{MyList}
  \InsertFirstItem{MyList}{Third value}
  \ShowList{MyList}
\end{SideBySideExample}
\par\medskip

As you can see, the list is in the reverse order than the user
one. With \cmd{\InsertFirstItem}, list behaves like stack.

The value written in the list can be nearly anything. For
instance:\par\medskip

\begin{SideBySideExample}
  \NewList{MyList}
  \InsertFirstItem{MyList}{{two}{groups}}
  \InsertFirstItem{MyList}{\textbf{text}}
  \InsertFirstItem{MyList}{special^}
  \InsertFirstItem{MyList}{end}
  \ShowList{MyList}
\end{SideBySideExample}
\par\medskip

\paragraph[\string\InsertLastItem]{\cmd{\InsertLastItem}}
\Cmd{\InsertLastItem} is like \cmd{\InsertFirstItem} but the insertion
is made at the end of the list. As for the previous command, it takes
two arguments: the list name and the value to insert. The previous
example give:\par\medskip

\begin{SideBySideExample}
  \NewList{MyList}
  \InsertLastItem{MyList}{{two}{groups}}
  \InsertLastItem{MyList}{\textbf{text}}
  \InsertLastItem{MyList}{special^}
  \InsertLastItem{MyList}{end}
  \ShowList{MyList}
\end{SideBySideExample}
\par\medskip

With this command, the list behaves as queue.

\paragraph[\string\InsertItem]{\cmd{\InsertItem}}
\Cmd{\InsertItem} is like \cmd{\InsertFirstItem} but the insertion
is made to a specified position of the list. With this command, the list
behaves as an array.

The index starts from zero and if the list has $n$ elements then the
index can't be greater than $n$. When a value is inserted in position
$k$, then the previous values from $k$ to $n-1$ are shifted one
position to the right.

\cmd{\InsertItem} takes three arguments: the list name, the position
and the value. Here is an example:\par\medskip

\begin{SideBySideExample}
  \NewList{MyList}
  \InsertLastItem{MyList}{first}
  \InsertLastItem{MyList}{\textbf{second}}
  \InsertLastItem{MyList}{special^}
  \InsertLastItem{MyList}{end}
  \ShowList{MyList}
  \InsertItem{MyList}{2}{\textbf{insert!}}
  \InsertItem{MyList}{2}{\textbf{other!}}
  \InsertItem{MyList}{6}{real end}
  \ShowList{MyList}
\end{SideBySideExample}
\par\medskip

As you can see, when the two values is inserted at position~2, the
value \texttt{special\textasciicircum} is shifted from position~2 to
position~4. Moreover, it's possible to insert a value to the
nonexistent position $n$ when the length of the list is $n$: it's the
only one possibility to specify a nonexistent index.

\paragraph[\string\InsertRandfomItem]{\cmd{\InsertRandomItem}}
This is the first command which use random numbers. We will see later
how to manage random numbers themselves.

\Cmd{\InsertRandomItem} command works as the \cmd{\InsertItem} one
but the position is selected randomly by \TeX{}. Then there is only two
arguments: the list name and the value to insert. Here is an example:
\RLsetrandomseed 1\par\medskip

\begin{SideBySideExample}
  \NewList{MyList}
  \InsertRandomItem{MyList}{first}
  \InsertRandomItem{MyList}{second}
  \InsertRandomItem{MyList}{third}
  \InsertRandomItem{MyList}{fourth}
  \InsertRandomItem{MyList}{fifth}
  \ShowList{MyList}
\end{SideBySideExample}
\par\medskip

\paragraph[\string\InsertList]{\cmd{\InsertList}}
\Cmd{\InsertList} allows to do in one shot what the \cmd{\InsertItem}
can do in several ones. The principle of this command is to insert all
the items of a list inside a second one. You have to give three
arguments: the list which receive, the index to start insertion, and
the list to insert. For instance:\par\medskip

\begin{SideBySideExample}
  \NewList{MyList}
  \NewList{OtherList}
  \InsertLastItem{MyList}{first in M}
  \InsertLastItem{MyList}{fourth in M}
  \InsertLastItem{MyList}{fifth in M}
  \InsertLastItem{OtherList}{second in O}
  \InsertLastItem{OtherList}{third in O}
  \InsertList{MyList}{1}{OtherList}
  \ShowList{MyList}
  \ShowList{OtherList}
\end{SideBySideExample}
\par\medskip

As you can see, the second list is unchanged after operation.

Package \package{randomlist} checks that both lists exists and that
index is compatible with list. Otherwise an error message will be
raised.

It's possible to insert an empty list:\par\medskip

\begin{SideBySideExample}
  \NewList{MyList}
  \NewList{OtherList}
  \InsertLastItem{MyList}{first}
  \InsertLastItem{MyList}{second}
  \InsertLastItem{MyList}{third}
  \InsertList{MyList}{1}{OtherList}
  \ShowList{MyList}
\end{SideBySideExample}
\par\medskip

\subsubsection{Extract commands}
\paragraph[\string\ExtractFirstItem]{\cmd{\ExtractFirstItem}}
The four commands \cmd{\Extract...Item} are the inverse one of the
four commands \cmd{\Insert...Item}.

\Cmd{\ExtractFirstItem} extract the first value of a list and store it
in a macro. The other elements of the list are shifted left (the list
length decreases by one). This command takes two argument: the list
name and the macro name where the value is stored. The last argument
is just the name of the macro, \emph{i.e.} the macro name without the
backslash. For example:\par\medskip

\begin{SideBySideExample}
  \NewList{MyList}
  \InsertLastItem{MyList}{\TeX}
  \InsertLastItem{MyList}{is}
  \InsertLastItem{MyList}{very}
  \InsertLastItem{MyList}{powerful}
  \ExtractFirstItem{MyList}{MyMacro}
  The first element was ``\MyMacro''.

  \ShowList{MyList}
\end{SideBySideExample}
\par\medskip

When you extract an element from a list, the list length decreases by
one. It explains why it's forbidden to extract an element from an
empty list. If you try  it,
\begin{Verbatim}[xrightmargin=0pt,xleftmargin=0.5\linewidth]
  \NewList{MyList}
  \ExtractFirstItem{MyList}{MyMacro}
\end{Verbatim}
you have the error message:
\begin{Verbatim}[frame=none]
  ! Package randomlist Error: List MyList is empty.
\end{Verbatim}

\paragraph[\string\ExtractLastItem]{\cmd{\ExtractLastItem}}
\Cmd{\ExtractLastItem} behaves like \cmd{\ExtractFirstItem} but the
element extracted is the last one. Thus there is no shifting, there is
just a decrementation of the list length.

Here is an example:\par\medskip

\begin{SideBySideExample}
  \NewList{MyList}
  \InsertLastItem{MyList}{\TeX}
  \InsertLastItem{MyList}{is}
  \InsertLastItem{MyList}{very}
  \InsertLastItem{MyList}{powerful}
  \ExtractLastItem{MyList}{MyMacro}
  The last element was ``\MyMacro''.

  \ShowList{MyList}
\end{SideBySideExample}
\par\medskip

\paragraph[\string\ExtractItem]{\cmd{\ExtractItem}}
\Cmd{\ExtractItem} behaves like \cmd{\ExtractFirstItem} but the
element extracted is the one indicated by its index. The command takes
three argument: the list name, the index of element to extract, the
macro used to store the element extracted. Don't forget that indexes
start from zero. Here is an example:\par\medskip

\begin{SideBySideExample}
  \NewList{MyList}
  \InsertLastItem{MyList}{\TeX}
  \InsertLastItem{MyList}{is}
  \InsertLastItem{MyList}{very}
  \InsertLastItem{MyList}{powerful}
  \ExtractItem{MyList}{2}{MyMacro}
  The third element was ``\MyMacro''.

  \ShowList{MyList}
\end{SideBySideExample}
\par\medskip

There isn't anything special. The length of the list decreases by one
and elements are shifted accordingly to the extracted one.

\paragraph[\string\ExtractRandomItem]{\cmd{\ExtractRandomItem}}
\Cmd{\ExtractRandomItem} works like the previous
\cmd{\ExtractItem}. Here, the index is selected randomly by the
computer. Then there are only two arguments: the list name and the
macro to store the extracted element:\par\medskip

\RLsetrandomseed 3
\begin{SideBySideExample}
  \NewList{MyList}
  \InsertLastItem{MyList}{\TeX}
  \InsertLastItem{MyList}{is}
  \InsertLastItem{MyList}{very}
  \InsertLastItem{MyList}{powerful}
  \ExtractRandomItem{MyList}{MyMacro}
  ``\MyMacro'' was extracted.

  \ShowList{MyList}
\end{SideBySideExample}
\par\medskip

Even the extraction is made on a random index, it's forbidden to
extract something from an empty list. Then, the code:
\begin{Verbatim}[xrightmargin=0pt,xleftmargin=0.5\linewidth]
  \NewList{MyList}
  \ExtractRandomItem{MyList}{MyMacro}
\end{Verbatim}
gives the usual error message:
\begin{Verbatim}[frame=none]
  ! Package randomlist Error: List MyList is empty.
\end{Verbatim}

\paragraph[\string\ExtractList]{\cmd{\ExtractList}}
The commands \cmd{\Extract...Item} extract one item and store it in a
macro. With the command \Cmd{\ExtractList} we can extract several
items and put them in a list. \cmd{\ExtractList} asks for four
arguments:
\begin{enumerate}
\item the main list;
\item the starting index;
\item the ending index;
\item the list which receive extracted values.
\end{enumerate}
Here is an example:\par\medskip

\begin{SideBySideExample}
  \NewList{MyList}
  \NewList{OtherList}
  \InsertLastItem{MyList}{first}
  \InsertLastItem{MyList}{second}
  \InsertLastItem{MyList}{third}
  \InsertLastItem{MyList}{fourth}
  \InsertLastItem{MyList}{fifth}
  \InsertLastItem{MyList}{sixth}
  \ExtractList{MyList}{2}{4}{OtherList}

  \ShowList{MyList}
  \ShowList{OtherList}
\end{SideBySideExample}
\par\medskip

Obviously, \package{randomlist} checks list and indexes. You can have
the start index and the last index equals. In this case,
\cmd{\ExtractList} behaves like \cmd{\ExtractItem} but the extracted
value is put in a list rather than in a macro:\par\medskip

\begin{SideBySideExample}
  \NewList{MyList}
  \NewList{OtherList}
  \InsertLastItem{MyList}{first}
  \InsertLastItem{MyList}{second}
  \InsertLastItem{MyList}{third}
  \InsertLastItem{MyList}{fourth}
  \InsertLastItem{MyList}{fifth}
  \InsertLastItem{MyList}{sixth}
  \ExtractList{MyList}{2}{2}{OtherList}

  \ShowList{MyList}
  \ShowList{OtherList}
\end{SideBySideExample}
\par\medskip

\subsubsection{Set commands}
\paragraph[\string\SetFirstItem]{\cmd{\SetFirstItem}}
The commands \cmd{\Set...Item} modify the existing values of
list. \Cmd{\SetFirstItem} modify the first value.\par\medskip

\begin{SideBySideExample}
  \NewList{MyList}
  \InsertLastItem{MyList}{\TeX}
  \InsertLastItem{MyList}{is}
  \InsertLastItem{MyList}{very}
  \InsertLastItem{MyList}{powerful}
  \SetFirstItem{MyList}{\LaTeX}
  \ShowList{MyList}
\end{SideBySideExample}
\par\medskip

If a list is empty, there is the classic error message about empty
list.

\paragraph[\string\SetLastItem]{\cmd{\SetLastItem}}
\Cmd{\SetLastItem} acts like \cmd{\SetFirstItem} but at the end of the
list.

\paragraph[\string\SetItem]{\cmd{\SetItem}}
\Cmd{\SetItem} acts like the previous commands. It takes three
arguments: the list name, the index, the new value:\par\medskip

\begin{SideBySideExample}
  \NewList{MyList}
  \InsertLastItem{MyList}{\TeX}
  \InsertLastItem{MyList}{is}
  \InsertLastItem{MyList}{very}
  \InsertLastItem{MyList}{powerful}
  \SetItem{MyList}{2}{quiet}
  \ShowList{MyList}
\end{SideBySideExample}
\par\medskip

If the index doesn't exist, an error message is showed. Code:
\begin{Verbatim}[xrightmargin=0pt,xleftmargin=0.5\linewidth]
  \NewList{MyList}
  \InsertLastItem{MyList}{\TeX}
  \InsertLastItem{MyList}{is}
  \InsertLastItem{MyList}{very}
  \InsertLastItem{MyList}{powerful}
  \SetItem{MyList}{4}{isn't it?}
\end{Verbatim}
gives the error message:
\begin{Verbatim}[frame=none]
  ! Package randomlist Error: Index 4 is greater than last index of list MyList.
\end{Verbatim}

\paragraph[\string\SetRandomItem]{\cmd{\SetRandomItem}}
\Cmd{\SetRandomItem} acts like the previous one but the index is
selected randomly. The list must be non empty. Here is an
example:\par\medskip 

\begin{SideBySideExample}
  \NewList{MyList}
  \InsertLastItem{MyList}{\TeX}
  \InsertLastItem{MyList}{is}
  \InsertLastItem{MyList}{really}
  \InsertLastItem{MyList}{very}
  \InsertLastItem{MyList}{powerful}
  \SetRandomItem{MyList}{snap!}
  \ShowList{MyList}
\end{SideBySideExample}
\par\medskip

\paragraph[\string\SetList]{\cmd{\SetList}}
Insert value one by one inside a list could be tiresome especially if
you have many values. Package \package{randomlist} allows to insert
many items in a row using the macro \Cmd{\SetList}. Items are
separated with comma. For instance:\par\medskip 

\begin{SideBySideExample}
  \NewList{MyList}
  \SetList{MyList}{\TeX, is, very,
    powerful}
  \ShowList{MyList}
\end{SideBySideExample}
\par\medskip

As you can see, spaces aren't discarded. A more satisfactory
presentation would be:\par\medskip 

\begin{SideBySideExample}
  \NewList{MyList}
  \SetList{MyList}{\TeX,is,very,%
    powerful}
  \ShowList{MyList}
\end{SideBySideExample}
\par\medskip

\paragraph[\string\CopyList]{\cmd{\CopyList}}
Copy a list in another one. Both lists must exists: this command don't
create list since only \cmd{\NewList} can do that. Here is an
example:\par\medskip

\begin{SideBySideExample}
  \NewList{MyList}
  \NewList{OtherList}
  \SetList{MyList}{\TeX,is,very,%
                   powerful}
  \CopyList{MyList}{OtherList}
  \ShowList{OtherList}
\end{SideBySideExample}
\par\medskip

\subsubsection{Get commands}
\paragraph[\string\GetFirstItem]{\cmd{\GetFirstItem}}
The \cmd{\get...list} look for a value in a list. They don't change
the list. The index must exist elsewhere an error message will be
show. That is, for first, last, and random variant, list must be non
empty.

\Cmd{\GetFirstItem} put the first value of a list into a macro. The
arguments of the command are: the list name, the macro:\par\medskip

\begin{SideBySideExample}
  \NewList{MyList}
  \SetList{MyList}{\TeX,is,so,cute}
  \GetFirstItem{MyList}{MyMacro}
  The first element is ``\MyMacro''

  \ShowList{MyList}
\end{SideBySideExample}
\par\medskip

\paragraph[\string\GetLastItem]{\cmd{\GetLastItem}}
\Cmd{\GetLastItem} acts like \cmd{\GetFirstItem} but give the last value.

\paragraph[\string\GetItem]{\cmd{\GetItem}}
\Cmd{\GetItem} acts like the previous one but give the value of the
element $k$ where $k$ is the second argument. Pay attention that
indexes start from zero. Then the index $k$ maps to the $k+1$st
element of the list.\par\medskip

\begin{SideBySideExample}
  \NewList{MyList}
  \SetList{MyList}{\TeX,is,so,cute}
  \GetItem{MyList}{2}{MyMacro}
  The third element is ``\MyMacro''

  \ShowList{MyList}
\end{SideBySideExample}
\par\medskip

Package \package{randomlist} offers an other syntax to access to an
item: \cmd{\<namelist>[<index>]}. Thus, we can write the previous
example like that:\par\medskip

\begin{SideBySideExample}
  \NewList{MyList}
  \SetList{MyList}{\TeX,is,so,cute}
  The third element is ``\MyList[2]''
\end{SideBySideExample}
\par\medskip

\paragraph[\string\GetRandomItem]{\cmd{\GetRandomItem}}
\Cmd{\GetRandomItem} give the value of a randomly selected element of a
list.\par\medskip

\begin{SideBySideExample}
  \NewList{MyList}
  \SetList{MyList}{\TeX,is,so,cute}
  \GetRandomItem{MyList}{MyMacro}
  The random element is ``\MyMacro''

  \ShowList{MyList}
\end{SideBySideExample}
\par\medskip

\paragraph[\string\GetList]{\cmd{\GetList}}
\Cmd{\GetList} builds a sub-list. Arguments are those of
\cmd{\ExtractList}, that is, the read list, the first index, the last
index, and the written list.\par\medskip

\begin{SideBySideExample}
  \NewList{MyList}
  \NewList{OtherList}
  \SetList{MyList}{X1,X2,X3,X4,X5,X6}
  \GetList{MyList}{2}{4}{OtherList}
  \ShowList{MyList}
  \ShowList{OtherList}
\end{SideBySideExample}
\par\medskip

Contrary to what \cmd{\ExtractItem} do, \cmd{\GetList} don't modify
the source list.

\section{Database}
\subsection{Simple database}
Package \package{randomlist} offers some features about databases. In
fact that was the first aim of this package: to be able to product one
assignment for one pupil (with all assignments different).

For \package{randomlist} a database is a list. For instance the next
example shows an usual list which is used as a database. We'll see
later real databases with records and fields. For now, our
database has records and each record has one single field: the name
and first name of our pupils. In order to parse all entries of
database \package{randomlist} offers the commands
\cmd{\ForEach...Item}. These command extract one by one all the
elements of a list and typeset, for each element, its third
argument. Second argument give the macro name where element is stored.
For these commands, the macro name is given without the
backslash. Depending how the extraction is made, we have the three
commands: \Cmd{\ForEachFirstItem}, \cmd{\ForEachLastItem}, and
\cmd{\ForEachRandomItem}. In fact, the readind is made with an
extraction but, as the work is made in a group, after the
\cmd{\ForEach...} command, the list is restored.

\subsubsection[\protect\texttt{\protect\textbackslash ForEachFirstItem}]{\cmd{\ForEachFirstItem}}

\begin{SideBySideExample}
  \NewList{Pupils}
  \SetList{Pupils}{Alfred Aho,%
    Charles Babbage,Gregory Chaintin,%
    Edsger Dijkstra}
  \ForEachFirstItem{Pupils}{Name}{%
    Test for \Name\par
    blah blah blah\dots\par\smallskip
  }
\end{SideBySideExample}
\par\medskip

\subsubsection[\protect\texttt{\protect\textbackslash ForEachLastItem}]{\cmd{\ForEachLastItem}}
\Cmd{\ForEachLastItem} acts like \cmd{\ForEachFirstItem} but the
reading is made in reverse order:\par\medskip

\begin{SideBySideExample}
  \NewList{Pupils}
  \SetList{Pupils}{Alfred Aho,%
    Charles Babbage,Gregory Chaintin,%
    Edsger Dijkstra}
  \ForEachLastItem{Pupils}{Name}{%
    Test for \Name\par
    blah blah blah\dots\par\smallskip
  }
\end{SideBySideExample}
\par\medskip

\subsubsection[\protect\texttt{\protect\textbackslash ForEachRandomItem}]{\cmd{\ForEachRandomItem}}
\Cmd{\ForEachRandomItem} acts like the previous commands but the
reading is made randomly. In the next example, we can see that the
list is restored after the command
\cmd{\ForEachRandomItem}:\par\medskip

\begin{SideBySideExample}
  \NewList{Pupils}
  \SetList{Pupils}{Alfred Aho,%
    Charles Babbage,Gregory Chaintin,%
    Edsger Dijkstra}
  \ForEachRandomItem{Pupils}{Name}{%
    Test for \Name\par
    blah blah blah\dots\par\smallskip
  }
  \ShowList{Pupils}
\end{SideBySideExample}
\par\medskip

You can put a command \cmd{\ForEach} inside another one. There is no
limits (but the stacks of \TeX{}):\par\medskip

\begin{SideBySideExample}
  \NewList{L-Man}
  \NewList{L-Woman}
  \SetList{L-Man}{Alfred,Charles,Gregory}
  \SetList{L-Woman}{Ada,Grace,Adele}
  \ForEachRandomItem{L-Man}{Man}{%
    \ForEachRandomItem{L-Woman}{Woman}{%
      \Man{} and \Woman{} are computer
      scientists\par
    }
  }
\end{SideBySideExample}
\par\medskip

Actually, there is a bug that don't allow fragile commands inside
lists when they are read with \cmd{\ForEach...Item} commands. I hope
that the next version of \package{randomlist} will fix this!

\subsection{Database with fields}
Each record of a database is read as a set of fields. In fact it's a
sequence of groups. \package{randomlist} allow to read each field with
the macro \Cmd{\ReadFieldItem}. In order to make life easy,
\package{randomlist} allow to read whole database from files with the
command \Cmd{\ReadFileList}.

This command read a field in a record and store it in a macro. It
takes three arguments: a whole record or a macro containing the whole
record, the rank of the field (starting zero), and a macro to store
the value of this field. For instance:\par\medskip

\begin{SideBySideExample}
  \def\record{{ein}{un}{one}}
  \ReadFieldItem{\record}{0}{Zahl}
  \ReadFieldItem{\record}{1}{Nombre}
  \ReadFieldItem{\record}{2}{Number}
  \Nombre\ French, \Zahl\ German,
  and \Number\ English.
\end{SideBySideExample}
\par\medskip

If there are less fields than the indicating rank then an
error message is raised:
\begin{Verbatim}[xrightmargin=0pt,xleftmargin=0.5\linewidth]
  \def\record{{ein}{un}{one}}
  \ReadFieldItem{\record}{3}{Stuff}
\end{Verbatim}
give the error message:
\begin{Verbatim}[frame=none]
  ! Package randomlist Error: There aren't enough fields in the record.
\end{Verbatim}
Remember that fields are numbered starting from zero!

Obviously, the power of \Cmd{\ReadFieldItem} comes with list and real
database. For instance:\par\medskip

\begin{SideBySideExample}
  \NewList{Languages}
  \SetList{Languages}{{France}{un},%
    {Germany}{ein},{England}{one}}
  \ForEachFirstItem{Languages}{Unit}{%
    \ReadFieldItem{\Unit}{0}{Country}%
    \ReadFieldItem{\Unit}{1}{Number}%
    You say ``\Number'' in \Country.\par
  }
\end{SideBySideExample}
\par\medskip

It's not very handy to write a whole database inside a \LaTeX{}
source. \package{randomlist} allows to load a data base reading a
extern file. For that, there is the command \Cmd{\ReadFileList}. This
command takes two mandatory arguments: the name of the database and
the name of the file.

The file \texttt{pythagoras.dat} (page~\pageref{file-pythagoras})
shows 100~lines of three numbers separated by comma. When this file is
read, the data base contains 100~records with three fields. That is,
by default, \package{randomlist} read CSV files (Comma Separated
Values).\par\medskip

\begin{SideBySideExample}
  \NewList{Pyth}
  \ReadFileList{Pyth}{pythagoras.dat}
  \GetItem{Pyth}{10}{triple}
  \ReadFieldItem{\triple}{0}{triplea}
  \ReadFieldItem{\triple}{1}{tripleb}
  Do you know that
  $\sqrt{\triplea^2+\tripleb^2}$ is an
  integer?
\end{SideBySideExample}
\par\medskip

\NewList{Pyth}
\ReadFileList{Pyth}{pythagoras.dat}
\GetItem{Pyth}{10}{triple}
\ReadFieldItem{\triple}{0}{triplea}
\ReadFieldItem{\triple}{1}{tripleb}
\ReadFieldItem{\triple}{2}{triplec}
To those who check the triple page~\pageref{file-pythagoras}, don't
forget that the 10th rank maps with line number~11, that is,
``\texttt{\triplea,\tripleb,\triplec}''. Moreover, you have the result
to the operation : $\sqrt{\triplea^2+\tripleb^2}=\triplec$.

A file could have any structure. In particular, it could have one or
several title lines. It's the case for the file \texttt{pupils.dat}
(page~\pageref{file-pupils}) where the first line is obviously a title
line. You have just to extract this or these lines to obtain a
``classical'' database.\par\medskip

\begin{SideBySideExample}
  \NewList{Class}
  \ReadFileList{Class}{pupils.dat}
  \ExtractFirstItem{Class}{NULL}
  \GetRandomItem{Class}{pupil}
  \ReadFieldItem{\pupil}{0}{Name}
  \ReadFieldItem{\pupil}{1}{FName}
  \ReadFieldItem{\pupil}{2}{Note}
  Your child \Name{} \FName{} has
  \Note{} this term. This is
  \if A\Note very \fi
  \if C\Note not \fi
  good.
\end{SideBySideExample}
\par\medskip\noindent
Processing this way, you have got a database with real datas (no title
data).

The file could be in another format than CSV. In fact, you can define
a field separator (comma by default) and a string delimiter (double
quote by default) which allow to put a field separator inside a
field.
%The syntax isn't very rich. Unlike real CSV file, there isn't
%the syntax \texttt{""} to obtain a single \texttt{"} (or with an other
%field separator).
To indicate other symbols than comma and double
quote, the command \cmd{\ReadFileList} accept an optional argument
which declare the field separator and the string encloser by two
characters.

For instance, the file \texttt{comets.dat} (see page~
\pageref{file-comets}) has ``\texttt{\string|}'' as field
separator. Therefore, the calling syntax becomes:\par\medskip

\begingroup
\catcode`\|=12
\begin{SideBySideExample}
  \NewList{Comets}
  \ReadFileList[|"]{Comets}{comets.dat}
  \ExtractFirstItem{Comets}{NULL}
  \ExtractFirstItem{Comets}{NULL}
  \GetRandomItem{Comets}{comet}
  \ReadFieldItem{\comet}{1}{Name}
  \ReadFieldItem{\comet}{2}{Discover}
  \ReadFieldItem{\comet}{3}{Year}
  \ReadFieldItem{\comet}{4}{Period}
  The comet \Name{} was discovered by
  \Discover{} in \Year{}.
  \unless\ifx\Period\empty
    Its period is \Period{} years.
  \fi
\end{SideBySideExample}
\endgroup
\par\medskip

Observe that we have two ``title lines'' to discard. As each line
begins by a field separator, the first field of each record is
empty. Thus we extract field starting one (not zero). We test if a
period is empty because of 18D/Perrine-Mrkos comet.

\subsection{Tricks, things, and other matters}
\subsubsection{Random number}
In the package \package{randomlist}, the (pseudo) random numbers are
processed by the macro \Cmd{\RLuniformdeviate}\texttt{\{<n>\}\{<macro>\}}
(choose a random integer number between 0 and $\text{\key{n}}-1$ and
store it in |\<macro>|) and \Cmd{\RLsetrandomseed} (set the seed).

When you say nothing, the seed is calculated with the current date
(year, month, day, hour and minute). That is, if you run
\texttt{latex} twice with a delay greater than one minute, you will
have two different results. Sometime, it's what you want, sometime
it's annoying.

Under \LaTeX{}, you can set the seed with the package option
\texttt{seed} with the syntax:
\begin{Verbatim}[xrightmargin=0pt,xleftmargin=0.5\linewidth]
  \usepackage[seed=<value>]{randomlist}
\end{Verbatim}
where \texttt{<value>} is an integer value. If \texttt{<value>} is
zero then the seed is calculated using actual time, year, month and
day.

Under \TeX{} you have to use the command \Cmd{\RLsetrandomseed} to
give the seed value. As for the option, with a zero value, the seed is
calculated using actual time, year, month and day. The syntax is
simple:
\begin{Verbatim}[xrightmargin=0pt,xleftmargin=0.5\linewidth]
  \RLsetrandomseed{<value>}
\end{Verbatim}
Of course, this command is available under \LaTeX{}.

\subsubsection{Loop}
For complex material with several databases, it could be useful to use
external loop such \cmd{\foreach} from \package{pgffor} package or
\cmd{\multido} from \package{multido} package. In fact, this is a good
idea but not the best! These commands (\cmd{\foreach} and
\cmd{\multido}) work inside a group at each loop. With
\package{randomlist} that doesn't work everytime since lists are
restored at each loop. For example, you can't extract element of a
list.

It is possible to read the list with \cmd{\Get...Item}. In this case,
you should probably use the command \Cmd{\CountList} to know the size
of a list. This command takes two arguments: the list name and a macro
to store the number of elements. As usual you give only the name of
the macro (without the backslash). The big difference between extract
and get is that with random reading, you can avoid to have twice (or
more) the same element.

A real example is too long to be inserted here. We give only the code
source. The file \texttt{.tex} and the result \texttt{.pdf} are part
of the package distribution. In this example, we use two databases:
one for the pupils and the other one for the pythagorean triples. As
we read randomly the pythagorean triples and as we won't the same test
for two pupils, then we don't use the external loop described in the
latter paragraph.

\VerbatimInput[gobble=0, frame=none, numbers=left, xleftmargin=0pt,
               xrightmargin=0pt]{test.tex}

Be careful! When we extract triples inside the loop, we must be sure
that there is more triples than pupils elsewhere an error about an
empty list is raised.

Lines 15, 16, 17 read the data bases and extract the title line from
\texttt{pupils.dat}. After that, we enter in the main loop (lines~18
to~60).

At the beginning of the loop, we read the fields for the pupil (name,
first name and note) and the three fields of the pythagorean triple
(lines~20 to~26). It's here that we extract randomly a triple. Since
it's an extraction, another pupil will have another triple.

Lines 27 to 39 typeset the test and lines 40 to 59 typeset the answer
to the test. We test the note of the pupil to decide the type of
exercise: Pythagorean theorem to find the hypotenuse (easy) or
Pythagorean theorem to find a side (less easy).

\subsubsection{Internal}
You can access directly to a list. It's not recommended but\dots

If the list name is \texttt{LName} (pay attention to the letter case),
then the length of the list is \cmd{\LName-len} and the $n$th element
of the list (starting from zero) is \cmd{\LName-$n$}. When an element
is a record with several fields, those fields are inside braces. For
example the first element of list \texttt{Triples} (see last example)
is: \cmd{\Triples-0}~$=$~\texttt{\{119\}\{120\}\{169\}}. As you can
see, inside a list, the characters for separator field and for string
delimiter don't exist.

The authors don't see any situation where knowing internal is
important. If some users have good idea about it then writing to the
authors will be an appreciate initiative!

\section{\LaTeX{} Lists}
Package \package{randomlist} offers two other special commands which
allow to build random lists.

The first one is \Cmd{\RandomItemizeList} which build an itemize list
with random placement of items. Each item is a group.\par\medskip

\RLsetrandomseed{1}
\begin{SideBySideExample}
  \LaTeX{} is:
  \RandomItemizeList
    {magical}
    {logical}
    {practical}
    {clinical}
    {cynical}
\end{SideBySideExample}
\par\medskip

The second command is for enumerate list. It is
\Cmd{\RandomEnumerateList} and it acts like the previous
one:\par\medskip

\RLsetrandomseed{2}
\begin{SideBySideExample}
  \LaTeX{} is:
  \RandomEnumerateList
    {magical}
    {logical}
    {practical}
    {clinical}
    {cynical}
\end{SideBySideExample}
\par\medskip


\newpage
\newgeometry{left=4cm, right=2cm, top=2cm, bottom=2cm}
\section{Package \package{randomlist} code}
\DocInput{randomlist.dtx}
\newpage
\newgeometry{left=2cm, right=2cm, top=2cm, bottom=2cm}
\PrintChanges
\PrintIndex
\newpage
\appendix
\fvset{gobble=0,frame=none,numbers=left,xleftmargin=0pt,xrightmargin=0pt}
\section{File \texttt{pythagoras.dat}}
\label{file-pythagoras}
This file contains Pythagorean triples which have three digits. There
isn't all these triple. In fact the triple are built with the famous
formula $(u^2-v^2,2uv,u^2+v^2)$ with $u$ and $v$ positive integers
such $u>v$. Here are only the hundred first triples with three
digits.

\VerbatimInput{pythagoras.dat}
\newpage
\section{File \texttt{pupils.dat}}
\label{file-pupils}
This file shows a first line which isn't a data line.

\VerbatimInput{pupils.dat}
\newpage
\section{File \texttt{comets.dat}}
\label{file-comets}
This file use lines which aren't data lines and weird separator.

\VerbatimInput{comets.dat}
\end{document}
%</driver>
% \fi
%
% \changes{v0.1}{2013/05/03}{%
%   First release
% }
% \changes{v0.1a}{2013/05/06}{%
%   Fix bug about braces.
% }
% \changes{v0.1b}{2013/05/10}{%
%   Add \cmd{\initrandomlist}.
% }
% \changes{v1.0}{2015/12/27}{%
%   randomlist completly rewritten.
% }
% \changes{v1.1}{2016/01/16}{%
%   Tons of improvements due to Christian Tellechea.
% }
% \changes{v1.2}{2016/07/13}{%
%   First public release.
% }
% \changes{v1.3}{2017/09/11}{%
%   Random operations are \global.
% }
% \CheckSum{0}
% \iffalse
%<*latex>
% \fi
% \subsection{\LaTeX's wrapper}
% \subsubsection{Introduction}
% We start with release number and date.
%    \begin{macrocode}
\NeedsTeXFormat{LaTeX2e}[1995/06/01]
\ProvidesPackage{randomlist}
         [2017/09/11 v1.3 Package for random list (JCC, CT)]
%    \end{macrocode}
% \LaTeX's wrapper has the possibility to use option. There is only
% one option: the seed one. It requires the \package{(x)keyval} package.
%    \begin{macrocode}
\RequirePackage{xkeyval}
\DeclareOptionX{seed}{\gdef\RL@seed{#1}}
\ExecuteOptions{seed=0}
\ProcessOptionsX
%    \end{macrocode}
% We can now call the real randomlist code!
%    \begin{macrocode}
%%
%% This is file `randomlist.tex',
%% generated with the docstrip utility.
%%
%% The original source files were:
%%
%% randomlist.dtx  (with options: `tex')
%% 
%% randomlist.dtx
%% Copyleft 2013-2017 J.-C. Charpentier & C. Tellechea
%% Packages `randomlist' to use with (La)eTeX
%% Copyleft (L) 2013-2017 Jean-C\^ome Charpentier & Christian Tellechea.
%% \CharacterTable
%%  {Upper-case    \A\B\C\D\E\F\G\H\I\J\K\L\M\N\O\P\Q\R\S\T\U\V\W\X\Y\Z
%%   Lower-case    \a\b\c\d\e\f\g\h\i\j\k\l\m\n\o\p\q\r\s\t\u\v\w\x\y\z
%%   Digits        \0\1\2\3\4\5\6\7\8\9
%%   Exclamation   \!     Double quote  \"     Hash (number) \#
%%   Dollar        \$     Percent       \%     Ampersand     \&
%%   Acute accent  \'     Left paren    \(     Right paren   \)
%%   Asterisk      \*     Plus          \+     Comma         \,
%%   Minus         \-     Point         \.     Solidus       \/
%%   Colon         \:     Semicolon     \;     Less than     \<
%%   Equals        \=     Greater than  \>     Question mark \?
%%   Commercial at \@     Left bracket  \[     Backslash     \\
%%   Right bracket \]     Circumflex    \^     Underscore    \_
%%   Grave accent  \`     Left brace    \{     Vertical bar  \|
%%   Right brace   \}     Tilde         \~}
%%
\csname RandomListLoaded\endcsname
\let\RandomListLoaded\endinput
\edef\RLAtCatcode{\the\catcode`\@}
\catcode`\@=11
\ifx\@ifnextchar\@undefined
  \long\def\@ifnextchar#1#2#3{%
    \let\reserved@d=#1%
    \def\reserved@a{#2}%
    \def\reserved@b{#3}%
    \futurelet\@let@token\@ifnch}
  \def\@ifnch{%
    \ifx\@let@token\@sptoken
      \let\reserved@c\@xifnch
    \else
      \ifx\@let@token\reserved@d
        \let\reserved@c\reserved@a
      \else
        \let\reserved@c\reserved@b
      \fi
    \fi
    \reserved@c}
  \def\:{\let\@sptoken= } \:  %
  \def\:{\@xifnch} \expandafter\def\: {\futurelet\@let@token\@ifnch}
\fi
\ifx\PackageError\@undefined
  \long\def\@firstoftwo#1#2{#1}
  \long\def\@secondoftwo#1#2{#2}
  \def\@nnil{\@nil}%
  \alloc@7\write\chardef\sixt@@n\@unused
  \def\typeout#1{\immediate\write\@unused{#1}}%
  \def\@spaces{\space\space\space\space}
  \def\PackageError#1#2#3{%
    \begingroup
    \newlinechar`\^^J
    \edef\RL@temp{#3}%
    \expandafter\errhelp\expandafter{\RL@temp}%
    \typeout{%
      #1 error. \space See User's Manual for further information.^^J
      \@spaces\@spaces\@spaces\@spaces
      Type \space H <return> \space for immediate help.}%
    \errmessage{#2}%
    \endgroup
  }
\fi
\ifx\numexpr\@undefined
    \begingroup
    \newlinechar`\^^J
    \errhelp{Run under etex, pdftex, xetex, luatex, ... but not under
      tex}%
    \typeout{%
      randomlist error. \space See User's Manual for further information.^^J
      \@spaces\@spaces\@spaces\@spaces
      Type \space H <return> \space for immediate help.}%
    \errmessage{You can't use randomlist under tex without etex extension.}%
    \endgroup
\fi
\ifx\@gobble\@undefined
  \long\def\@gobble#1{}
\fi
\def\RL@addtomacro#1#2{\expandafter\def\expandafter#1\expandafter{#1#2}}
\def\RL@ifempty#1{%
  \ifcat\relax\detokenize{#1}\relax
    \expandafter\@firstoftwo
  \else
    \expandafter\@secondoftwo
  \fi
}
\newcount\RL@random
\newcount\RL@random@a
\newcount\RL@random@b
\def\RLsetrandomseed#1{%
  \ifnum#1=0
    \global\RL@random \numexpr \time + \year * \month * \day \relax
  \else
    \global\RL@random \numexpr \ifnum#1<0 -\fi#1 \relax
  \fi
}
\ifx\RL@seed\@undefined
  \RLsetrandomseed{0}
\else
  \RLsetrandomseed{\RL@seed}
\fi
\def\RL@nextrand{%
  \global\RL@random@a=\RL@random
  \global\divide\RL@random@a 127773
  \global\RL@random@b=\RL@random@a
  \global\multiply\RL@random@a -2836
  \global\multiply\RL@random@b -127773
  \global\advance\RL@random\RL@random@b
  \global\multiply\RL@random 16807
  \global\advance\RL@random\RL@random@a
  \ifnum\RL@random<0
    \global\advance\RL@random 2147483647
  \fi
}
\def\RLuniformdeviate#1#2{%
  \RL@nextrand
  \global\RL@random@a=\RL@random
  \global\RL@random@b=\RL@random
  \global\divide\RL@random@a \numexpr#1\relax
  \global\RL@random@b \numexpr\RL@random@b - \RL@random@a * (#1)\relax
  \expandafter\edef\csname #2\endcsname{\number\RL@random@b}%
}%
\newtoks\@ListOfList
\def\@ifIsList#1{%
  \expandafter\@ifIsList@\expandafter{#1}%
}
\def\@ifIsList@#1{%
  \def\@@ifIsList##1#1\sep##2\@@ifIsList{%
    \csname @\ifx\empty##2\empty second\else first\fi oftwo\endcsname
  }%
  \expandafter\@@ifIsList\the\@ListOfList#1\sep\@@ifIsList
}
\def\RL@lenof#1{\csname #1-len\endcsname}
\newif\if@EmptyListFound
\def\@ifIsListNotEmpty#1{%
  \global\@EmptyListFoundfalse
  \@ifIsList{#1}{%
    \ifnum\RL@lenof{#1}=0
      \global\@EmptyListFoundtrue
      \expandafter\@secondoftwo
    \else
      \expandafter\@firstoftwo
    \fi
  }%
  \@secondoftwo
}
\def\@NoListError#1{%
  \if@EmptyListFound
    \@EmptyListError{#1}%
    \global\@EmptyListFoundfalse
  \else
    \PackageError{randomlist}%
                 {List #1 doesn't exist}%
                 {Maybe you mistyped the list name?}%
  \fi
}
\def\@EmptyListError#1{%
  \if@EmptyListFound
  \PackageError{randomlist}%
               {List #1 is empty}%
               {Ask yourself why this list is empty.}%
}
\def\@OutOfRangeError#1#2{%
  \PackageError{randomlist}%
               {Index #2 is greater than last index of list #1}%
               {There aren't enough elements in the list.}%
}
\long\def\RL@nameldef#1{%
  \long\expandafter\def\csname #1\endcsname
}
\long\def\RL@nameledef#1{%
  \long\expandafter\edef\csname #1\endcsname
}
\long\def\RL@namelgdef#1{%
  \long\expandafter\gdef\csname #1\endcsname
}
\long\def\RL@namelxdef#1{%
  \long\expandafter\xdef\csname #1\endcsname
}
\def\RL@let#1#2{%
  \expandafter\let\csname#1\expandafter\endcsname\csname#2\endcsname
}
\long\def\RL@doafterfi#1\fi{\fi#1}
\def\RLfor#1=#2to#3\do{%
  \edef#1{\number\numexpr#2}%
  \edef\RL@sgncomp{\ifnum#1<\numexpr#3\relax>+\else<-\fi}%
  \expandafter\RLfor@i
  \csname RLfor@ii@\string#1\expandafter\endcsname\expandafter
  {\number\numexpr#3\expandafter}%
  \RL@sgncomp
  #1%
}
\long\def\RLfor@i#1#2#3#4#5#6{%
  \def#1{%
    \unless\ifnum#5#3#2\relax
    \RL@doafterfi{%
      #6%
      \edef#5{\number\numexpr#5#41\relax}%
      #1%
    }%
    \fi
  }%
  #1%
}
\def\NewList#1{%
  \@ifIsList{#1}{%
    \PackageError{randomlist}%
                 {List #1 already exists}%
                 {Use \string\ClearList.}%
  }%
  {%
    \ifcsname #1\endcsname
      \PackageError{randomlist}%
                   {Command \csname#1\endcsname already exists}%
                   {Creating list #1 defines a \csname#1\endcsname command.}%
    \else
      \RL@nameldef{#1-len}{0}%
      \@ListOfList\expandafter{\the\@ListOfList#1\sep}%
      \expandafter\def\csname #1\endcsname[##1]{%
        \ifnum##1>\csname#1-len\endcsname
          \relax
        \else
          \csname #1-##1\endcsname
        \fi
      }%
    \fi
  }%
}
\def\ClearList#1{%
  \@ifIsList{#1}{%
     \RL@nameldef{#1-len}{0}%
  }%
  {\@NoListError{#1}}%
}
\def\CopyList#1#2{%
  \@ifIsList{#1}{%
    \@ifIsList{#2}{%
      \RL@let{#2-len}{#1-len}%
      \ifnum\RL@lenof{#1}>0
        \RLfor\RL@iter=0 to \RL@lenof{#1}-1 \do{%
          \RL@let{#2-\RL@iter}{#1-\RL@iter}%
        }%
      \fi
    }%
    {\@NoListError{#2}}%
  }%
  {\@NoListError{#1}}%
}
\def\InsertList#1#2#3{%
  \@ifIsList{#1}{%
    \@ifIsList{#3}{%
      \ifnum #2>\RL@lenof{#1}
        \@OutOfRangeError{#1}{#2}%
      \else
        \ShiftList{#1}{#2}{\RL@lenof{#3}}%
        \ifnum\RL@lenof{#3}>0
          \RLfor\RL@iter=0 to \RL@lenof{#3}-1 \do{%
            \RL@let{#1-\number\numexpr\RL@iter+#2}{#3-\RL@iter}%
          }%
        \fi
      \fi
    }%
    {\@NoListError{#3}}%
  }%
  {\@NoListError{#1}}%
}
\newcount\RL@counti
\newcount\RL@countii
\newcount\RL@countiii
\def\ShowList#1{%
  \@ifIsList{#1}{%
    \ifhmode\par\noindent\fi
    \begingroup
      \ifdefined\ttfamily\ttfamily\else\tt\fi
      BEGIN\detokenize{{#1}}
      (\ifcase\RL@lenof{#1}
        empty list%
      \or
        1 element%
      \else
        \RL@lenof{#1} elements%
      \fi)\par
      \ifnum\RL@lenof{#1}>0
        \parindent=1em
        \RLfor\RL@iter=0 to \RL@lenof{#1}-1 \do {%
          #1[\RL@iter] = \expandafter\RL@meaning\csname
          #1-\RL@iter\endcsname
          \par
        }%
      \fi
      \noindent
      END\detokenize{{#1}}\par
    \endgroup
  }%
  {\@NoListError{#1}}%
}
\def\RL@meaning#1{\expandafter\RL@meaningi\meaning#1}
\expandafter\def\expandafter\RL@meaningi\expandafter#\expandafter1\string>{}
\def\CountList#1#2{%
  \@ifIsList{#1}%
  {\RL@nameledef{#2}{\RL@lenof{#1}}}%
  {\@NoListError{#1}}%
}
\def\ShiftList#1#2#3{%
  \@ifIsList{#1}%
  {%
    \unless\ifnum#3=0
      \ifnum\numexpr#2<0
        \PackageError{randomlist}%
                     {Negative index number}%
                     {Index must be equal or greater than 0}%
      \else
      \ifnum\numexpr#2>\RL@lenof{#1}\relax
        \PackageError{randomlist}%
                     {Index \number\numexpr #2\relax\space too big
                       (<=\RL@lenof{#1})}%
                     {Index must be equal or smaller than length of
                       the list}%
      \else
      \ifnum\numexpr#3>0
        \RLfor\RL@iter = \RL@lenof{#1} to #2 \do{%
          \RL@let{#1-\number\numexpr\RL@iter+#3}{#1-\RL@iter}%
        }%
        \RLfor\RL@iter = #2 to #2 + #3 - 1 \do{%
          \RL@nameldef{#1-\RL@iter}{}%
        }%
      \else
        \ifnum-#3>\numexpr#2\relax
          \PackageError{randomlist}%
                       {Negative shift to big}%
                       {When negative, shift must not be greater than index}%
        \else
          \RLfor\RL@iter=#2 to \RL@lenof{#1} \do{%
            \RL@let{#1-\number\numexpr\RL@iter+#3}{#1-\RL@iter}%
          }%
        \fi
      \fi
      \RL@nameledef{#1-len}{\number\numexpr\RL@lenof{#1} + #3}%
    \fi\fi\fi
  }%
  {\@NoListError{#1}}%
}
\long\def\InsertLastItem#1#2{%
  \@ifIsList{#1}
  {%
    \RL@nameldef{#1-\RL@lenof{#1}}{#2}%
    \RL@nameledef{#1-len}{\number\numexpr\RL@lenof{#1}+1}%
  }
  {\@NoListError{#1}}%
}
\long\def\InsertFirstItem#1#2{%
  \InsertItem{#1}{0}{#2}%
}
\long\def\InsertItem#1#2#3{%
  \@ifIsList{#1}%
  {%
    \ShiftList{#1}{#2}{1}%
    \RL@nameldef{#1-#2}{#3}%
  }%
  {\@NoListError{#1}}%
}
\long\def\InsertRandomItem#1#2{%
  \@ifIsList{#1}%
  {%
    \RLuniformdeviate{\RL@lenof{#1}+1}{RL@temp}%
    \InsertItem{#1}{\RL@temp}{#2}%
  }%
  {\@NoListError{#1}}%
}
\def\ExtractFirstItem#1#2{%
  \@ifIsList{#1}%
  {%
    \ExtractItem{#1}{0}{#2}%
  }%
  {\@NoListError{#1}}%
}
\def\ExtractLastItem#1#2{%
  \@ifIsListNotEmpty{#1}%
  {%
    \RL@let{#2}{#1-\number\numexpr\RL@lenof{#1}-1}%
    \RL@nameledef{#1-len}{\number\numexpr\RL@lenof{#1}-1}%
  }%
  {\@NoListError{#1}}%
}
\def\ExtractItem#1#2#3{%
  \@ifIsListNotEmpty{#1}%
  {%
    \RL@let{#3}{#1-#2}%
    \ShiftList{#1}{#2+1}{-1}%
  }%
  {\@NoListError{#1}}%
}
\def\ExtractRandomItem#1#2{%
  \@ifIsListNotEmpty{#1}%
  {%
    \RLuniformdeviate{\RL@lenof{#1}}{RL@temp}%
    \ExtractItem{#1}{\RL@temp}{#2}%
  }%
  {\@NoListError{#1}}%
}
\def\ExtractList#1#2#3#4{%
  \@ifIsList{#1}{%
    \@ifIsList{#4}{%
      \ifnum#2<\RL@lenof{#1}%
        \ifnum#3<\RL@lenof{#1}%
          \ifnum#2>#3\relax
            \RL@nameldef{#4-len}{0}%
          \else
            \RLfor\RL@iterextract=0 to #3 - #2 \do{%
              \RL@let{#4-\RL@iterextract}{#1-#2}%
              \ExtractItem{#1}{#2}{RL@temp}%
            }%
            \RL@nameledef{#4-len}{\number\numexpr #3 - #2 + 1}%
          \fi
        \else
          \@OutOfRangeError{#1}{#3}%
        \fi
      \else
        \@OutOfRangeError{#1}{#2}%
      \fi
    }%
    {\@NoListError{#4}}%
  }%
  {\@NoListError{#1}}%
}
\def\GetFirstItem#1#2{%
  \GetItem{#1}{0}{#2}%
}
\def\GetLastItem#1#2{%
  \GetItem{#1}{\number\numexpr\RL@lenof{#1}-1}{#2}%
}
\def\GetItem#1#2#3{%
  \@ifIsListNotEmpty{#1}
  {%
    \ifnum\numexpr\RL@lenof{#1}-1-#2<0
      \@OutOfRangeError{#1}{#2}%
    \else
      \RL@let{#3}{#1-#2}%
    \fi
  }
  {\@NoListError{#1}}%
}
\def\GetRandomItem#1#2{%
  \@ifIsListNotEmpty{#1}%
  {%
    \RLuniformdeviate{\RL@lenof{#1}}{RL@temp}%
    \GetItem{#1}{\RL@temp}{#2}%
  }%
  {\@NoListError{#1}}%
}
\def\GetList#1#2#3#4{%
  \@ifIsList{#1}{%
    \@ifIsList{#4}{%
      \ifnum#2<\RL@lenof{#1}%
        \ifnum#3<\RL@lenof{#1}%
          \ifnum#2>#3\relax
            \RL@nameldef{#4-len}{0}%
          \else
            \RLfor\RL@iter=#2 to #3 \do{%
              \RL@let{#4-\number\numexpr \RL@iter - #2}{#1-\RL@iter}%
            }%
            \RL@nameledef{#4-len}{\number\numexpr #3 - #2 + 1}%
          \fi
        \else
          \@OutOfRangeError{#1}{#3}%
        \fi
      \else
        \@OutOfRangeError{#1}{#2}%
      \fi
    }%
    {\@NoListError{#4}}%
  }%
  {\@NoListError{#1}}%
}
\long\def\SetFirstItem#1#2{%
  \SetItem{#1}{0}{#2}%
}
\long\def\SetLastItem#1#2{%
  \SetItem{#1}{\number\numexpr\RL@lenof{#1}-1}{#2}%
}
\long\def\SetItem#1#2#3{%
  \@ifIsListNotEmpty{#1}%
  {%
    \ifnum\numexpr\RL@lenof{#1}-1-#2<0
      \@OutOfRangeError{#1}{#2}%
    \else
      \RL@nameldef{#1-#2}{#3}%
    \fi
  }%
  {\@NoListError{#1}}%%
}
\long\def\SetRandomItem#1#2{%
  \@ifIsListNotEmpty{#1}%
  {%
    \RLuniformdeviate{\RL@lenof{#1}}{RL@temp}%
    \SetItem{#1}{\RL@temp}{#2}%
  }%
  {\@NoListError{#1}}%
}
\def\SetList#1#2{%
  \@ifIsList{#1}%
  {%
    \ClearList{#1}%
    \def\RL@name{#1}%
    \RL@setlist#2,\@nil,%
  }%
  {\@NoListError{#1}}%
}
\long\def\RL@setlist#1,{%
  \def\RL@arg{#1}%
  \unless\ifx\RL@arg\@nnil
    \InsertLastItem{\RL@name}{#1}%
    \expandafter\RL@setlist
  \fi
}
\long\def\ForEachFirstItem#1#2#3{%
  \begingroup
    \RLfor \RL@var = 0 to \RL@lenof{#1}-1 \do{%
      \ExtractFirstItem{#1}{#2}%
      #3%
    }%
  \endgroup
}
\long\def\ForEachLastItem#1#2#3{%
  \begingroup
    \RLfor \RL@var = 0 to \RL@lenof{#1}-1 \do{%
      \ExtractLastItem{#1}{#2}%
      #3%
    }%
  \endgroup
}
\long\def\ForEachRandomItem#1#2#3{%
  \begingroup
    \RLfor \RL@var = 0 to \RL@lenof{#1}-1 \do{%
      \ExtractRandomItem{#1}{#2}%
      #3%
    }%
  \endgroup
}
\long\def\ReadFieldItem#1#2#3{%
  \RL@counti #2\relax
  \expandafter\RL@ReadFieldItem#1\@nil
  \expandafter\let\csname#3\endcsname\RL@temp
}
\long\def\RL@ReadFieldItem{%
  \@ifnextchar\bgroup{\RL@@ReadFieldItem}{\RL@@ReadFieldItemError}%
}
\long\def\RL@@ReadFieldItem#1{%
  \ifnum\RL@counti=\z@
    \def\RL@temp{#1}%
    \expandafter\RL@@ReadFieldItemEnd
  \else
    \advance\RL@counti \m@ne
    \expandafter\RL@ReadFieldItem
  \fi
}
\long\def\RL@@ReadFieldItemEnd#1\@nil{}
\long\def\RL@@ReadFieldItemError#1\@nil{%
  \PackageError{randomlist}%
               {There aren't enough fields in the record}%
               {Pay attention that field number starts from zero.}%
}
\def\RL@SetDelimiters#1#2#3\@nil{%
  \def\RL@markstrings##1{%
    \let\RL@accu\empty
    \expandafter\RL@markstrings@i##1#2\@nil#2%
    \let##1=\RL@accu
  }%
  \def\RL@markstrings@i##1#2##2#2{%
    \RL@addtomacro\RL@accu{##1}%
    \def\RL@current{##2}%
    \unless\ifx\@nnil\RL@current
      \RL@addtomacro\RL@accu{\RL@string{##2}}%
      \expandafter\RL@markstrings@i
    \fi
  }%
  \def\RL@unmarkstrings##1{%
    \let\RL@accuA\empty
    \expandafter\RL@unmarkstrings@i##1\RL@string\@nil
    \let##1=\RL@accuA
  }%
  \def\RL@unmarkstrings@i##1\RL@string##2{%
    \RL@addtomacro\RL@accuA{##1}%
    \def\RL@current{##2}%
    \unless\ifx\@nnil\RL@current
      \RL@ifempty{##2}%
        {\RL@addtomacro\RL@accuA{#2}}%
        {\RL@addtomacro\RL@accuA{##2}}%
      \expandafter\RL@unmarkstrings@i
    \fi
  }%
  \def\RL@parsefields##1{%
    \let\RL@accu\empty
    \expandafter\RL@parsefields@i##1#1\@nil#1%
    \let##1=\RL@accu
  }%
  \def\RL@parsefields@i##1#1{%
    \def\RL@current{##1}%
    \unless\ifx\@nnil\RL@current
      \RL@unmarkstrings\RL@current
      \RL@removefirstspaces\RL@current
      \RL@removelastspaces \RL@current
      \expandafter\RL@addtomacro\expandafter\RL@accu\expandafter
        {\expandafter{\RL@current}}%
      \expandafter\RL@parsefields@i
    \fi
  }%
}
\newread\RL@hdle
\def\@ppar{\par}
\def\ReadFileList{\@ifnextchar[{\@ReadFileList}{\@ReadFileList[,"]}}
\def\@ReadFileList[#1]#2#3{%
  \openin \RL@hdle = #3
  \ifeof\RL@hdle
    \PackageError{randomlist}%
      {File #3 doesn't exist}%
      {Verify its name, its extension, its location, its permissions.}%
  \else
    \RL@ifempty{#1}%
      {%
        \PackageError{randomlist}
          {Optional argument empty: [,"] inserted}
          {Do not leave an optional argument empty}%
          \RL@SetDelimiters,"\@nil
      }
      {\RL@SetDelimiters#1"\@nil}%
    \loop
      \read\RL@hdle to \RL@buffer
      \unless\ifx\RL@buffer\@ppar
        \RL@markstrings\RL@buffer
        \RL@parsefields\RL@buffer
        \def\RL@accuA{\InsertLastItem{#2}}%
        \expandafter\RL@accuA\expandafter{\RL@buffer}%
      \fi
      \ifeof\RL@hdle\else
    \repeat
  \fi
}
\def\RL@ifspacefirst#1{%
  \RL@ifspacefirst@i#1A \@nil
}
\expandafter\def\expandafter\RL@ifspacefirst@i
  \expandafter#\expandafter1\space#2\@nil{%
    \RL@ifempty{#1}%
}
\def\RL@removefirstspaces#1{%
  \expandafter\RL@ifspacefirst\expandafter{#1}
    {\expandafter\removefistspace@i#1\@nil#1}
    {}%
}
\expandafter\def\expandafter\removefistspace@i\space#1\@nil#2{%
  \def#2{#1}%
  \RL@removefirstspaces#2%
}
\edef\RL@restorecatcodezero{\catcode0=\number\catcode0\relax}
\catcode0=12
\def\RL@removelastspaces#1{%
  \expandafter\def\expandafter#1\expandafter{%
    \romannumeral\expandafter
    \RL@removelastspaces@i\expandafter\relax#1^^00 ^^00\@nil
  }%
}
\def\RL@removelastspaces@i#1 ^^00{\RL@removelastspaces@ii#1^^00}
\def\RL@removelastspaces@ii#1^^00#2\@nil{%
  \RL@ifspacefirst{#2}
    {\RL@removelastspaces@i#1^^00 ^^00\@nil}
    {\expandafter\z@\@gobble#1}%
}
\RL@restorecatcodezero
\catcode`\@=\RLAtCatcode\relax
\endinput
%%
%% End of file `randomlist.tex'.

%    \end{macrocode}
% \subsubsection{\LaTeX{} lists}
% Obviously, \LaTeX{} lists are useful only with \LaTeX{}!
% \begin{macro}{\RandomItemizeList}
% Build an itemize list with random placement of items.
%    \begin{macrocode}
\NewList{*RandomList*}
\def\RandomItemizeList{%
  \def\RL@Type{itemize}%
  \ClearList{*RandomList*}%
  \@ifnextchar\bgroup{\@randomlist}{\@@randomlist}%
}
\long\def\@randomlist#1{%
  \InsertRandomItem{*RandomList*}{#1}%
  \@ifnextchar\bgroup{\@randomlist}{\@@randomlist}%
}
\def\@@randomlist{%
  \long\edef\RL@body{\noexpand\begin{\RL@Type}}%
  \RLfor \RL@var = 0 to \RL@lenof{*RandomList*}-1 \do{%
    \long\edef\RL@body{%
      \unexpanded\expandafter{\RL@body}%
      \unexpanded\expandafter{%
        \expandafter\item \csname *RandomList*-\RL@var\endcsname
      }%
    }%
  }%
  \long\edef\RL@body{\unexpanded\expandafter{\RL@body}\noexpand\end{\RL@Type}}%
  \RL@body
}
%    \end{macrocode}
% \end{macro}
% \begin{macro}{\RandomEnumerateList}
% Like |randomitemize| but for enumerate list.
%    \begin{macrocode}
\newcommand*\RandomEnumerateList{%
  \def\RL@Type{enumerate}
  \ClearList{*RandomList*}%
  \@ifnextchar\bgroup{\@randomlist}{\@@randomlist}
}
%    \end{macrocode}
% \end{macro}
% That's all for the \LaTeX's wrapper!
% \iffalse
%</latex>
%<*tex>
% \fi
% \subsection{\TeX{} code}
% At the beginning, we have to deal with multiple call and |@|'s
% catcode.
%    \begin{macrocode}
\csname RandomListLoaded\endcsname
\let\RandomListLoaded\endinput
\edef\RLAtCatcode{\the\catcode`\@}
\catcode`\@=11
%    \end{macrocode}
% If we aren't under \LaTeX{} then we need some \LaTeX{}
% commands. It's just a copy of \LaTeX{}$2\epsilon$ code.
%    \begin{macrocode}
\ifx\@ifnextchar\@undefined
%    \end{macrocode}
% Definition of |\@ifnextchar|.
%    \begin{macrocode}
  \long\def\@ifnextchar#1#2#3{%
    \let\reserved@d=#1%
    \def\reserved@a{#2}%
    \def\reserved@b{#3}%
    \futurelet\@let@token\@ifnch}
  \def\@ifnch{%
    \ifx\@let@token\@sptoken
      \let\reserved@c\@xifnch
    \else
      \ifx\@let@token\reserved@d
        \let\reserved@c\reserved@a
      \else
        \let\reserved@c\reserved@b
      \fi
    \fi
    \reserved@c}
  \def\:{\let\@sptoken= } \:  %
  \def\:{\@xifnch} \expandafter\def\: {\futurelet\@let@token\@ifnch}
\fi
%    \end{macrocode}
% Definition of |\PackageError| and some \LaTeX{} functions when
% running under \TeX{}.
%    \begin{macrocode}
\ifx\PackageError\@undefined
  \long\def\@firstoftwo#1#2{#1}
  \long\def\@secondoftwo#1#2{#2}
  \def\@nnil{\@nil}%
  \alloc@7\write\chardef\sixt@@n\@unused
  \def\typeout#1{\immediate\write\@unused{#1}}%
  \def\@spaces{\space\space\space\space}
  \def\PackageError#1#2#3{%
    \begingroup
    \newlinechar`\^^J
    \edef\RL@temp{#3}%
    \expandafter\errhelp\expandafter{\RL@temp}%
    \typeout{%
      #1 error. \space See User's Manual for further information.^^J
      \@spaces\@spaces\@spaces\@spaces
      Type \space H <return> \space for immediate help.}%
    \errmessage{#2}%
    \endgroup
  }
\fi
%    \end{macrocode}
% We check if we work with an engine which contain at least
% e\TeX{}.
%    \begin{macrocode}
\ifx\numexpr\@undefined
    \begingroup
    \newlinechar`\^^J
    \errhelp{Run under etex, pdftex, xetex, luatex, ... but not under
      tex}%
    \typeout{%
      randomlist error. \space See User's Manual for further information.^^J
      \@spaces\@spaces\@spaces\@spaces
      Type \space H <return> \space for immediate help.}%
    \errmessage{You can't use randomlist under tex without etex extension.}%
    \endgroup
\fi
%    \end{macrocode}
% \begin{macro}{\@gobble}
% Redefine |\@gobble| if needed.
%    \begin{macrocode}
\ifx\@gobble\@undefined
  \long\def\@gobble#1{}
\fi
%    \end{macrocode}
% \end{macro}
% \begin{macro}{\RL@addtomacro}
% We needs to add some code to some macros sometimes.
%    \begin{macrocode}
\def\RL@addtomacro#1#2{\expandafter\def\expandafter#1\expandafter{#1#2}}
%    \end{macrocode}
% \end{macro}
% \begin{macro}{\RL@ifempty}
% Test if something is empty. Execute code according to the answer.
%    \begin{macrocode}
\def\RL@ifempty#1{%
  \ifcat\relax\detokenize{#1}\relax
    \expandafter\@firstoftwo
  \else
    \expandafter\@secondoftwo
  \fi
}
%    \end{macrocode}
% \end{macro}
% PDF\LaTeX, and X$_\mathrm{E}$\TeX know the primitives
% |\pdfsetrandomseed| and |\pdfuniformedeviate| but lua\TeX{} didn't
% know those primitives. Then we have to process ``by hand''!
% \begin{macro}{\RLsetrandomseed}
% We define the macro which give the initial seed. If the argument is
% zero, the value is a mixture of actual time, day, month and year. If
% the argument is nonzero, we process a new randomseed.
%
% The actual random number is stored in |\RL@random|.
%    \begin{macrocode}
\newcount\RL@random
\newcount\RL@random@a
\newcount\RL@random@b
\def\RLsetrandomseed#1{%
  \ifnum#1=0
    \global\RL@random \numexpr \time + \year * \month * \day \relax
  \else
    \global\RL@random \numexpr \ifnum#1<0 -\fi#1 \relax
  \fi
}
%    \end{macrocode}
% If |\RL@seed| exists -- that is, if we run under \LaTeX{} -- we process
% the seed (option of \LaTeX{} package). Otherwise, we use the zero
% value.
%    \begin{macrocode}
\ifx\RL@seed\@undefined
  \RLsetrandomseed{0}
\else
  \RLsetrandomseed{\RL@seed}
\fi
%    \end{macrocode}
% \end{macro}
% \begin{macro}{\RL@nextrand}
% Process the next random number using Linear Congruentiel Generator
% with Shrage's metthod.
%    \begin{macrocode}
\def\RL@nextrand{%
%    \end{macrocode}
% Use the LCG with :
% \[x_{n+1} = 7^5 \times x_n \pmod{2^{31}-1}.\]
% For that we take:
% \begin{itemize}
% \item $7^5=16807$;
% \item $2^{31}-1 = 2147483647$;
% \item $q = \mathrm{E}\left(\frac{2^{31}-1}{7^5}\right)=127773$;
% \item $r = 2^{31}-1 \pmod{7^5} = 2836$.
% \end{itemize}
% Then:
% \[x_{n+1} = 7^5(x_n \pmod{q}) - r\times\mathrm{E}\left(\frac{x_n}{q}\right).\]
% If $x_{n+1} < 0$ then $x_{n+1} = x_{n+1} + 2^{31}-1$
%    \begin{macrocode}
  \global\RL@random@a=\RL@random
  \global\divide\RL@random@a 127773
  \global\RL@random@b=\RL@random@a
  \global\multiply\RL@random@a -2836
  \global\multiply\RL@random@b -127773
  \global\advance\RL@random\RL@random@b
  \global\multiply\RL@random 16807
  \global\advance\RL@random\RL@random@a
%    \end{macrocode}
% If random number is negative add $2^{31}-1$.
%    \begin{macrocode}
  \ifnum\RL@random<0
    \global\advance\RL@random 2147483647
  \fi
}
%    \end{macrocode}
% \end{macro}
% \begin{macro}{\RLuniformdeviate}
% Use |\RL@nextrand| to calculate a random integer between 0
% (inclusive) and |#1| (exclusive). Store the result in macro |#2|.
%    \begin{macrocode}
\def\RLuniformdeviate#1#2{%
%    \end{macrocode}
% Compute the next random number |\RL@random|.
%    \begin{macrocode}
  \RL@nextrand
%    \end{macrocode}
% Compute $|\RL@random| \pmod{\mathtt{\string#1}}$.
%    \begin{macrocode}
  \global\RL@random@a=\RL@random
  \global\RL@random@b=\RL@random
  \global\divide\RL@random@a \numexpr#1\relax
  \global\RL@random@b \numexpr\RL@random@b - \RL@random@a * (#1)\relax
  \expandafter\edef\csname #2\endcsname{\number\RL@random@b}%
}%
%    \end{macrocode}
% \end{macro}
% \subsubsection{Introduction and first commands}
% \begin{macro}{\@ifIsList}
% Test if a list exists and then executes true code or false code. For
% that the list of list names is stored inside the token register
% |\@ListOfList|. Each name is separed to the next one by a ``|\sep|''
% markup.
%    \begin{macrocode}
\newtoks\@ListOfList
%    \end{macrocode}
% |\@ifIsList| test if the list |#1| exists. If yes then it executes the
% next argument else it executes the third argument. Test must be
% executed on an expanded argument.
%    \begin{macrocode}
\def\@ifIsList#1{%
  \expandafter\@ifIsList@\expandafter{#1}%
}
\def\@ifIsList@#1{%
  \def\@@ifIsList##1#1\sep##2\@@ifIsList{%
    \csname @\ifx\empty##2\empty second\else first\fi oftwo\endcsname
  }%
  \expandafter\@@ifIsList\the\@ListOfList#1\sep\@@ifIsList
}
%    \end{macrocode}
% \end{macro}
% \begin{macro}{\RL@lenof}
% Shortcut allowing to get the len of a list
%    \begin{macrocode}
\def\RL@lenof#1{\csname #1-len\endcsname}
%    \end{macrocode}
% \end{macro}
% \begin{macro}{\@ifIsListNotEmpty}
% Test if a list exist and isn't empty. If double yes then it executes
% the second argument else it executes the third one.
%    \begin{macrocode}
\newif\if@EmptyListFound
\def\@ifIsListNotEmpty#1{%
  \global\@EmptyListFoundfalse
  \@ifIsList{#1}{%
    \ifnum\RL@lenof{#1}=0
      \global\@EmptyListFoundtrue
      \expandafter\@secondoftwo
    \else
      \expandafter\@firstoftwo
    \fi
  }%
  \@secondoftwo
}
%    \end{macrocode}
% \end{macro}
% \begin{macro}{\@NoListError}
% Error for an unexisting list or an empty list.
%    \begin{macrocode}
\def\@NoListError#1{%
  \if@EmptyListFound
    \@EmptyListError{#1}%
    \global\@EmptyListFoundfalse
  \else
    \PackageError{randomlist}%
                 {List #1 doesn't exist}%
                 {Maybe you mistyped the list name?}%
  \fi
}
%    \end{macrocode}
% \end{macro}
% \begin{macro}{\@EmptyListError}
% Error for an empty list.
%    \begin{macrocode}
\def\@EmptyListError#1{%
  \if@EmptyListFound
  \PackageError{randomlist}%
               {List #1 is empty}%
               {Ask yourself why this list is empty.}%
}
%    \end{macrocode}
% \end{macro}
% \begin{macro}{\@OutOfRangeError}
% Error for index out of range.
%    \begin{macrocode}
\def\@OutOfRangeError#1#2{%
  \PackageError{randomlist}%
               {Index #2 is greater than last index of list #1}%
               {There aren't enough elements in the list.}%
}
%    \end{macrocode}
% \end{macro}
% \begin{macro}{\RL@nameldef}
% |\long| version of |\@namedef|.
%    \begin{macrocode}
\long\def\RL@nameldef#1{%
  \long\expandafter\def\csname #1\endcsname
}
%    \end{macrocode}
% \end{macro}
% \begin{macro}{\RL@nameledef}
% All the macros in \package{randomlist} are long ones. It's
% useless for now since there isn't argument but it's a precaution
% for the future.
%
% |\long| version of |\@nameedef|.
%    \begin{macrocode}
\long\def\RL@nameledef#1{%
  \long\expandafter\edef\csname #1\endcsname
}
%    \end{macrocode}
% \end{macro}
% \begin{macro}{\RL@namelgdef}
% |\long| version of |\@namegdef|.
%    \begin{macrocode}
\long\def\RL@namelgdef#1{%
  \long\expandafter\gdef\csname #1\endcsname
}
%    \end{macrocode}
% \end{macro}
% \begin{macro}{\RL@namelxdef}
% |\long| version of |\@namexdef|.
%    \begin{macrocode}
\long\def\RL@namelxdef#1{%
  \long\expandafter\xdef\csname #1\endcsname
}
%    \end{macrocode}
% \end{macro}
% \begin{macro}{\RL@let}
% |\let| between two macros (with just the names)
%    \begin{macrocode}
\def\RL@let#1#2{%
  \expandafter\let\csname#1\expandafter\endcsname\csname#2\endcsname
}
%    \end{macrocode}
% \end{macro}
% \begin{macro}{\RLfor}
% Loop without group. Syntax is |\RLfor<var>=<begin>to<end>\do|
%    \begin{macrocode}
\long\def\RL@doafterfi#1\fi{\fi#1}
\def\RLfor#1=#2to#3\do{%
%    \end{macrocode}
% Set the variable.
%    \begin{macrocode}
  \edef#1{\number\numexpr#2}%
%    \end{macrocode}
% Set |\RL@sgncomp| to |<+| or |>-| if variable is greater or less
% than end
%    \begin{macrocode}
  \edef\RL@sgncomp{\ifnum#1<\numexpr#3\relax>+\else<-\fi}%
%    \end{macrocode}
% Call auxiliary macro with five parameters:
%    \begin{macrocode}
  \expandafter\RLfor@i
%    \end{macrocode}
% First argument (sub-recursive macro name build with the <variable>
% name).
%    \begin{macrocode}
  \csname RLfor@ii@\string#1\expandafter\endcsname\expandafter
%    \end{macrocode}
% Second argument (max).
%    \begin{macrocode}
  {\number\numexpr#3\expandafter}%
%    \end{macrocode}
% Third and fourth arguments since |\RL@sgncomp| is ``|<+|'' or ``|>-|''.
%    \begin{macrocode}
  \RL@sgncomp
%    \end{macrocode}
% fifth argument (variable name).
%    \begin{macrocode}
  #1%
}
%    \end{macrocode}
% Auxiliary macro:
% \begin{itemize}
% \item |#1| recursive macro name (like |\RLfor@ii@<var>|;
% \item |#2| max integer;
% \item |#3| ``|<|'' or ``|>|'';
% \item |#4| ``|+|'' or ``|-|'' (incrementation or decrementation);
% \item |#5| variable name;
% \item |#6| code to execute.
% \end{itemize}
%    \begin{macrocode}
\long\def\RLfor@i#1#2#3#4#5#6{%
%    \end{macrocode}
% Define the recursive submacro.
%    \begin{macrocode}
  \def#1{%
%    \end{macrocode}
% While <var> isn't greater than max
%    \begin{macrocode}
    \unless\ifnum#5#3#2\relax
%    \end{macrocode}
% In order to have a tail recursion.
%    \begin{macrocode}
    \RL@doafterfi{%
%    \end{macrocode}
% Execute the loop code.
%    \begin{macrocode}
      #6%
%    \end{macrocode}
% Increment <variable> by one.
%    \begin{macrocode}
      \edef#5{\number\numexpr#5#41\relax}%
%    \end{macrocode}
% And repeat.
%    \begin{macrocode}
      #1%
    }%
    \fi
  }%
%    \end{macrocode}
% submacro recursive call.
%    \begin{macrocode}
  #1%
}
%    \end{macrocode}
% \end{macro}
% \subsubsection{General list commands}
% \begin{macro}{\NewList}
% The main structure is the list. A list |L| is a collection of macros
% |L-<n>| where |<n>| is an index (starting from zero) and a macro
% |L-len| which store the len of the list, i.e. the last index plus
% one.
%
% When a new list is created, its name is stored in |@ListOfList|. A
% macro is also created for accessing data.
%    \begin{macrocode}
\def\NewList#1{%
  \@ifIsList{#1}{%
%    \end{macrocode}
% If a list with the same name exists then raise an error.
%    \begin{macrocode}
    \PackageError{randomlist}%
                 {List #1 already exists}%
                 {Use \string\ClearList.}%
  }%
  {%
%    \end{macrocode}
% When a list |MyName| is created, the macros |\MyName| and
% |\MyName-len| are created and there will be macros |\MyName-<n>| to
% store data. Then \package{randomlist} prohibit the name |MyName| for
% a list if the macro |\MyName| already exists.
%    \begin{macrocode}
    \ifcsname #1\endcsname
      \PackageError{randomlist}%
                   {Command \csname#1\endcsname already exists}%
                   {Creating list #1 defines a \csname#1\endcsname command.}%
    \else
%    \end{macrocode}
% If everything is fine, create the len macro which store the len of
% the list (starting with 0);
%    \begin{macrocode}
      \RL@nameldef{#1-len}{0}%
%    \end{macrocode}
% append the list name to the list of names |\@ListOfList|;
%    \begin{macrocode}
      \@ListOfList\expandafter{\the\@ListOfList#1\sep}%
%    \end{macrocode}
% and create the |\Mylist[<index>]| macro.
%    \begin{macrocode}
      \expandafter\def\csname #1\endcsname[##1]{%
%    \end{macrocode}
% If index is to big, the macro is |\relax| (a sort of undefined
% without error).
%    \begin{macrocode}
        \ifnum##1>\csname#1-len\endcsname
          \relax
        \else
          \csname #1-##1\endcsname
        \fi
      }%
    \fi
  }%
}
%    \end{macrocode}
% \end{macro}
% \begin{macro}{\ClearList}
% |\ClearList| erases a list. It sets the length to zero. There is no
% need to erase all the |\Mylist-<index>| macros.
%    \begin{macrocode}
\def\ClearList#1{%
  \@ifIsList{#1}{%
%    \end{macrocode}
% Clear the list if it exists.
%    \begin{macrocode}
     \RL@nameldef{#1-len}{0}%
  }%
  {\@NoListError{#1}}%
}
%    \end{macrocode}
% \end{macro}
% \begin{macro}{\CopyList}
% Copy list |#1| in list |#2|.
%    \begin{macrocode}
\def\CopyList#1#2{%
  \@ifIsList{#1}{%
    \@ifIsList{#2}{%
      \RL@let{#2-len}{#1-len}%
      \ifnum\RL@lenof{#1}>0
        \RLfor\RL@iter=0 to \RL@lenof{#1}-1 \do{%
          \RL@let{#2-\RL@iter}{#1-\RL@iter}%
        }%
      \fi
    }%
    {\@NoListError{#2}}%
  }%
  {\@NoListError{#1}}%
}
%    \end{macrocode}
% \end{macro}
% \begin{macro}{\InsertList}
% Insert List |#3| to the list |#1| starting at index |#2|. 
%    \begin{macrocode}
\def\InsertList#1#2#3{%
  \@ifIsList{#1}{%
    \@ifIsList{#3}{%
      \ifnum #2>\RL@lenof{#1}
        \@OutOfRangeError{#1}{#2}%
      \else
        \ShiftList{#1}{#2}{\RL@lenof{#3}}%
        \ifnum\RL@lenof{#3}>0
          \RLfor\RL@iter=0 to \RL@lenof{#3}-1 \do{%
            \RL@let{#1-\number\numexpr\RL@iter+#2}{#3-\RL@iter}%
          }%
        \fi
      \fi
    }%
    {\@NoListError{#3}}%
  }%
  {\@NoListError{#1}}%
}
%    \end{macrocode}
% \end{macro}
% \begin{macro}{\ShowList}
% Macro for debugging purpose. First we declare some scratch count
% registers.
%    \begin{macrocode}
\newcount\RL@counti
\newcount\RL@countii
\newcount\RL@countiii
\def\ShowList#1{%
%    \end{macrocode}
% We show a list only if this list exists!
%    \begin{macrocode}
  \@ifIsList{#1}{%
    \ifhmode\par\noindent\fi
%    \end{macrocode}
% Typeset |BEGIN{MyList}|. As we typeset braces, we have to use
% ttfamily, then put the material inside a group.
%    \begin{macrocode}
    \begingroup
      \ifdefined\ttfamily\ttfamily\else\tt\fi
      BEGIN\detokenize{{#1}}
%    \end{macrocode}
% Typeset the number of elements.
%    \begin{macrocode}
      (\ifcase\RL@lenof{#1}
        empty list%
      \or
        1 element%
      \else
        \RL@lenof{#1} elements%
      \fi)\par
%    \end{macrocode}
% Loop to typeset element one after one.
%    \begin{macrocode}
      \ifnum\RL@lenof{#1}>0
        \parindent=1em
        \RLfor\RL@iter=0 to \RL@lenof{#1}-1 \do {%
          #1[\RL@iter] = \expandafter\RL@meaning\csname
          #1-\RL@iter\endcsname
          \par
        }%
      \fi
%    \end{macrocode}
% Typeset |END{MyList}|.
%    \begin{macrocode}
      \noindent
      END\detokenize{{#1}}\par
    \endgroup
  }%
  {\@NoListError{#1}}%
}
%    \end{macrocode}
% \end{macro}
% \begin{macro}{\RL@meaning}
% Like \TeX{} primitive |\meaning| without prefix |(\long) macro:->|:
%    \begin{macrocode}
\def\RL@meaning#1{\expandafter\RL@meaningi\meaning#1}
\expandafter\def\expandafter\RL@meaningi\expandafter#\expandafter1\string>{}
%    \end{macrocode}
% \end{macro}
% \begin{macro}{\CountList}
% Count the number of elements in the list |#1|. Store it in |#2|.
%    \begin{macrocode}
\def\CountList#1#2{%
  \@ifIsList{#1}%
  {\RL@nameledef{#2}{\RL@lenof{#1}}}%
  {\@NoListError{#1}}%
}
%    \end{macrocode}
% \end{macro}
% \subsubsection{Writing and reading list commands}
% \begin{macro}{\ShiftList}
% Shift the elements of a list left or right. The syntax is:
%
% |\ShiftList{list name}{start}{shift}|
%
% where |start| is the first index to shift and |shift| the number of
% shifting. If |shift| is positive, it is a right shift. If |shitf| is
% negative, it is a left shift.
%    \begin{macrocode}
\def\ShiftList#1#2#3{%
  \@ifIsList{#1}%
  {%
%    \end{macrocode}
% No action if |shift| is zero!
%    \begin{macrocode}
    \unless\ifnum#3=0
%    \end{macrocode}
% If |<start>| is negative, raise an error.
%    \begin{macrocode}
      \ifnum\numexpr#2<0
        \PackageError{randomlist}%
                     {Negative index number}%
                     {Index must be equal or greater than 0}%
      \else
%    \end{macrocode}
% If |<start>| is greater than the lists length, raise an error.
%    \begin{macrocode}
      \ifnum\numexpr#2>\RL@lenof{#1}\relax
        \PackageError{randomlist}%
                     {Index \number\numexpr #2\relax\space too big
                       (<=\RL@lenof{#1})}%
                     {Index must be equal or smaller than length of
                       the list}%
      \else
%    \end{macrocode}
% Here we have $0 \leq |<start>| \leq |len|(|<list>|)$.
%
% If |<shift>| is positive, we process a right shifting: it's alway
% possible.
%    \begin{macrocode}
      \ifnum\numexpr#3>0
        \RLfor\RL@iter = \RL@lenof{#1} to #2 \do{%
          \RL@let{#1-\number\numexpr\RL@iter+#3}{#1-\RL@iter}%
        }%
%    \end{macrocode}
% Empty the items out of shift part.
%    \begin{macrocode}
        \RLfor\RL@iter = #2 to #2 + #3 - 1 \do{%
          \RL@nameldef{#1-\RL@iter}{}%
        }%
      \else
        \ifnum-#3>\numexpr#2\relax
%    \end{macrocode}
% If the negative shifting is to big for index |#2| then raise an error.
%    \begin{macrocode}
          \PackageError{randomlist}%
                       {Negative shift to big}%
                       {When negative, shift must not be greater than index}%
        \else
%    \end{macrocode}
% Elsewhere, process the left shifting.
%    \begin{macrocode}
          \RLfor\RL@iter=#2 to \RL@lenof{#1} \do{%
            \RL@let{#1-\number\numexpr\RL@iter+#3}{#1-\RL@iter}%
          }%
        \fi
      \fi
%    \end{macrocode}
% Set the list length for both positive and negative shifting.
%    \begin{macrocode}
      \RL@nameledef{#1-len}{\number\numexpr\RL@lenof{#1} + #3}%
    \fi\fi\fi
  }%
  {\@NoListError{#1}}%
}
%    \end{macrocode}
% \end{macro}
% \begin{macro}{\InsertLastItem}
% Add an element |#2| at the end of the list |#1|.
%    \begin{macrocode}
\long\def\InsertLastItem#1#2{%
  \@ifIsList{#1}
  {%
    \RL@nameldef{#1-\RL@lenof{#1}}{#2}%
    \RL@nameledef{#1-len}{\number\numexpr\RL@lenof{#1}+1}%
  }
  {\@NoListError{#1}}%
}
%    \end{macrocode}
% \end{macro}
% \begin{macro}{\InsertFirstItem}
% Add an element |#2| at the beginning of the list |#1|. For that,
% shift right all the element and then put |#3| at |L[0]|.
%    \begin{macrocode}
\long\def\InsertFirstItem#1#2{%
  \InsertItem{#1}{0}{#2}%
}
%    \end{macrocode}
% \end{macro}
% \begin{macro}{\InsertItem}
% Add an element |#3| at the position |#2| of the list |#1|. For that,
% pass from |L[0]| to |L[#2-1]| then shift right from |L[#2]| to
% |L[len]| and finally put |#3| at |L[#2]|. To do this, we must have
% |#2| $\geq$ |L-len|.
%    \begin{macrocode}
\long\def\InsertItem#1#2#3{%
  \@ifIsList{#1}%
  {%
    \ShiftList{#1}{#2}{1}%
    \RL@nameldef{#1-#2}{#3}%
  }%
  {\@NoListError{#1}}%
}
%    \end{macrocode}
% \end{macro}
% \begin{macro}{\InsertRandomItem}
% Insert element |#2| in a random position of list |#1|.
%    \begin{macrocode}
\long\def\InsertRandomItem#1#2{%
  \@ifIsList{#1}%
  {%
    \RLuniformdeviate{\RL@lenof{#1}+1}{RL@temp}%
    \InsertItem{#1}{\RL@temp}{#2}%
  }%
  {\@NoListError{#1}}%
}
%    \end{macrocode}
% \end{macro}
% \begin{macro}{\ExtractFirstItem}
% Extract the first element of list |#1| and store it in |#2|.
%    \begin{macrocode}
\def\ExtractFirstItem#1#2{%
  \@ifIsList{#1}%
  {%
    \ExtractItem{#1}{0}{#2}%
  }%
  {\@NoListError{#1}}%
}
%    \end{macrocode}
% \end{macro}
% \begin{macro}{\ExtractLastItem}
% Extract the last element of list |#1| and store it in |#2|.
%    \begin{macrocode}
\def\ExtractLastItem#1#2{%
  \@ifIsListNotEmpty{#1}%
  {%
    \RL@let{#2}{#1-\number\numexpr\RL@lenof{#1}-1}%
    \RL@nameledef{#1-len}{\number\numexpr\RL@lenof{#1}-1}%
  }%
  {\@NoListError{#1}}%
}
%    \end{macrocode}
% \end{macro}
% \begin{macro}{\ExtractItem}
% Extract the element at the position |#2| of the list |#1| and store
% it in |#3|.
%    \begin{macrocode}
\def\ExtractItem#1#2#3{%
  \@ifIsListNotEmpty{#1}%
  {%
    \RL@let{#3}{#1-#2}%
    \ShiftList{#1}{#2+1}{-1}%
  }%
  {\@NoListError{#1}}%
}
%    \end{macrocode}
% \end{macro}
% \begin{macro}{\ExtractRandomItem}
% Extract element in a random position of list |#1| and store it in |#2|.
%    \begin{macrocode}
\def\ExtractRandomItem#1#2{%
  \@ifIsListNotEmpty{#1}%
  {%
    \RLuniformdeviate{\RL@lenof{#1}}{RL@temp}%
    \ExtractItem{#1}{\RL@temp}{#2}%
  }%
  {\@NoListError{#1}}%
}
%    \end{macrocode}
% \end{macro}
% \begin{macro}{\ExtractList}
% |ExtractList| extract a list from a list. There are four arguments:
% \begin{itemize}
% \item |#1| is the list from which the extraction is made;
% \item |#2| is the starting index of extraction;
% \item |#3| is the ending index of extraction;
% \item |#4| is the list which receive the extracted list.
% \end{itemize}
%    \begin{macrocode}
\def\ExtractList#1#2#3#4{%
%    \end{macrocode}
% In order to do something, |#1| and |#2| must be lists, and indexes
% |#2| and |#3| must be inside list |#1|.
%    \begin{macrocode}
  \@ifIsList{#1}{%
    \@ifIsList{#4}{%
      \ifnum#2<\RL@lenof{#1}%
        \ifnum#3<\RL@lenof{#1}%
          \ifnum#2>#3\relax
%    \end{macrocode}
% If $\text{start} > \text{end}$ we build an empty list.
%    \begin{macrocode}
            \RL@nameldef{#4-len}{0}%
          \else
%    \end{macrocode}
% If $\text{start} \leq \text{end}$ we build a real extracted list. We
% have to be careful because |\ExtractItem| uses the loop variable
% |\RL@iter|. Then we use another loop variable. 
%    \begin{macrocode}
            \RLfor\RL@iterextract=0 to #3 - #2 \do{%
              \RL@let{#4-\RL@iterextract}{#1-#2}%
              \ExtractItem{#1}{#2}{RL@temp}%
            }%
            \RL@nameledef{#4-len}{\number\numexpr #3 - #2 + 1}%
          \fi
        \else
          \@OutOfRangeError{#1}{#3}%
        \fi
      \else
        \@OutOfRangeError{#1}{#2}%
      \fi
    }%
    {\@NoListError{#4}}%
  }%
  {\@NoListError{#1}}%
}
%    \end{macrocode}
% \end{macro}
% \begin{macro}{\GetFirstItem}
% Get the first element of list |#1| and store it in |#2|.
%    \begin{macrocode}
\def\GetFirstItem#1#2{%
  \GetItem{#1}{0}{#2}%
}
%    \end{macrocode}
% \end{macro}
% \begin{macro}{\GetLastItem}
% Get the last element of list |#1| and store it in |#2|.
%    \begin{macrocode}
\def\GetLastItem#1#2{%
  \GetItem{#1}{\number\numexpr\RL@lenof{#1}-1}{#2}%
}
%    \end{macrocode}
% \end{macro}
% \begin{macro}{\GetItem}
% Get the element of rank |#2| of list |#1| and store it in |#3|.
%    \begin{macrocode}
\def\GetItem#1#2#3{%
  \@ifIsListNotEmpty{#1}
  {%
    \ifnum\numexpr\RL@lenof{#1}-1-#2<0
      \@OutOfRangeError{#1}{#2}%
    \else
      \RL@let{#3}{#1-#2}%
    \fi
  }
  {\@NoListError{#1}}%
}
%    \end{macrocode}
% \end{macro}
% \begin{macro}{\GetRandomItem}
% Get element in a random position of list |#1| and store it in |#2|.
%    \begin{macrocode}
\def\GetRandomItem#1#2{%
  \@ifIsListNotEmpty{#1}%
  {%
    \RLuniformdeviate{\RL@lenof{#1}}{RL@temp}%
    \GetItem{#1}{\RL@temp}{#2}%
  }%
  {\@NoListError{#1}}%
}
%    \end{macrocode}
% \end{macro}
% \begin{macro}{\GetList}
% |\GetList| copy a sub-list from a list. There are four arguments:
% \begin{itemize}
% \item |#1| is the list from which the reading is made;
% \item |#2| is the starting index of extraction;
% \item |#3| is the ending index of extraction;
% \item |#4| is the list which receive the readen items.
% \end{itemize}
%    \begin{macrocode}
\def\GetList#1#2#3#4{%
%    \end{macrocode}
% In order to do something, |#1| and |#2| must be lists, and indexes
% |#2| and |#3| must be inside list |#1|.
%    \begin{macrocode}
  \@ifIsList{#1}{%
    \@ifIsList{#4}{%
      \ifnum#2<\RL@lenof{#1}%
        \ifnum#3<\RL@lenof{#1}%
          \ifnum#2>#3\relax
%    \end{macrocode}
% If $\text{start} > \text{end}$ we build an empty list.
%    \begin{macrocode}
            \RL@nameldef{#4-len}{0}%
          \else
%    \end{macrocode}
% If $\text{start} \leq \text{end}$ we build a real extracted list.
%    \begin{macrocode}
            \RLfor\RL@iter=#2 to #3 \do{%
              \RL@let{#4-\number\numexpr \RL@iter - #2}{#1-\RL@iter}%
            }%
            \RL@nameledef{#4-len}{\number\numexpr #3 - #2 + 1}%
          \fi
        \else
          \@OutOfRangeError{#1}{#3}%
        \fi
      \else
        \@OutOfRangeError{#1}{#2}%
      \fi
    }%
    {\@NoListError{#4}}%
  }%
  {\@NoListError{#1}}%
}
%    \end{macrocode}
% \end{macro}
% \begin{macro}{\SetFirstItem}
% Set the first element of list |#1| with value |#2|.
%    \begin{macrocode}
\long\def\SetFirstItem#1#2{%
  \SetItem{#1}{0}{#2}%
}
%    \end{macrocode}
% \end{macro}
% \begin{macro}{\SetLastItem}
% Set the last element of list |#1| with value |#2|.
%    \begin{macrocode}
\long\def\SetLastItem#1#2{%
  \SetItem{#1}{\number\numexpr\RL@lenof{#1}-1}{#2}%
}
%    \end{macrocode}
% \end{macro}
% \begin{macro}{\SetItem}
% Set the |#2| element of list |#1| with value |#3|.
%    \begin{macrocode}
\long\def\SetItem#1#2#3{%
  \@ifIsListNotEmpty{#1}%
  {%
    \ifnum\numexpr\RL@lenof{#1}-1-#2<0
      \@OutOfRangeError{#1}{#2}%
    \else
      \RL@nameldef{#1-#2}{#3}%
    \fi
  }%
  {\@NoListError{#1}}%%
}
%    \end{macrocode}
% \end{macro}
% \begin{macro}{\SetRandomItem}
% Set element in a random position of list |#1| with value |#2|.
%    \begin{macrocode}
\long\def\SetRandomItem#1#2{%
  \@ifIsListNotEmpty{#1}%
  {%
    \RLuniformdeviate{\RL@lenof{#1}}{RL@temp}%
    \SetItem{#1}{\RL@temp}{#2}%
  }%
  {\@NoListError{#1}}%
}
%    \end{macrocode}
% \end{macro}
% \begin{macro}{\SetList}
% |\SetList| allow to give multiple values to a list. This function
% acts like a repetition of |\InsertLastItem|.
%    \begin{macrocode}
\def\SetList#1#2{%
  \@ifIsList{#1}%
  {%
    \ClearList{#1}%
    \def\RL@name{#1}%
    \RL@setlist#2,\@nil,%
  }%
  {\@NoListError{#1}}%
}
\long\def\RL@setlist#1,{%
  \def\RL@arg{#1}%
  \unless\ifx\RL@arg\@nnil
    \InsertLastItem{\RL@name}{#1}%
    \expandafter\RL@setlist
  \fi
}
%    \end{macrocode}
% \end{macro}
% \subsubsection{Loop on list}
% \begin{macro}{\ForEachFirstItem}
% |\ForEachFirstItem| typesets |#3| for each element of the list |#1|
% extracting the actual first element (stored in |#2|).
%    \begin{macrocode}
\long\def\ForEachFirstItem#1#2#3{%
  \begingroup
    \RLfor \RL@var = 0 to \RL@lenof{#1}-1 \do{%
      \ExtractFirstItem{#1}{#2}%
      #3%
    }%
  \endgroup
}
%    \end{macrocode}
% \end{macro}
% \begin{macro}{\ForEachLastItem}
% |\ForEachLastItem| typesets |#3| for each element of the list |#1|
% extracting the actual last element  (stored in |#2|).
%    \begin{macrocode}
\long\def\ForEachLastItem#1#2#3{%
  \begingroup
    \RLfor \RL@var = 0 to \RL@lenof{#1}-1 \do{%
      \ExtractLastItem{#1}{#2}%
      #3%
    }%
  \endgroup
}
%    \end{macrocode}
% \end{macro}
% \begin{macro}{\ForEachRandomItem}
% |\ForEachRandomItem| typesets |#3| for each element of the list |#1|
% extracting randomly an element (stored in |#2|).
%    \begin{macrocode}
\long\def\ForEachRandomItem#1#2#3{%
  \begingroup
    \RLfor \RL@var = 0 to \RL@lenof{#1}-1 \do{%
      \ExtractRandomItem{#1}{#2}%
      #3%
    }%
  \endgroup
}
%    \end{macrocode}
% \end{macro}
% \subsubsection{Database}
% \begin{macro}{\ReadFieldItem}
% Macro |\ReadFieldItem| read a field in a record.
%
% A record is a sequence of groups, each group is a field.
% \begin{itemize}
% \item |#1| is the record (sequence of groups;
% \item |#2| is the index of item (starting at zero);
% \item |#3| is the macro name which store the field.
% \end{itemize}
%    \begin{macrocode}
\long\def\ReadFieldItem#1#2#3{%
%    \end{macrocode}
% Store the field's index.
%    \begin{macrocode}
  \RL@counti #2\relax
%    \end{macrocode}
% Call the recursive macro
%    \begin{macrocode}
  \expandafter\RL@ReadFieldItem#1\@nil
%    \end{macrocode}
% Store the result in macro |\#3|.
%    \begin{macrocode}
  \expandafter\let\csname#3\endcsname\RL@temp
}
%    \end{macrocode}
% In fact the first recusive call check for a left brace. A record
% must contain at least one field otherwise an error message is
% raised.
%    \begin{macrocode}
\long\def\RL@ReadFieldItem{%
  \@ifnextchar\bgroup{\RL@@ReadFieldItem}{\RL@@ReadFieldItemError}%
}
%    \end{macrocode}
% 
%    \begin{macrocode}
\long\def\RL@@ReadFieldItem#1{%
  \ifnum\RL@counti=\z@
    \def\RL@temp{#1}%
    \expandafter\RL@@ReadFieldItemEnd
  \else
    \advance\RL@counti \m@ne
    \expandafter\RL@ReadFieldItem
  \fi
}
\long\def\RL@@ReadFieldItemEnd#1\@nil{}
\long\def\RL@@ReadFieldItemError#1\@nil{%
  \PackageError{randomlist}%
               {There aren't enough fields in the record}%
               {Pay attention that field number starts from zero.}%
}
%    \end{macrocode}
% \end{macro}
% \begin{macro}{\ReadFileList}
% First, we look for special delimiters for fields and strings. By
% default, the delimiter for fields is the comma and the delimiter for
% string is the double quote.
%    \begin{macrocode}
\def\RL@SetDelimiters#1#2#3\@nil{%
%    \end{macrocode}
% \begin{itemize}
% \item argument |#1| is the field separator;
% \item argument |#2| is the string delimiter;
% \item argument |#3| is the remainder to ignore.
% \end{itemize}
%    \begin{macrocode}
  \def\RL@markstrings##1{%
    \let\RL@accu\empty
    \expandafter\RL@markstrings@i##1#2\@nil#2%
    \let##1=\RL@accu
  }%
  \def\RL@markstrings@i##1#2##2#2{%
    \RL@addtomacro\RL@accu{##1}%
    \def\RL@current{##2}%
    \unless\ifx\@nnil\RL@current
      \RL@addtomacro\RL@accu{\RL@string{##2}}%
      \expandafter\RL@markstrings@i
    \fi
  }%
  \def\RL@unmarkstrings##1{%
    \let\RL@accuA\empty
    \expandafter\RL@unmarkstrings@i##1\RL@string\@nil
    \let##1=\RL@accuA
  }%
  \def\RL@unmarkstrings@i##1\RL@string##2{%
    \RL@addtomacro\RL@accuA{##1}%
    \def\RL@current{##2}%
    \unless\ifx\@nnil\RL@current
      \RL@ifempty{##2}%
        {\RL@addtomacro\RL@accuA{#2}}%
        {\RL@addtomacro\RL@accuA{##2}}%
      \expandafter\RL@unmarkstrings@i
    \fi
  }%
  \def\RL@parsefields##1{%
    \let\RL@accu\empty
    \expandafter\RL@parsefields@i##1#1\@nil#1%
    \let##1=\RL@accu
  }%
  \def\RL@parsefields@i##1#1{%
    \def\RL@current{##1}%
    \unless\ifx\@nnil\RL@current
      \RL@unmarkstrings\RL@current
      \RL@removefirstspaces\RL@current
      \RL@removelastspaces \RL@current
      \expandafter\RL@addtomacro\expandafter\RL@accu\expandafter
        {\expandafter{\RL@current}}%
      \expandafter\RL@parsefields@i
    \fi
  }%
}
%    \end{macrocode}
% The macro |\ReadFileList| uses a handle for the reading file. it
% needs also a macro to detect |\par|
%    \begin{macrocode}
\newread\RL@hdle
\def\@ppar{\par}
%    \end{macrocode}
% At first, |\ReadFileList| check for an optionnal argument giving
% delimiters. By default, delimiters are comma for field separator and
% double quote for string delimiter.
%    \begin{macrocode}
\def\ReadFileList{\@ifnextchar[{\@ReadFileList}{\@ReadFileList[,"]}}
%    \end{macrocode}
% \begin{itemize}
% \item |#1| contains the delimiters;
% \item |#2| is the data base name;
% \item |#3| is the file name.
% \end{itemize}
%    \begin{macrocode}
\def\@ReadFileList[#1]#2#3{%
  \openin \RL@hdle = #3
  \ifeof\RL@hdle
    \PackageError{randomlist}%
      {File #3 doesn't exist}%
      {Verify its name, its extension, its location, its permissions.}%
  \else
%    \end{macrocode}
% If the optionnal argument is empty then raise an error and take the
% comma and the double quote instead.
%    \begin{macrocode}
    \RL@ifempty{#1}%
      {%
        \PackageError{randomlist}
          {Optional argument empty: [,"] inserted}
          {Do not leave an optional argument empty}%
          \RL@SetDelimiters,"\@nil
      }
%    \end{macrocode}
% Else add double quote to for security.
%    \begin{macrocode}
      {\RL@SetDelimiters#1"\@nil}%
%    \end{macrocode}
% The main loop read each line of the file. Don't process anything if
% the line is empty (it could be the very end of the file).
%    \begin{macrocode}
    \loop
      \read\RL@hdle to \RL@buffer
      \unless\ifx\RL@buffer\@ppar
%    \end{macrocode}
% Mark the string
%    \begin{macrocode}
        \RL@markstrings\RL@buffer
%    \end{macrocode}
% and process the fields.
%    \begin{macrocode}
        \RL@parsefields\RL@buffer
%    \end{macrocode}
% Save current record with fields, that is, with sequence of groups.
%    \begin{macrocode}
        \def\RL@accuA{\InsertLastItem{#2}}%
        \expandafter\RL@accuA\expandafter{\RL@buffer}%
      \fi
      \ifeof\RL@hdle\else
    \repeat
  \fi
}
%    \end{macrocode}
% Check for a heading space.
%    \begin{macrocode}
\def\RL@ifspacefirst#1{%
  \RL@ifspacefirst@i#1A \@nil
}
\expandafter\def\expandafter\RL@ifspacefirst@i
  \expandafter#\expandafter1\space#2\@nil{%
    \RL@ifempty{#1}%
}
%    \end{macrocode}
%    \begin{macrocode}
% Remove all spaces at the start of argument (macro).
\def\RL@removefirstspaces#1{%
  \expandafter\RL@ifspacefirst\expandafter{#1}
    {\expandafter\removefistspace@i#1\@nil#1}
    {}%
}
\expandafter\def\expandafter\removefistspace@i\space#1\@nil#2{%
  \def#2{#1}%
  \RL@removefirstspaces#2%
}
%    \end{macrocode}
%    \begin{macrocode}
% Store |^^00|'s catcode
\edef\RL@restorecatcodezero{\catcode0=\number\catcode0\relax}
%    \end{macrocode}
% then set this catcode to other catcode.
%    \begin{macrocode}
% puis le modifie à 12.
\catcode0=12
%    \end{macrocode}
% Remove all heading and trailing spaces of argument (macro).
%    \begin{macrocode}
\def\RL@removelastspaces#1{%
  \expandafter\def\expandafter#1\expandafter{%
    \romannumeral\expandafter
    \RL@removelastspaces@i\expandafter\relax#1^^00 ^^00\@nil
  }%
}
\def\RL@removelastspaces@i#1 ^^00{\RL@removelastspaces@ii#1^^00}
\def\RL@removelastspaces@ii#1^^00#2\@nil{%
  \RL@ifspacefirst{#2}
    {\RL@removelastspaces@i#1^^00 ^^00\@nil}
    {\expandafter\z@\@gobble#1}%
}
%    \end{macrocode}
%    \begin{macrocode}
% Restore |^^00|'s catcode.
\RL@restorecatcodezero
%    \end{macrocode}
% \end{macro}
% At the very end of the package, we restore the |@|'s catcode.
%    \begin{macrocode}
\catcode`\@=\RLAtCatcode\relax
%    \end{macrocode}
% \iffalse
%</tex>
% \fi
