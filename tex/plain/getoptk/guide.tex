\documentclass{article}
\usepackage{palatino}
\usepackage[T1]{fontenc}
\usepackage[latin9]{inputenc}
\usepackage[dvipsnames]{xcolor}
\usepackage{hyperref}
\usepackage{graphicx}
\usepackage{microtype}
\hypersetup{%
	hyperindex,
	% pagebackref,
	pdftex,
	pdftitle={Interfaces to macros: the GETOPTK package}
	pdfdisplaydoctitle,
	% pdfpagemode=UseNone,
	breaklinks=true,
	% extension=pdf,
	bookmarks=true,
	% plainpages=false,
	colorlinks,
	linkcolor=BlueViolet,
	citecolor=BrickRed,
	urlcolor=Plum,
	pdfpagelabels,
	pdfpagelayout=TwoPageRight,
	pdfview=Fit,
	pdfstartview=Fit
}%
%
% Definitions
%
\newcommand\PACKAGE{\emph{getoptk}}
\newcommand\cs[1]{\texttt{\textbackslash#1}}
\newcommand\meta[1]{\penalty0 $\langle\textit{#1}\rangle$\penalty0 }
\begin{document}
\author{Michael Le Barbier Gr�newald}
\title{The GETOPTK package}%
\date{6th of June 2011}
\maketitle
\begin{abstract}
The~{\PACKAGE} package eases the definition of macros accepting
optional arguments in the same style as~\cs{hrule} or~\cs{hbox}.
It is meant to be used with \emph{plain {\TeX}}.
\end{abstract}
\section{Introduction}
A flexible way to pass optional arguments to a procedure is to rely on
\emph{dictionaries} of optional arguments, that is, a set of bindings
between formal names of arguments and their values.  Some~{\TeX}
primitives, like~\cs{hrule} or~\cs{hbox}, use such an interface style.
We call this style the~\emph{keyword} interface style.  There is no
facility in~{\TeX} to define new macros using the~\emph{keyword}
interface style.  The {\PACKAGE} package provides such a service.

\section{Quick guide}

In order to define a macro using the~\emph{keyword} interface style,
we have to setup first a~\emph{behaviour dictionary}
binding \emph{keywords} and \emph{behaviours}.  A~keyword introduces an
optional argument and a behaviour describes its effect, we will soon
show an example of this.  In the definition of the macro itself, we
first select the behaviour dictionary we want to use and
call~\cs{getoptk}, the control sequence responsible of the detection
of optional arguments.  In this call, we need to provide
a~\emph{callback} as argument to~\cs{getoptk}, this callback is a macro
taking control of the execution after~\cs{getoptk} has completed its
task.  It will be called with an argument, that is derived from the
list of optional arguments.

For explanatory purposes, let us assume that we want to define a
macro \cs{begindisplay} using the \emph{keyword} interface style and
accepting the following optional arguments:
\begin{description}
\item [ragged]
  Fill, but do not adjust the right margin (only left-justify).

\item [literal]
  Display block with literal font (usually fixed-width).  Useful for
  source code or simple tabbed or spaced text.

\item [file \meta{file name}]
  The file whose name, enclosed in curly braces, follows the
  \emph{file} keyword is read and displayed using the selected
  display type.

\item [offset \meta{dimen}]
  Use~\emph{dimen} as indentation for the display.

\end{description}
We first create a fresh new behaviour dictionary:
\begin{verbatim}
\newgetoptkdictionary{display}
\end{verbatim}
and fill it with behaviours:
\begin{verbatim}
\defgetoptkflag{ragged}{\raggedright}
\defgetoptkflag{literal}{\let\displayfont\literalfont}
\defgetoptktoks{file}{\input #1}
\defgetoptkdimen{dimen}{\displayindent=#1\relax}
\end{verbatim}
Besides registering the behaviours in the dictionary~\emph{display},
these commands also bind the behaviours to the following control
sequences:
\begin{verbatim}
\getoptk@behaviour@display@ragged
\getoptk@behaviour@display@literal
\getoptk@behaviour@display@file
\getoptk@behaviour@display@dimen
\end{verbatim}
The control sequences created with \cs{getoptkflag} must do not have an
argument, while those created by \cs{getoptktoks} or \cs{getoptkdimen}
do have one.  The definition of~\cs{begindisplay} is
\begin{verbatim}
\def\begindisplay{%
  \setgetoptkdictionary{display}%
  \getoptk\display@M
}
\end{verbatim}
The control sequence~\cs{getoptk} is such that the input text
\begin{verbatim}
\begindisplay file {chapter1} literal offset 20pt
\end{verbatim}
is \emph{replaced} by
\begin{verbatim}
\display@M{%
  \getoptk@behaviour@display@file{chapter1}%
  \getoptk@behaviour@display@literal
  \getoptk@behaviour@display@dimen{20pt}%
}
\end{verbatim}
so that \cs{display@M} can do its job and trigger the behaviours at
the appropriate time.


\section{Defining a behaviour dictionary}

The command \cs{newgetoptkdictionary}\meta{dictionary} creates a
behaviour dictionary that will be filled by subsequent calls to
commands binding keywords and behaviours.  The bindings commands are:
\begin{verbatim}
\defgetoptkflag
\defgetoptkcount
\defgetoptkdimen
\defgetoptkskip
\defgetoptktoks
\defgetoptkbracket
\end{verbatim}
They all must be called like in the previous examples, following the
pattern:
\par\medskip
\leftline{\meta{binding\_command}\meta{keyword}\meta{behaviour}}
\smallskip
\begin{description}
\newcommand\defgetoptkdescribe[1]{%
\item [defgetoptk#1]
  The \meta{behaviour} is the replacement text of a macro having one
  argument.  When it finds~\meta{keyword}, the \cs{getoptk} macro
  scans further for an argument that is valid right hand side value
  for a \emph{#1} register.  This argument will be supplied to the
  \meta{behaviour} when behaviours are triggered.\par
}
\item [defgetoptkflag]
  The \meta{behaviour} is the replacement text of a macro without
  arguments.  When it finds~\meta{keyword}, the \cs{getoptk} macro
  does not look for an argument but starts again scanning for
  keywords.
\defgetoptkdescribe{count}
\defgetoptkdescribe{dimen}
\defgetoptkdescribe{skip}
\defgetoptkdescribe{toks}
\item [defgetoptkbracket]
  The \meta{behaviour} is the replacement text of a macro having one
  argument.  When it finds~\meta{keyword}, the \cs{getoptk} macro
  scans further for an optional argument enclosed by square brackets.
  If such an argument is found, it is supplied to
  \meta{behaviour} when behaviours are triggered and the predicate
  \cs{ifgetoptkbracket} is bound to \cs{iftrue}.  If no such an
  argument is found, then the empty argument is supplied
  to~\meta{behaviour} when behaviours are triggered and the predicate
  \cs{ifgetoptkbracket} is bound to \cs{iffalse}.
\end{description}

\section{Licence}
The {\PACKAGE} software id copyright \copyright~2011 Michael Le
Barbier Gr�newald.
The {\PACKAGE} software is distributed under the terms of the
CeCILL-B licence, version~1.0.  See the files COPYING and COPYING-FR
in the distribution.
\end{document}

% Local Variables:
% ispell-dictionary: "en_GB-ise"
% End:
