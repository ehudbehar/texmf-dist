%%%%%%%%%%%%%%%%%%%%%%%%%%%%%%%%%%%%%%%%%%%%%%%%%%%%%%%%%%%%%%%%%%%%%%%%%%%%%%
%
% Module:    ZzTeX Horizontal Mode Facilities
%
% Synopsis:  This module provides various facilities that operate in
%            horizontal mode, within a paragraph.  See also ZZHMODEB.
%
% Author:    Paul C. Anagnostopoulos
% Created:   21 September 1989
%
% Copyright 1989--2020 by Paul C. Anagnostopoulos
% under The MIT License (opensource.org/licenses/MIT)
%
%%%%%%%%%%%%%%%%%%%%%%%%%%%%%%%%%%%%%%%%%%%%%%%%%%%%%%%%%%%%%%%%%%%%%%%%%%%%%%%%

%                       Starting Paragraphs
%                       -------- ----------


% The following token register is executed at the beginning of
% every paragraph.

\definetoks{\everyparagraph}
\everyparagraph = {}

% This is what we have to do at the beginning of every paragraph:

\everypar = {\requiredeverypar \the\everyparagraph}

% The following command ensures that we are in horizontal mode,
% without doing anything if we already are.

\def \ensurepar {\unhbox \voidbox}

% The following command fakes up the start of a paragraph while
% leaving us in vertical mode.

\def \fakepar {%
  \endgraf
  \vspace{\parskip}%
  \requiredeverypar}

%                       Paragraph Penalties
%                       --------- ---------


\definecount{\zclubsv}{-100000}         % Overridden \clubpenalty.
\definecount{\zclubct}{1}               % Overridden paragraph count.


\def \overrideclubpenalty #1{%                          {penalty}
  \if \eqlp{\zclubsv}{-100000}\global\zclubsv = \clubpenalty \fi
  \global\clubpenalty = #1\relax
  \global\zclubct = \if \vmodep -1\else 0\fi \relax}

\declareeverypar{\zclubpar}

\def \zclubpar {%
  \ifnum \zclubct=0
    \global\clubpenalty=\zclubsv
    \global\zclubsv=-100000\relax
  \fi
  \global\advance \zclubct 1\relax}

%                       Single-Character Commands
%                       ---------------- --------


%  Name         Description

%  \<space>     Control space
%  \!           Negative sixth space; negative thin space in math;
%                 callout in code
%  \"           Dieresis (umlaut) accent
%  \#           Pound sign
%  \$           Dollar sign
%  \%           Percent sign
%  \&           Ampersand
%  \'           Acute accent
%  \(           (reserved 
%  \)            for user)
%  \*           \discretionary{}{}{}
%  \+           Hair space
%  \,           Sixth space; thin space in math
%  \-           Discretionary hyphen
%  \.           Dot accent
%  \/           Italic correction
%  \0           \phantom{0}
%  \1 -- \9     (reserved for user)
%  \:           List element separator
%  \;           Third space; thick space in math
%  \<           (reserved for user)
%  \=           Macron accent; prefix for tag namespace
%  \>           (reserved for user)
%  \?           (reserved for user)
%  \@           At-sign
%  \H           Hungarian umlaut accent
%  \L           Polish suppressed L
%  \O           Scandinavian O slash
%  \P           Paragraph sign (pilcrow)
%  \S           Section sign
%  \[           (reserved for user)
%  \\           Backslash
%  \]           (reserved for user)
%  \^           Circumflex accent
%  \_           Underscore
%  \`           Grave accent
%  \b           Bar-under accent
%  \c           Cedilla accent
%  \d           Dot-under accent
%  \g           Ogonek accent
%  \h           Small horizontal space
%  \i           Dotless i
%  \j           Dotless j
%  \l           Polish suppressed l
%  \o           Scandinavian o slash
%  \r           ring accent
%  \t           Tie-after accent
%  \u           Breve accent
%  \v           Hacek accent
%  \x           \
%  \y            } (reserved for user, for index command abbreviations)
%  \z           /
%  \{           Left brace
%  \|           Vertical bar; \Vert in math mode
%  \}           Right brace
%  \~           Tilde accent

%                       Characters
%                       ----------


% These definitions are in alphabetical order.

\def \AA {%                             %~ Ring A (Angstrom). %^named_char
  \ensurepar
  \if \mathmodep
    \mtext{\AA}%
  \else
    \r{A}%
  \fi}

\def \aa {%                             %~ Ring a. %^named_char
  \ensurepar
  \if \mathmodep
    \mtext{\aa}%
  \else
    \r{a}%
  \fi}

\chardef \AE = "1D                      %~ Scandinavian AE.

\chardef \ae = "1A                      %~ Scandinavian ae.

\chardef \ampersand = `\&               %~ Ampersand.

\let \& = \ampersand

\chardef \atsign = `\@                  %~ At sign.

\let \@ = \atsign

\def \backslash {%                      %~ Backslash. %^named_char
  \ensurepar
  \if \mathmodep
    \zmbackslash
  \else \if \monoencodingp
    \char "5C\relax
  \else
    \PSbackslash
  \fi\fi}

\let \\=\backslash

\def \baselinebullet #1{%               {size} %~ Bullet set on baseline.
  \zrequiredoodad{\zdoodadfont}{#1}%
  \hbox{\zdoodadfont .}}

\def \bullet {%                         %~ Bullet. %^named_char
  \ensurepar
  \if \mathmodep
    \zmbullet
  \else
    \PSbullet
  \fi}

\def \cent {%                           %~ Cent sign. %^named_char
  \ensurepar
  \PScent}

\let \cents = \cent

\def \copyright {%                      %~ Copyright symbol. %^named_char
  \ensurepar
  \PScopyright}

\def \cdquote {''}                      %~ Close double quote. %^named_char

\def \dagger {%                         %~ Dagger. %^named_char
  \ensurepar
  \if \mathmodep
    \zmdagger
  \else
    \PSdagger
  \fi}

\def \ddagger {%                        %~ Double dagger. %^named_char
  \ensurepar
  \if \mathmodep
    \zmddagger
  \else
    \PSddagger
  \fi}

\def \dollar {%                         %~ Dollar sign. %^named_char
  \zprehyph{\dollarhyph}%
  \if \monoencodingp
    \char "24\relax
  \else
    \PSdollar
  \fi
  \zposthyph{\dollarhyph}}

\let \$ = \dollar

\def \ellipsis {%                       %~ Text ellipsis.
  \if \mathmodep
    \ldots
  \else
    \unbreakable\thickspace 
    .\unbreakable\thickspace
    .\unbreakable\thickspace
    .\thickspace
    \ignorespaces
  \fi}

\def \emdash {---}                      %~ Em dash. %^named_char

\def \endash {--}                       %~ En dash. %^named_char

\def \greater {%                        %~ Greater-than sign %^named_char
  \ensurepar
  \if \monoencodingp
    \char "3E\relax
  \else
    \PSgreater
  \fi}

\def \>{%                               %~ Greater-than sign. %^named_char
  \relax
  \if \mathmodep \plainerror \else \greater \fi}

\chardef \hash = `\#                    %~ Pound sign, hash, octothorpe.

\let \# = \hash

\def \hat {%                            %~ Hat symbol. %^named_char
  \ensurepar
  \if \mathmodep
    \zmhat
  \else
    \char "5E\relax
  \fi}

\chardef \i = "10                       %~ Dotless i.

\chardef \j = "11                       %~ Dotless j.

\def \L {\PSpolishL}                    %~ Uppercase Polish L. %^named_char

\def \l {\PSpolishl}                    %~ Lowercase Polish l. %^named_char

% \lbrace and \rbrace cannot start with \ensurepar, because that won't
% work following a \left or \right command.  However, the \relax is required.

\def \lbrace {%                         %~ Left brace. %^named_char
  \relax
  \if \mathmodep
    \zmlbrace
  \else \if \monoencodingp
    \char "7B\relax
  \else
    \PSbraceleft
  \fi\fi}

\let \{ = \lbrace

\def \less {%                           %~ Less-than sign. %^named_char
  \ensurepar
  \if \monoencodingp
    \char "3C\relax
  \else
    \PSless
  \fi}

\let \< = \less

\chardef \O = "1F                       %~ Slashed O.

\chardef \o = "1C                       %~ Slashed o.

\chardef \OE = "1E                      %~ OE diphthong.

\chardef \oe = "1B                      %~ oe diphthong.

\def \odquote {``}                      %~ Open double quote. %^named_char

\def \P {%                              %~ Paragraph symbol (pilcrow). %^named_char
  \ensurepar
  \PSparagraph}

\chardef \percent = `\%                 %~ Percent sign.

\let \% = \percent

\def \perthousand {%                    %~ Per thousand symbol. %^named_char
  \ensurepar
  \PSperthousand}

\def \prellipsis {%                     %~ Text ellipsis with no left space.
  .\unbreakable\thickspace
  .\unbreakable\thickspace
  .\thickspace
  \ignorespaces}

\def \rbrace {%                         %~ Right brace. %^named_char
  \relax
  \if \mathmodep
    \zmrbrace
  \else \if \monoencodingp
    \char "7D\relax
  \else
    \PSbraceright
  \fi\fi}

\let \} = \rbrace

\def \registered {%                     %~ Registered symbol. %^named_char
  \ensurepar
  \PSregistered}

\def \S {%                              %~ Section sign. %^named_char
  \ensurepar
  \PSsection}

\def \SS {%                             %~ Double section sign. %^named_char
  \ensurepar
  \PSsection\PSsection}

\chardef \ss = "19                      %~ German eszett.

\def \sterling {%                       %~ Pound sterling symbol. %^named_char
  \ensurepar
  \PSsterling}

\def \tilde {%                          %~ Tilde. %^named_char
  \ensurepar
  \if \mathmodep
    \zmtilde
  \else
    \char "7E\relax
  \fi}

\def \trademark {%                      %~ Trademark symbol. %^named_char
  \ensurepar
  \PStrademark}

\def \underscore {%                     %~ Underscore. %^named_char
  \ensurepar
  \zprehyph{\underscorehyph}%
  \if \monoencodingp
    \char "5F\relax
  \else
    \PSunderscore
  \fi
  \zposthyph{\underscorehyph}}

\let \_ = \underscore

\def \udquote {%                        %~ Undirected double quote. %^named_char
  \ensurepar
  \if \monoencodingp
    \char "22\relax
  \else
    \PSudquote
  \fi}

\def \usquote {%                        %~ Undirected single quote. %^named_char
  \ensurepar
  \if \monoencodingp
    \char "0D\relax
  \else
    \PSusquote
  \fi}

\def \verticalbar {%                    %~ Vertical bar. %^named_char
  \ensurepar
  \if \monoencodingp
    \char "7C\relax
  \else
    \PSbar
  \fi}

\def \|{%                               %~ Vertical bar. %^named_char
  \relax
  \if \mathmodep \Vert \else \verticalbar \fi}

%                       Accents
%                       -------


\def \"#1{{\accent "7F #1}}             %~ Umlaut accent. %^accent

\def \'#1{{\accent 19 #1}}              %~ Acute accent. %^accent

\def \.#1{{\accent 95 #1}}              %~ Dot accent. %^accent

\def \=#1{{\accent 22 #1}}              %~ Macron accent. %^accent

\def \^#1{{\accent 94 #1}}              %~ Circumflex accent. %^accent

\def \`#1{{\accent 18 #1}}              %~ Grave accent. %^accent

\def \~#1{{\accent "7E #1}}             %~ Tilde accent. %^accent

\def \b #1{%                            %~ Bar-under accent. %^accent
  \ozalign{\relax #1\cr
           \hidewidth \vbox to .2ex {\hbox{\char 22}\vss}\hidewidth}}

\def \c #1{%                            %~ Cedilla accent. %^accent
  \setbox \zboxa = \hbox{#1}%
  \if \dimeqlp{\ht\zboxa}{1ex}%
    \accent 24 #1%
  \else
    {\ooalign{\hidewidth \char 24 \hidewidth \cr
              \unhbox \zboxa}}%
  \fi}

\def \d #1{%                            %~ Dot-under accent. %^accent
  \ozalign{\relax #1\cr
           \hidewidth.\hidewidth}}

\def \g #1{%                            %~ Ogonek accent. %^accent
  \setbox \zboxa = \hbox{#1}%
  \if \dimeqlp{\ht\zboxa}{1ex}%
    \accent 154 #1%
  \else
    {\ooalign{\hidewidth \char 154 \hidewidth \cr
              \unhbox \zboxa}}%
  \fi}

\def \H #1{{\accent "7D #1}}            %~ Long Hungarian umlaut accent. %^accent

\def \r #1{{\accent "17 #1}}            %~ Ring accent. %^accent

\def \t #1{%                            %~ Tie-after accent. %^accent
  {\edef \znext {\the\font}\the\textfont1 \accent "7F \znext #1}}

\def \u #1{{\accent 21 #1}}             %~ Breve accent. %^accent

\def \v #1{{\accent 20 #1}}             %~ Hacek accent. %^accent

%                       Spaces
%                       ------


% TeX space primitives include the following:
%
%   <space>     Produce a normal space, obeying \spacefactor.
%   \<space>    Produce a normal space as if \spacefactor = 1000.

% Here are a set of "space" commands that produce breakable space:

%~ Specified amount of breakable horizontal space.

\def \hspace #1{\ensurepar \hskip #1\relax}%            {amount} %^hspace

%~ Specified amount of unbreakable horizontal space
%~ measured in tenths of a point.

\def \h #1;{%                                           tenths-points; %^hspace
  {\tdimena = .1pt
   \tdimena = #1\tdimena
   \kern \tdimena}}

\def \fiveemspace {\ensurepar \hskip 5em\relax}         %~ Breakable 5 em space. %^hspace

\def \fouremspace {\ensurepar \hskip 4em\relax}         %~ Breakable 4 em space. %^hspace

\def \threeemspace {\ensurepar \hskip 3em\relax}        %~ Breakable 3 em space. %^hspace

\def \twoemspace {\ensurepar \hskip 2em\relax}          %~ Breakable 2 em space. %^hspace

\def \emspace {\ensurepar \hskip 1em\relax}             %~ Breakable 1 em space. %^hspace

\def \enspace {\ensurepar \hskip .5em\relax}            %~ Breakable 1 en space. %^hspace

\def \digitspace {%                                     %~ Breakable digit space. %^hspace
  {\measuredigitwidth{\tdimena}%
   \ensurepar \hskip \tdimena}}

{\catcode`\| = \catactive
\gdef \digitspacebar {%
  \catcode`\| = \catactive
  \def |{\digitspace}}
}

\def \thirdspace {\ensurepar \hskip .333em\relax}       %~ Breakable 1/3 em space. %^hspace
\let \thickspace = \thirdspace
\def \negthickspace {\ensurepar \hskip -.333em\relax}   %~ Breakable negative 1/3 em space. %^hspace

\def \space { }                                         %~ Normal space.
\def \^^M{\ }                                           %~ Normal space.

\def \fontspace {\ensurepar \hskip \fontdimen2\font}    %~ Breakable font space. %^hspace

\def \visiblespace {%                                   %~ Visible space. %^hspace
  \ensurepar
   \hbox {%
     \measurespacewidth{\tdimena}%
     \kern .16ex
     \llap{\rule{width .08ex height .15ex depth .2ex}}%
     \lower .2ex \hbox{\rule{width \tdimena height .08ex depth 0pt}}%
     \rlap{\rule{width .08ex height .15ex depth .2ex}}%
     \kern .16ex}}

\def \fourthspace {\ensurepar \hskip .25em\relax}       %~ Breakable 1/4 em space. %^hspace

\def \fifthspace {\ensurepar \hskip .2em\relax}         %~ Breakable 1/5 em space. %^hspace
\let \thinspace = \fifthspace
\def \negfifthspace {\ensurepar \hskip -.2em\relax}     %~ Breakable negative 1/3 em space. %^hspace
\let \negthinspace = \negfifthspace

\def \sixthspace {\ensurepar \hskip .16667em\relax}     %~ Breakable 1/6 em space. %^hspace
\def \negsixthspace {\ensurepar \hskip -.16667em\relax} %~ Breakable negative 1/6 em space. %^hspace

\def \seventhspace {\ensurepar \hskip .14286em\relax}   %~ Breakable 1/7 em space. %^hspace
\let \hairspace = \seventhspace
\def \negseventhspace {\ensurepar \hskip -.14286em\relax} %~ Breakable negative 1/7 em space. %^hspace

\def \qqqqquad {\unbreakable\fiveemspace}               %~ Unbreakable 5 em space. %^hspace
\def \qqqquad {\unbreakable\fouremspace}                %~ Unbreakable 4 em space. %^hspace
\def \qqquad {\unbreakable\threeemspace}                %~ Unbreakable 3 em space. %^hspace
\def \qquad {\unbreakable\twoemspace}                   %~ Unbreakable 2 em space. %^hspace
\def \quad {\unbreakable\emspace}                       %~ Unbreakable 1 em space. %^hspace
\def \nut {\unbreakable\enspace}                        %~ Unbreakable 1 en space. %^hspace

%~ Unbreakable 1/3 em space.

\def \;{\relax\if \mathmodep \mthickspace \else \unbreakable\thickspace \fi} %^hspace

%~ Unbreakable 1/5 em space.

\def \,{\relax\if \mathmodep \mthinspace \else \unbreakable\thinspace \fi}  %^hspace

%~ Unbreakable negative 1/5 em space.

\def \!{\relax\if \mathmodep \mnegthinspace \else \unbreakable\negthinspace\fi}  %^hspace

\def \+{\unbreakable\hairspace}                         %~ Unbreakable 1/7 em space. %^hspace

%~ Specified amount of unbreakable horizontal space
%~ measured in hundredths of an em.

\def \k #1;{%                                   hundredths-em; %^hspace
  {\tdimena = .01em
   \tdimena = #1\tdimena
   \kern \tdimena}}

%~ This command retains horizontal space at the beginning of a line.
%~ Such space is normally discarded.

\def \retain {\vrule height 0pt depth 0pt width 0pt\relax} %^hspace

%~ This declaration causes spaces to be "obeyed," so that every space
%~ in the source is relevent.

\def \obeyspaces {%
  \catcode`\ =\catactive}

{\obeyspaces
\global\let =\ %
}

%                       Space Factors
%                       ----- -------


% IniTeX sets the space factor code of uppercase letters to 999 so
% periods after them will not end sentences.  That just gives people
% one more rule to remember.

  \sfcode`\A=1000 \sfcode`\B=1000 \sfcode`\C=1000 \sfcode`\D=1000
  \sfcode`\E=1000 \sfcode`\F=1000 \sfcode`\G=1000 \sfcode`\H=1000
  \sfcode`\I=1000 \sfcode`\J=1000 \sfcode`\K=1000 \sfcode`\L=1000
  \sfcode`\M=1000 \sfcode`\N=1000 \sfcode`\O=1000 \sfcode`\P=1000
  \sfcode`\Q=1000 \sfcode`\R=1000 \sfcode`\S=1000 \sfcode`\T=1000
  \sfcode`\U=1000 \sfcode`\V=1000 \sfcode`\W=1000 \sfcode`\X=1000
  \sfcode`\Y=1000 \sfcode`\Z=1000

% A few characters should not affect the space factor.

  \sfcode`\'=0
  \sfcode`\)=0
  \sfcode`\]=0


\def \setsentencespacing #1{%                           {flag}
  \if #1%
    \sfcode`\. = 2250
    \sfcode`\? = 2250
    \sfcode`\! = 2250
    \sfcode`\: = 2000
  \else
    \sfcode`\. = 1000
    \sfcode`\? = 1000
    \sfcode`\! = 1000
    \sfcode`\: = 1000
  \fi
  \relax}

%                       Numbers
%                       -------


\let \lcromannumeral = \romannumeral

%~ This command formats an integer as uppercase Roman numerals.

\def \ucromannumeral #1{%                               {number}
  \uppercase\expandafter{\romannumeral#1}}

%~ This command formats an integer as lowercase letters.

\def \lcletter #1{%                                     {number}
  \ifcase #1%
    0\or a\or b\or c\or d\or e\or f\or g\or h\or i\or j\or k\or l\or m\or
         n\or o\or p\or q\or r\or s\or t\or u\or v\or w\or x\or y\or z\else
         \number#1\fi}

%~ This command formats an integer as uppercase letters.

\def \ucletter #1{%                                     {number}
  \ifcase #1%
    0\or A\or B\or C\or D\or E\or F\or G\or H\or I\or J\or K\or L\or M\or
         N\or O\or P\or Q\or R\or S\or T\or U\or V\or W\or X\or Y\or Z\else
         \number#1\fi}

%                       Lines
%                       -----


\long\def \line #1{%                                    {text}
  \if \vmodep
    {\parfillskip = 0pt
     \noindent #1\retain \par}%
  \else
    \retain #1\tl
  \fi}

\long\def \leftline         #1{\line{#1\hfil}}%          {text}
\long\def \rightline        #1{\line{\hfil#1}}%          {text}
\long\def \centerline       #1{\line{\hfil#1\hfil}}%     {text}
\long\def \spreadline       #1#2{\line{{#1}\hfil{#2}}}%  {text1}{text2} 
\long\def \centerspreadline #1#2#3{%
  \line{\rlap{#1}\hfil #2\hfil \llap{#3}}}%              {text1}{text2}{text3}


% This declaration causes line breaks to be "obeyed," so that each
% results in a paragraph end.

{\catcode`\^^M = \catactive%
\gdef \obeylines {\catcode`\^^M = \active \let ^^M = \par}%
}

%                       Line Breaks
%                       ---- ------


\def \linebreak #1{%                            {penalty}
  \penalty #1\relax}

\def \tl {\break}

\def \nl {%
  \gluevaries{\znlvar}{\leftskip}%
  \if \znlvar \tl \else \hfil \break \fi}

\def \blankline {\ensurepar \nl}


% These commands are used to tie text on one line.

\def ~{\nobreak \ }                             % Unbreakable space (tie).

\def \tie {\ensurepar \hbox}

%                       Indentation
%                       -----------


% Note that this macro just sets \leftskip and \rightskip, and so
% blows away any \parrag.

\def \setindentation #1#2{%                     {left}{right}
  \leftskip = #1\relax
  \rightskip = #2\relax}


\def \alterindentation #1#2{%                   {left-delta}{right-delta}
  \leftskip = 1\leftskip
  \advance \leftskip by #1\relax
  \rightskip = 1\rightskip
  \advance \rightskip by #2\relax}

%                       Text Width
%                       ---- -----


% This command must be used AFTER the indentation is established with
% \setindentation or \alterindentation.

\def \settextwidth #1{%                                 {width}
  \if \dimneqlp{#1}{\naturalwidth}%
    \hsize = #1\relax
    \advance \hsize by \leftskip
    \advance \hsize by \rightskip
  \fi}

%                       Paragraph Raggedness
%                       --------- ----------


% This command must be used AFTER the indentation is established with
% \setindentation or \alterindentation.

\def \setparrag #1{%                                            {rag-dimen}
  \gluevaries{\zrsvar}{\rightskip}%
  \if \notp{\zrsvar}%
    \tdimena = #1\relax
    \advance \rightskip by 0pt plus \tdimena\relax
  \fi}

\zdeclareeveryvcontext{\setindentation{0pt}{0pt}\setparrag{0pt}}

%                       Paragraph Shapes
%                       --------- ------


\definecount{\zshparct}{0}
\definecount{\zshparrep}{0}
\definedimen{\zshpari}{0pt}
\definedimen{\zshparw}{0pt}


\def \nshapepar #1{%                            {[*repeat1,]indent1,width1,...}
  \if \emptyargp{#1}%
    \error{empshape}{The argument to \string\shapepar cannot be empty}%
  \fi
  \gdef \zshpartoks {}%
  \global\zshparct = 0
  {\zshapeparc #1,\zmark}%
  \parshape = \zshparct\zshpartoks\relax}

\def \zshapeparc #1#2,#3\zmark{%                [*repeat],indent,...
  \if \codeeqlp{#1}{*}%
    \global\zshparrep = #2\relax
    \def \znext {\zshapepare #3\zmark}%
  \else
    \global\zshparrep = 1
    \def \znext {\zshapepare #1#2,#3\zmark}%
  \fi
  \znext}

\def \zshapepare #1,#2,#3\zmark{%               indent,width,...
  \zshapepard #1,\false,\zshpari\zmark
  \global\zshpari = \tdimena
  \zshapepard #2,\true,\zshparw\zmark
  \global\zshparw = \tdimena
  \loop
    \if \posp{\zshparrep}%
      \xdef \zshpartoks {\zshpartoks \space \the\zshpari \space \the\zshparw}%
      \global\increment \zshparct
      \global\decrement \zshparrep
  \repeat
  \if \emptyargp{#3}%
    \let \znext = \relax
  \else
    \def \znext {\zshapeparc #3\zmark}%
  \fi
  \znext}

\def \zshapepard #1#2,#3,#4\zmark{%             dimen1 dimen-rest,width?,prev
  \if \codeeqlp{#1}{+}%
    \calculate \tdimena = {#4,+,#2}%
  \else\if \codeeqlp{#1}{-}%
    \calculate \tdimena = {#4,-,#2}%
  \else\if \andp{\codeeqlp{#1}{=}}{#3}%
    \calculate \tdimena = {\hsize,-,\zshpari}%
  \else
    \tdimena = #1#2\relax
    \if \dimeqlp{\tdimena}{\naturalwidth}%
      \calculate \tdimena = {\hsize,-,\zshpari}%
    \fi
  \fi\fi\fi
  \if #3%
    %%% Do we want to adjust the width by \leftskip or \rightskip?
  \fi}


%%%\definecount{\zshparct}{0}
\definedimen{\zshpard}{0pt}
\definedimen{\zshpare}{0pt}


\def \shapepar #1{%                             {indent1,width1,...}
  \if \emptyargp{#1}%
    \error{empshape}{The argument to \string\shapepar cannot be empty}%
  \fi
  \gdef \zshpartoks {}%
  \global\zshparct = 0
  {\zshapepara #1,\zmark}%
  \parshape = \zshparct\zshpartoks\relax}

\def \zshapepara #1,#2,#3\zmark{%               indent,width,...\zmark
  \zshpard = #1\relax
  \xdef \zshpartoks {\zshpartoks \space \the\zshpard}%
  \zshpare = #2\relax
  \if \dimeqlp{\zshpare}{\naturalwidth}%
    \zshpare = \hsize
    \advance \zshpare by -\zshpard
  \else
    \advance \zshpare by \leftskip
    \advance \zshpare by \rightskip
  \fi
  \xdef \zshpartoks {\zshpartoks \space \the\zshpare}%
  \global\increment \zshparct
  \if \emptyargp{#3}%
    \let \znext = \relax
  \else
    \def \znext {\zshapepara #3\zmark}%
  \fi
  \znext}


\def \hangalign #1#2#3{%                        {indent}{text1}{text2} 
  \endgraf
  {\setbox \zboxa = \hbox{#2}%
   \tdimena = \wd\zboxa
   \advance \tdimena by #1\relax
   \shapepar{#1,\naturalwidth,\tdimena,\naturalwidth}%
   \noindent \box\zboxa #3\par}}

\def \hindent #1{%                              {indent}
  \noindent \hspace{#1}%
  \ignorespaces}

%                       Adjustments
%                       -----------


%~ This page overrides the number of lines in a paragraph. The argument
%~ specifies the number of lines and should start with a plus or
%~ minus sign. TeX does not guarantee the change.

\def \overrideparlength #1{%                            {lines} %^page_makeup
  \looseness = #1\relax}


%~ This command is used in the middle of a paragraph, on a line by
%~ itself, to make an adjustment
%~ between two lines. A typical use is to force a page break:
%~   \adjustpar{\fullpagebreak}
%~ There is no need to put a percent sign (|%|) after the command.

\long\def \adjustpar #1{%                               {command} %^page_makeup
  \vadjust{#1}%
  \ignorespaces}


% This command is used to adjust the title of a chapter, section, etc.
% The argument looks like `\loc{commands}\loc{commands}...'
% where each \loc is:
%   \body       To adjust the title in the body of the text.
%   \mark       To adjust the title in the mark (running heads and feet).
%   \toc        To adjust the title in the table of contents.

\def \zadjloc {\body}   % Normally, \adjusttitle should adjust body.

\def \adjusttitle #1{%                                  {\loc{commands}...}
  {\def \body ##1{}%
   \def \mark ##1{}%
   \def \toc ##1{}%
   \expandafter\let \zadjloc = \zadjtitleb
   #1}}

\def \zadjtitleb #1{%
  #1%
  \def \znext {#1}%
  \if \tokeqlp{\znext}{\nl}%
    \let \znext = \ignorespaces
  \else
    \let \znext = \relax
  \fi
  \znext}

% These macros use \lspspace so they do the right thing while letterspacing.

\def \titlenl {\adjusttitle{\body{\nl}}\lspspace}
\def \titletl {\adjusttitle{\body{\tl}}\lspspace}
\def \tocnl   {\adjusttitle{\toc{\nl}}\lspspace}
\def \toctl   {\adjusttitle{\toc{\tl}}\lspspace}


% These definitions are used by the new title adjustment scheme,
% but need to be present even when it is not in use.

\chardef \zadjtitlenone     = 0
\chardef \zadjtitlebody     = 1
\chardef \zadjtitlemark     = 2
\chardef \zadjtitletoc      = 3
\chardef \zadjtitleminitoc  = 4

\definecount{\zadjtitle}{\zadjtitlebody}
